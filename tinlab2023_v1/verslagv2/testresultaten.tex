 
\section{testresultaten}
Om een goed verhaal op te stellen, moet vooraf aan enkele voorwaarden
worden voldaan. De eerste voorwaarde is de geschiktheid van het
afstudeerproject. Als een afstudeerproject niet tot keuzes leidt, kan
men zich afvragen of dat wel een echte afstudeeropdracht is. Een
afstudeerproject zonder onderzoeksaspecten is ook verdacht. Daarnaast
moet een afstudeerproject passen in het profiel van een opleiding om
beoordeelbaar te zijn. De andere voorwaarde voor goed een verhaal is
de registratie van werkzaamheden tijdens het a


\begin{tabular}{|l|l {2cm} |l|l|l|l|l|l|l|l|} \hline
	\multicolumn{10}{|l|}{REQ traceabilitymatrix}                                                               \\ \hline
	\multicolumn{4}{|l|}{project name}   &\multicolumn{6}{|l|}{Created designed by}                           \\ \hline
	\multicolumn{4}{|l|}{release no}   &\multicolumn{6}{|l|}{Created on}                           \\ \hline
	\multicolumn{4}{|l|}{version}   &\multicolumn{6}{|l|}{Reviewed on}                           \\ \hline
	\multicolumn{4}{|l|}{Test title}   &\multicolumn{6}{|l|}{Reviewed by}                           \\ \hline
	\multicolumn{4}{|l|}{Description}   &\multicolumn{6}{|l|}{ }                           \\ \hline 		
	\multicolumn{10}{|l|}{ }   																\\ \hline
	\multicolumn{10}{|l|}{Pre condition}                                                               \\ \hline
	\multicolumn{10}{|l|}{Dependencies}                                                               \\ \hline
	\multicolumn{10}{|l|}{ }   															\\ \hline
	\multicolumn{2}{|l|}{REQID} & descr & ST &Designdoc &codemod.&t.caseID &T.caseNme & manu& testedon \\ \hline
	 
\end{tabular}






\subsection{Testcases}		

Om een goed verhaal op te stellen, moet vooraf aan enkele voorwaarden
worden voldaan. De eerste voorwaarde is de geschiktheid van het
afstudeerproject. Als een afstudeerproject niet tot keuzes leidt, kan
men zich afvragen of dat wel een echte afstudeeropdracht is. Een
afstudeerproject zonder onderzoeksaspecten is ook verdacht. Daarnaast
moet een afstudeerproject passen in het profiel van een opleiding om
beoordeelbaar te zijn. De andere voorwaarde voor goed een verhaal is
de registratie van werkzaamheden tijdens het a



\begin{tabular}{|l|l|l|l|l|l|l|} \hline
	\multicolumn{7}{|l|}{project name}                                                               \\ \hline
	\multicolumn{4}{|l|}{Test case ID}   &\multicolumn{3}{|l|}{Test designed by}                           \\ \hline
	\multicolumn{4}{|l|}{test priority (low/medium/high)}   &\multicolumn{3}{|l|}{Test design date}                           \\ \hline
	\multicolumn{4}{|l|}{Module name}   &\multicolumn{3}{|l|}{Test executed by}                           \\ \hline
	\multicolumn{4}{|l|}{Test title}   &\multicolumn{3}{|l|}{Test execution date}                           \\ \hline
	\multicolumn{4}{|l|}{Description}   &\multicolumn{3}{|l|}{ }                           \\ \hline 		
	\multicolumn{7}{|l|}{ }   																\\ \hline
	\multicolumn{7}{|l|}{Pre condition}                                                               \\ \hline
	\multicolumn{7}{|l|}{Dependencies}                                                               \\ \hline
	\multicolumn{7}{|l|}{ }   															\\ \hline
	Step  &  Test steps & Test data & expected result &Acual result &Streee (pass or fail)&notes  \\ \hline
	
\end{tabular}

\subsubsection{Een schip komt aanvaren}	

\subsubsection{Een schip moet wachten}

\subsubsection{Een schip komt van boven naar beneden}

\subsubsection{De sluis is vol: hoe lang mkoet een schip wachten}


\subsubsection{Een schip wil in de sluis koers wijzigen}

\subsubsection{De volgorde in  de wachtrij}


\subsubsection{Aantal schepen in de wachtrij}	

\subsubsection{Maximale doorlooptijd }

\subsubsection{Minimale doorlooptijd 
	
	
	
	Black box testing: product test  zonder kennis van code, implementatie of interne paden
	Alpha testimg: bugs en issues testen voor de release; soort van acceptatietest
	Functioneel testen: require,emnnt testen op input/output en requirement omschrijving webinterface
	Domain testing: applicaties testen met normale input; waarvan het doel is om te testen of de app waardenbereik hanteert
	Gebruikeraaceptatieteast:
	Interface testing: om de communicatie tussen 2 interfaces te testen
	Vulnerability testing: testen op kwetsbaarheden webinterface
	Configuration testing: de app testen met meerdere hardware en software combinaties; verschillende browsers, OS; IoT cloud
	Applicatie test: script om de gehele applicatie te testen
	Negative testing: test de ap[p op negatieve waarden waaronder 
	Interoperabiliteit : test of het systeem kan communiceren met andere systemen; zoals cloud/wifi/netwerk
	Complianc testing: test of het systeem voldoiet aan de stabndaarden
	Loop testing: test over de loops in de app
	Component testing: elk systee,mcomponent wordt getest
	Dynamic testing: test de code in natuurlijke omgeving
	Parralel testing: systeem testen tegelijk op meerdere platformen
	Operational acceptance testing:
	Module testing: test de procedures, classes, subroutine of subprogramma’s
	Workflow testing: wordt de serie taken juist uitgevoerd
	Storage testing: wordt de relevantedata opgeslagen
	Recovery testing: wordt de systeem ook getest op hoe het reageert op cradshes
	Concurrency testing: hoe reageert de app als er moderne gebruikers tegelijkj toegevoegd zijn
	Thread testing: key functionaliteit van een thread testen
	Desctructive test: apparaat wordt geteast op het  omgaan met verkeerde informatie
	Test-driven-development: test ontwikkelen voor elke functionalitei
	Data driven testing:
	Monkey testing: de app testen zonder  vooraf gedefinieerd
	Embedded testing:
	Rest api testing:
	Gui testing:
	Automated vs manual testing:>
	Unit test: individuele componenten testen
	Integrale test: werkt de software als onderdeel van een groot systeem
	System test: test de werking van het volledige product
	Sanity Test: is het product rationeel ontwikkeld
	Smokey test: werken de kritische functionaliteiten
	Regressie test: test of een wijziging in de code geen invloed heeft op meer processen of taken
	non-functionele testebn
	Boundary testen: testen op partities
	Decision table testing: wat te  doen als er een incoerveld verkeerd is ingevuld
	Stat transition testing: toont het systeem de informatie bij een weijziging
	Agile teting
	Continuous testoing
	Performance test: hoeveel data
	Load testing: is laadtijd van het systeem relevant
	Stress test: test van het systeem onder zware condities
	Volumte testin: een volume aan data testen
	Scalability testing systeem testen bij extreem data verkeer
	Performance testing
	Response time testing: hoe lang durut het voordat het systeem reageert
	Benchmark testing: vergelijken van kwaliteitsattribh8uten
	Endurance testing: systeem testen over lange periode
	Reliabilityb testing: kan het systeem foutloos draaien over een lange periode
	Test metrics
	End-to-e nd testing: test hele systeem van end bnaa alle
	Externe interfaces: test op afhankelijkheden; data integriteit; communicatie met andere systemen, interfaces en databases
	Exploratory testing: testing on the fly
	Mutation testing: een controle op kleine wijzigingen van bijv namen van variabelen of functies
	Ad-hoc testing: willekeurig overal testen waar vermoedelijk fouten zitten in het key-drives-framework
	risk -based-testing: in kaart brengen van
	Backend testing: bijv dedatabase-tabel van telemetrie
	Localization testing: regio gericht testen
	Ontogonaal testen: testen van testcases
	Pilot testing: feasibility; time;cost;risk;performance
	Model base testen: testen volgens een model
	Grey box testing
	Whitelabeel testing: test hoe ebruiksvriendelijk het product is
	
	
	\subsection{Reparaties}
	
	\subsection{Resulaat annalyse}
	
	
	%%%%%%%%%%%%%%%%%%%%%%%%%%%%%%%%%%%%%%%%%%%%%%%%%%%%%%%%%%%%%%
	\subsubsection{Editor}
	
	
	%%%%%%%%%%%%%%%%%%%%%%%%%%%%%%%%%%%%%%%%%%%%%%%%%%%%%%%%%%%%%%
	
	\subsubsection{Concrete simulator}
	
	
	%%%%%%%%%%%%%%%%%%%%%%%%%%%%%%%%%%%%%%%%%%%%%%%%%%%%%%%%%%%%%%
	
	\subsubsection{verifier}
	%%%%%%%%%%%%%%%%%%%%%%%%%%%%%%%%%%%%%%%%%%%%%%%%%%%%%%%%%%%%%%
	
	\subsubsection{Yggdrasil}