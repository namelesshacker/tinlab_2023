\newpage
\section{Requirementsengineering}

\subsection{Rampen}
 

\subsubsection{Mali}
Een granaat explodeerd in een mortier
De medische zorg na het ongeval was neit voldoende


De algemeen militair verpleegkundige gaf aan het slachtoffer nar het vn-hospitaal in kidal te brengen
De chaauffeur van de bushmaster kende de locatie niet  en bracht het slachtoffer naar een door frane militairren bemand hospitaal mmet minder mediswche faciliteiten
Hierna alsnog overgebracht naar het vn-hospitaal.
Dit verlieop  neit door nederlandse maatstaven.
pas toen een nederlandse arts arrivveerde werd door de Tongolese artsen een buikoperatie uitgevoerd.
Dit gebrurde zonder adequate anesthesie.
Na de operatie werde de gewonde militair overgelogen naar nederland. En later naar nederland.


granaat stond niet op scherp en in afgegaan in veilige stand
Granaat werd opgeslagen in neit gekoelde containers waardoor deze aan te hoge temeperaturen zijn blootgesteld.
Door de comvinatie van vocht en warmte in de granaat zeer gevoelige explosieve stoffen werden gevormd.
Tijdens de oefening was de fatale granaat in de zon.
Het afsluitplaatje in de granaat bleek niet in staat om doorslag in veilige stand te voorkomen waarna de granaat explodeerde.
De moritren zijn aangeschaft bij de amerikanen. gredurende de aanschafperiode zijn procedures en controles op kwaliteit en veiligheid deels nagelaten.
Dit veiligheidsgarantie werd vermeld in het koopcontract.
Conclusie
Koopcontract werd niet goed doorgelezen
Geen controle op kwaliteit en veiligheid
Geen controle op kwaliteit en veiligheid
Zwakke plekken in het ontwerp
Geen controle op kwaliteit en veiligheid
opslag en gebruik in ongunstige condities

De aanwezige medische voorzieningen waren nite volgends de nederlandse militaire richtlijnen
Het ontbreek aan medische toetsing vanuit de defensie organisatie
twijfels die werden geuit binnen de defensieorganisae vonden geen wrrklank
Ok het ongeval tijdens de mortieroefening was voor defensie geen aanleuiding om de medische voorzienignen te evalueren.
De inrichting van veilige medische zorg voor nederlandse militairen in kidal is ondergeschikt gemaakt aan de voortgang van de missie.


https://www.youtube.com/watch?v=PC2ekl4SaNA 

 
Een granaat explodeerd in een mortier
De medische zorg na het ongeval was neit voldoende


De algemeen militair verpleegkundige gaf aan het slachtoffer nar het vn-hospitaal in kidal te brengen
De chaauffeur van de bushmaster kende de locatie niet  en bracht het slachtoffer naar een door frane militairren bemand hospitaal mmet minder mediswche faciliteiten
Hierna alsnog overgebracht naar het vn-hospitaal.
Dit verlieop  neit door nederlandse maatstaven.
pas toen een nederlandse arts arrivveerde werd door de Tongolese artsen een buikoperatie uitgevoerd.
Dit gebrurde zonder adequate anesthesie.
Na de operatie werde de gewonde militair overgelogen naar nederland. En later naar nederland.


granaat stond niet op scherp en in afgegaan in veilige stand
Granaat werd opgeslagen in neit gekoelde containers waardoor deze aan te hoge temeperaturen zijn blootgesteld.
Door de comvinatie van vocht en warmte in de granaat zeer gevoelige explosieve stoffen werden gevormd.
Tijdens de oefening was de fatale granaat in de zon.
Het afsluitplaatje in de granaat bleek niet in staat om doorslag in veilige stand te voorkomen waarna de granaat explodeerde.
De moritren zijn aangeschaft bij de amerikanen. gredurende de aanschafperiode zijn procedures en controles op kwaliteit en veiligheid deels nagelaten.
Dit veiligheidsgarantie werd vermeld in het koopcontract.
Conclusie
Koopcontract werd niet goed doorgelezen
Geen controle op kwaliteit en veiligheid
Geen controle op kwaliteit en veiligheid
Zwakke plekken in het ontwerp
Geen controle op kwaliteit en veiligheid
opslag en gebruik in ongunstige condities

De aanwezige medische voorzieningen waren nite volgends de nederlandse militaire richtlijnen
Het ontbreek aan medische toetsing vanuit de defensie organisatie
twijfels die werden geuit binnen de defensieorganisae vonden geen wrrklank
Ok het ongeval tijdens de mortieroefening was voor defensie geen aanleuiding om de medische voorzienignen te evalueren.
De inrichting van veilige medische zorg voor nederlandse militairen in kidal is ondergeschikt gemaakt aan de voortgang van de missie.


https://www.youtube.com/watch?v=PC2ekl4SaNA 
\subsubsection{tesla crash report}
Door een softwarefout zijn er situaties ontstaan waarin het systeem informatie een onvoldoende informatie positie had om de juiste beslissingen te maken. Of dat de informatieverwerking niet juist was.
 


\subsubsection{ethiek}
 

\subsubsection{ cyber aanval op Oekraïene }


Hackers konden door het versturen van corrupte emails zichzeklf toegang verschaffen tot  SCADA controle systemen. Door de dienstdoende operators uitgebreid te observeren.
first doing reconnaissance to study the networks and siphon operator credentials, then launching a synchronized assault in a well-choreographed dance.
Ondanks dat de elektriciteitescentrale soms nog beter was beveiligd dan in de VS. toch is het de hackers gelukt door medewerkers logging remotely into the SCADA network, the Supervisory Control and Data Acquisition network that controlled the grid, weren't required to use two-factor authentication, which allowed the attackers to hijack their credentials and gain crucial access to systems that controlled the breakers.
%https://en.wikipedia.org/wiki/Ukraine_power_grid_hack 


 


\subsubsection{schipholbrand}
 



\subsubsection{therac-25}
Softwarefout uit zich als hardwarefout de klachtafhandeling geen onderzoek geen second opinion is prioriteit wel 
gechecked na onderzoek bellen en geen prioriteit aanwezig te zijn alleen importeurs en fabriken mogen fouten 
in frabrieksinstellingen rapporteren 
Therac25 Systeem ligt plat veel voorkomende eror stdaardafhandeling om de error te verwerpen resultaat: 
de patient kreeg overdosis patient overleden onderzoek opgestart, stuatie niet reproduceerbar foutmarkering: 
gezien als uitzonderlijk, software aanpassing van groote magnitude 5; de oorzaak was waarschijlijk mechanisch 
maar neit vastgesteld; conceptueel odel niet aangepast probleemclassicificatie door autorititen het probleem 
en de impact daarvan anar beneden bijgesteld AEFL doe gedeeltelijke aanpassing om hardware na berisping 
Canadese autoriteit 
Derde patient overleden door eythema AECL wijst alle doodsoorzaken af AECL beweert dat geen vergeli- 
jkbare voorvalle bij andere machines of patienten zijn voorgekomen geen vervolgonderzoek vanwege garanties 
bedrijf gaat uit van geen mogelijke functionele fout 
vierde patient overleden aan overdodis ontstaan door bug in software onjuiste aanduiding bij de foutmelding 
verkeerde reactie/invoer ddoor operator communicatie tussen patient en operator werd onvoldoende gemon- 
itorred ( apparatuur niet aangesloten, en audio monitor kapot) engineer van AECL stelt geen fouten vast 
Engineer AECl kan fout niet reproduceren Geen communicate tussen bedrijf en uitgezonden technisci over 
vergelijkbare probleemgevallen 
vijfde geval malfunction 54 leidt tot overdosis en de dood fout gereproduceerd door operator bedrijf fout 
was daa entryspeed herpublicatie van de ongevallen en de eerdere ongevallen in de meia apparaat wel nog in 
gebruik genomen niet handig, waarschuwingsberichten en aanwijzingen voor een bugfix naar de gebruikers door 
druk van fda is bedrijf op zoek gegaan naar permanente oplossing 
zesde geval software fout door softwarefout otntstaat lightstruct .. op de patient na onderzoek door AECL 
blijkt niet alleen hardware de oorzak gebruikers direct geinformeerd oplossing gevonden, media ingeschakeld om 

transparantie af te dwingen door de gebruikersgroep en de FDA AECL gedwongen functionaliteit aan te passen 
Engineers hebben meer studie moeten maken van gebruikte technologie en onderhoudbaarheid daarvan 


\subsubsection{Ramp schietpartij militair ossendrecht }
Een militaire overleid op een schietbaan in ossendracht door onvoldoende begeleiding van cursisten, geen toezicht op de lokatie. E\r was een instructuur in opleiding die niet volledig was mmeegenomen in het poroces en ook was er geen baancommandant aanwezig. Geen van de aanwezig instructeurts had de juiste papieren om de cursisten te begeleiden. De aanwezig instruceur had geen zich op de instructeur in opleiding, evenmin de andere militairen. In de instructiehandleiding ontbreken richtlijnen voor bijzondere schietbanen. Ook was er geen keuring. Door personelstekort is er geen andacht besteed aan documentastie(een slyllabus) hoe en met welke risico’s oefeningnen moeten worden ingericht. Ok werd er vooraf geen veiliheidsanaklyse gedaan. Het gebrek aan lesmateriaal en deskundigen is gemeld binnen de defensieorganisatie maar dit heeft niet geleid tot enige verandering in de situatie.
Op een afgekeurde scheitbaan
Tezicht door een instructeur in opleiding die zelf geen persoonlijke begeleiding heeft gehad tijdens de uitvoering
Belangrijk is dat defensie haar taken kan uitvoeren met personeel dat is getraind in situaties die de risicos van de werkomgeving aan de cursisten kunnen laten zien.
Conclusie
Zonder gekwalificeerde instructuers.
Zonder toezicht
Zonder lesmateriaal
Zonder adequate veiligheidsanalyse
https://www.youtube.com/watch?v=6jmkDClGDHo 


\subsubsection{molukse treinkaping }
https://www.youtube.com/watch?v=h99Fe9XzzHI 

\subsubsection{vuurwerkramp in enschede }
https://www.enschede.nl/inhoud/commissie-oosting 
https://www.politie.nl/binaries/content/assets/politie/wob/00-landelijk/vuurwerkramp-enschede/bijlagen-rapport-vuurwerkramp-enschede.pdf 
https://www.researchgate.net/publication/254815008_Rampen_regels_richtlijnen 

 





\subsubsection{explosie in libabon, beirut 
}

Op 23 september 2013 voer het vrachtschip de Rhosus onder Moldavische vlag[7] van Batoemi in Georgië naar Beira in Mozambique met 2.750 ton ammoniumnitraat

Gezien het ernstige gevaar van het bewaren van deze goederen in de hangar onder ongeschikte klimatologische omstandigheden, herhalen we ons verzoek aan de marine-instantie om deze goederen onmiddellijk weer te exporteren om de veiligheid van de haven en de mensen die er werken te verzekeren, of om akkoord te gaan om ze te verkopen.
Voorafgaand aan de explosie was er een brand in een opslagplaats. 

https://www.hrw.org/report/2021/08/03/they-killed-us-inside/investigation-august-4-beirut-blast 
https://www.researchgate.net/publication/348325979_Beirut_Explosion_the_full_story 
https://reliefweb.int/sites/reliefweb.int/files/resources/CaseStudy_BeirutExplosion_TechBioHazardsweb.pdf 
\subsubsection{explosie tanjin china 
}

Later bleek uit een onderzoek van de Chinese autoriteiten dat de explosie overeenkwam met de ontploffing van 450 ton TNT.[6] 
De oorzaak van de explosie lag in de spontane zelfontbranding van 207 ton cellulosenitraat dat in containers was opgeslagen op het terminalterrein.[6] 
Verder lag op een tweede locatie nog eens 26 ton van dit explosieve materiaal opgeslagen.
De tweede ontploffing werd versterkt door de opslag van 800 ton kunstmest in de vorm van ammoniumnitraat in de nabijheid.[6]
De opslag van cellulosenitraat is aan strenge regels gebonden. Het moet koel en droog worden opgeslagen. De containers stonden buiten opgesteld in de brandende zon. De temperatuur liep op tot 36 °C en bereikte binnen de containers waarschijnlijk de 65 °C.[6] De verpakking van de cellulosenitraat droogde uit waardoor de ontploffing kon ontstaan. Op het terrein lagen meer gevaarlijke stoffen opgeslagen dan waarvoor vergunningen waren verstrekt.[6] Dit leidde tot een kettingreactie met grote schade tot gevolg. Door de brand en bluswater is in de directe omgeving veel milieuschade opgetreden.


https://www.hindawi.com/journals/joph/2019/1360805/ 

\subsubsection{bijlmerramp}
Motor 3 (de binnenste motor aan de rechtervleugel van het vliegtuig) brak af, beschadigde de vleugelkleppen en botste tegen motor 4 die vervolgens ook afbrak.
De ernst van de situatie werd op Schiphol niet goed ingezien. Dit kwam onder meer doordat lost in de luchtvaart de gebruikelijke term is om het verlies van motorvermogen te melden. Op Schiphol werd er dan ook van uitgegaan dat er twee motoren waren uitgevallen. Dat ze letterlijk verloren waren wist men niet. Gezien het grote aantal handelingen dat de bemanning in een paar minuten moest uitvoeren en de keuzes die de piloot maakte, veronderstelde de parlementaire enquêtecommissie die de ramp later zou onderzoeken dat ook de bemanning waarschijnlijk niet heeft geweten dat beide motoren van de rechtervleugel waren afgebroken. De buitenste motor van een 747 is vanuit de cockpit slechts met moeite zichtbaar en de binnenste motor helemaal niet.

Op de avond van de 4e oktober 1992 was landingsbaan 06 (de Kaagbaan) in gebruik. De piloot verzocht de luchtverkeersleiding op Schiphol echter een noodlanding te mogen maken op de Buitenveldertbaan (baan 27). Waarom hij juist deze baan koos, is nooit duidelijk geworden. Een keuze voor deze baan lag niet voor de hand; omdat de wind uit het noordoosten kwam, zou het toestel met flinke staartwind moeten landen. Langs de landingsbaan waren enkele grote brandweerwagens van Schiphol geplaatst. Deze zogeheten crashtenders moesten een brand tijdens de landing meteen blussen. Na de crash werd één zwarte doos teruggevonden. De bijbehorende band was in vier stukken gebroken, waardoor de laatste 2 minuten en 45 seconden ervan niet meer te gebruiken waren. De doos werd voor onderzoek naar Washington gestuurd en leverde uiteindelijk onderstaande informatie op.
Om goed uit te komen voor de landingsbaan vloog het beschadigde toestel eerst nog een rondje boven Amsterdam. Tijdens dit rondje gaf de gezagvoerder de copiloot opdracht de vleugelkleppen (flaps) uit te schuiven. Links schoven de kleppen uit, maar doordat de afgebroken motor 3 de rechtervleugel had beschadigd schoven de kleppen op die vleugel niet uit. Als gevolg hiervan kreeg het toestel links meer draagvermogen dan rechts. De piloot meldde aan de verkeersleiding dat er ook problemen met de flaps waren.
Aanvankelijk ging het aanvliegen van de Buitenveldertbaan goed. Op het moment dat het vliegtuig daalde tot onder de 1500 voet en snelheid minderde, raakte het echter compleet onbestuurbaar en maakte het een ongecontroleerde, scherpe bocht naar rechts. Over de radio was te horen dat de gezagvoerder zijn copiloot in het Hebreeuws opdracht gaf om alle kleppen in te trekken en het landingsgestel uit te klappen. Vervolgens meldde de copiloot in het Engels aan de luchtverkeersleider dat het toestel zou gaan neerstorten. Uit later onderzoek bleek dat het vliegtuig eerder enkel recht bleef vanwege de hoge snelheid (280 knopen, zijnde 519 km/u). Doordat de rechtervleugel beschadigd was, was het moeilijker om het vliegtuig recht te houden. Alleen de hoge snelheid zorgde ervoor dat er nog voldoende draagvermogen was. Toen bij het inzetten van de landing de snelheid verlaagd werd, werd het draagvermogen van de rechtervleugel echter dusdanig gering dat het toestel niet meer onder controle te houden was en een duikvlucht naar rechts maakte.

https://aviation-safety.net/database/record.php?id=19921004-2&lang=nl 


\subsubsection{slmramp}
Toen de Anthony Nesty Zanderij naderde, was het daar, anders dan het weerbericht had voorspeld, mistig. Het zicht was evenwel niet zo slecht dat er niet op zicht kon worden geland. Gezagvoerder Will Rogers besloot echter via het Instrument Landing System (ILS) te landen, hoewel dit niet betrouwbaar was en hij voor zo'n landing ook geen toestemming had. De gezagvoerder brak drie landingspogingen af. Bij de vierde poging negeerde de bemanning de automatische waarschuwing (GPWS) dat het toestel te laag vloog. Het toestel raakte op 25 meter hoogte twee bomen. Het rolde om de lengteas en stortte om 04.27 uur plaatselijke tijd ondersteboven neer.

Uit onderzoek bleek dat de papieren van de bemanning niet in orde waren. 
Geconcludeerd werd dat de gezagvoerder roekeloos had gehandeld door voor een ILS-landing te kiezen terwijl hij daar geen toestemming voor had, en door onvoldoende op de vlieghoogte te hebben gelet. 
De SLM werd verweten de kwalificaties van de bemanning onvoldoende te hebben gecontroleerd.

https://aviation-safety.net/investigation/cvr/transcripts/cvr_py764.php 
https://aviation-safety.net/database/record.php?id=19890607-2 

\subsubsection{ethiopian airlines}
Ethiopian Airlines Flight 302
Door problemen met de flight control
One minute into the flight, the first officer, acting on the instructions of the captain, reported a "flight control" problem to the control tower.
Two minutes into the flight, the plane's MCAS system activated, pitching the plane into a dive toward the ground. The pilots struggled to control it and managed to prevent the nose from diving further, but the plane continued to lose altitude.
The MCAS then activated again, dropping the nose even further down. The pilots then flipped a pair of switches to disable the electrical trim tab system, which also disabled the MCAS software. However, in shutting off the electrical trim system, they also shut off their ability to trim the stabilizer into a neutral position with the electrical switch located on their yokes. The only other possible way to move the stabilizer would be by cranking the wheel by hand, but because the stabilizer was located opposite to the elevator, strong aerodynamic forces were pushing on it.
As the pilots had inadvertently left the engines on full takeoff power, which caused the plane to accelerate at high speed, there was further pressure on the stabilizer. The pilots' attempts to manually crank the stabilizer back into position failed.
Three minutes into the flight, with the aircraft continuing to lose altitude and accelerating beyond its safety limits, the captain instructed the first officer to request permission from air traffic control to return to the airport. Permission was granted, and the air traffic controllers diverted other approaching flights. Following instructions from air traffic control, they turned the aircraft to the east, and it rolled to the right. The right wing came to point down as the turn steepened.
At 8:43, having struggled to keep the plane's nose from diving further by manually pulling the yoke, the captain asked the first officer to help him, and turned the electrical trim tab system back on in the hope that it would allow him to put the stabilizer back into neutral trim. However, in turning the trim system back on, he also reactivated the MCAS system, which pushed the nose further down. The captain and first officer attempted to raise the nose by manually pulling their yokes, but the aircraft continued to plunge toward the ground.

https://www.hindawi.com/journals/ijae/2014/472395/ 


\subsubsection{stint ongeluk}
Vier kinderen, een bestuurder kwamen om en een vijfde persoon , een kind raakte zwaargewond. Uit odnerzoek van bleek :
Foute torsieveer voor de gashendel werd geleverd
Geen van de drie onderzochte voertuigen haalden de wettelijk vereiste remvertraging
De automatische parkeerrem kan leiden tot gevaarlijke situaties wanneer deze ongewenst geactiveerd wordt tijdens het rijden. 
Het losraken van de nuldraad naar de gashendel leidt volgens TNO tot ongewenst versnellen van het voertuig en een oncontroleerbare situatie voor de bestuurder.
Voor alle drie onderzochte voertuigen geldt dat het ontbreken van een zitplaats leidt tot veiligheidsrisico’s voor remmen en sturen door de grotere kans dat de bestuurder van het voertuig valt. Als de bestuurder van een Stint valt, leidt dit in alle rijsituaties tot een onbeheersbare situatie


https://repository.tno.nl/islandora/object/uuid%3Acdef48df-da49-46b6-8678-5c62a88a0090 


\subsubsection{tjernobyl}
Een ramp bij een kernreacor in de sovjetunie. Door een bedieningsfout in een testprocedure werd het vermogen van de koelinstallaties negatief beinvloed. Door een ontwerpfout in de noodstopprocedure kon in het systeem niet snel genoeg schakelen om remmende invloed uit te oefenen op het toenemende vermogen van de reactorkernen. Met brand en eksplosie tot gevolg.
https://www-pub.iaea.org/MTCD/publications/PDF/Pub913e_web.pdf 

\subsubsection{ecounrt in de nerderlandse rechdspraak}
https://www.njb.nl/blogs/a-court-with-no-face-and-no-place/ 
http://www.e-court.nl/wp-content/uploads/2018/03/Procesreglement-e-Court-2017_20180201.pdf 


 

\subsubsection{ramp turkisch airlines}
Inadequaat handelen van de piloten ondanks een defecte hoogtemeter en onvolledige instructies van de luchtverkeersleiding/
https://catsr.vse.gmu.edu/SYST460/TA1951_AccidentReport.pdf 

Wat ging er allemaal mis bij de bovengenoemde rampen en ongelukken....... 

Wat hebben deze rampten te maken met de requirements en specificaties van deze odpracht? 


\subsubsection{automatisering van waterwerken}
https://hbo-kennisbank.nl/searchresult?q=sluizen 
artificial inte;lligence and water locks 
https://www.sciencedirect.com/science/article/pii/S0160791X21002165 
https://www.wwdmag.com/artificial-intelligence/sewer-monitoring-turns-ai 
https://www.anylogic.com/resources/articles/analysis-of-the-expansion-of-the-panama-canal-using-simulation-modeling-and-artificial-intelligence/ 
ai used in public infrastructure thesis 
https://blog.ferrovial.com/en/2020/10/how-artificial-intelligence-is-used-for-infrastructure-maintenance/ 
https://www.tilburguniversity.edu/about/schools/law/departments/plg/ai-public-sector 
artificaial used in water shipping 
artificial used in maritime transport 
artificaial used in water shipping 
ai used in maritime traffic 



https://www.ftm.nl/artikelen/waternet-verantwoordelijkheid-digitaal-wanbeleid 
https://open.overheid.nl/repository/ronl-bc28f344-af87-481a-aea6-58c481b4cdc8/1/pdf/ilt-onderzoeksrapport-stichting-waternet.pdf\ 
https://www.security.nl/posting/697815/Waternet+onder+verscherpt+toezicht+wegens+onvoldoende+grip+op+cybersecurity 
https://www.security.nl/posting/677368/Inspectie+doet+onderzoek+naar+Waternet+na+verzwegen+penetratietest 
https://www.parool.nl/nederland/onderzoeksraad-rijk-houdt-informatie-over-cyberveiligheid-achter-met-grote-risico-s~b673ec1f/?referrer=https%3A%2F%2Fwww.google.com%2F 


google: artiifcial intelligence for industrial control systems researchgate 
https://link.springer.com/chapter/10.1007/978-3-030-38557-6_7 
automation  for industrial control systems 
researchgate 
google scholar:automation of industrial control systems problems 
hindawi.com: problems industrial control systems 

\subsubsection{Artificial intelligence en water locks}
 

\subsection{ethiek}

 
Ethiek 



persuasive technology 
https://www.humanetech.com/youth/persuasive-technology 
https://www.minddistrict.com/blog/persuasive-technology-new-insights-in-behavioural-change 
https://www.sciencedirect.com/book/9781558606432/persuasive-technology 
https://spectrum.ieee.org/how-persuasive-technology-can-change-your-habits 
https://www.frontiersin.org/articles/10.3389/frai.2020.00007/full 
https://psmag.com/environment/captology-fogg-invisible-manipulative-power-persuasive-technology-81301 
https://www.makeuseof.com/what-is-persuasive-technology/ 
https://lib.ugent.be/catalog/rug01:001235489 
https://cyberpsychology.eu/article/view/12270 


\subsubsection{mode confusion }
Mode confusion treed op als geobserveerd gedrag van een technisch systeem niet past in het gedragspatroon 
dat de gebruiker in zijn beeldvorming heeft en ook niet met voorstellingsvermogen kan bevatten 


\subsection{Een goed model}

What is a Good Model?
To some extent, building good models is an art. Dijkstra's motto "Beauty is our business" applies to models as well as to programs. Nevertheless, we can state seven criteria for good models. These criteria are in some sense obvious, and any person with experience in modelling will often try to adhere to them. But surprisingly, our list of criteria has - to the best of our knowledge - not been described elsewhere in the literature, although most of them occur in a technical report of Mader, Wupper and Boon. (We see this as a clear indication of the lack of interest for the methodology of modeling in our field.) Often, the criteria are hard to meet and typically several of them are conflicting. In practice, a good model is often one which constitutes the best possible compromise, given the current state-of-the-art of tools for modelling and analysis. But a truly beautiful model meets all the criteria! We refer to Mader, Wupper and Boon for further links to related work in the areas of software engineering, requirements analysis, and design.
\paragraph{Specification}
A good model has a clearly specified object of modelling, that is, it is clear what thing the model describes. The object of modelling can be (a part of) an existing artefact or physical system, but it may also be a document that informally specifies a system or class of systems (for instance a protocol standard), and it may even be a collection of ideas of a design team about a system they construct, expressed orally and/or by some drawings on a whiteboard.
A good model has a clearly specified purpose and (ideally) contributes to the realization of that purpose. Possible purposes include: communication between stake holders, verification of specific properties (safety, liveness, timing,..), analysis and design space exploration, code generation, and test generation. A model can be descriptive or prescriptive. If a model has to serve several distinct purposes then often it is better to construct multiple models rather than one.
\paragraph{Traceable}
A good model is traceable: each structural element of a model either (1) corresponds to an aspect of the object of modelling, or (2) encodes some implicit domain knowledge, or (3) encodes some additional assumption. Additional assumptions are for instance required when a protocol s tandard is incomplete (e.g., it does not specify how to handle certain events in certain cases). Links between the structural elements of the model and the aspects of the object of modelling should be clearly documented. A distinction must always be made between properties of (a component of) a model and assumptions about the behavior of its environment.
\paragraph{Truthfullnes}
A good model is truthful: relevant properties of the model should also carry over to (hold for) the object of modelling. Typically, for each (relevant) behavior of the object of modelling there should be a corresponding behavior of the model, and/or for each behavior of the model there should be a corresponding behavior of the artefact. In the construction of models often idealizations or simplifications are necessary in order to allow for the use of a certain modeling formalism or in order to be able to analyze the model. In these cases, the model may not be entirely truthful. The modeller should always be explicit about such idealizations/simplifications, and have an argument why the properties of the idealized model still say something about the artefact. In the case of quantitative models this argument will typically involve some error margin. In the case of nondeterministic models it frequently occurs that a model ``overapproximates'' reality, and that certain behaviors that are possible in the model are not possible for the artefact.
\paragraph{Simplicity}
A good model is simple (but not too simple). Occam's razor is a principle particularly relevant to modelling: among models with roughly equal predictive power, the simplest one is the most desirable. Hence, the number of states and state variables should be as small as possible, and the level of atomicity of transitions should be as coarse grained as possible (but not coarser), i.e., the number of transitions should be minimal given the intended use of the model. Preferably, things should be written only once, and one should avoid ugly encodings. Preferably, the model uses stable, well-defined and well-understood concepts and semantics.
A good model is extensible and reusable, that is, it has been designed to evolve and be used beyond its original purpose. Typically, if one defines models in a modular and parametric way this allows for dimensioning, future extensions and modifications, especially if modules have well-defined interfaces. Ideally, a model should not just describe the specific system at hand: by appropriate instantiation and dimensioning it should be possible to model a whole class of similar systems.
\paragraph{intgeroperability}
A good model has been designed and encoded for interoperability and sharing of semantics. Model-driven development of an embedded system typically leads to a plethora of models, all presenting different views on and abstractions of the system. If a model is not somehow linked to other models, its usefulness will be limited. Ideally therefore, the relationships between all models should be properly defined, for instance via formal refinement relations.
Clearly, there are many relationships and dependencies between the criteria. If a model is traceble, that is, links between the structural elements of the model and the aspects of the object of modelling are clearly documented, then chances increase that the model will be thrutful. Also, if a model has been set up in a modular way, then one may apply a divide-and-conquer strategy both for establishing truthfulness of the model and for analysis. Etc, etc.

Last change made on 23/2/2010. Please send comments to Frits Vaandrager


http://www.cs.ru.nl/~fvaan/PV/what_is_a_good_model.html


artiekelen

Programming languages and energy efficiency’
Preliminary MW and MMW Reflection and Transmission Measurements of a Silicon Wafer under Illumination of Light for Reflected Phased Array Antennas
Looking at Hands in Autonomous Vehicles: A ConvNet Approach using Part Affinity Fields
Security of the Internet of Things: Vulnerabilities, Attacks and Countermeasures
Active Scan-Beam Reflectarray Antenna Loaded with Tunable Capacitator
\subsection{Requirements}

Om een goed verhaal op te stellen, moet vooraf aan enkele voorwaarden
worden voldaan. De eerste voorwaarde is de geschiktheid van het
afstudeerproject. Als een afstudeerproject niet tot keuzes leidt, kan
men zich afvragen of dat wel een echte afstudeeropdracht is. Een
afstudeerproject zonder onderzoeksaspecten is ook verdacht. Daarnaast
moet een afstudeerproject passen in het profiel van een opleiding om
beoordeelbaar te zijn. De andere voorwaarde voor goed een verhaal is
de registratie van werkzaamheden tijdens het a
\subsection{specificaties}

Om een goed verhaal op te stellen, moet vooraf aan enkele voorwaarden
worden voldaan. De eerste voorwaarde is de geschiktheid van het
afstudeerproject. Als een afstudeerproject niet tot keuzes leidt, kan
men zich afvragen of dat wel een echte afstudeeropdracht is. Een
afstudeerproject zonder onderzoeksaspecten is ook verdacht. Daarnaast
moet een afstudeerproject passen in het profiel van een opleiding om
beoordeelbaar te zijn. De andere voorwaarde voor goed een verhaal is
de registratie van werkzaamheden tijdens het a
\subsection{Het vier variabelen model}

Systemen (met daarin software) en de bijbehorende vier variabelen:
Monitored variabelen: door sensoren gekwanticeerde
fenomenen uit de omgeving
Controlled variabelen: door actuatoren \bestuurde"
fenomenen uit de omgeving
Input variabelen: data die de software als input gebruikt
Output variabelen: data die de software levert als output

\subsubsection{Monitored variabelen}

Om een goed verhaal op te stellen, moet vooraf aan enkele voorwaarden
worden voldaan. De eerste voorwaarde is de geschiktheid van het
afstudeerproject. Als een afstudeerproject niet tot keuzes leidt, kan
men zich afvragen of dat wel een echte afstudeeropdracht is. Een
afstudeerproject zonder onderzoeksaspecten is ook verdacht. Daarnaast
moet een afstudeerproject passen in het profiel van een opleiding om
beoordeelbaar te zijn. De andere voorwaarde voor goed een verhaal is
de registratie van werkzaamheden tijdens het a
\subsubsection{Controlled variabelen}

Om een goed verhaal op te stellen, moet vooraf aan enkele voorwaarden
worden voldaan. De eerste voorwaarde is de geschiktheid van het
afstudeerproject. Als een afstudeerproject niet tot keuzes leidt, kan
men zich afvragen of dat wel een echte afstudeeropdracht is. Een
afstudeerproject zonder onderzoeksaspecten is ook verdacht. Daarnaast
moet een afstudeerproject passen in het profiel van een opleiding om
beoordeelbaar te zijn. De andere voorwaarde voor goed een verhaal is
de registratie van werkzaamheden tijdens het a
\subsubsection{Input variabelen}

Om een goed verhaal op te stellen, moet vooraf aan enkele voorwaarden
worden voldaan. De eerste voorwaarde is de geschiktheid van het
afstudeerproject. Als een afstudeerproject niet tot keuzes leidt, kan
men zich afvragen of dat wel een echte afstudeeropdracht is. Een
afstudeerproject zonder onderzoeksaspecten is ook verdacht. Daarnaast
moet een afstudeerproject passen in het profiel van een opleiding om
beoordeelbaar te zijn. De andere voorwaarde voor goed een verhaal is
de registratie van werkzaamheden tijdens het a
\subsubsection{Output variabelen}

Om een goed verhaal op te stellen, moet vooraf aan enkele voorwaarden
worden voldaan. De eerste voorwaarde is de geschiktheid van het
afstudeerproject. Als een afstudeerproject niet tot keuzes leidt, kan
men zich afvragen of dat wel een echte afstudeeropdracht is. Een
afstudeerproject zonder onderzoeksaspecten is ook verdacht. Daarnaast
moet een afstudeerproject passen in het profiel van een opleiding om
beoordeelbaar te zijn. De andere voorwaarde voor goed een verhaal is
de registratie van werkzaamheden tijdens het a


\subsection{Specificaties}

Voor alle paden geldt als een schip vertrekt is de sluisdeur dicht. 
Voor alle paden geldt als stoplicht op rood, sluisdeuren dicht 
Is een schip vertrokken dan is de nivelleermachine uit. 
Er is geen pad waarvoor geldt dat een schip vertrekt vanuit de rechtersluisdeur en de linkersluisdeur is open, linkeruitvaartstoplich en de linkerinvaarstoplicht op groen en nivelleermachine is aan. 
Er is een pad waarvoor geldt dat de linkersluisdeur dicht is, de rechtersluisdeur dicht is, de linkerinvaarstoplicht is gelijk aan rood, linkeruitvaarstoploicht is op rood en rechteruitvaarstoplicht is op rood en rechterinvaarstoplicht op rood terwijl er geen schip in de sluis ligt. 

Geen deadlock 
• Voor geen enkel pad geldt dat als de deuren gesloten zijn volgens de kluis dat er een deur openstaat om 
een schip naar buiten te laten. 
• Voor alle paden geld dat als een sluis aan het voorbereiden is, dan zijn alle duren dcht. 
• Voor alle paden geld dat als een deur dicht is het aantal schepen in de kade gelijk is aan nul 
• Voor een enkel pad geld dat als het binnenstoplicht op groen staat dat het niet toegestaan in naar binnen 
te varen 
• Voor alle paden geldt dat de globale tijd langer is dan 30 tijdseenheden 
• Er is een pad waarvoor geld dat als een schip wilt stoppen dat er meer dan 5 schepen in de sluis zitten. 
• Voor alle paden geldt als schip vrtrekt is sluisdeur dicht 
• Voor alle paden geldt als stoplicht op rood sluisdeuren dicht en schip vertrollen dan is de nivelleermachine 
uit 
• Er is geen pad waarop een schip vertrekt vanuit de rechtersluisdeur en de linkersluisdeur is open en 
linkeruitaartstoplicht en linkeruitvaartsoplicht opgroen en nibelleermachine is aan 
• Er is een pad waarvoor geldt dat linkerslsuisdeuren dicht zijn, rechtersluisdeuren dicht zijn rechteruit- 
vaartstoplicht is rood en rechteruitvaartstoplicht is rood terwijl eer geen schip in de sluis licht 
• EEn stopluch staat altijd op groen als de deuren open staan en de pomp niet bezig is. 
• In geen enkele staat van de sluis behalve tussen de lowergate en uppergate en uppergate en lowergate en 
de staten AtArrivalLow en AtEnteringHigh is de wachttijd langer dan 5 tijdseenheden 
• Voor alle paden in een pomp geldt dat als water level lager is dan waterlaag pompwaterweg is altijd false 
• Voor alle paden gelft dat als water level hoger is dan waterhoog dan is pompwater altjd false 
• Het zal nooit gebeuren dat een pomp water toevoegt als deuren open zjn, geen schip in sluis en stoplicht 
op groen 
• Het kan gebeuren dat bij pompr het stoplicht op rood staat, het schip in de sluis en deur is dicht, en 
waterstand gelijk aan waterlaag 
• Er is een mogelijkheid dat vanuit pomp get stoplicht op rood wordt gezet en waterlevel gelijk is aan 
waterlaag 
• Het kan voorkomen dat bij state pompaan het waterniveau gelijk is aan waterlaag 
• Voor alle paden gelt dat er een mogelijkheid is dat deur is open/dicht en sluis nivelleert omhoog/omlaag 
14 




Kripke strucuten


De set van initiele staten

De transities tussen de states

De state labeling fuctie voor elke state met een set van atomischr proosities die waar zijn voor een state

Met D ={0,1};
S = D*D
S0 = {(1,1)}
R=
L
