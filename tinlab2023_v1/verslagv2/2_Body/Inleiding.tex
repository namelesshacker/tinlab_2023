
\newpage
\section{Introducie}
Zie hier een referentie naar Royce~\cite{royce1987managing} en nog een naar Clarke~\cite{modelchecking}\ldots 

\subsection{Inleiding}
Om een goed verhaal op te stellen, moet vooraf aan enkele voorwaarden
worden voldaan. De eerste voorwaarde is de geschiktheid van het
afstudeerproject. Als een afstudeerproject niet tot keuzes leidt, kan
men zich afvragen of dat wel een echte afstudeeropdracht is. Een
afstudeerproject zonder onderzoeksaspecten is ook verdacht. Daarnaast
moet een afstudeerproject passen in het profiel van een opleiding om
beoordeelbaar te zijn. De andere voorwaarde voor goed een verhaal is
de registratie van werkzaamheden tijdens het a
\subsection{Achtergrond}


Het ministerie van verkeer en Waterstaat wil in het kader van het klimaatakkoord en onderzoek laten uitvoeren naar de staat van het sluizenpark in Nederland. Het onderzoek moet zich richten op het ontwerpen en ontwikkelen van een geautomatiseerd sluismodel dat geschikt is voor een brede toepassing. In het onderzoek moet naar voren komen wat de huidige staat is van de sluizen met oog op veiligheid, efficiëntie, capaciteit, onderhoud, duurzaamheid en automatisering. Het onderzoek geeft aan hoe een volledig model worden opgeleverd opdat ontwerp van verschillend volledig geautomatiseerde sluizen in de toekomst geautomatiseerd kunnen worden.  


\subsection{Probleemanalyse} 





Na grondige analyse van het Nederlandse sluizen park is gebleken dat renovatie van een groot aantal sluizen noodzakelijk is.  Uit een eerste verkenning is gebleken  dat het gecombineerd renoveren en automatiseren van het Nederlandse Sluizen Park een aanzienlijke verbetering kan opleveren t.a.v. 
Op  het  ministerie  van  infrastructuur  en waterstaat is helaas onvoldoende kennis van ict en systemen aanwezig om eenen ander uit te voeren 



Waarom nu 
In  het  kader  van  het  onlangs  afgesloten  klimaatakkoord  heeft  de  Nederlandse Overheid  daarom  besloten  over  te  gaan  tot  een  ingrijpende  renovatie  van  de diverse  sluizen  die  ons  land  rijk  is.     

\subsection{Gewenste resultaat} 

Wij vragen u een model (of een onderling samenhangend aantal modellen)aan  te  leveren,  opdat  ontwerpen  van  verschillende,  volledig  geautomatiseerde sluizen in de toekomst gerealiseerd kunnen worden. 
Zoals  gesteld  in  de  brief  is  het  de  bedoeling  dat  een  sluis  gemodelleerd  worden  dat  bewezen  kan  worden  dat  de  te  bouwen  sluis  een  aantal  eigenschappen bezit.  

\subsection{Scope} 
Je krijgt daarbij veel vrijheid om zelf keuzes te maken.  Daaruit volgt: 
dat de onderbouwing van die keuzes even belangrijk is als het uiteindelijke resultaat. 
dat er ook zaken mogen mislukken.  Dat iets mislukt is niet erg, gesteld dat duidelijk maakt waarom het mislukt is en wat geprobeerd is.  Redenen van  falen  zijn  doorgaans  even  interessant,  zo  niet  interessanter,  als  desuccessen. 


\subsection{Deliverables}
 
Soortgelijke problemen (automatiseren openbare werken en infrastructuur, maintenance and automation in public
 infrastructure) 
 
 
 \subsection{research design}
 
 
 
 \begin{forest}
 	area/.style={%
 		fill=gray!10,draw,
 	},
 	method/.style={%
 		thick,
 		edge+={-Latex},
 	},
 	dir tree switch forking=at 2,
 	for tree={
 		draw,
 		align=center,
 		thin,
 		minimum height=1.5em,
 	},
 	[Tenant
 	[Customer user\\RBAC
 	[Roles\\Permissions
 	[Provisioning\\Permissions
 	[Device profile
 	]
 	]
 	]
 	[Dashboard\\ Alias navigator
 	[Device profile
 	[Rulechain
 	]
 	]
 	[Telemetry dashbpoard
 	]
 	[RPC dashboard
 	]
 	
 	]
 	[User group
 	[End user dashboard][IoT Device][device profile]
 	]
 	]
 	
 	
 	]
 \end{forest}

\subsection{Onderzoeksvragen}
 

Uit het onderzoek zal moeten blijken welke veiligheidseisen er zijn voor sluizen in nederland. Daarnaast welke factoren een rol spelen in de duurzaamheid van het sluispark.  
Hoe wordt de routinecontrole op de sluizen uitgevoerd?  
Welke automatisering is mogelijk met oog op veiligheid, efficientie en capaciteit?  
Welke criteria wegen zwaar in de ontwikkel- en onderhoudskosten van duurzame technologie?  

\subsection{Leeswijzer}


Contents
Certificate 
Declaration  
Abstract  
List of Abbreviations  
List of Figures 
INTRODUCTION  
Methodology 
Objective/Problem Statement  
SYSTEM DEVELOPMENT 
Block Diagram and Working  
HARDWARE AND SOFTWARE 
Hardware 
Component layout  
PCB Layout  
Software  
CONCLUSION  
Result and Performance  
Conclusion  
Application 
Future Scope  
REFERENCE
