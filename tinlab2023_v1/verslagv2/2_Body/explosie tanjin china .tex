
\subsubsection{explosie tanjin china 
}

Later bleek uit een onderzoek van de Chinese autoriteiten dat de explosie overeenkwam met de ontploffing van 450 ton TNT.[6] 
De oorzaak van de explosie lag in de spontane zelfontbranding van 207 ton cellulosenitraat dat in containers was opgeslagen op het terminalterrein.[6] 
Verder lag op een tweede locatie nog eens 26 ton van dit explosieve materiaal opgeslagen.
De tweede ontploffing werd versterkt door de opslag van 800 ton kunstmest in de vorm van ammoniumnitraat in de nabijheid.[6]
De opslag van cellulosenitraat is aan strenge regels gebonden. Het moet koel en droog worden opgeslagen. De containers stonden buiten opgesteld in de brandende zon. De temperatuur liep op tot 36 °C en bereikte binnen de containers waarschijnlijk de 65 °C.[6] De verpakking van de cellulosenitraat droogde uit waardoor de ontploffing kon ontstaan. Op het terrein lagen meer gevaarlijke stoffen opgeslagen dan waarvoor vergunningen waren verstrekt.[6] Dit leidde tot een kettingreactie met grote schade tot gevolg. Door de brand en bluswater is in de directe omgeving veel milieuschade opgetreden.


https://www.hindawi.com/journals/joph/2019/1360805/ 