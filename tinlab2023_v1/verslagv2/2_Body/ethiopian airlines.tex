

\subsubsection{ethiopian airlines}
Ethiopian Airlines Flight 302
Door problemen met de flight control
One minute into the flight, the first officer, acting on the instructions of the captain, reported a "flight control" problem to the control tower.
Two minutes into the flight, the plane's MCAS system activated, pitching the plane into a dive toward the ground. The pilots struggled to control it and managed to prevent the nose from diving further, but the plane continued to lose altitude.
The MCAS then activated again, dropping the nose even further down. The pilots then flipped a pair of switches to disable the electrical trim tab system, which also disabled the MCAS software. However, in shutting off the electrical trim system, they also shut off their ability to trim the stabilizer into a neutral position with the electrical switch located on their yokes. The only other possible way to move the stabilizer would be by cranking the wheel by hand, but because the stabilizer was located opposite to the elevator, strong aerodynamic forces were pushing on it.
As the pilots had inadvertently left the engines on full takeoff power, which caused the plane to accelerate at high speed, there was further pressure on the stabilizer. The pilots' attempts to manually crank the stabilizer back into position failed.
Three minutes into the flight, with the aircraft continuing to lose altitude and accelerating beyond its safety limits, the captain instructed the first officer to request permission from air traffic control to return to the airport. Permission was granted, and the air traffic controllers diverted other approaching flights. Following instructions from air traffic control, they turned the aircraft to the east, and it rolled to the right. The right wing came to point down as the turn steepened.
At 8:43, having struggled to keep the plane's nose from diving further by manually pulling the yoke, the captain asked the first officer to help him, and turned the electrical trim tab system back on in the hope that it would allow him to put the stabilizer back into neutral trim. However, in turning the trim system back on, he also reactivated the MCAS system, which pushed the nose further down. The captain and first officer attempted to raise the nose by manually pulling their yokes, but the aircraft continued to plunge toward the ground.

https://www.hindawi.com/journals/ijae/2014/472395/ 