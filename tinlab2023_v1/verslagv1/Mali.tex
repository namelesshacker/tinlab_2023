\subsubsection{Mali}
Een granaat explodeerd in een mortier
De medische zorg na het ongeval was neit voldoende


De algemeen militair verpleegkundige gaf aan het slachtoffer nar het vn-hospitaal in kidal te brengen
De chaauffeur van de bushmaster kende de locatie niet  en bracht het slachtoffer naar een door frane militairren bemand hospitaal mmet minder mediswche faciliteiten
Hierna alsnog overgebracht naar het vn-hospitaal.
Dit verlieop  neit door nederlandse maatstaven.
pas toen een nederlandse arts arrivveerde werd door de Tongolese artsen een buikoperatie uitgevoerd.
Dit gebrurde zonder adequate anesthesie.
Na de operatie werde de gewonde militair overgelogen naar nederland. En later naar nederland.


granaat stond niet op scherp en in afgegaan in veilige stand
Granaat werd opgeslagen in neit gekoelde containers waardoor deze aan te hoge temeperaturen zijn blootgesteld.
Door de comvinatie van vocht en warmte in de granaat zeer gevoelige explosieve stoffen werden gevormd.
Tijdens de oefening was de fatale granaat in de zon.
Het afsluitplaatje in de granaat bleek niet in staat om doorslag in veilige stand te voorkomen waarna de granaat explodeerde.
De moritren zijn aangeschaft bij de amerikanen. gredurende de aanschafperiode zijn procedures en controles op kwaliteit en veiligheid deels nagelaten.
Dit veiligheidsgarantie werd vermeld in het koopcontract.
Conclusie
Koopcontract werd niet goed doorgelezen
Geen controle op kwaliteit en veiligheid
Geen controle op kwaliteit en veiligheid
Zwakke plekken in het ontwerp
Geen controle op kwaliteit en veiligheid
opslag en gebruik in ongunstige condities

De aanwezige medische voorzieningen waren nite volgends de nederlandse militaire richtlijnen
Het ontbreek aan medische toetsing vanuit de defensie organisatie
twijfels die werden geuit binnen de defensieorganisae vonden geen wrrklank
Ok het ongeval tijdens de mortieroefening was voor defensie geen aanleuiding om de medische voorzienignen te evalueren.
De inrichting van veilige medische zorg voor nederlandse militairen in kidal is ondergeschikt gemaakt aan de voortgang van de missie.


https://www.youtube.com/watch?v=PC2ekl4SaNA 





\subsubsection{Mali}
Een granaat explodeerd in een mortier
De medische zorg na het ongeval was neit voldoende


De algemeen militair verpleegkundige gaf aan het slachtoffer nar het vn-hospitaal in kidal te brengen
De chaauffeur van de bushmaster kende de locatie niet  en bracht het slachtoffer naar een door frane militairren bemand hospitaal mmet minder mediswche faciliteiten
Hierna alsnog overgebracht naar het vn-hospitaal.
Dit verlieop  neit door nederlandse maatstaven.
pas toen een nederlandse arts arrivveerde werd door de Tongolese artsen een buikoperatie uitgevoerd.
Dit gebrurde zonder adequate anesthesie.
Na de operatie werde de gewonde militair overgelogen naar nederland. En later naar nederland.


granaat stond niet op scherp en in afgegaan in veilige stand
Granaat werd opgeslagen in neit gekoelde containers waardoor deze aan te hoge temeperaturen zijn blootgesteld.
Door de comvinatie van vocht en warmte in de granaat zeer gevoelige explosieve stoffen werden gevormd.
Tijdens de oefening was de fatale granaat in de zon.
Het afsluitplaatje in de granaat bleek niet in staat om doorslag in veilige stand te voorkomen waarna de granaat explodeerde.
De moritren zijn aangeschaft bij de amerikanen. gredurende de aanschafperiode zijn procedures en controles op kwaliteit en veiligheid deels nagelaten.
Dit veiligheidsgarantie werd vermeld in het koopcontract.
Conclusie
Koopcontract werd niet goed doorgelezen
Geen controle op kwaliteit en veiligheid
Geen controle op kwaliteit en veiligheid
Zwakke plekken in het ontwerp
Geen controle op kwaliteit en veiligheid
opslag en gebruik in ongunstige condities

De aanwezige medische voorzieningen waren nite volgends de nederlandse militaire richtlijnen
Het ontbreek aan medische toetsing vanuit de defensie organisatie
twijfels die werden geuit binnen de defensieorganisae vonden geen wrrklank
Ok het ongeval tijdens de mortieroefening was voor defensie geen aanleuiding om de medische voorzienignen te evalueren.
De inrichting van veilige medische zorg voor nederlandse militairen in kidal is ondergeschikt gemaakt aan de voortgang van de missie.


\url{https://www.youtube.com/watch?v=PC2ekl4SaNA }
