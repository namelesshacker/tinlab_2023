

\subsubsection{Ramp schietpartij militair ossendrecht }
Een militaire overleid op een schietbaan in ossendracht door onvoldoende begeleiding van cursisten, geen toezicht op de lokatie. E\r was een instructuur in opleiding die niet volledig was mmeegenomen in het poroces en ook was er geen baancommandant aanwezig. Geen van de aanwezig instructeurts had de juiste papieren om de cursisten te begeleiden. De aanwezig instruceur had geen zich op de instructeur in opleiding, evenmin de andere militairen. In de instructiehandleiding ontbreken richtlijnen voor bijzondere schietbanen. Ook was er geen keuring. Door personelstekort is er geen andacht besteed aan documentastie(een slyllabus) hoe en met welke risico’s oefeningnen moeten worden ingericht. Ok werd er vooraf geen veiliheidsanaklyse gedaan. Het gebrek aan lesmateriaal en deskundigen is gemeld binnen de defensieorganisatie maar dit heeft niet geleid tot enige verandering in de situatie.
Op een afgekeurde scheitbaan
Tezicht door een instructeur in opleiding die zelf geen persoonlijke begeleiding heeft gehad tijdens de uitvoering
Belangrijk is dat defensie haar taken kan uitvoeren met personeel dat is getraind in situaties die de risicos van de werkomgeving aan de cursisten kunnen laten zien.
Conclusie
Zonder gekwalificeerde instructuers.
Zonder toezicht
Zonder lesmateriaal
Zonder adequate veiligheidsanalyse
https://www.youtube.com/watch?v=6jmkDClGDHo 