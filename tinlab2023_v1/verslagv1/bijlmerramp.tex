

\subsubsection{bijlmerramp}
Motor 3 (de binnenste motor aan de rechtervleugel van het vliegtuig) brak af, beschadigde de vleugelkleppen en botste tegen motor 4 die vervolgens ook afbrak.
De ernst van de situatie werd op Schiphol niet goed ingezien. Dit kwam onder meer doordat lost in de luchtvaart de gebruikelijke term is om het verlies van motorvermogen te melden. Op Schiphol werd er dan ook van uitgegaan dat er twee motoren waren uitgevallen. Dat ze letterlijk verloren waren wist men niet. Gezien het grote aantal handelingen dat de bemanning in een paar minuten moest uitvoeren en de keuzes die de piloot maakte, veronderstelde de parlementaire enquêtecommissie die de ramp later zou onderzoeken dat ook de bemanning waarschijnlijk niet heeft geweten dat beide motoren van de rechtervleugel waren afgebroken. De buitenste motor van een 747 is vanuit de cockpit slechts met moeite zichtbaar en de binnenste motor helemaal niet.

Op de avond van de 4e oktober 1992 was landingsbaan 06 (de Kaagbaan) in gebruik. De piloot verzocht de luchtverkeersleiding op Schiphol echter een noodlanding te mogen maken op de Buitenveldertbaan (baan 27). Waarom hij juist deze baan koos, is nooit duidelijk geworden. Een keuze voor deze baan lag niet voor de hand; omdat de wind uit het noordoosten kwam, zou het toestel met flinke staartwind moeten landen. Langs de landingsbaan waren enkele grote brandweerwagens van Schiphol geplaatst. Deze zogeheten crashtenders moesten een brand tijdens de landing meteen blussen. Na de crash werd één zwarte doos teruggevonden. De bijbehorende band was in vier stukken gebroken, waardoor de laatste 2 minuten en 45 seconden ervan niet meer te gebruiken waren. De doos werd voor onderzoek naar Washington gestuurd en leverde uiteindelijk onderstaande informatie op.
Om goed uit te komen voor de landingsbaan vloog het beschadigde toestel eerst nog een rondje boven Amsterdam. Tijdens dit rondje gaf de gezagvoerder de copiloot opdracht de vleugelkleppen (flaps) uit te schuiven. Links schoven de kleppen uit, maar doordat de afgebroken motor 3 de rechtervleugel had beschadigd schoven de kleppen op die vleugel niet uit. Als gevolg hiervan kreeg het toestel links meer draagvermogen dan rechts. De piloot meldde aan de verkeersleiding dat er ook problemen met de flaps waren.
Aanvankelijk ging het aanvliegen van de Buitenveldertbaan goed. Op het moment dat het vliegtuig daalde tot onder de 1500 voet en snelheid minderde, raakte het echter compleet onbestuurbaar en maakte het een ongecontroleerde, scherpe bocht naar rechts. Over de radio was te horen dat de gezagvoerder zijn copiloot in het Hebreeuws opdracht gaf om alle kleppen in te trekken en het landingsgestel uit te klappen. Vervolgens meldde de copiloot in het Engels aan de luchtverkeersleider dat het toestel zou gaan neerstorten. Uit later onderzoek bleek dat het vliegtuig eerder enkel recht bleef vanwege de hoge snelheid (280 knopen, zijnde 519 km/u). Doordat de rechtervleugel beschadigd was, was het moeilijker om het vliegtuig recht te houden. Alleen de hoge snelheid zorgde ervoor dat er nog voldoende draagvermogen was. Toen bij het inzetten van de landing de snelheid verlaagd werd, werd het draagvermogen van de rechtervleugel echter dusdanig gering dat het toestel niet meer onder controle te houden was en een duikvlucht naar rechts maakte.

https://aviation-safety.net/database/record.php?id=19921004-2&lang=nl 