
 



\hoofdstuk{Theoretisch kader}

In het eerste hoofdstuk is duidelijk geworden wat de onderzoeksvraag is, namelijk ‘Hoe kan een geautomatiseerde sluis worden gemodeleerd met oog op ontwikkel- en onderhoudskosten,veiligheid, efficientie en capaciteit’. Door de toenemende complexiteit van systemen is het gebruik van modellen en de toepassing van timebased model checking  op industriele controle systemen een manier van modelleren van het systeem en de requirements zodat er een bijdagre kan worden geleverd aan de acceptatie van  simulatie-/modeltechniek voor de industrie.(‘https://link.springer.com/article/10.1007/s10626-020-00314-0’, 2020). Of dit ook het geval is bij het modellereren van sluizen is nu de vraag.

De bestudering van rampen aan de hand van het vier-variabelen model biedt maakt het analyseren mogelijk van rampsituaties. Van een aantal rampen is een beschrijving gegeven met datum, plaats en oorzaak. De analyse van de 4-variabelen modellen zal gebruikt worden voor de requirementsdefinitie, ontwerp en ontwikkeling van het sluismodel. 

De verschillende factoren en achtergronden die  samenhangen met het modelleren van een sluis zullen in dit hoofdstuk toegelicht worden. Bovendien worden er hypotheses gevormd die de basis vormen voor debeantwoording van de onderzoeksvraag. 




\paragraph{Wat is uppaal}

Wat is Uppaal
Uppaal is an integrated tool environment for modeling, simulation and verification of real-time systems, developed jointly by Basic Research in Computer Science at Aalborg University in Denmark and the Department of Information Technology at Uppsala University in Sweden. It is appropriate for systems that can be modeled as a collection of non-deterministic processes with finite control structure and real-valued clocks, communicating through channels or shared variables [WPD94, LPW97b]. Typical application areas include real-time controllers and communication protocols in particular, those where timing aspects are critical.


model checking

Wat is statistical model checking?
Dit verwijst naar verschillende technieken dfie worden gebruikt voor de monitoring van een systeem. Daarbij wordt vooral gelet op een specifieke eigenschap. Met de resultaten van de statsitieken wordt de juistheid van een ontwerp beoordeeld. Statistisch model checking wordt onder andere toegepast in systeembiologie, software engineering en industriele toepassingen.
https://www-verimag.imag.fr/Statistical-Model-Checking-814.html?lang=en#:~:text=Statistical%20Model%20Checking%20(SMC)%20is,from%20state%20space%20explosion%20issues.


\cite{inriaStatsMoodCheck}
\cite{ buddeModelChecker}
\cite{AGHASuervey }


Waarom gebruiken we statistisch model checking?
To overcome the above difficulties we propose to work with Statistical Model Checking [KZHHJ09,You05,You06,SVA04,SVA05,SVA05b] an approach that has recently been proposed as an alternative to avoid an exhaustive exploration of the state-space of the model. The core idea of the approach is to conduct some simulations of the system, monitor them, and then use results from the statistic area (including sequential hypothesis testing or Monte Carlo simulation) in order to decide whether the system satisfies the property or not with some degree of confidence. By nature, SMC is a compromise between testing and classical model checking techniques. Simulation-based methods are known to be far less memory and time intensive than exhaustive ones, and are oftentimes the only option. 
https://project.inria.fr/plasma-lab/statistical-model-checking/

Alternatief
Alternatieven voor Uppaal zijn Asynchronous Events,Vesta en MRMC.


\paragraph{MODE CONFUSION }
Mode confusion tredd op als gepbserveerd gedrag van een technisch systeem niet past in het gedragspatroon dat de gebruiker in zijn beeldvorming heeft  en ook niet met voorstellingsvermogen kan bevatten.
\paragraph{Wat is automatiseringsparadox}
Gemak dient de mens. Als er veel energie wordt gestoken in de ontwikkeling van hulmiddelen die taken van werknemers overemen heeft dat tot resultaat dat veel productieprocessen worden geautomatiseerd. De vraag is dan of vanuit mechnisch wereldpunt de robot niet de rol van de mens overneemt en of de mens nog de kwaliteiten heeft om het werk zelf te doen.
\cite{bicker21102016automatiseringsparadox }
\cite{vseautoparadox }
\cite{blogxot21112016slimapparaat }



\paragraph{4 variabelen model}





Het 4 variabelen model kort toegelicht
Monitored variabelen: door sensoren gekwantificeerde fenomenen uit de omgeving, bijv temperatuur

Controlled variabelen: door actuatoren \bestuurde fenomenen uit de omgeving
For example, monitored variables might be the pressure and temperature
inside a nuclear reactor while controlled variables might be visual and audible alarms, as well
as the trip signal that initiates a reactor shutdown; whenever the temperature or pressure reach
abnormal values, the alarms go off and the shutdown procedure is initiated

Input variabelen: data die de software als input gebruikt
Here, IN models the input hardware interface (sensors and analog-to-digital converters) and
relates values of monitored variables to values of input variables in the software. The input variables model the information about the environment that is available to the software. For example,
IN might model a pressure sensor that converts temperature values to analog voltages; these voltages are then converted via an A/D converter to integer values stored in a register accesible to the
software.

Output variabelen: data die de software levert als output
The output hardware interface (digital-to-analog converters and actuators) is modelled
by OUT, which relates values of the output variables of the software to values of controlled variables. An output variable might be, for instance, a boolean variable set by the software with the
understanding that the value true indicates that a reactor shutdown should occur and the value
false indicates the opposite



\paragraph{6 Variable model}
Optitatieve statements omschrijven de omgeving zoals we het willen zien vanwege de machine. 

Indicatieve statements omschrijven de omgeving zoals deze is los van de machine. 

Een requirement is een optitatief statement omdat ten doel heeft om de klantwens uit te drukken in een softwareontwikkel project. 

Domein kennis bestaut uit indicatieve uitspraken die vanuit het oogpunt van software ontwikkeling relevant zijn. 

Een specificatie is een optitatief statement met als doel direct implementeerbaar te zijn en ter verondersteuning van het natreven vande requirements. 

Drie verschillende type domeinkennis: domein eigenschappen, domein hypothesen, en verwachtingen. 

Domein eingenschappen  zijn beschrijvende statementsover een omgeving en zijn feiten.Domein hypotheses  zijn ook beschrijvende uitspraken over een omgeving, maar zijn aannames. 

Verwachtingen zijn ook aannames, maar dat zijn voorschrijvende uitspraken die behaald worden door actoren als personen, sensoren en actuators. 

  
\paragraph{Conceptueel model}



System requirement:
uitspraak over wereld fenomenen (gedeeld of niet) of doelen
die bereikt moeten worden.
met enige regelmaat informeel, niet precies geformuleerd.
Software requirement/specicatie:
uitspraak over gedeelde fenomenen of doelen die de machine
moet bereiken middels de onderdelen waar die machine uit
bestaat of middels de fenomenen waar de machine controle
over heeft.
doorgaans preciezer, meetbaar, exact geformuleerd.


Systemen gaan een zekere interactie aan met hun omgeving:
Sensoren: meten fenomenen uit de omgeving (temperatuur,
druk, licht, geluid, etc.)
actuatoren: veranderen iets in de omgeving (mechanische,
electrisch, pneumatisch, etc.)
Software:
Kan niet direct communiceren met de buitenwereld.
Snapt derhalve niets van de buitenwereld.
Kan alleen maar bestaan in en communiceren met het
systeem.


\paragraph{Requirementsengineering}

Om de juiste requirements te verzamelen en selecteren hebben we meer kennis nodig van de methoden hiervoor gebruikt in het domein van requirementsengineering. Daarom is een literatuurstudie gedaan naar rapporten en artikelen die ons meer informatie over dit onderwerp verschaffen.
 Uitdagingen in requirementsengineering zijn incomplete requirements en specifcates, veranderende requirements en specificates en grote, complexe oftwaresystemen.
 
 Het article the worlds a stage biedt inzicht in de requirementstechnieken voor een ambulance in london. In het artikel gaan de onderzoeks in op de volgende onderwerpen: 
 viewpoints, sociale ascpecten,evolutie, non-functional requirements, conflict resolution, traceability
 
 Goal of this paper is requirement  engineering on London aulance service
 Method of opinions: crew, staff, management, computational, transport, services
 Evolutioon: changes, specification and technology trade
 Environment: company policies, regulation, impact solution on organizational
 Non-functional aspect: communicatio problem, malfunctions, less critical isues: cost, tradeoff beween performance \& user interfaces
 vieuwpoint: is a subset of all system requirements expressible in a given requirements notation regardless of the stakeholders involved
 
 log change
 basic model vieuw
 hypertext vieuw
 data transmission problems
 continued difficulties
 installation problems
 problems caused by mistake
 tracebility requirements[selecting reliable information]
 PRE requirement specification traceability, repository baed approach
 1) compromise specification
 2) representatives
 3) agreement dimensions
 Domain: part of the worl in which the computer system effects will be felt, inclusing its peoples, organizational structure, related legislation, physical location and met only the compyter systems
 
 
 Het artikel "from inconsistencyhandling to non-conanical requirements management: a logical perspective" geeft enkele tips voor het omgaan met inconsistente requirements:
 
 1) identifying non-canonicalrequirements
 2) measuring them
 3) generate caandidate proposals for handling them
 4) choosing acccptable probosals
 5) revising them acccording to the proposals

Het artikel "managing inconsistent specification: reasoning, analysis, action" zoekt een ontologische benadering voor het omgaan met inconsistenties in de requirements specificaties.
Voor de omshrijving van een specificatie kun je gebruik maken van logica. Daarbij kun je onderschei maken in klasieke logica quasi -logica.
Wat ook een rol kan spelen in domain interpretatie. De achtergrond van de gebruikers speelt ook een rol.
Zo is er e=onderscheid te maken in de volgende groepen: users, customers, domain experts, designers,, manufacturers
graphical  textual specification

Basic constraint, legal constraint, cooperation constraint
1) scenatio  definition
2) scenario analysis
3) scenario consolidation


Hoe kan een systeem verder worden ontworpen op een manier dat non-functionele requirements worden geimplementeerd?
Hoe hangt dat ontwerp samen met aanpassingen van het functionele en structurele aspect van het systeem?

block[objects, classes, methods, messages, inheritance]
[goals,agents, alternative, events, actions,existence modalities,agent responsibilities]


Het artikel "representing and using nonfunctional requirements: a process-oriented approach"" gaat in op een het proces van requirements acquisitie. Hierbij in ogenschouw de acquisitie van prestaties, ontwerp en aanpasbaarheid.
product oriented
process oriented


Acquisitie Prestaties
user concern
-Hoe goed werkt het product
-Hoe goed wordt de bron gebruikt?>> Efficiency
-How veilig is het product >> integrity
-Met hoeveel zekerheid is uit  te sluiten dat het werkt >>Reliability
-Hoe goed werkt het product onder zware omstandigheden >> sustainability
-Hoe makkelijk is het product in gebruik >> usability
quality attribute


Acquisitie: Ontwerp
user concern
Hoe valide is het ontwerp
-Is ht ontwerp conform de requirements
-hoe makkelijk is het ontwerp te repareren
-Hoe makkelijk zijn de prestaties te verifieren

quality attribute


Acquisitie: Aanpasbaarheid
user concern
-hoe makkelijk is het om het product aan te passen
- hoe makkelijk is het om het product te updaten en/of uitbreiden>> expendability
- hoe makkelijk is het om een wijziging door te voeren>>flexibility
-hoe makkelijk is het om andere system aan te sluiten >> portability
- hoe makkelijk is het om het product te transporteren >> interoperability
-hoe makkelijk is het om te converteren tot een systeem gebruiksklaar voor communiceren met andere systemen>> reaseability
quality attribute




 \cite{jonkerTreurKlush200informativeAgents}
\cite{boehmBoseLeeRequirementsNegotiations}
\cite{muHungJinLiu2013inconsistencyReqs}
\cite{hunterNuseibeh1996manageSpecs}
\cite{myloloupos1992representingReqs}
\cite{zavePamela4darkCorners}
\cite{zavePAmela1997regEngineering}

%%%%%%%%%%%%%%%%%%%%%%%%%%%%%%%%%%%%%%%%%%%%%%%%%%%%%%%%%%%%%%%%%

what is a good software specification

\cite{fvaandrager2322010Goodmodel}
\cite{onix01102022devopmodel}
\cite{sulemani04012021softwareprocesmodel}
\cite{globalluxsoft18102017softdev}
\cite{wiegers30052022SRS}
\cite{muller06092020goodspecification}
\cite{informit30062008reqmanagement}
\cite{altexsoft15092020writingSRS}


\paragraph{Wat is een sluis}

\paragraph{Recente ontwikkelingen op het gebied van sluisautomatisering}

Het ministerie van verkeer en Waterstaat wil in het kader van het klimaatakkoord en onderzoek laten uitvoeren naar de staat van het sluizenpark in Nederland. Het onderzoek moet zich richten op het ontwerpen en ontwikkelen van een geautomatiseerd sluismodel dat geschikt is voor een brede toepassing. In het onderzoek moet naar voren komen wat de huidige staat is van de sluizen met oog op veiligheid, efficiëntie, capaciteit, onderhoud, duurzaamheid en automatisering. Het onderzoek geeft aan hoe een volledig model worden opgeleverd opdat ontwerp van verschillend volledig geautomatiseerde sluizen in de toekomst geautomatiseerd kunnen worden.  


\paragraph{Studie naar rampen aan de hand van het vier variabelen model}
\newline Voor deze studie is onderzoek gedaan naar verschillende rampen aan de hand van het vier variabelen model.
Elke ramp op deze manier categoriseren  kan ons helpen te bepalen in hoeverre requirements een rol kunnen spelen in de veiligheid van ons model.  Zo is er de bijlmerramp \cite{aviationsafety04101992airplaneCrashBijlmer}
, deze vond plaats op 04/10/1994. Dan nog de  ramp turkisch airlines vlucht 1951 op woensdag 25 februari 2009 \cite{catsr25022009Boeing737AmsterdamCrash}
\cite{zuilen23022019Tijdlijnpoldercrash}
\cite{wikinews04032009techfoutailines1951}
\cite{luchtvaartnieuws21012020boeing737conclusies}
\cite{adformatie280220209communicatiegebreken}
\cite{spinnael25022009onderzoekpolderbaancrash}
\cite{crashTurkishAirlines}
\cite{flightradar24}
\cite{flightstatstracker}. 
%%%%%%%%%%%%%%%%%%%%%%%%%%%%%%%%%%%%%%%%%%%%%%%%%%%%%%%%%%%%%%%%%
\newline \indent
De therac-25 June 1985 and January 1987. 
Medical lineair accelerators accelerate electrons to createhighenergy beams that can destroy tumors with minimal impact on the surrounding healthy tissue.
In the mid-1970s, AECL, developed a radical new "double-pass" concept for electron acceleration. A double passaccelerator needs much less spaceto develop comparableenergy levels because it folds the long  physical mechanismrequired to accelerate the electros, and it is more economic to produce.
Using this double pass concept AECL designed the  Therac-25, a dual mode lineair acelerator that can deliver either photonsat 25 MeVor electrons at various energy levels. Compared with theTerac-20 The Thrac-25 is notably more compact,, more versatile, and arguably easier to use. 
The higejr energy takes advantage of the phenomenon "depth dose": As the energy increases, the depth in the body at which maximum dose buildup occurs alse increases, sparing the tissue above the target area.
First, like the Therac-6 and the Therac-20, the Therac25 is conrolled by a PDP11. The Terac-6and Therac-20 had been designed around machines that already had histories of clinical use without computer control.
The therac-20 has idependent protective circuits for monitoring electron-beam scanning, plus mechanical interlocks for policing the machine and ensuring safe operation.
Finally some software for the machines was interrlatd or reused.
Eleven therac-25 were installed: five in the usand six in canada. Six accidents involving massive oerdoses to patients occured between 1985 and 1987. The machine was recalled in 1987 for extensive design changes, including hardware	 safeguards against errors.
Kennestone Regional Oncology Center 1985
Door rechtzaken waren managegers op de hoogte van de problemen en ongelukken. Maar er werd in het vervolg niet over gerapporteerd.
The treatment prescription printout failure was disabled at the time of the accident , so there was no hardcopyof the treatment data.
Ontario Cancer Foundation in 1985
Since the machine did not suspendand the control display indicated no dose was delivered to the patient, the operator went ahead with a second attempt at trratment by pressing the "P" key, expecting the machine to deliver the proper dose this time. This was standard operating procedure and, described in the "The operating interface" on p 24, Therac 25
oprators had become accustomed to freunt malfunctions that had no untowardconsequences for the patient. Again, the machine shut downin the same manner. The oeprator repeated this process four times after the original attempt- the display showing "no dose" delivered to the patient each time. After th fifth pause, the machine went into treatment suspedn, and a  hospital service technician was called.
The technician found nothing wrong with the machine. This was not an unusual scenario, according to the Therac-26 operator
Manufactureere response
Government and user response
Yakima Valley Memorial Hospital in 1985
Manufactureere response
Government and user response
East Texas Cncer Center, March 1986
Manufactureere response
Government and user response
East Texas Cncer Center, April 1986
Manufactureere response
Government and user response
Yakima Valley Memorial Hospital
Manufactureere response
Government and user response
	\cite{rogaway2004therac25},
\cite{wikiTherac25}, 
\cite{lynch2017theracRaceConditions},	\cite{lim1998theracdisaster}, 
\cite{fabio26102015therac25},	 	\cite{ethicsunwrappedTherac25}, 	\cite{casesHistoryTherac25},	 	\cite{caballero2019Therac25}, 	\cite{rose1994theracFatalDose}, 	\cite{levesonMITTherac25},
\cite{grant1978theracevaluation},	 	\cite{turnerTheracAccidentsInvestigations},	\cite{turner1993TheracAccidentsInvestigations}, 	\cite{wang2017industrialdesignengineering}, 	\cite{levesonturner1993theracpart2},	\cite{porelloTheraccFailure},\cite{theracIncidents}, 
\cite{huffbrown2004casestudyethicatherac}, 
\cite{sebowikimedicalradiation},	\cite{hsia1995testtherac25},	\cite{magsilvaTheracTesting},
\cite{chemeuropetherac25},	\cite{statsenko10102016Therackillerbug},	\cite{therac25casestudy},	\cite{thomas1994theracinLotos},	\cite{twitter2019programmerbehindtherac},	\cite{wikibookstherac}, 
\cite{bozdagTherac25},	\cite{levesonTurnerTheracAbstract}, 	\cite{stackexchange2021therac25code}.
%%%%%%%%%%%%%%%%%%%%%%%%%%%%%%%%%%%%%%%%%%%%%%%%%%%%%%%%%%%%%%%%%
\newline \indent
%Hoe werkt het
tesla autopilot features voor dataverzameling\cite{denneyjdsupraFeds},\cite{gritti24062020tesladataengine}.
% crashes
 De eerste tesla crash is van juni 2016 \url{https://impakter.com/tesla-autopilot-crashes-with-at-least-a-dozen-dead-whos-fault-man-or-machine/#:~:text=The%20first%20known%20death%20reportedly,trailer%20against%20the%20bright%20sky.}. En meerdere zouden volgen.
Een ongeluk in  de VS waarbij 2 inzittenden om het leven kwamen. Een persoon had plaats genomen als bijrijder en de andere persoon als passagier achter de stoel van de bestuurder. Waarschijnlijk was de autopiloot niet ingeschakeld.
\cite{anderson30042021secondteslacrash},\cite{raynal20042021probeTeslaCrash},\cite{firstpress11052021fatalnonautopilot},\cite{cochran18042021nodriverTeslaCrash},\cite{gitlin11052021autopilot},\cite{sommerfield12072021NHTSAmandateresult},\cite{hawkins30062021nhtsarequiresreporting},\cite{wilson19042021teslacrashregulators},\cite{mcfarland22042021selfdrivingrisks}
De situatie en oorzaken zijn bij elke ramp verschillend. 
Een automobilist heeft in een rit van 37 minuten slechts 25 seconden zijn handen aan het suur gehad ondanks de melding "Hands requireed not detected". Hiermee zijn de onderzoekers van de NTSB ervan uitgegaan dat de bestuurder de autopiloot bewschouwde als een volledig autonooom rijsyssteem in plaatst van een veligheidsmechanisme
\cite{oremus21062017fatalTeslaCrash}. Of in 
Mei 2015 als een besuurde foto's van zichzelf maakt in de testla zonder handen aan het stuur of voeten op het pedaal.
\cite{guardian15052021teslacrashHandsOnWheel}
Een faatale crash in 2016 waarbij de bestuurder  e veel vertrouwde op het semi-autonome rijtechnologie op het verkeerde type wegdek.
\cite{Puzzanghera13092017TeslaSharesBlame}
Onderzoek naar een fatale crash op 7 mei 2016 toont aan dat er beperkingen zitten aan de autopilot mode. Om specifiek te zijin is de automatische noodrem niet failsafe, blijkt uit onderzoek.
\cite{jaillet02022017teslaAutopilotLimitations}
\cite{reuters03102019teslaAutoParkingFail}
\cite{dowling23042021}
Op  April 17 2019 een autocrash waarbij het onduidelijk is of de autopiloot aan stond.
\cite{young05112021fatalTeslaReport}. Een auto ongelu waarbij een tesla is betrokken. De bestuurder was waarschijnlijk afgeleid door de games op zijn apple telefoon. De NTSB gaf aan dat het crash-avoidance systeem neit otnworpen is en ook geen crash atnuaor heeft gedetecteerd. Herdoor accelereerde de autopilot  het voertuig. Ook Faalde het systeem in het verschaffen van een crash aleter en werden de noodremmen niet geactiveerd.
\cite{tiungteslasoftwarecrash}
Er is ook een melding van een tesla waarvan de autopilot bots tegen een stilstaande politieauto
\cite{kierstein18032021teslaAutopilotCrashStationary}. Ook uit dit onderzoek blijkt dat er geen gebreken waren en dat het automaische remsysteem neit kapot was. De HNTSA concludeerder dat de bestuurder zelf geen actie ondernam door  bij te sturen of te remmen. In een eerder artikel kwam naar voren dat de tesla een autopilot krijgt die enkel camera's en GPS gebruikt; lidar of een radarsysteem wordt niet toegepast.
\cite{janssen20062017teslacrashdetailflorida}
Enkele fotos van crashes met autonome rijsysstemen \cite{saferoardsCrashesAutonomousvehicles}.
\cite{stephardson18032021revieuwingtesla}
%Onderzoeksrapport naar testla automatic vehicle control system
\cite{habib28062016NHTSATeslaReport},
\cite{darkReading17112020TeslaBackup},
\cite{heilweil26022020teslaAutopilot}
% overzicht
Tesla autopilot crashes met meer crashes en incidenten dan tot dan toe gerapporteerd
\cite{teslaFDSCrash}
De meest voorkomende crashes zijn stationaire objecten bij hoge snelheden, lane incursions from stationary objects, auti=opilot confusion at forks and gores.
\cite{teslaCrashesCauses}
\cite{teslacrashOvervieuw}
\cite{tesladeaths}
% veiligheidsrisico''
De veiligheidsrisicos van de tesla lopen uiteen. Zo zijn er risicos in de machinelearning technologie:
veiigheidsrisico Three Small Stickers in Intersection Can Cause Tesla Autopilot to Swerve Into Wrong Lane
\cite{evan01042019teslaautopilotIntersection},
\cite{lambert31062020q2safetyreport},de autopilot zelf
\cite{templeton06092019HTSBReportTesla}. Een studie door de consumntenbond in de VS toont aan dat hetautopilot systeem van de testla niet failsafe is. Zo zijn de sensoren, gebrukt voor detectie van een bestuurder negatief te beinvloeden.
\cite{dowling23042021autopilottricking} Maar ook andere problemen met de bluetooth 
\cite{wiredBloutoothHackTesla}, touch screen
\cite{preston14012021NHTSATeslaRecall},
Web-based attack crashes Tesla driver interface
\cite{leyden23032020TeslaInterfaceHack}.
Of zelfds de tesla batterij is veiligheidsvraagstuk geworden
\cite{mitchell01072020teslabatterycooling}.
Maar ook was een onderzoeker  was in staat om persoonlijke details van afgedankte voertuigonderdelen  te vekrijgen nadat deze waren afgekeurd vanwege upgrades en reparaties op consumentenvoertuigen.
\cite{stumpff04052020TeslaPersonalData}
Data-opslag in de cloud niet altijd bereikbaar.
\cite{mitchell24022020AIDataTesla}
%Wat er mis zou kunnen gegeaan wordt dru over gespeculeerd online.
%\cite{stackexchange102019teslacarmistake}
dodelijk ongeluk
\cite{fottrell03092018TeslaSecurityChecks},
softwarefout maakt diestal mogelijk
\cite{kirk26112020modelX}
fouten ontdekt in onderzoek
\cite{bbc24022021hyundaiBatteryFireFix},
tesla cloud gehacked
\cite{hawkins22102022}.
%Het AI aloritme vn Tesla
%\cite{rangaiah25022020teslaAI}
%Waarom deeplearning geen zelfrijdende auto's zal voortbrengen
%\cite{bdickson29072020teslalevelfive}
%Een survey naar de tesla gebruikers.
%\cite{randall05112019modelSurvey}
%maatschappelijk probleem
This analysis considers the potential impacts of completely self-driving vehicles on vehicular liability. 
\cite{griemannExaminSelfDriving}
Dan zijn er nog maatschappelijke problemen die de aanpak moeilijker maken.
Er is in de vs in verschillende staten een andere wetgeving
\cite{berry21042021teslacrashtexas}
\cite{hull23072021regulatorsaftercrash}
\cite{wikiTeslaAutopilot}
%oplossingen
Toch zijn er oplossingen en tegenmaatregelen.
tesla gaat advanced driver assistance systems inzetten met behulp van  passive visual, ultrasonic, en radar.
\cite{tasking07062017TeslaAugmentedSafety},\cite{ackerman01072016TeslaImperfect}
Safe system solutions door David Harkey
\cite{Harkey30052019SafeSystemVehicle}
%maatregelen
Voor elke auto uitgerust met een level 2 tot level 5 autonomy wordt nu standaard een rapport van van de crash opgvraagd door de NTSA. Dit in het kader van verder onderzoek waarbij de autoritait kijk naar  ziekenhuisbehandeling, fataliteit, airbag deployment.
\cite{szymkowski29062021nhtsaTeslaCrashReports}. 
%%%%%%%%%%%%%%%%%%%%%%%%%%%%%%%%%%%%%%%%%%%%%%%%%%%%%%%%%%%%%%%%%
\newline \indent De slmramp op  07/06/1989 \cite{espnSLMterugblik},\cite{dennisRosier01052020}
\cite{hassing07062020slmramp},\cite{amsterdamArchiefSLM},\cite{rtvOost06062019nabestaande},
\cite{breda07062021AndroSnel},\cite{andereTijdenSLMCrash},
\cite{aviationReport},\cite{aviationSLMCrashAccidentInvestigation},\cite{mcDonnelDouglasCommissionReportSLMCrash},
\cite{wikiSRFlight764},\cite{nos07062019SLMTerugblik},\cite{dagvantoenSLMCrash},\cite{waterkantNesty07061989},\cite{eduNandlalSRCrash},\cite{oldjetsSRAirways},\cite{cloudberg02012021srflight764},\cite{apnews07061989srplanecrash}.
%%%%%%%%%%%%%%%%%%%%%%%%%%%%%%%%%%%%%%%%%%%%%%%%%%%%%%%%%%%%%%%%%
\newline \indent De schipholbrand op 27/10/2005\cite{schipholbrand27102005video},\cite{schipholbrand27102005video},\cite{onderzoeksraad2610schipholoost},
\cite{schipholbrandvideoargos},\cite{nunl30052023feitenoverzicht},\cite{parlementairemonitorschipholbrand},\cite{videonpoNOVA13112008},\cite{rizoomes01052014schipholbrand},\cite{heuvelkroesschipholbrandcamerabeelden},
\cite{wikiSchipholbrand},\cite{schipholbrand27102005video},\cite{onderzoeksraad2610schipholoost},\cite{schipholbrandvideoargos},\cite{nunl30052023feitenoverzicht},\cite{singeluitgeverijenSchipholbrand},\cite{eenvandaagschipholbrand},\cite{parlementairemonitorschipholbrand},
\cite{videonpoNOVA13112008},\cite{rizoomes01052014schipholbrand},\cite{heuvelkroesschipholbrandcamerabeelden}. 
%%%%%%%%%%%%%%%%%%%%%%%%%%%%%%%%%%%%%%%%%%%%%%%%%%%%%%%%%%%%%%%%%
\newline \indent De explosie tanjin china 12/08/2015. 
Op 12 augustus 2015. Er waren twee explosies bij de Rulthai logistiek  faciliteit zorgde voor de opslag vn  gevaarlijke stoffen. De explosie zorgde voor de vernietiging van 12000 voertuigen, schade aan 17000 huize binnen een traal van 1 km. Er waren 173 doden inclusief brandweermensen.
Een van de explosies zorgde voor  een beving van 2.3 op de schaal van rigter.
De volgende factoren zouden een rol hebben gepeeld:
Een onjuiste afbakening van het opslagmaeriaal
Er was  weinig kennis bij de autoriteiten over  opslagmaterialen. Zo bleek er 7000 ton aan materiaal opgeslagen, dat is ruim 70 keer te maximaal toegestande hoeveelheid. 
Onverenigbaar grondgebruik in de nabije omgeving. Veel woonwijken met nar schatting 6000000 bewoners en 500 lokale bedrijvenin de buurt van de opslag gevaarlijke stoffen.
Opgeslagen materialen  waren: calcium carbine, sodium nitraat, potassium nitraat, amminiak nitraat en cyanide.
Ook is er veel kritiek geweest op de acties van de autoriteiten. Zo was er censuur vanuit de overheid op de journalistiek.
Ook was er naar alle warschijnlijkheid sprake van corruptie. Zo bleek achteraf dat een van de grootste aandeelhouders Dong Shexuang de zoon te zijn van een oud-politiechef in Tanjin haven, genaamd Dong Pijun
De overheid beloofde strengere toezicht en alle bedrijven moeten een risico-inventariatie maken en onderhouden\cite{jiang16042019TanjinExplosion},
\cite{staff31082015tanjinblastunrevealed},\cite{chinafile18082015tanjinexplosion},
\cite{pinghuang2410201TanjinFactreport},\cite{portoTanjinExplosionSight},\cite{imago17082015TanjinApartmentImages},\cite{trager14082015Chemicalblast},\cite{pangeramo27082015TanjinExplosion},\cite{ap06082020ammaniumnitrate},
\cite{morris14082015TanjinIndustryImpact},\cite{milesyu20082015exposingtoxicgovlines},\cite{artemis30032016tanjininsurance},\cite{aidenxiatanjinblast},
\cite{danwangTanjinflexreport},\cite{keyHighlightsTanjin},\cite{hartley13082015videofootage},\cite{odonnel01062017firetanjinblast2015},
\cite{fan15082015newyorkermistrustchina},\cite{yanlidongchinamediaframingTanjin},\cite{evans27092017TnjinInsurance},\cite{jasi26032019chineschemplant},\cite{shiqingTanjinExecutiveSentence},\cite{sophiebeach15082015},\cite{hamzeh05082020BeirutBlast},\cite{chemwatch18082015TanjiinExplosion},
\cite{thehindu15062019chinaExplosion},\cite{santagotimes24032019chinablast},
\cite{klingecorp28042020causedTanjin},\cite{mcgarryExplosions2017},\cite{roswnfeld13082015TanjinReports},
\cite{aria12082015explosionaTanjin},\cite{tremblay11022016chineseInvestigatorsTanjin},\cite{taylor13082015TanjinExplosianAftermath},
\cite{associatedPresss13082013},\cite{un20082015InvestigationTanjin},\cite{france2412082015TnjinExplosion},\cite{npr14082015TanjinCause},\cite{bbc05022016TanjinResponsibles},\cite{CBodeen15082015TanjinExplosion},\cite{reutersTanjinInsurance},\cite{yu082016evaluationTanjin2015},\cite{wiki2015TanjinExplosions},\cite{bbc17082015whathappenedTanjin},
\cite{mortimer19082016taijinexplosioncrater},\cite{internationallabourofficeChmControlTooliit},\cite{euTaxationCustomsICSC},
\cite{iloWHOChemSafetyCards}.
%%%%%%%%%%%%%%%%%%%%%%%%%%%%%%%%%%%%%%%%%%%%%%%%%%%%%%%%%%%%%%%%%
\newline \indent  De ethiopian airlinesop 10/03/2019\cite{caliskan09112013747boeingkalman},\cite{gates18112020boeingcrisis},
\cite{boeing737maxsoftwareprobles},\cite{avetisov19032019boeingmalwarestate},\cite{thompson23112020nationalsecurityboeing},
\cite{wiki737maxgroundings},\cite{campbell02052019boengcrashhumanerrors},
De oorzaak is de MCAS
\cite{hawkins22032019737maxairplanes},\cite{barnett05052019737maxcrisis}, \cite{thomas30082020737safest},\cite{boyle18112020737maxupgrade},\cite{bergstraburgess122019737maxMcasAlgorithm},\cite{737mcas},\cite{german190620217372yaftergrounded},\cite{beningo02052019boeinglessons},\cite{bloomberg26092019failedpred},\cite{afacwaLostSafeguards}, als een single point of failure \cite{uran05042019SPOF}
Angle-of-attack\cite{boeing737maxdisplay},
Behalve de MCAS waren er nog andere failures\cite{fehrm24112020737changes}, en ook deze failures \cite{dohertylindeman15032019737problems}
\cite{travis18042019737maxsoftwaredevop},
%\cite{easa27012021737maxsafereturn},
safety record van de boeing
\cite{touitou11032019737tragedies},
 Oplossingen zijn \cite{caa737modifications}. 
%%%%%%%%%%%%%%%%%%%%%%%%%%%%%%%%%%%%%%%%%%%%%%%%%%%%%%%%%%%%%%%%%
\newline \indent Het mortierongeluk in Mali op 06/04/2016. Aanwezige militair brengt slachtoffer naar de fransen, vervolgens naar de Tongolezen. Maar de kwaliteit van personeel liet te wensen over.
Er werd een Nederlandse arts overgevlogen. De slachtoffers werden overgevlogen naar Gao omvervolgens te worden oergevolgen naar Nederland.
Het ongeluk werd veroorzaakt door een kapot afsluitplaatje in de mortier. De granaat opslag in een niet gekoelde container. Dan was er vocht in de fatale granaat. Zodoende werden er explosieve stoffen gevormd in de granaat.
Tijdens de oefening werden de granaten warm in de zon. De granaat stond in veilie stand kon de explosie niet voorkomen.	\cite{ovvMortierOngevalMaliVideo} 
\cite{bnnvara13062018malirapport}
\cite{eucal11012021malimissieverlengd}
\cite{nos21052014zorgenmalimissie}
\cite{meijnders}
\cite{bnrwebredactie}
\cite{keultjes01062016malimissiecoalitie}
\cite{veenhof18012019}
\cite{isitman06012016militair}
\cite{nporadio11072016filmdemissie}
\cite{parlementairmonitor15122013mortierongeluk}
%%%%%%%%%%%%%%%%%%%%%%%%%%%%%%%%%%%%%%%%%%%%%%%%%%%%%%%%%%%%%%%%%
\newline \indent De ramp tjernobyl 26/04/1986. \cite{INSAVienna1992Chernobyl}
De mislukte veiligheidscontrole op 26 apeil 1986 01.24 uurin de sovjetuni leiddte tot explosies in een van de reactoren in de kerncentrale. De reactoren hadden geen veiligheidomhulling en de reactor bevat grote hoeveelheden brandbaar grafiet.
Door de explosie en de brand kwamen er radioactieve stoffen vrij.het gaat helemaal mis in de kernreactor 4. De warmteproductie nam  toe met een explosie tot gevolg.
31 mensen kwamen om, waaron veel mensen dagen later door stralingsziekte.
\cite{wikiTjernobyl},
\cite{rivmTjernobyl},
\cite{andereTijdenTjernobyl},
\cite{kingskey19042022tjernobyl},
\cite{erikbork26042023reactor4},
\cite{nosTjernobyl30jaarlater},
\cite{knmi04052021tjernobylbosbrand},
\cite{dodonovaKVIRisicoTjernobyl},
\cite{dumarey04062020verhaalTjernobylWaarheid},
\cite{sparkesNewScientistTjernoby},
\cite{kernenergiened26041986chronologiemaatregelen},
\cite{mapszoneReactor},
\cite{kernhistoriek15062021tjernobyl},
\cite{nucleairforumFeitenTjernobyl},
\cite{kernongevalTjernobylFancGov},
\cite{arendswolters062019lessenTjernobyl},\cite{damveld08052020tjernobyl},
\cite{deVriestjernobylHolland},\cite{ing3enieur29042015antistralingskoepel},
\cite{verschuur14012013tjernobylreports},\cite{paperlessarchivesTjernobyl},\cite{vargos082000tjernobylconcerns},\cite{mauroNuclearRiskSociety},\cite{vienna06092005LookingBack}
%%%%%%%%%%%%%%%%%%%%%%%%%%%%%%%%%%%%%%%%%%%%%%%%%%%%%%%%%%%%%%%%%
\newline \indent  Research case: De digitale aanval op de Oekrainese krachtcentrale op 23,december 2015

Op 23,december 2015  vind er een cyber aanval plaats op het elektriciteitsnet van de Oekraine. Dit was de eerste bekende aanval op een elektrisch contole  system.  Dit verslag geeft inzage in een analyse van de Ukraine cyber aanval,
inclusief hoe de actoren zich zelf toegang gavan tot het controle systeem, welke methoden de acoren hebben gebruikt voor reconnaissance en vastleggen van het systeem, een gedetailleerde omshrijving van de aanval op 15 December 2015, en de methoden die gebruikt zijn door de aanvallers om hun sporen uit te wissen en daarmee het het stoppen van schade toebrengen  nog moeilker maken. Daarnaast wordter  een gedetailleerde omschrijving gevevenv an de beveiliging van de SCADA ccontrol systemen gebaeerd op bst practices, inclusief het control network ontwerp, technieken voor whtelisting, monitoring en loggen, en  opleiding van personeel.
\cite{Whitehead2017ukrainepoweroutage}
\cite{noauthor_2022-nm}
\cite{zetter2016GridHack}
\cite{owens21032017ukrainemitigationstrategies}
\cite{cerulus2019FrontlineRussiaAttack}
\cite{grammatikis2019AttackIEC6087505104}
\cite{hidajat2016ScadaSimulator}
\cite{uscert20072021crashmalware}
\cite{zetter12062017malwareanalysis}
\cite{icsRussianHackingCyberWeapon}
\cite{usgovC2M2}
Dit verslag geeft inzage in een analyse van de Ukraine cyber aanval,
inclusief hoe de actoren zich zelf toegang gavan tot het controle systeem, welke methoden de acoren hebben gebruikt voor reconnaissance en vastleggen van het systeem, een gedetailleerde omshrijving van de aanval op 15 December 2015, en de methoden die gebruikt zijn door de aanvallers om hun sporen uit te wissen en daarmee het het stoppen van schade toebrengen  nog moeilker maken. Daarnaast wordter  een gedetailleerde omschrijving gevevenv an de beveiliging van de SCADA ccontrol systemen gebaeerd op bst practices, inclusief het control network ontwerp, technieken voor whtelisting, monitoring en loggen, en  opleiding van personeel.
\cite{Whitehead2017ukrainepoweroutage},\cite{zetter2016GridHack},\cite{boozallen2016lightwentout},\cite{finklejan2016UsBlamesRussianSandworm},\cite{desarnaud2017cyberattacks},\cite{caseli04112016intrusiondetectioncontrolsystem},\cite{rochascadatesting},\cite{hidajat2016ScadaSimulator},\cite{zetter2017moreDangerousMalware}.
Oop 23,december 2015  vind er een cyber aanval plaats op het elektriciteitsnet van de Oekraine. Dit was de eerste bekende aanval op een elektrisch controle  system met corrupte firmware. Daarnaas wordt er een telecom-based denial of service attack met  geautomatieerde systemen om het telefoonverkeer uit te schakelen.
\cite{Whitehead2017ukrainepoweroutage}
Uit onderzoek\cite{zetter2016GridHack} naar de aanval,  uitgevoerd door Oekraiene sen Amerikaanse militairenblijkt  bleek onder meer dat de power grids in sommige gevallen beter waren beveiligd dan de Amerikaanse. Desondanks was de viligheid niet optimaal door onder andere de  hetgegeven dat werknemers op afstand konden inloggen en geen gebruik van 2-stapsverificatie.
Oekraine wijst naar de russen \cite{zetter2016GridHack}, 
\cite{greenberg2017Cyberwartestlab},
\cite{boozallen2016lightwentout},
\cite{finkle08012016russiansandwormhackers},
\cite{zinets15022017ukrainechargesrussia},
\cite{mcelfresh2016cyberattackhowandwhy},
\cite{parkwalstorm11102017russiagridattack}.
{Situatie Oekraiene}
\cite{drago2017CrashOverride},
\cite{slowik2019ReassasUkraine2016Attack}.
{Situatie algemeen}
\cite{cerulus2019FrontlineRussiaAttack},
\cite{desarnaud2017cyberattacks},
\cite{dragos2019TargetedTransStation}.
{Factoren}
\cite{shehod2016gridadvantageus}
{Oorzaak}
\cite{rocha2017cybersecyrityanalysisScada},
\cite{2017crashoverridenostuxnet},
\cite{vijayan2017firstmalwareCausedOutage},
\cite{slowik2019ReassasUkraine2016Attack}.
{Gebruikte materialen}
\cite{2015ukrainegridattack},
\cite{industroyershortfact}
{Uitvoering van de aanval}
\cite{Whitehead2017ukrainepoweroutage},
\cite{boozallen2016lightwentout}.
{Oplossingen}
~\cite{Whitehead2017ukrainepoweroutage}
\cite{Whitehead2017ukrainepoweroutage}
\cite{boozallen2016lightwentout}
{spearfishing}
{blackenergy}
{remote access capabilities}
{serial-to-ethernet communication devices}
{telephony denial of service attacks}
{oplossingen}
Identificeer alle risicos en schrijf een plan foor het managen van de risico's.
Implementeer  effecteve controle  om het riico te managen.
Creeer een diepgaand model dat ervoor zor dat er efectieve en efficiente security controls worden uitgevoerd.
Aangaande de gebeurtenissen in de oekraiene kunnen de volgende security controls worden opgenomen in het securitymodel: Initial access to enterprise network, pivot in interprise network, elevate priviliges, maintainance access, gain access to control system, attack, attack complication, destroy hard drives.
\cite{Whitehead2017ukrainepoweroutage}
{Discussie}
{Verder lezen}
\cite{shahzad2014ScadaProtocolsPollingScenario},
\cite{grammatikis2019AttackIEC6087505104},
\cite{2017win32industroyer},
\cite{yadav2020reviewScadaArchitecture},
\cite{arrizabalaga2020surveyiiotProtocols},\cite{fauri2017EncryptionICS},\cite{resch31102019IEC62351secureCommunication},\cite{levalle2020FuzzingICSProtocols},\cite{blackhatusa2017},\cite{blackhatusa2017},\cite{abb30062017crashoverridenotification},\cite{spinner2018crashoverrideiot},\cite{njccicthreat08102017crashovverrideprofile},\cite{slowikvb2018crashoverride},\cite{crashoverridenetwork},\cite{wikiindustroyer},\cite{icsSecurityRussianHacking},\cite{holappa2017threattoElectricityNetworks}.
%%%%%%%%%%%%%%%%%%%%%%%%%%%%%%%%%%%%%%%%%%%%%%%%%%%%%%%%%%%%%%%%%
\newline \indent Dan zijn er nog andere ongelukken met de stint, de shietpartij op militairencomplex in ossendrecht, stint-ongeluk, de enschedese vuurwerkramp en de molukse treinkaping. Meer recentelijk de coronacrisis.
%%%%%%%%%%%%%%%%%%%%%%%%%%%%%%%%%%%%%%%%%%%%%%%%%%%%%%%%%%%%%%%%%





\paragraph{Safety critical systems}


https://www.icheme.org/media/8976/xxiv-poster-11.pdf
https://crpit.scem.westernsydney.edu.au/confpapers/CRPITV55Chambers.pdf
https://users.ece.cmu.edu/~koopman/des_s99/safety_critical/
WHAT ARE SAFETY-CRITICAL SYSTEMS?

Traditional Systems
Traditional areas that have been considered the home of safetycritical systems include medical care, commercial aircraft, nuclear
power, and weapons. Failure in these areas can quickly lead to
human life being put in danger, loss of equipment, and so on.

Non-traditional Systems
Emergency 911 service is an example of a critical infrastructure
application. Other examples are transportation control, banking
and financial systems, electricity generation and distribution, telecommunications, and the management of water systems

4.1 Technology


https://users.encs.concordia.ca/~ymzhang/courses/reliability/ICSE02Knight.pdf
https://www.dcs.gla.ac.uk/~johnson/teaching/safety/slides/pt2.pdf
https://www.dau.edu/tools/se-brainbook/Pages/Design%20Considerations/Critical-Safety-Item.aspx
https://daytonaero.com/wp-content/uploads/AC-17-01.pdf
https://nebula.esa.int/content/assessment-methodology-certification-safety-gnc-critical-space-systems
https://www.cs.unc.edu/~anderson/teach/comp790/papers/safety_critical_arch.pdf
https://www.cs.uct.ac.za/mit_notes/human_computer_interaction/htmls/ch02s10.html

1.       The Assembly is aware that the use of computers in safety-related applications is growing, particularly in areas such as control systems of aeroplanes, high-speed trains and nuclear power stations, medical equipment and medical records, anti-lock braking systems for vehicles and machine engineering in general, and last but not least, modern weapons and their guidance systems.

2.       Many recent accidents (for example, plane crashes due to computer failure, malfunctioning robot killing a mechanic, patient dying because of malfunctioning of computer-controlled intravenous drip, rocket launch failure traced to computer error, software piracy etc.) cause public concern and raise the question of the reliability of such systems.


How has the problem of safety-critical software arisen? Essentially from an ever-increasing complexity in engineering. One may compare the steam locomotive of 1830 with the APOLLO Moon spacecraft of 1970 as an example. In 1917 WM FARREN designed, supervised the construction of and testflew an aircraft - the CE 1 and with acceptable safety! [2]. Even in 1965 a chief designer would be familiar with all the decisions taken in the design of a complex product such as an aircraft or ship. The management operation was deeply hierarchical [3] , but as systems became more complex and design teams included more and more specialists it became necessary to formalise the interfaces between the specialist groups to gain benefit and yet maintain overall design disciplines. This led to the matrix design management system in the 1970s to cope with design teams 50 times larger than before [4].

A difficulty embodied in tackling the safety related to software in engineered products arises because of software complexity and the mathematical rigour of some parts of it distorts and clouds the fundamental processes of creative engineering design. 

Before discussing safety definitions and integrity a brief mention of design techniques to enhance safety. One way of increasing safety is to develop more reliable components and systems. At the outset, once the general preliminary design is defined there will be a "safety budget" allocating tolerable levels of integrity for every subsystem. Then Reliability Analysis evaluates the probability of failure and Failure Mode Effect and Criticality Analysis deals with the likely results of failure. Once the "life" of a part has been measured then the inspection and maintenance function will act to replace the part with a new one in good time. Another technique is to design an item to "fail-safe" i.e. even if it does fail it does not create a safety risk before the fault can be rectified. This has been extensively used on structures and coping with the development of fatigue cracks. "Fail- operate", "fault tolerant design" and "graceful degradation of systems" are other methods.



https://www.egbc.ca/getmedia/78073fda-5a83-4f0f-b12f-0a40dcbbc29d/EGBC-Safety-Critical-Software-V1-0.pdf.aspx
https://assembly.coe.int/nw/xml/XRef/X2H-Xref-ViewHTML.asp?FileID=7144&lang=EN
https://www.dlr.de/ft/en/desktopdefault.aspx/tabid-1360/1856_read-36215/
https://ieeexplore.ieee.org/document/1007998
https://coreavi.com/the-future-of-safety-critical-systems-in-the-emerging-autonomous-world/
https://verticalmag.com/features/whensafetymanagementsystemsfail/


fault stress
cause consequence analysis
hazops
fmeca/fha/fmea
fmeca = failure modes, effect and criticality anaysis
step 1 functional block diagram
setp 2 idendity failure modes ( complete failures, partial failure, intermittant failure, gradual failure)
step 3 accesss criticality
step 4 repeat for potential consequences
step 5 identify cause and occurence rate
step 6 deteermine detection failures
-type 1 the controls prevent the cause of failure mode from occuring, or reduce their rate of occurence
-type 2 hes controls detect the cause of the failure mode and the lead to corrective action
-type 3 these controls detect the failure mode before the product operaton, subsequent operations, or the end user
step 7  calculate risk priority numbers
RPN = secerity index * occurence index * detection index
step 8 finalyze hazard analysis
pra is aprt of hazard analysis, probaility risk analysis, the probability that product will work for T withour failure, R(T) = exp(-T/MTTF)
decision theory
risk = frequency * cost
mttf
Bellcore: reliability prediction procedure

Criticality level

36
41
58
59
62
73
84
104 Safety-critical software development
Software designed by
-hazard elminination
-hazard reduction
-hazrd controls
Software implementation issues
-dangerous practices
-choie of safe languages
105 leveson taxonomy of design techniques
-hazard elimination/avoidance: substitution, elimination, decoupling,	human error removal, removal of hazardous materials
-hazrd reduction:
--design for control
-incremental control
-intermediate states
-decision aids
-monitoring
-- add barriers
-hard/software lcoks
--minimize single point failures
-increase safety margins
-exploit redundancy
-allow for recovery

-hazard control: limit eposure ( back to normal fast exceptions), isolate and contain ( dont let things get worse, fail-safe (panic shut downs and watchdog code
-hazard minimization
software desigg techniques: fault taulerance
-avoid common mode failures
-nred for design diversity
- samre requiremenrs, different programmers, different contractors,
-redundant hardware mau duplicate, any faults if software is the same
- N-version programming, shared requirements or diffrent implementations where voring ensures agreement
-what about timing differences, comparison of continous values, wha is requirements wrong, performance of cost voting
-exception handling emchanism
- use run-time system to detect faults by raising exceptions or pass control to appropriate handler
-propagate o outmost scope hen fail
-recovery blocks: write acceptance test for modules, if it failes then eecut alternative
-must be able to restore the states: take a sapshot/checkpoint, if failure then restore snapshot
-control redundancy inclues: N-version programming, recovery blocks, exception handling
-error detecting/correcting codes
-chcksum agreements
-no task scheduler but bare machine
-restrict language subsets
-memory jumps
-overwrites
-semantics
-precision: interger, floating point, oeprations...
-data typing issues
-exception handling: is runtime recovery supported
-memory monitoring: guard againt  memory depletion
-separate compilation by typw checking agross modules

software development
-planning proces
* coordinate development activities
-software development processes
* requirements proces
* design process
* coding process
* integration proces
-software integral	 processes
* verification proces
* configuration management
* quality assurance
8 certification liaison
sofwtare developmnt key issues
- traceability and lifecycle focus
-designed engineering reps
-recommended practices
-design verification
-design validation

safety-critical software developemnt conclusions
-software design by
* hazard elimination
* hazard reduction
* hazard control
- software implementation issues: dangerous practices and choice of safe languages
Hardware Design: fault tolerant architectures
-the basiscs of hardware  management
*preferred pars list
* vendor and device selection
* critical devices, techniques and vendors
* device specifications
* sccreening
* part obsolescence
*FRACAS (Failure reporting, Analysis and corrective Action
Types of faults:
* Design faults: erroneous requirements, erroneous software, erroneous hardware
* management/ regulators
* Intermittent faults
-fault occurs and recurs over time
-fault connections can recur
* Transient faults
- fault occurs but may not recurr
-electromagnetif interference
* Permanent faults:
- fault persists
-physical damage to processor
- fault models
-hardware redundancy

Software faults
-specification errors
-coding errors
- tranaltion errors
-runtime errors

Active redundancy
Standby redundancy
Triple Modular redundancy

Fault detection
-runctionality checks
* routine to check hardware works
-signal comparisons
* copare signal in same units
-information redundancy
* parity checking, M out of N codes
-watchdog times
* reset if system times out
-bus monitoring
* check processor is alive
-power monitoring
* time to respond if power is lost


Validation and verification
-Verifiction is about proof and proof is about argument and an argument must be correct but not a mathematical holy grail
* does it meet the requirements
- show that  implementation is same as functional requirements
- too costly and time consuming all safety behaviour in specification
verification supported by:
* determinism (repeted tests)
* separate safety-critical functions
* well defined processes
* simplucity and decoupling
-Validation
* are the requirements any good
-during design
* external review before commission
* external review before commission
-during implementation
* additional constainrs discovered
* additional requirements emmerges
-during operations
* were the asumptions valid
* especialy environental factors
Validation summary of key issues
-who validates validator
* external agenets must be approved
-who validates validation
* clarify links to certification
- what happens if  validation fails
* must have feedback mechanism
* lnks to process improvement



222
228 mode confusion
241 individul human error
- slips, lapses and mistakes
- rasmussen: skill, rules, knowledge
- reason,generic error modelling
- risk homeostasis
242 what is error
* deviation from optimal performance
- very vieuw achieve the optimal
* failure to achieve desired outcome
-desired outcome can be unsafe
* departure from intended plan
-but environment may hange plan
244 types of errors
- slips
* correct plan but incorrect action
* more readily observed
-lapses
* correct plan but incorrect action
* failure of memory so more covert
-mistakes
* incorrect plan
* more complex, less understood
-human error modeling 
*analyse/distinguish error types
247 Skills, rules and knowledge: SKR
-signals
*sensory data from environment
* continuous variables
*Gibson direct perception
-signs
* indicate state of the environment
* with conventions for action
* activate stored pater into action
-symbols
* can be formally processed
8 related by convention to state
248
-skill-bases errors
* variability if human performance
-rule based errors
* isclassification of situations
* application fo wring rule
* incorrect recoll of correct rule
-knowdlegedebased errors
* incomplete/incorrect knwoledge
* worload and external constraints
249
- How do we account for slips ansd lapsesin SKR?
can we distinguish more detailed error forms and more diverse error forms?
Before an error is detected the operation is typically skill based
250 Monitoring failures
-Normal monitoring
* typical berore error is spotted
* preoprogrammed behaviours plus
* attentional checks on progress
-Attentional checks
* are actions acording to plan?
* willl plan still achieve outcome
- Failure in the checks pften leads to a slip or a lapse
-Reason also identifies overattention failues
251 Problem solving failures
-humans are pattern matchers
* prefer to use rules
* before effort of nkkowledge level
-local state information
* idexes stored problem hadling
* schemata, frames, scripts
-misapplication of good rules
* incorrect situation assessment
* over-generatisation of rules
-application of bad rules
* encoding deficiencies
* action deficiencies
252 knowledge based failures
-thematic vagabonding
* superficial analysis/ behaviour
* fit from issues to issue
-encysting
* myopic attention to small details
* metal-level issues may be ignored
-reason
* individual fails to recognise failure
* does not face up to consequences

254 GEMS: Error detection
- dont try to elmininate errors but focus on their detection
-self monitoring
* correction of postural deviations
* correction of motor responses
* detecton of speech errors
* detection of actin slips
*detection of problem solving error
-how do we support these activities
* standard checks procedures
* error hypotheses or suspicion
* use simulation based training
255
256 
258 GEMS: practical application
-Eliminate error affoardancecs
* increase visibility of task
* show users constraints on action
-Decision support systems
* dont just present events
* provide trend information
* what if subjunctve displays
* prostheses/ mental crushes
-Memory aids for maintainance
* often overlooked
*aviation task cards
* must maintain minatainance data
-improve training
* procedures or heuristics
* simulator or training
-error management
* avoid high-risk strategies
* high probability/cost of failure
-ecological interface design
* rasmussen and vincente
8 10 guidelines
-self-awaireness
* when might i make and error
* contentious
261 GEMS outstanding issues
-problem of intention
* is an error a slip or lapse
* is an error a mistake or intention
-give an observations of error
* afermath of accident/incident
*guilt, insecurity, fear, anger
-can we expect valid answers
- can we make valid inferences
263 Risk Homeostasis theory
265 individual human error
- slips, lapses and mistakes
- rasmusses: skill, rules, knowledge
- Reason: generic error modeling
- Risk homeostasis
266
- workload
-situation awareness
-crew resource management
267
-high workload
* stretches users resources
-low workload
* wasts user resources
* can inhibit ability to respons
-cannot be seen directly
* is infered from behaviour
269
- various approaches
* wickens on perceptual channels
* kanwowiz on problem solving
* hart on oerall experience
- holistic vs atomic approaches
* FAA a gestalt concept
* cannot measure in isolation
* many xperimentalists disagree
-single-user vs team approaches
* workload is dynamic
* shared/distributed between a team
* many previous studies ignore this
270 Workload
- how do we measure workload
-subjective ratings
* NASA TLX, task load index
* consider individual differences
-secondary tasks
* performance on aditional task
* obtrusive & difficult to generalise
-physiological measures
* heart rate, skin temperture
* lost of data but hard to interpret
271
- how to reduce workload
-function allocation
* staticof dynamic allocation
* to crew, systems or others (ATC)
-Automation
* but it can increase workload
* of change nature
-Crew recourswe management
*crew coordination
* deision making
* situation awareness
* more revew activities isnerted into standard operating procedures
\cite{winceckCriticalToSafety}
\cite{chambersHazardAnalysisSCS}
\cite{rslater1998SCSAnalysis}
\cite{knightchallengessafetyCritical}
\cite{johnson2006devsafetycritical}
\cite{daucriticalsafetyconsider}
\cite{fallsafedesign}
\cite{arForce2015VerificationExpectations}
\cite{nebulaassessment}
\cite{lalaArchitecturalPrinciples}
\cite{mitNotesSafetyCritical}
\cite{britishColumbia2020GuideSafetyCritical}
\cite{fulvio1993safetycriticalsystems}
\cite{dlrtabid}
\cite{knight2010SafetyCritical}
\cite{creavisafecritical}
\cite{valdes2018SafetybyAutomation}
\cite{2015whensafetymanagementsystemsfail}
\paragraph{Ondeerzoeksresultaten naar sluisbeveiliging}



Verouderde computersystemen zijn door de jaren heen gekoppeld aan netwerken, zodat ze op afstand te besturen zijn. Dit zorgt ervoor dat systemen kwetsbaar zijn voor aanvallen van buitenaf. De beveiliging is in de loop der jaren niet voldoende ontwikkeld om de infrastructuur goed te beveiligen.

Volgens het onderzoek is er de afgelopen jaren wel het nodige geïnvesteerd om de beveiliging op te schroeven, maar deze maatregelen zijn nog onvoldoende doorgevoerd.
https://www.nu.nl/internet/5814282/rekenkamer-waterwerken-niet-goed-beveiligd-tegen-cyberaanvallen.html
\cite{hdsr30092022lichtprojectieswaterliniesluizen}
rapport Digitale dijkverzwaring: cybersecurity en vitale waterwerken 
Crisisdocumentatie is verouderd en er worden geen volwaardige pentesten uitgevoerd. Uit het onderzoek blijkt dat nog niet alle vitale waterwerken rechtstreeks zijn aangesloten op het Security Operations Center (SOC) van Rijkswaterstaat. Hierdoor bestaat het risico dat RWS een cyberaanval niet of te laat detecteert. De minister van Infrastructuur en Waterstaat moet nog stappen zetten om aan de eigen doelstellingen voor cybersecurity te voldoen
De Algemene Rekenkamer beveelt de minister van Infrastructuur en Waterstaat ook aan om het actuele dreigingsniveau te onderzoeken en te besluiten of extra mensen en middelen nodig zijn. Ook is het voor een snelle en adequate reactie op een crisissituatie van essentieel belang dat informatie up-to-date is. Pentesten zouden integraal onderdeel uit moeten maken van de cybersecuritymaatregelen bij vitale waterwerken. Verder zou moeten worden bezien of medewerkers van het SOC beter moeten worden gescreend.

\cite{kramerZeeland}
Sluis Eefde kreeg niet alleen de onderhoudsbeurt, maar werd tevens uitgebreid met een tweede sluiskolk. Zo wil Rijkswaterstaat wachttijden voor de scheepvaart voorko

\cite{gww29032021kantelendesluisdeur}
Om de lokale bemanning, die de oren en ogen waren van de sluizen, te vervangen waren camera’s, communicatielijnen en software nodig. Hoge kwaliteit videobeelden, met echte kleuren en zonder enige vertraging zijn belangrijk voor de operators en zij moeten hierop kunnen vertrouwen. Er zijn verschillende testen gedaan met diverse camera’s en cameraposities om kleurechtheid te kunnen bieden onder alle omstandigheden. Het resultaat was een perfecte kleur op alle 70+ camera’s op iedere locatie.

Vertraging van videobeelden was een cruciale factor in dit project. Het is uiterst belangrijk dat de operator op zijn beeld ziet wat er daadwerkelijk op locatie gebeurt, zonder enige vertraging. Om te laten zien of er eventuele vertraging is, is er een speciale functie gecreëerd. Deze functie laat een rood kruis zien op het scherm wanneer de vertraging meer is dan 500 miliseconden. Zo ziet de operator direct of het beeld wat hij ziet actueel is. 

Een andere functie die voor dit project is gecreëerd, is bij de videobeelden aan te geven van welke kant van de sluis het camerabeeld is. Voor de operators is het belangrijk dat ze weten vanaf welke kant het vaartuig komt en waar deze naartoe vaart. Een simpele oplossing was om een blauw kader te maken om het videobeeld van de ene kant van de sluis en geen kader om het videobeeld van de andere kant. 


\cite{thkwaterwerken}
Het crisismodel kan beter, is de derde deelconclusie van de Algemene Rekenkamer. Er is geen specifiek scenario voor een crisis die wordt veroorzaakt door een cyberaanval. Ook ontbreekt inzicht in de effecten van een cybercrisis op andere sectoren, de zogeheten cascade-effecten. Tevens is de crisisdocumentatie op onderdelen verouderd.

\cite{rekenkamercybersecWater}
Ook maakt cyberveiligheid nog geen volwaardig onderdeel uit van reguliere inspecties.’ De Rekenkamer hamert erop dat alle vitale waterinfrastructuur zo snel mogelijk op het SOC wordt aangesloten. Ook zouden werknemers van Rijkswaterstaat die belangrijke waterkeringen bedienen beter gescreend moeten worden op hun antecedenten. Sollicitanten hoeven nu slechts een Verklaring Omtrent Gedrag te overleggen, maar dat is een heel lichte toets.

\cite{hackerWaterwerk}
deltawerken

\cite{kramerZeeland}
Volgens Rijkswaterstaat is het kostbaar en technisch uitdagend om klassieke automatiseringssystemen te moderniseren en wordt er daarom vooral ingezet op detectie van aanvallen en een adequate reactie daarop.
Uit het onderzoek blijkt dat Rijkswaterstaat de afgelopen jaren zelf van alle tunnels, bruggen, sluizen et cetera heeft vastgesteld welke cyberveiligheidsmaatregelen moeten worden genomen. Een groot deel van die maatregelen (ongeveer 60\%) was begin 2018 ook al uitgevoerd, maar Rijkswaterstaat ziet onvoldoende toe op de uitvoering van het resterend deel en heeft geen actueel overzicht van de overgebleven maatregelen.
De minister heeft een aantal waterwerken die Rijkswaterstaat beheert als vitaal aangewezen. . Uit het onderzoek blijkt dat nog niet alle vitale waterwerken rechtstreeks zijn aangesloten op het Security Operations Center (SOC) van Rijkswaterstaat. De ambitie om eind 2017 bij alle vitale waterwerken cyberaanvallen direct te kunnen detecteren was in het najaar van 2018 daarmee nog niet gerealiseerd. Hierdoor bestaat het risico dat RWS een cyberaanval niet of te laat detecteert.

\cite{cybersecWaterwerk}
Over de cyberbeveiliging van gemeenten en waterschappen wordt al langer geklaagd. Zo meldde EenVandaag al in 2012 dat rioolgemalen en sluizen gemakkelijk van afstand te bedienen waren, onder meer door bijzonder slechte wachtwoorden.

\cite{cybersecWaterschappen}
Rittal doet onderzoek naarop afstand besdienbare sluizen

\cite{cybersecZuidHolland}
Beveiligde VPN
M2M Services levert aan inmiddels 220 gemeenten en waterschappen beveiligde connectiviteitsoplossingen voor het beheer van pompen, riolen en gemalen. Om risico’s op beveiligingsincidenten te voorkomen maken wij gebruik van een VPN oplossing, waarbij de verbinding optimaal beveiligd is middels encryptie en authenticatie.

\cite{waterwerkNED}
Veiligheid op het water én op het land
Gebruik van lampbewaking 

\cite{veiligheidwaterland} 



\paragraph{ethiek}


Ethiek 



persuasive technology 
https://www.humanetech.com/youth/persuasive-technology 
\cite{humanTechpersuasiveTech}
https://www.minddistrict.com/blog/persuasive-technology-new-insights-in-behavioural-change 
https://www.sciencedirect.com/book/9781558606432/persuasive-technology 
https://spectrum.ieee.org/how-persuasive-technology-can-change-your-habits 
\cite{rezenfeld01012018persuasiveTecgHabits}
https://www.frontiersin.org/articles/10.3389/frai.2020.00007/full 
\cite{aldenaini28042020persuasiveTechTrends}
https://psmag.com/environment/captology-fogg-invisible-manipulative-power-persuasive-technology-81301 
\cite{larson14062017persuasivetechmanipulates}
https://www.makeuseof.com/what-is-persuasive-technology/ 
\cite{tanzem22012022persuasivetechchanginglives}
https://lib.ugent.be/catalog/rug01:001235489 
https://cyberpsychology.eu/article/view/12270 
\cite{tikkakuddonenpersuasiveTechnology}
%%%%%%%%%%%%%%%%%%%%%%%%%%%%%%%%%%%%%%%%%%%%%%%%%%%%%%%%%%%%%%%%%
\paragraph{Afbakening van requirements Wet en regelgeving voor sluizen}
Omdat we in deit onderzoek uitgaan van het uitbreiden van bestaande sluizen is er literatuurstudie gedaan naar sluizen. In de archieven van het ministerie van verkeer en waterstaat is er het rapport Design of waterlocks\cite{CivilEngineeringDivision}.
Het programma van requirements kunnen we in ons model niet helemaal overnemen. 
Zo zijn er precondities zaols topgrafie,bestaande watersluizen,waterlevel, wind, morphologie en bodemeigenschappen.

 

\paragraph{Analyse}
\paragraph{Conclusie}
%%%%%%%%%%%%%%%%%%%%%%%%%%%%%%%%%%%%%%%%%%%%%%%%%%%%%%%%%%%%%%%%%

%%%%%%%%%%%%%%%%%%%%%%%%%%%%%%%%%%%%%%%%%%%%%%%%%%%%%%%%%%%%%%%%%
 
\hoofdstuk{Uppaal model}


\paragraph{Inleiding}






	\subsubsection{De computation tree}

\xymatrix@ur@!R=2pc{%
	*+<1pc>[o][F-]{q_0}  \ar@(l,l)[]^<<<<{start} \ar@/^/[r]^0  \ar@/_/[d]_1 
	& *+<1pc>[o][F-]{q_1} \ar@(ul,ur)[]^{0}  \ar@/^/[d]^1 \\
	*+<1pc>[o][F-]{q_2} \ar@(dr,dl)[]^{1} \ar@/_/[r]_0 
	& *+<1pc>[o][F=]{q_3} \ar@(l,l)[]^>>>>{start}  \ar@(dr,dl)[]^{1} \\
 
 }








\subsection{Semantiek}




\paragraph{Variable}



Because we require that the transition relation of a kripke structuer us always total, we must extend the relation R if some state s has no successor. In this case, we modify R so that R(s,s) holds.
To illustrate the notions defined in this section we consider a simple system with variables x and y that range over D={0,1}. Thus, a valuation for the variables x and y is justa pair (d_1, d_2) $\in$ D x Dwhre d_1 is the value for x and d_2 is the value for y.


$\mathbb${A}  bestaat uit  een 4-tuple M = \{ S ,  S_{0}  , $\Re$ , L \} ~ met ~  daarin: \\
S:  ~ de  ~ verzamelingvan ~  alle ~  states ~  in  ~ het ~  systeem \\
S_{0} $\subseteq$ S: de verzameling van alle beginstates \\
$\Re$ $\subseteq$ $\mathbb${S} x S: de transitierelatie \\
L= S $\to$   ~ $2^{AP}$ ~ : de labels waarmee weiedere state labelen met atomaire propositiesdie waar zijn in die state\\
\\



A clock relation limits the occurrences among different
clocks/events, which are defined based on run and history.
A run corresponds to an execution of the system model
where the clocks tick/progress. The history of a clock c
represents the number of times c has ticked currently. A
probabilistic relation in PrCCSL is satisfied if and only if
the probability of the relation constraint being satisfied is
greater than or equal to the probability threshold p 2 [0; 1].
Given k runs = fR1; : : : ;Rkg, the probabilistic relations
in PrCCSL, including subclock, coincidence, exclusion,
precedence and causality are defined in Table II.
bron Formal Verification of Dynamic and Stochastic Behaviors for Automotive Systems

About transition
A transition is composed of
a unique source location
a unique target location
a guard, i.e. an enabling condition (g := x ∼ c|g ∧ g, where
∼∈ {<, ≤, =, ≥, >}
a label (that can be used for synchronization)
a subset (potentially empty) of clocks to be reset

a clock valuation is a function v: X $\trightarrow$ $R^+$ \\
v[Y:=0] is the valuation obtained from v by resetting clocks from Y:  \\

\begin{math}
	$v[Y:=0]$=\left\{
	\begin{array}{ll}
		1, & \mbox{0 x $\in$ Y}.\\
		0, & \mbox{otherwise}.
	\end{array}
	\right.
\end{math}
\\

v+d = flow of time (d units) \\
(v +d)(x) = v(x)+d  \\
v $\models$ c means that valuation v satisfies the constraint c

evaluation of a clock constraint (v $\models$ g) \\
\begin{enumerate}
	\item $\vee$ $\models$ g x  < k iff ν(x) < k
	\item $\vee$ $\models$ x ≤ k iff ν(x) ≤ k
	\item $\vee$ $\models$ g1 ∧ g2 iff ν $\models$ g1 and ν $\models$ g2
\end{enumerate} 
\\
\\
\\

(s', v") and (s,v) $\xrightarrow[]{a}$(s', v").

Action transitions correspond to the execution of a transition	 from T. We write (s,v) $\xrightarrow[]{a}$ (s', v'), where a \in $\Sigma$, provided that there is a transition $\langle$ s,a, $\varphi$, $\lambda$, s' $\rangle$ such that v satisfies $\varphi$ and v=[$\lambda$:=0].

a delay transition (s, v1)  $\xrightarrow[]{$$\delta$(d)}$ (s, v_1 + d_1) for some $d_1$ $\geq$ 0, and
an action transition   (s, v1 +d1)  $\xrightarrow[]{a}$ (s', v_1') such that $v_1$ + $d_1$ satisfies $\varphi$ and v'_1 = (v_1 + d_1)[$\lambda $:=0].



Real-time System = Discrete System + Clock Variables by Rajeev Alur

blz 2 actions
The state of a system changes over time. We refer to the state changes of a
system as actions. An action is a pair ($\sigma$,$\sigma$ ') of states that consists of a source
state $\sigma$ and a target state $\sigma$ '. Intuitively, if a system is in the source state $\sigma$,
then the action ($\sigma$,$\sigma$ ') takes the system into the target state $\sigma$'. We say that
an action is enabled in its source state and disabled in all other states. Two
actions ($\sigma$,$\sigma$ '1) and ($\sigma$,$\sigma$ '2) are consecutive if the second action is enabled in
the target state of the first action|i.e., if ($\sigma$ '1=$\sigma$ '2). The action ($\sigma$,$\sigma$ ') is a nul l
action if ($\sigma$=$\sigma$ ')
.

blz 6 clocks and delays

Formally, the action ($\sigma$,$\sigma$ ') is a system action if for all clock variables x, either
$\sigma$ '(x) = $\sigma$(x) or $\sigma$ '(x) = 0; the action ($\sigma$,$\sigma$ ') is a time action - or delay -if there
is a nonnegative real $\delta$ the duration of the delay|such that $\sigma$ ' = ($\sigma$,$\sigma$ '). System
actions have duration 0. Every null action is, by definition, both a system action and a delay of duration 0.



blz 7 Clock constraints
Let ($\sigma$, $\delta$) be a delay, let $\phi$ be a state predicate, and let $\psi$  be an action
predicate. The characteristic function of $\phi$ maps each nonnegative real e < $\delta$ to
1 if $\phi$ is true for $\sigma$ + e, and otherwise to 0; the characteristic function of   maps
e to 1 iff $\psi$   is enabled in $\sigma$ + e. A state or action predicate varies finitely over the
delay ($\sigma$, $\delta$) if its characteristic function has nitely many discontinuities in the
interval (0,$\delta$). Abstractly, we restrict ourselves to state predicates and action
predicates that vary nitely over all delays.


blz 8 Clock-constrained systems
A clock-constrained system S = ($\phi$, $\psi$ ) is a pair that consists of a timed state
predicate 0|the initial condition of S|and a timed action predicate $\psi$ |the
transition condition of S. The timed behavior $\sigma$ is a behavior of the clock-
constrained system S if (1) the initial condition of S is initially true for $\sigma$
and (2) the transition condition of S is invariantly true for $\sigma$. Every clock-
constrained system S denes, then, the set of its divergent behaviors, which is
denoted by [[S]].


The transition relation R of $\tau$(A) is obtained by combining the delay and action transitions. We will write (s,v) R(s', v') or (s, v) $\xRightarrow[]{f(x)}$   (s', v') if there exists s" and v" such that (s,v) $\xrightarrow[]{d}$ (s", v")$\xrightarrow[]{a}$ (s', v') for some d $\in$ $\Re$.


1 For a $\in$ $\Sigma$_1 $\cap$ $\Sigma$_2, if $\langle$ s1,a, $\varphi$, $\lambda_1$, s_1' $\rangle$ $\in$ $T_1$ and $\langle$ s2,a, $\varphi$, $\lambda_2$, $s_2'$ $\rangle$ $\in$ $T_2$ then T will contain the transition $\langle$ (s1,s2), a $\varphi$ , $\lambda_1$ $\cup$ $\lambda_2$, ($s_1$',$s_2$') $\rangle$
2. For a $\in$ $\Sigma_1$ - $\Sigma_2$, if $\langle$ s, a, $\varphi$, $\lambda$, s' $\in$ $T_1$ and t $\in$ $S_2$ then T will contain the transition $\langle$ (s,t),a, $\varphi$, $\lambda$, (s', t) $\rangle$
3. For a $\in$ $\Sigma$_2 - $\Sigma$_1, if $\langle$ s, a, $\varphi$, $\lambda$, s' $\in$ T_2 and t $\in$ $S_1$ then T will contain the transition $\langle$ (t,s),a, $\varphi$, $\lambda$, (t,s') $\rangle$


\xi \dots \langle \overleftarrow{ $\ell$_0}  , v_0  \rangle ,

$\xi$(t) = {\langle \overleftarrow{ $\ell$_0}  , v  \rangle | \exists i \in   \mathbb{N} \bullet (t_i \leq 
	t \leq t_{i+1} \wedge \overleftarrow{ $\ell$}  = \overleftarrow{ $\ell$_i}  \wedge v = v_i +t -t_i) }



is a sextuplet (L, `0, C, A, E, I), where
L is a set of positions
`0 ∈ L is the starting position,
C is a set of clocks
A is a set of actions, co-actions and internal τ -actions,
E ⊆ L × A × B(C) × 2
C × L is a set of edges between positions with
action, guard and a set of clocks that are reset, and
I : L → B(C) assigns invariants to positions.

Timed automation clock
Clock evaluation is a function of u : C → R≥0 from a set of clocks
to non-negative real numbers.
Let R
C be the set of all clock evaluations.
Let u0(x) = 0 for all x ∈ C.
Writing u ∈ I(`) will mean that u satisfies I(`).
It is possible to make a transition from a given state using action or
delay.


Timed Automata Semantics
Let (L, `0, C, A, E, I) be a timed automaton.
Semantics . . . a transition system with label hS, s0, →i, where
S ⊆ L × R
C is a set of states,
s0 = (`0, u0) is the initial state,
→⊆ S × (R≥0 ∪ A) × S is a transition relation such that
(`, u)  d → (`, u + d) if ∀d 0 : 0 ≤ d 0 ≤ d =⇒ u + d 0 ∈ I(`)

(`, u) a→ (` 0 , u0 ) if ∃e = (`, a, g, r, `0 ) ∈ E  e ∈ g, u0 = [r 7→ 0]u, u0 ∈ I(`0 ),

u + d maps each clock x ∈ C every hour to the value u(x) + d, for
d ∈ R≥0,


r 7→ 0]u indicates clock evaluation,
which maps every clock in r to 0 and agrees with u over C \ r







Voor het modelleren van een systeemhebben we nodig:
\newline
alle states van het systeem.
\newline
We stoppen deze in een verzameling \\
S: de verzamleing van alle states van een systeem\\
Elke individuele state noemen we $s_{0}$, $s_{1}$,...$s_{n}$.\\
\\
Ons   model   is   een   tupel   met   daarin   de   verzameling   states: M(S)\\
\\
De transities tussen states vormen een relatie
R $\subseteq$S xS\\
\\
De systemen die wij modelleren zijn reactief:Systemenkunnen eindeloos rondjes lopen door een aantal toestanden. \\
\\
Belangrijk gevolg: Voor elke state s \inS geldt dat er een state s' bestaat zodanig dat geldt R(s,s') \\
\\
Elke state heeft een uitgaande transities.
\\
Een transitierelatie, waarin elke stateeen uitgaande transitie heeft noemt men totaal. \\
Alle transitierelatiesin de  systemen die wij modelleren zijn totaal.
\\
\\
Om uitspraken te kunnen doen over ons systeem gebruiken we: \\
Een verzamelingatomaire poposities (AP):\\
proposities die niet verder op te delen zijnin kleinere/kortere proposities. \\
\\
Een labeling functie: L
De labeling  L= S \to   ~ $2^{AP}$ functie is een functie dieelke state "labeled"met een verrzameling atomaire propostities die waar zijn in die state.
\\
bron Formal Verification of Dynamic and Stochastic Behaviors for Automotive Systems


De safety en reachability requirements die formeel zijn gespecificeerd worden in Uppaal geverifieerd met de A en E state formule. Andrerere opreratoren zijn





\subsection{Formele specificaties: Queries}




Reachability Query:
Een veelvoorkomend type query is het controleren van de bereikbaarheid van een bepaalde toestand of toestandscombinatie in het systeem. Bijvoorbeeld, "Is het mogelijk om vanuit toestand A toestand B te bereiken?"

Is het mogelijk om vanuit toestand idle weer in toestand idle te komen?
Is het mogelijk om vanuit toestand deurDownOpen in toestand deurUpOpen te komen?

Invariant Query:
Een invariant is een eigenschap die altijd waar moet zijn tijdens de uitvoering van het systeem. Een query kan worden gebruikt om te controleren of een bepaalde toestand altijd aan een bepaalde voorwaarde voldoet. Bijvoorbeeld, "Is het zo dat altijd wanneer we in toestand C zijn, eigenschap X waar is?"

Is het zo dat altijd wanneer we in toestand deurDownClosed of deurUpClosed zijn, alle stoplichten op rood staan?
Is het zo dat altijd wanneer er een stoplicht op groen is, het waterniveau gelijk is aan de minimum of maximum?

Liveness Query:
Liveness verwijst naar de eigenschap dat bepaalde gebeurtenissen uiteindelijk zullen plaatsvinden. Een query kan worden gebruikt om te controleren of bepaalde gebeurtenissen in het systeem altijd zullen plaatsvinden, ongeacht de invoer. Bijvoorbeeld, "Zal gebeurtenis Y uiteindelijk altijd plaatsvinden?"

Reachability Specification:
Een reachability-specificatie controleert of het mogelijk is om een bepaalde toestand of toestandscombinatie in het systeem te bereiken. Het kan worden uitgedrukt als "Er bestaat een pad vanuit toestand A naar toestand B."

Invariant Specification:
Een invariant-specificatie controleert of een bepaalde eigenschap altijd waar moet zijn gedurende de uitvoering van het systeem. Bijvoorbeeld, "In toestand C moet eigenschap X altijd waar zijn."

Safety Specification:
Een safety-specificatie controleert of een bepaalde eigenschap nooit wordt overtreden tijdens de uitvoering van het systeem. Het kan worden uitgedrukt als "Op elk pad door het systeem moet eigenschap Y nooit waar zijn."

Liveness Specification:
Een liveness-specificatie controleert of bepaalde gebeurtenissen uiteindelijk altijd plaatsvinden. Het kan worden uitgedrukt als "Uiteindelijk zal gebeurtenis Z altijd optreden."

Fairness Specification:
Een fairness-specificatie beschrijft de eerlijkheidsvereisten van het systeem en kan bepalen hoe bepaalde gebeurtenissen worden behandeld, zodat ze niet voor onbepaalde tijd kunnen worden vermeden.

Controleerbaarheidspecificatie:
Een controleerbaarheidspecificatie geeft aan welke eigenschappen in het systeem kunnen worden gecontroleerd of welke aspecten van het systeem kunnen worden gestuurd.

\paragraph{Safety}
Safety Properties are used to verify that something
bad will never happen. Dit kan worden gespecificeerd met de volgende vergelijking





\begin{longtable} { |s|p{10cm}|p{8cm}| }
	\hline
	Scope & CCTL proposities \\
	\hline
	It is true ... & _ \\
	
	It is possible ... & E $\bigtriangleup$ \\
	it is inevitable ... & A  $\bigtriangleup$ \\
	it is at all time true ... & AG  \\
	It is at all times possible & AG E$\bigtriangleup$ \\
	it is at all time inevitable & AG A$\bigtriangleup$ \\
	- & \square ( a_0 \implies (( \lnot a_2 \wedge \lnot a_3 ) \mathcal{U} a_1 ) \vee ( \lnot a_2 \wedge \lnot a_3 )) \\
	AG(p) &  M, s \models AG(p) $\Leftrightarrow$     \forall \pi \in  \sqcap (M,s) \cdot \forall i \cdot M,\pi[i] \models p \\
	EG(p) &  M, s \models EG(p) $\Leftrightarrow$     \exists \pi \in  \sqcap (M,s) \cdot \forall i \cdot M,\pi[i] \models p \\
	AF(p) &   M, s \models AF(p) $\Leftrightarrow$     \forall \pi \in  \sqcap (M,s) \cdot \exists i \cdot M,\pi[i] \models p\\
	EF(p) &    M, s \models EG(p) $\Leftrightarrow$     \exists \pi \in  \sqcap (M,s) \cdot \forall i \cdot M,\pi[i] \models p\\
	AX(p) & M, s \models AX(p) $\Leftrightarrow$     \forall \pi \in  \sqcap (M,s) \cdot M,\pi[1] \models p \\
	EX(p) &  M, s \models EX(p) $\Leftrightarrow$     \forall \pi \in  \sqcap (M,s) \cdot M,\pi[1] \models p \\
	A(p \cup q) &  M, s \models  A(p \cup q)   $\Leftrightarrow$     \forall \pi \in  \sqcap (M,s) \cdot \exists k \cdot M,\pi[k] \models q \wedge ( \forall i \leq k \cdot M,\pi [i] \models p) \\
	E(p \cup q) & M, s \models  E(p \cup q)   $\Leftrightarrow$     \exists \pi \in  \sqcap (M,s) \cdot \exists k \cdot M,\pi[k] \models q \wedge ( \forall i \leq k \cdot M,\pi [i] \models p) \\
	A(p \Re q) &  M, s \models  A(p \Re q)   $\Leftrightarrow$     \forall \pi \in  \sqcap (M,s) \cdot \forall  k \cdot  \wedge ( \forall i \leq k \cdot M,\pi [i] \models \neg p) = (M,\pi [k] ]  \models q \\
	E(p \Re q) &  M, s \models  E(p \Re q)   $\Leftrightarrow$     \forall \pi \in  \sqcap (M,s) \cdot \forall  k \cdot  \wedge ( \forall i \leq k \cdot M,\pi [i] \models \neg p) = (M,\pi [k] ] \models q \\
	- &  \\
	- & M, s \models p $\Leftrightarrow$ p \in L(s) \\
	- &  M, s \models \not f1 $\Leftrightarrow$ M, s \nvdash f1 \\
	- & M, s \models f1 \vee f2 $\Leftrightarrow$ M,s \models f1 or M,s \nvdash f2 \\
	- &  M, s \models f1 \wedge f2 $\Leftrightarrow$  M,s \models f1 and M,s \nvdash f2 \\
	- &  M, s \models \mathrm{E} g_{1} $\Leftrightarrow$ there is a path \pi  from ~  s ~   such ~  that  ~ M, \pi \models g1 \\
	- &  M, s \models p $\Leftrightarrow$ for every path \pi  ~ starting from  ~  s, M, \pi \models g1 \\
	- & M, s \models p $\Leftrightarrow$ s is the first state of \piand M, s \models f1 \\
	- &  M, s \models \not g_{1} $\Leftrightarrow$ M, \pi  \nvdash g1 \\
	- & M, s \models p $\Leftrightarrow$  M, \pi  \models g1  or  M, \pi  M, \pi  \models g2 \\
	- &  M, s \models p $\Leftrightarrow$ M, \pi  \models g1  and  M, \pi  M, \pi  \models g2 \\
	- & M, s \models p $\Leftrightarrow$ M, $\pi^{1}$ \models g1 \\
	- &  M, s \models p $\Leftrightarrow$ there exists a k \ge 0, such that  ~ M, $\pi^{k}$  \models g1 \\
	- & M, s \models p $\Leftrightarrow$ for all i \ge 0,M,$\pi^{i}$ \models g1 \\
	- &  M, s \models g1 \bugcup g2 $\Leftrightarrow$ ~  there  ~ exists  ~ ak  ~ \ge  ~ 0 ~  such ~  that  ~ M,  ~ $\pi^{k}$ \models g2 \\
	- & and  ~ for  ~ all ~  0  ~ \le j < k, M,$\pi^{j}$ \models g1 \\
	- &  M, s \models p $\Leftrightarrow$ for all j \ge 0, if for ~  every  ~ i < j,M,$\pi^{i}$ \nvdash g1 then M,$\pi^{j}$ \models g2 \\
	\hline
	
	\caption{Resolution suffixes}
	\label{table:ressuffixes}
\end{longtable}

\begin{longtable} { |s|p{10cm}|p{8cm}| }
	\hline
	Main scope & CCTL Operations \\
	\hline
	- &  \\
	- &  \\
	- &  \\
	- &  \\
	\hline
	
	\caption{Resolution suffixes}
	\label{table:ressuffixes}
\end{longtable}

AG EF_[_x_,_y_] $\vee$


A (\on every path")
E (\there exists a path")
X (\next time")
G (\globally" or \always")
F (\eventually" or \nally")
U (\until")
R (\release")


\begin{tabular} { |s|p{6cm}|p{6cm}| }
	\hline
	Requirement & queries \\
	\hline
	It is true ... & A[] not maincontroller.rd1 imply \\
	
	It is possible ... &A[] maincontroller.rd1 imply \\
	it is inevitable ... & A[] not deadlock imply\\
	it is at all time true ... & E<> maincontroller.rd1 imply  \\
	It is at all times possible & E<> maincontroller.s7 \\
	it is at all time inevitable & E<> maincontroller.s7d\\
	\hline
\end{tabular}


\aqcap\\




\subsubsection{De computation tree}


\newline


\subsection{Evaluatie }


\forall x \, (P(x) \to Q(x)) & premise \\
\forall x \, P(x) & premise \\\hspace*{-30pt} \\


P(x_0) & $\forall x \, \mathrm{e}$ 2 \\
Q(x_0) & $\to \mathrm{e}$ 3, 4 \\

\forall x \, Q(x) & $\forall x \, \mathrm{i}$ 3--5 \\


\{a,b\} or \set†{a,b} \\
\langle a,b \rangle or \gens†{a,b} \\


f \colon A \to B \\

f \circ g \\
x \mapsto f(x) \\

\begin{align*}
	f \colon \mathbb{R} &\to \mathbb{R} \\
	x &\mapsto x^2
\end{align*}

\begin{tikzpicture}[>=latex',scale=0.5]
	% set node style
	
	\begin{dot2tex}[dot,tikz,codeonly,styleonly,options=-s -tmath]
		digraph G  {
			node [style="n"];
			p [label="+"];
			t [texlbl="\LaTeX"];
			6
			8
			10-> p;
			6 -> t;
			8 -> t;
			t -> p;
			{rank=same; 10;6;8}
		}
	\end{dot2tex}
	\begin{pgfonlayer}{background}
		\draw[rounded corners,fill=blue!20] (6.north west) -- (8.north east) -- (t.south east)--cycle;
	\end{pgfonlayer}
\end{tikzpicture}


\[\begin{tikzcd}[column sep=1cm]
	ABCDE\arrow[r, leftrightarrow, "\times"{anchor=center},"\text{label}","\text{label}"{below}]\arrow[d] & F\arrow[r]\arrow[d] & G\arrow[rr]\arrow[d] && H\arrow[d]\\
	ABCDEFGH\arrow[r, leftrightarrow, "\times"{anchor=center}]\arrow[d] & II\arrow[r]\arrow[d] & JJ\arrow[rr,"\text{very long label}"]\arrow[d] && KK\arrow[d]\\
	ABCD\arrow[r] & EEE\arrow[r] & FFF\arrow[rr] && GGG
\end{tikzcd}\]




\paragraph{Models}

\subparagraph{Maincontroller}
\[
\begin{tikzcd}%[every arrow/.append style=dash]  uncomment to remev arrowa
	& \tikz{\node[draw,circle]{1}} \ar{d}&  & \tikz{\node[draw,circle]{2}} \ar{d}  \ar{r} &  \ar{r} & \ar{r} & \ar{r} &  \ar{r}& \tikz{\node[draw,circle]{2}}\ar{d} \\
	\tikz{\node[draw,circle]{4}} \ar{d} & \tikz{\node[draw,circle]{2}}  \ar[bend right=15]{l} \ar{d} & & \ar{d} & \tikz{\node[draw,circle]{2}} &\tikz{\node[draw,circle]{2}}&\tikz{\node[draw,circle]{2}}&& \tikz{\node[draw,circle]{2}} \ar{d} &\\
	\tikz{\node[draw,circle]{4}} \ar{u} \ar{r}  & \tikz{\node[draw,circle]{3}} \ar{r}  & \tikz{\node[draw,circle]{5}} \ar{r} &  \tikz{\node[draw,circle]{6}} \ar{r}  \ar{ru} & \tikz{\node[draw,circle]{7}} \ar{r} & \tikz{\node[draw,circle]{7}} \ar[crossing over]{ul}  \ar{r} & \tikz{\node[draw,circle]{8}} \ar{r} & \tikz{\node[draw,circle]{9}} \ar{r} \ar{d} & \tikz{\node[draw,circle]{10}}  \\
	& & & &\tikz{\node[draw,circle]{1}} \ar[crossing over]{ul} \ar{ru}  & & \tikz{\node[draw,circle]{2}}  \ar[bend right=15]{r} & \tikz{\node[draw,circle]{2}} \ar[bend right=15]{l} \ar[bend right=15]{r}  & \tikz{\node[draw,circle]{2}} \ar[bend right=15]{l}
\end{tikzcd}
\]
 

\begin{tikzpicture}[>=latex',shorten >=1pt,node distance=2cm,on grid,auto,scale=0.2]

	\node[state] (q0-e) {$q_0/\epsilon$};
\node[state] (q0-1) [below right=of q0-e] {$q_0'/1$};
\node[state] (q1-0) [above right=of q0-1] {$q_1/0$};
\node[state] (q2-1) [below right=of q1-0] {$q_2/1$};

\node[state] (q3-0) [below left=of q0-1] {$q_3/0$};
\node[state,accepting] (q3-1) [below right=of q0-1] {$q_3'/1$};
\node[state] (0) [ left=of q3-0] {$q_3/0$};
\node[state] (1) [ left=of 0] {$q_3/0$};

\node[state,initial,accepting] (0) [ left=of 1] {$q_3/0$};

\node[state] (3) [ below right=of q2-1] {$2$};
\node[state] (4) [  right=of 3] {$4$};
\node[state] (5) [  right=of 4] {$5$};
\node[state] (6) [  right=of 5] {$6$};

\node[state] (7) [  above=of 1] {$7$};
\node[state] (8) [  above=of 7] {$8$};
\node[state] (9) [  right=of 5] {$9$};

\node[state] (10) [ below  left=of 4] {$10$};
\node[state] (11) [ below  right=of 4] {$11$};

\node[state] (12) [   above=of 5] {$11$};

\node[state] (13) [   above=of 12] {$11$};

\node[state] (14) [ below right  =of q3-0] {$11$};
\node[state] (15) [   below right =of 14] {$15$};
\node[state] (16) [   above right =of 15] {$16$};


\path[->] (q0-e) edge node {a} (q1-0);
\path[->] (q0-e) edge node {b} (q3-0);
\path[->] (q0-1) edge node {a} (q1-0);
\path[->] (q0-1) edge [bend right] node {b} (q3-0);
\path[->] (q1-0) edge node {a} (q3-1);
\path[->] (q1-0) edge node {b} (q2-1);
\path[->] (q2-1) edge node {a} (q0-1);
\path[->] (q2-1) edge node {b} (q3-1);
\path[->] (q3-0) edge node {a} (q3-1);
\path[->] (q3-0) edge [bend right] node {b} (q0-1);
\path[->] (q3-1) edge [loop below] node {a} (q3-1);
\path[->] (q3-1) edge node {b} (q0-1);

\path[->] (4) edge [bend right] node {b} (10);	
\path[->] (10) edge [bend right] node {b} (4);


	
\path[->] (4) edge [bend right] node {b} (11);	
\path[->] (11) edge [bend right] node {b} (4);
	

\end{tikzpicture}

 
\subparagraph{Labeling functions}



\subparagraph{Schip}


\begin{tikzpicture}[>=latex',shorten >=1pt,node distance=2cm,on grid,auto,scale=0.2]
	
	\node[state] (q0-e) {$q_0/\epsilon$};
	\node[state] (q0-1) [below right=of q0-e] {$q_0'/1$};
	\node[state] (q1-0) [above right=of q0-1] {$q_1/0$};
	\node[state] (q2-1) [below right=of q1-0] {$q_2/1$};
	
	\node[state] (q3-0) [below left=of q0-1] {$q_3/0$};
	\node[state,accepting] (q3-1) [below right=of q0-1] {$q_3'/1$};
	\node[state] (0) [ left=of q3-0] {$q_3/0$};
	\node[state] (1) [ left=of 0] {$q_3/0$};
	
	\node[state,initial,accepting] (0) [ left=of 1] {$q_3/0$};
	

	
	\path[->] (q0-e) edge node {a} (q1-0);
	\path[->] (q0-e) edge node {b} (q3-0);
	\path[->] (q0-1) edge node {a} (q1-0);
	\path[->] (q0-1) edge [bend right] node {b} (q3-0);
	\path[->] (q1-0) edge node {a} (q3-1);
	\path[->] (q1-0) edge node {b} (q2-1);
	\path[->] (q2-1) edge node {a} (q0-1);
	\path[->] (q2-1) edge node {b} (q3-1);
	\path[->] (q3-0) edge node {a} (q3-1);
	\path[->] (q3-0) edge [bend right] node {b} (q0-1);
	\path[->] (q3-1) edge [loop below] node {a} (q3-1);
	\path[->] (q3-1) edge node {b} (q0-1);
	

	
	
	
	
\end{tikzpicture}

\subparagraph{Deur}

\begin{tikzpicture}[>=latex',shorten >=1pt,node distance=2cm,on grid,auto,scale=0.2]
	
	\node[state] (q0-e) {$q_0/\epsilon$};
	\node[state] (q0-1) [below right=of q0-e] {$q_0'/1$};
	\node[state] (q1-0) [above right=of q0-1] {$q_1/0$};
	\node[state] (q2-1) [below right=of q1-0] {$q_2/1$};
	
	\node[state] (q3-0) [below left=of q0-1] {$q_3/0$};
	\node[state,initial,accepting] (q3-1) [below right=of q0-1] {$q_3'/1$};

	


	
	
	\path[->] (q0-e) edge node {a} (q1-0);
	\path[->] (q0-e) edge node {b} (q3-0);
	\path[->] (q0-1) edge node {a} (q1-0);
	\path[->] (q0-1) edge [bend right] node {b} (q3-0);
	\path[->] (q1-0) edge node {a} (q3-1);
	\path[->] (q1-0) edge node {b} (q2-1);
	\path[->] (q2-1) edge node {a} (q0-1);
	\path[->] (q2-1) edge node {b} (q3-1);
	\path[->] (q3-0) edge node {a} (q3-1);
	\path[->] (q3-0) edge [bend right] node {b} (q0-1);
	\path[->] (q3-1) edge [loop below] node {a} (q3-1);
	\path[->] (q3-1) edge node {b} (q0-1);
	

	
\end{tikzpicture}

\subparagraph{Stoplicht}


\begin{tikzpicture}[>=latex',shorten >=1pt,node distance=2cm,on grid,auto,scale=0.2]
	
	\node[state] (q0-e) {$q_0/\epsilon$};
	\node[state] (q0-1) [below right=of q0-e] {$q_0'/1$};
	\node[state] (q1-0) [above right=of q0-1] {$q_1/0$};
	\node[state] (q2-1) [below right=of q1-0] {$q_2/1$};
	
	\node[state,initial,accepting] (q3-0) [below left=of q0-1] {$q_3/0$};
	\node[state,accepting] (q3-1) [below right=of q0-1] {$q_3'/1$};



	
	
	\path[->] (q0-e) edge node {a} (q1-0);
	\path[->] (q0-e) edge node {b} (q3-0);
	\path[->] (q0-1) edge node {a} (q1-0);
	\path[->] (q0-1) edge [bend right] node {b} (q3-0);
	\path[->] (q1-0) edge node {a} (q3-1);
	\path[->] (q1-0) edge node {b} (q2-1);
	\path[->] (q2-1) edge node {a} (q0-1);
	\path[->] (q2-1) edge node {b} (q3-1);
	\path[->] (q3-0) edge node {a} (q3-1);
	\path[->] (q3-0) edge [bend right] node {b} (q0-1);
	\path[->] (q3-1) edge [loop below] node {a} (q3-1);
	\path[->] (q3-1) edge node {b} (q0-1);
	

	
	
	
\end{tikzpicture}

\subparagraph{pomp}



\begin{tikzpicture}[>=latex',shorten >=1pt,node distance=2cm,on grid,auto,scale=0.2]
	
	\node[state] (q0-e) {$q_0/\epsilon$};
	\node[state] (q0-1) [below right=of q0-e] {$q_0'/1$};
	\node[state] (q1-0) [above right=of q0-1] {$q_1/0$};
	\node[state] (q2-1) [below right=of q1-0] {$q_2/1$};
	
	\node[state] (q3-0) [below left=of q0-1] {$q_3/0$};
	\node[state,initial,accepting] (q3-1) [below right=of q0-1] {$q_3'/1$};



	
	
	\path[->] (q0-e) edge node {a} (q1-0);
	\path[->] (q0-e) edge node {b} (q3-0);
	\path[->] (q0-1) edge node {a} (q1-0);
	\path[->] (q0-1) edge [bend right] node {b} (q3-0);
	\path[->] (q1-0) edge node {a} (q3-1);
	\path[->] (q1-0) edge node {b} (q2-1);
	\path[->] (q2-1) edge node {a} (q0-1);
	\path[->] (q2-1) edge node {b} (q3-1);
	\path[->] (q3-0) edge node {a} (q3-1);
	\path[->] (q3-0) edge [bend right] node {b} (q0-1);
	\path[->] (q3-1) edge [loop below] node {a} (q3-1);
	\path[->] (q3-1) edge node {b} (q0-1);
	

	
\end{tikzpicture}


\hoofdstuk{Verificatie}
 We moeten aantonen dat een real-time programma voldoet aan de eisen opgesteld en gespecificeerd. De meest gebruikte methode voor het bewij
 
 zen van de correctheid van untimed programma's zijn aangepast voor timed programs.  We hebben nog geen aanpask gevonden voor het gebruik en bewijzen van correct gebruik van clocks.  Een bewijs voor het gebruik van real-time programmas met clocks is gegeven in T.A. Henzinger and P.W. Kopke. Verification methods for the di-
 vergent runs of clock systems
 
 In dit hoofdstuk formaliseren we de requirements ogegeven in de requiremenstlis tin hoofdstuk .. en bewijzen we de correcte toepassing met gebruik van de symbolic model-checker van Uppaal.
 Het systeem is gemodelleerd als een netwerk van meerdere timed automata: controller, sluis, stoplicht, deur, pomp en schip.
 
 Het bewijs vn corret gebruik kan ook worden aangetoond met help van bewijs voor inorrectgebruik
 
 
 Verificatie resultaten
 \paragraph{Het door ons uitgetippelde testpath of scenario}
 
 \paragraph{Timed automata}
 
 
\paragraph{Data variabelen}

\paragraph{Acties}
 
\paragraph{Clock regions}
\cite{clarke2000Modelchecking21}
\cite{clarke2000Modelchecking212}
\cite{clarke2000Modelchecking223}
\cite{clarke2000Modelchecking31}
\cite{clarke2000Modelchecking32}
\cite{clarke2000Modelchecking33}
\cite{clarke2000Modelchecking411}
\cite{clarke2000Modelchecking43}
\cite{clarke2000Modelchecking63}
\cite{clarke2000Modelchecking64}
\cite{clarke2000Modelchecking661}
\cite{clarke2000Modelchecking91}
\cite{clarke2000Modelchecking102}
\cite{clarke2000Modelchecking11}
\cite{clarke2000Modelchecking122}
\cite{clarke2000Modelchecking123}
\cite{clarke2000Modelchecking132}
\cite{clarke2000Modelchecking1321}
\cite{clarke2000Modelchecking152}
\cite{clarke2000Modelchecking171}
\cite{clarke2000Modelchecking172}
\cite{clarke2000Modelchecking173}
\cite{audioSemanticsBengtsson}
\cite{guidingAutomataBberm}
\cite{gearTransitionLindahl1}
\cite{gearTransitionLindahl2}
\cite{martinelliScada}
\cite{IgbalReconstructurintTransition1}
\cite{IgbalReconstructurintTransition2}
\cite{huangVerficationStoch}
\cite{bengtssonUppaalVerification}
\cite{pranaliVerificationWaterLevel}
\cite{alexandreUppaalDefinition}
\cite{behzadEvalQOS}
\cite{behzadVariablesQoS}
\cite{alur}
\cite{alurDenseRealTime}
\cite{alurSystemClok}
\cite{alurModelHybrid}
\cite{rijksoverheidSluizen}
\cite{rijksoverheidSluisStroomschema}

\paragraph{CTL logica}
Alle veiligheid en reachability requirements formeel gespecificeerd in hoofdstuk ... zijn geverifieerd in uppaal met gebruik an A en E state formulae. Deze zijn als volgt:
\newline \\
M, s $\models$ p $\Leftrightarrow$ p $\in$ L(s) \\
M, s $\models$ $\not$ f1 $\Leftrightarrow$ M, s $\nvdash$ f1 \\
M, s $\models$ f1 $\vee$ f2 $\Leftrightarrow$ M,s $\models$ f1 or M,s $\nvdash$ f2 \\
M, s $\models$ f1 $\wedge$ f2 $\Leftrightarrow$  M,s $\models$ f1 and M,s $\nvdash$ f2 \\
M, s $\models$ $\mathrm{E}$ $g_{1}$ $\Leftrightarrow$ there is a path $\pi$  from ~  s ~   such ~  that  ~ M, $\pi$ $\models$ g1 \\
M, s $\models$ p $\Leftrightarrow$ for every path $\pi$  ~ starting from  ~  s, M, $\pi$ $\models$ g1 \\
M, s $\models$ p $\Leftrightarrow$ s is the first state of $\piand$ M, s $\models$ f1 \\
M, s $\models$ $\not$ $g_{1}$ $\Leftrightarrow$ M, $\pi$  $\nvdash$ g1\\
M, s $\models$ p $\Leftrightarrow$  M, $\pi$  $\models$ g1  or  M, $\pi$  M, $\pi$  $\models$ g2\\
M, s $\models$ p $\Leftrightarrow$ M, $\pi$  $\models$ g1  and  M, $\pi$  M, $\pi$  $\models$ g2 \\
M, s $\models$ p $\Leftrightarrow$ M, $\pi^{1}$ $\models$ g1 \\
M, s $\models$ p $\Leftrightarrow$ there exists a k $\ge$ 0, such that  ~ M, $\pi^{k}$  $\models$ g1\\
M, s $\models$ p $\Leftrightarrow$ for all i $\ge$ 0,M,$\pi^{i}$ $\models$ g1 \\
M, s $\models$ g1 $\bugcup$ g2 $\Leftrightarrow$ ~  there  ~ exists  ~ ak  ~ $\ge$  ~ 0 ~  such ~  that  ~ M,  ~ $\pi^{k}$ $\models$ g2\\
and  ~ for  ~ all ~  0  ~ $\le$ j < k, M,$\pi^{j}$ $\models$ g1
M, s $\models$ p $\Leftrightarrow$ for all j $\ge$ 0, if for ~  every  ~ i < j,M,$\pi^{i}$ $\nvdash$ g1 then M,$\pi^{j}$ $\models$ g2\\


%%%%%%%%%%%%%%%%%%%%%%%%%%%%%%%%%%%%%%%%%%%%%%%%%%%%%%%%%%%%%%%%%


 

\hoofdstuk{Conclusie}

Wat hebben alle bovenstaande rampen/ongelukken gemeen? Veiligheid.
Bij de therac waren er diverse problemen: communicatie, doorontwikkeling, controle en toetsing
Was het makkelijk te onderzoeken? Waarom?
Bij de boeing 737 crashes was het probleem van controle en communicatie naar medewerkers
Was het makkelijk te onderzoeken? Waarom?

Uit de evaluatie van de china explosion 2015 tianjin komt naar voren dat communicatie, transparantie en veiligheid niet altijd prioriteit hadden bij de lokale autoriteiten
Was het makkelijk te onderzoeken? Waarom?

Bij de tesla autopilot crashes komen soms onvoldoende onderbouwde ontwerpkeuzes naar voren die niet goed zij  afgewogen tegenover het gedrag van de bestuurder
vlucht 1951
Was het makkelijk te onderzoeken? Waarom?

De ramp in Tsjernobyl toont aan hoe autoriteiten een ramp in de doofpot proberen te stoppen
Was het makkelijk te onderzoeken? Waarom?



Wat heb ik geleerd
Ik heb erg veel geleerd van het veilig opzetten van VPN’s. Een VPN opzettenhad ik namelijk nog nooit gedaan. Het opzetten van SSH en het aanmaken vanVM’s was al bekend. Ook had ik nog nooit met UDP sockets geprogrammeerd.Verder heb ik geleerd hoe ik in de praktijk een VM in een VLAN kan zetten enhoe VLAN’s netwerken van elkaar kunnen scheiden.Het leukste onderdeel van het project, was dat wonderbaarlijk mijn gekozenoplossing elegant werkte. UDP Servers en clients zijn gerealiseerd met minderdan enkele regels logisch scipt. Ik had aan genomen dat het werken met socketsin shell absoluut rampzalig zou uitpakken. Ik ben blij dat het opdracht zo vrijwas, zodat ik experimenteel kon zijn met mijn implementatie.



