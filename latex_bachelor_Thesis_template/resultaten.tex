
 



\hoofdstuk{Theoretisch kader}

In het eerste hoofdstuk is duidelijk geworden wat de onderzoeksvraag is, namelijk ‘Hoe kan een geautomatiseerde sluis worden gemodeleerd met oog op ontwikkel- en onderhoudskosten,veiligheid, efficientie en capaciteit’. Door de toenemende complexiteit van systemen is het gebruik van modellen en de toepassing van timebased model checking  op industriele controle systemen een manier van modelleren van het systeem en de requirements zodat er een bijdagre kan worden geleverd aan de acceptatie van  simulatie-/modeltechniek voor de industrie.(‘https://link.springer.com/article/10.1007/s10626-020-00314-0’, 2020). Of dit ook het geval is bij het modellereren van sluizen is nu de vraag.

De bestudering van rampen aan de hand van het vier-variabelen model biedt maakt het analyseren mogelijk van rampsituaties. Van een aantal rampen is een beschrijving gegeven met datum, plaats en oorzaak. De analyse van de 4-variabelen modellen zal gebruikt worden voor de requirementsdefinitie, ontwerp en ontwikkeling van het sluismodel. 

De verschillende factoren en achtergronden die  samenhangen met het modelleren van een sluis zullen in dit hoofdstuk toegelicht worden. Bovendien worden er hypotheses gevormd die de basis vormen voor debeantwoording van de onderzoeksvraag. 




\paragraph{MODE CONFUSION }
Mode confusion tredd op als gepbserveerd gedrag van een technisch systeem niet past in het gedragspatroon dat de gebruiker in zijn beeldvorming heeft  en ook niet met voorstellingsvermogen kan bevatten.
\paragraph{Wat is automatiseringsparadox}
Gemak dient de mens. Als er veel energie wordt gestoken in de ontwikkeling van hulmiddelen die taken van werknemers overemen heeft dat tot resultaat dat veel productieprocessen worden geautomatiseerd. De vraag is dan of vanuit mechnisch wereldpunt de robot niet de rol van de mens overneemt en of de mens nog de kwaliteiten heeft om het werk zelf te doen.
\cite{bicker21102016automatiseringsparadox }
\cite{vseautoparadox }
\cite{blogxot21112016slimapparaat }


\paragraph{Wat is een model}

\paragraph{in vivo model}
Levende organismendie in de werkelijkheid of in een laboriatrum vergelijkbare eigenschappen bezitten als bestaande fenomenen in de werkeljkheid. Deze objecten zijn vergelijkbaar met werkelijkobjecten en geven vergelijkbare resultaten
\paragraph{in vitro model}
Een model dat dezelfde condities biedt  buiten het onderzoeksobject om, maar is voldoende vergelijkbaar om vergelijkbare processen te simuleren.
Zowel invivo als in vitro modellen zijn beperkt door de materialen die beschikbaar ijn voor onderzoek en de arbeidsomstandigheden waaronder ze worden gebruikt. Desondanks zijn het geen werkelijke natuurlijke modellen dus vvoor een onderzoek kan boedt het geen volledige uitsluitsel.
\paragraph{In silicio model}
Ee veelzijdig object. Het verwijst naar simulaties die gebruik maken van wiskundige modellen in computer,een zijn dus afhankelijk van silicone chips. In silico model analyseert  wiskundige vergelijkingen om resultaten te geven onder bepaalde omstandigheden. Deze vergelijkingen vertellen iets over de correlatie van verschillende objecten van een wetenschappelijk onderzoek. OM deze modellen te kunnen gebruiken is het noodzakelijk te omschrijven waat de fenomenen in kwestie van onderzoek zijn door middel van getallen. Kwanttitatieve relaties kunnen worden geintegreerd in het model en waar deze relaties complex zijn is een computer noodzakelijk deze op telossen. Vaak worden hierbij verschillende mechanismen gebruikt. Als je bijvoorbeeld de prijsontwikkeling van een marsreep in kaart wilt brengen.
\paragraph{in simulacra model}

\paragraph{World and machine samenvatting}
Waarom zijn wij engineers? Omdat we bruikbare apparaten willen laten functioneren in de wereld waarin we leven. Dat doen we door de machine te beschrijven en deze beschrijving van instructies bieden we aan onze computer opdat deze als de attribuut en gedragingen uitleest zoals wij die hebben omschreven. Dit alles op basis van theoretische funderingen en praktisch inzicht. 

Het doel van een machine is om te worden geinstalleerd en te worden gebruikt. De eisen die we stellen zitten in de omgeving en in de wereld en de machine is slechts de oplossing die we bedenken om aan een eis te voldoen. 

De relatie machine-wereld world gecategoriseerd in: 
Het modelleer aspect: waar een machine de wereld simuleert 

Het interface aspect: waar er fysieke interactie is tussen de machine en de wereld 

Het engineering aspect: waar de machine zich gedraagt als een controlemotor gebruikmakend van de gedragingen van de omgeving in de wereld 

Het probleem aspect: waar de omgeving in de wereld en de omvang van het probleem invloed heeft op de machine en de oplossing 

Het modelleer  of simulatie aspect over een deel van de wereld. Er zijn data,object en proces modellen. Het doel van een model is toegang te geven tot informatie over die wereld. Door het opvangen van statische weergaven en gebeurtenissen kunnen wij deze gebruiken van opgeslagen informatie die we kunnen hergebruiken. Een model kan bruikbare informatie bevatten omdat zowel het model als de wereld warin het model zich bevind gemeenschappelijke omschrijvingen hebben die waar zijn voor zwel het model als voor de wereld. Daarbij moet gesteld worden dat de interpretatie van een model verschilt met een interpretatie van de wereld. 

Omdat zowel de wereld als de machine fysieke realiteiten zijn an niet slechts abstracties, zijn de gemeenschappelijke beschrijvingen slechts een deel van de werkelijheid van beide objecten. For elk object zijn er meerdere beschrijvingen. Toch maken niet alle omschrijvingen deel uit van het getoonde reportoire. Zoals niet alle eigenschappen van een boek; meer dan een auteur, pseudoniemen, een onderdeel van een reeks, een gerevisiteerde versie, worden gereflecteerd in een database.  

Het interface aspect. Een machine kan een probleem in de wereld oplossen als de wereld en de machine phenomena kunnen uitwisselen. Maar de participatie is niet symmetrisch: een status kan als phenomena worden uitgewisseld maar slechts een partij kan er invloed op uitoefenen maar beiden kunnen dezelfde status signaleren. 

Het engineering aspect gaat over requirements, specificaties, en programma’s. Requirements hebben betrekking op phenomena in de wereld. Een programma heeft alleen betrekking tot de machinale phenomena. Het doel van programma’s is om eigenschappen en gedragingen te omschrijven van de machine ten behoeve van de gebruiker. Tussen de requirements en de programma’s zitten de specificaties. Omdat programma’s dan wel beschrijvingen zijn van een gewenste machine, maar dat moeten beschrijvingen zijn van de  machines  die de computers kunnen uitvoeren zodanig dat de computer deze beschrijvingen ook zo kan interpreteren. De engineer moet  de eigenschappen van de wereld kennen en begrijpen en deze eigenschappen manipuleren en laten werken met als doel het dienen van het systeem. 

Het probleem aspect. Het onderscheid tussen specificatie en implementatie. Het probleem zit in de relatie van de machine en de wereld. De machine brengt de oplossing maar het probleem zit in de wereld. Een vertoog over een probleem moet dus gaan over de wereld en over de opvatting die de gebruiker heeft in de wereld. Omdat de wereld veelzijdig is moeten we ervan uit gaan dat er verschillende soorten problemen zijn. Een realistisch probleem wordt dus niet opgelost met een simpele hiërarchische structurele aanpak en een homogene decompositie maar met een paralleele structurele oplossing waar beide kanten van het probleem worden opgelost. 



Ontkenningen 

We hebben als engineers de taak om een machine te bouwen aan de hand van de specificaties opgeleverd door de opdrachtgever. Een engineer heeft niet als taak de fitheid voor een doeleind te onderzoeken, maar wel de haalbaarheid naar een doeleind aan de hand van kennis, tijd, resources, budget en ontwikkelmethodiek. Daaruit komt naar voren dat een engineer zich richt op: elicitation (schetsen van een requirement), description (omschrijving) en analyse van de requirements waaraan het systeem moet voldoen. Vertaalt naar de volgende vragen: Wat is precies de klantwens?  Wat is de precieze omschrijving van het probleem? Voor welke doelen wordt het systeem gebouwd? Welke functies moet het systeem hebben? 

Denial by hacking: obsessief bezig zijn met een systeem omdat het de gebruiker veel macht geeft. Een uitgebreidheid van een systeem zorgt er soms voor dat mensen niet meer geprikkeld zijn na te denken over probleemstellingen, domein beschrijvingen en analyse. 

Denial by a abstraction. Wiskundige benaderingen van werkelijke problemen is  een belangrijke intellectuele strategie om problemen te formuleren. Een software ontwikkelaar moet een probleem kunnen omschrijven in zo min mogelijk woorden, maar de complexiteit ligt in de oplossing. 

Denial by vagueness. De vaagheid van een omschrijving is terug te vinden in: 

Von Neumann’s principe ,Principe van reductionisme ,Shanley principe en het Montaingnes’s principe.
Het Von Neumand principe uitgelegd
Voor een vocabulair  moet een grondslag zijn ontwikkeld waarmee gesproken kan worden over de wereld en de machine. Belangrijke phenomenen moeten geindtifieerd worden, door middel van een grondregel  of ‘herkenningsregel’ moet een fenomeen worden herkend, en vervolgens het fenomeen een formele term geven die gebruikt wordt als duiding van een bepaalde omschrijving. Dan moet voor de formele term een symbool gevonden worden. Samen vormen de grondregel en het symbool een designatie. 

Principe van reductionisme 

Simpelweg het openbreken van termen met een weerlegbare definitie totdat alle begrippen die worden gebruikt om iets te duiden  niet meer te herconstrueren zijn in hun definitie. 

Shanley principe 

Er bestaan volgens dit principe geen scherpe verdelingen in de wereld zoals wetenschappers soms denken. Een strenge opvatting over de wereld waarin een individu geclassificeerd kan worden als een onsamenhangend geheel. Maar dat is slechts een opname van een beeld. De werkelijkheid staat soms toe dat een elementair individueel object in verschillende classificaties verschillende getypeerd kan worden in een andere setting of view. 

Montaignes principe 

De incative mood; gaat over wat we beweren waar te zijn. 

De optitative mood; gaat over wat we willen dat waar is 
\paragraph{4 variabelen model}





Het 4 variabelen model kort toegelicht
Monitored variabelen: door sensoren gekwantificeerde fenomenen uit de omgeving, bijv temperatuur

Controlled variabelen: door actuatoren \bestuurde fenomenen uit de omgeving
For example, monitored variables might be the pressure and temperature
inside a nuclear reactor while controlled variables might be visual and audible alarms, as well
as the trip signal that initiates a reactor shutdown; whenever the temperature or pressure reach
abnormal values, the alarms go off and the shutdown procedure is initiated

Input variabelen: data die de software als input gebruikt
Here, IN models the input hardware interface (sensors and analog-to-digital converters) and
relates values of monitored variables to values of input variables in the software. The input variables model the information about the environment that is available to the software. For example,
IN might model a pressure sensor that converts temperature values to analog voltages; these voltages are then converted via an A/D converter to integer values stored in a register accesible to the
software.

Output variabelen: data die de software levert als output
The output hardware interface (digital-to-analog converters and actuators) is modelled
by OUT, which relates values of the output variables of the software to values of controlled variables. An output variable might be, for instance, a boolean variable set by the software with the
understanding that the value true indicates that a reactor shutdown should occur and the value
false indicates the opposite



\paragraph{6 Variable model}
Optitatieve statements omschrijven de omgeving zoals we het willen zien vanwege de machine. 

Indicatieve statements omschrijven de omgeving zoals deze is los van de machine. 

Een requirement is een optitatief statement omdat ten doel heeft om de klantwens uit te drukken in een softwareontwikkel project. 

Domein kennis bestaut uit indicatieve uitspraken die vanuit het oogpunt van software ontwikkeling relevant zijn. 

Een specificatie is een optitatief statement met als doel direct implementeerbaar te zijn en ter verondersteuning van het natreven vande requirements. 

Drie verschillende type domeinkennis: domein eigenschappen, domein hypothesen, en verwachtingen. 

Domein eingenschappen  zijn beschrijvende statementsover een omgeving en zijn feiten.Domein hypotheses  zijn ook beschrijvende uitspraken over een omgeving, maar zijn aannames. 

Verwachtingen zijn ook aannames, maar dat zijn voorschrijvende uitspraken die behaald worden door actoren als personen, sensoren en actuators. 

Het verschil tussen essentie en incarnatie van een systeem. Een essentie bevestigd de  mogelijkheden dat een systeem moet hebben om te voldoen aan de eise, ongeacht hoe het systeem is geimlementeerd. De incarnatie bevestigd of omvat de mogelijjkheden die te maken hebben met details omtrent implementatie. Een heuristiek voor het identificeren van de essentie van een systeem is de aanname van perfecte technologie, ofwel de aanname dat de technologie binnen een systeem perfect is. Om essentie te indentificeren nemen we aan dat technologie buiten de machine om perfect is. Zouden we incarnatie overwegen dan wordt de aanname van perfecte machin-externe technologie opgeheven. 

Voor de documentatie van contextuele beslissingen en opties/alternatieven wordt de OVM (Orthogonale variability Model) gebruikt. Oorspronkelik was deze methode bedoeld om de variatiepunten en de variant van een productlijn samen met hun variabele afhankelijkheden( mandatory, optional, alternative)  en beperkende afhankelijkheden(requires en excludes)te omvatten. De variant kan worden gerelateerd aan een ontwikkelartefact zoals een requirement of een diagram als een zogenoemde artefact dependency. Een artefact is dan gedefinieerd als variabele. Voor de documentatie van de keuzen die we maken is een selectie model gemaakt. We gebruiken het OVM voor de documentatie van contextuele beslissingen die moeten worden genomen, opties en alternatieven die selecteerbaar zijn, en de afhankelijkheden tussen hen. met behulp van de artefact dependency relateren we de alternatieven aan variabele elementen van de AND/OR graaf. Voor documentatie van de keuzes gebruiken we ook een selectiemodel. De kracht van het OVM model en de voornaamste reden deze methode te gebruiken is dat deze is in staat is om een variant te relateren aan een geheel model, een model element, of een selectie van een model. 

AND/OR graaf wordt gebruikt voor de documentatie van refinement/decompositie of requirements. De AND/OR graaf is een directe, asyclische graaf met nodes knopen die requirements voorstellen en lijnen die AND-decomposities voorstellen en OR-decompositiestussen de requirements. Een decompositie van een requirement in een set van subrequirements R1,….Rn is een OR-decompositie iff die dusdanig aan een subrequirement voldoet en daarmee voldoet aan requirement R. Wat moet worden gedocumenteerd met betrekkig tot de AND/OR graaf is de abeargumentering waarom elkeAND/OR-decomopositie  voldoende is. 
\paragraph{Conceptueel model}



System requirement:
uitspraak over wereld fenomenen (gedeeld of niet) of doelen
die bereikt moeten worden.
met enige regelmaat informeel, niet precies geformuleerd.
Software requirement/specicatie:
uitspraak over gedeelde fenomenen of doelen die de machine
moet bereiken middels de onderdelen waar die machine uit
bestaat of middels de fenomenen waar de machine controle
over heeft.
doorgaans preciezer, meetbaar, exact geformuleerd.


Systemen gaan een zekere interactie aan met hun omgeving:
Sensoren: meten fenomenen uit de omgeving (temperatuur,
druk, licht, geluid, etc.)
actuatoren: veranderen iets in de omgeving (mechanische,
electrisch, pneumatisch, etc.)
Software:
Kan niet direct communiceren met de buitenwereld.
Snapt derhalve niets van de buitenwereld.
Kan alleen maar bestaan in en communiceren met het
systeem.


\paragraph{Requirementsengineering}

Om de juiste requirements te verzamelen en selecteren hebben we meer kennis nodig van de methoden hiervoor gebruikt in het domein van requirementsengineering. Daarom is een literatuurstudie gedaan naar rapporten en artikelen die ons meer informatie over dit onderwerp verschaffen.
 Uitdagingen in requirementsengineering zijn incomplete requirements en specifcates, veranderende requirements en specificates en grote, complexe oftwaresystemen.
 
 Het article the worlds a stage biedt inzicht in de requirementstechnieken voor een ambulance in london. In het artikel gaan de onderzoeks in op de volgende onderwerpen: 
 viewpoints, sociale ascpecten,evolutie, non-functional requirements, conflict resolution, traceability
 
 Goal of this paper is requirement  engineering on London aulance service
 Method of opinions: crew, staff, management, computational, transport, services
 Evolutioon: changes, specification and technology trade
 Environment: company policies, regulation, impact solution on organizational
 Non-functional aspect: communicatio problem, malfunctions, less critical isues: cost, tradeoff beween performance \& user interfaces
 vieuwpoint: is a subset of all system requirements expressible in a given requirements notation regardless of the stakeholders involved
 
 log change
 basic model vieuw
 hypertext vieuw
 data transmission problems
 continued difficulties
 installation problems
 problems caused by mistake
 tracebility requirements[selecting reliable information]
 PRE requirement specification traceability, repository baed approach
 1) compromise specification
 2) representatives
 3) agreement dimensions
 Domain: part of the worl in which the computer system effects will be felt, inclusing its peoples, organizational structure, related legislation, physical location and met only the compyter systems
 
 
 Het artikel "from inconsistencyhandling to non-conanical requirements management: a logical perspective" geeft enkele tips voor het omgaan met inconsistente requirements:
 
 1) identifying non-canonicalrequirements
 2) measuring them
 3) generate caandidate proposals for handling them
 4) choosing acccptable probosals
 5) revising them acccording to the proposals

Het artikel "managing inconsistent specification: reasoning, analysis, action" zoekt een ontologische benadering voor het omgaan met inconsistenties in de requirements specificaties.
Voor de omshrijving van een specificatie kun je gebruik maken van logica. Daarbij kun je onderschei maken in klasieke logica quasi -logica.
Wat ook een rol kan spelen in domain interpretatie. De achtergrond van de gebruikers speelt ook een rol.
Zo is er e=onderscheid te maken in de volgende groepen: users, customers, domain experts, designers,, manufacturers
graphical  textual specification

Basic constraint, legal constraint, cooperation constraint
1) scenatio  definition
2) scenario analysis
3) scenario consolidation


Hoe kan een systeem verder worden ontworpen op een manier dat non-functionele requirements worden geimplementeerd?
Hoe hangt dat ontwerp samen met aanpassingen van het functionele en structurele aspect van het systeem?

block[objects, classes, methods, messages, inheritance]
[goals,agents, alternative, events, actions,existence modalities,agent responsibilities]


Het artikel "representing and using nonfunctional requirements: a process-oriented approach"" gaat in op een het proces van requirements acquisitie. Hierbij in ogenschouw de acquisitie van prestaties, ontwerp en aanpasbaarheid.
product oriented
process oriented


Acquisitie Prestaties
user concern
-Hoe goed werkt het product
-Hoe goed wordt de bron gebruikt?>> Efficiency
-How veilig is het product >> integrity
-Met hoeveel zekerheid is uit  te sluiten dat het werkt >>Reliability
-Hoe goed werkt het product onder zware omstandigheden >> sustainability
-Hoe makkelijk is het product in gebruik >> usability
quality attribute


Acquisitie: Ontwerp
user concern
Hoe valide is het ontwerp
-Is ht ontwerp conform de requirements
-hoe makkelijk is het ontwerp te repareren
-Hoe makkelijk zijn de prestaties te verifieren

quality attribute


Acquisitie: Aanpasbaarheid
user concern
-hoe makkelijk is het om het product aan te passen
- hoe makkelijk is het om het product te updaten en/of uitbreiden>> expendability
- hoe makkelijk is het om een wijziging door te voeren>>flexibility
-hoe makkelijk is het om andere system aan te sluiten >> portability
- hoe makkelijk is het om het product te transporteren >> interoperability
-hoe makkelijk is het om te converteren tot een systeem gebruiksklaar voor communiceren met andere systemen>> reaseability
quality attribute




 \cite{jonkerTreurKlush200informativeAgents}
\cite{boehmBoseLeeRequirementsNegotiations}
\cite{muHungJinLiu2013inconsistencyReqs}
\cite{hunterNuseibeh1996manageSpecs}
\cite{myloloupos1992representingReqs}
\cite{zavePamela4darkCorners}
\cite{zavePAmela1997regEngineering}

%%%%%%%%%%%%%%%%%%%%%%%%%%%%%%%%%%%%%%%%%%%%%%%%%%%%%%%%%%%%%%%%%

what is a good software specification

\cite{fvaandrager2322010Goodmodel}
\cite{onix01102022devopmodel}
\cite{sulemani04012021softwareprocesmodel}
\cite{globalluxsoft18102017softdev}
\cite{wiegers30052022SRS}
\cite{muller06092020goodspecification}
\cite{informit30062008reqmanagement}
\cite{altexsoft15092020writingSRS}





\paragraph{bijlmerramp}

\begin{description}
	\item[Beschrijving] 04/10/1994
	\item[Datum en plaats] 
	\item[Oorzaak]
	%Beschrijf wat er mis ging in termen van het vier variabelen model/requirements/specificaties
\end{description}


\cite{aviationsafety04101992airplaneCrashBijlmer}

\paragraph{ramp turkisch airlines vlucht 1951}

\begin{description}
	\item[Beschrijving]
	\item[Datum en plaats] woensdag 25 februari 2009
	\item[Oorzaak]
	%Beschrijf wat er mis ging in termen van het vier variabelen model/requirements/specificaties
\end{description}
Op woensdag 25 februari 2009 start een vliegtuig van Turlisch airlines neer met 9 doden als gevolg.
Conclusies van het ovv luiden :Inadequaat handelen van de piloten ondanks een defecte hoogtemeter en onvolledige instructies van de luchtverkeersleiding.

\cite{catsr25022009Boeing737AmsterdamCrash}

\cite{zuilen23022019Tijdlijnpoldercrash}
\cite{wikinews04032009techfoutailines1951}
\cite{luchtvaartnieuws21012020boeing737conclusies}
\cite{adformatie280220209communicatiegebreken}
\cite{spinnael25022009onderzoekpolderbaancrash}
\cite{crashTurkishAirlines}
\cite{flightradar24}
\cite{flightstatstracker}


\paragraph{therac-25}


\begin{description}
	\item[Beschrijving]
	
	Softwarefout uit zich als hardwarefout de klachtafhandeling geen onderzoek geen second opinion is prioriteit wel 
	gechecked na onderzoek bellen en geen prioriteit aanwezig te zijn alleen importeurs en fabriken mogen fouten 
	in frabrieksinstellingen rapporteren 
	Therac25 Systeem ligt plat veel voorkomende eror stdaardafhandeling om de error te verwerpen resultaat: 
	de patient kreeg overdosis patient overleden onderzoek opgestart, stuatie niet reproduceerbar foutmarkering: 
	gezien als uitzonderlijk, software aanpassing van groote magnitude 5; de oorzaak was waarschijlijk mechanisch 
	maar neit vastgesteld; conceptueel odel niet aangepast probleemclassicificatie door autorititen het probleem 
	en de impact daarvan anar beneden bijgesteld AEFL doe gedeeltelijke aanpassing om hardware na berisping 
	Canadese autoriteit 
	Derde patient overleden door eythema AECL wijst alle doodsoorzaken af AECL beweert dat geen vergeli- 
	jkbare voorvalle bij andere machines of patienten zijn voorgekomen geen vervolgonderzoek vanwege garanties 
	bedrijf gaat uit van geen mogelijke functionele fout 
	vierde patient overleden aan overdodis ontstaan door bug in software onjuiste aanduiding bij de foutmelding 
	verkeerde reactie/invoer ddoor operator communicatie tussen patient en operator werd onvoldoende gemon- 
	itorred ( apparatuur niet aangesloten, en audio monitor kapot) engineer van AECL stelt geen fouten vast 
	Engineer AECl kan fout niet reproduceren Geen communicate tussen bedrijf en uitgezonden technisci over 
	vergelijkbare probleemgevallen 
	vijfde geval malfunction 54 leidt tot overdosis en de dood fout gereproduceerd door operator bedrijf fout 
	was daa entryspeed herpublicatie van de ongevallen en de eerdere ongevallen in de meia apparaat wel nog in 
	gebruik genomen niet handig, waarschuwingsberichten en aanwijzingen voor een bugfix naar de gebruikers door 
	druk van fda is bedrijf op zoek gegaan naar permanente oplossing 
	zesde geval software fout door softwarefout otntstaat lightstruct .. op de patient na onderzoek door AECL 
	blijkt niet alleen hardware de oorzak gebruikers direct geinformeerd oplossing gevonden, media ingeschakeld om transparantie af te dwingen door de gebruikersgroep en de FDA AECL gedwongen functionaliteit aan te passen 
	Engineers hebben meer studie moeten maken van gebruikte technologie en onderhoudbaarheid daarvan 
	sheets
	\cite{rogaway2004therac25}
	~\cite{wikiTherac25}
	reproduceren van de error. IN dit stuk wordt uitgelgd hoe het product werkt en waarom bepaalde beslssingen zijn genomen in de ontwerp/productiefase
	\cite{lynch2017theracRaceConditions}
	kort artikel met daarin een opsomming van alle fouten in het systeem en een korte uitleg
	\cite{lim1998theracdisaster}
	uitgebreid artikel over hoe de fout werd gereproduceerd en de resultaten daaruit voortkwamen. Alsnog werden er na de reproductie fase nog meer fouten gevonden.
	\cite{fabio26102015therac25}
	artikel
	\cite{ethicsunwrappedTherac25}
	onderzoeksartikel waarin de bug wordt uitgelgd: de racecondities, de bytepositie en het testen worden berkitiseerd envenals andere onderdelen van het softwareproces
	onrealistisch testplan. In dit artikel egt de auteur het belang nog eens uit van goede requirements en implementatie, niet de software is waar het probleem ligt
	geschiedenis
	\cite{casesHistoryTherac25}
	artikel
	\cite{caballero2019Therac25}
	computer error. De ongeval en de malfunction nog een keer uitgelegd
	\cite{rose1994theracFatalDose}
	rapport
	\cite{levesonMITTherac25}
	\cite{grant1978theracevaluation}
	onderzoeksartkel
	\cite{turnerTheracAccidentsInvestigations}
	\cite{turner1993TheracAccidentsInvestigations}
	uitgebreid artikel gaat hier ook wat meer over de hardware
	\cite{wang2017industrialdesignengineering}
	artikel waarin in 3 delen de problemaiekwordt blootgesteld
	\cite{levesonturner1993theracpart2}
	case study sheets
	artikel waarin vooral de fabriikant ervan langs krijgt
	\cite{porelloTheraccFailure}
	lessons learned. Vooral de begrippen betrouwbaarheid, welgevalligheid, veilgheid en gebruiksvriendelijkheid
	\cite{theracIncidents}
	root-cause analysis
	case study
	\cite{huffbrown2004casestudyethicatherac}
	case study
	\cite{sebowikimedicalradiation}
	opzetten van systematische acceptaatie test met therac als voorbeeld
	\cite{hsia1995testtherac25}
	artikel waarin een diagnose plaatvindt voor het bedrijf en de ingenieur/ontwerper
	\cite{magsilvaTheracTesting}
	rapport
	oorzaken aangegeven in artikel
	\cite{chemeuropetherac25}
	het onderzoek en enkele ontwerptekeningen en oplossingen
	\cite{statsenko10102016Therackillerbug}
	\cite{therac25casestudy}
	\cite{thomas1994theracinLotos},
	\cite{twitter2019programmerbehindtherac}
	wiki
	\cite{wikibookstherac}
	analyse
	\cite{bozdagTherac25}
	samenvatting
	\cite{levesonTurnerTheracAbstract}
	rapport over de fouten die de verschillende partijen hebben gemaakt( overheid, ingenieurs, bedrijf, operators) en de verbeterpunten
	onderzoeksrapport
	slides online over het technisch mankement
	Wat is er gebeurd, nou het volgende:
	Normal radiation treatments: 6,000 rads over a 3 week period, under certain conditions Therac-25 was delivering 60,000 rads during one session.
	En wat ging er mis?
	Paradigm Shift
	Therac-25 replaced expensive hardware safety interlocks with software controls
	Real-time software
	Design
	Race condition caused focusing element to be incorrectly set
	No indication of actual hardware settings
	Error messages appeared the same regardless of how important
	Error messages were difficult to understand
	All errors messages could be manually overridden
	oorzaak-gevolg diagram
	veiligheidsanalyse naar de rapportage van foutmeldingen, de beslissingsmatrix waarmee het programma wordt uitgevoerd en de software-analyse door een consultat
	\cite{stackexchange2021therac25code}
	
	\item[Datum en plaats] 
	June 1985 and January 1987
	\item[Oorzaak]
	%Beschrijf wat er mis ging in termen van het vier variabelen model/requirements/specificaties
\end{description}






\paragraph{tesla crash report}

\begin{description}
\item[Beschrijving]


Door een softwarefout zijn er situaties ontstaan waarin het systeem informatie een onvoldoende informatie positie had om de juiste beslissingen te maken. Of dat de informatieverwerking niet juist was.


tesla autopilot crashes

\cite{teslaFDSCrash}
\cite{teslaCrashesCauses}
\cite{teslacrashOvervieuw}
\cite{tesladeaths}
veiigheidsrisico

\cite{evan01042019teslaautopilotIntersection}
\cite{testVehicleSafetyReport}
veiligheidsrapport mbt autopilot
\cite{lambert31062020q2safetyreport}
consumentenrapport
bluetooth veiligheidsvraagstuk
\cite{wiredBloutoothHackTesla}
veiigheidsvraagstuk vanwege touch screen
\cite{preston14012021NHTSATeslaRecall}
veiligheidsvraagstuk
\cite{cio25112020belgianTeslaHack}
veiligheidsvraagstuk
rapport over autopilot
\cite{templeton06092019HTSBReportTesla}
de invloed van de bestuurder bij tesla ongeluk
veiligheidsvraagstuk
\cite{darkReading17112020TeslaBackup}
veiligheidsvraagstuk
\cite{leyden23032020TeslaInterfaceHack}
veiigheidsvraagstuk
\cite{huddlestonjr03042019ChineseTeslaHack}
veiligheidsvraagstuk
veiligheidsvraagstuk
\cite{heilweil26022020teslaAutopilot}
rapport over ongeluk
veiligheidsvraagstuk
veiligheidsvraagstuk
\cite{blanco04102019NHTSATesla}
veiligheidsvraagstuk
ransomware aanval op tesla
tesla batterij is veiligheidsvraagstuk geworden
\cite{mitchell01072020teslabatterycooling}
ongeluk
\cite{bbc26022020AutopilotCrash}
veiligheidsvraagstuk
veiligheidsvraagstuk
\cite{stumpff04052020TeslaPersonalData}
dodelijk ongeluk
\cite{levin08062018teslaautopilotsafety}
veiligheidsvraagstuk: ransomware
veiligheidsvraagstuk: medewerker in de fout
\cite{cbrook06082021TeslaInsideDataThreft}
\cite{shilling25022021Tesla}
veiligheidsvraagstuk: hackers je systeem laten testen
verdedigen tegenover ransomware
veiligheidsrisico
prijzen omlaag
autopilot
\cite{randall05112019modelSurvey}
malware door een medewerker
dodelijk ongeluk
\cite{fottrell03092018TeslaSecurityChecks}
waarom een tesla stelen bijna onmogelijk is



veiligheidsonderzoek



softwarefout maakt diestal mogelijk


\cite{kirk26112020modelX}
fouten ontdekt in onderzoek
\cite{bbc24022021hyundaiBatteryFireFix}
tesla cloud gehacked
\cite{hawkins22102022}
\cite{gritti24062020tesladataengine}
\cite{bouchard07052019teslaDeepLearning}
\cite{Srikanth2019teslabigdata}
\cite{rangaiah25022020teslaAI}
\cite{marr08012018taslabigdataAI}
\cite{bdickson29072020teslalevelfive}
\cite{dcruz17062022tesladesignthink}
\cite{mcfarland22042021selfdrivingrisks}
\cite{hawkins18032021fedgovinvest}
\cite{berry21042021teslacrashtexas}
\cite{hull23072021regulatorsaftercrash}
\cite{wikiTeslaAutopilot}
\cite{nhtsaAutomatedVehiclesSafety}
\cite{dowling23042021autopilottricking}
\cite{wilson19042021teslacrashregulators}
\cite{seamans22062021aikillerap}
\cite{mitchell24022020AIDataTesla}
\cite{denneyjdsupraFeds}
\cite{siddiqui22102020TeslaCriticism}
\cite{ackerman01072016TeslaImperfect}
\cite{greene04092019misuseautopilot}
\cite{michralli26112019ubserautocarcrsash}
\cite{pitmann21072021wrongfullautodeath}
\cite{stackexchange102019teslacarmistake}
\cite{tasking07062017TeslaAugmentedSafety}
\cite{griemannExaminSelfDriving}
\cite{Harkey30052019SafeSystemVehicle}


tesla crash report



\cite{shepardson18062021TeslaDeaths}
\cite{hawkins30062021nhtsarequiresreporting}
\cite{hawkins10052021autopilotnotavailable}
\cite{szymkowski29062021nhtsaTeslaCrashReports}
\cite{abc1112052021AutopilotNotinTeslaCrash}
\cite{ankel18062021regulatorsinvestigateAutopilot}
\cite{sommerfield12072021NHTSAmandateresult}
\cite{saferoardsCrashesAutonomousvehicles}
\cite{stephardson18032021revieuwingtesla}
\cite{krishner30062021NHTSAreport}
\cite{gitlin11052021autopilot}
\cite{mitchell19012017investigationstop}
\cite{gordon10052021teslaprelimreport}
\cite{shaper07062018}
\cite{cochran18042021nodriverTeslaCrash}
\cite{habib28062016NHTSATeslaReport}
\cite{firstpress11052021fatalnonautopilot}
\cite{raynal20042021probeTeslaCrash}
\cite{tiungteslasoftwarecrash}
\cite{globaltimes08052021guangdongcrash}
\cite{anderson30042021secondteslacrash}
\cite{oremus21062017fatalTeslaCrash}
\cite{guardian15052021teslacrashHandsOnWheel}
\cite{Puzzanghera13092017TeslaSharesBlame}
\cite{jaillet02022017teslaAutopilotLimitations}
\cite{reuters03102019teslaAutoParkingFail}
\cite{dowling23042021}
\cite{young05112021fatalTeslaReport}
\cite{kierstein18032021teslaAutopilotCrashStationary}
\cite{janssen20062017teslacrashdetailflorida}

tesla crash publications overview

\item[Datum en plaats] 
\url{https://www.autopilotreview.com/tesla-autopilot-accidents-causes/}
\url{https://www.skynettoday.com/briefs/tesla-investigations}
\url{https://www.tesladeaths.com/}
\url{https://www.washingtonpost.com/technology/2023/06/10/tesla-autopilot-crashes-elon-musk/}
\item[Oorzaak]
%Beschrijf wat er mis ging in termen van het vier variabelen model/requirements/specificaties
\end{description}

\paragraph{slmramp}

\begin{description}
\item[Beschrijving]

Toen de Anthony Nesty Zanderij naderde, was het daar, anders dan het weerbericht had voorspeld, mistig. Het zicht was evenwel niet zo slecht dat er niet op zicht kon worden geland. Gezagvoerder Will Rogers besloot echter via het Instrument Landing System (ILS) te landen, hoewel dit niet betrouwbaar was en hij voor zo'n landing ook geen toestemming had. De gezagvoerder brak drie landingspogingen af. Bij de vierde poging negeerde de bemanning de automatische waarschuwing (GPWS) dat het toestel te laag vloog. Het toestel raakte op 25 meter hoogte twee bomen. Het rolde om de lengteas en stortte om 04.27 uur plaatselijke tijd ondersteboven neer.
\item[Datum en plaats] 07/06/1989
\item[Oorzaak]
%Beschrijf wat er mis ging in termen van het vier variabelen model/requirements/specificaties

Uit onderzoek bleek dat de papieren van de bemanning niet in orde waren. 
Geconcludeerd werd dat de gezagvoerder roekeloos had gehandeld door voor een ILS-landing te kiezen terwijl hij daar geen toestemming voor had, en door onvoldoende op de vlieghoogte te hebben gelet. 
De SLM werd verweten de kwalificaties van de bemanning onvoldoende te hebben gecontroleerd.



\cite{espnSLMterugblik}
\cite{dennisRosier01052020}
\cite{hassing07062020slmramp}
\cite{amsterdamArchiefSLM}
\cite{rtvOost06062019nabestaande}
\cite{breda07062021AndroSnel}
\cite{andereTijdenSLMCrash}
%\cite{wikiSLMRamp}
database
\cite{aviationReport}
rapport
\cite{aviationSLMCrashAccidentInvestigation}
\cite{mcDonnelDouglasCommissionReportSLMCrash}
\cite{wikiSRFlight764}
\cite{nos07062019SLMTerugblik}
\cite{dagvantoenSLMCrash}
\cite{waterkantNesty07061989}
uitgebreid engels artikel
\cite{eduNandlalSRCrash}
ntsb investigtion
\cite{oldjetsSRAirways}
uitgebreid engels artikel
\cite{cloudberg02012021srflight764}
persbericht
\cite{apnews07061989srplanecrash}
Wat is de rol van de autoriteiten?
Welke andere betrokkeen? Enw at is hun verantwoordelijkheid
Hadden de negatieve gevolgen voorkomen kunnen worden?
Hoe werd er over veiligheid gedacht?

\end{description}



\paragraph{schipholbrand}

\begin{description}
\item[Beschrijving]

Wat is er gebeurd?
\cite{schipholbrand27102005video}
artikel
\cite{schipholbrand27102005video}
psychologische gevolgen
rapport
\cite{onderzoeksraad2610schipholoost}
artikel met video
herdenking
impact op de persoon
herdenking
\cite{schipholbrandvideoargos}
chronologie
\cite{nunl30052023feitenoverzicht}
tijdlijn
vervolgens van ministers
beeldanalyse en reconstructie
\cite{}
herdenking
korte samenvatting
rapport
artikel
verwijzing naar het rapport vanuit de politieke oppositie
beeld vanuit de gevangenisbewaarder
nationaliteit slachtoffers schipholbrand
verblijfsvergunning voor de slachtoffers
gen schadevergoeding voor de verdachte
verdachte voor de rechter
geen schadevergoeding voor verdachte
artikel wat ging er mis bji de schipholbrand
brand veroorzaakt door een peuk
smaadschrift
bewakers worden niet vervolgd
proces schipholbrand moet over en de brandveilgheid moet worden verbeterd
de rol van het parlement in de evaluatie
\cite{parlementairemonitorschipholbrand}
onderzoeksmemo
herdenking
herdenking
invloed van de ramp op samenleving
\cite{videonpoNOVA13112008}
opmerkelijk rapport gestolen in de nasleep
\cite{rizoomes01052014schipholbrand}
publicaties
\cite{heuvelkroesschipholbrandcamerabeelden}
Wat waren de regels destijds?
Waren de autoriteiten in staat om op tijd in te grijpen of om erger te voorkomen?
Wat is er gedaan om de veiligheid van illegalen en gevangenissbewaarders te verbeteren
Wat is er gebeurd?
\cite{wikiSchipholbrand},\cite{schipholbrand27102005video}
psychologische gevolgen
rapport
\cite{onderzoeksraad2610schipholoost}
artikel met video
herdenking
impact op de persoon
herdenking
\cite{schipholbrandvideoargos}
chronologie
\cite{nunl30052023feitenoverzicht}
tijdlijn
\cite{singeluitgeverijenSchipholbrand}
vervolgens van ministers
beeldanalyse en reconstructie
\cite{eenvandaagschipholbrand}
herdenking
korte samenvatting
rapport
artikel
verwijzing naar het rapport vanuit de politieke oppositie
beeld vanuit de gevangenisbewaarder
nationaliteit slachtoffers schipholbrand
verblijfsvergunning voor de slachtoffers
gen schadevergoeding voor de verdachte
verdachte voor de rechter
geen schadevergoeding voor verdachte
artikel wat ging er mis bji de schipholbrand
brand veroorzaakt door een peuk
smaadschrift
bewakers worden niet vervolgd
proces schipholbrand moet over en de brandveilgheid moet worden verbeterd
de rol van het parlement in de evaluatie
\cite{parlementairemonitorschipholbrand}
onderzoeksmemo
herdenking
herdenking
invloed van de ramp op samenleving
\cite{videonpoNOVA13112008}
opmerkelijk rapport gestolen in de nasleep
\cite{rizoomes01052014schipholbrand}
publicaties
\cite{heuvelkroesschipholbrandcamerabeelden}
Wat waren de regels destijds?
Waren de autoriteiten in staat om op tijd in te grijpen of om erger te voorkomen?
Wat is er gedaan om de veiligheid van illegalen en gevangenissbewaarders te verbeteren
\item[Datum en plaats] 27/10/2005
\item[Oorzaak]
%Beschrijf wat er mis ging in termen van het vier variabelen model/requirements/specificaties
\end{description}



\paragraph{explosie tanjin china }

\begin{description}
\item[Beschrijving]

Later bleek uit een onderzoek van de Chinese autoriteiten dat de explosie overeenkwam met de ontploffing van 450 ton TNT.[6] 
De oorzaak van de explosie lag in de spontane zelfontbranding van 207 ton cellulosenitraat dat in containers was opgeslagen op het terminalterrein.[6] 
Verder lag op een tweede locatie nog eens 26 ton van dit explosieve materiaal opgeslagen.
De tweede ontploffing werd versterkt door de opslag van 800 ton kunstmest in de vorm van ammoniumnitraat in de nabijheid.[6]
De opslag van cellulosenitraat is aan strenge regels gebonden. Het moet koel en droog worden opgeslagen. De containers stonden buiten opgesteld in de brandende zon. De temperatuur liep op tot 36 °C en bereikte binnen de containers waarschijnlijk de 65 °C.[6] De verpakking van de cellulosenitraat droogde uit waardoor de ontploffing kon ontstaan. Op het terrein lagen meer gevaarlijke stoffen opgeslagen dan waarvoor vergunningen waren verstrekt.[6] Dit leidde tot een kettingreactie met grote schade tot gevolg. Door de brand en bluswater is in de directe omgeving veel milieuschade opgetreden.


https://www.hindawi.com/journals/joph/2019/1360805/ 
\cite{jiang16042019TanjinExplosion}
verhaal van brandweermannen
\cite{staff31082015tanjinblastunrevealed}
artikel
\cite{chinafile18082015tanjinexplosion}
invloed van social media
\cite{pinghuang2410201TanjinFactreport}
gemaakte fouten
\cite{portoTanjinExplosionSight}
\cite{imago17082015TanjinApartmentImages}
\cite{trager14082015Chemicalblast}
\cite{pangeramo27082015TanjinExplosion}
vergelijking met andere explosies
\cite{ap06082020ammaniumnitrate}
invloed van de ramp op de industrie
\cite{morris14082015TanjinIndustryImpact}
is er sprake van een doofpot
\cite{milesyu20082015exposingtoxicgovlines}
eigendomsverzekering
\cite{artemis30032016tanjininsurance}
\cite{aidenxiatanjinblast}
effecten op de lange termijn
\cite{danwangTanjinflexreport}
\cite{keyHighlightsTanjin}
lessons learned
\cite{hartley13082015videofootage}
\cite{odonnel01062017firetanjinblast2015}
gevolgen voor de industrie
\cite{fan15082015newyorkermistrustchina}
framing vanuit de chinese media
\cite{yanlidongchinamediaframingTanjin}
\cite{evans27092017TnjinInsurance}
niewsartikel
\cite{jasi26032019chineschemplant}
\cite{shiqingTanjinExecutiveSentence}
toegang tot de ramplplek vanuit de okale journalistiek
\cite{sophiebeach15082015}
artikel
\cite{hamzeh05082020BeirutBlast}
\cite{chemwatch18082015TanjiinExplosion}
\cite{thehindu15062019chinaExplosion}
\cite{santagotimes24032019chinablast}
oorzaken
\cite{klingecorp28042020causedTanjin}
case study
\cite{mcgarryExplosions2017}
niewsartikel
\cite{roswnfeld13082015TanjinReports}
chronologische uiteenzetting
\cite{aria12082015explosionaTanjin}
corruptie
mismanagement als oorzaak
autoriteiten publiceren onderoeksrapport
\cite{tremblay11022016chineseInvestigatorsTanjin}
fotos van de rampplek
\cite{taylor13082015TanjinExplosianAftermath}
niuwesartiekel
\cite{associatedPresss13082013}
\cite{un20082015InvestigationTanjin}
\cite{france2412082015TnjinExplosion}
\cite{npr14082015TanjinCause}
123 verantwoordelijken
\cite{bbc05022016TanjinResponsibles}
lang artiekel
\cite{CBodeen15082015TanjinExplosion}
\cite{reutersTanjinInsurance}
\cite{yu082016evaluationTanjin2015}
\cite{wiki2015TanjinExplosions}
\cite{bbc17082015whathappenedTanjin}
\cite{mortimer19082016taijinexplosioncrater}
veiigheidshandhaving
\cite{internationallabourofficeChmControlTooliit}
\cite{euTaxationCustomsICSC}
\cite{iloWHOChemSafetyCards}

\item[Datum en plaats] 12/08/2015
\item[Oorzaak]
%Beschrijf wat er mis ging in termen van het vier variabelen model/requirements/specificaties
\end{description}

%@online{landryalameddine12112020beiruthelathsystem,	ALTauthor = {author},	ALTeditor = {editor},	title = {title},	date = {date},	url = {"https://bmchealthservres.biomedcentral.com/articles/10.1186/s12913-020-05906-y"},}
%@online{ID,	ALTauthor = {author},	ALTeditor = {editor},	title = {title},	date = {date},	url = {"https://news.sky.com/story/beirut-blast-cctv-captures-moment-huge-explosion-devastated-hospital-12047452"},}
%@online{ID,	ALTauthor = {author},	ALTeditor = {editor},	title = {title},	date = {date},	url = {"https://www.unodc.org/unodc/en/frontpage/2020/September/unodc-assists-lebanon-in-reestablishing-container-shipments-in-the-aftermath-of-the-port-of-beirut-explosion.html"},}
%@online{ID,	ALTauthor = {author},	ALTeditor = {editor},	title = {title},	date = {date},	url = {"https://reliefweb.int/sites/reliefweb.int/files/resources/LEB201-Lebanon-Emergency-Response.pdf"},}
%@online{yadav07082020handlingexplosivesBeirut,	ALTauthor = {author},	ALTeditor = {editor},	title = {title},	date = {date},	url = {"https://www.downtoearth.org.in/news/governance/beirut-blast-lessons-time-for-india-to-strengthen-handling-of-explosives-chemicals-72707"},}
%@online{graham21082020rootsImpactBeirutBlast,	ALTauthor = {author},	ALTeditor = {editor},	title = {title},	date = {date},	url = {"https://www.justsecurity.org/72122/the-cost-of-resilience-the-roots-and-impacts-of-the-beirut-blast/"},}
%@online{ID,	ALTauthor = {author},	ALTeditor = {editor},	title = {title},	date = {date},	url = {"https://www.fire-magazine.com/the-port-of-beirut-explosion-a-timely-reminder"},}
%@online{neusaeter07082020beirutexplosioneval,	ALTauthor = {author},	ALTeditor = {editor},	title = {title},	date = {date},	url = {"https://www.ctvnews.ca/sci-tech/mapping-the-beirut-explosion-what-the-impact-would-look-like-in-canadian-cities-1.5053932"},}

 


\paragraph{ethiopian airlines}

\begin{description}
\item[Beschrijving]
\item[Datum en plaats] 10/03/2019
\item[Oorzaak]
%Beschrijf wat er mis ging in termen van het vier variabelen model/requirements/specificaties
\end{description}
Ethiopian Airlines Flight 302



\cite{caliskan09112013747boeingkalman}
\cite{gates18112020boeingcrisis}
\cite{boeing737maxsoftwareprobles}
\cite{avetisov19032019boeingmalwarestate}
\cite{thompson23112020nationalsecurityboeing}
\cite{gates21032019FAAControlSystem}
\cite{faa18112020boeingreview}
\cite{wiki737maxgroundings}
\cite{campbell02052019boengcrashhumanerrors}
\cite{hawkins22032019737maxairplanes}
\cite{thomas30082020737safest}
\cite{boeing737maxdisplay}
\cite{fehrm24112020737changes}
\cite{travis18042019737maxsoftwaredevop}
\cite{barnett05052019737maxcrisis}
\cite{easa27012021737maxsafereturn}
\cite{touitou11032019737tragedies}
\cite{hemmerdinger02022021737maxdeliveries}
\cite{bielby27022021faaimprovesafety}
\cite{boyle18112020737maxupgrade}
\cite{bergstraburgess122019737maxMcasAlgorithm}
\cite{737mcas}
\cite{newburger17052019boeingcrisis}
\cite{arstechnica22012020737problems}
\cite{german190620217372yaftergrounded}
\cite{beningo02052019boeinglessons}
\cite{duran05042019boeingspof}
\cite{makichuck24012021737fearflying}
\cite{caa737modifications}
\cite{oestergaard14122020boeingdeliveries}
\cite{reenberg787flaws}
\cite{fitch16092020737backlogrisks}
\cite{willis27082020737maxfailures}
\cite{ostrower11062020more737changes}
\cite{hruska13122019faaknown737crashrate}
\cite{bloomberg26092019failedpred}
\cite{whiteman09072020boengcancelstock}
\cite{leopold09192019boeingreliability}
\cite{koenig11122019737crashesnofix}
\cite{dohertylindeman15032019737problems}
\cite{stodder02102019corruptoversight}
\cite{afacwaLostSafeguards}
\cite{swayne18032019profitssafety}
\cite{freed26022021liftaustraliaban}
\cite{reed15032019softwareattention}
\cite{news17032019softwareexplains}
\cite{legget21122020eu737maxsafe}
\cite{marketscreener0103221737chinarecertification}
\cite{euractiv22022021737firegrounds}
\cite{benny18022019737returnUAE}
\cite{biersmichel22022021777grounds}
\cite{reuters23022021777metalfatigue}



\paragraph{Mali}

\begin{description}
	\item[Beschrijving]
	
	Een granaat explodeerd in een mortier
	De medische zorg na het ongeval was neit voldoende
	
	
	De algemeen militair verpleegkundige gaf aan het slachtoffer nar het vn-hospitaal in kidal te brengen
	De chaauffeur van de bushmaster kende de locatie niet  en bracht het slachtoffer naar een door frane militairren bemand hospitaal mmet minder mediswche faciliteiten
	Hierna alsnog overgebracht naar het vn-hospitaal.
	Dit verlieop  neit door nederlandse maatstaven.
	pas toen een nederlandse arts arrivveerde werd door de Tongolese artsen een buikoperatie uitgevoerd.
	Dit gebrurde zonder adequate anesthesie.
	Na de operatie werde de gewonde militair overgelogen naar nederland. En later naar nederland.
	
	
	granaat stond niet op scherp en in afgegaan in veilige stand
	Granaat werd opgeslagen in neit gekoelde containers waardoor deze aan te hoge temeperaturen zijn blootgesteld.
	Door de comvinatie van vocht en warmte in de granaat zeer gevoelige explosieve stoffen werden gevormd.
	Tijdens de oefening was de fatale granaat in de zon.
	Het afsluitplaatje in de granaat bleek niet in staat om doorslag in veilige stand te voorkomen waarna de granaat explodeerde.
	De moritren zijn aangeschaft bij de amerikanen. gredurende de aanschafperiode zijn procedures en controles op kwaliteit en veiligheid deels nagelaten.
	Dit veiligheidsgarantie werd vermeld in het koopcontract.
	Conclusie
	Koopcontract werd niet goed doorgelezen
	Geen controle op kwaliteit en veiligheid
	Geen controle op kwaliteit en veiligheid
	Zwakke plekken in het ontwerp
	Geen controle op kwaliteit en veiligheid
	opslag en gebruik in ongunstige condities
	
	De aanwezige medische voorzieningen waren nite volgends de nederlandse militaire richtlijnen
	Het ontbreek aan medische toetsing vanuit de defensie organisatie
	twijfels die werden geuit binnen de defensieorganisae vonden geen wrrklank
	Ok het ongeval tijdens de mortieroefening was voor defensie geen aanleuiding om de medische voorzienignen te evalueren.
	De inrichting van veilige medische zorg voor nederlandse militairen in kidal is ondergeschikt gemaakt aan de voortgang van de missie.
	
	
	\cite{ovvMortierOngevalMaliVideo} 
	\cite{bnnvara13062018malirapport}
	\cite{eucal11012021malimissieverlengd}
	\cite{nos21052014zorgenmalimissie}
	\cite{meijnders}
	\cite{bnrwebredactie}
	\cite{keultjes01062016malimissiecoalitie}
	\cite{veenhof18012019}
	
	\cite{isitman06012016militair}
	\cite{nporadio11072016filmdemissie}
	\cite{parlementairmonitor15122013mortierongeluk}
	sollicitatie
	de bureaucratie
	aankomst
	interview van de burgerbevolking
	steun van de bevolking minuut 15:00
	de organisatie minuut 23:00
	De militaire briefing minuut 34:00
	prioriteit minuut 39:00
	briefing minuut 40:00
	de communicatie met ministerie over inlichten minuut 44:00
	\cite{DemissieFilm}
	
	
	\item[Datum en plaats] 06/04/2016
	\item[Oorzaak]
	%Beschrijf wat er mis ging in termen van het vier variabelen model/requirements/specificaties
\end{description}

\paragraph{tjernobyl}

\begin{description}
	\item[Beschrijving]
	\item[Datum en plaats] 26/04/1986
	\item[Oorzaak]
	%Beschrijf wat er mis ging in termen van het vier variabelen model/requirements/specificaties
\end{description}
Een ramp bij een kernreacor in de sovjetunie. Door een bedieningsfout in een testprocedure werd het vermogen van de koelinstallaties negatief beinvloed. Door een ontwerpfout in de noodstopprocedure kon in het systeem niet snel genoeg schakelen om remmende invloed uit te oefenen op het toenemende vermogen van de reactorkernen. Met brand en eksplosie tot gevolg.

\cite{INSAVienna1992Chernobyl}
Tsjernobyl
\cite{wikiTjernobyl}
\cite{rivmTjernobyl}
\cite{andereTijdenTjernobyl}
wat er is gebeurd en hoe het leven verdergaat
\cite{kingskey19042022tjernobyl}
pernsioenfondsen en de tjernobyl ramp
In 2021 worden mensen nog steeds blootgesteld blijkt ut een gezamelijk onderzoek van greenpeace en oekraiense wetenschappers
stijging van de nucliaire activiteit gemeten in tjernobyl
Het toerisme  aspect
De chronologie
\cite{erikbork26042023reactor4}
\cite{nosTjernobyl30jaarlater}
Dieren in de omgeving van tjernobyl
De chronologie
Echtreme droogte zorgd voor gevaar
\cite{knmi04052021tjernobylbosbrand}
\cite{dodonovaKVIRisicoTjernobyl}
Joernalistiek, entertainment en de waarheid
\cite{dumarey04062020verhaalTjernobylWaarheid}
Een onderzoek
Huidige gevolgen van de explosie van toen
\cite{sparkesNewScientistTjernoby}
De ramp, hoe de mensen ermee omgingen en hoe er nu geleef wordt
evaluatieonderzoek en amatregeen
\cite{kernenergiened26041986chronologiemaatregelen}
\cite{mapszoneReactor}
Invloed van de mens op de omgeving
Heroplevende splijtingsreacties
docu van schooltv
Radioactiviteit bereikt nederland
documentaire en maatregelen
\cite{kernhistoriek15062021tjernobyl}
Het verhaal van een overledende
Toerisme
toerisme
toerisme
Dieren in de omgevong
Toevluchtsoord voor vluchtelingen van de oorlog met russische seperatisten
Ouderen die terugkeerden naar hun woonplaats na de gedwongen verhuizing door de autoriteiten
De straling neemt weer toe
Lessen geleerd van tjernobyl
\cite{nucleairforumFeitenTjernobyl}
Toerisme
Bosbrand in tjernobyl
invloed van de ramp op belgie
\cite{kernongevalTjernobylFancGov}
Boek recensie
Fotos en berekeningen
ontmanteling en toerisme
Belangrijke lessen en overeenkomsten
De journalistieke waarheid van de koude oorlog
De lessen van
\cite{arendswolters062019lessenTjernobyl}
Een toristenattractie maken van tjernobyl
De radioactieve straling toen en nu
de 30km zone door de ogen van toeristen
artikel
stedentrip
rapport
\cite{damveld08052020tjernobyl}
slapend monster
docu
krantenartikel
hbo serie
docuserie
de  nieuwe sacrofaag
hulp aan slachtoffers
slapende reactor
krantenartikel
\cite{deVriestjernobylHolland}
hbo serie
internationale gevolgen
toerisme
nieuwe koepel
media communicatie
docu
dieren
koepel
koepel
\cite{ing3enieur29042015antistralingskoepel}
toerisme
toeristisch reiperspectief
toerisme
niwe koepel
overschakelen naar duurzaamheid
docu
tjernobyl wekt nu duurazme energie
toerisme
overeenkomsten tjernobyl en fukushima
drank en sla uit tjernobyl
geen efficiente opslag is mogelijk
wetenschappelijke artikelen
zaterdag 26 april 1986. Er vind routineonderhoud plaats bij reactor 4, De controle wordt uitegevoerd door de dagploeg. Vnwege een test wordt jhet koelsysteem uitgeschakeld. Door omstandigheden wordt de test uitgesteld en wordt de verantwoordelijkheid overgedragen aan de avondploeg.
De operator maakt bedieningsfouten waardoot de reactor bijna stil komt te liggen. En vervolgens probeert hij de reactor weer op gang te brengen. ondanks de snelle temperatuurstijging wordt het experiment doorgezet. Dan wordt ook het veiligheidssysteem stilgelgd. Terwijl het koelwater langzaam opwarmt, sluit hij de klep waarlangs de stoom naar de generator stroomt.

De temperatuur van de reactorstaven neemt daarna snel toe. Terwijl er een oncontroleerbare kettingreactie op gang komt, laat het personeel in paniek de regelstaven zakken om de warmteontwikkeling af te remmen. Het is dan echter al te laat. Door een ontwerpfout loopt het vermogen razendsnel op tot 33.000 megawatt, ruim tien keer hoger dan normaal.

In een oogwenk verandert al het koelwater in stoom. De ontploffing die daarop volgt, blaast het 2000 ton zware deksel van de reactor af.

In de ravage vat het gloeiend hete grafiet in de reactor spontaan vlam. De uitslaande brand en een tweede explosie voeren een radioactieve rookwolk tot 8 kilometer hoogte. 
In een poging het vuur in reactor 4 te doven, storten helikopters vanuit de lucht zand, lood en boorzuur in de reactorkern. Het mag echter niet baten.

Intussen is de nucleaire brandstof zo heet geworden dat die door de bodem van het reactorvat dreigt te smelten. Als dat gebeurt, kan het bluswater onder het vat in één klap verdampen en dreigt een derde explosie die een groot deel van Europa onbewoonbaar zal maken. Om dit te voorkomen moet het water hoe dan ook worden weggepompt.

Drie brandweermannen wagen zich daarvoor in de ruimte onder de reactor, blootgesteld aan 300 sievert per uur, 300.000 keer de dosis die een Nederlander jaarlijks maximaal mag oplopen. Ze slagen daarin, maar twee van hen overlijden enkele dagen later aan acute stralingsziekte.

Hoewel geigertellers de dag na de ramp onrustbarende waarden aangeven, slaat het plaatselijk bestuur geen alarm. De bevolking is het niet gewend om vragen te stellen.

De volgende dag blijkt er wel degelijk iets ernstigs aan de hand te zijn. In een lange rij bussen worden de 135.000 inwoners op 27 april uit het besmette gebied geëvacueerd, om er nooit meer terug te keren.

De ramp is dan nog steeds geen wereldnieuws. De Sovjetautoriteiten blijken er niet eens van op de hoogte te zijn – president Gorbatsjov klaagt later dat hij via Zweden aan zijn informatie moest komen.
\cite{verschuur14012013tjernobylreports}
\cite{paperlessarchivesTjernobyl}
\cite{vargos082000tjernobylconcerns}
\cite{mauroNuclearRiskSociety}
\cite{vienna06092005LookingBack}



\paragraph{ Research case: De digitale aanval op de Oekrainese krachtcentrale}

\begin{description}
	\item[Beschrijving]
	\item[Datum en plaats] 23,december 2015
	\item[Oorzaak]
	%Beschrijf wat er mis ging in termen van het vier variabelen model/requirements/specificaties
\end{description}

Op 23,december 2015  vind er een cyber aanval plaats op het elektriciteitsnet van de Oekraine. Dit was de eerste bekende aanval op een elektrisch contole  system.  Dit verslag geeft inzage in een analyse van de Ukraine cyber aanval,
inclusief hoe de actoren zich zelf toegang gavan tot het controle systeem, welke methoden de acoren hebben gebruikt voor reconnaissance en vastleggen van het systeem, een gedetailleerde omshrijving van de aanval op 15 December 2015, en de methoden die gebruikt zijn door de aanvallers om hun sporen uit te wissen en daarmee het het stoppen van schade toebrengen  nog moeilker maken. Daarnaast wordter  een gedetailleerde omschrijving gevevenv an de beveiliging van de SCADA ccontrol systemen gebaeerd op bst practices, inclusief het control network ontwerp, technieken voor whtelisting, monitoring en loggen, en  opleiding van personeel.
\cite{Whitehead2017ukrainepoweroutage}
\cite{noauthor_2022-nm}
\cite{zetter2016GridHack}
\cite{owens21032017ukrainemitigationstrategies}
\cite{cerulus2019FrontlineRussiaAttack}
\cite{grammatikis2019AttackIEC6087505104}
\cite{hidajat2016ScadaSimulator}
\cite{uscert20072021crashmalware}
\cite{zetter12062017malwareanalysis}
\cite{icsRussianHackingCyberWeapon}
\cite{usgovC2M2}
Dit verslag geeft inzage in een analyse van de Ukraine cyber aanval,
inclusief hoe de actoren zich zelf toegang gavan tot het controle systeem, welke methoden de acoren hebben gebruikt voor reconnaissance en vastleggen van het systeem, een gedetailleerde omshrijving van de aanval op 15 December 2015, en de methoden die gebruikt zijn door de aanvallers om hun sporen uit te wissen en daarmee het het stoppen van schade toebrengen  nog moeilker maken. Daarnaast wordter  een gedetailleerde omschrijving gevevenv an de beveiliging van de SCADA ccontrol systemen gebaeerd op bst practices, inclusief het control network ontwerp, technieken voor whtelisting, monitoring en loggen, en  opleiding van personeel.
\cite{Whitehead2017ukrainepoweroutage},\cite{zetter2016GridHack},\cite{boozallen2016lightwentout},\cite{finklejan2016UsBlamesRussianSandworm},\cite{desarnaud2017cyberattacks},\cite{caseli04112016intrusiondetectioncontrolsystem},\cite{rochascadatesting},\cite{hidajat2016ScadaSimulator},\cite{zetter2017moreDangerousMalware}.
Oop 23,december 2015  vind er een cyber aanval plaats op het elektriciteitsnet van de Oekraine. Dit was de eerste bekende aanval op een elektrisch controle  system met corrupte firmware. Daarnaas wordt er een telecom-based denial of service attack met  geautomatieerde systemen om het telefoonverkeer uit te schakelen.
\cite{Whitehead2017ukrainepoweroutage}
Uit onderzoek\cite{zetter2016GridHack} naar de aanval,  uitgevoerd door Oekraiene sen Amerikaanse militairenblijkt  bleek onder meer dat de power grids in sommige gevallen beter waren beveiligd dan de Amerikaanse. Desondanks was de viligheid niet optimaal door onder andere de  hetgegeven dat werknemers op afstand konden inloggen en geen gebruik van 2-stapsverificatie.
\subparagraph{Literaire analyse}
\subparagraph{Motief}
Oekraine wijst naar de russen \cite{zetter2016GridHack}, 
\cite{greenberg2017Cyberwartestlab},
\cite{boozallen2016lightwentout},
\cite{finkle08012016russiansandwormhackers},
\cite{zinets15022017ukrainechargesrussia},
\cite{mcelfresh2016cyberattackhowandwhy},
\cite{parkwalstorm11102017russiagridattack}.
\subparagraph{Situatie Oekraiene}
\cite{drago2017CrashOverride},
\cite{slowik2019ReassasUkraine2016Attack}.
\subparagraph{Situatie algemeen}
\cite{cerulus2019FrontlineRussiaAttack},
\cite{desarnaud2017cyberattacks},
\cite{dragos2019TargetedTransStation}.
\subparagraph{Factoren}
\cite{shehod2016gridadvantageus}
\subparagraph{Oorzaak}
\cite{rocha2017cybersecyrityanalysisScada},
\cite{2017crashoverridenostuxnet},
\cite{vijayan2017firstmalwareCausedOutage},
\cite{slowik2019ReassasUkraine2016Attack}.
\subparagraph{Gebruikte materialen}
\cite{2015ukrainegridattack},
\cite{industroyershortfact}
\subparagraph{Uitvoering van de aanval}
\cite{Whitehead2017ukrainepoweroutage},
\cite{boozallen2016lightwentout}.
\subparagraph{Oplossingen}
~\cite{Whitehead2017ukrainepoweroutage}
\subparagraph{Aanbevelingen}
\subparagraph{Resultaten}
\subparagraph{De aanval}
De reconstructie van de aanval bestaat uit een zetal stappen:
1. An initial email spear phishing attack lures recipients
into opening an attached Microsoft® document with a
macro that installs Black Energy 3 (BE3) onto
corporate workstations.
2. BE3 and other tools perform reconnaissance and
enumeration of the network and provide an initial
backdoor for the hackers into the corporate network.
3. As a result of network reconnaissance, the malicious
actors discover and access the oblenergos’ Microsoft
Active Directory® servers that contain corporate user
accounts and credentials.
4. With the harvested credentials, the malicious actors use
an encrypted tunnel from an external network to get
inside the oblenergo network, establishing a presence
on the oblenergo control system networks.
5. Malicious actors discover and access the control center
supervisory control and data acquisition (SCADA)
human-machine interface (HMI) servers and
substations. While a router separates corporate and
SCADA networks, the firewall rules are improperly
configured.
6. On December 23, 2015, at 3:30 p.m., the malicious
actors begin their power outage attacks by entering
operations and SCADA networks through backdoors on
the compromised SCADA workstations. The malicious
actors take control away from HMI operators and then
open breakers.
7. The malicious actors perform several other actions with
the intent of complicating the responses of control
operators and increasing the effort required to return the
system to normal operating conditions. These actions
include:
a. Launching a coordinated Telephony Denial of
Service (TDoS) attack that floods call centers to
prevent legitimate calls from getting through.
b. Disabling the UPSs for the control centers.
c. Corrupting the firmware on a remote terminal unit
(RTU) HMI module and serial-to-Ethernet port
servers.
8. Malicious actors execute KillDisk malware in an
attempt to wipe out the control center HMIs and pivotpoint workstations.
\cite{Whitehead2017ukrainepoweroutage}
\cite{boozallen2016lightwentout}
\subparagraph{spearfishing}
\subparagraph{blackenergy}
\subparagraph{remote access capabilities}
\subparagraph{serial-to-ethernet communication devices}
\subparagraph{telephony denial of service attacks}
\subparagraph{oplossingen}
Identificeer alle risicos en schrijf een plan foor het managen van de risico's.
Implementeer  effecteve controle  om het riico te managen.
Creeer een diepgaand model dat ervoor zor dat er efectieve en efficiente security controls worden uitgevoerd.
Aangaande de gebeurtenissen in de oekraiene kunnen de volgende security controls worden opgenomen in het securitymodel: Initial access to enterprise network, pivot in interprise network, elevate priviliges, maintainance access, gain access to control system, attack, attack complication, destroy hard drives.
\cite{Whitehead2017ukrainepoweroutage}
\subparagraph{Discussie}
\subparagraph{Verder lezen}
\cite{shahzad2014ScadaProtocolsPollingScenario},
\cite{grammatikis2019AttackIEC6087505104},
\cite{2017win32industroyer},
\cite{yadav2020reviewScadaArchitecture},
\cite{arrizabalaga2020surveyiiotProtocols},\cite{fauri2017EncryptionICS},\cite{resch31102019IEC62351secureCommunication},\cite{levalle2020FuzzingICSProtocols},\cite{blackhatusa2017},\cite{blackhatusa2017},\cite{abb30062017crashoverridenotification},\cite{spinner2018crashoverrideiot},\cite{njccicthreat08102017crashovverrideprofile},\cite{slowikvb2018crashoverride},\cite{crashoverridenetwork},\cite{wikiindustroyer},\cite{icsSecurityRussianHacking},\cite{holappa2017threattoElectricityNetworks}.

 




\paragraph{AlgemeenDeelonerzoek naar veiligheidsrisico's voor sluizen}
Omdat we in deit onderzoek uitgaan van het uitbreiden van bestaande sluizen is er literatuurstudie gedaan naar sluizen. In de archieven van het ministerie van verkeer en waterstaat is er het rapport Design of waterlocks\cite{CivilEngineeringDivision}.
Het programma van requirements kunnen we in ons model niet helemaal overnemen. 
Zo zijn er precondities zaols topgrafie,bestaande watersluizen,waterlevel, wind, morphologie en bodemeigenschappen.
Dan zijn er nog de functionele eigenschppen


De gebruikerseigenschappen

De onderhoudseisen

Omgevingseisen

omgevingseien in de constructiefase

Constructievoorschriften

benadering van de sluis en sluiskolk

Dimensies van het watersluiscomplex


Intake en discharge systemen

Deuren. operationele mechanismen



Sluishoofd


Sluiskolk

\paragraph{Wet en regelgeving voor sluizen}

\paragraph{Ondeerzoeksresultaten naar sluisbeveiliging}



Verouderde computersystemen zijn door de jaren heen gekoppeld aan netwerken, zodat ze op afstand te besturen zijn. Dit zorgt ervoor dat systemen kwetsbaar zijn voor aanvallen van buitenaf. De beveiliging is in de loop der jaren niet voldoende ontwikkeld om de infrastructuur goed te beveiligen.

Volgens het onderzoek is er de afgelopen jaren wel het nodige geïnvesteerd om de beveiliging op te schroeven, maar deze maatregelen zijn nog onvoldoende doorgevoerd.
https://www.nu.nl/internet/5814282/rekenkamer-waterwerken-niet-goed-beveiligd-tegen-cyberaanvallen.html
\cite{hdsr30092022lichtprojectieswaterliniesluizen}
rapport Digitale dijkverzwaring: cybersecurity en vitale waterwerken 
Crisisdocumentatie is verouderd en er worden geen volwaardige pentesten uitgevoerd. Uit het onderzoek blijkt dat nog niet alle vitale waterwerken rechtstreeks zijn aangesloten op het Security Operations Center (SOC) van Rijkswaterstaat. Hierdoor bestaat het risico dat RWS een cyberaanval niet of te laat detecteert. De minister van Infrastructuur en Waterstaat moet nog stappen zetten om aan de eigen doelstellingen voor cybersecurity te voldoen
De Algemene Rekenkamer beveelt de minister van Infrastructuur en Waterstaat ook aan om het actuele dreigingsniveau te onderzoeken en te besluiten of extra mensen en middelen nodig zijn. Ook is het voor een snelle en adequate reactie op een crisissituatie van essentieel belang dat informatie up-to-date is. Pentesten zouden integraal onderdeel uit moeten maken van de cybersecuritymaatregelen bij vitale waterwerken. Verder zou moeten worden bezien of medewerkers van het SOC beter moeten worden gescreend.

\cite{kramerZeeland}
Sluis Eefde kreeg niet alleen de onderhoudsbeurt, maar werd tevens uitgebreid met een tweede sluiskolk. Zo wil Rijkswaterstaat wachttijden voor de scheepvaart voorko

\cite{gww29032021kantelendesluisdeur}
Om de lokale bemanning, die de oren en ogen waren van de sluizen, te vervangen waren camera’s, communicatielijnen en software nodig. Hoge kwaliteit videobeelden, met echte kleuren en zonder enige vertraging zijn belangrijk voor de operators en zij moeten hierop kunnen vertrouwen. Er zijn verschillende testen gedaan met diverse camera’s en cameraposities om kleurechtheid te kunnen bieden onder alle omstandigheden. Het resultaat was een perfecte kleur op alle 70+ camera’s op iedere locatie.

Vertraging van videobeelden was een cruciale factor in dit project. Het is uiterst belangrijk dat de operator op zijn beeld ziet wat er daadwerkelijk op locatie gebeurt, zonder enige vertraging. Om te laten zien of er eventuele vertraging is, is er een speciale functie gecreëerd. Deze functie laat een rood kruis zien op het scherm wanneer de vertraging meer is dan 500 miliseconden. Zo ziet de operator direct of het beeld wat hij ziet actueel is. 

Een andere functie die voor dit project is gecreëerd, is bij de videobeelden aan te geven van welke kant van de sluis het camerabeeld is. Voor de operators is het belangrijk dat ze weten vanaf welke kant het vaartuig komt en waar deze naartoe vaart. Een simpele oplossing was om een blauw kader te maken om het videobeeld van de ene kant van de sluis en geen kader om het videobeeld van de andere kant. 


\cite{thkwaterwerken}
Het crisismodel kan beter, is de derde deelconclusie van de Algemene Rekenkamer. Er is geen specifiek scenario voor een crisis die wordt veroorzaakt door een cyberaanval. Ook ontbreekt inzicht in de effecten van een cybercrisis op andere sectoren, de zogeheten cascade-effecten. Tevens is de crisisdocumentatie op onderdelen verouderd.

\cite{rekenkamercybersecWater}
Ook maakt cyberveiligheid nog geen volwaardig onderdeel uit van reguliere inspecties.’ De Rekenkamer hamert erop dat alle vitale waterinfrastructuur zo snel mogelijk op het SOC wordt aangesloten. Ook zouden werknemers van Rijkswaterstaat die belangrijke waterkeringen bedienen beter gescreend moeten worden op hun antecedenten. Sollicitanten hoeven nu slechts een Verklaring Omtrent Gedrag te overleggen, maar dat is een heel lichte toets.

\cite{hackerWaterwerk}
deltawerken

\cite{kramerZeeland}
Volgens Rijkswaterstaat is het kostbaar en technisch uitdagend om klassieke automatiseringssystemen te moderniseren en wordt er daarom vooral ingezet op detectie van aanvallen en een adequate reactie daarop.
Uit het onderzoek blijkt dat Rijkswaterstaat de afgelopen jaren zelf van alle tunnels, bruggen, sluizen et cetera heeft vastgesteld welke cyberveiligheidsmaatregelen moeten worden genomen. Een groot deel van die maatregelen (ongeveer 60\%) was begin 2018 ook al uitgevoerd, maar Rijkswaterstaat ziet onvoldoende toe op de uitvoering van het resterend deel en heeft geen actueel overzicht van de overgebleven maatregelen.
De minister heeft een aantal waterwerken die Rijkswaterstaat beheert als vitaal aangewezen. . Uit het onderzoek blijkt dat nog niet alle vitale waterwerken rechtstreeks zijn aangesloten op het Security Operations Center (SOC) van Rijkswaterstaat. De ambitie om eind 2017 bij alle vitale waterwerken cyberaanvallen direct te kunnen detecteren was in het najaar van 2018 daarmee nog niet gerealiseerd. Hierdoor bestaat het risico dat RWS een cyberaanval niet of te laat detecteert.

\cite{cybersecWaterwerk}
Over de cyberbeveiliging van gemeenten en waterschappen wordt al langer geklaagd. Zo meldde EenVandaag al in 2012 dat rioolgemalen en sluizen gemakkelijk van afstand te bedienen waren, onder meer door bijzonder slechte wachtwoorden.

\cite{cybersecWaterschappen}
Rittal doet onderzoek naarop afstand besdienbare sluizen

\cite{cybersecZuidHolland}
Beveiligde VPN
M2M Services levert aan inmiddels 220 gemeenten en waterschappen beveiligde connectiviteitsoplossingen voor het beheer van pompen, riolen en gemalen. Om risico’s op beveiligingsincidenten te voorkomen maken wij gebruik van een VPN oplossing, waarbij de verbinding optimaal beveiligd is middels encryptie en authenticatie.

\cite{waterwerkNED}
Veiligheid op het water én op het land
Gebruik van lampbewaking 

\cite{veiligheidwaterland} 



\paragraph{ethiek}


Ethiek 



persuasive technology 
https://www.humanetech.com/youth/persuasive-technology 
\cite{humanTechpersuasiveTech}
https://www.minddistrict.com/blog/persuasive-technology-new-insights-in-behavioural-change 
https://www.sciencedirect.com/book/9781558606432/persuasive-technology 
https://spectrum.ieee.org/how-persuasive-technology-can-change-your-habits 
\cite{rezenfeld01012018persuasiveTecgHabits}
https://www.frontiersin.org/articles/10.3389/frai.2020.00007/full 
\cite{aldenaini28042020persuasiveTechTrends}
https://psmag.com/environment/captology-fogg-invisible-manipulative-power-persuasive-technology-81301 
\cite{larson14062017persuasivetechmanipulates}
https://www.makeuseof.com/what-is-persuasive-technology/ 
\cite{tanzem22012022persuasivetechchanginglives}
https://lib.ugent.be/catalog/rug01:001235489 
https://cyberpsychology.eu/article/view/12270 
\cite{tikkakuddonenpersuasiveTechnology}





\paragraph{Analyse}
\paragraph{Conclusie}

%%%%%%%%%%%%%%%%%%%%%%%%%%%%%%%%%%%%%%%%%%%%%%%%%%%%%%%%%%%%%%%%%
 
\hoofdstuk{Uppaal model}


\paragraph{Inleiding}




De meeste sluizen die zich in Nederland bevinden zijn schutsluizen; deze sluizen zijn bedoeld om boten, zowel vrachtschepen als pleziervaart afhangend van de locatie van de sluis, te verwerken. Om deze reden gaan wij deze dus ook verwerken in ons model. Mocht een sluis niet bedoeld zijn om boten te verwerken, dan zou dit model alsnog toegepast kunnen worden opp desbetreffende sluis.
Boten worden toegevoed aan de queue. Hoe dit gebeurt, dat ligt aan de specifieke sluis.  Sinds wij een template maken, hoeven wij geen rekening te hounden met hoe de schepen in de queue komen. Het enige wat wij hoeven te doen, is de data verwerken.





\paragraph{Aandachtspunten}
\begin{enumerate}
	\item Voorrang tussen schepen onderling in de sluis?
	\item Hoe lang mag een schip zich in de sluis bevinden?
\end{enumerate} 




\subparagraph{Afbakening}
\begin{itemize}
	\item Wat doet de sluis niet.
	\item De sluiss houdt geen rekening met links of rechtsrijdend verkeer vanuit de zeevaart
	\item De sluis heeft geen queue met daarin een id gekoppeld aan de sluis.
	\item De waterpomp wordt alleen aan en uitgezet
	\item De waterpomp houdt geen rekening met waterstand
	\item Houdt geen rekening met een schip in de sluis dat is blijven hangen.
\end{itemize}


\begin{enumerate}
	\item Een tweetal sluisdeuren. 
	\item Een sluiskolk waarin de schepen in- enuitvaren
	\item een stoplicht om een signaal af te geven voor invaren en uitvaren.
	\item Een nivelleermachine zorgt ervoor dat het water in de sluis op het gewenste niveau wordt gebracht
	\item Een control-system dat ervoor zorgt dat de opdrachten van de sluisbeheerder (geautomatiseerd) worden uitgevoerd
\end{enumerate}

Een schip komt aanvaren en meld zich aan bij de sluismeester. De sluismeester geeft een signaal aan het controlsystem voor het openen van de sluisdeuren, nadat geccontroleerd is of de nivelleermachine al klaar is. Als er ruimte is voor een invarend schip mag het schip dat zoich heeft aangemeld en toestemming heeft  in de sluis varen. Op het moment dat de sluis vol is gaan de sluisdeuren dicht. Eenmaal afgesloten kan de nivelleermachine beginnen om het water in de sluiskolk op het gewenste waterpeil te brengen. Als dit nivelleerprces is afgerond geeft  het controlsystem daan da de sleusdeuren open kunnen.  Als de sleusdeuren open zijn en het uitvaarsignaal is op groen dan moet het schip in de sluis de sluis uitvaren.

Uit het zojuist genoemnde scenario valt het volgende op te maken.
\begin{enumerate}
	\item Een schip geeft een signaal aan een sluismeester.
	\item Er wordt gekeken of er wel plek is in de sluis .
	\item Er wordt gekeken of de nivelleermachine is afgerond.
	\item Er wordt gekeken wat het niveo van de waterpeil in de sluiskolk is.
	\item Er wordt gekeken of de sluisdeuren gereed zijn voor invarende schepen.
\end{enumerate}



	\subsubsection{De computation tree}

\xymatrix@ur@!R=2pc{%
	*+<1pc>[o][F-]{q_0}  \ar@(l,l)[]^<<<<{start} \ar@/^/[r]^0  \ar@/_/[d]_1 
	& *+<1pc>[o][F-]{q_1} \ar@(ul,ur)[]^{0}  \ar@/^/[d]^1 \\
	*+<1pc>[o][F-]{q_2} \ar@(dr,dl)[]^{1} \ar@/_/[r]_0 
	& *+<1pc>[o][F=]{q_3} \ar@(l,l)[]^>>>>{start}  \ar@(dr,dl)[]^{1} \\

 }








\begin{tikzpicture}[>=latex',scale=0.5]
	% set node style
	
	\begin{dot2tex}[dot,tikz,codeonly,styleonly,options=-s -tmath]
		digraph G  {
			node [style="n"];
			p [label="+"];
			t [texlbl="\LaTeX"];
			6
			8
			10-> p;
			6 -> t;
			8 -> t;
			t -> p;
			{rank=same; 10;6;8}
		}
	\end{dot2tex}
	\begin{pgfonlayer}{background}
		\draw[rounded corners,fill=blue!20] (6.north west) -- (8.north east) -- (t.south east)--cycle;
	\end{pgfonlayer}
\end{tikzpicture}


\[\begin{tikzcd}[column sep=1cm]
	ABCDE\arrow[r, leftrightarrow, "\times"{anchor=center},"\text{label}","\text{label}"{below}]\arrow[d] & F\arrow[r]\arrow[d] & G\arrow[rr]\arrow[d] && H\arrow[d]\\
	ABCDEFGH\arrow[r, leftrightarrow, "\times"{anchor=center}]\arrow[d] & II\arrow[r]\arrow[d] & JJ\arrow[rr,"\text{very long label}"]\arrow[d] && KK\arrow[d]\\
	ABCD\arrow[r] & EEE\arrow[r] & FFF\arrow[rr] && GGG
\end{tikzcd}\]




\paragraph{Models}

\subparagraph{Maincontroller}
\[
\begin{tikzcd}%[every arrow/.append style=dash]  uncomment to remev arrowa
	& \tikz{\node[draw,circle]{1}} \ar{d}&  & \tikz{\node[draw,circle]{2}} \ar{d}  \ar{r} &  \ar{r} & \ar{r} & \ar{r} &  \ar{r}& \tikz{\node[draw,circle]{2}}\ar{d} \\
	\tikz{\node[draw,circle]{4}} \ar{d} & \tikz{\node[draw,circle]{2}}  \ar[bend right=15]{l} \ar{d} & & \ar{d} & \tikz{\node[draw,circle]{2}} &\tikz{\node[draw,circle]{2}}&\tikz{\node[draw,circle]{2}}&& \tikz{\node[draw,circle]{2}} \ar{d} &\\
	\tikz{\node[draw,circle]{4}} \ar{u} \ar{r}  & \tikz{\node[draw,circle]{3}} \ar{r}  & \tikz{\node[draw,circle]{5}} \ar{r} &  \tikz{\node[draw,circle]{6}} \ar{r}  \ar{ru} & \tikz{\node[draw,circle]{7}} \ar{r} & \tikz{\node[draw,circle]{7}} \ar[crossing over]{ul}  \ar{r} & \tikz{\node[draw,circle]{8}} \ar{r} & \tikz{\node[draw,circle]{9}} \ar{r} \ar{d} & \tikz{\node[draw,circle]{10}}  \\
	& & & &\tikz{\node[draw,circle]{1}} \ar[crossing over]{ul} \ar{ru}  & & \tikz{\node[draw,circle]{2}}  \ar[bend right=15]{r} & \tikz{\node[draw,circle]{2}} \ar[bend right=15]{l} \ar[bend right=15]{r}  & \tikz{\node[draw,circle]{2}} \ar[bend right=15]{l}
\end{tikzcd}
\]
 

\begin{tikzpicture}[>=latex',shorten >=1pt,node distance=2cm,on grid,auto,scale=0.2]

	\node[state] (q0-e) {$q_0/\epsilon$};
\node[state] (q0-1) [below right=of q0-e] {$q_0'/1$};
\node[state] (q1-0) [above right=of q0-1] {$q_1/0$};
\node[state] (q2-1) [below right=of q1-0] {$q_2/1$};

\node[state] (q3-0) [below left=of q0-1] {$q_3/0$};
\node[state,accepting] (q3-1) [below right=of q0-1] {$q_3'/1$};
\node[state] (0) [ left=of q3-0] {$q_3/0$};
\node[state] (1) [ left=of 0] {$q_3/0$};

\node[state,initial,accepting] (0) [ left=of 1] {$q_3/0$};

\node[state] (3) [ below right=of q2-1] {$2$};
\node[state] (4) [  right=of 3] {$4$};
\node[state] (5) [  right=of 4] {$5$};
\node[state] (6) [  right=of 5] {$6$};

\node[state] (7) [  above=of 1] {$7$};
\node[state] (8) [  above=of 7] {$8$};
\node[state] (9) [  right=of 5] {$9$};

\node[state] (10) [ below  left=of 4] {$10$};
\node[state] (11) [ below  right=of 4] {$11$};

\node[state] (12) [   above=of 5] {$11$};

\node[state] (13) [   above=of 12] {$11$};

\node[state] (14) [ below right  =of q3-0] {$11$};
\node[state] (15) [   below right =of 14] {$15$};
\node[state] (16) [   above right =of 15] {$16$};


\path[->] (q0-e) edge node {a} (q1-0);
\path[->] (q0-e) edge node {b} (q3-0);
\path[->] (q0-1) edge node {a} (q1-0);
\path[->] (q0-1) edge [bend right] node {b} (q3-0);
\path[->] (q1-0) edge node {a} (q3-1);
\path[->] (q1-0) edge node {b} (q2-1);
\path[->] (q2-1) edge node {a} (q0-1);
\path[->] (q2-1) edge node {b} (q3-1);
\path[->] (q3-0) edge node {a} (q3-1);
\path[->] (q3-0) edge [bend right] node {b} (q0-1);
\path[->] (q3-1) edge [loop below] node {a} (q3-1);
\path[->] (q3-1) edge node {b} (q0-1);

\path[->] (4) edge [bend right] node {b} (10);	
\path[->] (10) edge [bend right] node {b} (4);


	
\path[->] (4) edge [bend right] node {b} (11);	
\path[->] (11) edge [bend right] node {b} (4);
	

\end{tikzpicture}

 
\begin{tikzpicture}[shorten >=0.5pt,node distance=2cm,on grid,auto]
	\node[state,initial] (0) {$q_0$};
	\node[state] (1) [right=of 0] {$q_1$};
	\node[state] (2) [right=of 1] {$q_2$};
	\node[state] (3) [above=of 2] {$q_3$};
	\node[state] (4) [right=of 3] {$q_4$};
	\node[state] (5) [right=of 2] {$q_5$};
	\node[state] (6) [below=of 2] {$q_6$};
	\node[state] (7) [right=of 6] {$q_7$};
	\node[state] (8) [right=of 5] {$q_8$};
	\node[state] (9) [right=of 8] {$q_9$};
	\node[state] (10) [right=of 9] {$q_10$};
	\node[state] (11) [right=of 10] {$q_11$};
	\node[state] (12) [right=of 11] {$q_12$};
	\node[state] (13) [above=of 11] {$q_13$};
	\node[state] (14) [above=of 12] {$q_14$};
	\node[state] (15) [right=of 4] {$q_15$};
	 
	\node[state] (16) [right=of 15] {$q_16$};
	\node[state] (17) [right=of 16] {$q_17$};
	\node[state] (18) [right=of 17] {$q_18$};
	\node[state] (19) [right=of 7] {$q_19$};
	\node[state] (20) [right=of 19] {$q_20$};
	\node[state] (21) [right=of 20] {$q_21$};
	\node[state] (22) [right=of 21] {$q_22$};
	\node[state] (23) [right=of 22] {$q_23$};
	\node[state] (24) [right=of 23] {$q_24$};
	\node[state] (25) [above=of 1] {$q_25$};
	\node[state] (26) [above=of 25] {$q_26$};
	\node[state] (27) [right=of 1] {$q_27$};
	\node[state] (28) [right=of 1] {$q_28$};
	\node[state] (29) [right=of 1] {$q_29$};
	\node[state] (30) [right=of 1] {$q_30$};
	\node[state] (31) [right=of 1] {$q_31$};
	\node[state] (32) [right=of 1] {$q_32$};
	\path[->]
	(0) edge                    node {$B\:B\:R$} (1)
	(1) edge [loop above]       node {$0\:0\:R$} (1)
	(1) edge [loop below]       node {$1\:1\:R$} (1)
	(1) edge                    node {$E\:E\:L$} (2)
	%(2) edge [loop below]       node {$*\:*\:L$} (2)
	(2) edge                    node {$1\:*\:R$} (3)
	(2) edge                    node {$0\:*\:R$} (6)
	(3) edge [loop above]       node {$*\:*\:R$} (3)
	(3) edge                    node {$E\:E\:R$} (4)
	(4) edge [loop above]       node {$0\:0\:R$} (4)
	(4) edge [loop right]       node {} (4)
	(4) edge                    node {} (5)
	(6) edge [loop below]       node {$*\:*\:R$} (6)
	(6) edge                    node {$E\:E\:R$} (7)
	(7) edge [loop below]       node {$0\:0\:R$} (7)
	(7) edge [loop right]       node {$1\:1\:R$} (7)
	(7) edge                    node {$\Box\:0\:L$} (5)
	%(5) edge [loop right]       node {$0\:0\:L$} (5)
	(5) edge [loop right]       node {$1\:1\:L$} (5)
	(5) edge                    node {$E\:E\:L$} (2);
	(5) edge                    node {$E\:E\:L$} (15);
\end{tikzpicture}

\subparagraph{Labeling functions}


\[\begin{tikzcd}[column sep=1cm]
	ABCDE\arrow[r, leftrightarrow,] & F\arrow[r] & G\arrow[rr]&& H
\end{tikzcd}\]

\subparagraph{Schip}
\[
\begin{tikzcd}
	\tikz{\node[draw,circle]{1}} \ar{r} & \tikz{\node[draw,circle]{2}} \ar{d}\\
	\tikz{\node[draw,circle]{4}} \ar{u} \ar{ur} \ar[bend right=15]{r} & \tikz{\node[draw,circle]{3}} \ar[crossing over]{ul} \ar[bend right=15]{l}
\end{tikzcd}
\]
\subparagraph{Deur}
\[
\begin{tikzcd}
	\tikz{\node[draw,circle]{1}} \ar{r} & \tikz{\node[draw,circle]{2}} \ar{d}\\
	\tikz{\node[draw,circle]{4}} \ar{u} \ar{ur} \ar[bend right=15]{r} & \tikz{\node[draw,circle]{3}} \ar[crossing over]{ul} \ar[bend right=15]{l}
\end{tikzcd}
\]
\subparagraph{Stoplicht}
\[
\begin{tikzcd}
	\tikz{\node[draw,circle]{1}} \ar{r} & \tikz{\node[draw,circle]{2}} \ar{d}\\
	\tikz{\node[draw,circle]{4}} \ar{u} \ar{ur} \ar[bend right=15]{r} & \tikz{\node[draw,circle]{3}} \ar[crossing over]{ul} \ar[bend right=15]{l}
\end{tikzcd}
\]
\subparagraph{pomp}
\[
\begin{tikzcd}
	\tikz{\node[draw,circle]{1}} \ar{r} & \tikz{\node[draw,circle]{2}} \ar{d}\\
	\tikz{\node[draw,circle]{4}} \ar{u} \ar{ur} \ar[bend right=15]{r} & \tikz{\node[draw,circle]{3}} \ar[crossing over]{ul} \ar[bend right=15]{l}
\end{tikzcd}
\]
\subparagraph{Sensor}
\[
\begin{tikzcd}
	\tikz{\node[draw,circle]{1}} \ar{r} & \tikz{\node[draw,circle]{2}} \ar{d}\\
	\tikz{\node[draw,circle]{4}} \ar{u} \ar{ur} \ar[bend right=15]{r} & \tikz{\node[draw,circle]{3}} \ar[crossing over]{ul} \ar[bend right=15]{l}
\end{tikzcd}
\]
\hoofdstuk{Verificatie}
 We moeten aantonen dat een real-time programma voldoet aan de eisen opgesteld en gespecificeerd. De meest gebruikte methode voor het bewij
 
 zen van de correctheid van untimed programma's zijn aangepast voor timed programs.  We hebben nog geen aanpask gevonden voor het gebruik en bewijzen van correct gebruik van clocks.  Een bewijs voor het gebruik van real-time programmas met clocks is gegeven in T.A. Henzinger and P.W. Kopke. Verification methods for the di-
 vergent runs of clock systems
 
 In dit hoofdstuk formaliseren we de requirements ogegeven in de requiremenstlis tin hoofdstuk .. en bewijzen we de correcte toepassing met gebruik van de symbolic model-checker van Uppaal.
 Het systeem is gemodelleerd als een netwerk van meerdere timed automata: controller, sluis, stoplicht, deur, pomp en schip.
 
 Het bewijs vn corret gebruik kan ook worden aangetoond met help van bewijs voor inorrectgebruik
 
 
 Verificatie resultaten
 \paragraph{Het door ons uitgetippelde testpath of scenario}
 
 \paragraph{Timed automata}
 
 
\paragraph{Data variabelen}

\paragraph{Acties}
 
\paragraph{Clock regions}
\cite{clarke2000Modelchecking21}
\cite{clarke2000Modelchecking212}
\cite{clarke2000Modelchecking223}
\cite{clarke2000Modelchecking31}
\cite{clarke2000Modelchecking32}
\cite{clarke2000Modelchecking33}
\cite{clarke2000Modelchecking411}
\cite{clarke2000Modelchecking43}
\cite{clarke2000Modelchecking63}
\cite{clarke2000Modelchecking64}
\cite{clarke2000Modelchecking661}
\cite{clarke2000Modelchecking91}
\cite{clarke2000Modelchecking102}
\cite{clarke2000Modelchecking11}
\cite{clarke2000Modelchecking122}
\cite{clarke2000Modelchecking123}
\cite{clarke2000Modelchecking132}
\cite{clarke2000Modelchecking1321}
\cite{clarke2000Modelchecking152}
\cite{clarke2000Modelchecking171}
\cite{clarke2000Modelchecking172}
\cite{clarke2000Modelchecking173}
\cite{audioSemanticsBengtsson}
\cite{guidingAutomataBberm}
\cite{gearTransitionLindahl1}
\cite{gearTransitionLindahl2}
\cite{martinelliScada}
\cite{IgbalReconstructurintTransition1}
\cite{IgbalReconstructurintTransition2}
\cite{huangVerficationStoch}
\cite{bengtssonUppaalVerification}
\cite{pranaliVerificationWaterLevel}
\cite{alexandreUppaalDefinition}
\cite{behzadEvalQOS}
\cite{behzadVariablesQoS}
\cite{alur}
\cite{alurDenseRealTime}
\cite{alurSystemClok}
\cite{alurModelHybrid}
\cite{rijksoverheidSluizen}
\cite{rijksoverheidSluisStroomschema}

\paragraph{CTL logica}
Alle veiligheid en reachability requirements formeel gespecificeerd in hoofdstuk ... zijn geverifieerd in uppaal met gebruik an A en E state formulae. Deze zijn als volgt:
\newline \\
M, s $\models$ p $\Leftrightarrow$ p $\in$ L(s) \\
M, s $\models$ $\not$ f1 $\Leftrightarrow$ M, s $\nvdash$ f1 \\
M, s $\models$ f1 $\vee$ f2 $\Leftrightarrow$ M,s $\models$ f1 or M,s $\nvdash$ f2 \\
M, s $\models$ f1 $\wedge$ f2 $\Leftrightarrow$  M,s $\models$ f1 and M,s $\nvdash$ f2 \\
M, s $\models$ $\mathrm{E}$ $g_{1}$ $\Leftrightarrow$ there is a path $\pi$  from ~  s ~   such ~  that  ~ M, $\pi$ $\models$ g1 \\
M, s $\models$ p $\Leftrightarrow$ for every path $\pi$  ~ starting from  ~  s, M, $\pi$ $\models$ g1 \\
M, s $\models$ p $\Leftrightarrow$ s is the first state of $\piand$ M, s $\models$ f1 \\
M, s $\models$ $\not$ $g_{1}$ $\Leftrightarrow$ M, $\pi$  $\nvdash$ g1\\
M, s $\models$ p $\Leftrightarrow$  M, $\pi$  $\models$ g1  or  M, $\pi$  M, $\pi$  $\models$ g2\\
M, s $\models$ p $\Leftrightarrow$ M, $\pi$  $\models$ g1  and  M, $\pi$  M, $\pi$  $\models$ g2 \\
M, s $\models$ p $\Leftrightarrow$ M, $\pi^{1}$ $\models$ g1 \\
M, s $\models$ p $\Leftrightarrow$ there exists a k $\ge$ 0, such that  ~ M, $\pi^{k}$  $\models$ g1\\
M, s $\models$ p $\Leftrightarrow$ for all i $\ge$ 0,M,$\pi^{i}$ $\models$ g1 \\
M, s $\models$ g1 $\bugcup$ g2 $\Leftrightarrow$ ~  there  ~ exists  ~ ak  ~ $\ge$  ~ 0 ~  such ~  that  ~ M,  ~ $\pi^{k}$ $\models$ g2\\
and  ~ for  ~ all ~  0  ~ $\le$ j < k, M,$\pi^{j}$ $\models$ g1
M, s $\models$ p $\Leftrightarrow$ for all j $\ge$ 0, if for ~  every  ~ i < j,M,$\pi^{i}$ $\nvdash$ g1 then M,$\pi^{j}$ $\models$ g2\\


%%%%%%%%%%%%%%%%%%%%%%%%%%%%%%%%%%%%%%%%%%%%%%%%%%%%%%%%%%%%%%%%%


 

\hoofdstuk{Conclusie}

Wat hebben alle bovenstaande rampen/ongelukken gemeen? Veiligheid.
Bij de therac waren er diverse problemen: communicatie, doorontwikkeling, controle en toetsing
Was het makkelijk te onderzoeken? Waarom?
Bij de boeing 737 crashes was het probleem van controle en communicatie naar medewerkers
Was het makkelijk te onderzoeken? Waarom?

Uit de evaluatie van de china explosion 2015 tianjin komt naar voren dat communicatie, transparantie en veiligheid niet altijd prioriteit hadden bij de lokale autoriteiten
Was het makkelijk te onderzoeken? Waarom?

Bij de tesla autopilot crashes komen soms onvoldoende onderbouwde ontwerpkeuzes naar voren die niet goed zij  afgewogen tegenover het gedrag van de bestuurder
vlucht 1951
Was het makkelijk te onderzoeken? Waarom?

De ramp in Tsjernobyl toont aan hoe autoriteiten een ramp in de doofpot proberen te stoppen
Was het makkelijk te onderzoeken? Waarom?



Wat heb ik geleerd
Ik heb erg veel geleerd van het veilig opzetten van VPN’s. Een VPN opzettenhad ik namelijk nog nooit gedaan. Het opzetten van SSH en het aanmaken vanVM’s was al bekend. Ook had ik nog nooit met UDP sockets geprogrammeerd.Verder heb ik geleerd hoe ik in de praktijk een VM in een VLAN kan zetten enhoe VLAN’s netwerken van elkaar kunnen scheiden.Het leukste onderdeel van het project, was dat wonderbaarlijk mijn gekozenoplossing elegant werkte. UDP Servers en clients zijn gerealiseerd met minderdan enkele regels logisch scipt. Ik had aan genomen dat het werken met socketsin shell absoluut rampzalig zou uitpakken. Ik ben blij dat het opdracht zo vrijwas, zodat ik experimenteel kon zijn met mijn implementatie.



