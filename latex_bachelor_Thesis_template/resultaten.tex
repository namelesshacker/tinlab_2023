
 



\hoofdstuk{Theoretisch kader}

In het eerste hoofdstuk is duidelijk geworden wat de onderzoeksvraag is, namelijk ‘Hoe kan een geautomatiseerde sluis worden gemodeleerd met oog op ontwikkel- en onderhoudskosten,veiligheid, efficientie en capaciteit’. Door de toenemende complexiteit van systemen is het gebruik van modellen en de toepassing van timebased model checking  op industriele controle systemen een manier van modelleren van het systeem en de requirements zodat er een bijdagre kan worden geleverd aan de acceptatie van  simulatie-/modeltechniek voor de industrie.(‘https://link.springer.com/article/10.1007/s10626-020-00314-0’, 2020). Of dit ook het geval is bij het modellereren van sluizen is nu de vraag.

De bestudering van rampen aan de hand van het vier-variabelen model biedt maakt het analyseren mogelijk van rampsituaties. Van een aantal rampen is een beschrijving gegeven met datum, plaats en oorzaak. De analyse van de 4-variabelen modellen zal gebruikt worden voor de requirementsdefinitie, ontwerp en ontwikkeling van het sluismodel. 

De verschillende factoren en achtergronden die  samenhangen met het modelleren van een sluis zullen in dit hoofdstuk toegelicht worden. Bovendien worden er hypotheses gevormd die de basis vormen voor debeantwoording van de onderzoeksvraag. 




\paragraph{Wat is uppaal}

Wat is Uppaal
Uppaal is an integrated tool environment for modeling, simulation and verification of real-time systems, developed jointly by Basic Research in Computer Science at Aalborg University in Denmark and the Department of Information Technology at Uppsala University in Sweden. It is appropriate for systems that can be modeled as a collection of non-deterministic processes with finite control structure and real-valued clocks, communicating through channels or shared variables [WPD94, LPW97b]. Typical application areas include real-time controllers and communication protocols in particular, those where timing aspects are critical.


model checking

Wat is statistical model checking?
Dit verwijst naar verschillende technieken dfie worden gebruikt voor de monitoring van een systeem. Daarbij wordt vooral gelet op een specifieke eigenschap. Met de resultaten van de statsitieken wordt de juistheid van een ontwerp beoordeeld. Statistisch model checking wordt onder andere toegepast in systeembiologie, software engineering en industriele toepassingen.
https://www-verimag.imag.fr/Statistical-Model-Checking-814.html?lang=en#:~:text=Statistical%20Model%20Checking%20(SMC)%20is,from%20state%20space%20explosion%20issues.


\cite{inriaStatsMoodCheck}
\cite{ buddeModelChecker}
\cite{AGHASuervey }


Waarom gebruiken we statistisch model checking?
To overcome the above difficulties we propose to work with Statistical Model Checking [KZHHJ09,You05,You06,SVA04,SVA05,SVA05b] an approach that has recently been proposed as an alternative to avoid an exhaustive exploration of the state-space of the model. The core idea of the approach is to conduct some simulations of the system, monitor them, and then use results from the statistic area (including sequential hypothesis testing or Monte Carlo simulation) in order to decide whether the system satisfies the property or not with some degree of confidence. By nature, SMC is a compromise between testing and classical model checking techniques. Simulation-based methods are known to be far less memory and time intensive than exhaustive ones, and are oftentimes the only option. 
https://project.inria.fr/plasma-lab/statistical-model-checking/

Alternatief
Alternatieven voor Uppaal zijn Asynchronous Events,Vesta en MRMC.


\paragraph{MODE CONFUSION }
Mode confusion tredd op als gepbserveerd gedrag van een technisch systeem niet past in het gedragspatroon dat de gebruiker in zijn beeldvorming heeft  en ook niet met voorstellingsvermogen kan bevatten.
\paragraph{Wat is automatiseringsparadox}
Gemak dient de mens. Als er veel energie wordt gestoken in de ontwikkeling van hulmiddelen die taken van werknemers overemen heeft dat tot resultaat dat veel productieprocessen worden geautomatiseerd. De vraag is dan of vanuit mechnisch wereldpunt de robot niet de rol van de mens overneemt en of de mens nog de kwaliteiten heeft om het werk zelf te doen.
\cite{bicker21102016automatiseringsparadox }
\cite{vseautoparadox }
\cite{blogxot21112016slimapparaat }



\paragraph{4 variabelen model}





Het 4 variabelen model kort toegelicht
Monitored variabelen: door sensoren gekwantificeerde fenomenen uit de omgeving, bijv temperatuur

Controlled variabelen: door actuatoren \bestuurde fenomenen uit de omgeving
For example, monitored variables might be the pressure and temperature
inside a nuclear reactor while controlled variables might be visual and audible alarms, as well
as the trip signal that initiates a reactor shutdown; whenever the temperature or pressure reach
abnormal values, the alarms go off and the shutdown procedure is initiated

Input variabelen: data die de software als input gebruikt
Here, IN models the input hardware interface (sensors and analog-to-digital converters) and
relates values of monitored variables to values of input variables in the software. The input variables model the information about the environment that is available to the software. For example,
IN might model a pressure sensor that converts temperature values to analog voltages; these voltages are then converted via an A/D converter to integer values stored in a register accesible to the
software.

Output variabelen: data die de software levert als output
The output hardware interface (digital-to-analog converters and actuators) is modelled
by OUT, which relates values of the output variables of the software to values of controlled variables. An output variable might be, for instance, a boolean variable set by the software with the
understanding that the value true indicates that a reactor shutdown should occur and the value
false indicates the opposite




\paragraph{World and machine samenvatting}
Waarom zijn wij engineers? Omdat we bruikbare apparaten willen laten functioneren in de wereld waarin we leven. Dat doen we door de machine te beschrijven en deze beschrijving van instructies bieden we aan onze computer opdat deze als de attribuut en gedragingen uitleest zoals wij die hebben omschreven. Dit alles op basis van theoretische funderingen en praktisch inzicht. 

Het doel van een machine is om te worden geinstalleerd en te worden gebruikt. De eisen die we stellen zitten in de omgeving en in de wereld en de machine is slechts de oplossing die we bedenken om aan een eis te voldoen. 

De relatie machine-wereld world gecategoriseerd in: 
Het modelleer aspect: waar een machine de wereld simuleert 

Het interface aspect: waar er fysieke interactie is tussen de machine en de wereld 

Het engineering aspect: waar de machine zich gedraagt als een controlemotor gebruikmakend van de gedragingen van de omgeving in de wereld 

Het probleem aspect: waar de omgeving in de wereld en de omvang van het probleem invloed heeft op de machine en de oplossing 

Het modelleer  of simulatie aspect over een deel van de wereld. Er zijn data,object en proces modellen. Het doel van een model is toegang te geven tot informatie over die wereld. Door het opvangen van statische weergaven en gebeurtenissen kunnen wij deze gebruiken van opgeslagen informatie die we kunnen hergebruiken. Een model kan bruikbare informatie bevatten omdat zowel het model als de wereld warin het model zich bevind gemeenschappelijke omschrijvingen hebben die waar zijn voor zwel het model als voor de wereld. Daarbij moet gesteld worden dat de interpretatie van een model verschilt met een interpretatie van de wereld. 

Omdat zowel de wereld als de machine fysieke realiteiten zijn an niet slechts abstracties, zijn de gemeenschappelijke beschrijvingen slechts een deel van de werkelijheid van beide objecten. For elk object zijn er meerdere beschrijvingen. Toch maken niet alle omschrijvingen deel uit van het getoonde reportoire. Zoals niet alle eigenschappen van een boek; meer dan een auteur, pseudoniemen, een onderdeel van een reeks, een gerevisiteerde versie, worden gereflecteerd in een database.  

Het interface aspect. Een machine kan een probleem in de wereld oplossen als de wereld en de machine phenomena kunnen uitwisselen. Maar de participatie is niet symmetrisch: een status kan als phenomena worden uitgewisseld maar slechts een partij kan er invloed op uitoefenen maar beiden kunnen dezelfde status signaleren. 

Het engineering aspect gaat over requirements, specificaties, en programma’s. Requirements hebben betrekking op phenomena in de wereld. Een programma heeft alleen betrekking tot de machinale phenomena. Het doel van programma’s is om eigenschappen en gedragingen te omschrijven van de machine ten behoeve van de gebruiker. Tussen de requirements en de programma’s zitten de specificaties. Omdat programma’s dan wel beschrijvingen zijn van een gewenste machine, maar dat moeten beschrijvingen zijn van de  machines  die de computers kunnen uitvoeren zodanig dat de computer deze beschrijvingen ook zo kan interpreteren. De engineer moet  de eigenschappen van de wereld kennen en begrijpen en deze eigenschappen manipuleren en laten werken met als doel het dienen van het systeem. 

Het probleem aspect. Het onderscheid tussen specificatie en implementatie. Het probleem zit in de relatie van de machine en de wereld. De machine brengt de oplossing maar het probleem zit in de wereld. Een vertoog over een probleem moet dus gaan over de wereld en over de opvatting die de gebruiker heeft in de wereld. Omdat de wereld veelzijdig is moeten we ervan uit gaan dat er verschillende soorten problemen zijn. Een realistisch probleem wordt dus niet opgelost met een simpele hiërarchische structurele aanpak en een homogene decompositie maar met een paralleele structurele oplossing waar beide kanten van het probleem worden opgelost. 



Ontkenningen 

We hebben als engineers de taak om een machine te bouwen aan de hand van de specificaties opgeleverd door de opdrachtgever. Een engineer heeft niet als taak de fitheid voor een doeleind te onderzoeken, maar wel de haalbaarheid naar een doeleind aan de hand van kennis, tijd, resources, budget en ontwikkelmethodiek. Daaruit komt naar voren dat een engineer zich richt op: elicitation (schetsen van een requirement), description (omschrijving) en analyse van de requirements waaraan het systeem moet voldoen. Vertaalt naar de volgende vragen: Wat is precies de klantwens?  Wat is de precieze omschrijving van het probleem? Voor welke doelen wordt het systeem gebouwd? Welke functies moet het systeem hebben? 

Denial by hacking: obsessief bezig zijn met een systeem omdat het de gebruiker veel macht geeft. Een uitgebreidheid van een systeem zorgt er soms voor dat mensen niet meer geprikkeld zijn na te denken over probleemstellingen, domein beschrijvingen en analyse. 

Denial by a abstraction. Wiskundige benaderingen van werkelijke problemen is  een belangrijke intellectuele strategie om problemen te formuleren. Een software ontwikkelaar moet een probleem kunnen omschrijven in zo min mogelijk woorden, maar de complexiteit ligt in de oplossing. 

Denial by vagueness. De vaagheid van een omschrijving is terug te vinden in: 

Von Neumann’s principe ,Principe van reductionisme ,Shanley principe en het Montaingnes’s principe.
Het Von Neumand principe uitgelegd
Voor een vocabulair  moet een grondslag zijn ontwikkeld waarmee gesproken kan worden over de wereld en de machine. Belangrijke phenomenen moeten geindtifieerd worden, door middel van een grondregel  of ‘herkenningsregel’ moet een fenomeen worden herkend, en vervolgens het fenomeen een formele term geven die gebruikt wordt als duiding van een bepaalde omschrijving. Dan moet voor de formele term een symbool gevonden worden. Samen vormen de grondregel en het symbool een designatie. 

Principe van reductionisme 

Simpelweg het openbreken van termen met een weerlegbare definitie totdat alle begrippen die worden gebruikt om iets te duiden  niet meer te herconstrueren zijn in hun definitie. 

Shanley principe 

Er bestaan volgens dit principe geen scherpe verdelingen in de wereld zoals wetenschappers soms denken. Een strenge opvatting over de wereld waarin een individu geclassificeerd kan worden als een onsamenhangend geheel. Maar dat is slechts een opname van een beeld. De werkelijkheid staat soms toe dat een elementair individueel object in verschillende classificaties verschillende getypeerd kan worden in een andere setting of view. 

Montaignes principe 

De incative mood; gaat over wat we beweren waar te zijn. 

De optitative mood; gaat over wat we willen dat waar is 


\paragraph{6 Variable model}
Optitatieve statements omschrijven de omgeving zoals we het willen zien vanwege de machine. 
Indicatieve statements omschrijven de omgeving zoals deze is los van de machine. 
Een requirement is een optitatief statement omdat ten doel heeft om de klantwens uit te drukken in een softwareontwikkel project. 
Domein kennis bestaut uit indicatieve uitspraken die vanuit het oogpunt van software ontwikkeling relevant zijn. 
Een specificatie is een optitatief statement met als doel direct implementeerbaar te zijn en ter verondersteuning van het natreven vande requirements. 
Drie verschillende type domeinkennis: domein eigenschappen, domein hypothesen, en verwachtingen. 
Domein eingenschappen  zijn beschrijvende statementsover een omgeving en zijn feiten.Domein hypotheses  zijn ook beschrijvende uitspraken over een omgeving, maar zijn aannames. 
Verwachtingen zijn ook aannames, maar dat zijn voorschrijvende uitspraken die behaald worden door actoren als personen, sensoren en actuators. 

  
\paragraph{Conceptueel model}



System requirement:
uitspraak over wereld fenomenen (gedeeld of niet) of doelen
die bereikt moeten worden.
met enige regelmaat informeel, niet precies geformuleerd.
Software requirement/specicatie:
uitspraak over gedeelde fenomenen of doelen die de machine
moet bereiken middels de onderdelen waar die machine uit
bestaat of middels de fenomenen waar de machine controle
over heeft.
doorgaans preciezer, meetbaar, exact geformuleerd.


Systemen gaan een zekere interactie aan met hun omgeving:
Sensoren: meten fenomenen uit de omgeving (temperatuur,
druk, licht, geluid, etc.)
actuatoren: veranderen iets in de omgeving (mechanische,
electrisch, pneumatisch, etc.)
Software:
Kan niet direct communiceren met de buitenwereld.
Snapt derhalve niets van de buitenwereld.
Kan alleen maar bestaan in en communiceren met het
systeem.


\paragraph{Requirementsengineering}

Om de juiste requirements te verzamelen en selecteren hebben we meer kennis nodig van de methoden hiervoor gebruikt in het domein van requirementsengineering. Daarom is een literatuurstudie gedaan naar rapporten en artikelen die ons meer informatie over dit onderwerp verschaffen.
 Uitdagingen in requirementsengineering zijn incomplete requirements en specifcates, veranderende requirements en specificates en grote, complexe oftwaresystemen.
 
 Het article the worlds a stage biedt inzicht in de requirementstechnieken voor een ambulance in london. In het artikel gaan de onderzoeks in op de volgende onderwerpen: 
 viewpoints, sociale ascpecten,evolutie, non-functional requirements, conflict resolution, traceability
 
 Goal of this paper is requirement  engineering on London aulance service
 Method of opinions: crew, staff, management, computational, transport, services
 Evolutioon: changes, specification and technology trade
 Environment: company policies, regulation, impact solution on organizational
 Non-functional aspect: communicatio problem, malfunctions, less critical isues: cost, tradeoff beween performance \& user interfaces
 vieuwpoint: is a subset of all system requirements expressible in a given requirements notation regardless of the stakeholders involved
 
 log change
 basic model vieuw
 hypertext vieuw
 data transmission problems
 continued difficulties
 installation problems
 problems caused by mistake
 tracebility requirements[selecting reliable information]
 PRE requirement specification traceability, repository baed approach
 1) compromise specification
 2) representatives
 3) agreement dimensions
 Domain: part of the worl in which the computer system effects will be felt, inclusing its peoples, organizational structure, related legislation, physical location and met only the compyter systems
 
 \bibitem{ID} ... \LaTeX:\\ \url{}
 
 Het artikel "from inconsistencyhandling to non-conanical requirements management: a logical perspective" geeft enkele tips voor het omgaan met inconsistente requirements:
 
 1) identifying non-canonicalrequirements
 2) measuring them
 3) generate caandidate proposals for handling them
 4) choosing acccptable probosals
 5) revising them acccording to the proposals

Het artikel "managing inconsistent specification: reasoning, analysis, action" zoekt een ontologische benadering voor het omgaan met inconsistenties in de requirements specificaties.
Voor de omshrijving van een specificatie kun je gebruik maken van logica. Daarbij kun je onderschei maken in klasieke logica quasi -logica.
Wat ook een rol kan spelen in domain interpretatie. De achtergrond van de gebruikers speelt ook een rol.
Zo is er e=onderscheid te maken in de volgende groepen: users, customers, domain experts, designers,, manufacturers
graphical  textual specification

Basic constraint, legal constraint, cooperation constraint
1) scenatio  definition
2) scenario analysis
3) scenario consolidation


Hoe kan een systeem verder worden ontworpen op een manier dat non-functionele requirements worden geimplementeerd?
Hoe hangt dat ontwerp samen met aanpassingen van het functionele en structurele aspect van het systeem?

block[objects, classes, methods, messages, inheritance]
[goals,agents, alternative, events, actions,existence modalities,agent responsibilities]
 \bibitem{ID} ... \LaTeX:\\ \url{}

Het artikel "representing and using nonfunctional requirements: a process-oriented approach"" gaat in op een het proces van requirements acquisitie. Hierbij in ogenschouw de acquisitie van prestaties, ontwerp en aanpasbaarheid.
product oriented
process oriented


Acquisitie Prestaties
user concern
-Hoe goed werkt het product
-Hoe goed wordt de bron gebruikt?>> Efficiency
-How veilig is het product >> integrity
-Met hoeveel zekerheid is uit  te sluiten dat het werkt >>Reliability
-Hoe goed werkt het product onder zware omstandigheden >> sustainability
-Hoe makkelijk is het product in gebruik >> usability
quality attribute


Acquisitie: Ontwerp
user concern
Hoe valide is het ontwerp
-Is ht ontwerp conform de requirements
-hoe makkelijk is het ontwerp te repareren
-Hoe makkelijk zijn de prestaties te verifieren

quality attribute


Acquisitie: Aanpasbaarheid
user concern
-hoe makkelijk is het om het product aan te passen
- hoe makkelijk is het om het product te updaten en/of uitbreiden>> expendability
- hoe makkelijk is het om een wijziging door te voeren>>flexibility
-hoe makkelijk is het om andere system aan te sluiten >> portability
- hoe makkelijk is het om het product te transporteren >> interoperability
-hoe makkelijk is het om te converteren tot een systeem gebruiksklaar voor communiceren met andere systemen>> reaseability
quality attribute
 



 \cite{jonkerTreurKlush200informativeAgents}
\cite{boehmBoseLeeRequirementsNegotiations}
\cite{muHungJinLiu2013inconsistencyReqs}
\cite{hunterNuseibeh1996manageSpecs}
\cite{myloloupos1992representingReqs}
\cite{zavePamela4darkCorners}
\cite{zavePAmela1997regEngineering}

%%%%%%%%%%%%%%%%%%%%%%%%%%%%%%%%%%%%%%%%%%%%%%%%%%%%%%%%%%%%%%%%%

what is a good software specification

\cite{fvaandrager2322010Goodmodel}
\cite{onix01102022devopmodel}
\cite{sulemani04012021softwareprocesmodel}
\cite{globalluxsoft18102017softdev}
\cite{wiegers30052022SRS}
\cite{muller06092020goodspecification}
\cite{informit30062008reqmanagement}
\cite{altexsoft15092020writingSRS}


\paragraph{Wat is een sluis}
\cite{woudagemaalSluizen}
\cite{bardetsluizenAmsterdam}
\cite{historischesluizen}
\paragraph{Recente ontwikkelingen op het gebied van sluisautomatisering}

Het ministerie van verkeer en Waterstaat wil in het kader van het klimaatakkoord en onderzoek laten uitvoeren naar de staat van het sluizenpark in Nederland. Het onderzoek moet zich richten op het ontwerpen en ontwikkelen van een geautomatiseerd sluismodel dat geschikt is voor een brede toepassing. In het onderzoek moet naar voren komen wat de huidige staat is van de sluizen met oog op veiligheid, efficiëntie, capaciteit, onderhoud, duurzaamheid en automatisering. Het onderzoek geeft aan hoe een volledig model worden opgeleverd opdat ontwerp van verschillend volledig geautomatiseerde sluizen in de toekomst geautomatiseerd kunnen worden.  


\paragraph{Studie naar rampen aan de hand van het vier variabelen model}
\newline \indent Voor deze studie is onderzoek gedaan naar verschillende rampen aan de hand van het vier variabelen model.
Elke ramp op deze manier categoriseren  kan ons helpen te bepalen in hoeverre requirements een rol kunnen spelen in de veiligheid van ons model.  Zo is er de bijlmerramp \cite{aviationsafety04101992airplaneCrashBijlmer}
, deze vond plaats op 04/10/1994. 
Op de avond van de 4e oktober 1992 ware er bij het toetel van el al fluctuaties in de selheidsregulering, daioproblemen, fluctuaties in de voltage electriciteit van motor 3
%Motor 3 (de binnenste motor aan de rechtervleugel van het vliegtuig) brak af, beschadigde de vleugelkleppen en botste tegen motor 4 die vervolgens ook afbrak.
%De ernst van de situatie werd op Schiphol niet goed ingezien. Dit kwam onder meer doordat lost in de luchtvaart de gebruikelijke term is om het verlies van motorvermogen te melden. Op Schiphol werd er dan ook van uitgegaan dat er twee motoren waren uitgevallen. Dat ze letterlijk verloren waren wist men niet. Gezien het grote aantal handelingen dat de bemanning in een paar minuten moest uitvoeren en de keuzes die de piloot maakte, veronderstelde de parlementaire enquêtecommissie die de ramp later zou onderzoeken dat ook de bemanning waarschijnlijk niet heeft geweten dat beide motoren van de rechtervleugel waren afgebroken. De buitenste motor van een 747 is vanuit de cockpit slechts met moeite zichtbaar en de binnenste motor helemaal niet.
%Op de avond van de 4e oktober 1992 was landingsbaan 06 (de Kaagbaan) in gebruik. De piloot verzocht de luchtverkeersleiding op Schiphol echter een noodlanding te mogen maken op de Buitenveldertbaan (baan 27). Waarom hij juist deze baan koos, is nooit duidelijk geworden. Een keuze voor deze baan lag niet voor de hand; omdat de wind uit het noordoosten kwam, zou het toestel met flinke staartwind moeten landen. Langs de landingsbaan waren enkele grote brandweerwagens van Schiphol geplaatst. Deze zogeheten crashtenders moesten een brand tijdens de landing meteen blussen. Na de crash werd één zwarte doos teruggevonden. De bijbehorende band was in vier stukken gebroken, waardoor de laatste 2 minuten en 45 seconden ervan niet meer te gebruiken waren. De doos werd voor onderzoek naar Washington gestuurd en leverde uiteindelijk onderstaande informatie op.
%Om goed uit te komen voor de landingsbaan vloog het beschadigde toestel eerst nog een rondje boven Amsterdam. Tijdens dit rondje gaf de gezagvoerder de copiloot opdracht de vleugelkleppen (flaps) uit te schuiven. Links schoven de kleppen uit, maar doordat de afgebroken motor 3 de rechtervleugel had beschadigd schoven de kleppen op die vleugel niet uit. Als gevolg hiervan kreeg het toestel links meer draagvermogen dan rechts. De piloot meldde aan de verkeersleiding dat er ook problemen met de flaps waren.
%Aanvankelijk ging het aanvliegen van de Buitenveldertbaan goed. Op het moment dat het vliegtuig daalde tot onder de 1500 voet en snelheid minderde, raakte het echter compleet onbestuurbaar en maakte het een ongecontroleerde, scherpe bocht naar rechts. Over de radio was te horen dat de gezagvoerder zijn copiloot in het Hebreeuws opdracht gaf om alle kleppen in te trekken en het landingsgestel uit te klappen. Vervolgens meldde de copiloot in het Engels aan de luchtverkeersleider dat het toestel zou gaan neerstorten. Uit later onderzoek bleek dat het vliegtuig eerder enkel recht bleef vanwege de hoge snelheid (280 knopen, zijnde 519 km/u). 
Doordat de rechtervleugel beschadigd was, was het moeilijker om het vliegtuig recht te houden. Alleen de hoge snelheid zorgde ervoor dat er nog voldoende draagvermogen was. Toen bij het inzetten van de landing de snelheid verlaagd werd, werd het draagvermogen van de rechtervleugel echter dusdanig gering dat het toestel niet meer onder controle te houden was en een duikvlucht naar rechts maakte.
\cite{aviationsafety04101992airplaneCrashBijlmer}
 \cite{catsr25022009Boeing737AmsterdamCrash}
\cite{zuilen23022019Tijdlijnpoldercrash}
\cite{wikinews04032009techfoutailines1951}
\cite{luchtvaartnieuws21012020boeing737conclusies}
\cite{adformatie280220209communicatiegebreken}
\cite{spinnael25022009onderzoekpolderbaancrash}
\cite{crashTurkishAirlines}
\cite{flightradar24}
\cite{flightstatstracker}. 
%%%%%%%%%%%%%%%%%%%%%%%%%%%%%%%%%%%%%%%%%%%%%%%%%%%%%%%%%%%%%%%%%
\newline \indent Dan nog de  ramp turkisch airlines vlucht 1951 op woensdag 25 februari 2009
25 februar 2009
De automatische reactie van het toestel werd getriggerd door een fout gevoelige radio altimeter waardoor de automatische gashendel de energiemotot op actief stelde.
Inadequaat handelen van de piloten ondanks een defecte hoogtemeter en onvolledige instructies van de luchtverkeersleiding
\cite{catsr25022009Boeing737AmsterdamCrash}
\cite{zuilen23022019Tijdlijnpoldercrash}
\cite{wikinews04032009techfoutailines1951}
\cite{luchtvaartnieuws21012020boeing737conclusies}
\cite{adformatie280220209communicatiegebreken}
\cite{spinnael25022009onderzoekpolderbaancrash}
\cite{crashTurkishAirlines}
\cite{flightradar24}
\cite{flightstatstracker}
%%%%%%%%%%%%%%%%%%%%%%%%%%%%%%%%%%%%%%%%%%%%%%%%%%%%%%%%%%%%%%%%%
\newline \indent De therac-25. In de periode van Juni 1985 and Januarie 1987 zijn er meerdere ongelukken met dodelijka afloop door de implementatie van de Therac-25 bij de behandelig van huidkanker.
De therac-25 is een Medische lineaire versneller. Deze  versnellen elektronen om stralen met hoge energie te creëren die tumoren kunnen vernietigen met minimale impact op het omliggende gezonde weefsel.
Onderzoekers constateren dat er fouten zijn gemaakt tijdens de (her-)implementatie van systemen uit eeerdere productiemodellen.


Yakima Valley Memorial Hospital in 1985
Manufactureere response
Government and user response
East Texas Cncer Center, March 1986
Manufactureere response
Government and user response
East Texas Cncer Center, April 1986
Manufactureere response
Government and user response
Yakima Valley Memorial Hospital

Onderzoekers zijn van mening dat de terkortkokmigen in medische apparatuur niet geheel en altijd te verwijten zijn aan softwareproblemen. Zor is er ook een rol wegglegd voor Manufactureere response
Government and user response.

Medische lineaire versnellers versnellen elektronen om stralen met hoge energie te creëren die tumoren kunnen vernietigen met minimale impact op het omliggende gezonde weefsel.

Manufactureere response
Government and user response
Yakima Valley Memorial Hospital in 1985
Manufactureere response
Government and user response
East Texas Cncer Center, March 1986
Manufactureere response
Government and user response
East Texas Cncer Center, April 1986
Manufactureere response
Government and user response
Yakima Valley Memorial Hospital

Manufactureere response
Government and user response


Veel fouten in safety-critical systeem. In geval van therac spreken we van een systeemongeluk, complexe interacties tusse verschillende componnten en activiteiten. In het artikel woden 6 ongevallen omschreven.
In het eerste geval is het neit helemaal duidelijk wat er is gebeurd.
In het tweede geval is er sprake van onvolmaakte microswitchtes welke	 een ambigu bericht produceert voor de computer.
In het derde geval zijn er open slots in de blocking trays.
In het Vierde geval  heeft de operator verkeerde prescriptie-data ingevoerd. De opertor drukt op return en bevestigd alsnog de invoer. Op een gegeven moment komt er een foutmelding "malfunction 54". De opertor was bekend met de machine en drulte op de knop "p" van proceed.
In het vijfde geval was er een verkeerde invoer voorschift data waardoor verkeerde toets werd gedrukt. Na aanpassen werd de return-toets ingedrukt. Er rad een fout op met de melding "malfunction 54" Na reproductie van de melding bleek dat de ionische amer neit gezouten bleek te zijn
In het zesde geval vergat de operator de files te verwijderen onder de patent. Er werd straling geenten maar de console gaf aan dat er geen dosisratio was gemeten. De operator drukte op de knop "p" om het proces te pauzeren.
Gerelateerde theac-20 problemen
Terwijl de therac 20 afhankelijk was van mechanische vergrendelingen werd er bij de therac-25 software gebruikt

Software problemen zijn onder andere
-slechte software engineering/designing praktijken
-er is een machine gebouw dat afhankelijk is van software voor veilgheidsoperaties
= de fout in de code is niet zo belangrijk als een geheel onveilig ontwer

cleaning the  bendory-magnet variable nistead of at the end of the frame
race conditioning to indicate prescription entry is still in progress


user response
- poor screen refresh subroutines that left trash and erroneous information on the oeprating console
- tape loading problems upon startup whwre dscouraed icluded the use of photom tables to trigger the interlock system in the event of a  load faul instead of a checksum

Gebruikersgroepen klagen over het tekort aan aoftware-evaluaties en 2) een tekort aan hard-copy audit trials om foutmeldinen in beeld te krijgen


%
%
%Medical lineair accelerators accelerate electrons to createhighenergy beams that can destroy tumors with minimal impact on the surrounding healthy tissue.
%In the mid-1970s, AECL, developed a radical new "double-pass" concept for electron acceleration. A double passaccelerator needs much less spaceto develop comparableenergy levels because it folds the long  physical mechanismrequired to accelerate the electros, and it is more economic to produce.
%Using this double pass concept AECL designed the  Therac-25, a dual mode lineair acelerator that can deliver either photonsat 25 MeVor electrons at various energy levels. Compared with theTerac-20 The Thrac-25 is notably more compact,, more versatile, and arguably easier to use. 
%The higejr energy takes advantage of the phenomenon "depth dose": As the energy increases, the depth in the body at which maximum dose buildup occurs alse increases, sparing the tissue above the target area.
%First, like the Therac-6 and the Therac-20, the Therac25 is conrolled by a PDP11. The Terac-6and Therac-20 had been designed around machines that already had histories of clinical use without computer control.
%The therac-20 has idependent protective circuits for monitoring electron-beam scanning, plus mechanical interlocks for policing the machine and ensuring safe operation.
%Finally some software for the machines was interrlatd or reused.
%Eleven therac-25 were installed: five in the usand six in canada. Six accidents involving massive oerdoses to patients occured between 1985 and 1987. The machine was recalled in 1987 for extensive design changes, including hardware	 safeguards against errors.
%Kennestone Regional Oncology Center 1985
%Door rechtzaken waren managegers op de hoogte van de problemen en ongelukken. Maar er werd in het vervolg niet over gerapporteerd.
%The treatment prescription printout failure was disabled at the time of the accident , so there was no hardcopyof the treatment data.
%Ontario Cancer Foundation in 1985
%Since the machine did not suspendand the control display indicated no dose was delivered to the patient, the operator went ahead with a second attempt at trratment by pressing the "P" key, expecting the machine to deliver the proper dose this time. This was standard operating procedure and, described in the "The operating interface" on p 24, Therac 25
%oprators had become accustomed to freunt malfunctions that had no untowardconsequences for the patient. Again, the machine shut downin the same manner. The oeprator repeated this process four times after the original attempt- the display showing "no dose" delivered to the patient each time. After th fifth pause, the machine went into treatment suspedn, and a  hospital service technician was called.
%The technician found nothing wrong with the machine. This was not an unusual scenario, according to the Therac-26 operator
%Manufactureere response
%Government and user response
%Yakima Valley Memorial Hospital in 1985
%Manufactureere response
%Government and user response
%East Texas Cncer Center, March 1986
%Manufactureere response
%Government and user response
%East Texas Cncer Center, April 1986
%Manufactureere response
%Government and user response
%Yakima Valley Memorial Hospital
%Manufactureere response
%Government and user response
%Softwarefout uit zich als hardwarefout de klachtafhandeling geen onderzoek geen second opinion is prioriteit wel 
%gechecked na onderzoek bellen en geen prioriteit aanwezig te zijn alleen importeurs en fabriken mogen fouten 
%in frabrieksinstellingen rapporteren 
%Therac25 Systeem ligt plat veel voorkomende eror stdaardafhandeling om de error te verwerpen resultaat: 
%de patient kreeg overdosis patient overleden onderzoek opgestart, stuatie niet reproduceerbar foutmarkering: 
%gezien als uitzonderlijk, software aanpassing van groote magnitude 5; de oorzaak was waarschijlijk mechanisch 
%maar neit vastgesteld; conceptueel odel niet aangepast probleemclassicificatie door autorititen het probleem 
%en de impact daarvan anar beneden bijgesteld AEFL doe gedeeltelijke aanpassing om hardware na berisping 
%Canadese autoriteit 
%Derde patient overleden door eythema AECL wijst alle doodsoorzaken af AECL beweert dat geen vergeli- 
%jkbare voorvalle bij andere machines of patienten zijn voorgekomen geen vervolgonderzoek vanwege garanties 
%bedrijf gaat uit van geen mogelijke functionele fout 
%vierde patient overleden aan overdodis ontstaan door bug in software onjuiste aanduiding bij de foutmelding 
%verkeerde reactie/invoer ddoor operator communicatie tussen patient en operator werd onvoldoende gemon- 
%itorred ( apparatuur niet aangesloten, en audio monitor kapot) engineer van AECL stelt geen fouten vast 
%Engineer AECl kan fout niet reproduceren Geen communicate tussen bedrijf en uitgezonden technisci over 
%vergelijkbare probleemgevallen 
%vijfde geval malfunction 54 leidt tot overdosis en de dood fout gereproduceerd door operator bedrijf fout 
%was daa entryspeed herpublicatie van de ongevallen en de eerdere ongevallen in de meia apparaat wel nog in 
%gebruik genomen niet handig, waarschuwingsberichten en aanwijzingen voor een bugfix naar de gebruikers door 
%druk van fda is bedrijf op zoek gegaan naar permanente oplossing 
%zesde geval software fout door softwarefout otntstaat lightstruct .. op de patient na onderzoek door AECL 
%blijkt niet alleen hardware de oorzak gebruikers direct geinformeerd oplossing gevonden, media ingeschakeld om transparantie af te dwingen door de gebruikersgroep en de FDA AECL gedwongen functionaliteit aan te passen 
%Engineers hebben meer studie moeten maken van gebruikte technologie en onderhoudbaarheid daarvan 
%sheets
%\cite{rogaway2004therac25}
%~\cite{wikiTherac25}
%reproduceren van de error. IN dit stuk wordt uitgelgd hoe het product werkt en waarom bepaalde beslssingen zijn genomen in de ontwerp/productiefase
%\cite{lynch2017theracRaceConditions}
%kort artikel met daarin een opsomming van alle fouten in het systeem en een korte uitleg
%\cite{lim1998theracdisaster}
%uitgebreid artikel over hoe de fout werd gereproduceerd en de resultaten daaruit voortkwamen. Alsnog werden er na de reproductie fase nog meer fouten gevonden.
%\cite{fabio26102015therac25}
%artikel
%\cite{ethicsunwrappedTherac25}
%onderzoeksartikel waarin de bug wordt uitgelgd: de racecondities, de bytepositie en het testen worden berkitiseerd envenals andere onderdelen van het softwareproces
%onrealistisch testplan. In dit artikel egt de auteur het belang nog eens uit van goede requirements en implementatie, niet de software is waar het probleem ligt
%geschiedenis
%\cite{casesHistoryTherac25}
%artikel
%\cite{caballero2019Therac25}
%computer error. De ongeval en de malfunction nog een keer uitgelegd
%\cite{rose1994theracFatalDose}
%rapport
%\cite{levesonMITTherac25}
%\cite{grant1978theracevaluation}
%onderzoeksartkel
%\cite{turnerTheracAccidentsInvestigations}
%\cite{turner1993TheracAccidentsInvestigations}
%uitgebreid artikel gaat hier ook wat meer over de hardware
%\cite{wang2017industrialdesignengineering}
%artikel waarin in 3 delen de problemaiekwordt blootgesteld
%\cite{levesonturner1993theracpart2}
%case study sheets
%artikel waarin vooral de fabriikant ervan langs krijgt
%\cite{porelloTheraccFailure}
%lessons learned. Vooral de begrippen betrouwbaarheid, welgevalligheid, veilgheid en gebruiksvriendelijkheid
%\cite{theracIncidents}
%root-cause analysis
%case study
%\cite{huffbrown2004casestudyethicatherac}
%case study
%\cite{sebowikimedicalradiation}
%opzetten van systematische acceptaatie test met therac als voorbeeld
%\cite{hsia1995testtherac25}
%artikel waarin een diagnose plaatvindt voor het bedrijf en de ingenieur/ontwerper
%\cite{magsilvaTheracTesting}
%rapport
%oorzaken aangegeven in artikel
%\cite{chemeuropetherac25}
%het onderzoek en enkele ontwerptekeningen en oplossingen
%\cite{statsenko10102016Therackillerbug}
%\cite{therac25casestudy}
%\cite{thomas1994theracinLotos},
%\cite{twitter2019programmerbehindtherac}
%wiki
%\cite{wikibookstherac}
%analyse
%\cite{bozdagTherac25}
%samenvatting
\cite{levesonTurnerTheracAbstract}
%rapport over de fouten die de verschillende partijen hebben gemaakt( overheid, ingenieurs, bedrijf, operators) en de verbeterpunten
%onderzoeksrapport
%slides online over het technisch mankement
%Wat is er gebeurd, nou het volgende:
%Normal radiation treatments: 6,000 rads over a 3 week period, under certain conditions Therac-25 was delivering 60,000 rads during one session.
%En wat ging er mis?
%Paradigm Shift
%Therac-25 replaced expensive hardware safety interlocks with software controls
%Real-time software
%Design
%Race condition caused focusing element to be incorrectly set
%No indication of actual hardware settings
%Error messages appeared the same regardless of how important
%Error messages were difficult to understand
%All errors messages could be manually overridden
%oorzaak-gevolg diagram
%veiligheidsanalyse naar de rapportage van foutmeldingen, de beslissingsmatrix waarmee het programma wordt uitgevoerd en de software-analyse door een consultat
\cite{stackexchange2021therac25code}
\cite{rogaway2004therac25},
\cite{wikiTherac25}, 
\cite{lynch2017theracRaceConditions},	\cite{lim1998theracdisaster}, 
\cite{fabio26102015therac25},	 	\cite{ethicsunwrappedTherac25}, 	\cite{casesHistoryTherac25},	 	\cite{caballero2019Therac25}, 	\cite{rose1994theracFatalDose}, 	\cite{levesonMITTherac25},
\cite{grant1978theracevaluation},	 	\cite{turnerTheracAccidentsInvestigations},	\cite{turner1993TheracAccidentsInvestigations}, 	\cite{wang2017industrialdesignengineering}, 	\cite{levesonturner1993theracpart2},	\cite{porelloTheraccFailure},\cite{theracIncidents}, 
\cite{huffbrown2004casestudyethicatherac}, 
\cite{sebowikimedicalradiation},	\cite{hsia1995testtherac25},	\cite{magsilvaTheracTesting},
\cite{chemeuropetherac25},	\cite{statsenko10102016Therackillerbug},	\cite{therac25casestudy},	\cite{thomas1994theracinLotos},	\cite{twitter2019programmerbehindtherac},	\cite{wikibookstherac}, 
\cite{bozdagTherac25},	\cite{levesonTurnerTheracAbstract}, 	\cite{stackexchange2021therac25code}.
%%%%%%%%%%%%%%%%%%%%%%%%%%%%%%%%%%%%%%%%%%%%%%%%%%%%%%%%%%%%%%%%%
\newline \indent
%Hoe werkt het
tesla autopilot features voor dataverzameling\cite{denneyjdsupraFeds},\cite{gritti24062020tesladataengine}.
% crashes
 De eerste tesla crash is van juni 2016 \cite{impakterTeslaCrash}. En meerdere zouden volgen.
Een ongeluk in  de VS waarbij 2 inzittenden om het leven kwamen. Een persoon had plaats genomen als bijrijder en de andere persoon als passagier achter de stoel van de bestuurder. Waarschijnlijk was de autopiloot niet ingeschakeld.
\cite{anderson30042021secondteslacrash},\cite{raynal20042021probeTeslaCrash},\cite{firstpress11052021fatalnonautopilot},\cite{cochran18042021nodriverTeslaCrash},\cite{gitlin11052021autopilot},\cite{sommerfield12072021NHTSAmandateresult},\cite{hawkins30062021nhtsarequiresreporting},\cite{wilson19042021teslacrashregulators},\cite{mcfarland22042021selfdrivingrisks}
De situatie en oorzaken zijn bij elke ramp verschillend. 
Een automobilist heeft in een rit van 37 minuten slechts 25 seconden zijn handen aan het suur gehad ondanks de melding "Hands requireed not detected". Hiermee zijn de onderzoekers van de NTSB ervan uitgegaan dat de bestuurder de autopiloot bewschouwde als een volledig autonooom rijsyssteem in plaatst van een veligheidsmechanisme
\cite{oremus21062017fatalTeslaCrash}. Of in 
Mei 2015 als een besuurde foto's van zichzelf maakt in de testla zonder handen aan het stuur of voeten op het pedaal.
\cite{guardian15052021teslacrashHandsOnWheel}
Een faatale crash in 2016 waarbij de bestuurder  e veel vertrouwde op het semi-autonome rijtechnologie op het verkeerde type wegdek.
\cite{Puzzanghera13092017TeslaSharesBlame}
Onderzoek naar een fatale crash op 7 mei 2016 toont aan dat er beperkingen zitten aan de autopilot mode. Om specifiek te zijin is de automatische noodrem niet failsafe, blijkt uit onderzoek.
\cite{jaillet02022017teslaAutopilotLimitations}
\cite{reuters03102019teslaAutoParkingFail}
\cite{dowling23042021}
Op  April 17 2019 een autocrash waarbij het onduidelijk is of de autopiloot aan stond.
\cite{young05112021fatalTeslaReport}. Een auto ongelu waarbij een tesla is betrokken. De bestuurder was waarschijnlijk afgeleid door de games op zijn apple telefoon. De NTSB gaf aan dat het crash-avoidance systeem neit otnworpen is en ook geen crash atnuaor heeft gedetecteerd. Herdoor accelereerde de autopilot  het voertuig. Ook Faalde het systeem in het verschaffen van een crash aleter en werden de noodremmen niet geactiveerd.
\cite{tiungteslasoftwarecrash}
Er is ook een melding van een tesla waarvan de autopilot bots tegen een stilstaande politieauto
\cite{kierstein18032021teslaAutopilotCrashStationary}. Ook uit dit onderzoek blijkt dat er geen gebreken waren en dat het automaische remsysteem neit kapot was. De HNTSA concludeerder dat de bestuurder zelf geen actie ondernam door  bij te sturen of te remmen. In een eerder artikel kwam naar voren dat de tesla een autopilot krijgt die enkel camera's en GPS gebruikt; lidar of een radarsysteem wordt niet toegepast.
\cite{janssen20062017teslacrashdetailflorida}
Enkele fotos van crashes met autonome rijsysstemen \cite{saferoardsCrashesAutonomousvehicles}.
\cite{stephardson18032021revieuwingtesla}
%Onderzoeksrapport naar testla automatic vehicle control system
\cite{habib28062016NHTSATeslaReport},
\cite{darkReading17112020TeslaBackup},
\cite{heilweil26022020teslaAutopilot}
% overzicht
Tesla autopilot crashes met meer crashes en incidenten dan tot dan toe gerapporteerd
\cite{teslaFDSCrash}
De meest voorkomende crashes zijn stationaire objecten bij hoge snelheden, lane incursions from stationary objects, auti=opilot confusion at forks and gores.
\cite{teslaCrashesCauses}
\cite{teslacrashOvervieuw}
\cite{tesladeaths}
% veiligheidsrisico''
De veiligheidsrisicos van de tesla lopen uiteen. Zo zijn er risicos in de machinelearning technologie:
veiigheidsrisico Three Small Stickers in Intersection Can Cause Tesla Autopilot to Swerve Into Wrong Lane
\cite{evan01042019teslaautopilotIntersection},
\cite{lambert31062020q2safetyreport},de autopilot zelf
\cite{templeton06092019HTSBReportTesla}. Een studie door de consumntenbond in de VS toont aan dat hetautopilot systeem van de testla niet failsafe is. Zo zijn de sensoren, gebrukt voor detectie van een bestuurder negatief te beinvloeden.
\cite{dowling23042021autopilottricking} Maar ook andere problemen met de bluetooth 
\cite{wiredBloutoothHackTesla}, touch screen
\cite{preston14012021NHTSATeslaRecall},
Web-based attack crashes Tesla driver interface
\cite{leyden23032020TeslaInterfaceHack}.
Of zelfds de tesla batterij is veiligheidsvraagstuk geworden
\cite{mitchell01072020teslabatterycooling}.
Maar ook was een onderzoeker  was in staat om persoonlijke details van afgedankte voertuigonderdelen  te vekrijgen nadat deze waren afgekeurd vanwege upgrades en reparaties op consumentenvoertuigen.
\cite{stumpff04052020TeslaPersonalData}
Data-opslag in de cloud niet altijd bereikbaar.
\cite{mitchell24022020AIDataTesla}
%Wat er mis zou kunnen gegeaan wordt dru over gespeculeerd online.
%\cite{stackexchange102019teslacarmistake}
dodelijk ongeluk
\cite{fottrell03092018TeslaSecurityChecks},
softwarefout maakt diestal mogelijk
\cite{kirk26112020modelX}
fouten ontdekt in onderzoek
\cite{bbc24022021hyundaiBatteryFireFix},
tesla cloud gehacked
\cite{hawkins22102022}.
This analysis considers the potential impacts of completely self-driving vehicles on vehicular liability. 
\cite{griemannExaminSelfDriving}
Dan zijn er nog maatschappelijke problemen die de aanpak moeilijker maken.
Er is in de vs in verschillende staten een andere wetgeving
\cite{berry21042021teslacrashtexas}
\cite{hull23072021regulatorsaftercrash}
\cite{wikiTeslaAutopilot}
%oplossingen
Toch zijn er oplossingen en tegenmaatregelen.
tesla gaat advanced driver assistance systems inzetten met behulp van  passive visual, ultrasonic, en radar.
\cite{tasking07062017TeslaAugmentedSafety},\cite{ackerman01072016TeslaImperfect}
Safe system solutions door David Harkey
\cite{Harkey30052019SafeSystemVehicle}
%maatregelen
Voor elke auto uitgerust met een level 2 tot level 5 autonomy wordt nu standaard een rapport van van de crash opgvraagd door de NTSA. Dit in het kader van verder onderzoek waarbij de autoritait kijk naar  ziekenhuisbehandeling, fataliteit, airbag deployment.
\cite{szymkowski29062021nhtsaTeslaCrashReports}. 
Door een softwarefout zijn er situaties ontstaan waarin het systeem informatie een onvoldoende informatie positie had om de juiste beslissingen te maken. Of dat de informatieverwerking niet juist was.
\cite{teslaFDSCrash}
\cite{teslaCrashesCauses}
\cite{teslacrashOvervieuw}
\cite{tesladeaths}
veiigheidsrisico
\cite{evan01042019teslaautopilotIntersection}
\cite{testVehicleSafetyReport}
veiligheidsrapport mbt autopilot
\cite{lambert31062020q2safetyreport}
consumentenrapport
bluetooth veiligheidsvraagstuk
\cite{wiredBloutoothHackTesla}
veiigheidsvraagstuk vanwege touch screen
\cite{preston14012021NHTSATeslaRecall}
veiligheidsvraagstuk
\cite{cio25112020belgianTeslaHack}
veiligheidsvraagstuk
rapport over autopilot
\cite{templeton06092019HTSBReportTesla}
de invloed van de bestuurder bij tesla ongeluk
veiligheidsvraagstuk
\cite{darkReading17112020TeslaBackup}
veiligheidsvraagstuk
\cite{leyden23032020TeslaInterfaceHack}
veiigheidsvraagstuk
\cite{huddlestonjr03042019ChineseTeslaHack}
veiligheidsvraagstuk
veiligheidsvraagstuk
\cite{heilweil26022020teslaAutopilot}
rapport over ongeluk
veiligheidsvraagstuk
veiligheidsvraagstuk
\cite{blanco04102019NHTSATesla}
veiligheidsvraagstuk
ransomware aanval op tesla
tesla batterij is veiligheidsvraagstuk geworden
\cite{mitchell01072020teslabatterycooling}
ongeluk
\cite{bbc26022020AutopilotCrash}
veiligheidsvraagstuk
veiligheidsvraagstuk
\cite{stumpff04052020TeslaPersonalData}
dodelijk ongeluk
\cite{levin08062018teslaautopilotsafety}
veiligheidsvraagstuk: ransomware
veiligheidsvraagstuk: medewerker in de fout
\cite{cbrook06082021TeslaInsideDataThreft}
\cite{shilling25022021Tesla}
veiligheidsvraagstuk: hackers je systeem laten testen
verdedigen tegenover ransomware
veiligheidsrisico
prijzen omlaag
autopilot
\cite{randall05112019modelSurvey}
malware door een medewerker
dodelijk ongeluk
\cite{fottrell03092018TeslaSecurityChecks}
waarom een tesla stelen bijna onmogelijk is
veiligheidsonderzoek
softwarefout maakt diestal mogelijk
\cite{kirk26112020modelX}
fouten ontdekt in onderzoek
\cite{bbc24022021hyundaiBatteryFireFix}
tesla cloud gehacked
\cite{hawkins22102022}
\cite{gritti24062020tesladataengine}
\cite{bouchard07052019teslaDeepLearning}
\cite{Srikanth2019teslabigdata}
\cite{rangaiah25022020teslaAI}
\cite{marr08012018taslabigdataAI}
\cite{bdickson29072020teslalevelfive}
\cite{dcruz17062022tesladesignthink}
\cite{mcfarland22042021selfdrivingrisks}
\cite{hawkins18032021fedgovinvest}
\cite{berry21042021teslacrashtexas}
\cite{hull23072021regulatorsaftercrash}
\cite{wikiTeslaAutopilot}
\cite{nhtsaAutomatedVehiclesSafety}
\cite{dowling23042021autopilottricking}
\cite{wilson19042021teslacrashregulators}
\cite{seamans22062021aikillerap}
\cite{mitchell24022020AIDataTesla}
\cite{denneyjdsupraFeds}
\cite{siddiqui22102020TeslaCriticism}
\cite{ackerman01072016TeslaImperfect}
\cite{greene04092019misuseautopilot}
\cite{michralli26112019ubserautocarcrsash}
\cite{pitmann21072021wrongfullautodeath}
\cite{stackexchange102019teslacarmistake}
\cite{tasking07062017TeslaAugmentedSafety}
\cite{griemannExaminSelfDriving}
\cite{Harkey30052019SafeSystemVehicle}
tesla crash report
\cite{shepardson18062021TeslaDeaths}
\cite{hawkins30062021nhtsarequiresreporting}
\cite{hawkins10052021autopilotnotavailable}
\cite{szymkowski29062021nhtsaTeslaCrashReports}
\cite{abc1112052021AutopilotNotinTeslaCrash}
\cite{ankel18062021regulatorsinvestigateAutopilot}
\cite{sommerfield12072021NHTSAmandateresult}
\cite{saferoardsCrashesAutonomousvehicles}
\cite{stephardson18032021revieuwingtesla}
\cite{krishner30062021NHTSAreport}
\cite{gitlin11052021autopilot}
\cite{mitchell19012017investigationstop}
\cite{gordon10052021teslaprelimreport}
\cite{shaper07062018}
\cite{cochran18042021nodriverTeslaCrash}
\cite{habib28062016NHTSATeslaReport}
\cite{firstpress11052021fatalnonautopilot}
\cite{raynal20042021probeTeslaCrash}
\cite{tiungteslasoftwarecrash}
\cite{globaltimes08052021guangdongcrash}
\cite{anderson30042021secondteslacrash}
\cite{oremus21062017fatalTeslaCrash}
\cite{guardian15052021teslacrashHandsOnWheel}
\cite{Puzzanghera13092017TeslaSharesBlame}
\cite{jaillet02022017teslaAutopilotLimitations}
\cite{reuters03102019teslaAutoParkingFail}
\cite{dowling23042021}
\cite{young05112021fatalTeslaReport}
\cite{kierstein18032021teslaAutopilotCrashStationary}
\cite{janssen20062017teslacrashdetailflorida}
%%%%%%%%%%%%%%%%%%%%%%%%%%%%%%%%%%%%%%%%%%%%%%%%%%%%%%%%%%%%%%%%%
\newline \indent De slmramp op  07/06/1989.
Toen de Anthony Nesty Zanderij naderde, was het daar, anders dan het weerbericht had voorspeld, mistig. Het zicht was evenwel niet zo slecht dat er niet op zicht kon worden geland. Gezagvoerder Will Rogers besloot echter via het Instrument Landing System (ILS) te landen, hoewel dit niet betrouwbaar was en hij voor zo'n landing ook geen toestemming had. De gezagvoerder brak drie landingspogingen af. Bij de vierde poging negeerde de bemanning de automatische waarschuwing (GPWS) dat het toestel te laag vloog. Het toestel raakte op 25 meter hoogte twee bomen. Het rolde om de lengteas en stortte om 04.27 uur plaatselijke tijd ondersteboven neer.
Uit onderzoek bleek dat de papieren van de bemanning niet in orde ware door nalatigheid in de crew-member screening
Geconcludeerd werd dat de gezagvoerder roekeloos had gehandeld door voor een ILS-landing te kiezen terwijl hij daar geen toestemming voor had, en door onvoldoende op de vlieghoogte te hebben gelet. 
De SLM werd verweten de kwalificaties van de bemanning onvoldoende te hebben gecontroleerd.
Oorzaak: het roekeloos besturen door de kapitein onder de minimum hoogte leidde tot collissie met een boom.
\cite{espnSLMterugblik},\cite{dennisRosier01052020}
\cite{hassing07062020slmramp},\cite{amsterdamArchiefSLM},\cite{rtvOost06062019nabestaande},
\cite{breda07062021AndroSnel},\cite{andereTijdenSLMCrash},
\cite{aviationReport},\cite{aviationSLMCrashAccidentInvestigation},\cite{mcDonnelDouglasCommissionReportSLMCrash},
\cite{wikiSRFlight764},\cite{nos07062019SLMTerugblik},\cite{dagvantoenSLMCrash},\cite{waterkantNesty07061989},\cite{eduNandlalSRCrash},\cite{oldjetsSRAirways},\cite{cloudberg02012021srflight764},\cite{apnews07061989srplanecrash}.
%%%%%%%%%%%%%%%%%%%%%%%%%%%%%%%%%%%%%%%%%%%%%%%%%%%%%%%%%%%%%%%%%
\newline \indent De schipholbrand op 27/10/2005\cite{schipholbrand27102005video},\cite{schipholbrand27102005video},\cite{onderzoeksraad2610schipholoost},
\cite{schipholbrandvideoargos},\cite{nunl30052023feitenoverzicht},\cite{parlementairemonitorschipholbrand},\cite{videonpoNOVA13112008},\cite{rizoomes01052014schipholbrand},\cite{heuvelkroesschipholbrandcamerabeelden},
\cite{wikiSchipholbrand},\cite{schipholbrand27102005video},\cite{onderzoeksraad2610schipholoost},\cite{schipholbrandvideoargos},\cite{nunl30052023feitenoverzicht},\cite{singeluitgeverijenSchipholbrand},\cite{eenvandaagschipholbrand},\cite{parlementairemonitorschipholbrand},
\cite{videonpoNOVA13112008},\cite{rizoomes01052014schipholbrand},\cite{heuvelkroesschipholbrandcamerabeelden}. 
27/10/2005
11 doden onder mirgranten in de cellenclomplexen van schiphol-oost. Doodsoorzaak van de slachtffoffers isverstikking. Het gebrouw voldeed niet aan de eisen voor brandveiligheid, personeel was niet goed getraind voor dergelijke situaties en de hulpverlening kwam door verschillende factoren te laat op gang.
%
%Wat is er gebeurd?
%\cite{schipholbrand27102005video}
%artikel
%\cite{schipholbrand27102005video}
%psychologische gevolgen
%rapport
%\cite{onderzoeksraad2610schipholoost}
%artikel met video
%herdenking
%impact op de persoon
%herdenking
%\cite{schipholbrandvideoargos}
%chronologie
%\cite{nunl30052023feitenoverzicht}
%tijdlijn
%vervolgens van ministers
%beeldanalyse en reconstructie
%\cite{}
%herdenking
%korte samenvatting
%rapport
%artikel
%verwijzing naar het rapport vanuit de politieke oppositie
%beeld vanuit de gevangenisbewaarder
%nationaliteit slachtoffers schipholbrand
%verblijfsvergunning voor de slachtoffers
%gen schadevergoeding voor de verdachte
%verdachte voor de rechter
%geen schadevergoeding voor verdachte
%artikel wat ging er mis bji de schipholbrand
%brand veroorzaakt door een peuk
%smaadschrift
%bewakers worden niet vervolgd
%proces schipholbrand moet over en de brandveilgheid moet worden verbeterd
%de rol van het parlement in de evaluatie
%\cite{parlementairemonitorschipholbrand}
%onderzoeksmemo
%herdenking
%herdenking
%invloed van de ramp op samenleving
%\cite{videonpoNOVA13112008}
%opmerkelijk rapport gestolen in de nasleep
%\cite{rizoomes01052014schipholbrand}
%publicaties
%\cite{heuvelkroesschipholbrandcamerabeelden}
%Wat waren de regels destijds?
%Waren de autoriteiten in staat om op tijd in te grijpen of om erger te voorkomen?
%Wat is er gedaan om de veiligheid van illegalen en gevangenissbewaarders te verbeteren
%Wat is er gebeurd?
%\cite{wikiSchipholbrand},\cite{schipholbrand27102005video}
%psychologische gevolgen
%rapport
%\cite{onderzoeksraad2610schipholoost}
%artikel met video
%herdenking
%impact op de persoon
%herdenking
%\cite{schipholbrandvideoargos}
%chronologie
%\cite{nunl30052023feitenoverzicht}
%tijdlijn
%\cite{singeluitgeverijenSchipholbrand}
%vervolgens van ministers
%beeldanalyse en reconstructie
%\cite{eenvandaagschipholbrand}
%herdenking
%korte samenvatting
%rapport
%artikel
%verwijzing naar het rapport vanuit de politieke oppositie
%beeld vanuit de gevangenisbewaarder
%nationaliteit slachtoffers schipholbrand
%verblijfsvergunning voor de slachtoffers
%gen schadevergoeding voor de verdachte
%verdachte voor de rechter
%geen schadevergoeding voor verdachte
%artikel wat ging er mis bji de schipholbrand
%brand veroorzaakt door een peuk
%smaadschrift
%bewakers worden niet vervolgd
%proces schipholbrand moet over en de brandveilgheid moet worden verbeterd
%de rol van het parlement in de evaluatie
%\cite{parlementairemonitorschipholbrand}
%onderzoeksmemo
%herdenking
%herdenking
%invloed van de ramp op samenleving
%\cite{videonpoNOVA13112008}
%opmerkelijk rapport gestolen in de nasleep
%\cite{rizoomes01052014schipholbrand}
%publicaties
%\cite{heuvelkroesschipholbrandcamerabeelden}
%Wat waren de regels destijds?
%Waren de autoriteiten in staat om op tijd in te grijpen of om erger te voorkomen?
%Wat is er gedaan om de veiligheid van illegalen en gevangenissbewaarders te verbeteren
%%%%%%%%%%%%%%%%%%%%%%%%%%%%%%%%%%%%%%%%%%%%%%%%%%%%%%%%%%%%%%%%%
\newline \indent De explosie tanjin china 12/08/2015. 
Op 12 augustus 2015. Er waren twee explosies bij de Rulthai logistiek  faciliteit zorgde voor de opslag vn  gevaarlijke stoffen. De explosie zorgde voor de vernietiging van 12000 voertuigen, schade aan 17000 huize binnen een traal van 1 km. Er waren 173 doden inclusief brandweermensen.
Een van de explosies zorgde voor  een beving van 2.3 op de schaal van rigter.
De volgende factoren zouden een rol hebben gepeeld:
Een onjuiste afbakening van het opslagmaeriaal
Er was  weinig kennis bij de autoriteiten over  opslagmaterialen. Zo bleek er 7000 ton aan materiaal opgeslagen, dat is ruim 70 keer te maximaal toegestande hoeveelheid. 
Onverenigbaar grondgebruik in de nabije omgeving. Veel woonwijken met nar schatting 6000000 bewoners en 500 lokale bedrijvenin de buurt van de opslag gevaarlijke stoffen.
Opgeslagen materialen  waren: calcium carbine, sodium nitraat, potassium nitraat, amminiak nitraat en cyanide.
Ook is er veel kritiek geweest op de acties van de autoriteiten. Zo was er censuur vanuit de overheid op de journalistiek.
Ook was er naar alle warschijnlijkheid sprake van corruptie. Zo bleek achteraf dat een van de grootste aandeelhouders Dong Shexuang de zoon te zijn van een oud-politiechef in Tanjin haven, genaamd Dong Pijun
De overheid beloofde strengere toezicht en alle bedrijven moeten een risico-inventariatie maken en onderhouden\cite{jiang16042019TanjinExplosion},
\cite{staff31082015tanjinblastunrevealed},\cite{chinafile18082015tanjinexplosion},
\cite{pinghuang2410201TanjinFactreport},\cite{portoTanjinExplosionSight},\cite{imago17082015TanjinApartmentImages},\cite{trager14082015Chemicalblast},\cite{pangeramo27082015TanjinExplosion},\cite{ap06082020ammaniumnitrate},
\cite{morris14082015TanjinIndustryImpact},\cite{milesyu20082015exposingtoxicgovlines},\cite{artemis30032016tanjininsurance},\cite{aidenxiatanjinblast},
\cite{danwangTanjinflexreport},\cite{keyHighlightsTanjin},\cite{hartley13082015videofootage},\cite{odonnel01062017firetanjinblast2015},
\cite{fan15082015newyorkermistrustchina},\cite{yanlidongchinamediaframingTanjin},\cite{evans27092017TnjinInsurance},\cite{jasi26032019chineschemplant},\cite{shiqingTanjinExecutiveSentence},\cite{sophiebeach15082015},\cite{hamzeh05082020BeirutBlast},\cite{chemwatch18082015TanjiinExplosion},
\cite{thehindu15062019chinaExplosion},\cite{santagotimes24032019chinablast},
\cite{klingecorp28042020causedTanjin},\cite{mcgarryExplosions2017},\cite{roswnfeld13082015TanjinReports},
\cite{aria12082015explosionaTanjin},\cite{tremblay11022016chineseInvestigatorsTanjin},\cite{taylor13082015TanjinExplosianAftermath},
\cite{associatedPresss13082013},\cite{un20082015InvestigationTanjin},\cite{france2412082015TnjinExplosion},\cite{npr14082015TanjinCause},\cite{bbc05022016TanjinResponsibles},\cite{CBodeen15082015TanjinExplosion},\cite{reutersTanjinInsurance},\cite{yu082016evaluationTanjin2015},\cite{wiki2015TanjinExplosions},\cite{bbc17082015whathappenedTanjin},
\cite{mortimer19082016taijinexplosioncrater},\cite{internationallabourofficeChmControlTooliit},\cite{euTaxationCustomsICSC},
\cite{iloWHOChemSafetyCards}.
Later bleek uit een onderzoek van de Chinese autoriteiten dat de explosie overeenkwam met de ontploffing van 450 ton TNT.[6] 
De oorzaak van de explosie lag in de spontane zelfontbranding van 207 ton cellulosenitraat dat in containers was opgeslagen op het terminalterrein.[6] 
Verder lag op een tweede locatie nog eens 26 ton van dit explosieve materiaal opgeslagen.
De tweede ontploffing werd versterkt door de opslag van 800 ton kunstmest in de vorm van ammoniumnitraat in de nabijheid.[6]
De opslag van cellulosenitraat is aan strenge regels gebonden. Het moet koel en droog worden opgeslagen. De containers stonden buiten opgesteld in de brandende zon. De temperatuur liep op tot 36 °C en bereikte binnen de containers waarschijnlijk de 65 °C.[6] De verpakking van de cellulosenitraat droogde uit waardoor de ontploffing kon ontstaan. Op het terrein lagen meer gevaarlijke stoffen opgeslagen dan waarvoor vergunningen waren verstrekt.[6] Dit leidde tot een kettingreactie met grote schade tot gevolg. Door de brand en bluswater is in de directe omgeving veel milieuschade opgetreden.
https://www.hindawi.com/journals/joph/2019/1360805/ 
\cite{jiang16042019TanjinExplosion}
verhaal van brandweermannen
\cite{staff31082015tanjinblastunrevealed}
artikel
\cite{chinafile18082015tanjinexplosion}
invloed van social media
\cite{pinghuang2410201TanjinFactreport}
gemaakte fouten
\cite{portoTanjinExplosionSight}
\cite{imago17082015TanjinApartmentImages}
\cite{trager14082015Chemicalblast}
\cite{pangeramo27082015TanjinExplosion}
vergelijking met andere explosies
\cite{ap06082020ammaniumnitrate}
invloed van de ramp op de industrie
\cite{morris14082015TanjinIndustryImpact}
is er sprake van een doofpot
\cite{milesyu20082015exposingtoxicgovlines}
eigendomsverzekering
\cite{artemis30032016tanjininsurance}
\cite{aidenxiatanjinblast}
effecten op de lange termijn
\cite{danwangTanjinflexreport}
\cite{keyHighlightsTanjin}
lessons learned
\cite{hartley13082015videofootage}
\cite{odonnel01062017firetanjinblast2015}
gevolgen voor de industrie
\cite{fan15082015newyorkermistrustchina}
framing vanuit de chinese media
\cite{yanlidongchinamediaframingTanjin}
\cite{evans27092017TnjinInsurance}
niewsartikel
\cite{jasi26032019chineschemplant}
\cite{shiqingTanjinExecutiveSentence}
toegang tot de ramplplek vanuit de okale journalistiek
\cite{sophiebeach15082015}
artikel
\cite{hamzeh05082020BeirutBlast}
\cite{chemwatch18082015TanjiinExplosion}
\cite{thehindu15062019chinaExplosion}
\cite{santagotimes24032019chinablast}
oorzaken
\cite{klingecorp28042020causedTanjin}
case study
\cite{mcgarryExplosions2017}
niewsartikel
\cite{roswnfeld13082015TanjinReports}
chronologische uiteenzetting
\cite{aria12082015explosionaTanjin}
corruptie
mismanagement als oorzaak
autoriteiten publiceren onderoeksrapport
\cite{tremblay11022016chineseInvestigatorsTanjin}
fotos van de rampplek
\cite{taylor13082015TanjinExplosianAftermath}
niuwesartiekel
\cite{associatedPresss13082013}
\cite{un20082015InvestigationTanjin}
\cite{france2412082015TnjinExplosion}
\cite{npr14082015TanjinCause}
123 verantwoordelijken
\cite{bbc05022016TanjinResponsibles}
lang artiekel
\cite{CBodeen15082015TanjinExplosion}
\cite{reutersTanjinInsurance}
\cite{yu082016evaluationTanjin2015}
\cite{wiki2015TanjinExplosions}
\cite{bbc17082015whathappenedTanjin}
\cite{mortimer19082016taijinexplosioncrater}
veiigheidshandhaving
\cite{internationallabourofficeChmControlTooliit}
\cite{euTaxationCustomsICSC}
\cite{iloWHOChemSafetyCards}.
%%%%%%%%%%%%%%%%%%%%%%%%%%%%%%%%%%%%%%%%%%%%%%%%%%%%%%%%%%%%%%%%%
\newline \indent  De ethiopian airlinesop 10/03/2019\cite{caliskan09112013747boeingkalman},\cite{gates18112020boeingcrisis},
\cite{boeing737maxsoftwareprobles},\cite{avetisov19032019boeingmalwarestate},\cite{thompson23112020nationalsecurityboeing},
\cite{wiki737maxgroundings},\cite{campbell02052019boengcrashhumanerrors},
De oorzaak is de MCAS
\cite{hawkins22032019737maxairplanes},\cite{barnett05052019737maxcrisis}, \cite{thomas30082020737safest},\cite{boyle18112020737maxupgrade},\cite{bergstraburgess122019737maxMcasAlgorithm},\cite{737mcas},\cite{german190620217372yaftergrounded},\cite{beningo02052019boeinglessons},\cite{bloomberg26092019failedpred},\cite{afacwaLostSafeguards}, als een single point of failure \cite{uran05042019SPOF}
Angle-of-attack\cite{boeing737maxdisplay},
Behalve de MCAS waren er nog andere failures\cite{fehrm24112020737changes}, en ook deze failures \cite{dohertylindeman15032019737problems}
\cite{travis18042019737maxsoftwaredevop},
%\cite{easa27012021737maxsafereturn},
safety record van de boeing
\cite{touitou11032019737tragedies},
 Oplossingen zijn \cite{caa737modifications}. 
 Ethiopian Airlines Flight 302
In maart 2019 stot vlucht ET302 van Ethiopian airlines neer. De oorzaak ligt bij het mcas flight control system. Dit systeem werd geimplementeerd om kosten te reduceren en opleidingen voor piloten  in te korten. De niweueboeing 737 max model veresite test in volledige flight simulators. Nieuwe faa regels vereisten ondersteuning bij het uitvoeren van enkele manouvres. Tijdens testvluchten uitgevoerd binnen een jaar voor certificatie werd het pitch-up fenomeen geconstateerd waarop het mcas systeem werd aangepast. het werd nu getriggerd door een enkele angle-of-attack sensor.
In eht systeem zaten nu 3 fouten.
- MCAS wordt getriggerd woor enkele sensor zonder vertraging
- Het ontwerp staat toe dat  in situaties waar de angle-of-attack fout is de mCAS wordt geactiveerd
- systeem kreeg onnidig bevoegdheid controle om de neus bij te sturen
- Waarschuwingslicht bij fouten in de angle-of-attack werkte niet door  softwarefout. Het werd ook niet kritisch ebvonden door ethiopian airlines. geplande updates door boeing pas in 2020
- een losstande fout in de microprocessor va de controle computer kan vergelijkbare situaties doen voorkomen zonder dat mcas wordt geactiveerd

Fouten vielen niet op omdat FAA test uitbesteedde aan boeing. Contact tussen de organisaties verliep op management niveau. Boeing instrueerde niet alle piloten ovver mCAs. Het werd gezien als een achtergron relief systeem.

% One minute into the flight, the first officer, acting on the instructions of the captain, reported a "flight control" problem to the control tower.
% Two minutes into the flight, the plane's MCAS system activated, pitching the plane into a dive toward the ground. The pilots struggled to control it and managed to prevent the nose from diving further, but the plane continued to lose altitude.
% The MCAS then activated again, dropping the nose even further down. The pilots then flipped a pair of switches to disable the electrical trim tab system, which also disabled the MCAS software. However, in shutting off the electrical trim system, they also shut off their ability to trim the stabilizer into a neutral position with the electrical switch located on their yokes. The only other possible way to move the stabilizer would be by cranking the wheel by hand, but because the stabilizer was located opposite to the elevator, strong aerodynamic forces were pushing on it.
% As the pilots had inadvertently left the engines on full takeoff power, which caused the plane to accelerate at high speed, there was further pressure on the stabilizer. The pilots' attempts to manually crank the stabilizer back into position failed.
% Three minutes into the flight, with the aircraft continuing to lose altitude and accelerating beyond its safety limits, the captain instructed the first officer to request permission from air traffic control to return to the airport. Permission was granted, and the air traffic controllers diverted other approaching flights. Following instructions from air traffic control, they turned the aircraft to the east, and it rolled to the right. The right wing came to point down as the turn steepened.
% At 8:43, having struggled to keep the plane's nose from diving further by manually pulling the yoke, the captain asked the first officer to help him, and turned the electrical trim tab system back on in the hope that it would allow him to put the stabilizer back into neutral trim. However, in turning the trim system back on, he also reactivated the MCAS system, which pushed the nose further down. The captain and first officer attempted to raise the nose by manually pulling their yokes, but the aircraft continued to plunge toward the ground.
 \cite{caliskan09112013747boeingkalman}
 \cite{gates18112020boeingcrisis}
 \cite{boeing737maxsoftwareprobles}
 \cite{avetisov19032019boeingmalwarestate}
 \cite{thompson23112020nationalsecurityboeing}
 \cite{gates21032019FAAControlSystem}
 \cite{faa18112020boeingreview}
 \cite{wiki737maxgroundings}
 \cite{campbell02052019boengcrashhumanerrors}
 \cite{hawkins22032019737maxairplanes}
 \cite{thomas30082020737safest}
 \cite{boeing737maxdisplay}
 \cite{fehrm24112020737changes}
 \cite{travis18042019737maxsoftwaredevop}
 \cite{barnett05052019737maxcrisis}
 \cite{easa27012021737maxsafereturn}
 \cite{touitou11032019737tragedies}
 \cite{hemmerdinger02022021737maxdeliveries}
 \cite{bielby27022021faaimprovesafety}
 \cite{boyle18112020737maxupgrade}
 \cite{bergstraburgess122019737maxMcasAlgorithm}
 \cite{737mcas}
 \cite{newburger17052019boeingcrisis}
 \cite{arstechnica22012020737problems}
 \cite{german190620217372yaftergrounded}
 \cite{beningo02052019boeinglessons}
 \cite{duran05042019boeingspof}
 \cite{makichuck24012021737fearflying}
 \cite{caa737modifications}
 \cite{oestergaard14122020boeingdeliveries}
 \cite{reenberg787flaws}
 \cite{fitch16092020737backlogrisks}
 \cite{willis27082020737maxfailures}
 \cite{ostrower11062020more737changes}
 \cite{hruska13122019faaknown737crashrate}
 \cite{bloomberg26092019failedpred}
 \cite{whiteman09072020boengcancelstock}
 \cite{leopold09192019boeingreliability}
 \cite{koenig11122019737crashesnofix}
 \cite{dohertylindeman15032019737problems}
 \cite{stodder02102019corruptoversight}
 \cite{afacwaLostSafeguards}
 \cite{swayne18032019profitssafety}
 \cite{freed26022021liftaustraliaban}
 \cite{reed15032019softwareattention}
 \cite{news17032019softwareexplains}
 \cite{legget21122020eu737maxsafe}
 \cite{marketscreener0103221737chinarecertification}
 \cite{euractiv22022021737firegrounds}
 \cite{benny18022019737returnUAE}
 \cite{biersmichel22022021777grounds}
 \cite{reuters23022021777metalfatigue}
%%%%%%%%%%%%%%%%%%%%%%%%%%%%%%%%%%%%%%%%%%%%%%%%%%%%%%%%%%%%%%%%%
\newline \indent Het mortierongeluk in Mali op 06/04/2016. Aanwezige militair brengt slachtoffer naar de fransen, vervolgens naar de Tongolezen. Maar de kwaliteit van personeel liet te wensen over.
Er werd een Nederlandse arts overgevlogen. De slachtoffers werden overgevlogen naar Gao omvervolgens te worden oergevolgen naar Nederland.
Het ongeluk werd veroorzaakt door een kapot afsluitplaatje in de mortier. De granaat opslag in een niet gekoelde container. Dan was er vocht in de fatale granaat. Zodoende werden er explosieve stoffen gevormd in de granaat.
Tijdens de oefening werden de granaten warm in de zon. De granaat stond in veilie stand kon de explosie niet voorkomen.
granaat stond niet op scherp en in afgegaan in veilige stand
Granaat werd opgeslagen in neit gekoelde containers waardoor deze aan te hoge temeperaturen zijn blootgesteld.
Door de comvinatie van vocht en warmte in de granaat zeer gevoelige explosieve stoffen werden gevormd.
Tijdens de oefening was de fatale granaat in de zon.
Het afsluitplaatje in de granaat bleek niet in staat om doorslag in veilige stand te voorkomen waarna de granaat explodeerde.
De moritren zijn aangeschaft bij de amerikanen. gredurende de aanschafperiode zijn procedures en controles op kwaliteit en veiligheid deels nagelaten.
Dit veiligheidsgarantie werd vermeld in het koopcontract.
Conclusie
Koopcontract werd niet goed doorgelezen
Geen controle op kwaliteit en veiligheid
Geen controle op kwaliteit en veiligheid
Zwakke plekken in het ontwerp
Geen controle op kwaliteit en veiligheid
opslag en gebruik in ongunstige condities
De aanwezige medische voorzieningen waren nite volgends de nederlandse militaire richtlijnen
Het ontbreek aan medische toetsing vanuit de defensie organisatie
twijfels die werden geuit binnen de defensieorganisae vonden geen wrrklank
Ok het ongeval tijdens de mortieroefening was voor defensie geen aanleuiding om de medische voorzienignen te evalueren.
De inrichting van veilige medische zorg voor nederlandse militairen in kidal is ondergeschikt gemaakt aan de voortgang van de missie.
\cite{ovvMortierOngevalMaliVideo} 
\cite{bnnvara13062018malirapport}
\cite{eucal11012021malimissieverlengd}
\cite{nos21052014zorgenmalimissie}
\cite{meijnders}
\cite{bnrwebredactie}
\cite{keultjes01062016malimissiecoalitie}
\cite{veenhof18012019}
\cite{isitman06012016militair}
\cite{nporadio11072016filmdemissie}
\cite{parlementairmonitor15122013mortierongeluk}
%%%%%%%%%%%%%%%%%%%%%%%%%%%%%%%%%%%%%%%%%%%%%%%%%%%%%%%%%%%%%%%%%
\newline \indent De ramp tjernobyl 26/04/1986. \cite{INSAVienna1992Chernobyl}
De mislukte veiligheidscontrole op 26 apeil 1986 01.24 uurin de sovjetuni leiddte tot explosies in een van de reactoren in de kerncentrale. De reactoren hadden geen veiligheidomhulling en de reactor bevat grote hoeveelheden brandbaar grafiet.
Door de explosie en de brand kwamen er radioactieve stoffen vrij.het gaat helemaal mis in de kernreactor 4. De warmteproductie nam  toe met een explosie tot gevolg.
31 mensen kwamen om, waaron veel mensen dagen later door stralingsziekte.
 Op 26 april 1986. Techici bij kerncentrale 4 voerden een slcht opgezet/ ontwerpen experiment uit. De  kracht regulering werd uitgeschakeld evenals veiligheidssystemen. 
Een ramp bij een kernreacor in de sovjetunie. Door een bedieningsfout in een testprocedure werd het vermogen van de koelinstallaties negatief beinvloed. Door een ontwerpfout in de noodstopprocedure kon in het systeem niet snel genoeg schakelen om remmende invloed uit te oefenen op het toenemende vermogen van de reactorkernen. Met brand en eksplosie tot gevolg.
\cite{INSAVienna1992Chernobyl}
Tsjernobyl
\cite{wikiTjernobyl}
\cite{rivmTjernobyl}
\cite{andereTijdenTjernobyl}
wat er is gebeurd en hoe het leven verdergaat
\cite{kingskey19042022tjernobyl}
\cite{erikbork26042023reactor4}
\cite{nosTjernobyl30jaarlater}
%Dieren in de omgeving van tjernobyl
%De chronologie
%Echtreme droogte zorgd voor gevaar
%\cite{knmi04052021tjernobylbosbrand}
%\cite{dodonovaKVIRisicoTjernobyl}
%Joernalistiek, entertainment en de waarheid
%\cite{dumarey04062020verhaalTjernobylWaarheid}
%Een onderzoek
%Huidige gevolgen van de explosie van toen
%\cite{sparkesNewScientistTjernoby}
%De ramp, hoe de mensen ermee omgingen en hoe er nu geleef wordt
%evaluatieonderzoek en amatregeen
%\cite{kernenergiened26041986chronologiemaatregelen}
%\cite{mapszoneReactor}
%Invloed van de mens op de omgeving
%Heroplevende splijtingsreacties
%docu van schooltv
%Radioactiviteit bereikt nederland
%documentaire en maatregelen
%\cite{kernhistoriek15062021tjernobyl}
%Het verhaal van een overledende
%Toerisme
%toerisme
%toerisme
%Dieren in de omgevong
%Toevluchtsoord voor vluchtelingen van de oorlog met russische seperatisten
%Ouderen die terugkeerden naar hun woonplaats na de gedwongen verhuizing door de autoriteiten
%De straling neemt weer toe
%Lessen geleerd van tjernobyl
%\cite{nucleairforumFeitenTjernobyl}
%Toerisme
%Bosbrand in tjernobyl
%invloed van de ramp op belgie
%\cite{kernongevalTjernobylFancGov}
%Boek recensie
%Fotos en berekeningen
%ontmanteling en toerisme
%Belangrijke lessen en overeenkomsten
%De journalistieke waarheid van de koude oorlog
%De lessen van
%\cite{arendswolters062019lessenTjernobyl}
%Een toristenattractie maken van tjernobyl
%De radioactieve straling toen en nu
%de 30km zone door de ogen van toeristen
%artikel
%stedentrip
%rapport
%\cite{damveld08052020tjernobyl}
%slapend monster
%docu
%krantenartikel
%hbo serie
%docuserie
%de  nieuwe sacrofaag
%hulp aan slachtoffers
%slapende reactor
%krantenartikel
%\cite{deVriestjernobylHolland}
%hbo serie
%internationale gevolgen
%toerisme
%nieuwe koepel
%media communicatie
%docu
%dieren
%koepel
%koepel
%\cite{ing3enieur29042015antistralingskoepel}
%toerisme
%toeristisch reiperspectief
%toerisme
%niwe koepel
%overschakelen naar duurzaamheid
%docu
%tjernobyl wekt nu duurazme energie
%toerisme
%overeenkomsten tjernobyl en fukushima
%drank en sla uit tjernobyl
%geen efficiente opslag is mogelijk
%wetenschappelijke artikelen
%zaterdag 26 april 1986. Er vind routineonderhoud plaats bij reactor 4, De controle wordt uitegevoerd door de dagploeg. Vnwege een test wordt jhet koelsysteem uitgeschakeld. Door omstandigheden wordt de test uitgesteld en wordt de verantwoordelijkheid overgedragen aan de avondploeg.
%De operator maakt bedieningsfouten waardoot de reactor bijna stil komt te liggen. En vervolgens probeert hij de reactor weer op gang te brengen. ondanks de snelle temperatuurstijging wordt het experiment doorgezet. Dan wordt ook het veiligheidssysteem stilgelgd. Terwijl het koelwater langzaam opwarmt, sluit hij de klep waarlangs de stoom naar de generator stroomt.
%De temperatuur van de reactorstaven neemt daarna snel toe. Terwijl er een oncontroleerbare kettingreactie op gang komt, laat het personeel in paniek de regelstaven zakken om de warmteontwikkeling af te remmen. Het is dan echter al te laat. Door een ontwerpfout loopt het vermogen razendsnel op tot 33.000 megawatt, ruim tien keer hoger dan normaal.
%In een oogwenk verandert al het koelwater in stoom. De ontploffing die daarop volgt, blaast het 2000 ton zware deksel van de reactor af.
%In de ravage vat het gloeiend hete grafiet in de reactor spontaan vlam. De uitslaande brand en een tweede explosie voeren een radioactieve rookwolk tot 8 kilometer hoogte. 
%In een poging het vuur in reactor 4 te doven, storten helikopters vanuit de lucht zand, lood en boorzuur in de reactorkern. Het mag echter niet baten.
%Intussen is de nucleaire brandstof zo heet geworden dat die door de bodem van het reactorvat dreigt te smelten. Als dat gebeurt, kan het bluswater onder het vat in één klap verdampen en dreigt een derde explosie die een groot deel van Europa onbewoonbaar zal maken. Om dit te voorkomen moet het water hoe dan ook worden weggepompt.
%Drie brandweermannen wagen zich daarvoor in de ruimte onder de reactor, blootgesteld aan 300 sievert per uur, 300.000 keer de dosis die een Nederlander jaarlijks maximaal mag oplopen. Ze slagen daarin, maar twee van hen overlijden enkele dagen later aan acute stralingsziekte.
%Hoewel geigertellers de dag na de ramp onrustbarende waarden aangeven, slaat het plaatselijk bestuur geen alarm. De bevolking is het niet gewend om vragen te stellen.
%De volgende dag blijkt er wel degelijk iets ernstigs aan de hand te zijn. In een lange rij bussen worden de 135.000 inwoners op 27 april uit het besmette gebied geëvacueerd, om er nooit meer terug te keren.
%De ramp is dan nog steeds geen wereldnieuws. De Sovjetautoriteiten blijken er niet eens van op de hoogte te zijn – president Gorbatsjov klaagt later dat hij via Zweden aan zijn informatie moest komen.
\cite{verschuur14012013tjernobylreports}
\cite{paperlessarchivesTjernobyl}
\cite{vargos082000tjernobylconcerns}
\cite{mauroNuclearRiskSociety}
\cite{vienna06092005LookingBack}
\cite{wikiTjernobyl},
\cite{rivmTjernobyl},
\cite{andereTijdenTjernobyl},
\cite{kingskey19042022tjernobyl},
\cite{erikbork26042023reactor4},
\cite{nosTjernobyl30jaarlater},
\cite{knmi04052021tjernobylbosbrand},
\cite{dodonovaKVIRisicoTjernobyl},
\cite{dumarey04062020verhaalTjernobylWaarheid},
\cite{sparkesNewScientistTjernoby},
\cite{kernenergiened26041986chronologiemaatregelen},
\cite{mapszoneReactor},
\cite{kernhistoriek15062021tjernobyl},
\cite{nucleairforumFeitenTjernobyl},
\cite{kernongevalTjernobylFancGov},
\cite{arendswolters062019lessenTjernobyl},\cite{damveld08052020tjernobyl},
\cite{deVriestjernobylHolland},\cite{ing3enieur29042015antistralingskoepel},
\cite{verschuur14012013tjernobylreports},\cite{paperlessarchivesTjernobyl},\cite{vargos082000tjernobylconcerns},\cite{mauroNuclearRiskSociety},\cite{vienna06092005LookingBack}
%%%%%%%%%%%%%%%%%%%%%%%%%%%%%%%%%%%%%%%%%%%%%%%%%%%%%%%%%%%%%%%%%
\newline \indent  De digitale aanval op de Oekrainese krachtcentrale op 23,december 2015
Op 23,december 2015  vind er een cyber aanval plaats op het elektriciteitsnet van de Oekraine. Dit was de eerste bekende aanval op een elektrisch contole  system.  Dit verslag geeft inzage in een analyse van de Ukraine cyber aanval,
inclusief hoe de actoren zich zelf toegang gavan tot het controle systeem, welke methoden de acoren hebben gebruikt voor reconnaissance en vastleggen van het systeem, een gedetailleerde omshrijving van de aanval op 15 December 2015, en de methoden die gebruikt zijn door de aanvallers om hun sporen uit te wissen en daarmee het het stoppen van schade toebrengen  nog moeilker maken. Daarnaast wordter  een gedetailleerde omschrijving gevevenv an de beveiliging van de SCADA ccontrol systemen gebaeerd op bst practices, inclusief het control network ontwerp, technieken voor whtelisting, monitoring en loggen, en  opleiding van personeel.
\cite{Whitehead2017ukrainepoweroutage}
\cite{noauthor_2022-nm}
\cite{zetter2016GridHack}
\cite{owens21032017ukrainemitigationstrategies}
\cite{cerulus2019FrontlineRussiaAttack}
\cite{grammatikis2019AttackIEC6087505104}
\cite{hidajat2016ScadaSimulator}
\cite{uscert20072021crashmalware}
\cite{zetter12062017malwareanalysis}
\cite{icsRussianHackingCyberWeapon}
\cite{usgovC2M2}
Dit verslag geeft inzage in een analyse van de Ukraine cyber aanval,
inclusief hoe de actoren zich zelf toegang gavan tot het controle systeem, welke methoden de acoren hebben gebruikt voor reconnaissance en vastleggen van het systeem, een gedetailleerde omshrijving van de aanval op 15 December 2015, en de methoden die gebruikt zijn door de aanvallers om hun sporen uit te wissen en daarmee het het stoppen van schade toebrengen  nog moeilker maken. Daarnaast wordter  een gedetailleerde omschrijving gevevenv an de beveiliging van de SCADA ccontrol systemen gebaeerd op bst practices, inclusief het control network ontwerp, technieken voor whtelisting, monitoring en loggen, en  opleiding van personeel.
\cite{Whitehead2017ukrainepoweroutage},\cite{zetter2016GridHack},\cite{boozallen2016lightwentout},\cite{finklejan2016UsBlamesRussianSandworm},\cite{desarnaud2017cyberattacks},\cite{caseli04112016intrusiondetectioncontrolsystem},\cite{rochascadatesting},\cite{hidajat2016ScadaSimulator},\cite{zetter2017moreDangerousMalware}.
Oop 23,december 2015  vind er een cyber aanval plaats op het elektriciteitsnet van de Oekraine. Dit was de eerste bekende aanval op een elektrisch controle  system met corrupte firmware. Daarnaas wordt er een telecom-based denial of service attack met  geautomatieerde systemen om het telefoonverkeer uit te schakelen.
\cite{Whitehead2017ukrainepoweroutage}
Uit onderzoek\cite{zetter2016GridHack} naar de aanval,  uitgevoerd door Oekraiene sen Amerikaanse militairenblijkt  bleek onder meer dat de power grids in sommige gevallen beter waren beveiligd dan de Amerikaanse. Desondanks was de viligheid niet optimaal door onder andere de  hetgegeven dat werknemers op afstand konden inloggen en geen gebruik van 2-stapsverificatie.
Oekraine wijst naar de russen \cite{zetter2016GridHack}, 
\cite{greenberg2017Cyberwartestlab},
\cite{boozallen2016lightwentout},
\cite{finkle08012016russiansandwormhackers},
\cite{zinets15022017ukrainechargesrussia},
\cite{mcelfresh2016cyberattackhowandwhy},
\cite{parkwalstorm11102017russiagridattack}.
{Situatie Oekraiene}
\cite{drago2017CrashOverride},
\cite{slowik2019ReassasUkraine2016Attack}.
{Situatie algemeen}
\cite{cerulus2019FrontlineRussiaAttack},
\cite{desarnaud2017cyberattacks},
\cite{dragos2019TargetedTransStation}.
{Factoren}
\cite{shehod2016gridadvantageus}
{Oorzaak}
\cite{rocha2017cybersecyrityanalysisScada},
\cite{2017crashoverridenostuxnet},
\cite{vijayan2017firstmalwareCausedOutage},
\cite{slowik2019ReassasUkraine2016Attack}.
{Gebruikte materialen}
\cite{2015ukrainegridattack},
\cite{industroyershortfact}
{Uitvoering van de aanval}
\cite{Whitehead2017ukrainepoweroutage},
\cite{boozallen2016lightwentout}.
{Oplossingen}
~\cite{Whitehead2017ukrainepoweroutage}
\cite{Whitehead2017ukrainepoweroutage}
\cite{boozallen2016lightwentout}
{spearfishing}
{blackenergy}
{remote access capabilities}
{serial-to-ethernet communication devices}
{telephony denial of service attacks}
{oplossingen}
Identificeer alle risicos en schrijf een plan foor het managen van de risico's.
Implementeer  effecteve controle  om het riico te managen.
Creeer een diepgaand model dat ervoor zor dat er efectieve en efficiente security controls worden uitgevoerd.
Aangaande de gebeurtenissen in de oekraiene kunnen de volgende security controls worden opgenomen in het securitymodel: Initial access to enterprise network, pivot in interprise network, elevate priviliges, maintainance access, gain access to control system, attack, attack complication, destroy hard drives.
\cite{Whitehead2017ukrainepoweroutage}
{Discussie}
{Verder lezen}
\cite{shahzad2014ScadaProtocolsPollingScenario},
\cite{grammatikis2019AttackIEC6087505104},
\cite{2017win32industroyer},
\cite{yadav2020reviewScadaArchitecture},
\cite{arrizabalaga2020surveyiiotProtocols},\cite{fauri2017EncryptionICS},\cite{resch31102019IEC62351secureCommunication},\cite{levalle2020FuzzingICSProtocols},\cite{blackhatusa2017},\cite{blackhatusa2017},\cite{abb30062017crashoverridenotification},\cite{spinner2018crashoverrideiot},\cite{njccicthreat08102017crashovverrideprofile},\cite{slowikvb2018crashoverride},\cite{crashoverridenetwork},\cite{wikiindustroyer},\cite{icsSecurityRussianHacking},\cite{holappa2017threattoElectricityNetworks}.
%%%%%%%%%%%%%%%%%%%%%%%%%%%%%%%%%%%%%%%%%%%%%%%%%%%%%%%%%%%%%%%%%

%%%%%%%%%%%%%%%%%%%%%%%%%%%%%%%%%%%%%%%%%%%%%%%%%%%%%%%%%%%%%%%%%
\newline \indent explosie in libabon, beirut 
Op 23 september 2013 voer het vrachtschip de Rhosus onder Moldavische vlag[7] van Batoemi in Georgië naar Beira in Mozambique met 2.750 ton ammoniumnitraat
Gezien het ernstige gevaar van het bewaren van deze goederen in de hangar onder ongeschikte klimatologische omstandigheden, herhalen we ons verzoek aan de marine-instantie om deze goederen onmiddellijk weer te exporteren om de veiligheid van de haven en de mensen die er werken te verzekeren, of om akkoord te gaan om ze te verkopen.
Voorafgaand aan de explosie was er een brand in een opslagplaats. 
\cite{hrw03082021investigateBeirutBlast}
\cite{souaibyElHussein112020Beirutstory}
\cite{ifrc2020chemicalexplosionBeirutPort}
%%%%%%%%%%%%%%%%%%%%%%%%%%%%%%%%%%%%%%%%%%%%%%%%%%%%%%%%%%%%%%%%%
\newline \indent  stint ongeluk
Vier kinderen, een bestuurder kwamen om en een vijfde persoon , een kind raakte zwaargewond. Uit odnerzoek van bleek :
Foute torsieveer voor de gashendel werd geleverd
Geen van de drie onderzochte voertuigen haalden de wettelijk vereiste remvertraging
De automatische parkeerrem kan leiden tot gevaarlijke situaties wanneer deze ongewenst geactiveerd wordt tijdens het rijden. 
Het losraken van de nuldraad naar de gashendel leidt volgens TNO tot ongewenst versnellen van het voertuig en een oncontroleerbare situatie voor de bestuurder.
Voor alle drie onderzochte voertuigen geldt dat het ontbreken van een zitplaats leidt tot veiligheidsrisico’s voor remmen en sturen door de grotere kans dat de bestuurder van het voertuig valt. Als de bestuurder van een Stint valt, leidt dit in alle rijsituaties tot een onbeheersbare situatie
\cite{TNOStint}
%%%%%%%%%%%%%%%%%%%%%%%%%%%%%%%%%%%%%%%%%%%%%%%%%%%%%%%%%%%%%%%%%
\newline \indent vuurwerkramp in enschede 
\cite{boogers092002RampenRegelsRichtlijnen}
Wat waren de afspraken omtrent vuurwerkopslag?
Waarom werden de voorschriften neit nageleefd?
%%%%%%%%%%%%%%%%%%%%%%%%%%%%%%%%%%%%%%%%%%%%%%%%%%%%%%%%%%%%%%%%%
\newline \indent  ecourt in nederlandse rechtspraak
niet odnerzocht

\cite{sprongken19032018CourtProcedureDigital}

\cite{PROCESREGLEMENTEcourt}
%%%%%%%%%%%%%%%%%%%%%%%%%%%%%%%%%%%%%%%%%%%%%%%%%%%%%%%%%%%%%%%%%
\newline \indent  molukse treinkaping 

\cite{molukseTreinkaping}
%%%%%%%%%%%%%%%%%%%%%%%%%%%%%%%%%%%%%%%%%%%%%%%%%%%%%%%%%%%%%%%%%
\newline \indent Ramp schietpartij militair ossendrecht. Een militaire overleed op een schietbaan in ossendracht door onvoldoende begeleiding van cursisten, geen toezicht op de lokatie. E\r was een instructuur in opleiding die niet volledig was mmeegenomen in het poroces en ook was er geen baancommandant aanwezig. Geen van de aanwezig instructeurts had de juiste papieren om de cursisten te begeleiden. De aanwezig instruceur had geen zich op de instructeur in opleiding, evenmin de andere militairen. In de instructiehandleiding ontbreken richtlijnen voor bijzondere schietbanen. Ook was er geen keuring. Door personelstekort is er geen andacht besteed aan documentastie(een slyllabus) hoe en met welke risico’s oefeningnen moeten worden ingericht. Ok werd er vooraf geen veiliheidsanaklyse gedaan. Het gebrek aan lesmateriaal en deskundigen is gemeld binnen de defensieorganisatie maar dit heeft niet geleid tot enige verandering in de situatie.
Op een afgekeurde scheitbaan
Tezicht door een instructeur in opleiding die zelf geen persoonlijke begeleiding heeft gehad tijdens de uitvoering
Belangrijk is dat defensie haar taken kan uitvoeren met personeel dat is getraind in situaties die de risicos van de werkomgeving aan de cursisten kunnen laten zien.
Conclusie
Zonder gekwalificeerde instructuers.
Zonder toezicht
Zonder lesmateriaal
Zonder adequate veiligheidsanalyse
\cite{ovvVideoOssendrecht}
\cite{oVVSchietongevalOssendrecht}
\cite{nos22032016ossendrecht}
\cite{ovv04042016lessenongevalossendrecht}
\cite{quekelboere10052017doodossendrecht}

%%%%%%%%%%%%%%%%%%%%%%%%%%%%%%%%%%%%%%%%%%%%%%%%%%%%%%%%%%%%%%%%%
\newline \indent Dan zijn er nog andere ongelukken met de stint, de shietpartij op militairencomplex in ossendrecht, stint-ongeluk, de enschedese vuurwerkramp en de molukse treinkaping. Meer recentelijk de coronacrisis.
%%%%%%%%%%%%%%%%%%%%%%%%%%%%%%%%%%%%%%%%%%%%%%%%%%%%%%%%%%%%%%%%%


% 
% 
%\paragraph{ethiek}
%
%
%Ethiek 
%
%
%
%persuasive technology 
%https://www.humanetech.com/youth/persuasive-technology 
%\cite{humanTechpersuasiveTech}
%https://www.minddistrict.com/blog/persuasive-technology-new-insights-in-behavioural-change 
%https://www.sciencedirect.com/book/9781558606432/persuasive-technology 
%https://spectrum.ieee.org/how-persuasive-technology-can-change-your-habits 
%\cite{rezenfeld01012018persuasiveTecgHabits}
%https://www.frontiersin.org/articles/10.3389/frai.2020.00007/full 
%\cite{aldenaini28042020persuasiveTechTrends}
%https://psmag.com/environment/captology-fogg-invisible-manipulative-power-persuasive-technology-81301 
%\cite{larson14062017persuasivetechmanipulates}
%https://www.makeuseof.com/what-is-persuasive-technology/ 
%\cite{tanzem22012022persuasivetechchanginglives}
%https://lib.ugent.be/catalog/rug01:001235489 
%https://cyberpsychology.eu/article/view/12270 
%\cite{tikkakuddonenpersuasiveTechnology}
%

%
%
%\paragraph{Ondeerzoeksresultaten naar sluisbeveiliging}
%
%
%
%Verouderde computersystemen zijn door de jaren heen gekoppeld aan netwerken, zodat ze op afstand te besturen zijn. Dit zorgt ervoor dat systemen kwetsbaar zijn voor aanvallen van buitenaf. De beveiliging is in de loop der jaren niet voldoende ontwikkeld om de infrastructuur goed te beveiligen.
%
%Volgens het onderzoek is er de afgelopen jaren wel het nodige geïnvesteerd om de beveiliging op te schroeven, maar deze maatregelen zijn nog onvoldoende doorgevoerd.
%https://www.nu.nl/internet/5814282/rekenkamer-waterwerken-niet-goed-beveiligd-tegen-cyberaanvallen.html
%\cite{hdsr30092022lichtprojectieswaterliniesluizen}
%rapport Digitale dijkverzwaring: cybersecurity en vitale waterwerken 
%Crisisdocumentatie is verouderd en er worden geen volwaardige pentesten uitgevoerd. Uit het onderzoek blijkt dat nog niet alle vitale waterwerken rechtstreeks zijn aangesloten op het Security Operations Center (SOC) van Rijkswaterstaat. Hierdoor bestaat het risico dat RWS een cyberaanval niet of te laat detecteert. De minister van Infrastructuur en Waterstaat moet nog stappen zetten om aan de eigen doelstellingen voor cybersecurity te voldoen
%De Algemene Rekenkamer beveelt de minister van Infrastructuur en Waterstaat ook aan om het actuele dreigingsniveau te onderzoeken en te besluiten of extra mensen en middelen nodig zijn. Ook is het voor een snelle en adequate reactie op een crisissituatie van essentieel belang dat informatie up-to-date is. Pentesten zouden integraal onderdeel uit moeten maken van de cybersecuritymaatregelen bij vitale waterwerken. Verder zou moeten worden bezien of medewerkers van het SOC beter moeten worden gescreend.
%
% 
%
%
%\cite{thkwaterwerken}
%Het crisismodel kan beter, is de derde deelconclusie van de Algemene Rekenkamer. Er is geen specifiek scenario voor een crisis die wordt veroorzaakt door een cyberaanval. Ook ontbreekt inzicht in de effecten van een cybercrisis op andere sectoren, de zogeheten cascade-effecten. Tevens is de crisisdocumentatie op onderdelen verouderd.
%
%\cite{rekenkamercybersecWater}
%Ook maakt cyberveiligheid nog geen volwaardig onderdeel uit van reguliere inspecties.’ De Rekenkamer hamert erop dat alle vitale waterinfrastructuur zo snel mogelijk op het SOC wordt aangesloten. Ook zouden werknemers van Rijkswaterstaat die belangrijke waterkeringen bedienen beter gescreend moeten worden op hun antecedenten. Sollicitanten hoeven nu slechts een Verklaring Omtrent Gedrag te overleggen, maar dat is een heel lichte toets.
%
%\cite{hackerWaterwerk}
%deltawerken
%
%\cite{kramerZeeland}
%Volgens Rijkswaterstaat is het kostbaar en technisch uitdagend om klassieke automatiseringssystemen te moderniseren en wordt er daarom vooral ingezet op detectie van aanvallen en een adequate reactie daarop.
%Uit het onderzoek blijkt dat Rijkswaterstaat de afgelopen jaren zelf van alle tunnels, bruggen, sluizen et cetera heeft vastgesteld welke cyberveiligheidsmaatregelen moeten worden genomen. Een groot deel van die maatregelen (ongeveer 60\%) was begin 2018 ook al uitgevoerd, maar Rijkswaterstaat ziet onvoldoende toe op de uitvoering van het resterend deel en heeft geen actueel overzicht van de overgebleven maatregelen.
%De minister heeft een aantal waterwerken die Rijkswaterstaat beheert als vitaal aangewezen. . Uit het onderzoek blijkt dat nog niet alle vitale waterwerken rechtstreeks zijn aangesloten op het Security Operations Center (SOC) van Rijkswaterstaat. De ambitie om eind 2017 bij alle vitale waterwerken cyberaanvallen direct te kunnen detecteren was in het najaar van 2018 daarmee nog niet gerealiseerd. Hierdoor bestaat het risico dat RWS een cyberaanval niet of te laat detecteert.
%
%\cite{cybersecWaterwerk}
%Over de cyberbeveiliging van gemeenten en waterschappen wordt al langer geklaagd. Zo meldde EenVandaag al in 2012 dat rioolgemalen en sluizen gemakkelijk van afstand te bedienen waren, onder meer door bijzonder slechte wachtwoorden.
%
%\cite{cybersecWaterschappen}
%Rittal doet onderzoek naarop afstand besdienbare sluizen
%
%\cite{cybersecZuidHolland}
%Beveiligde VPN
%M2M Services levert aan inmiddels 220 gemeenten en waterschappen beveiligde connectiviteitsoplossingen voor het beheer van pompen, riolen en gemalen. Om risico’s op beveiligingsincidenten te voorkomen maken wij gebruik van een VPN oplossing, waarbij de verbinding optimaal beveiligd is middels encryptie en authenticatie.
%
%\cite{waterwerkNED}
%Veiligheid op het water én op het land
%Gebruik van lampbewaking 
%
%\cite{veiligheidwaterland} 

%%%%%%%%%%%%%%%%%%%%%%%%%%%%%%%%%%%%%%%%%%%%%%%%%%%%%%%%%%%%%%%%%




\paragraph{Safety critical systems}

\cite{winceckCriticalToSafety}
\cite{chambersHazardAnalysisSCS}
\cite{rslater1998SCSAnalysis}
Traditional Systems
Traditional areas that have been considered the home of safetycritical systems include medical care, commercial aircraft, nuclear
power, and weapons. Failure in these areas can quickly lead to
human life being put in danger, loss of equipment, and so on.

Non-traditional Systems
Emergency 911 service is an example of a critical infrastructure
application. Other examples are transportation control, banking
and financial systems, electricity generation and distribution, telecommunications, and the management of water systems

4.1 Technology

https://users.encs.concordia.ca/~ymzhang/courses/reliability/ICSE02Knight.pdf
\cite{knightchallengessafetyCritical}

https://www.dcs.gla.ac.uk/~johnson/teaching/safety/slides/pt2.pdf
\cite{johnson2006devsafetycritical}
\cite{daucriticalsafetyconsider}
\cite{fallsafedesign}
\cite{arForce2015VerificationExpectations}
\cite{nebulaassessment}
\cite{lalaArchitecturalPrinciples}
\cite{mitNotesSafetyCritical}
\cite{britishColumbia2020GuideSafetyCritical}
1.       The Assembly is aware that the use of computers in safety-related applications is growing, particularly in areas such as control systems of aeroplanes, high-speed trains and nuclear power stations, medical equipment and medical records, anti-lock braking systems for vehicles and machine engineering in general, and last but not least, modern weapons and their guidance systems.

2.       Many recent accidents (for example, plane crashes due to computer failure, malfunctioning robot killing a mechanic, patient dying because of malfunctioning of computer-controlled intravenous drip, rocket launch failure traced to computer error, software piracy etc.) cause public concern and raise the question of the reliability of such systems.


How has the problem of safety-critical software arisen? Essentially from an ever-increasing complexity in engineering. One may compare the steam locomotive of 1830 with the APOLLO Moon spacecraft of 1970 as an example. In 1917 WM FARREN designed, supervised the construction of and testflew an aircraft - the CE 1 and with acceptable safety! [2]. Even in 1965 a chief designer would be familiar with all the decisions taken in the design of a complex product such as an aircraft or ship. The management operation was deeply hierarchical [3] , but as systems became more complex and design teams included more and more specialists it became necessary to formalise the interfaces between the specialist groups to gain benefit and yet maintain overall design disciplines. This led to the matrix design management system in the 1970s to cope with design teams 50 times larger than before [4].

A difficulty embodied in tackling the safety related to software in engineered products arises because of software complexity and the mathematical rigour of some parts of it distorts and clouds the fundamental processes of creative engineering design. 

Before discussing safety definitions and integrity a brief mention of design techniques to enhance safety. One way of increasing safety is to develop more reliable components and systems. At the outset, once the general preliminary design is defined there will be a "safety budget" allocating tolerable levels of integrity for every subsystem. Then Reliability Analysis evaluates the probability of failure and Failure Mode Effect and Criticality Analysis deals with the likely results of failure. Once the "life" of a part has been measured then the inspection and maintenance function will act to replace the part with a new one in good time. Another technique is to design an item to "fail-safe" i.e. even if it does fail it does not create a safety risk before the fault can be rectified. This has been extensively used on structures and coping with the development of fatigue cracks. "Fail- operate", "fault tolerant design" and "graceful degradation of systems" are other methods.


\cite{fulvio1993safetycriticalsystems}
\cite{dlrtabid}
\cite{knight2010SafetyCritical}
\cite{creavisafecritical}
\cite{valdes2018SafetybyAutomation}

https://verticalmag.com/features/whensafetymanagementsystemsfail/
\cite{2015whensafetymanagementsystemsfail}
\paragraph{Ondeerzoeksresultaten naar sluisbeveiliging}



Verouderde computersystemen zijn door de jaren heen gekoppeld aan netwerken, zodat ze op afstand te besturen zijn. Dit zorgt ervoor dat systemen kwetsbaar zijn voor aanvallen van buitenaf. De beveiliging is in de loop der jaren niet voldoende ontwikkeld om de infrastructuur goed te beveiligen.

Volgens het onderzoek is er de afgelopen jaren wel het nodige geïnvesteerd om de beveiliging op te schroeven, maar deze maatregelen zijn nog onvoldoende doorgevoerd.

\cite{hdsr30092022lichtprojectieswaterliniesluizen}
rapport Digitale dijkverzwaring: cybersecurity en vitale waterwerken 
Crisisdocumentatie is verouderd en er worden geen volwaardige pentesten uitgevoerd. Uit het onderzoek blijkt dat nog niet alle vitale waterwerken rechtstreeks zijn aangesloten op het Security Operations Center (SOC) van Rijkswaterstaat. Hierdoor bestaat het risico dat RWS een cyberaanval niet of te laat detecteert. De minister van Infrastructuur en Waterstaat moet nog stappen zetten om aan de eigen doelstellingen voor cybersecurity te voldoen
De Algemene Rekenkamer beveelt de minister van Infrastructuur en Waterstaat ook aan om het actuele dreigingsniveau te onderzoeken en te besluiten of extra mensen en middelen nodig zijn. Ook is het voor een snelle en adequate reactie op een crisissituatie van essentieel belang dat informatie up-to-date is. Pentesten zouden integraal onderdeel uit moeten maken van de cybersecuritymaatregelen bij vitale waterwerken. Verder zou moeten worden bezien of medewerkers van het SOC beter moeten worden gescreend
\cite{thkwaterwerken}.Het crisismodel kan beter, is de derde deelconclusie van de Algemene Rekenkamer. Er is geen specifiek scenario voor een crisis die wordt veroorzaakt door een cyberaanval. Ook ontbreekt inzicht in de effecten van een cybercrisis op andere sectoren, de zogeheten cascade-effecten. Tevens is de crisisdocumentatie op onderdelen verouderd\cite{rekenkamercybersecWater}.
Ook maakt cyberveiligheid nog geen volwaardig onderdeel uit van reguliere inspecties.’ De Rekenkamer hamert erop dat alle vitale waterinfrastructuur zo snel mogelijk op het SOC wordt aangesloten. Ook zouden werknemers van Rijkswaterstaat die belangrijke waterkeringen bedienen beter gescreend moeten worden op hun antecedenten. Sollicitanten hoeven nu slechts een Verklaring Omtrent Gedrag te overleggen, maar dat is een heel lichte toets
\cite{hackerWaterwerk}.
Volgens Rijkswaterstaat\cite{kramerZeeland} is het kostbaar en technisch uitdagend om klassieke automatiseringssystemen te moderniseren en wordt er daarom vooral ingezet op detectie van aanvallen en een adequate reactie daarop.
Uit het onderzoek blijkt dat Rijkswaterstaat de afgelopen jaren zelf van alle tunnels, bruggen, sluizen et cetera heeft vastgesteld welke cyberveiligheidsmaatregelen moeten worden genomen. Een groot deel van die maatregelen (ongeveer 60\%) was begin 2018 ook al uitgevoerd, maar Rijkswaterstaat ziet onvoldoende toe op de uitvoering van het resterend deel en heeft geen actueel overzicht van de overgebleven maatregelen.
De minister heeft een aantal waterwerken die Rijkswaterstaat beheert als vitaal aangewezen. . Uit het onderzoek blijkt dat nog niet alle vitale waterwerken rechtstreeks zijn aangesloten op het Security Operations Center (SOC) van Rijkswaterstaat. De ambitie om eind 2017 bij alle vitale waterwerken cyberaanvallen direct te kunnen detecteren was in het najaar van 2018 daarmee nog niet gerealiseerd. Hierdoor bestaat het risico dat RWS een cyberaanval niet of te laat detecteert\cite{cybersecWaterwerk}.
Over de cyberbeveiliging van gemeenten en waterschappen wordt al langer geklaagd. Zo meldde EenVandaag al in 2012 dat rioolgemalen en sluizen gemakkelijk van afstand te bedienen waren, onder meer door bijzonder slechte wachtwoorden
\cite{cybersecWaterschappen}.
Rittal doet onderzoek naarop afstand besdienbare sluizen\cite{cybersecZuidHolland}.
Beveiligde VPN
M2M Services levert aan inmiddels 220 gemeenten en waterschappen beveiligde connectiviteitsoplossingen voor het beheer van pompen, riolen en gemalen. Om risico’s op beveiligingsincidenten te voorkomen maken wij gebruik van een VPN oplossing, waarbij de verbinding optimaal beveiligd is middels encryptie en authenticatie\cite{waterwerkNED}.
Veiligheid op het water én op het land Gebruik van lampbewaking \cite{veiligheidwaterland}. 



%%%%%%%%%%%%%%%%%%%%%%%%%%%%%%%%%%%%%%%%%%%%%%%%%%%%%%%%%%%%%%%%%
\paragraph{Afbakening van requirements Wet en regelgeving voor sluizen}
Omdat we in deit onderzoek uitgaan van het uitbreiden van bestaande sluizen is er literatuurstudie gedaan naar sluizen. In de archieven van het ministerie van verkeer en waterstaat is er het rapport Design of waterlocks\cite{CivilEngineeringDivision}.
Het programma van requirements kunnen we in ons model niet helemaal overnemen. 
Zo zijn er precondities zaols topgrafie,bestaande watersluizen,waterlevel, wind, morphologie en bodemeigenschappen.

 

\paragraph{Analyse}
\paragraph{Conclusie}
%%%%%%%%%%%%%%%%%%%%%%%%%%%%%%%%%%%%%%%%%%%%%%%%%%%%%%%%%%%%%%%%%

%%%%%%%%%%%%%%%%%%%%%%%%%%%%%%%%%%%%%%%%%%%%%%%%%%%%%%%%%%%%%%%%%
 
\hoofdstuk{Uppaal model}


\paragraph{Inleiding}


\begin{center}
	\figuur{scale=0.45,angle=180}{plaatjes/sluispassage.png}{PDFA}{Vaste breedte
		(pdf)}
\end{center}













\hoofdstuk{Verificatie}
 We moeten aantonen dat een real-time programma voldoet aan de eisen opgesteld en gespecificeerd. De meest gebruikte methode voor het bewij
 
 zen van de correctheid van untimed programma's zijn aangepast voor timed programs.  We hebben nog geen aanpask gevonden voor het gebruik en bewijzen van correct gebruik van clocks.  Een bewijs voor het gebruik van real-time programmas met clocks is gegeven in T.A. Henzinger and P.W. Kopke. Verification methods for the di-
 vergent runs of clock systems
 
 In dit hoofdstuk formaliseren we de requirements ogegeven in de requiremenstlis tin hoofdstuk .. en bewijzen we de correcte toepassing met gebruik van de symbolic model-checker van Uppaal.
 Het systeem is gemodelleerd als een netwerk van meerdere timed automata: controller, sluis, stoplicht, deur, pomp en schip.
 
 Het bewijs vn corret gebruik kan ook worden aangetoond met help van bewijs voor inorrectgebruik
 
 
 
\paragraph{Semantiek}


About transition
A transition is composed of
a unique source location
a unique target location
a guard, i.e. an enabling condition (g := x ∼ c|g ∧ g, where
∼∈ {<, ≤, =, ≥, >}
a label (that can be used for synchronization)
a subset (potentially empty) of clocks to be reset

a clock valuation is a function v: X $\trightarrow$ $R^+$ \\
v[Y:=0] is the valuation obtained from v by resetting clocks from Y:  \\

\begin{math}
	$v[Y:=0]$=\left\{
	\begin{array}{ll}
		1, & \mbox{0 x $\in$ Y}.\\
		0, & \mbox{otherwise}.
	\end{array}
	\right.
\end{math}
\\

v+d = flow of time (d units) \\
(v +d)(x) = v(x)+d  \\
v $\models$ c means that valuation v satisfies the constraint c

evaluation of a clock constraint (v $\models$ g) \\
\begin{enumerate}
	\item $\vee$ $\models$ g x  < k iff ν(x) < k
	\item $\vee$ $\models$ x ≤ k iff ν(x) ≤ k
	\item $\vee$ $\models$ g1 ∧ g2 iff ν $\models$ g1 and ν $\models$ g2
\end{enumerate} 
\\
\\
\\

(s', v") and (s,v) $\xrightarrow[]{a}$(s', v").

Action transitions correspond to the execution of a transition	 from T. We write (s,v) $\xrightarrow[]{a}$ (s', v'), where a \in $\Sigma$, provided that there is a transition $\langle$ s,a, $\varphi$, $\lambda$, s' $\rangle$ such that v satisfies $\varphi$ and v=[$\lambda$:=0].

a delay transition (s, v1)  $\xrightarrow[]{$$\delta$(d)}$ (s, v_1 + d_1) for some $d_1$ $\geq$ 0, and
an action transition   (s, v1 +d1)  $\xrightarrow[]{a}$ (s', v_1') such that $v_1$ + $d_1$ satisfies $\varphi$ and v'_1 = (v_1 + d_1)[$\lambda $:=0].



 Defnition 3.1 Timed automata
A timed automation is een 6-tuple A $\langle$ $\sigma$ ,S,$S_0$, X, I, T$\rangle$, waarin geldt dat $\sigma$ een eindige alfabet voorsteld; S is een eindig srt van lcoaties (states); $S_0$ $\subseteq$ S is een set van initiele locaties; X is een set van clocks; I : S $\rightarrow$ $\Phi$(x) is de locatie invariant; en T $\subseteq$ S $\times$ $\sigma$ $\times$ $\Phi$(X) $\times$ $2^x$ $\times$ S is een set van transities.
De 5-tuple $\langle$ s, a, $\phi$ , $\lambda$, $s^$ $\rangle$ $\in$ T is een ransitie van locatie s naar locatie s' die correspondeert met actie gelabeld als a, De constraint op de clock $\phi$ speificeert wanneer de transitite wordt uitgevoerd, en $\lambda$ $\subseteq$ X is de set van clocks die worden gereset wanneer de transitie wordt uitgevoerd.

 De semantics van een timed automationA is gedefinieerd door de associatie met een transitiesysteem., T(A). Ten alle tijde geldt, de configuraties van een globale state van het systeem worden gemodelleerd door de gegeven timed automation in een bepaalde locatie, s, van de automation en clock interpretatie, v, welke een reele waarde toekent aan een klok. Dus de configuratie is een paar (s,v) waar s $\in$ S en v : $\rightarrow$ $R^+$. De set van initiele configuraties is gegeven door de set { (s,v) | s $\in$ $\wedge$ $\forall$ x $\in$ X $[v(x) =0]$}. In andere woorden, de set van initiele locaties waarin alle clocs gelijk gezet zijn aan 0.
 \cite{03CHAPTER3}
 
 Operatoren, definities clock zones en regios's zijn we in artikel . \cite{04_giWorkshop2000}
 \cite{pelanekFormal} 
 In het artikel worden de formele definitie en verificatie va   state transitites uitgelegd met de daarbij behorende eindige set van states Q, q0 als initiele state, $\sigma$ is de einndige set van symbole ut de input alfabet, de se van transities vertegenwoordogd door E, de set van states 'F'
 \cite{jiyanpatil07TOC}
 Met deze kennis zou het mogelijk zijn om een Verification and control of real-time systems te kunnen uitvoeren in timed automata
 \cite{latin06}
 \cite{realtimeForms}
 \cite{verification}
 \cite{uppaal}
 De delay en ation transites zijn wel in dit verslag opgenomen. Maar de clock regions, regions automata, en zone automata zijn niet overgenomen in onze studie.

 

 \cite{03CHAPTER3}
 Wel is duidelijk de Formal Syntax and Semantics en de TEMPORAL LOGICS. Maar de vertaling van 
 Network of Timed Automata naar Uppaal query language. A TCTL formula φ is satisfiable iff there is a labeled timed automaton
 Wel is overgenomen de Definition 3.7 (Satisfaction relation) 
 M = $\langle$A, µ$\rangle$ and a state s ∈$\in$S, such that M, s $\models$ φ.
 \cite{isbn9789526031033}
 Timed automata [4] [57] are hereby to model timed systems. These are finite-state automata
 equipped with clocks used to specify constraints on the amount of time that can elapse
 between two events (blz 46). Timed kripke structures (blz63) (blz 69) (blz 78) blz 99.
 \cite{nourollahi20191215}
 \cite{Lecture2}
 Timed Transition Systems (blz 3). Clock Constraints. (blz 4) Timed Automata.  (blz 4). Timed Computation Tree Logic.  (blz 6,7)
 \cite{LIPIcs-TIME-2021-12}
 Properties and Temporal L
 \cite{mctutorial}
 \cite{FULLTEXT02}
 \cite{stanfordRealtime}
 \cite{baierKatoenModelChecking}
 
 
 \paragraph{Timed automata}
  Timed automata [4] [57] are hereby to model timed systems. These are finite-state automata
 equipped with clocks used to specify constraints on the amount of time that can elapse
 between two events (blz 46). Timed kripke structures (blz63) (blz 69) (blz 78) blz 99.
 \cite{nourollahi20191215}
 
\paragraph{Data variabelen}
Dat variabelen zijn onder andere: water hoog  en laag, en aanal schepen in de queue.
\paragraph{Acties}
 Acties in het model zijn onder andere: invaren, uitvaren, deuren openen en sluiten, nivelleren
\paragraph{Clock regions}
\cite{clarke2000Modelchecking}
\cite[p.~5]{clarke2000Modelchecking}
\cite{clarke2000Modelchecking21}
\cite{clarke2000Modelchecking212}
\cite{clarke2000Modelchecking223}
\cite{clarke2000Modelchecking31}
\cite{clarke2000Modelchecking32}
\cite{clarke2000Modelchecking33}
\cite{clarke2000Modelchecking411}
\cite{clarke2000Modelchecking43}
\cite{clarke2000Modelchecking63}
\cite{clarke2000Modelchecking64}
\cite{clarke2000Modelchecking661}
\cite{clarke2000Modelchecking91}
\cite{clarke2000Modelchecking102}
\cite{clarke2000Modelchecking11}
\cite{clarke2000Modelchecking122}
\cite{clarke2000Modelchecking123}
\cite{clarke2000Modelchecking132}
\cite{clarke2000Modelchecking1321}
\cite{clarke2000Modelchecking152}
\cite{clarke2000Modelchecking171}
\cite{clarke2000Modelchecking172}
\cite{clarke2000Modelchecking173}

\cite{audioSemanticsBengtsson}
\cite{guidingAutomataBberm}
\cite{gearTransitionLindahl1}
\cite{gearTransitionLindahl2}
\cite{martinelliScada}
\cite{IgbalReconstructurintTransition1}
\cite{IgbalReconstructurintTransition2}
\cite{huangVerficationStoch}
\cite{bengtssonUppaalVerification}
\cite{pranaliVerificationWaterLevel}
\cite{alexandreUppaalDefinition}
\cite{behzadEvalQOS}
\cite{behzadVariablesQoS}
\cite{alur}
\cite{alurDenseRealTime}
\cite{alurSystemClok}
\cite{alurModelHybrid}
\cite{rijksoverheidSluizen}
\cite{rijksoverheidSluisStroomschema}

\paragraph{CTL logica}
Alle veiligheid en reachability requirements formeel gespecificeerd in hoofdstuk ... zijn geverifieerd in uppaal met gebruik an A en E state formulae. Deze zijn als volgt:
$\sim$, $\xi$ , $\cong$,$\overset{\Delta}{=}$ or equal by definition, $\uplus$
\newline \\
Om aan te tonen dat de gedefinieerde specificaties altijd geldig zijn moet de basisi specificatie inductief worden opgelost. \cite{latin06} blz 73,82,83,90,91,92,93,98,104,156,197, 225, 236, 315, 317, 318\cite[p.~318]{realtimeForms}  

%%%%%%%%%%%%%%%%%%%%%%%%%%%%%%%%%%%%%%%%%%%%%%%%%%%%%%%%%%%%%%%%%
M, s $\models$ p $\Leftrightarrow$ p $\in$ L(s) \\
M, s $\models$ $\not$ f1 $\Leftrightarrow$ M, s $\nvdash$ f1 \\
M, s $\models$ f1 $\vee$ f2 $\Leftrightarrow$ M,s $\models$ f1 or M,s $\nvdash$ f2 \\
M, s $\models$ f1 $\wedge$ f2 $\Leftrightarrow$  M,s $\models$ f1 and M,s $\nvdash$ f2 \\
M, s $\models$ $\mathrm{E}$ $g_{1}$ $\Leftrightarrow$ there is a path $\pi$  from ~  s ~   such ~  that  ~ M, $\pi$ $\models$ g1 \\
M, s $\models$ p $\Leftrightarrow$ for every path $\pi$  ~ starting from  ~  s, M, $\pi$ $\models$ g1 \\
M, s $\models$ p $\Leftrightarrow$ s is the first state of $\piand$ M, s $\models$ f1 \\
M, s $\models$ $\not$ $g_{1}$ $\Leftrightarrow$ M, $\pi$  $\nvdash$ g1\\
M, s $\models$ p $\Leftrightarrow$  M, $\pi$  $\models$ g1  or  M, $\pi$  M, $\pi$  $\models$ g2\\
M, s $\models$ p $\Leftrightarrow$ M, $\pi$  $\models$ g1  and  M, $\pi$  M, $\pi$  $\models$ g2 \\
M, s $\models$ p $\Leftrightarrow$ M, $\pi^{1}$ $\models$ g1 \\
M, s $\models$ p $\Leftrightarrow$ there exists a k $\ge$ 0, such that  ~ M, $\pi^{k}$  $\models$ g1\\
M, s $\models$ p $\Leftrightarrow$ for all i $\ge$ 0,M,$\pi^{i}$ $\models$ g1 \\
M, s $\models$ g1 $\bugcup$ g2 $\Leftrightarrow$ ~  there  ~ exists  ~ ak  ~ $\ge$  ~ 0 ~  such ~  that  ~ M,  ~ $\pi^{k}$ $\models$ g2\\
and  ~ for  ~ all ~  0  ~ $\le$ j < k, M,$\pi^{j}$ $\models$ g1
M, s $\models$ p $\Leftrightarrow$ for all j $\ge$ 0, if for ~  every  ~ i < j,M,$\pi^{i}$ $\nvdash$ g1 then M,$\pi^{j}$ $\models$ g2\\
%%%%%%%%%%%%%%%%%%%%%%%%%%%%%%%%%%%%%%%%%%%%%%%%%%%%%%%%%%%%%%%%%
Safety properties
Following L. Lamport, a safety property states that
something bad must never happen. The “bad thing” represents a
critical system state that should never occur, for instance a train
being inside a crossing with the gates open. Taking a Boolean observable C : Time −→ {0, 1}, where C(t) = 1 expresses that at
time t the system is in the critical state, this safety property can be
expressed by the formula:
$\forall$ t \in Time $\dot{}$ $\neg$ C(t)


Here C(t) abbreviates C(t) = 1 and thus ¬C(t) denotes that at time
t the system is not in the critical state. Thus for all time points it
is not the case that the system is in the critical state.
In general, a safety property is characterised as a property that can be falsified in bounded time. In case of (1.1) exhibiting a single
time point t0 with C(t0) suffices to show that (1.1) does not hold.
In the example, a crossing with permanently closed gates is safe,
but it is unacceptable for the waiting cars and pedestrians. Therefore
we need other types of properties.
%%%%%%%%%%%%%%%%%%%%%%%%%%%%%%%%%%%%%%%%%%%%%%%%%%%%%%%%%%%%%%%%%
liveness properties
Safety properties state what may or may not occur,
but do not require that anything ever does happen. Liveness properties state what must occur. The simplest form of a liveness property guarantees that something good eventually does happen. The
“good thing” represents a desirable system state, for instance the
gates being open for the road traffic. Taking a Boolean observable
G : Time −→ {0, 1}, where G(t) = 1 expresses that at time t the
system is in the good state, this liveness property can be expressed
by the formula: 
$\exists$t $\in$ Time $\dot{}$ G(t).
In other words, there exists a time point in which the system is in the
good state. Note that this property cannot be falsified in bounded
time. If for any time point t0 only ¬G(t) has been observed for
t ≤ t0, we cannot complain that (1.2) is violated because eventually
does not say how long it will take for the good state to occur.
Such liveness property is not strong enough in the context of realtime systems. Here one would like to see a time bound when the
good state occurs. This brings us to the next kind of property.
%%%%%%%%%%%%%%%%%%%%%%%%%%%%%%%%%%%%%%%%%%%%%%%%%%%%%%%%%%%%%%%%%
bounded response properties

A bounded response property states that
a desired system reaction to an input occurs within a time interval
[b, e] with lower bound b ∈ Time and upper bound e ∈ Time where
b ≤ e. For example, whenever a pedestrian at a traffic light pushes
the button to cross the road, the light for pedestrians should turn
green within a time interval of, say, [10, 15]. The need for an upper
bound is clear: the pedestrian wants to cross the road within a short
time (and not eventually). However, also a lower bound is needed
because the traffic light must not change from green to red instantaneously, but only after a yellow phase of, say, 10 seconds to allow
cars to slow down gently.
With P(t) representing the pushing of the button at time t and
G(t) representing a green traffic light for the pedestrians at time t,
we can express the desired property by the formula
$\forall$ t1 $\in$ Time $\dot{}$ (P(t1) $\rightarrow$ $\exists$t2 \in [t1 + 10, t1 + 15] $\dot{}$ G(t2))
Note that this property can be falsified in bounded time. When
for some time point t1 with P(t1) we find out that during the time
interval [t1 + 10, t1 + 15] no green light for the pedestrians appeared,
property (1.3) is violated.
%%%%%%%%%%%%%%%%%%%%%%%%%%%%%%%%%%%%%%%%%%%%%%%%%%%%%%%%%%%%%%%%%
Duration properties

A duration property is more subtle. It requires that
for observation intervals [b, e] satisfying a certain condition A(b, e)
the accumulated time in which the system is in a certain critical
state has an upper bound u(b, e). For example, the leak state of a
gas burner, where gas escapes without a flame burning, should occur
at most 5% of the time of a whole day.
To measure the accumulated time t of a critical state C(t) in a
given interval [b, e] we use the integral notion of mathematical calculus:

\[ \int_{b}^{e} C(t) \,dx \]	

Then the duration property can be expressed by a formula:


\[
$\forall$ b,e $\in$ Time $\bullet$ A(b,e) =\int_b^{e}C(t)\,\mathrm{d}t $\leq$  u(b,e)
\]
%%%%%%%%%%%%%%%%%%%%%%%%%%%%%%%%%%%%%%%%%%%%%%%%%%%%%%%%%%%%%%%%%
\paragraph{Andere duration properties}
Queries voor een  time based specificatie in Uppaal worden volgens literatuur \cite{04_giWorkshop2000} gedefinieerd als:

It is at all times possible that a weak sequence A with time interval(s) [x, y]
occurs
It is at all times possible that a weak sequence A with time interval(s) [x, y] does
not occur
It is at all times possible that a strong sequence A with time interval(s) [x, y]
occurs
It is at all times possible that an element of set A occurs within the interval [x, y]
. It is at all times possible that all elements of set A occur simultaneously within the
interval [x, y]
It is at all times possible that all elements of set A occur exclusively within the
interval [x, y]
It is at all times possible that an element of set A never occurs within the
interval [x, y
. It is at all times possible that all elements of set A never occur simultaneously
within the interval [x, y]
. It is at all times possible that all elements of set A never occur exclusively within
the interval [x, y]
 It is at all times true that if a strong sequence A with time interval(s) [x1, y1] occurs
then it must happen within [x2, y2] time unit(s) that an element of set B occurs
It is inevitable that if all elements of set A occur simultaneously within the
interval [x1, y1] then it is possible at some time later that a weak sequence B with
time interval(s) [x2, y2] occurs
. It is at all times true that if all elements of set A always occur simultaneously
within the interval [x1, y1] then it must happen in exactly [z] time unit(s) that all
elements of set B occur simultaneously within the interval [x2, y2]

AG EF_[_x_,_y_] $\vee$
%%%%%%%%%%%%%%%%%%%%%%%%%%%%%%%%%%%%%%%%%%%%%%%%%%%%%%%%%%%%%%%%%

\hoofdstuk{Conclusie}

Wat hebben alle bovenstaande rampen/ongelukken gemeen? Veiligheid.
Bij de therac waren er diverse problemen: communicatie, doorontwikkeling, controle en toetsing
Was het makkelijk te onderzoeken? Waarom?
Bij de boeing 737 crashes was het probleem van controle en communicatie naar medewerkers
Was het makkelijk te onderzoeken? Waarom?

Uit de evaluatie van de china explosion 2015 tianjin komt naar voren dat communicatie, transparantie en veiligheid niet altijd prioriteit hadden bij de lokale autoriteiten
Was het makkelijk te onderzoeken? Waarom?

Bij de tesla autopilot crashes komen soms onvoldoende onderbouwde ontwerpkeuzes naar voren die niet goed zij  afgewogen tegenover het gedrag van de bestuurder
vlucht 1951
Was het makkelijk te onderzoeken? Waarom?

De ramp in Tsjernobyl toont aan hoe autoriteiten een ramp in de doofpot proberen te stoppen
Was het makkelijk te onderzoeken? Waarom?



Wat heb ik geleerd
Ik heb erg veel geleerd van het veilig opzetten van VPN’s. Een VPN opzettenhad ik namelijk nog nooit gedaan. Het opzetten van SSH en het aanmaken vanVM’s was al bekend. Ook had ik nog nooit met UDP sockets geprogrammeerd.Verder heb ik geleerd hoe ik in de praktijk een VM in een VLAN kan zetten enhoe VLAN’s netwerken van elkaar kunnen scheiden.Het leukste onderdeel van het project, was dat wonderbaarlijk mijn gekozenoplossing elegant werkte. UDP Servers en clients zijn gerealiseerd met minderdan enkele regels logisch scipt. Ik had aan genomen dat het werken met socketsin shell absoluut rampzalig zou uitpakken. Ik ben blij dat het opdracht zo vrijwas, zodat ik experimenteel kon zijn met mijn implementatie.



