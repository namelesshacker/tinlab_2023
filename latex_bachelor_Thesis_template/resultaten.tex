
 



\hoofdstuk{Theoretisch kader}

In het eerste hoofdstuk is duidelijk geworden wat de onderzoeksvraag is, namelijk ‘Hoe kan een geautomatiseerde sluis worden gemodeleerd met oog op ontwikkel- en onderhoudskosten,veiligheid, efficientie en capaciteit’. Door de toenemende complexiteit van systemen is het gebruik van modellen en de toepassing van timebased model checking  op industriele controle systemen een manier van modelleren van het systeem en de requirements zodat er een bijdagre kan worden geleverd aan de acceptatie van  simulatie-/modeltechniek voor de industrie.(‘https://link.springer.com/article/10.1007/s10626-020-00314-0’, 2020). Of dit ook het geval is bij het modellereren van sluizen is nu de vraag.

De bestudering van rampen aan de hand van het vier-variabelen model biedt maakt het analyseren mogelijk van rampsituaties. Van een aantal rampen is een beschrijving gegeven met datum, plaats en oorzaak. De analyse van de 4-variabelen modellen zal gebruikt worden voor de requirementsdefinitie, ontwerp en ontwikkeling van het sluismodel. 

De verschillende factoren en achtergronden die  samenhangen met het modelleren van een sluis zullen in dit hoofdstuk toegelicht worden. Bovendien worden er hypotheses gevormd die de basis vormen voor debeantwoording van de onderzoeksvraag. 




\paragraph{Wat is uppaal}

Wat is Uppaal
Uppaal is an integrated tool environment for modeling, simulation and verification of real-time systems, developed jointly by Basic Research in Computer Science at Aalborg University in Denmark and the Department of Information Technology at Uppsala University in Sweden. It is appropriate for systems that can be modeled as a collection of non-deterministic processes with finite control structure and real-valued clocks, communicating through channels or shared variables [WPD94, LPW97b]. Typical application areas include real-time controllers and communication protocols in particular, those where timing aspects are critical.


model checking

Wat is statistical model checking?
Dit verwijst naar verschillende technieken dfie worden gebruikt voor de monitoring van een systeem. Daarbij wordt vooral gelet op een specifieke eigenschap. Met de resultaten van de statsitieken wordt de juistheid van een ontwerp beoordeeld. Statistisch model checking wordt onder andere toegepast in systeembiologie, software engineering en industriele toepassingen.
https://www-verimag.imag.fr/Statistical-Model-Checking-814.html?lang=en#:~:text=Statistical%20Model%20Checking%20(SMC)%20is,from%20state%20space%20explosion%20issues.


\cite{inriaStatsMoodCheck}
\cite{ buddeModelChecker}
\cite{AGHASuervey }


Waarom gebruiken we statistisch model checking?
To overcome the above difficulties we propose to work with Statistical Model Checking [KZHHJ09,You05,You06,SVA04,SVA05,SVA05b] an approach that has recently been proposed as an alternative to avoid an exhaustive exploration of the state-space of the model. The core idea of the approach is to conduct some simulations of the system, monitor them, and then use results from the statistic area (including sequential hypothesis testing or Monte Carlo simulation) in order to decide whether the system satisfies the property or not with some degree of confidence. By nature, SMC is a compromise between testing and classical model checking techniques. Simulation-based methods are known to be far less memory and time intensive than exhaustive ones, and are oftentimes the only option. 
https://project.inria.fr/plasma-lab/statistical-model-checking/

Alternatief
Alternatieven voor Uppaal zijn Asynchronous Events,Vesta en MRMC.


\paragraph{MODE CONFUSION }
Mode confusion tredd op als gepbserveerd gedrag van een technisch systeem niet past in het gedragspatroon dat de gebruiker in zijn beeldvorming heeft  en ook niet met voorstellingsvermogen kan bevatten.
\paragraph{Wat is automatiseringsparadox}
Gemak dient de mens. Als er veel energie wordt gestoken in de ontwikkeling van hulmiddelen die taken van werknemers overemen heeft dat tot resultaat dat veel productieprocessen worden geautomatiseerd. De vraag is dan of vanuit mechnisch wereldpunt de robot niet de rol van de mens overneemt en of de mens nog de kwaliteiten heeft om het werk zelf te doen.
\cite{bicker21102016automatiseringsparadox }
\cite{vseautoparadox }
\cite{blogxot21112016slimapparaat }



\paragraph{4 variabelen model}





Het 4 variabelen model kort toegelicht
Monitored variabelen: door sensoren gekwantificeerde fenomenen uit de omgeving, bijv temperatuur

Controlled variabelen: door actuatoren \bestuurde fenomenen uit de omgeving
For example, monitored variables might be the pressure and temperature
inside a nuclear reactor while controlled variables might be visual and audible alarms, as well
as the trip signal that initiates a reactor shutdown; whenever the temperature or pressure reach
abnormal values, the alarms go off and the shutdown procedure is initiated

Input variabelen: data die de software als input gebruikt
Here, IN models the input hardware interface (sensors and analog-to-digital converters) and
relates values of monitored variables to values of input variables in the software. The input variables model the information about the environment that is available to the software. For example,
IN might model a pressure sensor that converts temperature values to analog voltages; these voltages are then converted via an A/D converter to integer values stored in a register accesible to the
software.

Output variabelen: data die de software levert als output
The output hardware interface (digital-to-analog converters and actuators) is modelled
by OUT, which relates values of the output variables of the software to values of controlled variables. An output variable might be, for instance, a boolean variable set by the software with the
understanding that the value true indicates that a reactor shutdown should occur and the value
false indicates the opposite



\paragraph{6 Variable model}
Optitatieve statements omschrijven de omgeving zoals we het willen zien vanwege de machine. 

Indicatieve statements omschrijven de omgeving zoals deze is los van de machine. 

Een requirement is een optitatief statement omdat ten doel heeft om de klantwens uit te drukken in een softwareontwikkel project. 

Domein kennis bestaut uit indicatieve uitspraken die vanuit het oogpunt van software ontwikkeling relevant zijn. 

Een specificatie is een optitatief statement met als doel direct implementeerbaar te zijn en ter verondersteuning van het natreven vande requirements. 

Drie verschillende type domeinkennis: domein eigenschappen, domein hypothesen, en verwachtingen. 

Domein eingenschappen  zijn beschrijvende statementsover een omgeving en zijn feiten.Domein hypotheses  zijn ook beschrijvende uitspraken over een omgeving, maar zijn aannames. 

Verwachtingen zijn ook aannames, maar dat zijn voorschrijvende uitspraken die behaald worden door actoren als personen, sensoren en actuators. 

  
\paragraph{Conceptueel model}



System requirement:
uitspraak over wereld fenomenen (gedeeld of niet) of doelen
die bereikt moeten worden.
met enige regelmaat informeel, niet precies geformuleerd.
Software requirement/specicatie:
uitspraak over gedeelde fenomenen of doelen die de machine
moet bereiken middels de onderdelen waar die machine uit
bestaat of middels de fenomenen waar de machine controle
over heeft.
doorgaans preciezer, meetbaar, exact geformuleerd.


Systemen gaan een zekere interactie aan met hun omgeving:
Sensoren: meten fenomenen uit de omgeving (temperatuur,
druk, licht, geluid, etc.)
actuatoren: veranderen iets in de omgeving (mechanische,
electrisch, pneumatisch, etc.)
Software:
Kan niet direct communiceren met de buitenwereld.
Snapt derhalve niets van de buitenwereld.
Kan alleen maar bestaan in en communiceren met het
systeem.


\paragraph{Requirementsengineering}

Om de juiste requirements te verzamelen en selecteren hebben we meer kennis nodig van de methoden hiervoor gebruikt in het domein van requirementsengineering. Daarom is een literatuurstudie gedaan naar rapporten en artikelen die ons meer informatie over dit onderwerp verschaffen.
 Uitdagingen in requirementsengineering zijn incomplete requirements en specifcates, veranderende requirements en specificates en grote, complexe oftwaresystemen.
 
 Het article the worlds a stage biedt inzicht in de requirementstechnieken voor een ambulance in london. In het artikel gaan de onderzoeks in op de volgende onderwerpen: 
 viewpoints, sociale ascpecten,evolutie, non-functional requirements, conflict resolution, traceability
 
 Goal of this paper is requirement  engineering on London aulance service
 Method of opinions: crew, staff, management, computational, transport, services
 Evolutioon: changes, specification and technology trade
 Environment: company policies, regulation, impact solution on organizational
 Non-functional aspect: communicatio problem, malfunctions, less critical isues: cost, tradeoff beween performance \& user interfaces
 vieuwpoint: is a subset of all system requirements expressible in a given requirements notation regardless of the stakeholders involved
 
 log change
 basic model vieuw
 hypertext vieuw
 data transmission problems
 continued difficulties
 installation problems
 problems caused by mistake
 tracebility requirements[selecting reliable information]
 PRE requirement specification traceability, repository baed approach
 1) compromise specification
 2) representatives
 3) agreement dimensions
 Domain: part of the worl in which the computer system effects will be felt, inclusing its peoples, organizational structure, related legislation, physical location and met only the compyter systems
 
 
 Het artikel "from inconsistencyhandling to non-conanical requirements management: a logical perspective" geeft enkele tips voor het omgaan met inconsistente requirements:
 
 1) identifying non-canonicalrequirements
 2) measuring them
 3) generate caandidate proposals for handling them
 4) choosing acccptable probosals
 5) revising them acccording to the proposals

Het artikel "managing inconsistent specification: reasoning, analysis, action" zoekt een ontologische benadering voor het omgaan met inconsistenties in de requirements specificaties.
Voor de omshrijving van een specificatie kun je gebruik maken van logica. Daarbij kun je onderschei maken in klasieke logica quasi -logica.
Wat ook een rol kan spelen in domain interpretatie. De achtergrond van de gebruikers speelt ook een rol.
Zo is er e=onderscheid te maken in de volgende groepen: users, customers, domain experts, designers,, manufacturers
graphical  textual specification

Basic constraint, legal constraint, cooperation constraint
1) scenatio  definition
2) scenario analysis
3) scenario consolidation


Hoe kan een systeem verder worden ontworpen op een manier dat non-functionele requirements worden geimplementeerd?
Hoe hangt dat ontwerp samen met aanpassingen van het functionele en structurele aspect van het systeem?

block[objects, classes, methods, messages, inheritance]
[goals,agents, alternative, events, actions,existence modalities,agent responsibilities]


Het artikel "representing and using nonfunctional requirements: a process-oriented approach"" gaat in op een het proces van requirements acquisitie. Hierbij in ogenschouw de acquisitie van prestaties, ontwerp en aanpasbaarheid.
product oriented
process oriented


Acquisitie Prestaties
user concern
-Hoe goed werkt het product
-Hoe goed wordt de bron gebruikt?>> Efficiency
-How veilig is het product >> integrity
-Met hoeveel zekerheid is uit  te sluiten dat het werkt >>Reliability
-Hoe goed werkt het product onder zware omstandigheden >> sustainability
-Hoe makkelijk is het product in gebruik >> usability
quality attribute


Acquisitie: Ontwerp
user concern
Hoe valide is het ontwerp
-Is ht ontwerp conform de requirements
-hoe makkelijk is het ontwerp te repareren
-Hoe makkelijk zijn de prestaties te verifieren

quality attribute


Acquisitie: Aanpasbaarheid
user concern
-hoe makkelijk is het om het product aan te passen
- hoe makkelijk is het om het product te updaten en/of uitbreiden>> expendability
- hoe makkelijk is het om een wijziging door te voeren>>flexibility
-hoe makkelijk is het om andere system aan te sluiten >> portability
- hoe makkelijk is het om het product te transporteren >> interoperability
-hoe makkelijk is het om te converteren tot een systeem gebruiksklaar voor communiceren met andere systemen>> reaseability
quality attribute




 \cite{jonkerTreurKlush200informativeAgents}
\cite{boehmBoseLeeRequirementsNegotiations}
\cite{muHungJinLiu2013inconsistencyReqs}
\cite{hunterNuseibeh1996manageSpecs}
\cite{myloloupos1992representingReqs}
\cite{zavePamela4darkCorners}
\cite{zavePAmela1997regEngineering}

%%%%%%%%%%%%%%%%%%%%%%%%%%%%%%%%%%%%%%%%%%%%%%%%%%%%%%%%%%%%%%%%%

what is a good software specification

\cite{fvaandrager2322010Goodmodel}
\cite{onix01102022devopmodel}
\cite{sulemani04012021softwareprocesmodel}
\cite{globalluxsoft18102017softdev}
\cite{wiegers30052022SRS}
\cite{muller06092020goodspecification}
\cite{informit30062008reqmanagement}
\cite{altexsoft15092020writingSRS}


\paragraph{Wat is een sluis}

\paragraph{Recente ontwikkelingen op het gebied van sluisautomatisering}

Het ministerie van verkeer en Waterstaat wil in het kader van het klimaatakkoord en onderzoek laten uitvoeren naar de staat van het sluizenpark in Nederland. Het onderzoek moet zich richten op het ontwerpen en ontwikkelen van een geautomatiseerd sluismodel dat geschikt is voor een brede toepassing. In het onderzoek moet naar voren komen wat de huidige staat is van de sluizen met oog op veiligheid, efficiëntie, capaciteit, onderhoud, duurzaamheid en automatisering. Het onderzoek geeft aan hoe een volledig model worden opgeleverd opdat ontwerp van verschillend volledig geautomatiseerde sluizen in de toekomst geautomatiseerd kunnen worden.  


\paragraph{Studie naar rampen aan de hand van het vier variabelen model}
\newline Voor deze studie is onderzoek gedaan naar verschillende rampen aan de hand van het vier variabelen model.
Elke ramp op deze manier categoriseren  kan ons helpen te bepalen in hoeverre requirements een rol kunnen spelen in de veiligheid van ons model.  Zo is er de bijlmerramp \cite{aviationsafety04101992airplaneCrashBijlmer}
, deze vond plaats op 04/10/1994. Dan nog de  ramp turkisch airlines vlucht 1951 op woensdag 25 februari 2009 \cite{catsr25022009Boeing737AmsterdamCrash}
\cite{zuilen23022019Tijdlijnpoldercrash}
\cite{wikinews04032009techfoutailines1951}
\cite{luchtvaartnieuws21012020boeing737conclusies}
\cite{adformatie280220209communicatiegebreken}
\cite{spinnael25022009onderzoekpolderbaancrash}
\cite{crashTurkishAirlines}
\cite{flightradar24}
\cite{flightstatstracker}. 
%%%%%%%%%%%%%%%%%%%%%%%%%%%%%%%%%%%%%%%%%%%%%%%%%%%%%%%%%%%%%%%%%
\newline \indent
De therac-25 June 1985 and January 1987. 
Medical lineair accelerators accelerate electrons to createhighenergy beams that can destroy tumors with minimal impact on the surrounding healthy tissue.
In the mid-1970s, AECL, developed a radical new "double-pass" concept for electron acceleration. A double passaccelerator needs much less spaceto develop comparableenergy levels because it folds the long  physical mechanismrequired to accelerate the electros, and it is more economic to produce.
Using this double pass concept AECL designed the  Therac-25, a dual mode lineair acelerator that can deliver either photonsat 25 MeVor electrons at various energy levels. Compared with theTerac-20 The Thrac-25 is notably more compact,, more versatile, and arguably easier to use. 
The higejr energy takes advantage of the phenomenon "depth dose": As the energy increases, the depth in the body at which maximum dose buildup occurs alse increases, sparing the tissue above the target area.
First, like the Therac-6 and the Therac-20, the Therac25 is conrolled by a PDP11. The Terac-6and Therac-20 had been designed around machines that already had histories of clinical use without computer control.
The therac-20 has idependent protective circuits for monitoring electron-beam scanning, plus mechanical interlocks for policing the machine and ensuring safe operation.
Finally some software for the machines was interrlatd or reused.
Eleven therac-25 were installed: five in the usand six in canada. Six accidents involving massive oerdoses to patients occured between 1985 and 1987. The machine was recalled in 1987 for extensive design changes, including hardware	 safeguards against errors.
Kennestone Regional Oncology Center 1985
Door rechtzaken waren managegers op de hoogte van de problemen en ongelukken. Maar er werd in het vervolg niet over gerapporteerd.
The treatment prescription printout failure was disabled at the time of the accident , so there was no hardcopyof the treatment data.
Ontario Cancer Foundation in 1985
Since the machine did not suspendand the control display indicated no dose was delivered to the patient, the operator went ahead with a second attempt at trratment by pressing the "P" key, expecting the machine to deliver the proper dose this time. This was standard operating procedure and, described in the "The operating interface" on p 24, Therac 25
oprators had become accustomed to freunt malfunctions that had no untowardconsequences for the patient. Again, the machine shut downin the same manner. The oeprator repeated this process four times after the original attempt- the display showing "no dose" delivered to the patient each time. After th fifth pause, the machine went into treatment suspedn, and a  hospital service technician was called.
The technician found nothing wrong with the machine. This was not an unusual scenario, according to the Therac-26 operator
Manufactureere response
Government and user response
Yakima Valley Memorial Hospital in 1985
Manufactureere response
Government and user response
East Texas Cncer Center, March 1986
Manufactureere response
Government and user response
East Texas Cncer Center, April 1986
Manufactureere response
Government and user response
Yakima Valley Memorial Hospital
Manufactureere response
Government and user response
	\cite{rogaway2004therac25},
\cite{wikiTherac25}, 
\cite{lynch2017theracRaceConditions},	\cite{lim1998theracdisaster}, 
\cite{fabio26102015therac25},	 	\cite{ethicsunwrappedTherac25}, 	\cite{casesHistoryTherac25},	 	\cite{caballero2019Therac25}, 	\cite{rose1994theracFatalDose}, 	\cite{levesonMITTherac25},
\cite{grant1978theracevaluation},	 	\cite{turnerTheracAccidentsInvestigations},	\cite{turner1993TheracAccidentsInvestigations}, 	\cite{wang2017industrialdesignengineering}, 	\cite{levesonturner1993theracpart2},	\cite{porelloTheraccFailure},\cite{theracIncidents}, 
\cite{huffbrown2004casestudyethicatherac}, 
\cite{sebowikimedicalradiation},	\cite{hsia1995testtherac25},	\cite{magsilvaTheracTesting},
\cite{chemeuropetherac25},	\cite{statsenko10102016Therackillerbug},	\cite{therac25casestudy},	\cite{thomas1994theracinLotos},	\cite{twitter2019programmerbehindtherac},	\cite{wikibookstherac}, 
\cite{bozdagTherac25},	\cite{levesonTurnerTheracAbstract}, 	\cite{stackexchange2021therac25code}.
%%%%%%%%%%%%%%%%%%%%%%%%%%%%%%%%%%%%%%%%%%%%%%%%%%%%%%%%%%%%%%%%%
\newline \indent
%Hoe werkt het
tesla autopilot features voor dataverzameling\cite{denneyjdsupraFeds},\cite{gritti24062020tesladataengine}.
% crashes
 De eerste tesla crash is van juni 2016 \url{https://impakter.com/tesla-autopilot-crashes-with-at-least-a-dozen-dead-whos-fault-man-or-machine/#:~:text=The%20first%20known%20death%20reportedly,trailer%20against%20the%20bright%20sky.}. En meerdere zouden volgen.
Een ongeluk in  de VS waarbij 2 inzittenden om het leven kwamen. Een persoon had plaats genomen als bijrijder en de andere persoon als passagier achter de stoel van de bestuurder. Waarschijnlijk was de autopiloot niet ingeschakeld.
\cite{anderson30042021secondteslacrash},\cite{raynal20042021probeTeslaCrash},\cite{firstpress11052021fatalnonautopilot},\cite{cochran18042021nodriverTeslaCrash},\cite{gitlin11052021autopilot},\cite{sommerfield12072021NHTSAmandateresult},\cite{hawkins30062021nhtsarequiresreporting},\cite{wilson19042021teslacrashregulators},\cite{mcfarland22042021selfdrivingrisks}
De situatie en oorzaken zijn bij elke ramp verschillend. 
Een automobilist heeft in een rit van 37 minuten slechts 25 seconden zijn handen aan het suur gehad ondanks de melding "Hands requireed not detected". Hiermee zijn de onderzoekers van de NTSB ervan uitgegaan dat de bestuurder de autopiloot bewschouwde als een volledig autonooom rijsyssteem in plaatst van een veligheidsmechanisme
\cite{oremus21062017fatalTeslaCrash}. Of in 
Mei 2015 als een besuurde foto's van zichzelf maakt in de testla zonder handen aan het stuur of voeten op het pedaal.
\cite{guardian15052021teslacrashHandsOnWheel}
Een faatale crash in 2016 waarbij de bestuurder  e veel vertrouwde op het semi-autonome rijtechnologie op het verkeerde type wegdek.
\cite{Puzzanghera13092017TeslaSharesBlame}
Onderzoek naar een fatale crash op 7 mei 2016 toont aan dat er beperkingen zitten aan de autopilot mode. Om specifiek te zijin is de automatische noodrem niet failsafe, blijkt uit onderzoek.
\cite{jaillet02022017teslaAutopilotLimitations}
\cite{reuters03102019teslaAutoParkingFail}
\cite{dowling23042021}
Op  April 17 2019 een autocrash waarbij het onduidelijk is of de autopiloot aan stond.
\cite{young05112021fatalTeslaReport}. Een auto ongelu waarbij een tesla is betrokken. De bestuurder was waarschijnlijk afgeleid door de games op zijn apple telefoon. De NTSB gaf aan dat het crash-avoidance systeem neit otnworpen is en ook geen crash atnuaor heeft gedetecteerd. Herdoor accelereerde de autopilot  het voertuig. Ook Faalde het systeem in het verschaffen van een crash aleter en werden de noodremmen niet geactiveerd.
\cite{tiungteslasoftwarecrash}
Er is ook een melding van een tesla waarvan de autopilot bots tegen een stilstaande politieauto
\cite{kierstein18032021teslaAutopilotCrashStationary}. Ook uit dit onderzoek blijkt dat er geen gebreken waren en dat het automaische remsysteem neit kapot was. De HNTSA concludeerder dat de bestuurder zelf geen actie ondernam door  bij te sturen of te remmen. In een eerder artikel kwam naar voren dat de tesla een autopilot krijgt die enkel camera's en GPS gebruikt; lidar of een radarsysteem wordt niet toegepast.
\cite{janssen20062017teslacrashdetailflorida}
Enkele fotos van crashes met autonome rijsysstemen \cite{saferoardsCrashesAutonomousvehicles}.
\cite{stephardson18032021revieuwingtesla}
%Onderzoeksrapport naar testla automatic vehicle control system
\cite{habib28062016NHTSATeslaReport},
\cite{darkReading17112020TeslaBackup},
\cite{heilweil26022020teslaAutopilot}
% overzicht
Tesla autopilot crashes met meer crashes en incidenten dan tot dan toe gerapporteerd
\cite{teslaFDSCrash}
De meest voorkomende crashes zijn stationaire objecten bij hoge snelheden, lane incursions from stationary objects, auti=opilot confusion at forks and gores.
\cite{teslaCrashesCauses}
\cite{teslacrashOvervieuw}
\cite{tesladeaths}
% veiligheidsrisico''
De veiligheidsrisicos van de tesla lopen uiteen. Zo zijn er risicos in de machinelearning technologie:
veiigheidsrisico Three Small Stickers in Intersection Can Cause Tesla Autopilot to Swerve Into Wrong Lane
\cite{evan01042019teslaautopilotIntersection},
\cite{lambert31062020q2safetyreport},de autopilot zelf
\cite{templeton06092019HTSBReportTesla}. Een studie door de consumntenbond in de VS toont aan dat hetautopilot systeem van de testla niet failsafe is. Zo zijn de sensoren, gebrukt voor detectie van een bestuurder negatief te beinvloeden.
\cite{dowling23042021autopilottricking} Maar ook andere problemen met de bluetooth 
\cite{wiredBloutoothHackTesla}, touch screen
\cite{preston14012021NHTSATeslaRecall},
Web-based attack crashes Tesla driver interface
\cite{leyden23032020TeslaInterfaceHack}.
Of zelfds de tesla batterij is veiligheidsvraagstuk geworden
\cite{mitchell01072020teslabatterycooling}.
Maar ook was een onderzoeker  was in staat om persoonlijke details van afgedankte voertuigonderdelen  te vekrijgen nadat deze waren afgekeurd vanwege upgrades en reparaties op consumentenvoertuigen.
\cite{stumpff04052020TeslaPersonalData}
Data-opslag in de cloud niet altijd bereikbaar.
\cite{mitchell24022020AIDataTesla}
%Wat er mis zou kunnen gegeaan wordt dru over gespeculeerd online.
%\cite{stackexchange102019teslacarmistake}
dodelijk ongeluk
\cite{fottrell03092018TeslaSecurityChecks},
softwarefout maakt diestal mogelijk
\cite{kirk26112020modelX}
fouten ontdekt in onderzoek
\cite{bbc24022021hyundaiBatteryFireFix},
tesla cloud gehacked
\cite{hawkins22102022}.
%Het AI aloritme vn Tesla
%\cite{rangaiah25022020teslaAI}
%Waarom deeplearning geen zelfrijdende auto's zal voortbrengen
%\cite{bdickson29072020teslalevelfive}
%Een survey naar de tesla gebruikers.
%\cite{randall05112019modelSurvey}
%maatschappelijk probleem
This analysis considers the potential impacts of completely self-driving vehicles on vehicular liability. 
\cite{griemannExaminSelfDriving}
Dan zijn er nog maatschappelijke problemen die de aanpak moeilijker maken.
Er is in de vs in verschillende staten een andere wetgeving
\cite{berry21042021teslacrashtexas}
\cite{hull23072021regulatorsaftercrash}
\cite{wikiTeslaAutopilot}
%oplossingen
Toch zijn er oplossingen en tegenmaatregelen.
tesla gaat advanced driver assistance systems inzetten met behulp van  passive visual, ultrasonic, en radar.
\cite{tasking07062017TeslaAugmentedSafety},\cite{ackerman01072016TeslaImperfect}
Safe system solutions door David Harkey
\cite{Harkey30052019SafeSystemVehicle}
%maatregelen
Voor elke auto uitgerust met een level 2 tot level 5 autonomy wordt nu standaard een rapport van van de crash opgvraagd door de NTSA. Dit in het kader van verder onderzoek waarbij de autoritait kijk naar  ziekenhuisbehandeling, fataliteit, airbag deployment.
\cite{szymkowski29062021nhtsaTeslaCrashReports}. 
%%%%%%%%%%%%%%%%%%%%%%%%%%%%%%%%%%%%%%%%%%%%%%%%%%%%%%%%%%%%%%%%%
\newline \indent De slmramp op  07/06/1989 \cite{espnSLMterugblik},\cite{dennisRosier01052020}
\cite{hassing07062020slmramp},\cite{amsterdamArchiefSLM},\cite{rtvOost06062019nabestaande},
\cite{breda07062021AndroSnel},\cite{andereTijdenSLMCrash},
\cite{aviationReport},\cite{aviationSLMCrashAccidentInvestigation},\cite{mcDonnelDouglasCommissionReportSLMCrash},
\cite{wikiSRFlight764},\cite{nos07062019SLMTerugblik},\cite{dagvantoenSLMCrash},\cite{waterkantNesty07061989},\cite{eduNandlalSRCrash},\cite{oldjetsSRAirways},\cite{cloudberg02012021srflight764},\cite{apnews07061989srplanecrash}.
%%%%%%%%%%%%%%%%%%%%%%%%%%%%%%%%%%%%%%%%%%%%%%%%%%%%%%%%%%%%%%%%%
\newline \indent De schipholbrand op 27/10/2005\cite{schipholbrand27102005video},\cite{schipholbrand27102005video},\cite{onderzoeksraad2610schipholoost},
\cite{schipholbrandvideoargos},\cite{nunl30052023feitenoverzicht},\cite{parlementairemonitorschipholbrand},\cite{videonpoNOVA13112008},\cite{rizoomes01052014schipholbrand},\cite{heuvelkroesschipholbrandcamerabeelden},
\cite{wikiSchipholbrand},\cite{schipholbrand27102005video},\cite{onderzoeksraad2610schipholoost},\cite{schipholbrandvideoargos},\cite{nunl30052023feitenoverzicht},\cite{singeluitgeverijenSchipholbrand},\cite{eenvandaagschipholbrand},\cite{parlementairemonitorschipholbrand},
\cite{videonpoNOVA13112008},\cite{rizoomes01052014schipholbrand},\cite{heuvelkroesschipholbrandcamerabeelden}. 
%%%%%%%%%%%%%%%%%%%%%%%%%%%%%%%%%%%%%%%%%%%%%%%%%%%%%%%%%%%%%%%%%
\newline \indent De explosie tanjin china 12/08/2015. 
Op 12 augustus 2015. Er waren twee explosies bij de Rulthai logistiek  faciliteit zorgde voor de opslag vn  gevaarlijke stoffen. De explosie zorgde voor de vernietiging van 12000 voertuigen, schade aan 17000 huize binnen een traal van 1 km. Er waren 173 doden inclusief brandweermensen.
Een van de explosies zorgde voor  een beving van 2.3 op de schaal van rigter.
De volgende factoren zouden een rol hebben gepeeld:
Een onjuiste afbakening van het opslagmaeriaal
Er was  weinig kennis bij de autoriteiten over  opslagmaterialen. Zo bleek er 7000 ton aan materiaal opgeslagen, dat is ruim 70 keer te maximaal toegestande hoeveelheid. 
Onverenigbaar grondgebruik in de nabije omgeving. Veel woonwijken met nar schatting 6000000 bewoners en 500 lokale bedrijvenin de buurt van de opslag gevaarlijke stoffen.
Opgeslagen materialen  waren: calcium carbine, sodium nitraat, potassium nitraat, amminiak nitraat en cyanide.
Ook is er veel kritiek geweest op de acties van de autoriteiten. Zo was er censuur vanuit de overheid op de journalistiek.
Ook was er naar alle warschijnlijkheid sprake van corruptie. Zo bleek achteraf dat een van de grootste aandeelhouders Dong Shexuang de zoon te zijn van een oud-politiechef in Tanjin haven, genaamd Dong Pijun
De overheid beloofde strengere toezicht en alle bedrijven moeten een risico-inventariatie maken en onderhouden\cite{jiang16042019TanjinExplosion},
\cite{staff31082015tanjinblastunrevealed},\cite{chinafile18082015tanjinexplosion},
\cite{pinghuang2410201TanjinFactreport},\cite{portoTanjinExplosionSight},\cite{imago17082015TanjinApartmentImages},\cite{trager14082015Chemicalblast},\cite{pangeramo27082015TanjinExplosion},\cite{ap06082020ammaniumnitrate},
\cite{morris14082015TanjinIndustryImpact},\cite{milesyu20082015exposingtoxicgovlines},\cite{artemis30032016tanjininsurance},\cite{aidenxiatanjinblast},
\cite{danwangTanjinflexreport},\cite{keyHighlightsTanjin},\cite{hartley13082015videofootage},\cite{odonnel01062017firetanjinblast2015},
\cite{fan15082015newyorkermistrustchina},\cite{yanlidongchinamediaframingTanjin},\cite{evans27092017TnjinInsurance},\cite{jasi26032019chineschemplant},\cite{shiqingTanjinExecutiveSentence},\cite{sophiebeach15082015},\cite{hamzeh05082020BeirutBlast},\cite{chemwatch18082015TanjiinExplosion},
\cite{thehindu15062019chinaExplosion},\cite{santagotimes24032019chinablast},
\cite{klingecorp28042020causedTanjin},\cite{mcgarryExplosions2017},\cite{roswnfeld13082015TanjinReports},
\cite{aria12082015explosionaTanjin},\cite{tremblay11022016chineseInvestigatorsTanjin},\cite{taylor13082015TanjinExplosianAftermath},
\cite{associatedPresss13082013},\cite{un20082015InvestigationTanjin},\cite{france2412082015TnjinExplosion},\cite{npr14082015TanjinCause},\cite{bbc05022016TanjinResponsibles},\cite{CBodeen15082015TanjinExplosion},\cite{reutersTanjinInsurance},\cite{yu082016evaluationTanjin2015},\cite{wiki2015TanjinExplosions},\cite{bbc17082015whathappenedTanjin},
\cite{mortimer19082016taijinexplosioncrater},\cite{internationallabourofficeChmControlTooliit},\cite{euTaxationCustomsICSC},
\cite{iloWHOChemSafetyCards}.
%%%%%%%%%%%%%%%%%%%%%%%%%%%%%%%%%%%%%%%%%%%%%%%%%%%%%%%%%%%%%%%%%
\newline \indent  De ethiopian airlinesop 10/03/2019\cite{caliskan09112013747boeingkalman},\cite{gates18112020boeingcrisis},
\cite{boeing737maxsoftwareprobles},\cite{avetisov19032019boeingmalwarestate},\cite{thompson23112020nationalsecurityboeing},
\cite{wiki737maxgroundings},\cite{campbell02052019boengcrashhumanerrors},
De oorzaak is de MCAS
\cite{hawkins22032019737maxairplanes},\cite{barnett05052019737maxcrisis}, \cite{thomas30082020737safest},\cite{boyle18112020737maxupgrade},\cite{bergstraburgess122019737maxMcasAlgorithm},\cite{737mcas},\cite{german190620217372yaftergrounded},\cite{beningo02052019boeinglessons},\cite{bloomberg26092019failedpred},\cite{afacwaLostSafeguards}, als een single point of failure \cite{uran05042019SPOF}
Angle-of-attack\cite{boeing737maxdisplay},
Behalve de MCAS waren er nog andere failures\cite{fehrm24112020737changes}, en ook deze failures \cite{dohertylindeman15032019737problems}
\cite{travis18042019737maxsoftwaredevop},
%\cite{easa27012021737maxsafereturn},
safety record van de boeing
\cite{touitou11032019737tragedies},
 Oplossingen zijn \cite{caa737modifications}. 
%%%%%%%%%%%%%%%%%%%%%%%%%%%%%%%%%%%%%%%%%%%%%%%%%%%%%%%%%%%%%%%%%
\newline \indent Het mortierongeluk in Mali op 06/04/2016. Aanwezige militair brengt slachtoffer naar de fransen, vervolgens naar de Tongolezen. Maar de kwaliteit van personeel liet te wensen over.
Er werd een Nederlandse arts overgevlogen. De slachtoffers werden overgevlogen naar Gao omvervolgens te worden oergevolgen naar Nederland.
Het ongeluk werd veroorzaakt door een kapot afsluitplaatje in de mortier. De granaat opslag in een niet gekoelde container. Dan was er vocht in de fatale granaat. Zodoende werden er explosieve stoffen gevormd in de granaat.
Tijdens de oefening werden de granaten warm in de zon. De granaat stond in veilie stand kon de explosie niet voorkomen.	\cite{ovvMortierOngevalMaliVideo} 
\cite{bnnvara13062018malirapport}
\cite{eucal11012021malimissieverlengd}
\cite{nos21052014zorgenmalimissie}
\cite{meijnders}
\cite{bnrwebredactie}
\cite{keultjes01062016malimissiecoalitie}
\cite{veenhof18012019}
\cite{isitman06012016militair}
\cite{nporadio11072016filmdemissie}
\cite{parlementairmonitor15122013mortierongeluk}
%%%%%%%%%%%%%%%%%%%%%%%%%%%%%%%%%%%%%%%%%%%%%%%%%%%%%%%%%%%%%%%%%
\newline \indent De ramp tjernobyl 26/04/1986. \cite{INSAVienna1992Chernobyl}
De mislukte veiligheidscontrole op 26 apeil 1986 01.24 uurin de sovjetuni leiddte tot explosies in een van de reactoren in de kerncentrale. De reactoren hadden geen veiligheidomhulling en de reactor bevat grote hoeveelheden brandbaar grafiet.
Door de explosie en de brand kwamen er radioactieve stoffen vrij.het gaat helemaal mis in de kernreactor 4. De warmteproductie nam  toe met een explosie tot gevolg.
31 mensen kwamen om, waaron veel mensen dagen later door stralingsziekte.
\cite{wikiTjernobyl},
\cite{rivmTjernobyl},
\cite{andereTijdenTjernobyl},
\cite{kingskey19042022tjernobyl},
\cite{erikbork26042023reactor4},
\cite{nosTjernobyl30jaarlater},
\cite{knmi04052021tjernobylbosbrand},
\cite{dodonovaKVIRisicoTjernobyl},
\cite{dumarey04062020verhaalTjernobylWaarheid},
\cite{sparkesNewScientistTjernoby},
\cite{kernenergiened26041986chronologiemaatregelen},
\cite{mapszoneReactor},
\cite{kernhistoriek15062021tjernobyl},
\cite{nucleairforumFeitenTjernobyl},
\cite{kernongevalTjernobylFancGov},
\cite{arendswolters062019lessenTjernobyl},\cite{damveld08052020tjernobyl},
\cite{deVriestjernobylHolland},\cite{ing3enieur29042015antistralingskoepel},
\cite{verschuur14012013tjernobylreports},\cite{paperlessarchivesTjernobyl},\cite{vargos082000tjernobylconcerns},\cite{mauroNuclearRiskSociety},\cite{vienna06092005LookingBack}
%%%%%%%%%%%%%%%%%%%%%%%%%%%%%%%%%%%%%%%%%%%%%%%%%%%%%%%%%%%%%%%%%
\newline \indent  Research case: De digitale aanval op de Oekrainese krachtcentrale op 23,december 2015

Op 23,december 2015  vind er een cyber aanval plaats op het elektriciteitsnet van de Oekraine. Dit was de eerste bekende aanval op een elektrisch contole  system.  Dit verslag geeft inzage in een analyse van de Ukraine cyber aanval,
inclusief hoe de actoren zich zelf toegang gavan tot het controle systeem, welke methoden de acoren hebben gebruikt voor reconnaissance en vastleggen van het systeem, een gedetailleerde omshrijving van de aanval op 15 December 2015, en de methoden die gebruikt zijn door de aanvallers om hun sporen uit te wissen en daarmee het het stoppen van schade toebrengen  nog moeilker maken. Daarnaast wordter  een gedetailleerde omschrijving gevevenv an de beveiliging van de SCADA ccontrol systemen gebaeerd op bst practices, inclusief het control network ontwerp, technieken voor whtelisting, monitoring en loggen, en  opleiding van personeel.
\cite{Whitehead2017ukrainepoweroutage}
\cite{noauthor_2022-nm}
\cite{zetter2016GridHack}
\cite{owens21032017ukrainemitigationstrategies}
\cite{cerulus2019FrontlineRussiaAttack}
\cite{grammatikis2019AttackIEC6087505104}
\cite{hidajat2016ScadaSimulator}
\cite{uscert20072021crashmalware}
\cite{zetter12062017malwareanalysis}
\cite{icsRussianHackingCyberWeapon}
\cite{usgovC2M2}
Dit verslag geeft inzage in een analyse van de Ukraine cyber aanval,
inclusief hoe de actoren zich zelf toegang gavan tot het controle systeem, welke methoden de acoren hebben gebruikt voor reconnaissance en vastleggen van het systeem, een gedetailleerde omshrijving van de aanval op 15 December 2015, en de methoden die gebruikt zijn door de aanvallers om hun sporen uit te wissen en daarmee het het stoppen van schade toebrengen  nog moeilker maken. Daarnaast wordter  een gedetailleerde omschrijving gevevenv an de beveiliging van de SCADA ccontrol systemen gebaeerd op bst practices, inclusief het control network ontwerp, technieken voor whtelisting, monitoring en loggen, en  opleiding van personeel.
\cite{Whitehead2017ukrainepoweroutage},\cite{zetter2016GridHack},\cite{boozallen2016lightwentout},\cite{finklejan2016UsBlamesRussianSandworm},\cite{desarnaud2017cyberattacks},\cite{caseli04112016intrusiondetectioncontrolsystem},\cite{rochascadatesting},\cite{hidajat2016ScadaSimulator},\cite{zetter2017moreDangerousMalware}.
Oop 23,december 2015  vind er een cyber aanval plaats op het elektriciteitsnet van de Oekraine. Dit was de eerste bekende aanval op een elektrisch controle  system met corrupte firmware. Daarnaas wordt er een telecom-based denial of service attack met  geautomatieerde systemen om het telefoonverkeer uit te schakelen.
\cite{Whitehead2017ukrainepoweroutage}
Uit onderzoek\cite{zetter2016GridHack} naar de aanval,  uitgevoerd door Oekraiene sen Amerikaanse militairenblijkt  bleek onder meer dat de power grids in sommige gevallen beter waren beveiligd dan de Amerikaanse. Desondanks was de viligheid niet optimaal door onder andere de  hetgegeven dat werknemers op afstand konden inloggen en geen gebruik van 2-stapsverificatie.
Oekraine wijst naar de russen \cite{zetter2016GridHack}, 
\cite{greenberg2017Cyberwartestlab},
\cite{boozallen2016lightwentout},
\cite{finkle08012016russiansandwormhackers},
\cite{zinets15022017ukrainechargesrussia},
\cite{mcelfresh2016cyberattackhowandwhy},
\cite{parkwalstorm11102017russiagridattack}.
{Situatie Oekraiene}
\cite{drago2017CrashOverride},
\cite{slowik2019ReassasUkraine2016Attack}.
{Situatie algemeen}
\cite{cerulus2019FrontlineRussiaAttack},
\cite{desarnaud2017cyberattacks},
\cite{dragos2019TargetedTransStation}.
{Factoren}
\cite{shehod2016gridadvantageus}
{Oorzaak}
\cite{rocha2017cybersecyrityanalysisScada},
\cite{2017crashoverridenostuxnet},
\cite{vijayan2017firstmalwareCausedOutage},
\cite{slowik2019ReassasUkraine2016Attack}.
{Gebruikte materialen}
\cite{2015ukrainegridattack},
\cite{industroyershortfact}
{Uitvoering van de aanval}
\cite{Whitehead2017ukrainepoweroutage},
\cite{boozallen2016lightwentout}.
{Oplossingen}
~\cite{Whitehead2017ukrainepoweroutage}
\cite{Whitehead2017ukrainepoweroutage}
\cite{boozallen2016lightwentout}
{spearfishing}
{blackenergy}
{remote access capabilities}
{serial-to-ethernet communication devices}
{telephony denial of service attacks}
{oplossingen}
Identificeer alle risicos en schrijf een plan foor het managen van de risico's.
Implementeer  effecteve controle  om het riico te managen.
Creeer een diepgaand model dat ervoor zor dat er efectieve en efficiente security controls worden uitgevoerd.
Aangaande de gebeurtenissen in de oekraiene kunnen de volgende security controls worden opgenomen in het securitymodel: Initial access to enterprise network, pivot in interprise network, elevate priviliges, maintainance access, gain access to control system, attack, attack complication, destroy hard drives.
\cite{Whitehead2017ukrainepoweroutage}
{Discussie}
{Verder lezen}
\cite{shahzad2014ScadaProtocolsPollingScenario},
\cite{grammatikis2019AttackIEC6087505104},
\cite{2017win32industroyer},
\cite{yadav2020reviewScadaArchitecture},
\cite{arrizabalaga2020surveyiiotProtocols},\cite{fauri2017EncryptionICS},\cite{resch31102019IEC62351secureCommunication},\cite{levalle2020FuzzingICSProtocols},\cite{blackhatusa2017},\cite{blackhatusa2017},\cite{abb30062017crashoverridenotification},\cite{spinner2018crashoverrideiot},\cite{njccicthreat08102017crashovverrideprofile},\cite{slowikvb2018crashoverride},\cite{crashoverridenetwork},\cite{wikiindustroyer},\cite{icsSecurityRussianHacking},\cite{holappa2017threattoElectricityNetworks}.
%%%%%%%%%%%%%%%%%%%%%%%%%%%%%%%%%%%%%%%%%%%%%%%%%%%%%%%%%%%%%%%%%
\newline \indent Dan zijn er nog andere ongelukken met de stint, de shietpartij op militairencomplex in ossendrecht, stint-ongeluk, de enschedese vuurwerkramp en de molukse treinkaping. Meer recentelijk de coronacrisis.
%%%%%%%%%%%%%%%%%%%%%%%%%%%%%%%%%%%%%%%%%%%%%%%%%%%%%%%%%%%%%%%%%





\paragraph{Safety critical systems}
\cite{winceckCriticalToSafety}
\cite{chambersHazardAnalysisSCS}
\cite{rslater1998SCSAnalysis}
\cite{knightchallengessafetyCritical}
\cite{johnson2006devsafetycritical}
\cite{daucriticalsafetyconsider}
\cite{fallsafedesign}
\cite{arForce2015VerificationExpectations}
\cite{nebulaassessment}
\cite{lalaArchitecturalPrinciples}
\cite{mitNotesSafetyCritical}
\cite{britishColumbia2020GuideSafetyCritical}
\cite{fulvio1993safetycriticalsystems}
\cite{dlrtabid}
\cite{knight2010SafetyCritical}
\cite{creavisafecritical}
\cite{valdes2018SafetybyAutomation}
\cite{2015whensafetymanagementsystemsfail}
\paragraph{Ondeerzoeksresultaten naar sluisbeveiliging}



Verouderde computersystemen zijn door de jaren heen gekoppeld aan netwerken, zodat ze op afstand te besturen zijn. Dit zorgt ervoor dat systemen kwetsbaar zijn voor aanvallen van buitenaf. De beveiliging is in de loop der jaren niet voldoende ontwikkeld om de infrastructuur goed te beveiligen.

Volgens het onderzoek is er de afgelopen jaren wel het nodige geïnvesteerd om de beveiliging op te schroeven, maar deze maatregelen zijn nog onvoldoende doorgevoerd.
https://www.nu.nl/internet/5814282/rekenkamer-waterwerken-niet-goed-beveiligd-tegen-cyberaanvallen.html
\cite{hdsr30092022lichtprojectieswaterliniesluizen}
rapport Digitale dijkverzwaring: cybersecurity en vitale waterwerken 
Crisisdocumentatie is verouderd en er worden geen volwaardige pentesten uitgevoerd. Uit het onderzoek blijkt dat nog niet alle vitale waterwerken rechtstreeks zijn aangesloten op het Security Operations Center (SOC) van Rijkswaterstaat. Hierdoor bestaat het risico dat RWS een cyberaanval niet of te laat detecteert. De minister van Infrastructuur en Waterstaat moet nog stappen zetten om aan de eigen doelstellingen voor cybersecurity te voldoen
De Algemene Rekenkamer beveelt de minister van Infrastructuur en Waterstaat ook aan om het actuele dreigingsniveau te onderzoeken en te besluiten of extra mensen en middelen nodig zijn. Ook is het voor een snelle en adequate reactie op een crisissituatie van essentieel belang dat informatie up-to-date is. Pentesten zouden integraal onderdeel uit moeten maken van de cybersecuritymaatregelen bij vitale waterwerken. Verder zou moeten worden bezien of medewerkers van het SOC beter moeten worden gescreend.

\cite{kramerZeeland}
Sluis Eefde kreeg niet alleen de onderhoudsbeurt, maar werd tevens uitgebreid met een tweede sluiskolk. Zo wil Rijkswaterstaat wachttijden voor de scheepvaart voorko

\cite{gww29032021kantelendesluisdeur}
Om de lokale bemanning, die de oren en ogen waren van de sluizen, te vervangen waren camera’s, communicatielijnen en software nodig. Hoge kwaliteit videobeelden, met echte kleuren en zonder enige vertraging zijn belangrijk voor de operators en zij moeten hierop kunnen vertrouwen. Er zijn verschillende testen gedaan met diverse camera’s en cameraposities om kleurechtheid te kunnen bieden onder alle omstandigheden. Het resultaat was een perfecte kleur op alle 70+ camera’s op iedere locatie.

Vertraging van videobeelden was een cruciale factor in dit project. Het is uiterst belangrijk dat de operator op zijn beeld ziet wat er daadwerkelijk op locatie gebeurt, zonder enige vertraging. Om te laten zien of er eventuele vertraging is, is er een speciale functie gecreëerd. Deze functie laat een rood kruis zien op het scherm wanneer de vertraging meer is dan 500 miliseconden. Zo ziet de operator direct of het beeld wat hij ziet actueel is. 

Een andere functie die voor dit project is gecreëerd, is bij de videobeelden aan te geven van welke kant van de sluis het camerabeeld is. Voor de operators is het belangrijk dat ze weten vanaf welke kant het vaartuig komt en waar deze naartoe vaart. Een simpele oplossing was om een blauw kader te maken om het videobeeld van de ene kant van de sluis en geen kader om het videobeeld van de andere kant. 


\cite{thkwaterwerken}
Het crisismodel kan beter, is de derde deelconclusie van de Algemene Rekenkamer. Er is geen specifiek scenario voor een crisis die wordt veroorzaakt door een cyberaanval. Ook ontbreekt inzicht in de effecten van een cybercrisis op andere sectoren, de zogeheten cascade-effecten. Tevens is de crisisdocumentatie op onderdelen verouderd.

\cite{rekenkamercybersecWater}
Ook maakt cyberveiligheid nog geen volwaardig onderdeel uit van reguliere inspecties.’ De Rekenkamer hamert erop dat alle vitale waterinfrastructuur zo snel mogelijk op het SOC wordt aangesloten. Ook zouden werknemers van Rijkswaterstaat die belangrijke waterkeringen bedienen beter gescreend moeten worden op hun antecedenten. Sollicitanten hoeven nu slechts een Verklaring Omtrent Gedrag te overleggen, maar dat is een heel lichte toets.

\cite{hackerWaterwerk}
deltawerken

\cite{kramerZeeland}
Volgens Rijkswaterstaat is het kostbaar en technisch uitdagend om klassieke automatiseringssystemen te moderniseren en wordt er daarom vooral ingezet op detectie van aanvallen en een adequate reactie daarop.
Uit het onderzoek blijkt dat Rijkswaterstaat de afgelopen jaren zelf van alle tunnels, bruggen, sluizen et cetera heeft vastgesteld welke cyberveiligheidsmaatregelen moeten worden genomen. Een groot deel van die maatregelen (ongeveer 60\%) was begin 2018 ook al uitgevoerd, maar Rijkswaterstaat ziet onvoldoende toe op de uitvoering van het resterend deel en heeft geen actueel overzicht van de overgebleven maatregelen.
De minister heeft een aantal waterwerken die Rijkswaterstaat beheert als vitaal aangewezen. . Uit het onderzoek blijkt dat nog niet alle vitale waterwerken rechtstreeks zijn aangesloten op het Security Operations Center (SOC) van Rijkswaterstaat. De ambitie om eind 2017 bij alle vitale waterwerken cyberaanvallen direct te kunnen detecteren was in het najaar van 2018 daarmee nog niet gerealiseerd. Hierdoor bestaat het risico dat RWS een cyberaanval niet of te laat detecteert.

\cite{cybersecWaterwerk}
Over de cyberbeveiliging van gemeenten en waterschappen wordt al langer geklaagd. Zo meldde EenVandaag al in 2012 dat rioolgemalen en sluizen gemakkelijk van afstand te bedienen waren, onder meer door bijzonder slechte wachtwoorden.

\cite{cybersecWaterschappen}
Rittal doet onderzoek naarop afstand besdienbare sluizen

\cite{cybersecZuidHolland}
Beveiligde VPN
M2M Services levert aan inmiddels 220 gemeenten en waterschappen beveiligde connectiviteitsoplossingen voor het beheer van pompen, riolen en gemalen. Om risico’s op beveiligingsincidenten te voorkomen maken wij gebruik van een VPN oplossing, waarbij de verbinding optimaal beveiligd is middels encryptie en authenticatie.

\cite{waterwerkNED}
Veiligheid op het water én op het land
Gebruik van lampbewaking 

\cite{veiligheidwaterland} 



\paragraph{ethiek}


Ethiek 



persuasive technology 
https://www.humanetech.com/youth/persuasive-technology 
\cite{humanTechpersuasiveTech}
https://www.minddistrict.com/blog/persuasive-technology-new-insights-in-behavioural-change 
https://www.sciencedirect.com/book/9781558606432/persuasive-technology 
https://spectrum.ieee.org/how-persuasive-technology-can-change-your-habits 
\cite{rezenfeld01012018persuasiveTecgHabits}
https://www.frontiersin.org/articles/10.3389/frai.2020.00007/full 
\cite{aldenaini28042020persuasiveTechTrends}
https://psmag.com/environment/captology-fogg-invisible-manipulative-power-persuasive-technology-81301 
\cite{larson14062017persuasivetechmanipulates}
https://www.makeuseof.com/what-is-persuasive-technology/ 
\cite{tanzem22012022persuasivetechchanginglives}
https://lib.ugent.be/catalog/rug01:001235489 
https://cyberpsychology.eu/article/view/12270 
\cite{tikkakuddonenpersuasiveTechnology}
%%%%%%%%%%%%%%%%%%%%%%%%%%%%%%%%%%%%%%%%%%%%%%%%%%%%%%%%%%%%%%%%%
\paragraph{Afbakening van requirements Wet en regelgeving voor sluizen}
Omdat we in deit onderzoek uitgaan van het uitbreiden van bestaande sluizen is er literatuurstudie gedaan naar sluizen. In de archieven van het ministerie van verkeer en waterstaat is er het rapport Design of waterlocks\cite{CivilEngineeringDivision}.
Het programma van requirements kunnen we in ons model niet helemaal overnemen. 
Zo zijn er precondities zaols topgrafie,bestaande watersluizen,waterlevel, wind, morphologie en bodemeigenschappen.

 

Preconditions
Topography
By means of maps (land, water, river, sea, ownership, regional and zoning plans) a detailed description
of the environment should be provided, including any planned changes to existing situations, in so far
as this is of importance to the lock and adjoining lock approaches. Special attention should be paid to
historical, natural and scientific values. The maps should also show sewerage, cables and mains as well
as drainage facilities in the area concerned.
Existing lock (locks)
Water levels (approx.)
Wind
%%%%%%%%%%%%%%%%%%%%%%%%%%%%%%%%%%%%%%%%%%%%%%%%%%%%%%%%%%%%%%%%%
Morphology
Soil characteristics
Functional requirements
Functional requirements regarding navigation
%%%%%%%%%%%%%%%%%%%%%%%%%%%%%%%%%%%%%%%%%%%%%%%%%%%%%%%%%%%%%%%%%
General
Lock approaches
Primarily as part of the traffic management in locking
Stop over harbour
Harbour of refuge
Compulsory harbour
Hazardous substances
Leading jetties
Chamber and heads
The principal dimensions
The design
The facilities and equipment
Functional requirements regarding the water retaining (structure)
%%%%%%%%%%%%%%%%%%%%%%%%%%%%%%%%%%%%%%%%%%%%%%%%%%%%%%%%%%%%%%%%%
\newline \indent Dan zijn er nog de functionele eigenschappen.
 Functional requirements regarding water management
General
Limiting water loss
Separation of salt and fresh water or clean and polluted water
Water intake and discharge
%%%%%%%%%%%%%%%%%%%%%%%%%%%%%%%%%%%%%%%%%%%%%%%%%%%%%%%%%%%%%%%%%
\newline \indent Functional requirements regarding the crossing, dry infrastructure
Roads
Cables and mains
%%%%%%%%%%%%%%%%%%%%%%%%%%%%%%%%%%%%%%%%%%%%%%%%%%%%%%%%%%%%%%%%%
\newline \indent  User requirements
%%%%%%%%%%%%%%%%%%%%%%%%%%%%%%%%%%%%%%%%%%%%%%%%%%%%%%%%%%%%%%%%%
\newline \indent Levels
Locking levels
Situating the lock
Accessibility
Smoothness and safety of dealing with traffic
Design levels
Normative High Water (NHW)
Locking level high water gate
%%%%%%%%%%%%%%%%%%%%%%%%%%%%%%%%%%%%%%%%%%%%%%%%%%%%%%%%%%%%%%%%%
\newline \indent Possible preference for separating different kinds of vessels
Separation in using line-up area, waiting area and chamber
Separating vessels during locking
Separation of vessels during over night stop
Separation for use of the leading jetty (leidende steiger)
Leading jetty for seagoing vessels
Leading jetty for inland navigation
Leading jetty for recreational navigation
%%%%%%%%%%%%%%%%%%%%%%%%%%%%%%%%%%%%%%%%%%%%%%%%%%%%%%%%%%%%%%%%%
\newline \indent Mooring facilities in chamber and lock approach
Chamber
Lock approaches
Leading jetty
%%%%%%%%%%%%%%%%%%%%%%%%%%%%%%%%%%%%%%%%%%%%%%%%%%%%%%%%%%%%%%%%%
\newline \indent Operating times
%%%%%%%%%%%%%%%%%%%%%%%%%%%%%%%%%%%%%%%%%%%%%%%%%%%%%%%%%%%%%%%%%
\newline \indent Levelling times
%%%%%%%%%%%%%%%%%%%%%%%%%%%%%%%%%%%%%%%%%%%%%%%%%%%%%%%%%%%%%%%%%
\newline \indent Operational management
%%%%%%%%%%%%%%%%%%%%%%%%%%%%%%%%%%%%%%%%%%%%%%%%%%%%%%%%%%%%%%%%%
Process descriptions
%%%%%%%%%%%%%%%%%%%%%%%%%%%%%%%%%%%%%%%%%%%%%%%%%%%%%%%%%%%%%%%%%
Normal locking process
%%%%%%%%%%%%%%%%%%%%%%%%%%%%%%%%%%%%%%%%%%%%%%%%%%%%%%%%%%%%%%%%%
Obstructions
%%%%%%%%%%%%%%%%%%%%%%%%%%%%%%%%%%%%%%%%%%%%%%%%%%%%%%%%%%%%%%%%%
High water retaining structure
%%%%%%%%%%%%%%%%%%%%%%%%%%%%%%%%%%%%%%%%%%%%%%%%%%%%%%%%%%%%%%%%%
Intake/discharge
%%%%%%%%%%%%%%%%%%%%%%%%%%%%%%%%%%%%%%%%%%%%%%%%%%%%%%%%%%%%%%%%%
Salt /freshwater or clean/polluted water
%%%%%%%%%%%%%%%%%%%%%%%%%%%%%%%%%%%%%%%%%%%%%%%%%%%%%%%%%%%%%%%%%
Information for operational management
%%%%%%%%%%%%%%%%%%%%%%%%%%%%%%%%%%%%%%%%%%%%%%%%%%%%%%%%%%%%%%%%%
Procedures and facilities for negative operational situations
%%%%%%%%%%%%%%%%%%%%%%%%%%%%%%%%%%%%%%%%%%%%%%%%%%%%%%%%%%%%%%%%%
Power supply
Levelling%%%%%%%%%%%%%%%%%%%%%%%%%%%%%%%%%%%%%%%%%%%%%%%%%%%%%%%%%%%%%%%%%
Collisions
%%%%%%%%%%%%%%%%%%%%%%%%%%%%%%%%%%%%%%%%%%%%%%%%%%%%%%%%%%%%%%%%%
Too low/too high water levels and inspections
%%%%%%%%%%%%%%%%%%%%%%%%%%%%%%%%%%%%%%%%%%%%%%%%%%%%%%%%%%%%%%%%%
Problems with ice
%%%%%%%%%%%%%%%%%%%%%%%%%%%%%%%%%%%%%%%%%%%%%%%%%%%%%%%%%%%%%%%%%
\newline \indent Operating
Situating the control building
Local control facilities
Means of communication
Choice (partly) automated and self-service
Remote control of locks
%%%%%%%%%%%%%%%%%%%%%%%%%%%%%%%%%%%%%%%%%%%%%%%%%%%%%%%%%%%%%%%%%
\newline \indent Illumination, signalling and boarding
Illumination (for details, see Lit. [2.1])
Ship crews and operating personnel must take into account that comfort is decreased during locking that
takes place through the night. Given the decreased visibility and orientation, extra effort is required. This
effort has to be kept as low as possible in order to prevent decreased safety. For this purpose, suitable
and economically sound illumination of the lock complex is essential.
The lighting has to be geared to the ever-increasing use of central control at locks and has to be aimed
at places where activities (manoeuvres, tying and untying, going on land) are executed.
The locations drawing the attention of the individual captain for instance, are the free area, the line-up
and waiting area, the chamber entrance, the chamber, lock grounds, chamber exit and the outlet area
to the unlit waterway. The attention of operating personnel will particularly focus on the vessels in the
line-up and waiting areas, inbound vessels, the chamber, the gates, the lock grounds and the sailing of
outbound vessels.
Given the necessity of illuminating the lock and lock approaches, a number of general minimum conditions
are set. This illumination is compulsory and could be included in the design plan:
• a clear view of the lock complex has to be provided for the benefit of orientation from the water;
• the illumination has to be sufficiently even;
• during arrival and departure dazzling, which is often caused by excessive glare of lock parts because
of cameras etc., should be prevented;
• in the control building the illumination should be adjusted to the outside environment and images
recorded as TV pictures should have such contrast and definition that the operating personnel is given
sufficient information;
• uniformity in the illumination plan for the setup of light towers, height of points of light and light
colour is desired.
In Lit. [2.1], as extension of these conditions, a number of specific recommendations are made that are
of importance to the design.
%%%%%%%%%%%%%%%%%%%%%%%%%%%%%%%%%%%%%%%%%%%%%%%%%%%%%%%%%%%%%%%%%
Required illumination level
For the average value of illumination intensity on horizontal surfaces of the above-mentioned lock
parts, 10 lux is adhered to. On vertical surfaces that are more often more striking due to the perpendicular
directional view, a lower value of 3.5 lux can be used.
At a number of critical parts of the lock (both for the captain and the lock master) a larger contrast is
desired and can be achieved by stronger illumination of areas that should be in the light or providing
these with white markings. The latter is preferable. At critical lock parts such as gates and leading
jetties, the vertical illumination strength should be higher: 7 lux. On the chamber and mooring area
where accurate visibility is required, the previously stated values of 10 lux for horizontal and 3.5 lux
for vertical apply. The waiting area and the free area, where illumination is mostly for orientation,
require an illumination level of 5 lux horizontal respectively 3.5 lux vertical.
%%%%%%%%%%%%%%%%%%%%%%%%%%%%%%%%%%%%%%%%%%%%%%%%%%%%%%%%%%%%%%%%%
Surrounding illumination and guidance
Misleading illumination in the surrounding area can give the captain a wrong picture of the course of
the waterway that provides access to the lock chamber. This can be prevented if the waterway or the
lock complex is illuminated over a sufficient length or by adapting the surrounding illumination to the
illumination of the complex. For visual guidance, differences in illumination strength at crossings
should not exceed a factor 2.
%%%%%%%%%%%%%%%%%%%%%%%%%%%%%%%%%%%%%%%%%%%%%%%%%%%%%%%%%%%%%%%%%
Uniformity
For the uniformity (E) of the illumination, a minimum value of Emin/Emax = 0.3 should be adhered
to for both vertical and horizontal areas.
%%%%%%%%%%%%%%%%%%%%%%%%%%%%%%%%%%%%%%%%%%%%%%%%%%%%%%%%%%%%%%%%%
Glare
Unsafe situations due to dazzling should be avoided. The correct combination of armature, lamp and
positioning is of importance.
%%%%%%%%%%%%%%%%%%%%%%%%%%%%%%%%%%%%%%%%%%%%%%%%%%%%%%%%%%%%%%%%%
Colour recognition and kind of lamp
The colour of the light is one of the factors in the recognition of boards and signalling. Both white
and yellow light can be used.
In the lamp choice of illumination, both high-pressure and low-pressure lamps as well as energy
saving lamps qualify. In the application of low-pressure (monochromatic) sodium (vapour) light,
colour recognition is impossible. If this is the case, separate illumination of traffic signs is recommended.
%%%%%%%%%%%%%%%%%%%%%%%%%%%%%%%%%%%%%%%%%%%%%%%%%%%%%%%%%%%%%%%%%
Marking
White markings are a good and inexpensive tool for obtaining sufficient contrast in the dark while using
little light. Marking vertical surfaces, such as guiding structures and guard walls, to support the visual
guidance of navigation is very effective.
%%%%%%%%%%%%%%%%%%%%%%%%%%%%%%%%%%%%%%%%%%%%%%%%%%%%%%%%%%%%%%%%%
Signalling
Signalling should be executed according to the stipulations of the Police Regulations on Inland
Navigation (‘Binnenvaart Politie Reglement’ (BPR))and the Rhine Navigation Police Regulations
(‘Rijnvaart Politie Reglement’ (RPR)), (Lit. [2.4]).Signal indication and lock illumination choices should be
adjusted to terrain illumination of the lock for the benefit of colour recognition; it should have sufficient
attention value.
%%%%%%%%%%%%%%%%%%%%%%%%%%%%%%%%%%%%%%%%%%%%%%%%%%%%%%%%%%%%%%%%%
Boarding
Boards should be executed in accordance with the stipulations of the BPR and RPR, (Lit. [2.4]).The colour
recognition could be (substantially) reduced due to the terrain illumination. Sufficient attention should
be paid to adjusting the illumination or to separate board illumination.
%%%%%%%%%%%%%%%%%%%%%%%%%%%%%%%%%%%%%%%%%%%%%%%%%%%%%%%%%%%%%%%%%
Illumination plan
The user requirements for illumination should be incorporated in an illumination design plan.
The chamber depth (distance between low normative water level and the lock coping) and the chamber
width are of great importance. In Lit. [2.1] examples are provided for a number of chamber width
categories (5-13 m, 13-20 m, 20-24 m, larger than 24 m; chamber depth about. 5 m) of the resulting
illumination characteristics (such as illumination strength and uniformity), departing from the relationship
between lock design and the given characteristics of illumination installation (such as positioning
and illumination facilities).
%%%%%%%%%%%%%%%%%%%%%%%%%%%%%%%%%%%%%%%%%%%%%%%%%%%%%%%%%%%%%%%%%
\newline \indent Power supply
Emergency power supply is required for vital parts of the installation so that, in case of malfunction,
it can automatically take over the energy supply within minutes. A no-break facility is required for
installation parts that lose data in case of power loss. In addition, emergency lights should be present.

In essence, power is obtained from the public network. In consultation with the local power company,
assessments have to be made about where this is possible and whether the connection contains
sufficient capacity or whether this will have to be adjusted. Of importance is the total capacity required,
voltage variations and frequency of the energy to be supplied. In addition to capacity for lock operation,
the capacity for construction (civil and steel) will have to be determined. It could be taken into consideration
whether the cables for construction could later become part of the supply for the lock.
The lock complex should contain the necessary facilities for high tension, transformers and low-tension
equipment. In addition, room is reserved and facilities provided for cable location lines from the low-tension
area to the various lock parts (cable racks, cable channels, cable shafts, lead-through pipes etc.)
Take into account the other cables and mains required for lock operation as well as those for third parties
(Par. 2.3.4.2). For emergency power supply generators and no-break installations, see Par. 2.4.6.3.
%%%%%%%%%%%%%%%%%%%%%%%%%%%%%%%%%%%%%%%%%%%%%%%%%%%%%%%%%%%%%%%%%
\newline \indent Availability
Introduction
Causes of non-availability
Water levels above and below locking levels
Guidelines on the boundaries of locking levels are provided in Par. 2.4.1.1 (maximum and minimum
locking levels). Overall, this results in non-availability smaller than 2% of the time.
The specific boundaries should be set on economic grounds.
Too much wind, bad visibility
%%%%%%%%%%%%%%%%%%%%%%%%%%%%%%%%%%%%%%%%%%%%%%%%%%%%%%%%%%%%%%%%%
Malfunctions of installations, operating mechanisms and operating
Based on the previously mentioned economic considerations, requirements will have to be drafted for
the design of the lock or the series of locks for the acceptable risk of failure of these facilities. As an
example, the values applied for the renovation of the ‘Zuider- en de Kleine sluisin IJmuiden’ are stated
(Lit. [2.13]). Not available due to:
• malfunction installations : ≤ 0,5% of the time
• malfunction operating mechanisms : ≤ 0,5% of the time
• malfunction operation : ≤ 0,25% of the time
The number of times that malfunction occurs could also be a determining factor.
Not every malfunction results in complete obstruction. The objective is to limit the duration of the malfunction
as much as possible (alerting, responding, spare parts).
For emergency power supply and no-break installations, please see Par. 2.4.6.3.
%%%%%%%%%%%%%%%%%%%%%%%%%%%%%%%%%%%%%%%%%%%%%%%%%%%%%%%%%%%%%%%%%
Collisions
For non-availability due to collisions, at best a forecast can be made, based on the information available
for similar locks with a corresponding navigation volume. As an example, the ‘Zuidersluis bij IJmuiden’
(Lit. [2.13]) is mentioned, where the non-availability due to significant damage due to collisions amounted
to 17 hours per annum (about 0.2% of the time). Other locks could provide a different picture.
Within economically acceptable boundaries, the objective will be to limit the collisions and consequences
thereof. The accent is placed on gates (and operating mechanisms), moveable bridges and – to a
lesser degree – on berthing jetties and guide structures.
Measures to decrease risk of collision are, among others:
• good design of approach jetties (Par. 2.3.1.3 and 2.4.2.2);
• positioning of the flooring of moveable bridges – in opened condition – outside the outer walls of the
lock (Par. 2.3.4.1);
• anti-collision structures in front of the gates (Par. 2.4.11.1). This is an expensive facility that will only
be applied in special cases;
• protection of operating mechanism on gates. Preventing collisions with the operating mechanism can
be effected by fitting a tail end to the gate and connecting this to the operating mechanism
(Renovation Oranjesluizen). An extended operating mechanism chamber could also be used so that
the vulnerable cylinder rod cannot be hit in the lock (Middensluis IJmuiden).
Measures to limit the duration of the repairs (obstruction) are, among others, having the spare gates and
spare parts available (Par. 2.5.2 en 2.5.3).
Maintenance
%%%%%%%%%%%%%%%%%%%%%%%%%%%%%%%%%%%%%%%%%%%%%%%%%%%%%%%%%%%%%%%%%
\newline \indent Protecting constructions from damage
%%%%%%%%%%%%%%%%%%%%%%%%%%%%%%%%%%%%%%%%%%%%%%%%%%%%%%%%%%%%%%%%%
Collision protection for gates
Mitre gates and pivot (leaf) gates must be fitted with wood fender on the outside surfaces of the
opened gates to protect the construction from damage caused by inbound and outbound vessels. Wood
fender can also be fitted to other gates in places where they might be hit by vessels.
In special circumstances (for instance Wijk bij Duurstede, Tiel, Belfeld, Panheel, Twenthe-kanaal) trap
constructions are positioned in front of the closed gates. The energy of vessels that do not stop in time
is absorbed here and the construction prevents the gates from being hit (see par. 17.3.3). For this purpose,
cables (cable nets) and friction drums can be used. For the circumstances and setup of these constructions,
we refer to Lit. [2.15]. It does concern expensive constructions for which the investments will
have to be weighed against the risk of failure of the water retaining structure, the navigation interests
etc.
Anti-collision devices protecting lock gates could be economically sound at high-lift locks.
%%%%%%%%%%%%%%%%%%%%%%%%%%%%%%%%%%%%%%%%%%%%%%%%%%%%%%%%%%%%%%%%%
Collision protection for concrete and sheet pile constructions
Construction surfaces against which vessels moor or along which they shave, have to be as smooth as
possible in order to guide well and limit potential damage (construction and vessel). For inland navigation,
a concrete structure meets the requirements. In the case of other construction materials such as
sheet pile, the flat surface should be made of wooden or synthetic posts and rails wherever possible. This
system can be limited to the day surfaces that vessels meet.
Additional facilities are necessary in places where concrete surfaces are interrupted or come to an end
because of expansion joints, gate and ladder recesses. In the case of expansion joints, it will be sufficient
to use (sizeable) bevelled edges, steel corner protection profiles should be applied in recesses. Corner
guards made of tropical hardwood can also be fitted, especially where it concerns rugged navigation
such as tug-pushed dumb barges and sea-going vessels. As protection from hawsers etc, the top of the
wall should be fitted with steel capstone profiles. In locks for large ocean going vessels, floating wooden
frames (the Netherlands) or rubber wheel fenders (Belgium) are used.
The facilities are intended to minimize damage to vessels and constructions, but also to prevent backing
up and friction effects during mooring and unmooring of vessels with large side surfaces, thereby
decreasing the pass through time.
%%%%%%%%%%%%%%%%%%%%%%%%%%%%%%%%%%%%%%%%%%%%%%%%%%%%%%%%%%%%%%%%%
Facilities against vandalism
Lightning protection
%%%%%%%%%%%%%%%%%%%%%%%%%%%%%%%%%%%%%%%%%%%%%%%%%%%%%%%%%%%%%%%%%
Safety
Facilities for drowning persons
For rescuing people who accidentally end up in the water, ladders should be fitted to the chamber wall
and to (high) smooth walls in the lock approach. At the upper end, these ladders are equipped with
handgrips. For offering help from the quayside, life-saving devices (life buoy, hooks) should be present
on the lock coping in a clearly visible place. Ladders in the chamber and the lock approach also have an
accessibility function. For locations and distances, also see par. 2.4.13.2 and 2.4.13.3.
%%%%%%%%%%%%%%%%%%%%%%%%%%%%%%%%%%%%%%%%%%%%%%%%%%%%%%%%%%%%%%%%%
Safety facilities
Design and management of safety facilities of personnel will be executed in accordance with Health and
Safety Regulations, construction regulations, labour regulations and safety regulations (CE directives).
A number of facilities are mentioned below.
Railings are attached to the top of gates. If the lock coping is more than 2.5 m above minimum locking
level, fencing is placed behind the bollards. This fencing is always desirable where it concerns recreational
navigation and where tourists are allowed on the lock coping.
In the technical areas, workshops, bridges, control portals, rolling gate casings and the like, where work
is executed and people walk around where there are differences in height in the surrounding area,
railings are provided. From a height difference of 0.60 m or more with the surrounding area, a railing
has to be provided at 1 – 1.10 m. Height differences of more than 12 m require the railing to be placed
at a height of 1.20. Often, additional protection against falling is provided from height differences of
more than 2.5 m such as safety lines, lifelines, harness belts and the like.
Steel ladders should not be in regular use. Straight stairs, a spiral staircase or step ladders should be
installed. Ladders can be used between vertical (90o) and 75o and be equipped with simple round rungs.
The ladder width is between 0.38 and 0.46 m and the step distance is between 0.25 – 0.20 m.
If the ladder connects with the (landing) coping, the distance between the styles of the ladder should be
enlarged to 0.60 and it has to be connected to the railing. If the ladders are higher than 3.60 m, they
have to be provided with a safety cage. This cage has an inside measurement of 0.76 m and starts from
2.40 m above the ground. At ladder heights above 6 m, an intermediate landing is required.
Basement chambers that could possibly flood (for instance those of operating mechanisms of mitre
gates) have to be provided with an exit that can be opened from the inside. In addition, sufficient
natural ventilation will be required as well as plunger pumps.
The area in which the operating mechanisms are working need to be shielded from the environment to
ensure that nobody gets stuck between machine parts. The lock complex should have sufficient and visible
First Aid provisions.
%%%%%%%%%%%%%%%%%%%%%%%%%%%%%%%%%%%%%%%%%%%%%%%%%%%%%%%%%%%%%%%%%
Fire fighting
Accessibility of lock and lock approaches
Lock Infrastructure
Accessibility of vessels in the lock
Accessibility of vessels in the lock approache
Accessibility of vessels with dangerous goods in the lock approaches
%%%%%%%%%%%%%%%%%%%%%%%%%%%%%%%%%%%%%%%%%%%%%%%%%%%%%%%%%%%%%%%%%
\newline \indent Supplemental client wishes
%%%%%%%%%%%%%%%%%%%%%%%%%%%%%%%%%%%%%%%%%%%%%%%%%%%%%%%%%%%%%%%%%
\newline \indent Life span requirements
Design life span of lock complex
Steel parts
Electrical installations
Hardware and software
Sheet pile constructions
Guiding structures
Maintenance requirements
Maintenance strategy
Spare gates
Spare parts and materials
%%%%%%%%%%%%%%%%%%%%%%%%%%%%%%%%%%%%%%%%%%%%%%%%%%%%%%%%%%%%%%%%%
To lay lock open (or not)
Nowadays, it is no longer usual to lay open the complete lock for maintenance. The reasons are that it
is often too costly (measures required against floating up) and that the main construction of chamber
and heads are maintenance free, the probable exception being wood fenders for sheet pile constructions
and floating frames at sea locks. The latter parts should be easy to replace. Incidental repairs to head
constructions could be executed by divers or in diving bells.
Inspection and maintenance focus on gate supports (sill and side seals), fulcrums, and gate conduction,
in other words, parts that are located in the head. There are two possibilities:
1. Lying open a head, for which stop log weirs or dewatering weirs and rabbets are necessary.
2. Removable pivot-inspection chambers and other local steel dewatering means for the fulcrums,
support and gate condition. This also includes the dewatering stop logs for the gate recesses for lift
and roller-bearing gates.
Gate supports and rabbets are also required for the drainage. These means for water removal are stored
in the near vicinity in a highly accessible place and could possibly be used for several locks.
The choice between two possibilities depends on the inspection and maintenance frequency, the costs
and the duration of the obstruction for navigation. Option 1, in which too much space is laid open is, in
essence, usually only applied at smaller locks.
%%%%%%%%%%%%%%%%%%%%%%%%%%%%%%%%%%%%%%%%%%%%%%%%%%%%%%%%%%%%%%%%%
Accessibility for personnel
%%%%%%%%%%%%%%%%%%%%%%%%%%%%%%%%%%%%%%%%%%%%%%%%%%%%%%%%%%%%%%%%%
Monitoring
Monitoring is a permanent measuring and registration system for normative parameters for the condition
of structures, the loads and stresses that they are submitted to and the degree in which corrosion
processes have progressed. Even though the application in construction is still limited, it is necessary to
keep up with the rapid developments. Monitoring is useful, certainly for places of lock structures that are
difficult to inspect (for instance at soil facing side) and for erosion processes that are hardly visible on the
surface (such as chloride penetration).
Cathodic protection can be used as a monitoring system at the same time.
Electrical installation, hard- en software
Storage areas and workshops
Environmental requirements in the use phase
Aesthetics
%%%%%%%%%%%%%%%%%%%%%%%%%%%%%%%%%%%%%%%%%%%%%%%%%%%%%%%%%%%%%%%%%
\newline \indent In ons model houden we geen rekening met omgevingseisen zoals de materialen gebruiket voor de bouw, recreatie, bodemvervuiling, grondwaterverlies. Oo is er geen rekening gehouden met verkeer, communicatiekabels onderwater en netspanningskabels.

Environmental requirements with regard to building materials
Recreation
Environmental requirements in the construction phase
Required building site and final grounds
Polluted soil
Groundwater withdrawal
Upkeep/maintenance of road and navigation traffic, cables and mains
Upkeep/maintenance of the water retaining structure
%%%%%%%%%%%%%%%%%%%%%%%%%%%%%%%%%%%%%%%%%%%%%%%%%%%%%%%%%%%%%%%%%
\newline \indent Permits and procedures at the construction of a lock
Construction permits and zoning plan amendments
Demolition permit
Flood Defence Act
Environmental Management Act (M.E.R.)
Act on Earth Removal
Pollution of Surface Waters Act
Groundwater Act permit
Water management Act
Soil Protection Act
Nature Conservation Act
Management of Waterways and Public Works Act (Wet beheer RWS-werken)
Noise Abatement Act
Provincial Road Ordinance
Building Materials (Soil and Surface Waters Protection) Decree
Other permits and exemptions
Standards and guidelines
Standards
Guidelines
%%%%%%%%%%%%%%%%%%%%%%%%%%%%%%%%%%%%%%%%%%%%%%%%%%%%%%%%%%%%%%%%%
\paragraph{Checklist}


\paragraph{Analyse}
\paragraph{Conclusie}
%%%%%%%%%%%%%%%%%%%%%%%%%%%%%%%%%%%%%%%%%%%%%%%%%%%%%%%%%%%%%%%%%

\hoofdstuk{Requirements}
\paragraph{Inleiding}

De meeste sluizen die zich in Nederland bevinden zijn schutsluizen; deze sluizen zijn bedoeld om boten, zowel vrachtschepen als pleziervaart afhangend van de locatie van de sluis, te verwerken. Om deze reden gaan wij deze dus ook verwerken in ons model. Mocht een sluis niet bedoeld zijn om boten te verwerken, dan zou dit model alsnog toegepast kunnen worden opp desbetreffende sluis.
Boten worden toegevoed aan de queue. Hoe dit gebeurt, dat ligt aan de specifieke sluis.  Sinds wij een template maken, hoeven wij geen rekening te hounden met hoe de schepen in de queue komen. Het enige wat wij hoeven te doen, is de data verwerken.





\paragraph{Aandachtspunten}
\begin{enumerate}
	\item Voorrang tussen schepen onderling in de sluis?
	\item Hoe lang mag een schip zich in de sluis bevinden?
\end{enumerate} 




\subparagraph{Afbakening}
\begin{itemize}
	\item Wat doet de sluis niet.
	\item De sluiss houdt geen rekening met links of rechtsrijdend verkeer vanuit de zeevaart
	\item De sluis heeft geen queue met daarin een id gekoppeld aan de sluis.
	\item De waterpomp wordt alleen aan en uitgezet
	\item De waterpomp houdt geen rekening met waterstand
	\item Houdt geen rekening met een schip in de sluis dat is blijven hangen.
\end{itemize}


\begin{enumerate}
	\item Een tweetal sluisdeuren. 
	\item Een sluiskolk waarin de schepen in- enuitvaren
	\item een stoplicht om een signaal af te geven voor invaren en uitvaren.
	\item Een nivelleermachine zorgt ervoor dat het water in de sluis op het gewenste niveau wordt gebracht
	\item Een control-system dat ervoor zorgt dat de opdrachten van de sluisbeheerder (geautomatiseerd) worden uitgevoerd
\end{enumerate}

Een schip komt aanvaren en meld zich aan bij de sluismeester. De sluismeester geeft een signaal aan het controlsystem voor het openen van de sluisdeuren, nadat geccontroleerd is of de nivelleermachine al klaar is. Als er ruimte is voor een invarend schip mag het schip dat zoich heeft aangemeld en toestemming heeft  in de sluis varen. Op het moment dat de sluis vol is gaan de sluisdeuren dicht. Eenmaal afgesloten kan de nivelleermachine beginnen om het water in de sluiskolk op het gewenste waterpeil te brengen. Als dit nivelleerprces is afgerond geeft  het controlsystem daan da de sleusdeuren open kunnen.  Als de sleusdeuren open zijn en het uitvaarsignaal is op groen dan moet het schip in de sluis de sluis uitvaren.

Uit het zojuist genoemnde scenario valt het volgende op te maken.
\begin{enumerate}
	\item Een schip geeft een signaal aan een sluismeester.
	\item Er wordt gekeken of er wel plek is in de sluis .
	\item Er wordt gekeken of de nivelleermachine is afgerond.
	\item Er wordt gekeken wat het niveo van de waterpeil in de sluiskolk is.
	\item Er wordt gekeken of de sluisdeuren gereed zijn voor invarende schepen.
\end{enumerate}

\paragraph{overzicht}

\paragraph{Conclusie}

%%%%%%%%%%%%%%%%%%%%%%%%%%%%%%%%%%%%%%%%%%%%%%%%%%%%%%%%%%%%%%%%%
 
\hoofdstuk{Uppaal model}


\paragraph{Inleiding}






	\subsubsection{De computation tree}

\xymatrix@ur@!R=2pc{%
	*+<1pc>[o][F-]{q_0}  \ar@(l,l)[]^<<<<{start} \ar@/^/[r]^0  \ar@/_/[d]_1 
	& *+<1pc>[o][F-]{q_1} \ar@(ul,ur)[]^{0}  \ar@/^/[d]^1 \\
	*+<1pc>[o][F-]{q_2} \ar@(dr,dl)[]^{1} \ar@/_/[r]_0 
	& *+<1pc>[o][F=]{q_3} \ar@(l,l)[]^>>>>{start}  \ar@(dr,dl)[]^{1} \\
 
 }








\begin{tikzpicture}[>=latex',scale=0.5]
	% set node style
	
	\begin{dot2tex}[dot,tikz,codeonly,styleonly,options=-s -tmath]
		digraph G  {
			node [style="n"];
			p [label="+"];
			t [texlbl="\LaTeX"];
			6
			8
			10-> p;
			6 -> t;
			8 -> t;
			t -> p;
			{rank=same; 10;6;8}
		}
	\end{dot2tex}
	\begin{pgfonlayer}{background}
		\draw[rounded corners,fill=blue!20] (6.north west) -- (8.north east) -- (t.south east)--cycle;
	\end{pgfonlayer}
\end{tikzpicture}


\[\begin{tikzcd}[column sep=1cm]
	ABCDE\arrow[r, leftrightarrow, "\times"{anchor=center},"\text{label}","\text{label}"{below}]\arrow[d] & F\arrow[r]\arrow[d] & G\arrow[rr]\arrow[d] && H\arrow[d]\\
	ABCDEFGH\arrow[r, leftrightarrow, "\times"{anchor=center}]\arrow[d] & II\arrow[r]\arrow[d] & JJ\arrow[rr,"\text{very long label}"]\arrow[d] && KK\arrow[d]\\
	ABCD\arrow[r] & EEE\arrow[r] & FFF\arrow[rr] && GGG
\end{tikzcd}\]




\paragraph{Models}

\subparagraph{Maincontroller}
\[
\begin{tikzcd}%[every arrow/.append style=dash]  uncomment to remev arrowa
	& \tikz{\node[draw,circle]{1}} \ar{d}&  & \tikz{\node[draw,circle]{2}} \ar{d}  \ar{r} &  \ar{r} & \ar{r} & \ar{r} &  \ar{r}& \tikz{\node[draw,circle]{2}}\ar{d} \\
	\tikz{\node[draw,circle]{4}} \ar{d} & \tikz{\node[draw,circle]{2}}  \ar[bend right=15]{l} \ar{d} & & \ar{d} & \tikz{\node[draw,circle]{2}} &\tikz{\node[draw,circle]{2}}&\tikz{\node[draw,circle]{2}}&& \tikz{\node[draw,circle]{2}} \ar{d} &\\
	\tikz{\node[draw,circle]{4}} \ar{u} \ar{r}  & \tikz{\node[draw,circle]{3}} \ar{r}  & \tikz{\node[draw,circle]{5}} \ar{r} &  \tikz{\node[draw,circle]{6}} \ar{r}  \ar{ru} & \tikz{\node[draw,circle]{7}} \ar{r} & \tikz{\node[draw,circle]{7}} \ar[crossing over]{ul}  \ar{r} & \tikz{\node[draw,circle]{8}} \ar{r} & \tikz{\node[draw,circle]{9}} \ar{r} \ar{d} & \tikz{\node[draw,circle]{10}}  \\
	& & & &\tikz{\node[draw,circle]{1}} \ar[crossing over]{ul} \ar{ru}  & & \tikz{\node[draw,circle]{2}}  \ar[bend right=15]{r} & \tikz{\node[draw,circle]{2}} \ar[bend right=15]{l} \ar[bend right=15]{r}  & \tikz{\node[draw,circle]{2}} \ar[bend right=15]{l}
\end{tikzcd}
\]
 

\begin{tikzpicture}[>=latex',shorten >=1pt,node distance=2cm,on grid,auto,scale=0.2]

	\node[state] (q0-e) {$q_0/\epsilon$};
\node[state] (q0-1) [below right=of q0-e] {$q_0'/1$};
\node[state] (q1-0) [above right=of q0-1] {$q_1/0$};
\node[state] (q2-1) [below right=of q1-0] {$q_2/1$};

\node[state] (q3-0) [below left=of q0-1] {$q_3/0$};
\node[state,accepting] (q3-1) [below right=of q0-1] {$q_3'/1$};
\node[state] (0) [ left=of q3-0] {$q_3/0$};
\node[state] (1) [ left=of 0] {$q_3/0$};

\node[state,initial,accepting] (0) [ left=of 1] {$q_3/0$};

\node[state] (3) [ below right=of q2-1] {$2$};
\node[state] (4) [  right=of 3] {$4$};
\node[state] (5) [  right=of 4] {$5$};
\node[state] (6) [  right=of 5] {$6$};

\node[state] (7) [  above=of 1] {$7$};
\node[state] (8) [  above=of 7] {$8$};
\node[state] (9) [  right=of 5] {$9$};

\node[state] (10) [ below  left=of 4] {$10$};
\node[state] (11) [ below  right=of 4] {$11$};

\node[state] (12) [   above=of 5] {$11$};

\node[state] (13) [   above=of 12] {$11$};

\node[state] (14) [ below right  =of q3-0] {$11$};
\node[state] (15) [   below right =of 14] {$15$};
\node[state] (16) [   above right =of 15] {$16$};


\path[->] (q0-e) edge node {a} (q1-0);
\path[->] (q0-e) edge node {b} (q3-0);
\path[->] (q0-1) edge node {a} (q1-0);
\path[->] (q0-1) edge [bend right] node {b} (q3-0);
\path[->] (q1-0) edge node {a} (q3-1);
\path[->] (q1-0) edge node {b} (q2-1);
\path[->] (q2-1) edge node {a} (q0-1);
\path[->] (q2-1) edge node {b} (q3-1);
\path[->] (q3-0) edge node {a} (q3-1);
\path[->] (q3-0) edge [bend right] node {b} (q0-1);
\path[->] (q3-1) edge [loop below] node {a} (q3-1);
\path[->] (q3-1) edge node {b} (q0-1);

\path[->] (4) edge [bend right] node {b} (10);	
\path[->] (10) edge [bend right] node {b} (4);


	
\path[->] (4) edge [bend right] node {b} (11);	
\path[->] (11) edge [bend right] node {b} (4);
	

\end{tikzpicture}

 
\subparagraph{Labeling functions}



\subparagraph{Schip}


\begin{tikzpicture}[>=latex',shorten >=1pt,node distance=2cm,on grid,auto,scale=0.2]
	
	\node[state] (q0-e) {$q_0/\epsilon$};
	\node[state] (q0-1) [below right=of q0-e] {$q_0'/1$};
	\node[state] (q1-0) [above right=of q0-1] {$q_1/0$};
	\node[state] (q2-1) [below right=of q1-0] {$q_2/1$};
	
	\node[state] (q3-0) [below left=of q0-1] {$q_3/0$};
	\node[state,accepting] (q3-1) [below right=of q0-1] {$q_3'/1$};
	\node[state] (0) [ left=of q3-0] {$q_3/0$};
	\node[state] (1) [ left=of 0] {$q_3/0$};
	
	\node[state,initial,accepting] (0) [ left=of 1] {$q_3/0$};
	

	
	\path[->] (q0-e) edge node {a} (q1-0);
	\path[->] (q0-e) edge node {b} (q3-0);
	\path[->] (q0-1) edge node {a} (q1-0);
	\path[->] (q0-1) edge [bend right] node {b} (q3-0);
	\path[->] (q1-0) edge node {a} (q3-1);
	\path[->] (q1-0) edge node {b} (q2-1);
	\path[->] (q2-1) edge node {a} (q0-1);
	\path[->] (q2-1) edge node {b} (q3-1);
	\path[->] (q3-0) edge node {a} (q3-1);
	\path[->] (q3-0) edge [bend right] node {b} (q0-1);
	\path[->] (q3-1) edge [loop below] node {a} (q3-1);
	\path[->] (q3-1) edge node {b} (q0-1);
	

	
	
	
	
\end{tikzpicture}

\subparagraph{Deur}

\begin{tikzpicture}[>=latex',shorten >=1pt,node distance=2cm,on grid,auto,scale=0.2]
	
	\node[state] (q0-e) {$q_0/\epsilon$};
	\node[state] (q0-1) [below right=of q0-e] {$q_0'/1$};
	\node[state] (q1-0) [above right=of q0-1] {$q_1/0$};
	\node[state] (q2-1) [below right=of q1-0] {$q_2/1$};
	
	\node[state] (q3-0) [below left=of q0-1] {$q_3/0$};
	\node[state,initial,accepting] (q3-1) [below right=of q0-1] {$q_3'/1$};

	


	
	
	\path[->] (q0-e) edge node {a} (q1-0);
	\path[->] (q0-e) edge node {b} (q3-0);
	\path[->] (q0-1) edge node {a} (q1-0);
	\path[->] (q0-1) edge [bend right] node {b} (q3-0);
	\path[->] (q1-0) edge node {a} (q3-1);
	\path[->] (q1-0) edge node {b} (q2-1);
	\path[->] (q2-1) edge node {a} (q0-1);
	\path[->] (q2-1) edge node {b} (q3-1);
	\path[->] (q3-0) edge node {a} (q3-1);
	\path[->] (q3-0) edge [bend right] node {b} (q0-1);
	\path[->] (q3-1) edge [loop below] node {a} (q3-1);
	\path[->] (q3-1) edge node {b} (q0-1);
	

	
\end{tikzpicture}

\subparagraph{Stoplicht}


\begin{tikzpicture}[>=latex',shorten >=1pt,node distance=2cm,on grid,auto,scale=0.2]
	
	\node[state] (q0-e) {$q_0/\epsilon$};
	\node[state] (q0-1) [below right=of q0-e] {$q_0'/1$};
	\node[state] (q1-0) [above right=of q0-1] {$q_1/0$};
	\node[state] (q2-1) [below right=of q1-0] {$q_2/1$};
	
	\node[state,initial,accepting] (q3-0) [below left=of q0-1] {$q_3/0$};
	\node[state,accepting] (q3-1) [below right=of q0-1] {$q_3'/1$};



	
	
	\path[->] (q0-e) edge node {a} (q1-0);
	\path[->] (q0-e) edge node {b} (q3-0);
	\path[->] (q0-1) edge node {a} (q1-0);
	\path[->] (q0-1) edge [bend right] node {b} (q3-0);
	\path[->] (q1-0) edge node {a} (q3-1);
	\path[->] (q1-0) edge node {b} (q2-1);
	\path[->] (q2-1) edge node {a} (q0-1);
	\path[->] (q2-1) edge node {b} (q3-1);
	\path[->] (q3-0) edge node {a} (q3-1);
	\path[->] (q3-0) edge [bend right] node {b} (q0-1);
	\path[->] (q3-1) edge [loop below] node {a} (q3-1);
	\path[->] (q3-1) edge node {b} (q0-1);
	

	
	
	
\end{tikzpicture}

\subparagraph{pomp}



\begin{tikzpicture}[>=latex',shorten >=1pt,node distance=2cm,on grid,auto,scale=0.2]
	
	\node[state] (q0-e) {$q_0/\epsilon$};
	\node[state] (q0-1) [below right=of q0-e] {$q_0'/1$};
	\node[state] (q1-0) [above right=of q0-1] {$q_1/0$};
	\node[state] (q2-1) [below right=of q1-0] {$q_2/1$};
	
	\node[state] (q3-0) [below left=of q0-1] {$q_3/0$};
	\node[state,initial,accepting] (q3-1) [below right=of q0-1] {$q_3'/1$};



	
	
	\path[->] (q0-e) edge node {a} (q1-0);
	\path[->] (q0-e) edge node {b} (q3-0);
	\path[->] (q0-1) edge node {a} (q1-0);
	\path[->] (q0-1) edge [bend right] node {b} (q3-0);
	\path[->] (q1-0) edge node {a} (q3-1);
	\path[->] (q1-0) edge node {b} (q2-1);
	\path[->] (q2-1) edge node {a} (q0-1);
	\path[->] (q2-1) edge node {b} (q3-1);
	\path[->] (q3-0) edge node {a} (q3-1);
	\path[->] (q3-0) edge [bend right] node {b} (q0-1);
	\path[->] (q3-1) edge [loop below] node {a} (q3-1);
	\path[->] (q3-1) edge node {b} (q0-1);
	

	
\end{tikzpicture}


\hoofdstuk{Verificatie}
 We moeten aantonen dat een real-time programma voldoet aan de eisen opgesteld en gespecificeerd. De meest gebruikte methode voor het bewij
 
 zen van de correctheid van untimed programma's zijn aangepast voor timed programs.  We hebben nog geen aanpask gevonden voor het gebruik en bewijzen van correct gebruik van clocks.  Een bewijs voor het gebruik van real-time programmas met clocks is gegeven in T.A. Henzinger and P.W. Kopke. Verification methods for the di-
 vergent runs of clock systems
 
 In dit hoofdstuk formaliseren we de requirements ogegeven in de requiremenstlis tin hoofdstuk .. en bewijzen we de correcte toepassing met gebruik van de symbolic model-checker van Uppaal.
 Het systeem is gemodelleerd als een netwerk van meerdere timed automata: controller, sluis, stoplicht, deur, pomp en schip.
 
 Het bewijs vn corret gebruik kan ook worden aangetoond met help van bewijs voor inorrectgebruik
 
 
 Verificatie resultaten
 \paragraph{Het door ons uitgetippelde testpath of scenario}
 
 \paragraph{Timed automata}
 
 
\paragraph{Data variabelen}

\paragraph{Acties}
 
\paragraph{Clock regions}
\cite{clarke2000Modelchecking21}
\cite{clarke2000Modelchecking212}
\cite{clarke2000Modelchecking223}
\cite{clarke2000Modelchecking31}
\cite{clarke2000Modelchecking32}
\cite{clarke2000Modelchecking33}
\cite{clarke2000Modelchecking411}
\cite{clarke2000Modelchecking43}
\cite{clarke2000Modelchecking63}
\cite{clarke2000Modelchecking64}
\cite{clarke2000Modelchecking661}
\cite{clarke2000Modelchecking91}
\cite{clarke2000Modelchecking102}
\cite{clarke2000Modelchecking11}
\cite{clarke2000Modelchecking122}
\cite{clarke2000Modelchecking123}
\cite{clarke2000Modelchecking132}
\cite{clarke2000Modelchecking1321}
\cite{clarke2000Modelchecking152}
\cite{clarke2000Modelchecking171}
\cite{clarke2000Modelchecking172}
\cite{clarke2000Modelchecking173}
\cite{audioSemanticsBengtsson}
\cite{guidingAutomataBberm}
\cite{gearTransitionLindahl1}
\cite{gearTransitionLindahl2}
\cite{martinelliScada}
\cite{IgbalReconstructurintTransition1}
\cite{IgbalReconstructurintTransition2}
\cite{huangVerficationStoch}
\cite{bengtssonUppaalVerification}
\cite{pranaliVerificationWaterLevel}
\cite{alexandreUppaalDefinition}
\cite{behzadEvalQOS}
\cite{behzadVariablesQoS}
\cite{alur}
\cite{alurDenseRealTime}
\cite{alurSystemClok}
\cite{alurModelHybrid}
\cite{rijksoverheidSluizen}
\cite{rijksoverheidSluisStroomschema}

\paragraph{CTL logica}
Alle veiligheid en reachability requirements formeel gespecificeerd in hoofdstuk ... zijn geverifieerd in uppaal met gebruik an A en E state formulae. Deze zijn als volgt:
\newline \\
M, s $\models$ p $\Leftrightarrow$ p $\in$ L(s) \\
M, s $\models$ $\not$ f1 $\Leftrightarrow$ M, s $\nvdash$ f1 \\
M, s $\models$ f1 $\vee$ f2 $\Leftrightarrow$ M,s $\models$ f1 or M,s $\nvdash$ f2 \\
M, s $\models$ f1 $\wedge$ f2 $\Leftrightarrow$  M,s $\models$ f1 and M,s $\nvdash$ f2 \\
M, s $\models$ $\mathrm{E}$ $g_{1}$ $\Leftrightarrow$ there is a path $\pi$  from ~  s ~   such ~  that  ~ M, $\pi$ $\models$ g1 \\
M, s $\models$ p $\Leftrightarrow$ for every path $\pi$  ~ starting from  ~  s, M, $\pi$ $\models$ g1 \\
M, s $\models$ p $\Leftrightarrow$ s is the first state of $\piand$ M, s $\models$ f1 \\
M, s $\models$ $\not$ $g_{1}$ $\Leftrightarrow$ M, $\pi$  $\nvdash$ g1\\
M, s $\models$ p $\Leftrightarrow$  M, $\pi$  $\models$ g1  or  M, $\pi$  M, $\pi$  $\models$ g2\\
M, s $\models$ p $\Leftrightarrow$ M, $\pi$  $\models$ g1  and  M, $\pi$  M, $\pi$  $\models$ g2 \\
M, s $\models$ p $\Leftrightarrow$ M, $\pi^{1}$ $\models$ g1 \\
M, s $\models$ p $\Leftrightarrow$ there exists a k $\ge$ 0, such that  ~ M, $\pi^{k}$  $\models$ g1\\
M, s $\models$ p $\Leftrightarrow$ for all i $\ge$ 0,M,$\pi^{i}$ $\models$ g1 \\
M, s $\models$ g1 $\bugcup$ g2 $\Leftrightarrow$ ~  there  ~ exists  ~ ak  ~ $\ge$  ~ 0 ~  such ~  that  ~ M,  ~ $\pi^{k}$ $\models$ g2\\
and  ~ for  ~ all ~  0  ~ $\le$ j < k, M,$\pi^{j}$ $\models$ g1
M, s $\models$ p $\Leftrightarrow$ for all j $\ge$ 0, if for ~  every  ~ i < j,M,$\pi^{i}$ $\nvdash$ g1 then M,$\pi^{j}$ $\models$ g2\\


%%%%%%%%%%%%%%%%%%%%%%%%%%%%%%%%%%%%%%%%%%%%%%%%%%%%%%%%%%%%%%%%%


 

\hoofdstuk{Conclusie}

Wat hebben alle bovenstaande rampen/ongelukken gemeen? Veiligheid.
Bij de therac waren er diverse problemen: communicatie, doorontwikkeling, controle en toetsing
Was het makkelijk te onderzoeken? Waarom?
Bij de boeing 737 crashes was het probleem van controle en communicatie naar medewerkers
Was het makkelijk te onderzoeken? Waarom?

Uit de evaluatie van de china explosion 2015 tianjin komt naar voren dat communicatie, transparantie en veiligheid niet altijd prioriteit hadden bij de lokale autoriteiten
Was het makkelijk te onderzoeken? Waarom?

Bij de tesla autopilot crashes komen soms onvoldoende onderbouwde ontwerpkeuzes naar voren die niet goed zij  afgewogen tegenover het gedrag van de bestuurder
vlucht 1951
Was het makkelijk te onderzoeken? Waarom?

De ramp in Tsjernobyl toont aan hoe autoriteiten een ramp in de doofpot proberen te stoppen
Was het makkelijk te onderzoeken? Waarom?



Wat heb ik geleerd
Ik heb erg veel geleerd van het veilig opzetten van VPN’s. Een VPN opzettenhad ik namelijk nog nooit gedaan. Het opzetten van SSH en het aanmaken vanVM’s was al bekend. Ook had ik nog nooit met UDP sockets geprogrammeerd.Verder heb ik geleerd hoe ik in de praktijk een VM in een VLAN kan zetten enhoe VLAN’s netwerken van elkaar kunnen scheiden.Het leukste onderdeel van het project, was dat wonderbaarlijk mijn gekozenoplossing elegant werkte. UDP Servers en clients zijn gerealiseerd met minderdan enkele regels logisch scipt. Ik had aan genomen dat het werken met socketsin shell absoluut rampzalig zou uitpakken. Ik ben blij dat het opdracht zo vrijwas, zodat ik experimenteel kon zijn met mijn implementatie.



