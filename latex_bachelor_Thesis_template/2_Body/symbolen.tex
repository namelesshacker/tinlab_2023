% NPROP12.STY -- NOAO observing proposal form.
% version 1.2 the same as version 1.0 with just a few changes (8/15/98 - jvb)
% version 1.1 the same as version 1.0 with just a few changes (2/2/98 - jvb)
% version 1.0 based on kpprop14, with added tables -- djb Jul 7 1997 

% First, identifying information about this style file.

\def\revtex@ver{1.2}		% Version number of this file.
\def\revtex@date{15 Aug 98}	% Revision date.
\def\revtex@author{CDB/DJB}	% This file's author.
\def\revtex@org{NOAO}		% Sponsoring organization.
\def\revtex@genre{observing proposal}	% Document "type".

% ifundefined sequence
\def\ifundefined#1{\expandafter\ifx\csname#1\endcsname\relax}

% provide compatibility with previous versions
\newcommand{\lunarphase}{\lunardays}
\newcommand{\whykpno}{\whynoao}
\newcommand{\whyctio}{\whynoao}
\newcommand{\expdesign}{\feasibility}

\def\revtex@pageid{\xdef\@thefnmark{\null}
    \@footnotetext{\revtex@org\space
    \revtex@genre\space \LaTeX\ macros v\revtex@ver.}}

\let\ltx@enddoc=\enddocument
\def\enddocument{\vfill\revtex@pageid\ltx@enddoc}

% Page design/layout macros.  Page sizes are at the end of the file.

\def\baselinestretch{0.98}	% Tighten up baselines a little.

\def\ps@kpprophead{\def\@oddfoot{}\def\@evenfoot{}
    \def\@oddhead{\hbox to\textwidth{\small\sl Proposal \#\@propid
	\ifx\@empty\@rcptdate\relax\else, received \@rcptdate\fi
	\hfill Page \thepage}}\let\@evenhead\@oddhead
    \def\chaptermark##1{\markright {{\ifnum \c@secnumdepth>\m@ne
	\@chapapp\ \thechapter. \ \fi ##1}}}}


\newdimen\prop@idboxwidth

\def\prop@idbox{\prop@idboxwidth\textwidth
    \advance\prop@idboxwidth by-2.3\fboxsep
    \fbox{\hbox to\prop@idboxwidth{
    \ifx\@rcptdate\@empty{\it Date:\/} \today
    \else{\it Date received:\/} \@rcptdate\fi\hfil
    {\it TAC:\/} \@taccategory \hspace{1.4cm}
    {\it Proposal number:\/} \@propid }}}

\def\ques@font{\it}
\def\instruct@font{\small\sl}

% Proposal ID and receipt date will be filled in at NOAO.
%
%    \proposalid{NUMBER}
%    \received{DATE}

\def\received#1{\gdef\@rcptdate{#1}} \received{}
\def\proposalid#1{\gdef\@propid{#1}} \proposalid{\hspace{1in}}
\def\taccategory#1{\gdef\@taccategory{#1}} \taccategory{\hspace{1in}}
\def\observatory#1{\gdef\@observatory{#1}} \def\@observatory{}

% Below is the markup that the observing team needs to supply.
% Instructions for filling in the form using these commands are in
% a template proposal form file as LaTeX comments.


% print a different header based on \observatory field
\def\@tempkpno{kpno}
\def\@tempctio{ctio}
\def\prop@head{NOAO Observing Proposal\\
	\ifx\@observatory\@tempkpno
		{Kitt Peak National Observatory}
	\else\ifx\@observatory\@tempctio
		{Cerro Tololo Inter-American Observatory}
	\else
		\vspace*{0.1cm}\makebox[0.6\textwidth]{Observatory: \hrulefill}
	\fi\fi
}




%
%    \title{TEXT}
\def\title#1{\null\vspace{-2\headheight}\vspace{-\headsep}
    \begin{center}\large\bf\prop@head\end{center}\prop@idbox\par
    \vspace*{.2in}
    \def\@savetitle{#1}\ifx\@empty\@savetitle
    \makebox[\textwidth]{\large{\bf TITLE:} \hrulefill}\par
    \makebox[\textwidth]{\hrulefill}\par
    \else{{\large{\bf TITLE:}} \@savetitle}\fi}


\def\abstract{\par\vspace{0ex}\vbox to2.35in\bgroup\noindent
    {\bf Abstract of Scientific Justification} {\small\sl (may be made publicly available for accepted proposals)}{\bf :} \parindent\z@}
\let\ltx@endabstract=\endabstract
\def\endabstract{\ltx@endabstract\vfil\egroup\vspace{\fill}}

% Observer identification.  These items are buffered so that they can be
% specified in any order, subject only to the restriction that the observer
% name be given first.  The same commands are used to identify the PI as
% well as the co-investigators; the formatting is controlled separately,
% and differences arise in the use of LaTeX environments, below.
%
% Each member of the observing team is identified with several bits of
% information.
%
%    \name{OBSERVER NAME}
%    \affil{AFFILIATION}
%    \address{POSTAL ADDRESS}
%    \emailaddress{EMAIL ADDRESS}
%    \phone{TELEPHONE NUMBER}
%    \fax{FAX NUMBER}
%    \invstatus{P, T, G, U, or O}
%
% Note that the fax number does not print on the form.  There is not
% enough room on the cover page the way it is currently laid out.

\def\name#1{\gdef\obs@name{#1}}
\def\affil#1{\gdef\obs@affil{#1}}
\def\address#1{\gdef\obs@address{#1}}
\def\emailaddress#1{\gdef\obs@email{#1}}
\def\phone#1{\gdef\obs@phone{#1}}
\def\fax#1{\gdef\obs@fax{#1}}
\def\invstatus#1{\gdef\obs@invstatus{#1}}
\def\gradstudent#1{\gdef\obs@gradstudent{#1}}  % for compatibility (ignored)
\def\permaddress#1{\gdef\obs@permaddress{#1}} % for future use (ignored)

\def\clear@obs{\gdef\obs@name{}\gdef\obs@affil{}\gdef\obs@address{}
    \gdef\obs@email{}\gdef\obs@phone{}\gdef\obs@fax{}\gdef\obs@invstatus{}
    \gdef\obs@gradstudent{}\gdef\obs@permaddress{}}

% Formatting of PI and CO-I data are controlled by the next two macros.

% Now here is some dicey stuff.  We want to solve a variety of problems with
% one form and one set of macros, stemming from the need to print rational
% forms, whether or not anything has been filled in electronically.  In
% other words, if somebody runs a *blank* electronic form through LaTeX,
% a usable form should emerge from the laser printer, with blanks in the
% appropriate places, etc.  If the form is filled in electronically, the
% rules should be replaced with the stuff the user fills in.
%
% Fine.  Except that real estate on the first page is quite limited, and
% we also need a way of letting the user who fills in information with the
% editor know when he/she is being verbose.  We achieve this by forcing
% certain pieces of information into boxes of fixed size.  These contain
% either an \hrulefill, or text supplied by the author.  If too much text
% is provided, TeX will bitch about the overfull hbox.

\def\yn@rule{\rule{0.3in}{0.4pt}}
\newcount\@maxnumcois \@maxnumcois=3

\def\prt@piblock{\begin{quote}\begin{tabular}{ll}
    \bf Principal Investigator: \fld@dlm \obs@name\\
    \bf Institution: \fld@dlm \obs@affil\\
    \bf Email: \fld@dlm \obs@email\\
    \bf Telephone: \fld@dlm \obs@phone\\
    \bf Fax: \fld@dlm \obs@fax\\
    \end{tabular}\end{quote}}

\def\prt@coiblock{\begin{tabbing}
    \makebox[.25\textwidth]{\null}\=\makebox[.25\textwidth]{\null}\=
    \makebox[.25\textwidth]{\null}\=\kill
    \hbox to.40\textwidth{\obs@tag\space%
	\ifx\@empty\obs@name\hrulefill\space\else\obs@name\hfill\fi}%
    \hbox to.14\textwidth{Status:\space%
	\ifx\@empty\obs@invstatus\yn@rule\else\obs@invstatus\fi\hfill}%
    \hbox to.46\textwidth{%
	\ifx\@empty\obs@email Email: \hrulefill\else\hfill\obs@email\fi}\\[.2ex]
    \hbox to.80\textwidth{%
	\ifx\@empty\obs@address Address: \hrulefill\space\else\obs@affil,
	\obs@address\hfill\fi}%
    \hbox to.20\textwidth{%
	\ifx\@empty\obs@phone Phone: \hrulefill\else\hfill\obs@phone\fi}%
    \end{tabbing}}

% Principal investigator and co-investigator environments.

% Note that after the first three CoI blocks, I quit printing the CoI
% information.  This is not especially nice.  It would be better if the
% names (and possibly the grad student status) of the additional CoIs
% were listed on the proposal form someplace.  Since the cover page is
% so full at this point, my sense is that these would have to be collected
% into end notes somehow.  This would be hard to program, and so it should
% be deemed a very necessary requirement.  Perhaps some other display
% format would be acceptable, one that would require less programming.

\newcounter{CoI}
\def\obs@tag{{\bf CoI:}}

\newenvironment{PI}{\clear@obs}{\def\obs@tag{{\bf PI:}}\prt@coiblock}

\newenvironment{CoI}{\clear@obs\stepcounter{CoI}}{\ifnum\c@CoI>\@maxnumcois
    \relax\else\vspace*{-5ex}\prt@coiblock\fi}

% Questions for first page.
%
%   \thesis{Y or N}
%   \longterm{Y or N}
%   \longtermdetails{BRIEF DETAILS}
%   \queue{Y or N}
%   \unusabledates{BRIEF DETAILS}

\def\chk@setblank#1{\def\@kptmpa{#1}\ifx\@empty\@kptmpa
    \def\@kptmpa{\yn@rule}\fi\mbox{\@kptmpa}}

\def\chk@setblankline#1{\def\@kptmpa{#1}\ifx\@empty\@kptmpa
    \def\@kptmpa{\rule{\textwidth}{0.4pt}}\fi
    \par\makebox[\textwidth][l]{\@kptmpa}}

\def\thesis@head{{\ques@font Is this proposal part of a PhD thesis?
    If `Y', you must send a letter; see instructions.}}
\def\longterm@head{{\ques@font Are you requesting long-term status?
    If `Y', please give details on the line below.}}
\def\queue@head{{\ques@font Would you like these observations carried
    out in ``queue'' mode?}}
\def\undates@head{{\ques@font Scheduling constraints and non-usable dates.}}

\def\thesis#1{\par$\bullet$\space\thesis@head\quad\chk@setblank{#1}}
\def\longterm#1{\par$\bullet$\space\longterm@head\quad\chk@setblank{#1}
    \gdef\@longterm{#1}}
\def\longtermdetails#1{\def\@tempa{N}\ifx\@longterm\@tempa\relax\else
    \chk@setblankline{#1}\fi}
\def\queue#1{\par$\bullet$\space\queue@head\quad\chk@setblank{#1}}
\def\unusabledates#1{\par$\bullet$\space\undates@head\par\chk@setblankline{#1}}

% Observing run environment.  This is used to describe observing parameters
% in a very succinct manner, to be presented in an abbreviated form on the
% first page.  I have chosen to use multiple \begin{obsrun}-\end{obsrun}
% groupings, as opposed to a repeating set of parameter identifying macros.

\newcounter{obsrun}
\newenvironment{obsrun}{\stepcounter{obsrun}}%
    {\ifnum\c@obsrun=3\par\prt@obsruns\par\fi}

% Observing run parameters to be specified by the investigators.
%
%   \telescope{IDENT}
%   \instrument{IDENT(S)}
%   \numnights{NUMBER}
%   \numdays{NUMBER}		% removed from printed table
%   \lunardays{NUMBER}
%   \optimaldates{RANGE}
%   \acceptabledates{RANGE}

% These are pretty amazing-looking.  They are fake "arrays", actually.

\def\telescope#1{\global\@namedef{obs\@Alph{\c@obsrun}@telid}{#1}}
\def\instrument#1{\global\@namedef{obs\@Alph{\c@obsrun}@inst}{#1}}
\def\numnights#1{\global\@namedef{obs\@Alph{\c@obsrun}@numnights}{#1}}
\def\numdays#1{\global\@namedef{obs\@Alph{\c@obsrun}@numdays}{#1}}
\def\lunardays#1{\global\@namedef{obs\@Alph{\c@obsrun}@lunardays}{#1}}
\def\optimaldates#1{\global\@namedef{obs\@Alph{\c@obsrun}@optdates}{#1}}
\def\acceptabledates#1{\global\@namedef{obs\@Alph{\c@obsrun}@accdates}{#1}}

\def\obsrun@head{Summary of observing runs requested for this project}

% This one is not actually used.  It is work in progress...

\def\obsblock@kpnoprop{
\@whilenum\c@obsrun <3
\do {\typeout{Run \the\c@obsrun.}
\@arabic{\c@obsrun} &
\@nameuse{obs\@Alph{\c@obsrun}@telid} &
\@nameuse{obs\@Alph{\c@obsrun}@inst} \\
\global\advance\c@obsrun by\@ne}
& \@nameuse{obs\@Alph{\c@obsrun}@numnights} &
\@nameuse{obs\@Alph{\c@obsrun}@numdays} &
\@nameuse{obs\@Alph{\c@obsrun}@lunardays} &
\@nameuse{obs\@Alph{\c@obsrun}@optdates} &
\@nameuse{obs\@Alph{\c@obsrun}@accdates}
}

% Note that I am taking serious advantage of the requirement that only
% three of these blocks may be specified: I am explicitly printing
% three sets of obsrun data and three sets only.

\def\prt@obsruns{\begin{center}{\small\bf\obsrun@head}\\[1ex]
    \setcounter{obsrun}{1}
    \begin{tabular*}{1.0\textwidth}
	{|c||c|l@{\extracolsep{\fill}}c|}
    \multicolumn{1}{c}{\scriptsize Run} &
    \multicolumn{1}{c}{\scriptsize Telescope} &
    \multicolumn{1}{c}{\scriptsize Instrument, detectors, gratings, filters,
    camera optics, multislits, etc.}\\
    \hline
    1 & \obsA@telid & \obsA@inst &\\
    \hline
    2 & \obsB@telid & \obsB@inst &\\
    \hline
    3 & \obsC@telid & \obsC@inst &\\
    \hline
    \end{tabular*}\\
    \vspace{2ex}
    \begin{tabular*}{1.0\textwidth}
	{|c||c|c|@{\extracolsep{\fill}}*{2}{c|}}
    \multicolumn{1}{c}{\scriptsize Run} &
    \multicolumn{1}{c}{\scriptsize No. nights} &
    \multicolumn{1}{c}{\scriptsize Moon age (d)} &
    \multicolumn{1}{c}{\scriptsize Optimal dates} &
    \multicolumn{1}{c}{\scriptsize Acceptable dates}\\
    \hline
    1 & \obsA@numnights & \obsA@lunardays &
        \obsA@optdates & \obsA@accdates \\
    \hline
    2 & \obsB@numnights & \obsB@lunardays &
        \obsB@optdates & \obsB@accdates  \\
    \hline
    3 & \obsC@numnights & \obsC@lunardays &
        \obsC@optdates & \obsC@accdates \\
    \hline
    \end{tabular*}
    \end{center}}

% The scientific justification and the proposed observing program are
% typeset (printed) as they are specified by the author.  Specific
% observations are summarized in the "obsrun" environment (above).
%
%    \justification
%    \feasibility
%    \whynoao
%    \thepast

\newenvironment{references}{\subsubsection*{References}}{\relax}
\let\reference=\par

\def\propsection{\@startsection{section}{1}{\z@}
    {7pt plus 4pt minus 2pt}{-1em}{\normalsize\bf}}
\def\subsection{\@startsection{subsection}{2}{\z@}
    {-3.25ex plus -1ex minus -.2ex}{1.5ex plus .2ex}{\normalsize\bf}}

\def\justif@head{{\instruct@font Be sure to include overall significance
    to astronomy. Limit text to one page with figures (up to three), 
    captions and references on no more than two additional pages.\par}}
\def\feas@head{{\instruct@font Justify the number of nights requested
    as well as the specific telescope, instruments, and lunar phase.
    If you've requested long-term status, justify why this is necessary
    for successful completion of the science.  List objects, coordinates, 
    and magnitudes (or surface brightness, if appropriate), desired S/N ratio, 
    wavelength coverage and resolution (or include in Target Tables section). 
    {\bf NOTE:} If this project involves observations on other telescopes as 
    well, list all telescopes on which you have applied for or been granted 
    time for observations related to this project.  For each, indicate the 
    nature of the observations, and describe the importance of the 
    observations proposed here in the context of the entire program.
    For WIYN proposals, see special WIYN instructions.\par}}
\def\whynoao@head{{\instruct@font (For CTIO proposals only.)
Explain why access to the southern hemisphere is needed to achieve your
scientific goals.\par}}
\def\past@head{{\instruct@font List allocations of telescope time on 
    NOAO facilities to the Principal Investigator during the past 2 years, 
    together with the current
    status of the data (cite publications where appropriate).  Mark with
    an asterisk those allocations of time related to the current proposal.\par}}

\def\justification{\newpage\propsection{\fbox{Scientific Justification}}
    \justif@head}
\def\feasibility{\newpage\propsection{\fbox{Experimental Design}}\feas@head}
\def\whynoao{\vfill\propsection{\fbox{Why CTIO?}}\whynoao@head}
\def\thepast{\vfill\propsection{\fbox{Previous Use of NOAO Facilities}}
    \past@head}

\def\relatedwork#1{\fbox{$\star$}\space#1}

% vvv New Stuff for NOAOPROP tables vvv

\def\cleart@obs {
	\gdef\obs@objid{}
	\gdef\obs@object{}
	\gdef\obs@ra{}
	\gdef\obs@dec{}
	\gdef\obs@epoch{}
	\gdef\obs@magnitude{}
	\gdef\obs@filter{}
	\gdef\obs@exptime{}
	\gdef\obs@nexposures{}
	\gdef\obs@moondays{}
	\gdef\obs@skycond{}
	\gdef\obs@seeing{}
	\gdef\obs@obscomment{}
	\gdef\obs@grating{}
	\gdef\obs@order{}
	\gdef\obs@wstart{}
	\gdef\obs@wend{}
	\gdef\obs@cable{}
	\gdef\obs@camera{}
}


\def\obstargets@head{{\instruct@font Provide information for your intended observations in tabular form.  This information is required for WIYN
proposals but is optional for all other telescopes.\par}}
\def\obstargets{\vfill\newpage\propsection{\fbox{Target Tables}}\obstargets@head}

\def\obs@head{Observations for the \@nameuse{instrname@\@instrid}}


\def\@instrid{imager}
\newenvironment{targettable}[1]{\cleart@obs\def\@instrname{#1}\def\obs@list{}}%
    {\begin{center}\small\bf\obs@head\\[1.2ex]%
    \footnotesize\tabcolsep2\p@%
    \begin{tabular}{\@nameuse{pream@\@instrid}}%
    \@nameuse{obshead@\@instrid}\\\hline\\[-1.8ex]%
    \obs@list\end{tabular}\end{center}


% if they entered a grating, print out the spectral table-footer info
%	\ifundefined{obs@grating}{}
	\ifx\@empty\obs@grating{}	
	\else{
		\footnotesize
		\begin{tabular}{ll@{\hspace{1.5cm}}ll}
		Grating:	& \obs@grating 	& 
			Order: 	& \obs@order  \\
		Cable: 	& \obs@cable 	& 
			Camera: 	& \obs@camera \\
		$\lambda_{\rm start}$: &  \obs@wstart & 
			$\lambda_{\rm end}$:  & \obs@wend
		\end{tabular} \vspace*{3ex}
		\normalsize
	     }
	\fi

}

% Observing parameters.  Not all of these are needed for each instrument
% combination.
%
%    \objid{OBJECT ID}
%    \object{OBJECT NAME}
%    \ra{RIGHT ASCENSION}
%    \dec{DECLINATION}
%    \epoch{COORDINATE EPOCH}
%    \magnitude{MAGNITUDE}
%    \filter{FILTER}
%    \exptime{EXPOSURE TIME}
%    \nexposures{NUMBER OF EXPOSURES}
%    \moondays{DAYS FROM NEW MOON}
%    \skycond{SKY CONDITION}
%    \seeing{SEEING LIMIT}
%    \obscomment{COMMENT}



\def\obsblock@add{
    \ifx\obs@list\@empty
          \@temptokena=\expandafter{\obs@list}
    \else \@temptokena=\expandafter{\obs@list\\}\fi
    \xdef\obs@list{\the\@temptokena\@nameuse{obsblock@\@instrid}}}

\def\objid#1{\gdef\obs@objid{#1}}
\def\object#1{\gdef\obs@object{#1}}
\def\ra#1{\gdef\obs@ra{#1}}
\def\dec#1{\gdef\obs@dec{#1}}
\def\epoch#1{\gdef\obs@epoch{#1}}
\def\magnitude#1{\gdef\obs@magnitude{#1}}
\def\filter#1{\gdef\obs@filter{#1}}
\def\exptime#1{\gdef\obs@exptime{#1}}
\def\nexposures#1{\gdef\obs@nexposures{#1}}
\def\moondays#1{\gdef\obs@moondays{#1}}
\def\skycond#1{\gdef\obs@skycond{#1}}
\def\seeing#1{\gdef\obs@seeing{#1}}
\def\obscomment#1{\gdef\obs@obscomment{#1}\obsblock@add}

\def\grating#1{\gdef\obs@grating{#1}}
\def\order#1{\gdef\obs@order{#1}}
\def\wstart#1{\gdef\obs@wstart{#1}}
\def\wend#1{\gdef\obs@wend{#1}}
\def\cable#1{\gdef\obs@cable{#1}}
\def\camera#1{\gdef\obs@camera{#1}}



\def\instrname@imager{\@instrname}
\def\pream@imager{ccccccccccccl}

\def\obshead@imager{
{\shortstack{Obj\\ID}} & {Object} & {$\alpha$} & {$\delta$} & {Epoch} &
{Mag.} & {Filter} & {\shortstack{Exp.\\time}} &
{\shortstack{\# of\\exp.}} &
{\shortstack{Lunar\\days}} & {Sky} & {Seeing} & {Comment}
}

\def\obsblock@imager{
\obs@objid & \obs@object & \obs@ra & \obs@dec & \obs@epoch &
\obs@magnitude & \obs@filter & \obs@exptime & \obs@nexposures &
\obs@moondays & \obs@skycond & \obs@seeing & \obs@obscomment
}



% ^^^ New Stuff for NOAOPROP tables^^^



%  ****************************************
%  *           EPS INCLUSIONS             *
%  ****************************************

% Include Rokicki's epsf.sty file explicitly.

\@ifundefined{epsfbox}{%   EPSF.TEX macro file:
%   Written by Tomas Rokicki of Radical Eye Software, 29 Mar 1989.
%   Revised by Don Knuth, 3 Jan 1990.
%   Revised by Tomas Rokicki to accept bounding boxes with no
%      space after the colon, 18 Jul 1990.
%
%   TeX macros to include an Encapsulated PostScript graphic.
%   Works by finding the bounding box comment,
%   calculating the correct scale values, and inserting a vbox
%   of the appropriate size at the current position in the TeX document.
%
%   To use with the center environment of LaTeX, preface the \epsffile
%   call with a \leavevmode.  (LaTeX should probably supply this itself
%   for the center environment.)
%
%   To use, simply say
%   \input epsf           % somewhere early on in your TeX file
%   \epsfbox{filename.ps} % where you want to insert a vbox for a figure
%
%   Alternatively, you can type
%
%   \epsfbox[0 0 30 50]{filename.ps} % to supply your own BB
%
%   which will not read in the file, and will instead use the bounding
%   box you specify.
%
%   The effect will be to typeset the figure as a TeX box, at the
%   point of your \epsfbox command. By default, the graphic will have its
%   `natural' width (namely the width of its bounding box, as described
%   in filename.ps). The TeX box will have depth zero.
%
%   You can enlarge or reduce the figure by saying
%     \epsfxsize=<dimen> \epsfbox{filename.ps}
%   (or
%     \epsfysize=<dimen> \epsfbox{filename.ps})
%   instead. Then the width of the TeX box will be \epsfxsize and its
%   height will be scaled proportionately (or the height will be
%   \epsfysize and its width will be scaled proportiontally).  The
%   width (and height) is restored to zero after each use.
%
%   A more general facility for sizing is available by defining the
%   \epsfsize macro.    Normally you can redefine this macro
%   to do almost anything.  The first parameter is the natural x size of
%   the PostScript graphic, the second parameter is the natural y size
%   of the PostScript graphic.  It must return the xsize to use, or 0 if
%   natural scaling is to be used.  Common uses include:
%
%      \epsfxsize  % just leave the old value alone
%      0pt         % use the natural sizes
%      #1          % use the natural sizes
%      \hsize      % scale to full width
%      0.5#1       % scale to 50% of natural size
%      \ifnum#1>\hsize\hsize\else#1\fi  % smaller of natural, hsize
%
%   If you want TeX to report the size of the figure (as a message
%   on your terminal when it processes each figure), say `\epsfverbosetrue'.
%
\newread\epsffilein    % file to \read
\newif\ifepsffileok    % continue looking for the bounding box?
\newif\ifepsfbbfound   % success?
\newif\ifepsfverbose   % report what you're making?
\newif\ifepsfdraft     % use draft mode?
\newdimen\epsfxsize    % horizontal size after scaling
\newdimen\epsfysize    % vertical size after scaling
\newdimen\epsftsize    % horizontal size before scaling
\newdimen\epsfrsize    % vertical size before scaling
\newdimen\epsftmp      % register for arithmetic manipulation
\newdimen\pspoints     % conversion factor
%
\pspoints=1bp          % Adobe points are `big'
\epsfxsize=0pt         % Default value, means `use natural size'
\epsfysize=0pt         % ditto
%
\def\epsfbox#1{\global\def\epsfllx{72}\global\def\epsflly{72}%
   \global\def\epsfurx{540}\global\def\epsfury{720}%
   \def\lbracket{[}\def\testit{#1}\ifx\testit\lbracket
   \let\next=\epsfgetlitbb\else\let\next=\epsfnormal\fi\next{#1}}%
%
\def\epsfgetlitbb#1#2 #3 #4 #5]#6{\epsfgrab #2 #3 #4 #5 .\\%
   \epsfsetgraph{#6}}%
%
\def\epsfnormal#1{\epsfgetbb{#1}\epsfsetgraph{#1}}%
%
\def\epsfgetbb#1{%
%
%   The first thing we need to do is to open the
%   PostScript file, if possible.
%
\openin\epsffilein=#1
\ifeof\epsffilein\errmessage{I couldn't open #1, will ignore it}\else
%
%   Okay, we got it. Now we'll scan lines until we find one that doesn't
%   start with %. We're looking for the bounding box comment.
%
   {\epsffileoktrue \chardef\other=12
    \def\do##1{\catcode`##1=\other}\dospecials \catcode`\ =10
    \loop
       \read\epsffilein to \epsffileline
       \ifeof\epsffilein\epsffileokfalse\else
%
%   We check to see if the first character is a % sign;
%   if not, we stop reading (unless the line was entirely blank);
%   if so, we look further and stop only if the line begins with
%   `%%BoundingBox:'.
%
          \expandafter\epsfaux\epsffileline:. \\%
       \fi
   \ifepsffileok\repeat
   \ifepsfbbfound\else
    \ifepsfverbose\message{No bounding box comment in #1; using defaults}\fi\fi
   }\closein\epsffilein\fi}%
%
%   Now we have to calculate the scale and offset values to use.
%   First we compute the natural sizes.
%
\def\epsfclipon{\def\epsfclipstring{ clip}}%
\def\epsfclipoff{\def\epsfclipstring{\ifepsfdraft\space clip\fi}}%
\epsfclipoff
%
\def\epsfsetgraph#1{%
   \epsfrsize=\epsfury\pspoints
   \advance\epsfrsize by-\epsflly\pspoints
   \epsftsize=\epsfurx\pspoints
   \advance\epsftsize by-\epsfllx\pspoints
%
%   If `epsfxsize' is 0, we default to the natural size of the picture.
%   Otherwise we scale the graph to be \epsfxsize wide.
%
   \epsfxsize\epsfsize\epsftsize\epsfrsize
   \ifnum\epsfxsize=0 \ifnum\epsfysize=0
      \epsfxsize=\epsftsize \epsfysize=\epsfrsize
      \epsfrsize=0pt
%
%   We have a sticky problem here:  TeX doesn't do floating point arithmetic!
%   Our goal is to compute y = rx/t. The following loop does this reasonably
%   fast, with an error of at most about 16 sp (about 1/4000 pt).
%
     \else\epsftmp=\epsftsize \divide\epsftmp\epsfrsize
       \epsfxsize=\epsfysize \multiply\epsfxsize\epsftmp
       \multiply\epsftmp\epsfrsize \advance\epsftsize-\epsftmp
       \epsftmp=\epsfysize
       \loop \advance\epsftsize\epsftsize \divide\epsftmp 2
       \ifnum\epsftmp>0
          \ifnum\epsftsize<\epsfrsize\else
             \advance\epsftsize-\epsfrsize \advance\epsfxsize\epsftmp \fi
       \repeat
       \epsfrsize=0pt
     \fi
   \else \ifnum\epsfysize=0
     \epsftmp=\epsfrsize \divide\epsftmp\epsftsize
     \epsfysize=\epsfxsize \multiply\epsfysize\epsftmp
     \multiply\epsftmp\epsftsize \advance\epsfrsize-\epsftmp
     \epsftmp=\epsfxsize
     \loop \advance\epsfrsize\epsfrsize \divide\epsftmp 2
     \ifnum\epsftmp>0
        \ifnum\epsfrsize<\epsftsize\else
           \advance\epsfrsize-\epsftsize \advance\epsfysize\epsftmp \fi
     \repeat
     \epsfrsize=0pt
    \else
     \epsfrsize=\epsfysize
    \fi
   \fi
%
%  Finally, we make the vbox and stick in a \special that dvips can parse.
%
   \ifepsfverbose\message{#1: width=\the\epsfxsize, height=\the\epsfysize}\fi
   \epsftmp=10\epsfxsize \divide\epsftmp\pspoints
   \vbox to\epsfysize{\vfil\hbox to\epsfxsize{%
      \ifnum\epsfrsize=0\relax
        \special{PSfile=\ifepsfdraft psdraft.ps\else#1\fi\space
             llx=\epsfllx\space lly=\epsflly\space
             urx=\epsfurx\space ury=\epsfury\space rwi=\number\epsftmp
             \epsfclipstring}%
      \else
        \epsfrsize=10\epsfysize \divide\epsfrsize\pspoints
        \special{PSfile=\ifepsfdraft psdraft.ps\else#1\fi\space
             llx=\epsfllx\space lly=\epsflly\space
             urx=\epsfurx\space ury=\epsfury\space rwi=\number\epsftmp\space
             rhi=\number\epsfrsize \epsfclipstring}%
      \fi
      \hfil}}%
\global\epsfxsize=0pt\global\epsfysize=0pt}%
%
%   We still need to define the tricky \epsfaux macro. This requires
%   a couple of magic constants for comparison purposes.
%
{\catcode`\%=12 \global\let\epsfpercent=%\global\def\epsfbblit{%BoundingBox}}%
%
%   So we're ready to check for `%BoundingBox:' and to grab the
%   values if they are found.
%
\long\def\epsfaux#1#2:#3\\{\ifx#1\epsfpercent
   \def\testit{#2}\ifx\testit\epsfbblit
      \epsfgrab #3 . . . \\%
      \epsffileokfalse
      \global\epsfbbfoundtrue
   \fi\else\ifx#1\par\else\epsffileokfalse\fi\fi}%
%
%   Here we grab the values and stuff them in the appropriate definitions.
%
\def\epsfempty{}%
\def\epsfgrab #1 #2 #3 #4 #5\\{%
\global\def\epsfllx{#1}\ifx\epsfllx\epsfempty
      \epsfgrab #2 #3 #4 #5 .\\\else
   \global\def\epsflly{#2}%
   \global\def\epsfurx{#3}\global\def\epsfury{#4}\fi}%
%
%   We default the epsfsize macro.
%
\def\epsfsize#1#2{\epsfxsize}
%
%   Finally, another definition for compatibility with older macros.
%
\let\epsffile=\epsfbox


}{\relax}

% Simplified EPS inclusion macros so we can see how this goes...
% These are layered on Rokicki's dvips material, and are dependent
% on the author's use of that DVI driver.
%
%    \plotone{EPSFILE}
%    \plottwo{EPSFILE}{EPSFILE}
%    \plotfiddle{EPSFILE}{VSIZE}{ROT}{HSF}{VSF}{HTRANS}{VTRANS}
%
% \plotone inserts the plot in a space that is \columnwidth wide; the
% plot is scaled so the horizontal dimension fits in the text width,
% and the vertical dimension is scaled to maintain the aspect ratio.
% \plottwo inserts two plots next to each other in one \columnwidth,
% sort of like "two-up" mode.
%
%    EPSFILE    name of file with EPS
%
% The following arguments are for the \plotfiddle macro which formats
% the \special itself, prepares vspace, etc.  This completely bypasses
% Rokicki's macros that attempt to rationalize the EPS BoundingBox with
% the LaTeX page dimensions.
%
%    VSIZE      vertical white space to allow for plot
%    ROT        rotation angle
%    HSF        horiz scale factor
%    VSF        vert scale factor
%    HTRANS     horiz translation
%    VTRANS     vert translation

%\epsfverbosetrue

\def\eps@scaling{.95}
\def\epsscale#1{\gdef\eps@scaling{#1}}

\def\plotone#1{\centering \leavevmode
    \epsfxsize=\eps@scaling\columnwidth \epsfbox{#1}}

\def\plottwo#1#2{\centering \leavevmode
    \epsfxsize=.45\columnwidth \epsfbox{#1} \hfil
    \epsfxsize=.45\columnwidth \epsfbox{#2}}

\def\plotfiddle#1#2#3#4#5#6#7{\centering \leavevmode
    \vbox to#2{\rule{0pt}{#2}}
    \special{psfile=#1 voffset=#7 hoffset=#6 vscale=#5 hscale=#4 angle=#3}}

% Conveniences - compatible with AASTeX v4.0.

\def\deg{\hbox{$^\circ$}}
\def\sun{\hbox{$\odot$}}
\def\earth{\hbox{$\oplus$}}
\def\lesssim{\mathrel{\hbox{\rlap{\hbox{\lower4pt\hbox{$\sim$}}}\hbox{$<$}}}}
\def\gtrsim{\mathrel{\hbox{\rlap{\hbox{\lower4pt\hbox{$\sim$}}}\hbox{$>$}}}}
\def\sq{\hbox{\rlap{$\sqcap$}$\sqcup$}}
\def\arcdeg{\hbox{$^\circ$}}
\def\arcmin{\hbox{$^\prime$}}
\def\arcsec{\hbox{$^{\prime\prime}$}}
\def\fd{\hbox{$.\!\!^{\rm d}$}}
\def\fh{\hbox{$.\!\!^{\rm h}$}}
\def\fm{\hbox{$.\!\!^{\rm m}$}}
\def\fs{\hbox{$.\!\!^{\rm s}$}}
\def\fdg{\hbox{$.\!\!^\circ$}}
\def\farcm{\hbox{$.\mkern-4mu^\prime$}}
\def\farcs{\hbox{$.\!\!^{\prime\prime}$}}
\def\fp{\hbox{$.\!\!^{\scriptscriptstyle\rm p}$}}
\def\micron{\hbox{$\mu$m}}
\let\la=\lesssim			% For Springer A&A compliance...
\let\ga=\gtrsim
\def\case#1#2{\hbox{$\frac{#1}{#2}$}}
\def\slantfrac#1#2{\hbox{$\,^#1\!/_#2$}}
\def\onehalf{\slantfrac{1}{2}}
\def\onethird{\slantfrac{1}{3}}
\def\twothirds{\slantfrac{2}{3}}
\def\onequarter{\slantfrac{1}{4}}
\def\threequarters{\slantfrac{3}{4}}
\def\ubvr{\hbox{$U\!BV\!R$}}            
\def\ub{\hbox{$U\!-\!B$}}               
\def\bv{\hbox{$B\!-\!V$}}               
\def\vr{\hbox{$V\!-\!R$}}               
\def\ur{\hbox{$U\!-\!R$}}
\def\ion#1#2{#1$\;${\small\rm\@Roman{#2}}\relax}

\let\jnl@style=\rm
\def\ref@jnl#1{{\jnl@style#1}}
\def\aj{\ref@jnl{AJ}}                   
\def\araa{\ref@jnl{ARA\&A}}             
\def\apj{\ref@jnl{ApJ}}                 
\def\apjl{\ref@jnl{ApJ}}                
\def\apjs{\ref@jnl{ApJS}}               
\def\ao{\ref@jnl{Appl.~Opt.}}           
\def\apss{\ref@jnl{Ap\&SS}}             
\def\aap{\ref@jnl{A\&A}}                
\def\aapr{\ref@jnl{A\&A~Rev.}}          
\def\aaps{\ref@jnl{A\&AS}}              
\def\azh{\ref@jnl{AZh}}                 
\def\baas{\ref@jnl{BAAS}}               
\def\jrasc{\ref@jnl{JRASC}}             
\def\memras{\ref@jnl{MmRAS}}            
\def\mnras{\ref@jnl{MNRAS}}             
\def\pra{\ref@jnl{Phys.~Rev.~A}}        
\def\prb{\ref@jnl{Phys.~Rev.~B}}        
\def\prc{\ref@jnl{Phys.~Rev.~C}}        
\def\prd{\ref@jnl{Phys.~Rev.~D}}        
\def\pre{\ref@jnl{Phys.~Rev.~E}}        
\def\prl{\ref@jnl{Phys.~Rev.~Lett.}}    
\def\pasp{\ref@jnl{PASP}}               
\def\pasj{\ref@jnl{PASJ}}               
\def\qjras{\ref@jnl{QJRAS}}             
\def\skytel{\ref@jnl{S\&T}}             
\def\solphys{\ref@jnl{Sol.~Phys.}}      
\def\sovast{\ref@jnl{Soviet~Ast.}}      
\def\ssr{\ref@jnl{Space~Sci.~Rev.}}     
\def\zap{\ref@jnl{ZAp}}                 
\def\nat{\ref@jnl{Nature}}              
\def\iaucirc{\ref@jnl{IAU~Circ.}}
\def\aplett{\ref@jnl{Astrophys.~Lett.}}
\def\apspr{\ref@jnl{Astrophys.~Space~Phys.~Res.}}
\def\bain{\ref@jnl{Bull.~Astron.~Inst.~Netherlands}}
\def\fcp{\ref@jnl{Fund.~Cosmic~Phys.}}
\def\gca{\ref@jnl{Geochim.~Cosmochim.~Acta}}
\def\grl{\ref@jnl{Geophys.~Res.~Lett.}}
\def\jcp{\ref@jnl{J.~Chem.~Phys.}}      
\def\jgr{\ref@jnl{J.~Geophys.~Res.}}    
\def\jqsrt{\ref@jnl{J.~Quant.~Spec.~Radiat.~Transf.}}
\def\memsai{\ref@jnl{Mem.~Soc.~Astron.~Italiana}}
\def\nphysa{\ref@jnl{Nucl.~Phys.~A}}
\def\physrep{\ref@jnl{Phys.~Rep.}}
\def\physscr{\ref@jnl{Phys.~Scr}}
\def\planss{\ref@jnl{Planet.~Space~Sci.}}
\def\procspie{\ref@jnl{Proc.~SPIE}}
\let\astap=\aap
\let\apjlett=\apjl
\let\apjsupp=\apjs
\let\applopt=\ao

% Initialization.

\textwidth 6.5in
\textheight 9.0in
\oddsidemargin \z@
\evensidemargin \z@
\topmargin \z@
\headheight .2in
\headsep .2in
\footheight \z@
\parindent \z@
\parskip 1ex

\voffset=-0.25in
%\hoffset=-0.25in

\ps@kpprophead
\setcounter{secnumdepth}{0}

\thispagestyle{empty}