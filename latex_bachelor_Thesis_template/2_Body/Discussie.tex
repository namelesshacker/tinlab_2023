


\section{Conclusie}
Bij de beoordeling van het gezamenlijke gedeelte staan de volgende drie vragencentraal: 
•Hoe zijn de wensen van de opdrachtgever ge ̈ınterpreteerd?  
Tot wat voorrequirements/specificaties  leiden  deze?     
Anders  gezegd:   Wat  betekentveilig, effici ̈ent, etc.  en wat heb je aan bronnen geraadpleegd om tot eengoede  analyse  te  komen?   (Dit  laatste  hoef  je  niet  te  beschrijven:   hetblijkt immers uit citaten of verwijzingen die je gebruikt.)•\ 

Het model: 
–De modelcriteria van Vaandrager zijn op allerlei manieren tegenstrij-dig.   
Welke keuzes en afwegingen heb je gemaakt en waarom?–Gemodelleerde onderdelen.–Werking van het model 
.•Verificatie: 
–Wat heb je geverifieerd, waarom en hoe? 
–Als je ietsnietkon verifi ̈eren, waarom dan niet? 
–Een  harde  eis  isdater  een  aantal  eigenschappen  geverifieerd  zijn. 
We  modelleren  een  systeem  immers  middels  Kripke  structuren  omharde  uitspraken  over  eigenschappen  van  zo’  n  systeem  te  kunnendoen.Enkele belangrijke opmerkingen: 
•Sommige studenten denken dat ze bij het schrijven van verslagen per kiloworden betaald. . . Het is beslist niet de bedoeling om kilometers tekst teproduceren.  Beter een goed lopend, gestructureerd, korter document daneen waterig plofverslag. 
•Voor het model geldt grofweg hetzelfde:  Deze opdracht is niet een wed-strijdje  complexiteit.  Minimalisme/simplisme  is,  mits  goed  onderbouwd,beter dan een ingewikkeld vlaggeschip. 
•Wat helpt bij het schrijven van een gestructureerd document is het van tevoren  neerzetten  van  een  “skelet”  met  kernwoorden.  Deze  kernwoordenkun je uitsplitsen in begrippen die eronder vallen en zo krijg je, grofweg, dehoofdstukken, subhoofdstukken, etc.  al op papier en kun je de structuurvan je document beter in de gaten houden. 


In de brief is te lezen dat het ministerie van infrastructuur en waterstaat een mo-del wil hebben voor een sluis die volledig geautomatiseerd werkt.  Dit betekentdat er ergens in je model een component moet zitten die, so to speak, de baas is.Die component is verantwoordelijk voor, bijvoorbeeld, het aan- en uitzetten vanpompen,  het  openen  en  sluiten  van  sluisdeuren,  etc.   Deze  “mastercontroller”deelt de lakens uit en communiceert met andere componenten. 

3.3	Queues, integers en verificatieSchepen kunnen gemodelleerd worden met simpele (bounded) integers die in een wachtrij geplaatst worden.  E ́en en ander betekent dat er ergens in je model eenqueue  (of  zelfs  meerdere)  geprogrammeerd  moet  worden.   Wanneer  we  echterdrie queues programmeren met een lengte van vijf,  loopt bij het verifi ̈eren vaneen  eigenschap  het  aantal  states  al  zo  hoog  op  dat  verifi ̈eren  onmogelijk  isgeworden. . . 
Er zijn nu verschillende alternatieven: 
•de queue lengte kleiner maken 
•minder queues gebruiken 
•queues helemaal niet gebruiken 
Die laatste optie impliceert dat een simpele integer gebruikt wordt om het aantalschepen in de sluis bij te houden.  Je bent nu niet meer in staat een individueelschip  te  “volgen”,  maar  je  hebt  de  state  explosion  wel  enorm  teruggebracht.Voorbeideopties is iets te zeggen en het is daarom toegestaan om een model teveranderen of “uit te kleden”, zodat een eigenschap die eerder niet verifieerbaarwas, dat alsnog wordt. . . 

Werken  met  meerdere,  onderling  samenhangende,  modellen  die  gemaaktworden  om  er  specifieke  eigenschappen  mee  te  verifi ̈eren  is  een  normale  zaaken  derhalve  gewoon  toegestaan.   Let  er,  als  je  dit  doet,  wel  op  dat  je  goeddocumenteert wat er met welk model geverifieerd is. 
3.4	Onverwachte omstandigheden 
Wanneer  een  “echt”  systeem  gebouwd  en  in  gebruik  genomen  wordt,  kan  ermet een werkend systeem van alles misgaan: 
•sensor gaat kapot 
•menselijke fout 
•mechanische fout 
•. . .Je hoeft bij het modelleren geen rekening te houden met dit soort omstandighe-den.  Ga er van uit dat een gemodelleerd onderdeel doet wat het geacht wordtte doen. 
3.5	Simplisme vs.  realisme 
Wanneer  we  in  ons  systeem  met  waterhoogte  willen  werken,  zal  die  waardeergens vandaan moeten komen.  Het is realistisch om een sensor te modellerendie  de  waterhoogte  “uitleest”  en  doorgeeft  aan  het  systeem.   Dit  maakt  hetmodel realistischer, maar ook complexer. 
Het  doelbewustnietmodelleren  van  een  sensor  is  dan  ook  verdedigbaar: 
Met  het  verkregen  simplisme  ga  je  de  state  explosion  tegen  en  dat  maakt  hetverifi ̈eren van eigenschappen mogelijk.  Ook hier geldt dat je daar zelf keuzes inmag maken, gesteld dat je ze onderbouwt. 

3.6	Liveness 
Het  verifi ̈eren  van  liveness  kan  voor  subtiele  problemen  zorgen.   Deze  komen met enige regelmaat voort uit wat we in de handleiding lezen: 
The syntax pq denotes a leads to property meaning that whenever pholds eventually q will hold as well. Since Uppaal uses timed automata as theinput model, this has to be interpreted not only over action transitions but alsoover delay transitions.Anders  gezegd:   Als  je  in  je  model  een  state  hebt  zitten  zonder  invariantdie  het  systeem  vroeg  of  laat  uit  die  state  dwingt,  kan  het  systeem  eindeloos 
in die state blijven hangen en zal liveness verificatie niet slagen.  Het verhelpendaarvan kan een hoop werk opleveren. . . 

\section{Discussie}
Het argument `pathname' is de relatieve of absolute locatie van het
bronbestand, de map(pen) gecombineerd met de bestandsnaam. Als je
broncode van een bronbestand laadt, ben je zeker dat de broncode in je
\LaTeX{}-document altijd actueel is en hou je het \LaTeX{}-document
overzichtelijk. Als de broncode niet in dezelfde map of een submap van
het \LaTeX{}-document staat of je gebruikt absolute `pathnames', dan
is het mogelijk dat het verslag niet op andere computers gecompileerd
kan worden. Bij het inleveren van je afstudeerverslag in
\LaTeX{}-formaat zal je hiermee rekening moeten houden.
\subsection{Uitdagingen}

Het argument `pathname' is de relatieve of absolute locatie van het
bronbestand, de map(pen) gecombineerd met de bestandsnaam. Als je
broncode van een bronbestand laadt, ben je zeker dat de broncode in je
\LaTeX{}-document altijd actueel is en hou je het \LaTeX{}-document
overzichtelijk. Als de broncode niet in dezelfde map of een submap van
het \LaTeX{}-document staat of je gebruikt absolute `pathnames', dan
is het mogelijk dat het verslag niet op andere computers gecompileerd
kan worden. Bij het inleveren van je afstudeerverslag in
\LaTeX{}-formaat zal je hiermee rekening moeten houden.
\subsection{Data availability stabdard}
Het argument `pathname' is de relatieve of absolute locatie van het
bronbestand, de map(pen) gecombineerd met de bestandsnaam. Als je
broncode van een bronbestand laadt, ben je zeker dat de broncode in je
\LaTeX{}-document altijd actueel is en hou je het \LaTeX{}-document
overzichtelijk. Als de broncode niet in dezelfde map of een submap van
het \LaTeX{}-document staat of je gebruikt absolute `pathnames', dan
is het mogelijk dat het verslag niet op andere computers gecompileerd
kan worden. Bij het inleveren van je afstudeerverslag in
\LaTeX{}-formaat zal je hiermee rekening moeten houden.

\subsection{Recommended readings}
Het argument `pathname' is de relatieve of absolute locatie van het
bronbestand, de map(pen) gecombineerd met de bestandsnaam. Als je
broncode van een bronbestand laadt, ben je zeker dat de broncode in je
\LaTeX{}-document altijd actueel is en hou je het \LaTeX{}-document
overzichtelijk. Als de broncode niet in dezelfde map of een submap van
het \LaTeX{}-document staat of je gebruikt absolute `pathnames', dan
is het mogelijk dat het verslag niet op andere computers gecompileerd
kan worden. Bij het inleveren van je afstudeerverslag in
\LaTeX{}-formaat zal je hiermee rekening moeten houden.


\subsection{Reflectie}

Ik heb erg veel geleerd van het analyseren van de vershillende requirements en specificaties en het opzetten van een model in Uppaal. Een dergelijk model opzetten had ik namelijk nog nooit gedaan. Het uitvoeren van onderzoek heb ik eerder gedaan. Ook de toetsing van het model met behulp van proposities heb ik nog nooit gedaan. Verder heb ik de kennis die had van programmeren/ design pattersn gebruikt om de verschillende templates in mijn Uppaal model van elkaar te onderscheiden. Het leukste onderdeel van het project vond ik hoe mijn templatemodel deadlockvrij werkte. Voor de verificatie van het model heb ik veel achtergrondinformatie opgezet, en het is mooi om te zien dat je met enkele duidelijke zinnen kan aantonen of een propositie geldig is of niet.  Verder had ik moeite met het opstellen van de juiste veiligheidseisen bij het model. Ik had aangenomen dat ik het project niet zou halen omdat ik de opdracht niet in teamverband heb uitgevoerd. Ik ben toch blij dat ik een concept heb opgeleverd dat ik kan toetsen aan de doormijzef opgestelde eisen en dat ik met mijn huidige kennis de proposities uit de requirements kan toetsen.
