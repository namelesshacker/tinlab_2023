 
\label{chapter:Conclusions}
\thispagestyle{myheadings}

% set this to the location of the figures for this chapter. it may
% also want to be ../Figures/2_Body/ or something. make sure that
% it has a trailing directory separator (i.e., '/')!
\graphicspath{{4_Conclusion/Figures/}}


\section{Conclusies}


\newpage
%\section{Summary of the thesis}
\section{Eindverantwoording}

Ik heb erg veel geleerd van het analyseren van de vershillende requirements en specificaties en het opzetten van een model in Uppaal. Een dergelijk model opzetten had ik namelijk nog nooit gedaan. Het uitvoeren van onderzoek heb ik eerder gedaan. Ook de toetsing van het model met behulp van proposities heb ik nog nooit gedaan. Verder heb ik de kennis die had van programmeren/ design pattersn gebruikt om de verschillende templates in mijn Uppaal model van elkaar te onderscheiden. Het leukste onderdeel van het project vond ik hoe mijn templatemodel deadlockvrij werkte. Voor de verificatie van het model heb ik veel achtergrondinformatie opgezet, en het is mooi om te zien dat je met enkele duidelijke zinnen kan aantonen of een propositie geldig is of niet.  Verder had ik moeite met het opstellen van de juiste veiligheidseisen bij het model. Ik had aangenomen dat ik het project niet zou halen omdat ik de opdracht niet in teamverband heb uitgevoerd. Ik ben toch blij dat ik een concept heb opgeleverd dat ik kan toetsen aan de doormijzef opgestelde eisen en dat ik met mijn huidige kennis de proposities uit de requirements kan toetsen.



{\bf Important}: In the list of references at the end of thesis, abbreviated journal and conference titles aren't allowed. Either you must put the full title in each item, or create a List of Abbreviations at the beginning of the references, with the abbreviations in one column on the left (arranged in alphabetical order), and the corresponding full title in a second column on the right.  Some abbreviations, such as IEEE, SIGMOD, ACM, have become standardized and accepted by librarians, so those should not be spelled out in full.



\newpage
 
\section{Discussie}
\subsection{Future work}
\subsubsection{Hoogte waterniveau}

\subsubsection{type deuren naar waterniveau}
De sluis kan ook rekening houden met waterniveau van hoog naar laag.
Als een schip naar binnen vaart moet de sluis weten welk schip ook werer naar buiten vaart en aan welke kant.

\subsubsection{voorrang uitvarend op invarend}
Als een schip uitvaart komt er een moment dat een sluis ruimte vrij heeft. Voordat de sluisdeur sluit nadat eenschip is vertrokken kan er nog een polling worden gedaan naar alle schepen in de buurt om te zien of deze willen en kunnen invaren.

\subsubsection{stoplicht invarend en stoplicht uitvarend}
Een handige functionaliteit is dat voor invarende schepen er een stoplicht is en voor uitvarende schepen. Anders ontstaat er een probleem van collission. 

\subsubsection{Volgorde}
Kunnen aantonen dat schepen kunnen worden behandeld met voorkeur, wie het eerst komt die het eerste in behandeling wordt genomen.

 


\cite{para}
\cite{nuseibeh2000requirements}
\cite{modelchecking}
\cite{leveson1993investigation}
\cite{royce1987managing}