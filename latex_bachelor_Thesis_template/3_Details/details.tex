\chapter{Notulen}
\label{chapter:Details}
\thispagestyle{myheadings}

% set this to the location of the figures for this chapter. it may
% also want to be ../Figures/2_Body/ or something. make sure that
% it has a trailing directory separator (i.e., '/')!
\graphicspath{{3_Details/Figures/}}
%
%The use of Type 1 fonts and font embedding into the document are both dependent on a specific Latex installation and even on operating system. There is a good chance that it will work with no problem for you. However, should your thesis PDF be returned, please consider the following remedies discovered by students over many years. 
%
%\section{Type 1 fonts}
%
%All Boston University thesis and dissertation submissions must use only Type 1 fonts to assure high-quality rendering. Type 3 fonts are not acceptable.
%
%For some students adding the following two lines in ``thesis.tex'' preamble has worked:\\
%%
%{\tt
%$\backslash$usepackage[T1]\{fontenc\}\\
%$\backslash$usepackage{pslatex}
%   } 
%
%
%The easiest way to check if fonts are embedded well and of what type, is to use Adobe Acrobat's Preflight -- it shows exactly where the Type 3 fonts are in the thesis. You can learn more here: \url{https://community.adobe.com/t5/acrobat/figure-out-where-a-specific-font-is-used-in-a-pdf/m-p/10880057?page=1#M238035}
%
%If you don't have Adobe Acrobat (BU students get it for free), you can quickly check which fonts have which type by looking into Files $>>$ Properties $>>$ Fonts, but it doesn't tell where the text with a specific font type is.
%
%{\bf Linux/Unix}: If you are using LaTeX or Unix, the problem is that, by default, LaTeX uses Type 3 fonts. Since most users have a tendency to use the default settings, then Type 3 fonts will be used by default. You can try to change the first line in the preamble in ``thesis.tex'' to:\\
%%
%{\tt $\backslash$documentstyle[12pt,times,letterpaper]\{report\}}
%
%\noindent
%since then Times fonts will be used (which are not Type 3). If there are mathematical formulas in the text, it is better to use:\\
%%
%{\tt $\backslash$documentstyle[12pt,times,mathptm,letterpaper]\{report\}}
%
%
%\section{Font embedding}
%
%All fonts must be embedded into the final PDF file. If they are not, sometimes equations may look strange or may not show up at all for several pages. This is often due to unembedded font problem. Should you have a font-embedding issue, this page may prove useful:\\
%%
%\url{https://www.karlrupp.net/2016/01/embed-all-fonts-in-pdfs-latex-pdflatex}
%
%For those using Overleaf, this page might help:
%\url{https://www.overleaf.com/learn/latex/Questions/My_submission_was_rejected_by_the_journal_because_%22Font_XYZ_is_not_embedded%22._What_can_I_do%3F}




\usepackage{pgf}
\newcommand\setform{\pgfqkeys{/form }}
\setform{field1/.store in=\fieldi,
	field2/.store in=\fieldii,
}


%\addtolength{\oddsidemargin}{-.875in}
%\addtolength{\evensidemargin}{-.875in}
%\addtolength{\textwidth}{1.75in}

%\addtolength{\topmargin}{-.875in}
%\addtolength{\textheight}{1.75in}


\newcommand\myform{%
	\fboxrule=0.4pt
	
	
	\fbox{\begin{minipage}{\textwidth}
			\fbox{\begin{minipage}[t][3cm][t]{0.25\textwidth}
					Naam vergadering
			\end{minipage}}%
			\fbox{\begin{minipage}[t][3cm][t]{0.25\textwidth}
					Datum en plats
			\end{minipage}}%
			\fbox{\begin{minipage}[t][3cm][t]{0.44\textwidth}
					Namen aanwezigen
			\end{minipage}}
			\fbox{\begin{minipage}[t][1cm][t]{0.98\textwidth}
					Opening en goedkeuring
			\end{minipage}}
			
	\end{minipage}}
	
	\fbox{\begin{minipage}{\textwidth}
			
			\fbox{\begin{minipage}[t][3cm][t]{0.98\textwidth}
					Ingekomen stukken en rondvraag
			\end{minipage}}
	\end{minipage}}
	
	\fbox{\begin{minipage}{\textwidth}
			\fbox{\begin{minipage}[t][10cm][t]{0.98\textwidth}
					Sluiting
			\end{minipage}}%
			
			
	\end{minipage}}
}


\newpage

\begin{tabular}{*{15}{|l|l|l|l|l|l|l|}} \hline
  
	  \multicolumn{2}{|l|}{Onderwerp}   &\multicolumn{2}{|l|}{Besluit}   &\multicolumn{2}{|l|}{Wie}           &\multicolumn{2}{|l|}{Gereed}                \\ \hline
	\multicolumn{2}{|l|}{ }   &\multicolumn{2}{|l|}{ }   &\multicolumn{2}{|l|}{ }           &\multicolumn{2}{|l|}{ }                \\ \hline
	
	
 \multicolumn{7}{|l|}{ }   															\\ \hline
 Step  &  Test steps & Test data & expected result &Acual result &(pass or fail)&notes  \\ \hline
	 
	
\end{tabular}




\setform{field1 = G. Wales,
	field2 = Mathematics}
\myform


\begin{tabular}{*{8} {|l|l|l|l|}} \hline
 
	Onderwerp & Besluit & Wie &Gereed \\ \hline
	
\end{tabular}

Naam
Datum en plaats
namen van aanwezigen
Verzendlijst



Openingen
Goedkeuring
ingekomen stukken
Rondvraag
Sluiting


Onderwerp, besluit, wie, gereed



\begin{center}  
	\begin{tabular}{ | l | l | l | p{5cm} |} % you can change the dimension according to the spacing requirements  
		\hline  
		Onderwerp & Besluit & Wie &Gereed \\ \hline  
		Orange & Fruit & Vitamin C & It is fruit, which is full of nutrients and low in calories. They can promote clear, healthy skin and also lowers the risk for many diseases. It reduces cholesterol and also helps in building a healthy immune system.\\ \hline  
		
		Cauliflower & vegetable & B-Vitamins & It is the vegetable, which is high in fiber and B-Vitamins. It also provides antioxidants, which help in fighting or protect against cancer. It enhances digestion and has many other nutrients.\\ \hline  
		
	\end{tabular}  
\end{center}  


\chapter{Queries}
\begin{verbatim}
	
	Queries
	Sluis.Draining-->Deuren.laag_open
	Deuren.laag_open-->Stoplicht.Green
	E<> (Ship.ship_can_move&&Stoplicht.Green)
	A[] not (Stoplicht.Green && not (Deuren.hoog_open||Deuren.laag_open||Deuren.stopgaplow1||Deuren.stopgaplow2||Deuren.stopgaphigh1||Deuren.stopgaphigh2))
	A[] not ((Deuren.hoog_open||Deuren.laag_open||Deuren.Opening_laag||Deuren.Opening_hoog||Deuren.Closing_hoog||Deuren.Closing_laag) && (Sluis.Draining||Sluis.Filling||Sluis.draining2||Sluis.Filling2))
	Sensor.Wait-->Sensor.Wait
	Stoplicht.Green-->Stoplicht.Green
	(Deuren.hoog_open||Deuren.laag_open)-->(Deuren.laag_open||Deuren.hoog_open)
	Deuren.laag_open-->Deuren.Closed
	Deuren.hoog_open-->Deuren.Closed
	Deuren.Closed-->Stoplicht.Red
	Ship.ship_can_move-->Deuren.Closed
	Deuren.hoog_open-->Stoplicht.Green
	Ship.ship_can_move-->Stoplicht.Green
	A[] not (Deuren.laag_open && Deuren.hoog_open)
	Ship.ship_can_move-->Ship.ship_can_move
	A[] not (Deuren.laag_open && Sluis.water != Sluis.water_laag)
	A[] not (Deuren.hoog_open && Sluis.water != Sluis.water_hoog)
	A[]not deadlock
	
	Project declaraties
	//Declarations
	
	chan boot_hoog;
	chan boot_laag;
	chan changedoor_low;
	chan changedoor_high;
	chan ship_moves;
	chan ship_abletomove;
	chan changelight;
	
	\\Sluis declaraties
	const int water_laag=0;
	const int water_hoog=10;
	const int water_median=(water_hoog+water_laag)/2;
	int[water_laag,water_hoog] water=water_median;
	clock x;
	\\Stoplicht declaraties
	
	\\Ship declaraties
	clock x;
	\\Sensor declaraties
	
	\\Deuren declaraties
	bool stoplicht_hoog=false;
	bool stoplicht_laag=false;
	clock x;
	
	
	\\System declaraties
	system Deuren,Sensor,Sluis,Ship,Stoplicht;
	
	
	Uitleg
	Als het schip boven is, dan is waterlvel gelijk aan hoog, filling valve is dicht, lower gates zijn gesloten, uppergates zijn open,empty valve is dicht. 
	Schip is in waterlock, waterlevel is hoog, filling valve is dicht, lower gates gesloten, upper gates gesloten, empty valve is open. 
	Schip is dan laag, waterlevel gelijk aan laag, filling valve is dicht, lowergates zijn open, uppergates zijn dicht, empty valve is dicht.
	AtArrivalHigh
	
	AtArrivalLow
	Als schip beneden is dan is waterlevel gelijk aan laag, filling valve is dicht, lower gates zijn open, upper gates zijn dicht, empty valve is open. 
	Schip is in water lock, waterlevel is laag, flilling valve is open, lower gates zijn gesloten, upper gates zij gesloten, empty valve is dicht,. 
	Schip is dan hoog, waterlevel is gelijk aan hoog, filling valve is dicht, uppergates zijn open, lowergates zijn dicht, filling valve is dicht
	
	
	
\end{verbatim}


\chapter{Testresultaten}



\subsubsection{onderdeleel van de test}



\begin{tabular}{*{15}{|l|l|l|l|l|l|l|}} \hline
	\multicolumn{7}{|l|}{project name}                                                               \\ \hline
	\multicolumn{4}{|l|}{Test case ID}   &\multicolumn{3}{|l|}{Test designed by}                           \\ \hline
	\multicolumn{4}{|l|}{test priority (low/medium/high)}   &\multicolumn{3}{|l|}{Test design date}                           \\ \hline
	\multicolumn{4}{|l|}{Module name}   &\multicolumn{3}{|l|}{Test executed by}                           \\ \hline
	\multicolumn{4}{|l|}{Test title}   &\multicolumn{3}{|l|}{Test execution date}                           \\ \hline
	\multicolumn{4}{|l|}{Description}   &\multicolumn{3}{|l|}{ }                           \\ \hline 		
	\multicolumn{7}{|l|}{ }   																\\ \hline
	\multicolumn{7}{|l|}{Pre condition}                                                               \\ \hline
	\multicolumn{7}{|l|}{Dependencies}                                                               \\ \hline
	\multicolumn{7}{|l|}{ }   															\\ \hline
	Step  &  Test steps & Test data & expected result &Acual result &(pass or fail)&notes  \\ \hline
	
\end{tabular}





\chapter{Reflectie}

Ik heb erg veel geleerd van het analyseren van de vershillende requirements en specificaties en het opzetten van een model in Uppaal. Een dergelijk model opzetten had ik namelijk nog nooit gedaan. Het uitvoeren van onderzoek heb ik eerder gedaan. Ook de toetsing van het model met behulp van proposities heb ik nog nooit gedaan. Verder heb ik de kennis die had van programmeren/ design pattersn gebruikt om de verschillende templates in mijn Uppaal model van elkaar te onderscheiden. Het leukste onderdeel van het project vond ik hoe mijn templatemodel deadlockvrij werkte. Voor de verificatie van het model heb ik veel achtergrondinformatie opgezet, en het is mooi om te zien dat je met enkele duidelijke zinnen kan aantonen of een propositie geldig is of niet.  Verder had ik moeite met het opstellen van de juiste veiligheidseisen bij het model. Ik had aangenomen dat ik het project niet zou halen omdat ik de opdracht niet in teamverband heb uitgevoerd. Ik ben toch blij dat ik een concept heb opgeleverd dat ik kan toetsen aan de doormijzef opgestelde eisen en dat ik met mijn huidige kennis de proposities uit de requirements kan toetsen.


\usepackage{pgf}
\newcommand\setform{\pgfqkeys{/form }}
\setform{field1/.store in=\fieldi,
	field2/.store in=\fieldii,
}


%\addtolength{\oddsidemargin}{-.875in}
%\addtolength{\evensidemargin}{-.875in}
%\addtolength{\textwidth}{1.75in}

%\addtolength{\topmargin}{-.875in}
%\addtolength{\textheight}{1.75in}


\newcommand\myform{%
	\fboxrule=0.4pt
	
	
	\fbox{\begin{minipage}{\textwidth}
			\fbox{\begin{minipage}[t][3cm][t]{0.25\textwidth}
					Betrokken partij
			\end{minipage}}%
			\fbox{\begin{minipage}[t][3cm][t]{0.25\textwidth}
					Verantwoordelijk
			\end{minipage}}%
			\fbox{\begin{minipage}[t][3cm][t]{0.44\textwidth}
					datetimestamp here
			\end{minipage}}
			\fbox{\begin{minipage}[t][1cm][t]{0.98\textwidth}
					Korte notitie 
			\end{minipage}}
			
	\end{minipage}}
	
	\fbox{\begin{minipage}{\textwidth}
			
			\fbox{\begin{minipage}[t][3cm][t]{0.98\textwidth}
					test
			\end{minipage}}
	\end{minipage}}
	
	\fbox{\begin{minipage}{\textwidth}
			\fbox{\begin{minipage}[t][10cm][t]{0.98\textwidth}
					Foto incident
			\end{minipage}}%
			
			
	\end{minipage}}
}



\setform{field1 = G. Wales,
	field2 = Mathematics}
\myform





