\hoofdstuk{rampen extra}

\paragraph{bijlmerramp}

\begin{description}
	\item[Beschrijving]
	\item[Datum en plaats] 
	\item[Oorzaak]
	%Beschrijf wat er mis ging in termen van het vier variabelen model/requirements/specificaties
\end{description}
Motor 3 (de binnenste motor aan de rechtervleugel van het vliegtuig) brak af, beschadigde de vleugelkleppen en botste tegen motor 4 die vervolgens ook afbrak.
De ernst van de situatie werd op Schiphol niet goed ingezien. Dit kwam onder meer doordat lost in de luchtvaart de gebruikelijke term is om het verlies van motorvermogen te melden. Op Schiphol werd er dan ook van uitgegaan dat er twee motoren waren uitgevallen. Dat ze letterlijk verloren waren wist men niet. Gezien het grote aantal handelingen dat de bemanning in een paar minuten moest uitvoeren en de keuzes die de piloot maakte, veronderstelde de parlementaire enquêtecommissie die de ramp later zou onderzoeken dat ook de bemanning waarschijnlijk niet heeft geweten dat beide motoren van de rechtervleugel waren afgebroken. De buitenste motor van een 747 is vanuit de cockpit slechts met moeite zichtbaar en de binnenste motor helemaal niet.

Op de avond van de 4e oktober 1992 was landingsbaan 06 (de Kaagbaan) in gebruik. De piloot verzocht de luchtverkeersleiding op Schiphol echter een noodlanding te mogen maken op de Buitenveldertbaan (baan 27). Waarom hij juist deze baan koos, is nooit duidelijk geworden. Een keuze voor deze baan lag niet voor de hand; omdat de wind uit het noordoosten kwam, zou het toestel met flinke staartwind moeten landen. Langs de landingsbaan waren enkele grote brandweerwagens van Schiphol geplaatst. Deze zogeheten crashtenders moesten een brand tijdens de landing meteen blussen. Na de crash werd één zwarte doos teruggevonden. De bijbehorende band was in vier stukken gebroken, waardoor de laatste 2 minuten en 45 seconden ervan niet meer te gebruiken waren. De doos werd voor onderzoek naar Washington gestuurd en leverde uiteindelijk onderstaande informatie op.
Om goed uit te komen voor de landingsbaan vloog het beschadigde toestel eerst nog een rondje boven Amsterdam. Tijdens dit rondje gaf de gezagvoerder de copiloot opdracht de vleugelkleppen (flaps) uit te schuiven. Links schoven de kleppen uit, maar doordat de afgebroken motor 3 de rechtervleugel had beschadigd schoven de kleppen op die vleugel niet uit. Als gevolg hiervan kreeg het toestel links meer draagvermogen dan rechts. De piloot meldde aan de verkeersleiding dat er ook problemen met de flaps waren.
Aanvankelijk ging het aanvliegen van de Buitenveldertbaan goed. Op het moment dat het vliegtuig daalde tot onder de 1500 voet en snelheid minderde, raakte het echter compleet onbestuurbaar en maakte het een ongecontroleerde, scherpe bocht naar rechts. Over de radio was te horen dat de gezagvoerder zijn copiloot in het Hebreeuws opdracht gaf om alle kleppen in te trekken en het landingsgestel uit te klappen. Vervolgens meldde de copiloot in het Engels aan de luchtverkeersleider dat het toestel zou gaan neerstorten. Uit later onderzoek bleek dat het vliegtuig eerder enkel recht bleef vanwege de hoge snelheid (280 knopen, zijnde 519 km/u). Doordat de rechtervleugel beschadigd was, was het moeilijker om het vliegtuig recht te houden. Alleen de hoge snelheid zorgde ervoor dat er nog voldoende draagvermogen was. Toen bij het inzetten van de landing de snelheid verlaagd werd, werd het draagvermogen van de rechtervleugel echter dusdanig gering dat het toestel niet meer onder controle te houden was en een duikvlucht naar rechts maakte.

\cite{aviationsafety04101992airplaneCrashBijlmer}

\paragraph{ramp turkisch airlines vlucht 1951}

\begin{description}
	\item[Beschrijving]
	\item[Datum en plaats] 
	\item[Oorzaak]
	%Beschrijf wat er mis ging in termen van het vier variabelen model/requirements/specificaties
\end{description}
Inadequaat handelen van de piloten ondanks een defecte hoogtemeter en onvolledige instructies van de luchtverkeersleiding/

\cite{catsr25022009Boeing737AmsterdamCrash}

\cite{zuilen23022019Tijdlijnpoldercrash}
\cite{wikinews04032009techfoutailines1951}
\cite{luchtvaartnieuws21012020boeing737conclusies}
\cite{adformatie280220209communicatiegebreken}
\cite{spinnael25022009onderzoekpolderbaancrash}
\cite{crashTurkishAirlines}
\cite{flightradar24}
\cite{flightstatstracker}


\paragraph{tjernobyl}

\begin{description}
	\item[Beschrijving]
	\item[Datum en plaats] 
	\item[Oorzaak]
	%Beschrijf wat er mis ging in termen van het vier variabelen model/requirements/specificaties
\end{description}
Een ramp bij een kernreacor in de sovjetunie. Door een bedieningsfout in een testprocedure werd het vermogen van de koelinstallaties negatief beinvloed. Door een ontwerpfout in de noodstopprocedure kon in het systeem niet snel genoeg schakelen om remmende invloed uit te oefenen op het toenemende vermogen van de reactorkernen. Met brand en eksplosie tot gevolg.

\cite{INSAVienna1992Chernobyl}
Tsjernobyl
\cite{wikiTjernobyl}
\cite{rivmTjernobyl}
\cite{andereTijdenTjernobyl}
wat er is gebeurd en hoe het leven verdergaat
\cite{kingskey19042022tjernobyl}
pernsioenfondsen en de tjernobyl ramp
In 2021 worden mensen nog steeds blootgesteld blijkt ut een gezamelijk onderzoek van greenpeace en oekraiense wetenschappers
stijging van de nucliaire activiteit gemeten in tjernobyl
Het toerisme  aspect
De chronologie
\cite{erikbork26042023reactor4}
\cite{nosTjernobyl30jaarlater}
Dieren in de omgeving van tjernobyl
De chronologie
Echtreme droogte zorgd voor gevaar
\cite{knmi04052021tjernobylbosbrand}
\cite{dodonovaKVIRisicoTjernobyl}
Joernalistiek, entertainment en de waarheid
\cite{dumarey04062020verhaalTjernobylWaarheid}
Een onderzoek
Huidige gevolgen van de explosie van toen
\cite{sparkesNewScientistTjernoby}
De ramp, hoe de mensen ermee omgingen en hoe er nu geleef wordt
evaluatieonderzoek en amatregeen
\cite{kernenergiened26041986chronologiemaatregelen}
\cite{mapszoneReactor}
Invloed van de mens op de omgeving
Heroplevende splijtingsreacties
docu van schooltv
Radioactiviteit bereikt nederland
documentaire en maatregelen
\cite{kernhistoriek15062021tjernobyl}
Het verhaal van een overledende
Toerisme
toerisme
toerisme
Dieren in de omgevong
Toevluchtsoord voor vluchtelingen van de oorlog met russische seperatisten
Ouderen die terugkeerden naar hun woonplaats na de gedwongen verhuizing door de autoriteiten
De straling neemt weer toe
Lessen geleerd van tjernobyl
\cite{nucleairforumFeitenTjernobyl}
Toerisme
Bosbrand in tjernobyl
invloed van de ramp op belgie
\cite{kernongevalTjernobylFancGov}
Boek recensie
Fotos en berekeningen
ontmanteling en toerisme
Belangrijke lessen en overeenkomsten
De journalistieke waarheid van de koude oorlog
De lessen van
\cite{arendswolters062019lessenTjernobyl}
Een toristenattractie maken van tjernobyl
De radioactieve straling toen en nu
de 30km zone door de ogen van toeristen
artikel
stedentrip
rapport
\cite{damveld08052020tjernobyl}
slapend monster
docu
krantenartikel
hbo serie
docuserie
de  nieuwe sacrofaag
hulp aan slachtoffers
slapende reactor
krantenartikel
\cite{deVriestjernobylHolland}
hbo serie
internationale gevolgen
toerisme
nieuwe koepel
media communicatie
docu
dieren
koepel
koepel
\cite{ing3enieur29042015antistralingskoepel}
toerisme
toeristisch reiperspectief
toerisme
niwe koepel
overschakelen naar duurzaamheid
docu
tjernobyl wekt nu duurazme energie
toerisme
overeenkomsten tjernobyl en fukushima
drank en sla uit tjernobyl
geen efficiente opslag is mogelijk
wetenschappelijke artikelen
zaterdag 26 april 1986. Er vind routineonderhoud plaats bij reactor 4, De controle wordt uitegevoerd door de dagploeg. Vnwege een test wordt jhet koelsysteem uitgeschakeld. Door omstandigheden wordt de test uitgesteld en wordt de verantwoordelijkheid overgedragen aan de avondploeg.
De operator maakt bedieningsfouten waardoot de reactor bijna stil komt te liggen. En vervolgens probeert hij de reactor weer op gang te brengen. ondanks de snelle temperatuurstijging wordt het experiment doorgezet. Dan wordt ook het veiligheidssysteem stilgelgd. Terwijl het koelwater langzaam opwarmt, sluit hij de klep waarlangs de stoom naar de generator stroomt.

De temperatuur van de reactorstaven neemt daarna snel toe. Terwijl er een oncontroleerbare kettingreactie op gang komt, laat het personeel in paniek de regelstaven zakken om de warmteontwikkeling af te remmen. Het is dan echter al te laat. Door een ontwerpfout loopt het vermogen razendsnel op tot 33.000 megawatt, ruim tien keer hoger dan normaal.

In een oogwenk verandert al het koelwater in stoom. De ontploffing die daarop volgt, blaast het 2000 ton zware deksel van de reactor af.

In de ravage vat het gloeiend hete grafiet in de reactor spontaan vlam. De uitslaande brand en een tweede explosie voeren een radioactieve rookwolk tot 8 kilometer hoogte. 
In een poging het vuur in reactor 4 te doven, storten helikopters vanuit de lucht zand, lood en boorzuur in de reactorkern. Het mag echter niet baten.

Intussen is de nucleaire brandstof zo heet geworden dat die door de bodem van het reactorvat dreigt te smelten. Als dat gebeurt, kan het bluswater onder het vat in één klap verdampen en dreigt een derde explosie die een groot deel van Europa onbewoonbaar zal maken. Om dit te voorkomen moet het water hoe dan ook worden weggepompt.

Drie brandweermannen wagen zich daarvoor in de ruimte onder de reactor, blootgesteld aan 300 sievert per uur, 300.000 keer de dosis die een Nederlander jaarlijks maximaal mag oplopen. Ze slagen daarin, maar twee van hen overlijden enkele dagen later aan acute stralingsziekte.

Hoewel geigertellers de dag na de ramp onrustbarende waarden aangeven, slaat het plaatselijk bestuur geen alarm. De bevolking is het niet gewend om vragen te stellen.

De volgende dag blijkt er wel degelijk iets ernstigs aan de hand te zijn. In een lange rij bussen worden de 135.000 inwoners op 27 april uit het besmette gebied geëvacueerd, om er nooit meer terug te keren.

De ramp is dan nog steeds geen wereldnieuws. De Sovjetautoriteiten blijken er niet eens van op de hoogte te zijn – president Gorbatsjov klaagt later dat hij via Zweden aan zijn informatie moest komen.
\cite{verschuur14012013tjernobylreports}
\cite{paperlessarchivesTjernobyl}
\cite{vargos082000tjernobylconcerns}
\cite{mauroNuclearRiskSociety}
\cite{vienna06092005LookingBack}

\paragraph{therac-25}

\begin{description}
	\item[Beschrijving]
	\item[Datum en plaats] 
	\item[Oorzaak]
	%Beschrijf wat er mis ging in termen van het vier variabelen model/requirements/specificaties
\end{description}
Softwarefout uit zich als hardwarefout de klachtafhandeling geen onderzoek geen second opinion is prioriteit wel 
gechecked na onderzoek bellen en geen prioriteit aanwezig te zijn alleen importeurs en fabriken mogen fouten 
in frabrieksinstellingen rapporteren 
Therac25 Systeem ligt plat veel voorkomende eror stdaardafhandeling om de error te verwerpen resultaat: 
de patient kreeg overdosis patient overleden onderzoek opgestart, stuatie niet reproduceerbar foutmarkering: 
gezien als uitzonderlijk, software aanpassing van groote magnitude 5; de oorzaak was waarschijlijk mechanisch 
maar neit vastgesteld; conceptueel odel niet aangepast probleemclassicificatie door autorititen het probleem 
en de impact daarvan anar beneden bijgesteld AEFL doe gedeeltelijke aanpassing om hardware na berisping 
Canadese autoriteit 
Derde patient overleden door eythema AECL wijst alle doodsoorzaken af AECL beweert dat geen vergeli- 
jkbare voorvalle bij andere machines of patienten zijn voorgekomen geen vervolgonderzoek vanwege garanties 
bedrijf gaat uit van geen mogelijke functionele fout 
vierde patient overleden aan overdodis ontstaan door bug in software onjuiste aanduiding bij de foutmelding 
verkeerde reactie/invoer ddoor operator communicatie tussen patient en operator werd onvoldoende gemon- 
itorred ( apparatuur niet aangesloten, en audio monitor kapot) engineer van AECL stelt geen fouten vast 
Engineer AECl kan fout niet reproduceren Geen communicate tussen bedrijf en uitgezonden technisci over 
vergelijkbare probleemgevallen 
vijfde geval malfunction 54 leidt tot overdosis en de dood fout gereproduceerd door operator bedrijf fout 
was daa entryspeed herpublicatie van de ongevallen en de eerdere ongevallen in de meia apparaat wel nog in 
gebruik genomen niet handig, waarschuwingsberichten en aanwijzingen voor een bugfix naar de gebruikers door 
druk van fda is bedrijf op zoek gegaan naar permanente oplossing 
zesde geval software fout door softwarefout otntstaat lightstruct .. op de patient na onderzoek door AECL 
blijkt niet alleen hardware de oorzak gebruikers direct geinformeerd oplossing gevonden, media ingeschakeld om transparantie af te dwingen door de gebruikersgroep en de FDA AECL gedwongen functionaliteit aan te passen 
Engineers hebben meer studie moeten maken van gebruikte technologie en onderhoudbaarheid daarvan 
sheets
\cite{rogaway2004therac25}
~\cite{wikiTherac25}
reproduceren van de error. IN dit stuk wordt uitgelgd hoe het product werkt en waarom bepaalde beslssingen zijn genomen in de ontwerp/productiefase
\cite{lynch2017theracRaceConditions}
kort artikel met daarin een opsomming van alle fouten in het systeem en een korte uitleg
\cite{lim1998theracdisaster}
uitgebreid artikel over hoe de fout werd gereproduceerd en de resultaten daaruit voortkwamen. Alsnog werden er na de reproductie fase nog meer fouten gevonden.
\cite{fabio26102015therac25}
artikel
\cite{ethicsunwrappedTherac25}
onderzoeksartikel waarin de bug wordt uitgelgd: de racecondities, de bytepositie en het testen worden berkitiseerd envenals andere onderdelen van het softwareproces
onrealistisch testplan. In dit artikel egt de auteur het belang nog eens uit van goede requirements en implementatie, niet de software is waar het probleem ligt
geschiedenis
\cite{casesHistoryTherac25}
artikel
\cite{caballero2019Therac25}
computer error. De ongeval en de malfunction nog een keer uitgelegd
\cite{rose1994theracFatalDose}
rapport
\cite{levesonMITTherac25}
\cite{grant1978theracevaluation}
onderzoeksartkel
\cite{turnerTheracAccidentsInvestigations}
\cite{turner1993TheracAccidentsInvestigations}
uitgebreid artikel gaat hier ook wat meer over de hardware
\cite{wang2017industrialdesignengineering}
artikel waarin in 3 delen de problemaiekwordt blootgesteld
\cite{levesonturner1993theracpart2}
case study sheets
artikel waarin vooral de fabriikant ervan langs krijgt
\cite{porelloTheraccFailure}
lessons learned. Vooral de begrippen betrouwbaarheid, welgevalligheid, veilgheid en gebruiksvriendelijkheid
\cite{theracIncidents}
root-cause analysis
case study
\cite{huffbrown2004casestudyethicatherac}
case study
\cite{sebowikimedicalradiation}
opzetten van systematische acceptaatie test met therac als voorbeeld
\cite{hsia1995testtherac25}
artikel waarin een diagnose plaatvindt voor het bedrijf en de ingenieur/ontwerper
\cite{magsilvaTheracTesting}
rapport
oorzaken aangegeven in artikel
\cite{chemeuropetherac25}
het onderzoek en enkele ontwerptekeningen en oplossingen
\cite{statsenko10102016Therackillerbug}
\cite{therac25casestudy}
\cite{thomas1994theracinLotos},
\cite{twitter2019programmerbehindtherac}
wiki
\cite{wikibookstherac}
analyse
\cite{bozdagTherac25}
samenvatting
\cite{levesonTurnerTheracAbstract}
rapport over de fouten die de verschillende partijen hebben gemaakt( overheid, ingenieurs, bedrijf, operators) en de verbeterpunten
onderzoeksrapport
slides online over het technisch mankement
Wat is er gebeurd, nou het volgende:
Normal radiation treatments: 6,000 rads over a 3 week period, under certain conditions Therac-25 was delivering 60,000 rads during one session.
En wat ging er mis?
Paradigm Shift
Therac-25 replaced expensive hardware safety interlocks with software controls
Real-time software
Design
Race condition caused focusing element to be incorrectly set
No indication of actual hardware settings
Error messages appeared the same regardless of how important
Error messages were difficult to understand
All errors messages could be manually overridden
oorzaak-gevolg diagram
veiligheidsanalyse naar de rapportage van foutmeldingen, de beslissingsmatrix waarmee het programma wordt uitgevoerd en de software-analyse door een consultat
\cite{stackexchange2021therac25code}






\paragraph{tesla crash report}

\begin{description}
	\item[Beschrijving]
	\item[Datum en plaats] 
	\item[Oorzaak]
	%Beschrijf wat er mis ging in termen van het vier variabelen model/requirements/specificaties
\end{description}
Door een softwarefout zijn er situaties ontstaan waarin het systeem informatie een onvoldoende informatie positie had om de juiste beslissingen te maken. Of dat de informatieverwerking niet juist was.


tesla autopilot crashes

\cite{teslaFDSCrash}
\cite{teslaCrashesCauses}
\cite{teslacrashOvervieuw}
\cite{tesladeaths}
veiigheidsrisico

\cite{evan01042019teslaautopilotIntersection}
\cite{testVehicleSafetyReport}
veiligheidsrapport mbt autopilot
\cite{lambert31062020q2safetyreport}
consumentenrapport
bluetooth veiligheidsvraagstuk
\cite{wiredBloutoothHackTesla}
veiigheidsvraagstuk vanwege touch screen
\cite{preston14012021NHTSATeslaRecall}
veiligheidsvraagstuk
\cite{cio25112020belgianTeslaHack}
veiligheidsvraagstuk
rapport over autopilot
\cite{templeton06092019HTSBReportTesla}
de invloed van de bestuurder bij tesla ongeluk
veiligheidsvraagstuk
\cite{darkReading17112020TeslaBackup}
veiligheidsvraagstuk
\cite{leyden23032020TeslaInterfaceHack}
veiigheidsvraagstuk
\cite{huddlestonjr03042019ChineseTeslaHack}
veiligheidsvraagstuk
veiligheidsvraagstuk
\cite{heilweil26022020teslaAutopilot}
rapport over ongeluk
veiligheidsvraagstuk
veiligheidsvraagstuk
\cite{blanco04102019NHTSATesla}
veiligheidsvraagstuk
ransomware aanval op tesla
tesla batterij is veiligheidsvraagstuk geworden
\cite{mitchell01072020teslabatterycooling}
ongeluk
\cite{bbc26022020AutopilotCrash}
veiligheidsvraagstuk
veiligheidsvraagstuk
\cite{stumpff04052020TeslaPersonalData}
dodelijk ongeluk
\cite{levin08062018teslaautopilotsafety}
veiligheidsvraagstuk: ransomware
veiligheidsvraagstuk: medewerker in de fout
\cite{cbrook06082021TeslaInsideDataThreft}
\cite{shilling25022021Tesla}
veiligheidsvraagstuk: hackers je systeem laten testen
verdedigen tegenover ransomware
veiligheidsrisico
prijzen omlaag
autopilot
\cite{randall05112019modelSurvey}
malware door een medewerker
dodelijk ongeluk
\cite{fottrell03092018TeslaSecurityChecks}
waarom een tesla stelen bijna onmogelijk is



veiligheidsonderzoek



softwarefout maakt diestal mogelijk


\cite{kirk26112020modelX}
fouten ontdekt in onderzoek
\cite{bbc24022021hyundaiBatteryFireFix}
tesla cloud gehacked
\cite{hawkins22102022}
\cite{gritti24062020tesladataengine}
\cite{bouchard07052019teslaDeepLearning}
\cite{Srikanth2019teslabigdata}
\cite{rangaiah25022020teslaAI}
\cite{marr08012018taslabigdataAI}
\cite{bdickson29072020teslalevelfive}
\cite{dcruz17062022tesladesignthink}
\cite{mcfarland22042021selfdrivingrisks}
\cite{hawkins18032021fedgovinvest}
\cite{berry21042021teslacrashtexas}
\cite{hull23072021regulatorsaftercrash}
\cite{wikiTeslaAutopilot}
\cite{nhtsaAutomatedVehiclesSafety}
\cite{dowling23042021autopilottricking}
\cite{wilson19042021teslacrashregulators}
\cite{seamans22062021aikillerap}
\cite{mitchell24022020AIDataTesla}
\cite{denneyjdsupraFeds}
\cite{siddiqui22102020TeslaCriticism}
\cite{ackerman01072016TeslaImperfect}
\cite{greene04092019misuseautopilot}
\cite{michralli26112019ubserautocarcrsash}
\cite{pitmann21072021wrongfullautodeath}
\cite{stackexchange102019teslacarmistake}
\cite{tasking07062017TeslaAugmentedSafety}
\cite{griemannExaminSelfDriving}
\cite{Harkey30052019SafeSystemVehicle}


tesla crash report



\cite{shepardson18062021TeslaDeaths}
\cite{hawkins30062021nhtsarequiresreporting}
\cite{hawkins10052021autopilotnotavailable}
\cite{szymkowski29062021nhtsaTeslaCrashReports}
\cite{abc1112052021AutopilotNotinTeslaCrash}
\cite{ankel18062021regulatorsinvestigateAutopilot}
\cite{sommerfield12072021NHTSAmandateresult}
\cite{saferoardsCrashesAutonomousvehicles}
\cite{stephardson18032021revieuwingtesla}
\cite{krishner30062021NHTSAreport}
\cite{gitlin11052021autopilot}
\cite{mitchell19012017investigationstop}
\cite{gordon10052021teslaprelimreport}
\cite{shaper07062018}
\cite{cochran18042021nodriverTeslaCrash}
\cite{habib28062016NHTSATeslaReport}
\cite{firstpress11052021fatalnonautopilot}
\cite{raynal20042021probeTeslaCrash}
\cite{tiungteslasoftwarecrash}
\cite{globaltimes08052021guangdongcrash}
\cite{anderson30042021secondteslacrash}
\cite{oremus21062017fatalTeslaCrash}
\cite{guardian15052021teslacrashHandsOnWheel}
\cite{Puzzanghera13092017TeslaSharesBlame}
\cite{jaillet02022017teslaAutopilotLimitations}
\cite{reuters03102019teslaAutoParkingFail}
\cite{dowling23042021}
\cite{young05112021fatalTeslaReport}
\cite{kierstein18032021teslaAutopilotCrashStationary}
\cite{janssen20062017teslacrashdetailflorida}

tesla crash publications overview

\paragraph{slmramp}

\begin{description}
	\item[Beschrijving]
	\item[Datum en plaats] 
	\item[Oorzaak]
	%Beschrijf wat er mis ging in termen van het vier variabelen model/requirements/specificaties
\end{description}
Toen de Anthony Nesty Zanderij naderde, was het daar, anders dan het weerbericht had voorspeld, mistig. Het zicht was evenwel niet zo slecht dat er niet op zicht kon worden geland. Gezagvoerder Will Rogers besloot echter via het Instrument Landing System (ILS) te landen, hoewel dit niet betrouwbaar was en hij voor zo'n landing ook geen toestemming had. De gezagvoerder brak drie landingspogingen af. Bij de vierde poging negeerde de bemanning de automatische waarschuwing (GPWS) dat het toestel te laag vloog. Het toestel raakte op 25 meter hoogte twee bomen. Het rolde om de lengteas en stortte om 04.27 uur plaatselijke tijd ondersteboven neer.

Uit onderzoek bleek dat de papieren van de bemanning niet in orde waren. 
Geconcludeerd werd dat de gezagvoerder roekeloos had gehandeld door voor een ILS-landing te kiezen terwijl hij daar geen toestemming voor had, en door onvoldoende op de vlieghoogte te hebben gelet. 
De SLM werd verweten de kwalificaties van de bemanning onvoldoende te hebben gecontroleerd.



\cite{espnSLMterugblik}
\cite{dennisRosier01052020}
\cite{hassing07062020slmramp}
\cite{amsterdamArchiefSLM}
\cite{rtvOost06062019nabestaande}
\cite{breda07062021AndroSnel}
\cite{andereTijdenSLMCrash}
%\cite{wikiSLMRamp}
database
\cite{aviationReport}
rapport
\cite{aviationSLMCrashAccidentInvestigation}
\cite{mcDonnelDouglasCommissionReportSLMCrash}
\cite{wikiSRFlight764}
\cite{nos07062019SLMTerugblik}
\cite{dagvantoenSLMCrash}
\cite{waterkantNesty07061989}
uitgebreid engels artikel
\cite{eduNandlalSRCrash}
ntsb investigtion
\cite{oldjetsSRAirways}
uitgebreid engels artikel
\cite{cloudberg02012021srflight764}
persbericht
\cite{apnews07061989srplanecrash}
Wat is de rol van de autoriteiten?
Welke andere betrokkeen? Enw at is hun verantwoordelijkheid
Hadden de negatieve gevolgen voorkomen kunnen worden?
Hoe werd er over veiligheid gedacht?



\paragraph{schipholbrand}

\begin{description}
	\item[Beschrijving]
	\item[Datum en plaats] 
	\item[Oorzaak]
	%Beschrijf wat er mis ging in termen van het vier variabelen model/requirements/specificaties
\end{description}
Om een goed verhaal op te stellen, moet vooraf aan enkele voorwaarden
worden voldaan. De eerste voorwaarde is de geschiktheid van het
afstudeerproject. Als een afstudeerproject niet tot keuzes leidt, kan
men zich afvragen of dat wel een echte afstudeeropdracht is. Een
afstudeerproject zonder onderzoeksaspecten is ook verdacht. Daarnaast
moet een afstudeerproject passen in het profiel van een opleiding om
beoordeelbaar te zijn. De andere voorwaarde voor goed een verhaal is
de registratie van werkzaamheden tijdens het a
Wat is er gebeurd?
\cite{schipholbrand27102005video}
artikel
\cite{schipholbrand27102005video}
psychologische gevolgen
rapport
\cite{onderzoeksraad2610schipholoost}
artikel met video
herdenking
impact op de persoon
herdenking
\cite{schipholbrandvideoargos}
chronologie
\cite{nunl30052023feitenoverzicht}
tijdlijn
vervolgens van ministers
beeldanalyse en reconstructie
\cite{}
herdenking
korte samenvatting
rapport
artikel
verwijzing naar het rapport vanuit de politieke oppositie
beeld vanuit de gevangenisbewaarder
nationaliteit slachtoffers schipholbrand
verblijfsvergunning voor de slachtoffers
gen schadevergoeding voor de verdachte
verdachte voor de rechter
geen schadevergoeding voor verdachte
artikel wat ging er mis bji de schipholbrand
brand veroorzaakt door een peuk
smaadschrift
bewakers worden niet vervolgd
proces schipholbrand moet over en de brandveilgheid moet worden verbeterd
de rol van het parlement in de evaluatie
\cite{parlementairemonitorschipholbrand}
onderzoeksmemo
herdenking
herdenking
invloed van de ramp op samenleving
\cite{videonpoNOVA13112008}
opmerkelijk rapport gestolen in de nasleep
\cite{rizoomes01052014schipholbrand}
publicaties
\cite{heuvelkroesschipholbrandcamerabeelden}
Wat waren de regels destijds?
Waren de autoriteiten in staat om op tijd in te grijpen of om erger te voorkomen?
Wat is er gedaan om de veiligheid van illegalen en gevangenissbewaarders te verbeteren
Wat is er gebeurd?
\cite{wikiSchipholbrand},\cite{schipholbrand27102005video}
psychologische gevolgen
rapport
\cite{onderzoeksraad2610schipholoost}
artikel met video
herdenking
impact op de persoon
herdenking
\cite{schipholbrandvideoargos}
chronologie
\cite{nunl30052023feitenoverzicht}
tijdlijn
\cite{singeluitgeverijenSchipholbrand}
vervolgens van ministers
beeldanalyse en reconstructie
\cite{eenvandaagschipholbrand}
herdenking
korte samenvatting
rapport
artikel
verwijzing naar het rapport vanuit de politieke oppositie
beeld vanuit de gevangenisbewaarder
nationaliteit slachtoffers schipholbrand
verblijfsvergunning voor de slachtoffers
gen schadevergoeding voor de verdachte
verdachte voor de rechter
geen schadevergoeding voor verdachte
artikel wat ging er mis bji de schipholbrand
brand veroorzaakt door een peuk
smaadschrift
bewakers worden niet vervolgd
proces schipholbrand moet over en de brandveilgheid moet worden verbeterd
de rol van het parlement in de evaluatie
\cite{parlementairemonitorschipholbrand}
onderzoeksmemo
herdenking
herdenking
invloed van de ramp op samenleving
\cite{videonpoNOVA13112008}
opmerkelijk rapport gestolen in de nasleep
\cite{rizoomes01052014schipholbrand}
publicaties
\cite{heuvelkroesschipholbrandcamerabeelden}
Wat waren de regels destijds?
Waren de autoriteiten in staat om op tijd in te grijpen of om erger te voorkomen?
Wat is er gedaan om de veiligheid van illegalen en gevangenissbewaarders te verbeteren

\paragraph{explosie tanjin china }

\begin{description}
	\item[Beschrijving]
	\item[Datum en plaats] 
	\item[Oorzaak]
	%Beschrijf wat er mis ging in termen van het vier variabelen model/requirements/specificaties
\end{description}

Later bleek uit een onderzoek van de Chinese autoriteiten dat de explosie overeenkwam met de ontploffing van 450 ton TNT.[6] 
De oorzaak van de explosie lag in de spontane zelfontbranding van 207 ton cellulosenitraat dat in containers was opgeslagen op het terminalterrein.[6] 
Verder lag op een tweede locatie nog eens 26 ton van dit explosieve materiaal opgeslagen.
De tweede ontploffing werd versterkt door de opslag van 800 ton kunstmest in de vorm van ammoniumnitraat in de nabijheid.[6]
De opslag van cellulosenitraat is aan strenge regels gebonden. Het moet koel en droog worden opgeslagen. De containers stonden buiten opgesteld in de brandende zon. De temperatuur liep op tot 36 °C en bereikte binnen de containers waarschijnlijk de 65 °C.[6] De verpakking van de cellulosenitraat droogde uit waardoor de ontploffing kon ontstaan. Op het terrein lagen meer gevaarlijke stoffen opgeslagen dan waarvoor vergunningen waren verstrekt.[6] Dit leidde tot een kettingreactie met grote schade tot gevolg. Door de brand en bluswater is in de directe omgeving veel milieuschade opgetreden.


https://www.hindawi.com/journals/joph/2019/1360805/ 
\cite{jiang16042019TanjinExplosion}
verhaal van brandweermannen
\cite{staff31082015tanjinblastunrevealed}
artikel
\cite{chinafile18082015tanjinexplosion}
invloed van social media
\cite{pinghuang2410201TanjinFactreport}
gemaakte fouten
\cite{portoTanjinExplosionSight}
\cite{imago17082015TanjinApartmentImages}
\cite{trager14082015Chemicalblast}
\cite{pangeramo27082015TanjinExplosion}
vergelijking met andere explosies
\cite{ap06082020ammaniumnitrate}
invloed van de ramp op de industrie
\cite{morris14082015TanjinIndustryImpact}
is er sprake van een doofpot
\cite{milesyu20082015exposingtoxicgovlines}
eigendomsverzekering
\cite{artemis30032016tanjininsurance}
\cite{aidenxiatanjinblast}
effecten op de lange termijn
\cite{danwangTanjinflexreport}
\cite{keyHighlightsTanjin}
lessons learned
\cite{hartley13082015videofootage}
\cite{odonnel01062017firetanjinblast2015}
gevolgen voor de industrie
\cite{fan15082015newyorkermistrustchina}
framing vanuit de chinese media
\cite{yanlidongchinamediaframingTanjin}
\cite{evans27092017TnjinInsurance}
niewsartikel
\cite{jasi26032019chineschemplant}
\cite{shiqingTanjinExecutiveSentence}
toegang tot de ramplplek vanuit de okale journalistiek
\cite{sophiebeach15082015}
artikel
\cite{hamzeh05082020BeirutBlast}
\cite{chemwatch18082015TanjiinExplosion}
\cite{thehindu15062019chinaExplosion}
\cite{santagotimes24032019chinablast}
oorzaken
\cite{klingecorp28042020causedTanjin}
case study
\cite{mcgarryExplosions2017}
niewsartikel
\cite{roswnfeld13082015TanjinReports}
chronologische uiteenzetting
\cite{aria12082015explosionaTanjin}
corruptie
mismanagement als oorzaak
autoriteiten publiceren onderoeksrapport
\cite{tremblay11022016chineseInvestigatorsTanjin}
fotos van de rampplek
\cite{taylor13082015TanjinExplosianAftermath}
niuwesartiekel
\cite{associatedPresss13082013}
\cite{un20082015InvestigationTanjin}
\cite{france2412082015TnjinExplosion}
\cite{npr14082015TanjinCause}
123 verantwoordelijken
\cite{bbc05022016TanjinResponsibles}
lang artiekel
\cite{CBodeen15082015TanjinExplosion}
\cite{reutersTanjinInsurance}
\cite{yu082016evaluationTanjin2015}
\cite{wiki2015TanjinExplosions}
\cite{bbc17082015whathappenedTanjin}
\cite{mortimer19082016taijinexplosioncrater}
veiigheidshandhaving
\cite{internationallabourofficeChmControlTooliit}
\cite{euTaxationCustomsICSC}
\cite{iloWHOChemSafetyCards}

%@online{landryalameddine12112020beiruthelathsystem,	ALTauthor = {author},	ALTeditor = {editor},	title = {title},	date = {date},	url = {"https://bmchealthservres.biomedcentral.com/articles/10.1186/s12913-020-05906-y"},}
%@online{ID,	ALTauthor = {author},	ALTeditor = {editor},	title = {title},	date = {date},	url = {"https://news.sky.com/story/beirut-blast-cctv-captures-moment-huge-explosion-devastated-hospital-12047452"},}
%@online{ID,	ALTauthor = {author},	ALTeditor = {editor},	title = {title},	date = {date},	url = {"https://www.unodc.org/unodc/en/frontpage/2020/September/unodc-assists-lebanon-in-reestablishing-container-shipments-in-the-aftermath-of-the-port-of-beirut-explosion.html"},}
%@online{ID,	ALTauthor = {author},	ALTeditor = {editor},	title = {title},	date = {date},	url = {"https://reliefweb.int/sites/reliefweb.int/files/resources/LEB201-Lebanon-Emergency-Response.pdf"},}
%@online{yadav07082020handlingexplosivesBeirut,	ALTauthor = {author},	ALTeditor = {editor},	title = {title},	date = {date},	url = {"https://www.downtoearth.org.in/news/governance/beirut-blast-lessons-time-for-india-to-strengthen-handling-of-explosives-chemicals-72707"},}
%@online{graham21082020rootsImpactBeirutBlast,	ALTauthor = {author},	ALTeditor = {editor},	title = {title},	date = {date},	url = {"https://www.justsecurity.org/72122/the-cost-of-resilience-the-roots-and-impacts-of-the-beirut-blast/"},}
%@online{ID,	ALTauthor = {author},	ALTeditor = {editor},	title = {title},	date = {date},	url = {"https://www.fire-magazine.com/the-port-of-beirut-explosion-a-timely-reminder"},}
%@online{neusaeter07082020beirutexplosioneval,	ALTauthor = {author},	ALTeditor = {editor},	title = {title},	date = {date},	url = {"https://www.ctvnews.ca/sci-tech/mapping-the-beirut-explosion-what-the-impact-would-look-like-in-canadian-cities-1.5053932"},}




\paragraph{ethiopian airlines}

\begin{description}
	\item[Beschrijving]
	\item[Datum en plaats] 
	\item[Oorzaak]
	%Beschrijf wat er mis ging in termen van het vier variabelen model/requirements/specificaties
\end{description}
Ethiopian Airlines Flight 302
Door problemen met de flight control
One minute into the flight, the first officer, acting on the instructions of the captain, reported a "flight control" problem to the control tower.
Two minutes into the flight, the plane's MCAS system activated, pitching the plane into a dive toward the ground. The pilots struggled to control it and managed to prevent the nose from diving further, but the plane continued to lose altitude.
The MCAS then activated again, dropping the nose even further down. The pilots then flipped a pair of switches to disable the electrical trim tab system, which also disabled the MCAS software. However, in shutting off the electrical trim system, they also shut off their ability to trim the stabilizer into a neutral position with the electrical switch located on their yokes. The only other possible way to move the stabilizer would be by cranking the wheel by hand, but because the stabilizer was located opposite to the elevator, strong aerodynamic forces were pushing on it.
As the pilots had inadvertently left the engines on full takeoff power, which caused the plane to accelerate at high speed, there was further pressure on the stabilizer. The pilots' attempts to manually crank the stabilizer back into position failed.
Three minutes into the flight, with the aircraft continuing to lose altitude and accelerating beyond its safety limits, the captain instructed the first officer to request permission from air traffic control to return to the airport. Permission was granted, and the air traffic controllers diverted other approaching flights. Following instructions from air traffic control, they turned the aircraft to the east, and it rolled to the right. The right wing came to point down as the turn steepened.
At 8:43, having struggled to keep the plane's nose from diving further by manually pulling the yoke, the captain asked the first officer to help him, and turned the electrical trim tab system back on in the hope that it would allow him to put the stabilizer back into neutral trim. However, in turning the trim system back on, he also reactivated the MCAS system, which pushed the nose further down. The captain and first officer attempted to raise the nose by manually pulling their yokes, but the aircraft continued to plunge toward the ground.


\cite{caliskan09112013747boeingkalman}
\cite{gates18112020boeingcrisis}
\cite{boeing737maxsoftwareprobles}
\cite{avetisov19032019boeingmalwarestate}
\cite{thompson23112020nationalsecurityboeing}
\cite{gates21032019FAAControlSystem}
\cite{faa18112020boeingreview}
\cite{wiki737maxgroundings}
\cite{campbell02052019boengcrashhumanerrors}
\cite{hawkins22032019737maxairplanes}
\cite{thomas30082020737safest}
\cite{boeing737maxdisplay}
\cite{fehrm24112020737changes}
\cite{travis18042019737maxsoftwaredevop}
\cite{barnett05052019737maxcrisis}
\cite{easa27012021737maxsafereturn}
\cite{touitou11032019737tragedies}
\cite{hemmerdinger02022021737maxdeliveries}
\cite{bielby27022021faaimprovesafety}
\cite{boyle18112020737maxupgrade}
\cite{bergstraburgess122019737maxMcasAlgorithm}
\cite{737mcas}
\cite{newburger17052019boeingcrisis}
\cite{arstechnica22012020737problems}
\cite{german190620217372yaftergrounded}
\cite{beningo02052019boeinglessons}
\cite{duran05042019boeingspof}
\cite{makichuck24012021737fearflying}
\cite{caa737modifications}
\cite{oestergaard14122020boeingdeliveries}
\cite{reenberg787flaws}
\cite{fitch16092020737backlogrisks}
\cite{willis27082020737maxfailures}
\cite{ostrower11062020more737changes}
\cite{hruska13122019faaknown737crashrate}
\cite{bloomberg26092019failedpred}
\cite{whiteman09072020boengcancelstock}
\cite{leopold09192019boeingreliability}
\cite{koenig11122019737crashesnofix}
\cite{dohertylindeman15032019737problems}
\cite{stodder02102019corruptoversight}
\cite{afacwaLostSafeguards}
\cite{swayne18032019profitssafety}
\cite{freed26022021liftaustraliaban}
\cite{reed15032019softwareattention}
\cite{news17032019softwareexplains}
\cite{legget21122020eu737maxsafe}
\cite{marketscreener0103221737chinarecertification}
\cite{euractiv22022021737firegrounds}
\cite{benny18022019737returnUAE}
\cite{biersmichel22022021777grounds}
\cite{reuters23022021777metalfatigue}

\paragraph{ethiek}


Ethiek 



persuasive technology 
https://www.humanetech.com/youth/persuasive-technology 
\cite{humanTechpersuasiveTech}
https://www.minddistrict.com/blog/persuasive-technology-new-insights-in-behavioural-change 
https://www.sciencedirect.com/book/9781558606432/persuasive-technology 
https://spectrum.ieee.org/how-persuasive-technology-can-change-your-habits 
\cite{rezenfeld01012018persuasiveTecgHabits}
https://www.frontiersin.org/articles/10.3389/frai.2020.00007/full 
\cite{aldenaini28042020persuasiveTechTrends}
https://psmag.com/environment/captology-fogg-invisible-manipulative-power-persuasive-technology-81301 
\cite{larson14062017persuasivetechmanipulates}
https://www.makeuseof.com/what-is-persuasive-technology/ 
\cite{tanzem22012022persuasivetechchanginglives}
https://lib.ugent.be/catalog/rug01:001235489 
https://cyberpsychology.eu/article/view/12270 
\cite{tikkakuddonenpersuasiveTechnology}


\paragraph{ Research case: De digitale aanval op de Oekrainese krachtcentrale}

\begin{description}
	\item[Beschrijving]
	\item[Datum en plaats] 
	\item[Oorzaak]
	%Beschrijf wat er mis ging in termen van het vier variabelen model/requirements/specificaties
\end{description}

op 23,december 2015  vind er een cyber aanval plaats op het elektriciteitsnet van de Oekraine. Dit was de eerste bekende aanval op een elektrisch contole  system.  Dit verslag geeft inzage in een analyse van de Ukraine cyber aanval,
inclusief hoe de actoren zich zelf toegang gavan tot het controle systeem, welke methoden de acoren hebben gebruikt voor reconnaissance en vastleggen van het systeem, een gedetailleerde omshrijving van de aanval op 15 December 2015, en de methoden die gebruikt zijn door de aanvallers om hun sporen uit te wissen en daarmee het het stoppen van schade toebrengen  nog moeilker maken. Daarnaast wordter  een gedetailleerde omschrijving gevevenv an de beveiliging van de SCADA ccontrol systemen gebaeerd op bst practices, inclusief het control network ontwerp, technieken voor whtelisting, monitoring en loggen, en  opleiding van personeel.
\cite{Whitehead2017ukrainepoweroutage}
\cite{noauthor_2022-nm}
\cite{zetter2016GridHack}
\cite{owens21032017ukrainemitigationstrategies}
\cite{cerulus2019FrontlineRussiaAttack}
\cite{grammatikis2019AttackIEC6087505104}
\cite{hidajat2016ScadaSimulator}
\cite{uscert20072021crashmalware}
\cite{zetter12062017malwareanalysis}
\cite{icsRussianHackingCyberWeapon}
\cite{usgovC2M2}
Dit verslag geeft inzage in een analyse van de Ukraine cyber aanval,
inclusief hoe de actoren zich zelf toegang gavan tot het controle systeem, welke methoden de acoren hebben gebruikt voor reconnaissance en vastleggen van het systeem, een gedetailleerde omshrijving van de aanval op 15 December 2015, en de methoden die gebruikt zijn door de aanvallers om hun sporen uit te wissen en daarmee het het stoppen van schade toebrengen  nog moeilker maken. Daarnaast wordter  een gedetailleerde omschrijving gevevenv an de beveiliging van de SCADA ccontrol systemen gebaeerd op bst practices, inclusief het control network ontwerp, technieken voor whtelisting, monitoring en loggen, en  opleiding van personeel.
\cite{Whitehead2017ukrainepoweroutage},\cite{zetter2016GridHack},\cite{boozallen2016lightwentout},\cite{finklejan2016UsBlamesRussianSandworm},\cite{desarnaud2017cyberattacks},\cite{caseli04112016intrusiondetectioncontrolsystem},\cite{rochascadatesting},\cite{hidajat2016ScadaSimulator},\cite{zetter2017moreDangerousMalware}.
Oop 23,december 2015  vind er een cyber aanval plaats op het elektriciteitsnet van de Oekraine. Dit was de eerste bekende aanval op een elektrisch controle  system met corrupte firmware. Daarnaas wordt er een telecom-based denial of service attack met  geautomatieerde systemen om het telefoonverkeer uit te schakelen.
\cite{Whitehead2017ukrainepoweroutage}
Uit onderzoek\cite{zetter2016GridHack} naar de aanval,  uitgevoerd door Oekraiene sen Amerikaanse militairenblijkt  bleek onder meer dat de power grids in sommige gevallen beter waren beveiligd dan de Amerikaanse. Desondanks was de viligheid niet optimaal door onder andere de  hetgegeven dat werknemers op afstand konden inloggen en geen gebruik van 2-stapsverificatie.
\subparagraph{Literaire analyse}
\subparagraph{Motief}
Oekraine wijst naar de russen \cite{zetter2016GridHack}, 
\cite{greenberg2017Cyberwartestlab},
\cite{boozallen2016lightwentout},
\cite{finkle08012016russiansandwormhackers},
\cite{zinets15022017ukrainechargesrussia},
\cite{mcelfresh2016cyberattackhowandwhy},
\cite{parkwalstorm11102017russiagridattack}.
\subparagraph{Situatie Oekraiene}
\cite{drago2017CrashOverride},
\cite{slowik2019ReassasUkraine2016Attack}.
\subparagraph{Situatie algemeen}
\cite{cerulus2019FrontlineRussiaAttack},
\cite{desarnaud2017cyberattacks},
\cite{dragos2019TargetedTransStation}.
\subparagraph{Factoren}
\cite{shehod2016gridadvantageus}
\subparagraph{Oorzaak}
\cite{rocha2017cybersecyrityanalysisScada},
\cite{2017crashoverridenostuxnet},
\cite{vijayan2017firstmalwareCausedOutage},
\cite{slowik2019ReassasUkraine2016Attack}.
\subparagraph{Gebruikte materialen}
\cite{2015ukrainegridattack},
\cite{industroyershortfact}
\subparagraph{Uitvoering van de aanval}
\cite{Whitehead2017ukrainepoweroutage},
\cite{boozallen2016lightwentout}.
\subparagraph{Oplossingen}
~\cite{Whitehead2017ukrainepoweroutage}
\subparagraph{Aanbevelingen}
\subparagraph{Resultaten}
\subparagraph{De aanval}
1. An initial email spear phishing attack lures recipients
into opening an attached Microsoft® document with a
macro that installs Black Energy 3 (BE3) onto
corporate workstations.
2. BE3 and other tools perform reconnaissance and
enumeration of the network and provide an initial
backdoor for the hackers into the corporate network.
3. As a result of network reconnaissance, the malicious
actors discover and access the oblenergos’ Microsoft
Active Directory® servers that contain corporate user
accounts and credentials.
4. With the harvested credentials, the malicious actors use
an encrypted tunnel from an external network to get
inside the oblenergo network, establishing a presence
on the oblenergo control system networks.
5. Malicious actors discover and access the control center
supervisory control and data acquisition (SCADA)
human-machine interface (HMI) servers and
substations. While a router separates corporate and
SCADA networks, the firewall rules are improperly
configured.
6. On December 23, 2015, at 3:30 p.m., the malicious
actors begin their power outage attacks by entering
operations and SCADA networks through backdoors on
the compromised SCADA workstations. The malicious
actors take control away from HMI operators and then
open breakers.
7. The malicious actors perform several other actions with
the intent of complicating the responses of control
operators and increasing the effort required to return the
system to normal operating conditions. These actions
include:
a. Launching a coordinated Telephony Denial of
Service (TDoS) attack that floods call centers to
prevent legitimate calls from getting through.
b. Disabling the UPSs for the control centers.
c. Corrupting the firmware on a remote terminal unit
(RTU) HMI module and serial-to-Ethernet port
servers.
8. Malicious actors execute KillDisk malware in an
attempt to wipe out the control center HMIs and pivotpoint workstations.
\cite{Whitehead2017ukrainepoweroutage}
\cite{boozallen2016lightwentout}
\subparagraph{spearfishing}
\subparagraph{blackenergy}
\subparagraph{remote access capabilities}
\subparagraph{serial-to-ethernet communication devices}
\subparagraph{telephony denial of service attacks}
\subparagraph{oplossingen}
Identificeer alle risicos en schrijf een plan foor het managen van de risico's.
Implementeer  effecteve controle  om het riico te managen.
Creeer een diepgaand model dat ervoor zor dat er efectieve en efficiente security controls worden uitgevoerd.
Aangaande de gebeurtenissen in de oekraiene kunnen de volgende security controls worden opgenomen in het securitymodel: Initial access to enterprise network, pivot in interprise network, elevate priviliges, maintainance access, gain access to control system, attack, attack complication, destroy hard drives.
\cite{Whitehead2017ukrainepoweroutage}
\subparagraph{Discussie}
\subparagraph{Verder lezen}
\cite{shahzad2014ScadaProtocolsPollingScenario},
\cite{grammatikis2019AttackIEC6087505104},
\cite{2017win32industroyer},
\cite{yadav2020reviewScadaArchitecture},
\cite{arrizabalaga2020surveyiiotProtocols},\cite{fauri2017EncryptionICS},\cite{resch31102019IEC62351secureCommunication},\cite{levalle2020FuzzingICSProtocols},\cite{blackhatusa2017},\cite{blackhatusa2017},\cite{abb30062017crashoverridenotification},\cite{spinner2018crashoverrideiot},\cite{njccicthreat08102017crashovverrideprofile},\cite{slowikvb2018crashoverride},\cite{crashoverridenetwork},\cite{wikiindustroyer},\cite{icsSecurityRussianHacking},\cite{holappa2017threattoElectricityNetworks}.




\paragraph{Mali}

\begin{description}
	\item[Beschrijving]
	\item[Datum en plaats] 
	\item[Oorzaak]
	%Beschrijf wat er mis ging in termen van het vier variabelen model/requirements/specificaties
\end{description}
Een granaat explodeerd in een mortier
De medische zorg na het ongeval was neit voldoende


De algemeen militair verpleegkundige gaf aan het slachtoffer nar het vn-hospitaal in kidal te brengen
De chaauffeur van de bushmaster kende de locatie niet  en bracht het slachtoffer naar een door frane militairren bemand hospitaal mmet minder mediswche faciliteiten
Hierna alsnog overgebracht naar het vn-hospitaal.
Dit verlieop  neit door nederlandse maatstaven.
pas toen een nederlandse arts arrivveerde werd door de Tongolese artsen een buikoperatie uitgevoerd.
Dit gebrurde zonder adequate anesthesie.
Na de operatie werde de gewonde militair overgelogen naar nederland. En later naar nederland.


granaat stond niet op scherp en in afgegaan in veilige stand
Granaat werd opgeslagen in neit gekoelde containers waardoor deze aan te hoge temeperaturen zijn blootgesteld.
Door de comvinatie van vocht en warmte in de granaat zeer gevoelige explosieve stoffen werden gevormd.
Tijdens de oefening was de fatale granaat in de zon.
Het afsluitplaatje in de granaat bleek niet in staat om doorslag in veilige stand te voorkomen waarna de granaat explodeerde.
De moritren zijn aangeschaft bij de amerikanen. gredurende de aanschafperiode zijn procedures en controles op kwaliteit en veiligheid deels nagelaten.
Dit veiligheidsgarantie werd vermeld in het koopcontract.
Conclusie
Koopcontract werd niet goed doorgelezen
Geen controle op kwaliteit en veiligheid
Geen controle op kwaliteit en veiligheid
Zwakke plekken in het ontwerp
Geen controle op kwaliteit en veiligheid
opslag en gebruik in ongunstige condities

De aanwezige medische voorzieningen waren nite volgends de nederlandse militaire richtlijnen
Het ontbreek aan medische toetsing vanuit de defensie organisatie
twijfels die werden geuit binnen de defensieorganisae vonden geen wrrklank
Ok het ongeval tijdens de mortieroefening was voor defensie geen aanleuiding om de medische voorzienignen te evalueren.
De inrichting van veilige medische zorg voor nederlandse militairen in kidal is ondergeschikt gemaakt aan de voortgang van de missie.


\cite{ovvMortierOngevalMaliVideo} 
\cite{bnnvara13062018malirapport}
\cite{eucal11012021malimissieverlengd}
\cite{nos21052014zorgenmalimissie}
\cite{meijnders}
\cite{bnrwebredactie}
\cite{keultjes01062016malimissiecoalitie}
\cite{veenhof18012019}

\cite{isitman06012016militair}
\cite{nporadio11072016filmdemissie}
\cite{parlementairmonitor15122013mortierongeluk}
sollicitatie
de bureaucratie
aankomst
interview van de burgerbevolking
steun van de bevolking minuut 15:00
de organisatie minuut 23:00
De militaire briefing minuut 34:00
prioriteit minuut 39:00
briefing minuut 40:00
de communicatie met ministerie over inlichten minuut 44:00
\cite{DemissieFilm}
\paragraph{Analyse}
\paragraph{Conclusie}



\paragraph{Deelonderzoeken}

\paragraph{AlgemeenDeelonerzoek naar veiligheidsrisico's voor sluizen}
\paragraph{Wet en regelgeving voor sluizen}

\paragraph{Ondeerzoeksresultaten naar sluisbeveiliging}



Verouderde computersystemen zijn door de jaren heen gekoppeld aan netwerken, zodat ze op afstand te besturen zijn. Dit zorgt ervoor dat systemen kwetsbaar zijn voor aanvallen van buitenaf. De beveiliging is in de loop der jaren niet voldoende ontwikkeld om de infrastructuur goed te beveiligen.

Volgens het onderzoek is er de afgelopen jaren wel het nodige geïnvesteerd om de beveiliging op te schroeven, maar deze maatregelen zijn nog onvoldoende doorgevoerd.
https://www.nu.nl/internet/5814282/rekenkamer-waterwerken-niet-goed-beveiligd-tegen-cyberaanvallen.html
\cite{hdsr30092022lichtprojectieswaterliniesluizen}
rapport Digitale dijkverzwaring: cybersecurity en vitale waterwerken 
Crisisdocumentatie is verouderd en er worden geen volwaardige pentesten uitgevoerd. Uit het onderzoek blijkt dat nog niet alle vitale waterwerken rechtstreeks zijn aangesloten op het Security Operations Center (SOC) van Rijkswaterstaat. Hierdoor bestaat het risico dat RWS een cyberaanval niet of te laat detecteert. De minister van Infrastructuur en Waterstaat moet nog stappen zetten om aan de eigen doelstellingen voor cybersecurity te voldoen
De Algemene Rekenkamer beveelt de minister van Infrastructuur en Waterstaat ook aan om het actuele dreigingsniveau te onderzoeken en te besluiten of extra mensen en middelen nodig zijn. Ook is het voor een snelle en adequate reactie op een crisissituatie van essentieel belang dat informatie up-to-date is. Pentesten zouden integraal onderdeel uit moeten maken van de cybersecuritymaatregelen bij vitale waterwerken. Verder zou moeten worden bezien of medewerkers van het SOC beter moeten worden gescreend.

\cite{kramerZeeland}
Sluis Eefde kreeg niet alleen de onderhoudsbeurt, maar werd tevens uitgebreid met een tweede sluiskolk. Zo wil Rijkswaterstaat wachttijden voor de scheepvaart voorko

\cite{gww29032021kantelendesluisdeur}
Om de lokale bemanning, die de oren en ogen waren van de sluizen, te vervangen waren camera’s, communicatielijnen en software nodig. Hoge kwaliteit videobeelden, met echte kleuren en zonder enige vertraging zijn belangrijk voor de operators en zij moeten hierop kunnen vertrouwen. Er zijn verschillende testen gedaan met diverse camera’s en cameraposities om kleurechtheid te kunnen bieden onder alle omstandigheden. Het resultaat was een perfecte kleur op alle 70+ camera’s op iedere locatie.

Vertraging van videobeelden was een cruciale factor in dit project. Het is uiterst belangrijk dat de operator op zijn beeld ziet wat er daadwerkelijk op locatie gebeurt, zonder enige vertraging. Om te laten zien of er eventuele vertraging is, is er een speciale functie gecreëerd. Deze functie laat een rood kruis zien op het scherm wanneer de vertraging meer is dan 500 miliseconden. Zo ziet de operator direct of het beeld wat hij ziet actueel is. 

Een andere functie die voor dit project is gecreëerd, is bij de videobeelden aan te geven van welke kant van de sluis het camerabeeld is. Voor de operators is het belangrijk dat ze weten vanaf welke kant het vaartuig komt en waar deze naartoe vaart. Een simpele oplossing was om een blauw kader te maken om het videobeeld van de ene kant van de sluis en geen kader om het videobeeld van de andere kant. 


\cite{thkwaterwerken}
Het crisismodel kan beter, is de derde deelconclusie van de Algemene Rekenkamer. Er is geen specifiek scenario voor een crisis die wordt veroorzaakt door een cyberaanval. Ook ontbreekt inzicht in de effecten van een cybercrisis op andere sectoren, de zogeheten cascade-effecten. Tevens is de crisisdocumentatie op onderdelen verouderd.

\cite{rekenkamercybersecWater}
Ook maakt cyberveiligheid nog geen volwaardig onderdeel uit van reguliere inspecties.’ De Rekenkamer hamert erop dat alle vitale waterinfrastructuur zo snel mogelijk op het SOC wordt aangesloten. Ook zouden werknemers van Rijkswaterstaat die belangrijke waterkeringen bedienen beter gescreend moeten worden op hun antecedenten. Sollicitanten hoeven nu slechts een Verklaring Omtrent Gedrag te overleggen, maar dat is een heel lichte toets.

\cite{hackerWaterwerk}
deltawerken

\cite{kramerZeeland}
Volgens Rijkswaterstaat is het kostbaar en technisch uitdagend om klassieke automatiseringssystemen te moderniseren en wordt er daarom vooral ingezet op detectie van aanvallen en een adequate reactie daarop.
Uit het onderzoek blijkt dat Rijkswaterstaat de afgelopen jaren zelf van alle tunnels, bruggen, sluizen et cetera heeft vastgesteld welke cyberveiligheidsmaatregelen moeten worden genomen. Een groot deel van die maatregelen (ongeveer 60\%) was begin 2018 ook al uitgevoerd, maar Rijkswaterstaat ziet onvoldoende toe op de uitvoering van het resterend deel en heeft geen actueel overzicht van de overgebleven maatregelen.
De minister heeft een aantal waterwerken die Rijkswaterstaat beheert als vitaal aangewezen. . Uit het onderzoek blijkt dat nog niet alle vitale waterwerken rechtstreeks zijn aangesloten op het Security Operations Center (SOC) van Rijkswaterstaat. De ambitie om eind 2017 bij alle vitale waterwerken cyberaanvallen direct te kunnen detecteren was in het najaar van 2018 daarmee nog niet gerealiseerd. Hierdoor bestaat het risico dat RWS een cyberaanval niet of te laat detecteert.

\cite{cybersecWaterwerk}
Over de cyberbeveiliging van gemeenten en waterschappen wordt al langer geklaagd. Zo meldde EenVandaag al in 2012 dat rioolgemalen en sluizen gemakkelijk van afstand te bedienen waren, onder meer door bijzonder slechte wachtwoorden.

\cite{cybersecWaterschappen}
Rittal doet onderzoek naarop afstand besdienbare sluizen

\cite{cybersecZuidHolland}
Beveiligde VPN
M2M Services levert aan inmiddels 220 gemeenten en waterschappen beveiligde connectiviteitsoplossingen voor het beheer van pompen, riolen en gemalen. Om risico’s op beveiligingsincidenten te voorkomen maken wij gebruik van een VPN oplossing, waarbij de verbinding optimaal beveiligd is middels encryptie en authenticatie.

\cite{waterwerkNED}
Veiligheid op het water én op het land
Gebruik van lampbewaking 

\cite{veiligheidwaterland} 


\subsubsection{explosie in libabon, beirut }
\begin{description}
	\item[Beschrijving]
	\item[Datum en plaats] 
	\item[Oorzaak]
	%Beschrijf wat er mis ging in termen van het vier variabelen model/requirements/specificaties
\end{description}
Op 23 september 2013 voer het vrachtschip de Rhosus onder Moldavische vlag[7] van Batoemi in Georgië naar Beira in Mozambique met 2.750 ton ammoniumnitraat

Gezien het ernstige gevaar van het bewaren van deze goederen in de hangar onder ongeschikte klimatologische omstandigheden, herhalen we ons verzoek aan de marine-instantie om deze goederen onmiddellijk weer te exporteren om de veiligheid van de haven en de mensen die er werken te verzekeren, of om akkoord te gaan om ze te verkopen.
Voorafgaand aan de explosie was er een brand in een opslagplaats. 


\cite{hrw03082021investigateBeirutBlast}

\cite{souaibyElHussein112020Beirutstory}

\cite{ifrc2020chemicalexplosionBeirutPort}




\subsubsection{stint ongeluk}

\begin{description}
	\item[Beschrijving]
	\item[Datum en plaats] 
	\item[Oorzaak]
	%Beschrijf wat er mis ging in termen van het vier variabelen model/requirements/specificaties
\end{description}
Vier kinderen, een bestuurder kwamen om en een vijfde persoon , een kind raakte zwaargewond. Uit odnerzoek van bleek :
Foute torsieveer voor de gashendel werd geleverd
Geen van de drie onderzochte voertuigen haalden de wettelijk vereiste remvertraging
De automatische parkeerrem kan leiden tot gevaarlijke situaties wanneer deze ongewenst geactiveerd wordt tijdens het rijden. 
Het losraken van de nuldraad naar de gashendel leidt volgens TNO tot ongewenst versnellen van het voertuig en een oncontroleerbare situatie voor de bestuurder.
Voor alle drie onderzochte voertuigen geldt dat het ontbreken van een zitplaats leidt tot veiligheidsrisico’s voor remmen en sturen door de grotere kans dat de bestuurder van het voertuig valt. Als de bestuurder van een Stint valt, leidt dit in alle rijsituaties tot een onbeheersbare situatie


\cite{TNOStint}




\subsubsection{vuurwerkramp in enschede }

\cite{boogers092002RampenRegelsRichtlijnen}

Wat waren de afspraken omtrent vuurwerkopslag?
Waarom werden de voorschriften neit nageleefd?




\subsubsection{ecourt in nederlandse rechtspraak}

\begin{description}
	\item[Beschrijving]
	\item[Datum en plaats] 
	\item[Oorzaak]
	%Beschrijf wat er mis ging in termen van het vier variabelen model/requirements/specificaties
\end{description}
niet odnerzocht
https://www.njb.nl/blogs/a-court-with-no-face-and-no-place/ 
\cite{sprongken19032018CourtProcedureDigital}
http://www.e-court.nl/wp-content/uploads/2018/03/Procesreglement-e-Court-2017_20180201.pdf
\cite{PROCESREGLEMENTEcourt}




\paragraph{molukse treinkaping }

\begin{description}
	\item[Beschrijving]
	\item[Datum en plaats] 
	\item[Oorzaak]
	%Beschrijf wat er mis ging in termen van het vier variabelen model/requirements/specificaties
\end{description}
https://www.youtube.com/watch?v=h99Fe9XzzHI 
\cite{molukseTreinkaping}


\subsubsection{Ramp schietpartij militair ossendrecht }

\begin{description}
	\item[Beschrijving]
	\item[Datum en plaats] 
	\item[Oorzaak]
	%Beschrijf wat er mis ging in termen van het vier variabelen model/requirements/specificaties
\end{description}
Een militaire overleid op een schietbaan in ossendracht door onvoldoende begeleiding van cursisten, geen toezicht op de lokatie. E\r was een instructuur in opleiding die niet volledig was mmeegenomen in het poroces en ook was er geen baancommandant aanwezig. Geen van de aanwezig instructeurts had de juiste papieren om de cursisten te begeleiden. De aanwezig instruceur had geen zich op de instructeur in opleiding, evenmin de andere militairen. In de instructiehandleiding ontbreken richtlijnen voor bijzondere schietbanen. Ook was er geen keuring. Door personelstekort is er geen andacht besteed aan documentastie(een slyllabus) hoe en met welke risico’s oefeningnen moeten worden ingericht. Ok werd er vooraf geen veiliheidsanaklyse gedaan. Het gebrek aan lesmateriaal en deskundigen is gemeld binnen de defensieorganisatie maar dit heeft niet geleid tot enige verandering in de situatie.
Op een afgekeurde scheitbaan
Tezicht door een instructeur in opleiding die zelf geen persoonlijke begeleiding heeft gehad tijdens de uitvoering
Belangrijk is dat defensie haar taken kan uitvoeren met personeel dat is getraind in situaties die de risicos van de werkomgeving aan de cursisten kunnen laten zien.
Conclusie
Zonder gekwalificeerde instructuers.
Zonder toezicht
Zonder lesmateriaal
Zonder adequate veiligheidsanalyse
https://www.youtube.com/watch?v=6jmkDClGDHo 
\cite{oVVSchietongevalOssendrecht}
\cite{nos22032016ossendrecht}
\cite{ovv04042016lessenongevalossendrecht}
\cite{quekelboere10052017doodossendrecht}


Wat is de rol van defensie?
Wat is er gedaan om de veligheid van de medewerkers te waarborgen?
Waarom zijn deze regels niet nageleefd?
Wat zijn de gevolgen?
Zijn de acties die naderhand zijn ondernomen wel redelijk naar de slachtoffers, het nationale veiligheisbeeld en de medewerkers?
