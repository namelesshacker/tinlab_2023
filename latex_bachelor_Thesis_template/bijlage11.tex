\hoofdstuk{World a machine samenvating}

\paragraph{World and machine samenvatting}
Waarom zijn wij engineers? Omdat we bruikbare apparaten willen laten functioneren in de wereld waarin we leven. Dat doen we door de machine te beschrijven en deze beschrijving van instructies bieden we aan onze computer opdat deze als de attribuut en gedragingen uitleest zoals wij die hebben omschreven. Dit alles op basis van theoretische funderingen en praktisch inzicht. 

Het doel van een machine is om te worden geinstalleerd en te worden gebruikt. De eisen die we stellen zitten in de omgeving en in de wereld en de machine is slechts de oplossing die we bedenken om aan een eis te voldoen. 

De relatie machine-wereld world gecategoriseerd in: 
Het modelleer aspect: waar een machine de wereld simuleert 

Het interface aspect: waar er fysieke interactie is tussen de machine en de wereld 

Het engineering aspect: waar de machine zich gedraagt als een controlemotor gebruikmakend van de gedragingen van de omgeving in de wereld 

Het probleem aspect: waar de omgeving in de wereld en de omvang van het probleem invloed heeft op de machine en de oplossing 

Het modelleer  of simulatie aspect over een deel van de wereld. Er zijn data,object en proces modellen. Het doel van een model is toegang te geven tot informatie over die wereld. Door het opvangen van statische weergaven en gebeurtenissen kunnen wij deze gebruiken van opgeslagen informatie die we kunnen hergebruiken. Een model kan bruikbare informatie bevatten omdat zowel het model als de wereld warin het model zich bevind gemeenschappelijke omschrijvingen hebben die waar zijn voor zwel het model als voor de wereld. Daarbij moet gesteld worden dat de interpretatie van een model verschilt met een interpretatie van de wereld. 

Omdat zowel de wereld als de machine fysieke realiteiten zijn an niet slechts abstracties, zijn de gemeenschappelijke beschrijvingen slechts een deel van de werkelijheid van beide objecten. For elk object zijn er meerdere beschrijvingen. Toch maken niet alle omschrijvingen deel uit van het getoonde reportoire. Zoals niet alle eigenschappen van een boek; meer dan een auteur, pseudoniemen, een onderdeel van een reeks, een gerevisiteerde versie, worden gereflecteerd in een database.  

Het interface aspect. Een machine kan een probleem in de wereld oplossen als de wereld en de machine phenomena kunnen uitwisselen. Maar de participatie is niet symmetrisch: een status kan als phenomena worden uitgewisseld maar slechts een partij kan er invloed op uitoefenen maar beiden kunnen dezelfde status signaleren. 

Het engineering aspect gaat over requirements, specificaties, en programma’s. Requirements hebben betrekking op phenomena in de wereld. Een programma heeft alleen betrekking tot de machinale phenomena. Het doel van programma’s is om eigenschappen en gedragingen te omschrijven van de machine ten behoeve van de gebruiker. Tussen de requirements en de programma’s zitten de specificaties. Omdat programma’s dan wel beschrijvingen zijn van een gewenste machine, maar dat moeten beschrijvingen zijn van de  machines  die de computers kunnen uitvoeren zodanig dat de computer deze beschrijvingen ook zo kan interpreteren. De engineer moet  de eigenschappen van de wereld kennen en begrijpen en deze eigenschappen manipuleren en laten werken met als doel het dienen van het systeem. 

Het probleem aspect. Het onderscheid tussen specificatie en implementatie. Het probleem zit in de relatie van de machine en de wereld. De machine brengt de oplossing maar het probleem zit in de wereld. Een vertoog over een probleem moet dus gaan over de wereld en over de opvatting die de gebruiker heeft in de wereld. Omdat de wereld veelzijdig is moeten we ervan uit gaan dat er verschillende soorten problemen zijn. Een realistisch probleem wordt dus niet opgelost met een simpele hiërarchische structurele aanpak en een homogene decompositie maar met een paralleele structurele oplossing waar beide kanten van het probleem worden opgelost. 



Ontkenningen 

We hebben als engineers de taak om een machine te bouwen aan de hand van de specificaties opgeleverd door de opdrachtgever. Een engineer heeft niet als taak de fitheid voor een doeleind te onderzoeken, maar wel de haalbaarheid naar een doeleind aan de hand van kennis, tijd, resources, budget en ontwikkelmethodiek. Daaruit komt naar voren dat een engineer zich richt op: elicitation (schetsen van een requirement), description (omschrijving) en analyse van de requirements waaraan het systeem moet voldoen. Vertaalt naar de volgende vragen: Wat is precies de klantwens?  Wat is de precieze omschrijving van het probleem? Voor welke doelen wordt het systeem gebouwd? Welke functies moet het systeem hebben? 

Denial by hacking: obsessief bezig zijn met een systeem omdat het de gebruiker veel macht geeft. Een uitgebreidheid van een systeem zorgt er soms voor dat mensen niet meer geprikkeld zijn na te denken over probleemstellingen, domein beschrijvingen en analyse. 

Denial by a abstraction. Wiskundige benaderingen van werkelijke problemen is  een belangrijke intellectuele strategie om problemen te formuleren. Een software ontwikkelaar moet een probleem kunnen omschrijven in zo min mogelijk woorden, maar de complexiteit ligt in de oplossing. 

Denial by vagueness. De vaagheid van een omschrijving is terug te vinden in: 

Von Neumann’s principe ,Principe van reductionisme ,Shanley principe en het Montaingnes’s principe.
Het Von Neumand principe uitgelegd
Voor een vocabulair  moet een grondslag zijn ontwikkeld waarmee gesproken kan worden over de wereld en de machine. Belangrijke phenomenen moeten geindtifieerd worden, door middel van een grondregel  of ‘herkenningsregel’ moet een fenomeen worden herkend, en vervolgens het fenomeen een formele term geven die gebruikt wordt als duiding van een bepaalde omschrijving. Dan moet voor de formele term een symbool gevonden worden. Samen vormen de grondregel en het symbool een designatie. 

Principe van reductionisme 

Simpelweg het openbreken van termen met een weerlegbare definitie totdat alle begrippen die worden gebruikt om iets te duiden  niet meer te herconstrueren zijn in hun definitie. 

Shanley principe 

Er bestaan volgens dit principe geen scherpe verdelingen in de wereld zoals wetenschappers soms denken. Een strenge opvatting over de wereld waarin een individu geclassificeerd kan worden als een onsamenhangend geheel. Maar dat is slechts een opname van een beeld. De werkelijkheid staat soms toe dat een elementair individueel object in verschillende classificaties verschillende getypeerd kan worden in een andere setting of view. 

Montaignes principe 

De incative mood; gaat over wat we beweren waar te zijn. 

De optitative mood; gaat over wat we willen dat waar is 