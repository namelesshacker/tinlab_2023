\hoofdstuk{}






main.py [-h] [--version] [-d] [-c FILE] [-g FILE] [-t FILE] [-o FILE]





32,38,40,44,61,75,78,79,81,82,85,86,201,202,213,306,308,325\




Timeliness
https://www.eecs.yorku.ca/course_archive/2014-15/F/1090/slides/04_TheoremCalculation.pdf
https://math.stackexchange.com/questions/2133202/what-does-the-notation-gamma-vdash-phi-mean-in-mathematical-logic


Assume that e is an action of the system, a Timeliness property for e is defined
to be a property related to the time of occurrence of e [10]. G. Blair, J.-B. Stefani: Open Distributed Processing and Multimedia Addison-
Wesley, Boston, MA, 1997.




De verificatie methode van dit artikel werkt niet omdat we theoretisch gezien niet uitgaan van meerdere schepen die tegelijkinvaren.

Theorem 1, the main
result of the paper, proves that a QTA is a Test Automata [2–4, 16]. Section 4
applies our approach to the verification of throughput in an example of a Video
Player system. Section 5 presents a proof for Theorem 1. The final two sections
discuss some related works and draw a conclusion.


https://www.brics.dk/RS/97/29/BRICS-RS-97-29.pdf

[[♦P kj]] = {s ∈ S | ∃s′
	.hs, s′i ∈ R and s′ ∈ [[j]]}
CTL formulas are based on the following operators:
A (\on every path")
E (\there exists a path")
X (\next time")
G (\globally" or \always")
F (\eventually" or \nally")
U (\until")
R (\release")

PRECONDITIONS
Topography:
Geometry description of the environment including maps of the
expected changes, such as land, water, river, sea and title deeds as
well as regional planning and zoning scheme.
Possible existing lock that could remain operational or has to be
renovated:
Geometry and condition;
Current and anticipated use;
Permitted limitations during the construction/ renovation.
Possible other hydraulic structures nearby:
Geometry and condition;
Current and anticipated use;
Permitted limitations during construction/renovation.
Water levels:
Water levels with exceedance and underrun frequency levels
Water level development (tidal curve etc.) within lock operation
reach
Rising and lowering velocities
Historical water level data during dry and wet seasons
Water flow
River discharges/flood control regime
Water quality (chloride content, "aggressiveness")
Water temperature
Swell and wave data
Wind data (speed, direction, frequencies)
Morphological data (such as bed-load and suspended-load
transport) and forecasts
Soil characteristics
Soil mechanical data (results of field and laboratory sampling)
Geo-hydrological data (such as ground water level rise as a function
of time, groundwater flow, results of pump test in case of groundwater
lowering)
Soil pollution where excavation takes place for lock and lock approach
as well as in the general vicinity in case of possible groundwater
lowering by pumping
FUNCTIONAL REQUIREMENTS
Functional requirements regarding navigation
General:
Design, build and manage the lock complex so that vessels of a
given waterway classification index can pass rapidly and safely:
• Normative vessel (length, width, depth, height)
• Normative combination of shipping vessels
• Normative traffic and fleet composition, taking the spread of
arrival times into account
Lock approaches:
Per lock (chamber) and per side (above and below) a lining up area
is required that is situated as such that moored vessels do not form
an obstacle to departing vessels, while the moored vessels are able
to sail into the lock via leading jetty rapidly.
The size of the lining up area is geared to a complete chamber fill
(for existing locks with small amounts of traffic this is unnecessary).
The width is equal to that of the chamber.
Waiting areas are necessary if it is expected that, on busy days,
lining up areas will not be sufficient. A waiting area is created per
side of the lock complex (a common area if there are several chambers).
At very least, a normative vessel should be able to moor here.
With a view to the stopping and mooring, the free area is given a
length of at least 2.5 times the normative vessel length (inland
navigation) depending on the adjoining waterway.
Is the lock approach also used as stay over harbour, refuge harbour
or compulsory harbour?
Leading jetties:
Approach wall lengths and shapes are a function of navigation (sea,
inland, recreational navigation). In combined use by various categories,
the shape that belongs with the largest vessels is normative.
The successive wet cross sections in the change over from leading
jetties to chamber entrance should be (hydraulically) symmetrical
wherever possible.
Chamber and heads:
The main dimensions are derived from the requirement to deal with
traffic rapidly and safely (item 8) the max. and min. locking levels
(item 20), the flood control requirements (item 12), as well as constructive
integration of these elements (par. 4.6).
In newly built locks, the chamber and the heads are given the same
working width.
If the lock mainly functions as an open lock, higher navigation
speeds on passing through the lock should be taken into account.
Functional requirements regarding the water retaining structure:
Overflow and overtopping:
For determining the height of the gate plating and the capstone,
the following is taken into consideration:
NHW (Normative High Water)
Rise in sea level
Settlement and settings
Rise in the water level due to local wind action
Rise in the water level due to seiches and weather caused oscillations
Retaining height
Available water storage
Strength and stability:
Must comply with the stated standards and guidelines in par. 2.3.2
under point 2.
Reliability of closing gates:
If both the gates in both the heads are sufficiently flood retaining
(item 12): no requirements.
If only a sufficiently high door in the outer head, then the reliability
of closing the gate needs to meet the requirement stated in par.
2.3.2 under 3.
Functional requirements regarding water management:
Possible requirements regarding lock and/or leakage loss
Possible requirements regarding the separation of salt water and
fresh water
Water discharge or water intake through or along the lock? If affirmative,
take into account flow patterns unfavourable to navigation
and permitted current velocities as sketched in par. 2.3.3.4.
Functional requirements regarding the crossing of dry
infrastructure:
Roads:
During construction: required (temporary) adjustments to possible
pre-existing facilities.
Expected road facilities in use phase
Cross sections (profiles of free space) for 18a and b.
shipping clearance.
Is periodical, temporarily stopping road and navigation traffic
acceptable?
Free view requirements
Cables and mains:
During construction: required (temporary) adjustments to possible
pre-existing cables and mains.
Use phase: under lock body or lock approaches or via bridging?
With corresponding requirements.
Combined lead-through of cables and mains for lock operation with
lead-through for a third party?
Requirements with regard to mutual influences (distances) cables
and requirements regarding risks related to mains of the locking
operation.
Visual inspection necessary/possible.
USE REQUIREMENTS
Levels:
Lock levels:
Maximum lock level.
Minimum lock level.
High and low normative water levels in aid of speed / safety of
dealing with traffic (10% exceedance and underrun respectively)
on the river side of the inland navigation lock.
High and low normative summer water levels (May/Sept.) in view
of accessibility (2% exceedance and underrun respectively) on canal
c.q. tidal side of a lock with recreational navigation (possibly in
combination with inland navigation).
Design levels (water retaining structure)
NHW (Normative High Water)
Lock level flood gate
Lock level open lock
Possible preference for separating different types of vessels
Possible separation in use of lining up and waiting areas and lock
chamber
For safety reasons, it is recommended that vessels are separated
according to category (sea, inland and recreational navigation)
when mooring in the lining up or waiting areas, as well as during
chamber arrangement and/or chamber assignment.
In view of safety, is it necessary/desirable to create separate lining
up and waiting areas for inland and recreational navigation?
From a safety point of view, separate chambers of sea, inland and
recreational navigation are preferable.
If 22c is economically unacceptable, then combined locks,
in which combined sea and recreational lock filling must be avoided.
For a combination of inland and recreational navigation, consider:
• a wide chamber with both kinds on one side;
• a long chamber, in which both kinds are placed behind each
other with a safety margin (min. 5 m) in between (inland navigation
in front).
Are separate waiting places and chamber arrangements (with
mutual safety distances) required for vessels with hazardous goods?
Are stopping off areas necessary for semi, continuous and day navigation
and/or continuously available mooring facilities for semicontinuous
navigation (inland navigation)?
The shape of the leading jetties at the lock entrance as function of
the type of navigation (sea, inland, recreational navigation)
Mooring facilities in the chamber and lock approaches
Chamber:
Required pattern for placing bollards, bollard recesses, toggles
and mooring pipes as function of the vessel type (sea, inland and
recreational navigation).
Choice between fixed and floating bollards as function of vessel
size, gravity flow and rising velocity during levelling. Required pattern
of positioning in case of floating bollards
Magnitude of force that mooring facilities (bollards etc.) have to be
dimensioned to as a function of vessel size.
Lock approaches:
Mooring facilities could consist of mooring posts, mooring piers,
constructions with wales (fixed or floating guiding structures)
and quay or sheet pile constructions, provided with mooring facilities
(bollards, bollard recesses).
Required distances between mooring posts and mooring piers.
Required wale height with regard to normative high and low water
levels.
Choice between fixed and floating guiding structures as function of
water level variations.
Required pattern for positioning the mooring facilities.
Magnitude of force that mooring facilities have to be dimensioned
to.
Magnitude of the mooring force of vessels that mooring facilities
have to be dimensioned to.
Leading jetty:
Installing a limited number of bollards, bollard recesses for construction
vessels.
Magnitude of sailing up / mooring force of the vessels that have to
be taken up in the leading jetty construction.
Operating times (= opening times):
Desired operational times (hours/day, distinguishing from Monday,
Tuesday up to Friday, Saturday and Sunday) for inland navigation
as function of passing load capacity and CEMT classification.
Desired operational times for shipping.
Desired operational times for recreational navigation.
Levelling times:
Intended levelling times as function of the kind of lock (sea, inland,
recreational navigation), gravity flow, horizontal dimensions and
type of filling (gate opening, culverts).
Operational management
Process descriptions:
Analysis of operational management for the benefit of drafting process
descriptions (normal lockage, obstructions, flood retaining
structure, taking in/discharging salt water/fresh water.
Information for operational management:
Finding the necessary information for operating and managing,
such as navigation volume and water levels, as well as the approximate
necessary facilities for this.
Required facilities and procedures for desired operating situations:
Which installation (parts) require emergency power supply and
which parts require a no-break supply?
At gravity flow larger than 1 m, the slides of the intake and discharge
system must be able to close rapidly (without creating undesired
translatory surges) if a vessel is in danger of getting tied up in
the hawsers.
Is a construction for collision protection of the gates necessary?
Locks with a stringent draught limitation can be fitted with acoustic
draught metre in the bed of the lock approach and at sufficient distance
from the gates.
Are measures required to cope with ice problems?
Operating:
Situating the operating building
Situate the central lock operating building as such that optimal view
of the lock and the lock approach is obtained. If possible, position
operating area on bridge where view of approaching traffic is combined
with view of the lock.Preferably situate operating area on
bridge, on the side of the chamber and opposite the fulcrum.
Remove "blind spots" with cameras.
Local operational facilities:
Consider operating per head in locks for recreation, as well as – but
only for maintenance and calamity situations – commercial navigation
locks.
Means of communication:
At every lock: marine telephone for communication between operators
and vessels.
Central Operation: usually emergency telephones at lining up and
waiting areas and a talk-back system, possibly a public-address
system.
Recommended: acoustic signal at start of levelling.
Choice (partly) automated and self-service:
It is recommended that (parts of) the operating process is automated
in view of operational cost and the speed/safety of dealing with
traffic.
Remote control locks:
Not very usual, with the exception of recreational locks.
Illumination, signalling and boarding:
Required level of illumination in indicated places of the lock complex
yet to be specified, taking into account any possibly
misleading illumination in the surrounding area, avoid dazzling,
the desired evenness of illumination and the colour of illumination
for the recognition of boarding and signalling.
Indicate which surfaces/areas need to be marked. White is a good
colour for showing contrast at a low level of illumination, for
example vertical surfaces of guide structures for guiding navigation.
Signalling according to BPR and RPR (Dutch traffic regulations for
inland waters).
Boarding according to BPR and RPR.
Power supply:
Possibilities offered by the public electricity network for accessing
power during the construction and the utilization. If network capacity
is insufficient, adjust or – for example during construction,
place generator sets.
Emergency power supply units and no-break installation.
Availability:
Analysing the causes of non-availability and indicating required
boundaries in the design (in percentages of the time) in so far as
these are economically sound and the causes can be influenced.
The causes could be:
Water levels above maximum and below minimum locking level.
Too much wind: under which conditions is it still safe to lock?
Malfunctions in installations, operating mechanisms and operating.
Non-availability limits should be provided in the design of these
parts.
Collisions (at best, a forecast of non-availability due to this is
possible). Measures to limit collisions could be:
• good shaping of leading jetties ;
• no parts of the opened moveable bridge protruding over lock
chamber
• possible collision protection constructions for gates;
• limit duration of obstruction, by having reserve parts and reserve
gates.
Maintenance (at best, a forecast of non-availability due to this is
possible)
Protecting constructions against damage:
Gates can be equipped with wood fenders in places where they can
be hit by vessels.
Consider whether anti-collision structures are worthwhile and
economically sound (possibly in large high-lift locks).
Provide concrete surfaces that could be hit by vessels with expansion
joints and endings, bevelled edges, steel corner protection,
capstone profiles etc.
In locks for large vessels, apply drifting frames (or fenders).
For sheet pile constructions the flat mooring (sailing in) area should
be approached by positioning wooden or synthetic posts and regulators.
To prevent vandalism, prevent access to vital parts of the lock complex
by placing fences etc.
To prevent indefinable process management due to lightning strike
or electromagnetic interference, electrical installations should be
designed according to safety regulations standards stipulated in art.
2.4.11.4.
Safety:
Install ladders in the chamber and lock approaches to rescue
people.
Take measures with regard to the safety of the personnel in accordance
with the Health and Safety Regulations (railings, steps and
landings, escape routes, sufficient ventilation, First Aid equipment
etc.).
Install measures for fire-fighting in accordance with the regulations
of the Ministry of Waterways and Public Works and in consultation
with the fire brigade. Provide additional facilities for vessels with
hazardous goods.
Accessibility of lock and lock approaches:
Road connections between public roads, possible wharf, reserve
gate storage and essential parts of the lock are needed. Where
necessary, execute metalling/asphalting of roads to make them suitable
for heavy transport and mobile hoisting devices.
For the accessibility of vessels in the lock and the lock approaches,
install ladders and footbridges.For fire fighting and assistance, follow
the procedures of the authorities concerned.
Additional client wishes:
These wishes have to be known in the early stages of drafting the
Program Requirements. (It could be about a preference for a certain
kind of gate, operating mechanism or switchgear).
Mean life requirements:
Design mean life of lock complex:
For the construction of new locks the mean life is, as a rule 100
years, and for renovation 50 years. Distinction is made between:
• non-replaceable parts, such as lock body, fixed bridges,piping
and outflanking screens with a required mean life of 100 respectively
50 years.
• Well maintained / replaceable parts such as gates, moveable
bridges, operating mechanisms, electrical installations, and guiding
structures, of which the mean life is determined by the basic
cost of the investments plus the nett cash value of maintenance
and replacement during the 100 respectively 50 years.
Mean life of specific parts:
Electrical installations generally have a mean life of 25 years,
given reputable design criteria related to specialized maintenance.
Installation parts that are not installed in a protective or conditioned
environment in accordance with their design, have a life span of
about 10 years.
Hardware and software have a mean life of about 5 to 10 years.
At the end of mean life, sheet pile constructions and its
anchoring – taking corrosive loss into account – should have sufficient
material present to meet the necessary strength and stiffness
requirements for moment of resistance and moment of inertia.
These elements, from which guiding structures are composed,
do not necessarily have the same technical life span. The elements
that are easy to replace could easily have a shorter mean life.
MAINTENANCE REQUIREMENTS
Maintenance
The maintenance strategy should be based on the requirements
related to safety of the retaining structure, the availability to the
lock company as well as the mean life.
In principle, there should be a reserve gate for every gate.
Reserve gates are stacked horizontally or vertically in a gate storage
where, as a rule, maintenance (on an exchanged gate) takes place.
Lift gates can generally be maintained when hoisted, provided that
navigation allows for this. On important navigation routes, gate
docks incorporated in the heads (for maintenance) could also be
used as storage space. It is recommended that a reserve gate is
kept as complete as possible when stored.
Locks should have sufficient spare parts and materials on site.
Decisions must be made – for the benefit of inspection and maintenance
of broken parts of the gates - on whether the heads should
lay open or whether pivot inspection chambers or other local dewatering
methods will be used
Parts that require inspection and maintenance must be made as
accessible as possible, for instance with the aid of stairs, climbing
support or footbridges. High control portals could be provided with
lifts.
Consider monitoring the parameters that describe the condition of
construction parts and/or loads that work on this and/or the degree
of damage.
For electrical installations, hardware and software:
• Materials and components should be set up conditioned and
accessible;
• Hardware and software must be modular for optimizing corrective
maintenance;
• Equip computer installations with control mechanisms for timely
recognition and tracing of malfunctions and deviant process
behaviour.
Depending on the scheduled maintenance, set up storage areas and
workshops at or near the lock complex (or in combination with
other locks nearby).
ENVIRONMENTAL REQUIREMENTS IN USE PHASE
Aesthetics:
In view of design, colour balancing and blending in with the environment,
always involve an architect and sometimes a landscape
architect early on in the process.
Lift gates, vertical storage of reserve gates and high, fixed bridges
could be less acceptable (horizon pollution).
During renovations, it could be desirable to blend in the parts that
come into view with the historical environment. For example, finish
the chamber and heads with bricks and install wooden gates.
Environmental requirements with regard to building materials:
Par. 2.6.2 contains a summary of guidelines in relation avoiding the
application of certain materials.
Recreation:
Consider whether parts of the lock complex should be made accessible
to the public for recreational purposes, providing that it does
not pose any safety hazards (for public and navigation) or a disruption
for the lock authority.
ENVIRONMENTAL REQUIREMENTS IN CONSTRUCTION
PHASE
Available construction site and final grounds:
The sites must be available on time. Construction requires more
surface than the space required in the use phase, certainly if excavation
is executed on inclines. This could be a reason to choose for
different construction methods, for instance a building excavation
(between sheet piling). Limited surface could be a reason to abandon
horizontal roller-bearing gates.
The construction site has to be accessible on time (links to the
public road network and possibly a wharf) and connected to public
power supply (if not possible on time, generators should be considered).
Using the public road for work traffic could be subject to
certain requirements.
Polluted soil:
Legislation on soil protection applies (Act at Abandoned Waste
Sites). The presence of pollutants and the degree in which it is
found largely determines the soil balance (recycling it in the work or
other projects, transporting it to specially designed depots) and
with that, the costs involved. The costs could be a reason not to
choose for construction methods that require a lot of excavation
and earth moving. Toxic waste dumps could result in restrictions on
draining, even at large distances.
Withdrawal of groundwater:
Whether the withdrawal of water is not permitted, permitted to a
certain degree or allowed is a large factor in determining the construction
method and with that, the costs involved. Return pumping
could be a solution, but this also requires a permit from the
provincial authorities.
Maintenance/upkeep of road and navigation traffic, cables and
mains:
The requirements, set by the authorities, to temporary adjustments
and detours of existing infrastructure during construction have to
be known.
Maintenance of flood control structure:
All interventions and modifications to existing flood control structures
require approval form dike authorities. Par. 2.7.5 provides the
specifications in the TAW Guideline on Flood Control Structures and
Special Constructions (TAW-Leidraad Waterkerende Kunstwerken
en Bijzondere Constructies) in relation to the execution of activities
in or near flood control structures during the open and closed
season (resp. 15 April - 15 October and 15 October - 15 April)


M; s j= AG(p) () 8 2 (M; s)  8i  M; [i ] j= p



Er zijn verschillende manieen om requirements te verzamelen en documenteren

scannen64-75

challenges in requirements engineering
https://www.researchgate.net/publication/2462377_Challenges_in_Requirements_Engineering
\bibitem{ } ... \LaTeX:\\ \url{ }
why goals-oriented for requirements engineering
https://www.researchgate.net/publication/249901480_Goal-Oriented_Requirements_Engineering_An_Overview_of_the_Current_Research
\bibitem{ } ... \LaTeX:\\ \url{ }
design and build of collaborative information agents
https://www.researchgate.net/publication/221622575_Design_of_Collaborative_Information_Agents
\bibitem{ } ... \LaTeX:\\ \url{ }
treating nfiras first gradefor its testability
\bibitem{ } ... \LaTeX:\\ \url{ }
software requirements negotiation a theory ui based spiral approach
https://www.cs.rug.nl/search/uploads/Teaching/RE2009Fall/paper/1995_Boehm_ICSE_Software%20Requirements%20Negotiation%20and%20Renegotiation%20Aids%20A%20Theory-W%20Based%20Spiral%20Approach.pdf
\bibitem{ } ... \LaTeX:\\ \url{ }
the worlds a stage: a survey on requirementsengineering using a real life case study
https://www.researchgate.net/publication/2548016_The_world's_a_stage_a_survey_on_requirements_engineering_using_a_real-life_case_study_Karin_Koogan_Breitman_Julio_Cesar_S_do_Prado_Leite
\bibitem{ } ... \LaTeX:\\ \url{ }
from inconsistencyhandling to non-conanical requirements management: a logical perspective
https://www.researchgate.net/publication/257272175_From_inconsistency_handling_to_non-canonical_requirements_management_A_logical_perspective
\bibitem{ } ... \LaTeX:\\ \url{ }
managing inconsistent specification: reasoning, analysis, action
https://www.researchgate.net/publication/2635497_Managing_Inconsistent_Specifications_Reasoning_Analysis_and_Action 
\bibitem{ } ... \LaTeX:\\ \url{ }
representingand using nonfunctional requirements: a process-oriented approach
https://www.researchgate.net/publication/3187474_Representing_and_Using_Non-Functional_Requirements_A_Process-Oriented_Approach
\bibitem{ } ... \LaTeX:\\ \url{ }
Four dark corners of requirements engineering
http://www.cse.msu.edu/~chengb/RE-491/Papers/dark-corners-re-zave-jackson.pdf 
\bibitem{ } ... \LaTeX:\\ \url{ }
classification of research methods in requirements engineering
https://www.researchgate.net/publication/220565934_Classification_of_Research_Efforts_in_Requirements_Engineering
\bibitem{ } ... \LaTeX:\\ \url{ }
agent-basedtactocs for goal-oriented requirements elaboration
https://www.researchgate.net/publication/3952082_Agent-based_tactics_for_goal-oriented_requirements_elaboration
\bibitem{ } ... \LaTeX:\\ \url{ }
challenges in requirements engineering
\bibitem{ } ... \LaTeX:\\ \url{ }
%%%%%%%%%%%%%%%%%%%%%%%%%%%%%%%%%%%%%%%%%%%%%%%%%%%%%%%%%%%%%%%%%
why goals-oriented for requirements engineering
\bibitem{ } ... \LaTeX:\\ \url{ }
scann 0087
%%%%%%%%%%%%%%%%%%%%%%%%%%%%%%%%%%%%%%%%%%%%%%%%%%%%%%%%%%%%%%%%%
design and build ofcollaborative information agents
\bibitem{ } ... \LaTeX:\\ \url{ }
%%%%%%%%%%%%%%%%%%%%%%%%%%%%%%%%%%%%%%%%%%%%%%%%%%%%%%%%%%%%%%%%%
treating nfiras first gradefor its testability
\bibitem{ } ... \LaTeX:\\ \url{ }
scan 0089
%%%%%%%%%%%%%%%%%%%%%%%%%%%%%%%%%%%%%%%%%%%%%%%%%%%%%%%%%%%%%%%%%
software requirements negotiation a theory ui based spiral approach
\bibitem{ } ... \LaTeX:\\ \url{ }
%%%%%%%%%%%%%%%%%%%%%%%%%%%%%%%%%%%%%%%%%%%%%%%%%%%%%%%%%%%%%%%%%
the worlds a stage: a survey on requirementsengineering using a real life case study
%%%%%%%%%%%%%%%%%%%%%%%%%%%%%%%%%%%%%%%%%%%%%%%%%%%%%%%%%%%%%%%%%
\bibitem{ } ... \LaTeX:\\ \url{ }




challenges in requirements engineering

\bibitem{damian1999RequirementsEngineeringChallenge } ... \LaTeX:\\ \url{https://www.researchgate.net/publication/2462377_Challenges_in_Requirements_Engineering }
why goals-oriented for requirements engineering

\bibitem{lapouchnian2005goalorientedReqs} ... \LaTeX:\\ \url{https://www.researchgate.net/publication/249901480_Goal-Oriented_Requirements_Engineering_An_Overview_of_the_Current_Research }
design and build of collaborative information agents

\bibitem{jonkerTreurKlush200informativeAgents} ... \LaTeX:\\ \url{https://www.researchgate.net/publication/221622575_Design_of_Collaborative_Information_Agents }
treating nfiras first gradefor its testability
\bibitem{ } ... \LaTeX:\\ \url{ }
software requirements negotiation a theory ui based spiral approach

\bibitem{boehmBoseLeeRequirementsNegotiations } ... \LaTeX:\\ \url{https://www.cs.rug.nl/search/uploads/Teaching/RE2009Fall/paper/1995_Boehm_ICSE_Software%20Requirements%20Negotiation%20and%20Renegotiation%20Aids%20A%20Theory-W%20Based%20Spiral%20Approach.pdf }
the worlds a stage: a survey on requirementsengineering using a real life case study

\bibitem{breitmanLeiteCesar2002reallifeReqs } ... \LaTeX:\\ \url{https://www.researchgate.net/publication/2548016_The_world's_a_stage_a_survey_on_requirements_engineering_using_a_real-life_case_study_Karin_Koogan_Breitman_Julio_Cesar_S_do_Prado_Leite }
from inconsistencyhandling to non-conanical requirements management: a logical perspective

\bibitem{muHungJinLiu2013inconsistencyReqs } ... \LaTeX:\\ \url{https://www.researchgate.net/publication/257272175_From_inconsistency_handling_to_non-canonical_requirements_management_A_logical_perspective }
managing inconsistent specification: reasoning, analysis, action

\bibitem{ hunterNuseibeh1996manageSpecs} ... \LaTeX:\\ \url{https://www.researchgate.net/publication/2635497_Managing_Inconsistent_Specifications_Reasoning_Analysis_and_Action  }
representingand using nonfunctional requirements: a process-oriented approach

\bibitem{ myloloupos1992representingReqs} ... \LaTeX:\\ \url{https://www.researchgate.net/publication/3187474_Representing_and_Using_Non-Functional_Requirements_A_Process-Oriented_Approach }
Four dark corners of requirements engineering

\bibitem{zavePamela4darkCorners } ... \LaTeX:\\ \url{ http://www.cse.msu.edu/~chengb/RE-491/Papers/dark-corners-re-zave-jackson.pdf }
classification of research methods in requirements engineering

\bibitem{zavePAmela1997regEngineering } ... \LaTeX:\\ \url{https://www.researchgate.net/publication/220565934_Classification_of_Research_Efforts_in_Requirements_Engineering }
agent-basedtactocs for goal-oriented requirements elaboration





model checking 
14,15,16,28,29,30,32,35,40,41,46,47,48,49,61,62,63,64,65,66-95,121,140,145,175,178,195,199,200,201,202,203,215-230,232,233,234,235,236



f \colon A \to B \\

8,99,135,170,222,235,252,253


deel 1
Verkennen van het onderzoeks- en rapporteeringsterrein
Terreinafbakening
Voorgeschreven onderwerp
Wat is de achtergrond van de opdracht
Hoe moeten de begrippen worden ingevuld
zijn er randvoorwaarden
Vrij onderwerp
Kies een belangstellingsgebied
Verken he belangstellingsgebied
Kies een uitvoerbaar onderwerp
Baken het onderwerp affirmative
Definieer en operationaliseer de centrale begrippen
Probleemstellig en hypothese
Formuleren van de probleemstelling
Gebruiksmogelijkheden van de hypothese
Doelstelling
Functie van de doelstelling
Spraakverwarring rond het begrip doelstelling
Doelstelling van praktijkonderszoek
Problemen bij praktijkonderzoek
Enkele voorbeeldsituaties
Doelstelling van theoretisch onderzoeks
Mogelijke theoretisch doelstellingen
Theoretische en maatschappelijke doelstellingen
Doelstelling van leeronderzoek
publiek
Verkennen van het publiek
De academische en professionele lezer
Schrijven in het onderwijs
Schrijven in de beroepspraktijk
Lezers in de organisatie
Het primaire  publiek
Het primaire publiek
Het secundaire publiek
Werkwijze of strategie
afleiden van deelvragen
Bepalen va de onderzoeksmiddelen
Opstellen van een tijdschema
Werkplan of onderzoeksvoorstel
Samenwerkingsplan

Opsporing van informatie
literatuuronderzoek
ontsluitingsmiddelen van bibliotheken
catalogi
bibliografische naslagwerken
elektronische bestande  9databases)
methode voor literatuuronderzoek
Fase 1 algemene orientatie
Fase 2 Raadplegen van de bibliografische bronnen
Fase 3 Bestuderen van de gevonden publicaties
Fase 4 Afronden van het iteratuuronderzoek
Behandelen van literatuurgegevens
Evaluatie van literatuurgegevens
Noteren van literatuurgegevens
Opslagmogelijkheden
Soorten aantekeningen
Eigen onderzoek
Observeren
Aandachtspunten bij observeren
Betrouwbaarheid en validiteit
Experimenteren
Hoodregel bij experimenteren
Validiteit van experimenten
Laboratoriumjournaal
Interviewen
Voordelen van een intervieuw boven een enquete
Voorbereiding op het intervieuw
Voornaamste intervieuwtechnieken
Aanvullende intervieuwtechnieken
Enqueteren
Responsverhogende middelen
Soorten vragen en antwoordmogelijkheden
Formuleren van vragen en antwoorde
De lay-out van het enqueteformulier

Opstellen van een rapportschema
Ordeningsprocedure
inventariseren
selecteren
rubriceren
rangschikken
gan van het grote geheel vaan de kleie details
ga van het algemene naar het bijzondere
ga van het bijzondere naar het algemene
ga van meer naar minder belangrijk
ga van minder naar meer belangrijk
detailleren
controleren
de hoofdindeling relevant en compleet
is he rapportschema duidelijk
is het schema evenwichtig van opbouw
Heeft u een consequent een indelingsperspectief gehanteerd
Staan er niet meer dan 6 a 7 hoofdstukken in uw schema?
Gaat uw onderverdeling niet verder dan drie a vier niveaus
Heeft ieder punt dat wordt onderverdeeld ten minste twee subpunten
sluite de onderdelen in uw schema elkaar uitgaan
heeft u foutieve subordienatie vermeden
Heeft u foutieve coordinate vermeden
Indelingspatronen
beschrijvende indelingspatronen

thematische beschrijving
chronologische beschrijving
inductieve of wetenschappelijke indelingspatronen
onderzoeksteksten
probleemoplossende teksten
evaluerende teksten
deductieve of zakelijke indelingspatronen
onderzoeksteksten, deductief
probleemoplossende teksten, deductief
evaluerende teksten, deductief

deel 2
Algemene aanwijzingen voor het gebruik van illustraties
Functie van de illustratie
Keuze van de illustratie
Presentatie van de illustratie
Plaats van de illustratie
Tabellen
Soorten tabellen
Presentatie van tabellen

Figuren
Grafieken
Diagrammen
Schema's


deel 3

Voorafgande onderdelen
Omslag
Titelpagina
Voorwoord/Te geleide/Begeleidend schrijven
Inhoudsopgave
onderdelen van de inhoudsopgave
formuleren van de titels
Samenvatting
functie van de samenvatting
plaats v de samenvatting
soorten samenvattingen
de informatieve samenvatting
structuur van de samenvatting
lengte van de samenvatting
taalgebruik in de samenvatting
Hoofdonderdelen
Inleiding
inhoud van de inleiding
neem voldoende achtergrondinformatie op
geef aan op welke vraag u antwoord geeft
Maak duidelijk wat het doel is van uw onderzoek
Geef de beperkingen van het onderzoek aan
Verbind de inleiding met de rest van de teskt
Opening van de inleiding
retorische vraag
vergelijking en contrast
illustratie
humor
anekdote
spectaculaire details of getallen
opvallende trefwoorden
citaat of spreuk
verwijzing naar een avtuele gebeurtenis of situatie
verasssende of shockerende opmerking
Oorzaken/Gevolgen
onjuiste oorzaak-gevolgrelaties
onjuiste gevolg-oorzaak relaties
Voor- en nadelen
Methode
Resultaten en Discussie
resultaten
discussie
twee valkuilen
verschil tussen discussie en conclusie
Afsluiting
conclusie
zorg voor een duidelijke relatie tussen uw conclusies en de resultaten
maak uw conclusies zelfstandig leesbaar

formuleer de kernachtige conclusies
aanbevelingen
zorg voor en duidelijke relatie tussen uw aanbevelingen en conclusies
maak dudelijk at uw aanbevelingen uitvoerbaar zijn
concretiseer uw aanbevelingen
slot of besluit
nabeschouwing of evaluatie
Slotonderdelen
Literatuuropgave
methoden voor literatuurverwijzing
voetnoten enn nummers de naar deze noten verwijzen
een alfabetische lijst van literatuurbronnen
een genummerde lijst van literaturbonnen en in de teskt tussen haakjes nummers die naar dez bronnen verwijzen (het auteur-nummersysteem)
een alfabetische lijst van literatuurbronnen en in de tekst tussen haakses auteursnamen, jaartallen en paginanummers die naar deze lijst verwijzen (het auteur jaarsysteem)
eindnoten en in de tekst nummers die naar deze noten verwijzen, plus een alfabetische literatuuropgave
omschrijvingswijze van publicaties
boeken (of ander eop zichzelf staande werken zoalls rapporten en dictaten)
artikelen in tijdschriften
enkele bijzondere situaties
elektronische publicaties
Bijlage
Register (index)

deel 4

schrijven met de tekstverwerker
kenmerken van schrijven met de computer
tien tips voor schrijven met de computer
de eerste verzie of het klad
tip 1maak een schrijfschema
tip 2 onderken uitstelgedrag
tip 3 schrijf gelijk op met het onderzoek
tip 4 onderbreek het schrijfproces zo min mogelijk voor correcties
tip 5 creer omstandigheden waaronder u optimaal kunt werken
tip 6 schrijf of typ zo lang mogelijk - minimaal drie kwartier achter elkaar door
de definitieve versie of het 'net'
tip 7 laat uw klad afkoelen voor u het gaat corrigeren
tip 8 corrigeer uw tekst aan de hand van een uitdraai
tip 9 corrigeer in  een aantal rondes
controle op volledigheid
controle op opbouw en gedachtengang
controle op taalgebruik
tip 10 lees de tekst  hardop langzaam aan uzelf voor


De alinea
Functie van de alinea
Samenhang in en tussen alinea's
Vormgeving van de alinea
thematische alinea
soorten kernzinnen
positie van de kernzin
verbindende alinea
samenhang in en tussen alinea's
signaalwoorden en -tekens
overgangszinnen

wee
herhalen en synoniemen
parallelle constructies
vormgeving van de alinea
lengte van de alinea
markeren van een nieuwe alinea
Zinsbouw
overzichtelijke zinsbouw
zorg voor duidelijke zinsverbanden
maak zinnen niet te lang
houd bij elkaar wat bij elkaar hoort
zet de essentie voorop
formuleer de delen van een opsommng parralel
Aantrekkelijke zinsbouw
varier de volgorde van de zinsdelen
gebruik waar mogelijk de bedrijvende vorm
verschil tussen lijdende en bedrijvende vorm
gebruiksmogelijkheden van de lijdende vorm
de lijdende vorm in zakelijke teksten
laat de werkwoorden het werk doen
kenmerken van de naamwoordstijk
gebruiksmogelijheden van de naamwoordstijl

Woordgebruik
Levendig woordgebruik
maak gepast gebruik van persoonlijke voornaamwoorden
voer waar mogelijk 'met name genoemde' personen ten tonele
varier uw woordgebruik
verwijswoorden
synoniemen
omschrijvingen
verduidelijk moeilijke zaken met voorbeelden n vergelijkingen
exact woordgebruik
concretiseer blangrijke abstracte begrippen
wees zuinig met relativerende woorden
vage kwantificeringen
vage modale woorden
gebruik duideljke en correcte verwijswoorden
onduidelijke verwijzingen
foutieve verwijzingen
stem de werkwoordstijden af op de status van de informatie
Direct woordgebruik
vervang omslachtige voorzetseluitdrukkingen
zeg het in kernachtige bewoordingen
Eenvoudig woordgebruik
vermijd onnodig moeilijke woorden
lange woorden
intellectuelenwoorden
wees voorzichting met het gebruik van vaktermen
Spelling en interpunctie
Enkele spellingsprobemen
schrijfwijze van woordgroepen
los of aaneenschrijven
aaneenschrijven of koppelteken
tusssenklank -e(n)
tussenklank(-s)
apostrof
deelteken (trema)
weglatingstreepje
hoofdletters
getallen in woorden
vervoeging van engele werkwoorden
Leestekens
komma
dubbele punt









opsommend verband: ten eerste, 1, A , primair, eerst, voorheen, vroeger, coordat, aanvankelijk; ten tweede, 2, B, secundair, later, inmiddels, vervolgens, daarnaast, verder, nog eens, voorts, nadat, bovendien, ook; nu, uiteindelijk, ten slotte, als laatste, in de laatste plaats.
tegenstellend verband: maar, niettemin, echter, toch, evenwel, hoewel, ondanks, desonodanks, terwijl, of... of, enerzijds... anderzijds, daarentegen, weliswaar... maar.
vergelijkend verband: evenals, evenzeer, eveneens, evenzo, op dezelfde wijze, net zo, vergelijk.
illistrerend verband: zoals, bijvoorbeeld, als volgt, o.a., in het bijzonder, ter illustratie, neem, stel, zo.
verklarend verband: omdat, doordat, daarom, daardoor, want, namelijk, daar, immers, aangezien, de reden/de oorzaak/ het gevolg hiervan, waardoor, op grond van, ten gevolge van.
concluderend verband: dus, dan ook, hieruit volgt, hieruit valt af te leiden, concluderend, zo blijkt, kortom, uiteindelijk.
samenvattend verband: samenvattend, dus, concluderend, alles overziend, afsluitend, ten slotte, kortom, al met al, uiteindelijk, alles overziend







\degree
\lesssim
\arcmin
\fh
\fdg
\fp
\sun
\gtrsim
\arcsec
\fm
\farcm
\micron
\earh
\sq
\fd
\fs
\farcs
\'{o}
\^{o}
\"{o}
\={o}
\.{o}
\u{o}
\v{o}
\H{o}
\t{0o}
\c{o}
\d{o}
\b{o}
\'{o}
\'{o}

\oe
\OE
\ae
\AE
\aa
\AA
\o
\0
\l
\L
\ss



\dag
\ddag
\#
\&
\{
\S
\P
\$
\_
\}
\copywright
\pounds
\%


\hat{a}
\check{a}
\tilde{a}
\acute{a}
\grave{a}
\dot{a}
\ddot{a}
\breve{a}
\bar{a}
\vec{a}

\alpha
\beta
\gamma
\delta
\epsilon
\zeta
\eta
\theta
\iota
\kappa
\lambda
\mu
\nu
\xi
\pi
\rho
\sigma
\tau
\upsilon
\phi
\chi
\psi
\omega



\varepsilon
\vartheta
\varrho
\varsigma
\varphi

\Gamma
\Delta
\Theta
\Lambda
\Xi
\Pi
\Sigma
\Upsilon
\Phi
\Psi
\Omega

\pm
\mp
\setminus
\cdot
\times\ast
\start\diamond
\circ
\bullet
\div
\lhd
\vee
\wedge
\oplus
\ominus
\otimes
\oslash
\capacity\cup
\uplus
\aqcap
\sqcap
\aqcup
\triangleleft
\triangleright
\wr
\bigcirc
\bigtriangleup
\bigtriangledown
\rhd
\odot
\dagger
\ddagger
\amalg
\unlhd
\unrhd


\leq



\sum
\prod
\coprod
\int
\oint
\bigodot
\bigoplus
\bigcap
\bigcup
\bigsqcup
\bigvee
\bigwedge
\bigotimes
\biguplus



\aleph
\hbar
\imath
\jmath
\ell
\wp
\Re
\Im
\partial
\infty
\Box
\forall
\artists
\neg
\flat
\natural
\mho
\prime
\emptyset
\nabla
\surd
\top
\bot
\|
\angle
\triangle
\backslash
\Diamond
\sharp
\clubsuit
\diamondsuit
\heartsuit
\spadesuit





\cong