\newpae


\section{Research case: De digitale aanval op de Oekrainese krachtcentrale}
Dit verslag geeft inzage in een analyse van de Ukraine cyber aanval,
inclusief hoe de actoren zich zelf toegang gavan tot het controle systeem, welke methoden de acoren hebben gebruikt voor reconnaissance en vastleggen van het systeem, een gedetailleerde omshrijving van de aanval op 15 December 2015, en de methoden die gebruikt zijn door de aanvallers om hun sporen uit te wissen en daarmee het het stoppen van schade toebrengen  nog moeilker maken. Daarnaast wordter  een gedetailleerde omschrijving gevevenv an de beveiliging van de SCADA ccontrol systemen gebaeerd op bst practices, inclusief het control network ontwerp, technieken voor whtelisting, monitoring en loggen, en  opleiding van personeel.

https://na.eventscloud.com/file_uploads/aed4bc20e84d2839b83c18bcba7e2876_Owens1.pdf
https://www.wired.com/2016/03/inside-cunning-unprecedented-hack-ukraines-power-grid/
https://www.boozallen.com/content/dam/boozallen/documents/2016/09/ukraine-report-when-the-lights-went-out.pdf
https://www.reuters.com/article/us-ukraine-cybersecurity-sandworm-idUSKBN0UM00N20160108
https://www.mandiant.com/resources/blog/ukraine-and-sandworm-team
https://www.ifri.org/sites/default/files/atoms/files/desarnaud_cyber_attacks_energy_infrastructures_2017_2.pdf
https://ris.utwente.nl/ws/files/6028066/3-s2_0-B9780128015957000227.pdf
https://repositorio-aberto.up.pt/bitstream/10216/119066/2/315683.pdf
https://www.diva-portal.org/smash/get/diva2:1046339/FULLTEXT01.pdf
https://www.vice.com/en/article/zmeyg8/ukraine-power-grid-malware-crashoverride-industroyer



Oop 23,december 2015  vind er een cyber aanval plaats op het elektriciteitsnet van de Oekraine. Dit was de eerste bekende aanval op een elektrisch controle  system met corrupte firmware. Daarnaas wordt er een telecom-based denial of service attack met  geautomatieerde systemen om het telefoonverkeer uit te schakelen.
\cite{Whitehead2017ukrainepoweroutage}

Uit onderzoek\cite{zetter2016GridHack} naar de aanval,  uitgevoerd door Oekraiene sen Amerikaanse militairenblijkt  bleek onder meer dat de power grids in sommige gevallen beter waren beveiligd dan de Amerikaanse. Desondanks was de viligheid niet optimaal door onder andere de  hetgegeven dat werknemers op afstand konden inloggen en geen gebruik van 2-stapsverificatie.


\subsection{Literaire analyse}

\subsubsection{Motief}
Oekraine wijst naar de russen \cite{zetter2016GridHack}
https://www.wired.com/story/russian-hackers-attack-ukraine/
https://www.boozallen.com/content/dam/boozallen/documents/2016/09/ukraine-report-when-the-lights-went-out.pdf
https://www.reuters.com/article/us-ukraine-cybersecurity-sandworm/u-s-firm-blames-russian-sandworm-hackers-for-ukraine-outage-idUSKBN0UM00N20160108
https://www.reuters.com/article/us-ukraine-crisis-cyber-idUSKBN15U2CN
https://theconversation.com/cyberattack-on-ukraine-grid-heres-how-it-worked-and-perhaps-why-it-was-done-52802
https://jsis.washington.edu/news/cyberattack-critical-infrastructure-russia-ukrainian-power-grid-attacks/
\subsubsection{Situatie Oekraiene}
https://www.dragos.com/wp-content/uploads/CrashOverride-01.pdf
https://www.dragos.com/wp-content/uploads/CRASHOVERRIDE.pdf
\subsubsection{Situatie algemeen}
https://www.politico.eu/article/ukraine-cyber-war-frontline-russia-malware-attacks/
https://www.ifri.org/sites/default/files/atoms/files/desarnaud_cyber_attacks_energy_infrastructures_2017_2.pdf
https://www.cybersecurityintelligence.com/blog/attack-on-ukraines-power-grid-targeted-transmission-stations-4530.html

\subsubsection{Factoren}
http://web.mit.edu/smadnick/www/wp/2016-22.pdf
\subsubsection{Oorzaak}
https://www.sans.org/blog/confirmation-of-a-coordinated-attack-on-the-ukrainian-power-grid/
https://arstechnica.com/information-technology/2017/06/crash-override-malware-may-sabotage-electric-grids-but-its-no-stuxnet/
https://www.darkreading.com/threat-intelligence/first-malware-designed-solely-for-electric-grids-caused-2016-ukraine-outage
https://www.dragos.com/wp-content/uploads/CRASHOVERRIDE.pdf
\subsubsection{Gebruikte materialen}
https://en.wikipedia.org/wiki/2015_Ukraine_power_grid_hack
https://www.cisa.gov/news-events/alerts/2017/06/12/crashoverride-malware
https://rhebo.com/en/service/glossar/industroyer-25114/


\subsubsection{Uitvoering van de aanval}
https://na.eventscloud.com/file_uploads/aed4bc20e84d2839b83c18bcba7e2876_Owens1.pdf
https://www.boozallen.com/content/dam/boozallen/documents/2016/09/ukraine-report-when-the-lights-went-out.pdf
\subsubsection{Oplossingen}
https://na.eventscloud.com/file_uploads/aed4bc20e84d2839b83c18bcba7e2876_Owens1.pdf
https://www.cisa.gov/news-events/ics-alerts/ir-alert-h-16-056-01
\subsubsection{Aanbevelingen}

\subsection{Resultaten}
\subsubsection{De aanval}
1. An initial email spear phishing attack lures recipients
into opening an attached Microsoft® document with a
macro that installs Black Energy 3 (BE3) onto
corporate workstations.
2. BE3 and other tools perform reconnaissance and
enumeration of the network and provide an initial
backdoor for the hackers into the corporate network.
3. As a result of network reconnaissance, the malicious
actors discover and access the oblenergos’ Microsoft
Active Directory® servers that contain corporate user
accounts and credentials.
4. With the harvested credentials, the malicious actors use
an encrypted tunnel from an external network to get
inside the oblenergo network, establishing a presence
on the oblenergo control system networks.
5. Malicious actors discover and access the control center
supervisory control and data acquisition (SCADA)
human-machine interface (HMI) servers and
substations. While a router separates corporate and
SCADA networks, the firewall rules are improperly
configured.
6. On December 23, 2015, at 3:30 p.m., the malicious
actors begin their power outage attacks by entering
operations and SCADA networks through backdoors on
the compromised SCADA workstations. The malicious
actors take control away from HMI operators and then
open breakers.
7. The malicious actors perform several other actions with
the intent of complicating the responses of control
operators and increasing the effort required to return the
system to normal operating conditions. These actions
include:
a. Launching a coordinated Telephony Denial of
Service (TDoS) attack that floods call centers to
prevent legitimate calls from getting through.
b. Disabling the UPSs for the control centers.
c. Corrupting the firmware on a remote terminal unit
(RTU) HMI module and serial-to-Ethernet port
servers.
8. Malicious actors execute KillDisk malware in an
attempt to wipe out the control center HMIs and pivotpoint workstations.
https://na.eventscloud.com/file_uploads/aed4bc20e84d2839b83c18bcba7e2876_Owens1.pdf
https://www.boozallen.com/content/dam/boozallen/documents/2016/09/ukraine-report-when-the-lights-went-out.pdf
\subsubsection{spearfishing}
\subsubsection{blackenergy}
\subsubsection{remote access capabilities}
\subsubsection{serial-to-ethernet communication devices}
\subsubsection{telephony denial of service attacks}

\subsection{oplossingen}
Identificeer alle risicos en schrijf een plan foor het managen van de risico's.
Implementeer  effecteve controle  om het riico te managen.
Creeer een diepgaand model dat ervoor zor dat er efectieve en efficiente security controls worden uitgevoerd.
Aangaande de gebeurtenissen in de oekraiene kunnen de volgende security controls worden opgenomen in het securitymodel: Initial access to enterprise network, pivot in interprise network, elevate priviliges, maintainance access, gain access to control system, attack, attack complication, destroy hard drives.
\cite{Whitehead2017ukrainepoweroutage}

\subsection{Discussie}

\subsection{Verder lezen}
https://citeseerx.ist.psu.edu/viewdoc/download;jsessionid=0513EED48102FDAD1BD940260EF12B11?doi=10.1.1.548.7490&amp;rep=rep1&amp;type=pdf
https://scialert.net/fulltext/?doi=tasr.2014.396.405
https://www.researchgate.net/publication/333671061_Attacking_IEC-60870-5-104_SCADA_Systems
https://www.welivesecurity.com/wp-content/uploads/2017/06/Win32_Industroyer.pdf
https://blog.nettedautomation.com/2017/
https://arxiv.org/pdf/2001.02925.pdf
https://dl.acm.org/doi/fullHtml/10.1145/3381038
https://www.win.tue.nl/~setalle/2017_fauri_encryption.pdf
http://www.connectivity4ir.co.uk/article/175490/IEC-62351--Secure-communication-in-the-energy-industry.aspx
https://www.virsec.com/resources/blog/virsec-hack-analysis-deep-dive-into-industroyer-aka-crash-override
https://dreamlab.net/en/blog/post/fuzzing-ics-protocols/
https://www.blackhat.com/docs/us-17/wednesday/us-17-Staggs-Adventures-In-Attacking-Wind-Farm-Control-Networks.pdf
https://blog.checkpoint.com/research/crashoverride/
https://www.blackhat.com/us-17/briefings/schedule/#industroyercrashoverride-zero-things-cool-about-a-threat-group-targeting-the-power-grid-6159
https://search.abb.com/library/Download.aspx?DocumentID=9AKK107045A1003&amp;LanguageCode=en&amp;DocumentPartId=&amp;Action=Launch
https://iiot-world.com/ics-security/cybersecurity/five-cybersecurity-experts-about-crashoverride-malware-main-dangers-and-lessons-for-iiot/
https://www.csoonline.com/article/3200828/crash-override-malware-that-took-down-a-power-grid-may-have-been-a-test-run.html
https://www.paloaltonetworks.com/blog/2017/06/crashoverrideindustroyer-protections-palo-alto-networks-customers/
https://www.webopedia.com/definitions/crashoverride-industroyer-malware/
https://www.cyber.nj.gov/threat-center/threat-profiles/ics-malware-variants/crashoverride
https://www.nixu.com/blog/crashoverride-threat-electricity-networks
https://www.virusbulletin.com/virusbulletin/2019/03/vb2018-paper-anatomy-attack-detecting-and-defeating-crashoverride/
https://en.wikipedia.org/wiki/Crash_Override_Network
https://en.wikipedia.org/wiki/Industroyer
https://www.dragos.com/resource/crashoverride-analyzing-the-malware-that-attacks-power-grids/
https://www.wallix.com/blog/ics-security-russian-hacking
https://www.nixu.com/fi/node/53
https://control.com/forums/threads/comparison-between-iec60870-5-103-and-modbus-rtu.20317/