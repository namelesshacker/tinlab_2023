\hoofdstuk{Model}



moet de intitial state altijd in een loop zitten in uppaal?
wat zijn urgent channels?
rampen? er staat wel iets in de planning maar kan geen lessen of verdere documentatie of requirements terug vinden?	


gesprek wessel:
main controller slim dat direction een bool is. 
pomp is te slim, zoiu alleen maar aan of uit moeten gaan, of nog weg en in pompen maar meer niet. niets met waterlevel en aantal schepen.
schip: niet doen. als een schip zich aanmeld, dan gebeuren er dingen, maar gaat hij naar binnen? je weet niet wat dat schip gaat doen want menselijk gedrag. beter niet het schip uitgebreid maken, maar eerder de sluis. te veel aannames.

wessel model: alleen als wachtrij vol zit, doet de sluis iets.
deur heeft een parameter zodat er meerdere deuren in de simulator neergezet kunnnen worden. ook bij wachtrij.

stoplichen kunnen er wel in maar als je simpeler wilt, gaan die als eerste weg.
zes variabelen model is voorgesteld maar niet goed op gereageerd. alleen er van af weten is genoeg.
rampen alleen voor persoonlijk verslag



Om voor mezelf een beeld te krijgen van wat een sluis is en hoe deze moet werken is er een aantal foto's verzameld van sluizen.	



Uit deze afbeelding blijkt het volgende:
Hoogteverschip t.o.v NAP
2 sluisdeuren
stoplichten
Uit een onderzoek naar de werking van de verschillende sluizen in nederland wordt rekening gehouden met de aanmelding van sluizen en de gebruiktstijd van sluizen.

Met de aanmelding van schepen wordt omschreven welke acties er door de schipper de sluismeeter moet worden gedaan om de positie, tijdstip en lengte van een invarendship te communiveren.

Met de gebruikstijd wordt  de daadwerkelijke tijd aangeduid waarin het scheepsverkeer/waterverkeer gebruik kan maken van de sluis en onder welke voorwaarden zoals wachttijd, gewicht, terugvaarmogelijkheden etc).

Directe requirements van opdrachtgever:\\
Na grondige analyse van het Nederlandse sluizenpark is gebleken dat renova-tie van een groot aantal sluizen noodzakelijk is.  Een eerste verkenning heeft onsgeleerd dat het gecombineerd renoveren en automatiseren van het Nederlandsesluizenpark een aanzienlijke verbetering kan opleveren t.a.v.:\\
- veiligheid\\
- efficientie\\
- capaciteit\\
- onderhoudskosten\\
- duurzaamheid\\
In het kader van het onlangs afgesloten klimaatakkoord heeft de Nederlandseoverheid  daarom  besloten  over te gaan tot een ingrijpende renovatie van dediverse sluizen die ons land rijk is. Op het ministerie van infrastructuur en waterstaat is helaas onvoldoende kennis van ict en systemen aanwezig om eenen ander uit te voeren. Wij vragen u een model (of een onderling samenhangend aantal modellen)aan  te leveren, opdat ontwerpen van verschillende, volledig geautomatiseerde sluizen in de toekomst gerealiseerd kunnen worden.\\\\
Eigen inbreng van deze requirements:\\
Wij gaan er van uit dat het volgende van ons verwacht wordt:\\
Maak een model dat als template dient gebruikt te worden voor het automatiseren van verschillende soorten sluizen. Verder moeten overwegingen gemaakt worden die goed onderbouwd zijn.\\\\ Aangezien er van ons alleen een model verwacht wordt, zullen wij ons geheel focussen op de fundamentele werking van de sluis en hierbij zullen wij ons dus niet bezig  houden met fysieke eisen zoals veiligheidshekjes en borden. Onze focus ligt geheel op de werking van de sluis; elke state waar de sluis zich in mag bevinden en welke beslissingen de sluis moet maken op basis van bestaande protocols en benoemde eisen. \\\\
Deze requirements zullen hieronder uitgewerkt worden, per sluisonderdeel, deze bestaande uit de sluisdeuren, de sloplichten, de waterpomp en de boten.\\





\begin{itemize}
	\begin{minipage}{0.4\linewidth}
		\item Vooraanmelding
		\item informatie inwinnen
		\item operationele melding
		\item aankomst volgorde
		\item aanwijzen wachtplaats
		\item verstrekken informatie
		\item aanwijzen opstelplaats
		\item opstellen schutproces
		\item verstrekken informatie
		\item invaarvolgorde en ligplaats in sluis
		
	\end{minipage}
	\begin{minipage}{0.4\linewidth}
		
		\item gereedmaken voor invaren
		\item openen invaardeuren
		\item invaren toegestaan
		\item aanwijzingen voor invaren
		\item aanwijzingen tijdens afmeren
		\item invaren verboden
		\item sluiten invaardeuren
		\item start nivelleren
		\item stop nivelleren
		\item aanzwijzingen voor uitvaren
		\item openen uitvaardewuren
		\item uitvaren toegestaan
		
	\end{minipage}
	\begin{minipage}{0.4\linewidth}
		\item uitvaren
		\item operationele afmmelding
		\item utvaren verboden
		\item aanwijzing invaren nieuwe schepen
		\item invaren verboden
		\item deuren gesloten
	\end{minipage}
\end{itemize}






\paragraph{Requirements definitie}
Requirements zijn alleen die eisen die gesteld worden aan het gedrag of de kwaliteit van het systeem om te voorzien in de behoeften van een belanghebbende uit de business.




invaardeuren en uitvaardeuren
Gaan we uit van binnendeuren en buitendeuren? Er ontstaat dan een extra ruimte in de sluis. Hoeveel schepen kunnen in deze ruimte? Wat is de maximale wachtreij in deze ruimte en wat zijn de verkeersregels in deze nruimte?
invaarstoplicht en uitvaarstoplicht}
Als invaren is toegestaan hoe wordt dit dan doorgegeven aan de schepen in de sluis? moeten zij dan uit zichzelf wachten of krijgen zij een signaal dat zij wewl/niet mogen uitvaren? En moeten zij dan kiezen voor links, midden of rechts? Of maakt dat allemaal niets uit?

invaarwachtrij en uitvaarwachtrij
Als er meerder schepen in een sluiskolk zitten moet het systeem dan rekeneing houden met het schip dat als eerste is ingevaren en/of het langst in de sluis zit?


Sluisdeuren en stoplichten
De sluisdeuren aan weerszijde van de sluis  worden gebruikt om de toegang tot de sluiskolk mogelijk te maken en te bewaken in combinatie met de stoplicht.



Waterpomp
De waterpomp pompt water in de sluis of pompt water weg naar gelang de richting van het ingevaren schip.


Initially the clutch is closed
To open the clutch, it takes at least 100 ms and at most 150 ms
To close the cluch, it takes at least 100 ms and at most 150 ms
Initially the gearbox is neutral
To release the gear, it takes at least 100 ms and at most 200 ms.
To set a gear it takes at leasst 100 ms and at mose 300 ms.
The engine is always in a predefined state called initial when no gear is set.
To find zero torque in the engine, it takes at least 1150 ms and at most 400 ms. ut at 400 ms, the engine may enter an error state or find synchronous speed.
The  engine may regulate on synchronous speed in at most 500 ms.
When in an error state, the engine will regulate on synchrobous speed in at least 50 ms.


A gear change should ne performed within 1 seond (P6-p*,P3)
When an error arises, the system will reach a predefined error state marking the error (p9-p11)
The system should be able to use all gears ( p2-p3)
There will be no deadlocked stat in the system(p17)
When the system indicates gear neutral, the engine should  be in initial state (p12)
The gearbox controller will never indicate open or closed clutch when the clutch is closed or open respctively(p14)
The gearbox controller will never indicate gear set or geur neutral wen the gear is nog set or idle respectively (p15)
When the engine is regulating on torque, the clutch is closed (p16)

\paragraph{Aandachtspunten}
\begin{enumerate}
\item Voorrang tussen schepen onderling in de sluis?
\item Hoe lang mag een schip zich in de sluis bevinden?
\end{enumerate} 




\subparagraph{Afbakening}
\begin{itemize}
\item Wat doet de sluis niet.
\item De sluiss houdt geen rekening met links of rechtsrijdend verkeer vanuit de zeevaart
\item De sluis heeft geen queue met daarin een id gekoppeld aan de sluis.
\item De waterpomp wordt alleen aan en uitgezet
\item De waterpomp houdt geen rekening met waterstand
\item Houdt geen rekening met een schip in de sluis dat is blijven hangen.

\end{itemize}

\paragraph{Functionele en niet-functionele eisen}

\paragraph{specificaties}

\paragraph{Het vier variabelen model van de sluis}
Systemen (met daarin software) en de bijbehorende vier variabelen:
\subparagraph{Monitored variabelen}
: door sensoren gekwanticeerdefenomenen uit de omgeving
\subparagraph{Controlled variabelen}
door actuatoren bestuurde fenomenen uit de omgeving
\subparagraph{Input variabelen}
\subparagraph{Output variabelen}







Op basis van de schets kunnen we vaststellen dat een sluismodel uit de volgende onderdelen bestaat.

\begin{enumerate}
\item Een tweetal sluisdeuren. 
\item Een sluiskolk waarin de schepen in- enuitvaren
\item een stoplicht om een signaal af te geven voor invaren en uitvaren.
\item Een nivelleermachine zorgt ervoor dat het water in de sluis op het gewenste niveau wordt gebracht
\item Een control-system dat ervoor zorgt dat de opdrachten van de sluisbeheerder (geautomatiseerd) worden uitgevoerd
\end{enumerate}

Een schip komt aanvaren en meld zich aan bij de sluismeester. De sluismeester geeft een signaal aan het controlsystem voor het openen van de sluisdeuren, nadat geccontroleerd is of de nivelleermachine al klaar is. Als er ruimte is voor een invarend schip mag het schip dat zoich heeft aangemeld en toestemming heeft  in de sluis varen. Op het moment dat de sluis vol is gaan de sluisdeuren dicht. Eenmaal afgesloten kan de nivelleermachine beginnen om het water in de sluiskolk op het gewenste waterpeil te brengen. Als dit nivelleerprces is afgerond geeft  het controlsystem daan da de sleusdeuren open kunnen.  Als de sleusdeuren open zijn en het uitvaarsignaal is op groen dan moet het schip in de sluis de sluis uitvaren.

Uit het zojuist genoemnde scenario valt het volgende op te maken.
\begin{enumerate}
\item Een schip geeft een signaal aan een sluismeester.
\item Er wordt gekeken of er wel plek is in de sluis .
\item Er wordt gekeken of de nivelleermachine is afgerond.
\item Er wordt gekeken wat het niveo van de waterpeil in de sluiskolk is.
\item Er wordt gekeken of de sluisdeuren gereed zijn voor invarende schepen.
\end{enumerate}







4.2 5 en 6
Het Sluisbeheeerder model wordt getoond in fuguur[]. Het model is een uitbreiding van een schutsluis met alle condities en effecten. De kleuren in de automation verwijizen naar de kleuren in de staat van de automata . De template begint met een initiele lokatie start. De sluisbeheerder initieert het proces door een aangekomen schip te registreren metbehulp van een sychronizate met het channel... over de edge richtng de lokatie "aanmelden." Dit symboliseert een opstartprocedure, ook wordt een functie enqueeu_aanmeldLijst() gebruikt om de juiste waarden te geven aan lokale en globale avariabelen. De lokatie aanmelden regisseert het opstellen van schepen boven of beneden van de sluiskolk. De template Schip synchronizeerd met de template Sluisbeheerder met het channel move_down[id] of move_up[id] en bereikt daarmee de volgende lokatie afhankelijk af de sluis boven of beneden is worden de schepen die in de opstellijst voorkomen, max 2, klaargemaakt voor invaren.. De templates Stoplicht en sluisdeur synchroniseren met de channels ... call_Deur en call_stoplicht.
Het Sluisbeheerder model gebruikt de variabelen clock x, wachttijd_beneden, wachttijd_boven als invariant tussen de lokaties. Om op de hoogete te zijn van de invaar-/uitvaart van de verschillende schepen worden lijsten bijgehouden: list_wachtrij_beneden, list_pos_invaren_beneden, list_schepenInSluis, list_wachtrij_boven en list_pos_invaren_boven.

Het model voltooit de volgende transitie op basis van de waarde van de boolean $sluis_bove$ en $sluis_beneden$. en de lokale klok variabele x.
Vanaf de locatie invaarverbod_gecontroleerd  wordt gecontroleerd of er nog invarende schepen zijn die in de sluiskolk passen.
Op de lokatie sluiskolk gereed zijn er 1 of meer schepen in de sluis. Als er nog plek is in de sluiskolk n er is nog een schip klaar om in te varen dan wordt dit gecontroleerd, de functie enqueu() voegt het schip toe aan de queue van de sluiskolk. De functie deque() verwijdert de schip van de lijst met invarende schepen. De variabele sluis_boven of sluis_beneden is waar, bij de switch voor het sluiten van deuren en het aanroepe van het stoplicht nr gelang de positie van  de laate binnenvarende schip (boven of beneden). Hierna bereikt de automation sluiskolk_afgesloten.



De lokatie start_nivelleren kiest op basis van de variabelen sluis_boven en de variabelen sluis_beneden het nivellereingsprogramma.
Heet nivellereingsprogramma is Aof B. De keuze voor het programma wordt bepaald door de variabelen van het schip dat in de sluis zit.

De lokatie klaarmaken_voor_openen wordt bereikt als de   hoogte van de sluis  door het nivellereingsprogramma is bereikt.
De positie van de kluis is bepaald door de schepen in de sluis. Vanuit deze lokatie wordt gekeken off de stoplichten gereed moeten worden gemaakt en of de sluisdeuren open mogen.
Hierna volgt een transiie waarin de stoplichte op groen worden gezet en de sluisdeuren worden geopend voor de uitvaart van de schepen in de sluis.
Als alle schepen zijn uitgevaren die uit moeten varen, worden de stoplichten op groen gezet en de deuren gesloten.


De lokatie uitvaren_toegestaan heeft een verbinding(edge) met de lokatie sluis_afsluiten.
Er is een select statement, e:id_t gebruikt als onderdeel van het prototocol om alle uitvarende schepen uit de queue van de sluiskolk te halen, en wordt dan ook gebruikt door de synchronisatie met de channel leave om de schepen uit de sluiskolk te begeleiden. De edge hieraan gekoppeld bevat de functie deque() om de variabelen  van de sluiskolk te resetten.

Vanuit de positie van de sluis worden de schepen gesignaleerd op een invaarverbod en worden de deuren van de sluis gesloten.
De lokatie sluiskolk_afgesloten is bereikt.

Ship [guards, invariants, assignents, synchronizations, properties,aannames]
De template Schip begint bij de Init lokatie. De lokatie is verbonden met de lokatie aangekomen met een edge waarbij een synchronizatie wordt aangeroepen met de template sluisbeheerder. De clock wordt op nul gezet. De lokatie aangekomen is verbonden met de lokatie aangemeld. De edge bevat een synchronizatie waarmee de edge een synchronizatie uitvoert met de template Sluisbheheerder.
De volgende lokatie is  controleren. De edge waarmee de lokatie aangemeld in verbinding staat met de lokatie cnotroleren heeft een synchronisatie voor de template Sluisbeheerder. De lokatie controleren heeft ook een edge met de lokatie wachten. Een schip max maximaal 30 seconden wachten op de lokatie wachten voordat er een mogelijkheid is om opniew in aanmerking te komen voor een controle. Als een schip langer dan 30 tijdseenheden moet wachten de is er een mogelijkheid voor het schip te vertrekken. Hierbij eindigt het schip het invaarproces. Een schip kan dus na aanvaren maximaal 20 seconden wachten om toestemming te krijgen voor een positie invaren anders wordt deze verwezen naar een wachtrij.
Hierna volgdde lokate invarene. De lokatie invarene implieert dat een schip in een invaarproces is dat eindigt in de lokatie gestopt.
Hierop volgd de lokatie nivelleer_start. Hierop wordt een nivelleer_proces gestart. Daarbij is ee synchronisatie met de template Sluisbeheerder.
De lokatie nivelleer_stop is een lokate waarin het nivelleerproces al is gestopt. Van hieruit is er een edge met de lokatie klaar voor vertrek. De edge synchroniseert hiermee met de template Sluisbeheerder.
De lokatie klaar_voor_vertrek is verbonden met de lokatie Init. Met een guard x>=3 tijdseenheden mag een schip vertrekken.


Deur
De deur bevat de volgende lokaties: dicht, openend, open en sluitende.
Een deur sluit niet in een enkele actie. Het proces die een deur dooploopt zijn de processen openend en sluitende. De finale lokaties zijn open en dicht.

Nivelleermachine
De nivelleermachine begint bij de lokatie uit. Met een synchronisatie wordt een nivelleermachine aangezet. De automatie kiest een programma en werkt deze uit in de lokatie bezig. Als ht programma is afgerond volgt de lokatie klaar. Na elk nivelleerproces wordt de machine uitgezet

Stoplicht
Een stoplicht heeft twee lokaties: rood en groen.




\paragraph{Liveness}
Liveness properties are of the formn: something will eventually happen, e.g. when pressing the on button of the remote control of the television, then eventually the television should turn on. Or in a model of a  communication protocol, any message that has been sent should eventually be received.
\paragraph{Fairness}
\paragraph{Security}
Safety propertires are of the form: "something bad will never happen". For instance, in a model of a nuclear power plant, a safety propertymight be, that the operating temperature is always (invariantly) under a certain threshold, or that a meltdown never occurs. A variation of this property is that "something will possibly never happen".
For instance when playing a game, a safe state is one in which we can still win he game, hence we will possible not loose.
The system cannot reach states or enable events that are fornidden by the requirements
\paragraph{Performance}
There requirements limit the maximum time to perform when no recoverable errors occur.




