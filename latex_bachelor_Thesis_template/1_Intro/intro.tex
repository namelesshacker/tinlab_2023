\chapter{Inleiding}
\label{chapter:Inleiding}
\thispagestyle{myheadings}



%%%%%%%%%%%%%%%%%%%%%%%%%%%%%%%%%%%%%%%%%%%%%%%%%%%%%%%%%%%%%%%%%
	\newpage
\section{Inleiding}
In deze case study wordt %insert inleiding hier

\subsubsection{Algemeen}

Het ministerie van verkeer en Waterstaat wil in het kader van het klimaatakkoord en onderzoek laten uitvoeren naar de staat van het sluizenpark in Nederland. Het onderzoek moet zich richten op het ontwerpen en ontwikkelen van een geautomatiseerd sluismodel dat geschikt is voor een brede toepassing. In het onderzoek moet naar voren komen wat de huidige staat is van de sluizen met oog op veiligheid, efficiëntie, capaciteit, onderhoud, duurzaamheid en automatisering. Het onderzoek geeft aan hoe een volledig model worden opgeleverd opdat ontwerp van verschillend volledig geautomatiseerde sluizen in de toekomst geautomatiseerd kunnen worden.  

\subsubsection{Recente ontwikkelingen op het gebied van sluisautomatisering}

Het ministerie van verkeer en Waterstaat wil in het kader van het klimaatakkoord en onderzoek laten uitvoeren naar de staat van het sluizenpark in Nederland. Het onderzoek moet zich richten op het ontwerpen en ontwikkelen van een geautomatiseerd sluismodel dat geschikt is voor een brede toepassing. In het onderzoek moet naar voren komen wat de huidige staat is van de sluizen met oog op veiligheid, efficiëntie, capaciteit, onderhoud, duurzaamheid en automatisering. Het onderzoek geeft aan hoe een volledig model worden opgeleverd opdat ontwerp van verschillend volledig geautomatiseerde sluizen in de toekomst geautomatiseerd kunnen worden.  
\subsubsection{Wat is een sluis}

\subsubsection{Wat worrdt er omschreven en wat is er geleerd}

\subsubsection{Wat is uppaal}

Wat is Uppaal
Uppaal is an integrated tool environment for modeling, simulation and verification of real-time systems, developed jointly by Basic Research in Computer Science at Aalborg University in Denmark and the Department of Information Technology at Uppsala University in Sweden. It is appropriate for systems that can be modeled as a collection of non-deterministic processes with finite control structure and real-valued clocks, communicating through channels or shared variables [WPD94, LPW97b]. Typical application areas include real-time controllers and communication protocols in particular, those where timing aspects are critical.


model checking

Wat is statistical model checking?
Dit verwijst naar verschillende technieken dfie worden gebruikt voor de monitoring van een systeem. Daarbij wordt vooral gelet op een specifieke eigenschap. Met de resultaten van de statsitieken wordt de juistheid van een ontwerp beoordeeld. Statistisch model checking wordt onder andere toegepast in systeembiologie, software engineering en industriele toepassingen.
https://www-verimag.imag.fr/Statistical-Model-Checking-814.html?lang=en#:~:text=Statistical%20Model%20Checking%20(SMC)%20is,from%20state%20space%20explosion%20issues.

Model Checking (MC) [BK08,CGP99] is a widely recognized approach to guarantee correctness of a system. The technique relies on algorithms that check whether all executions of a system satisfy some properties stated in a specification logic. If this is the case, then the system is correct, else a bug is reported.
First implementations of model checking suffered from so-called state space explosion problems and could only be applied to small academic models. New techniques build on symbolic data structures and/or heuristics that make them capable of analyzing large-size systems that are part of our daily life
lassical model checking techniques are Boolean (either the system satisfies a property or it does not). Unfortunately such a view is extremely sensitive to changes made in the design and is not able to quantify their impacts (both minor and major changes may reverse the verification outcome).  This view is now obsolete: the designers need a finer analysis that allows to quantify the impacts of any change in the design. This has motivated the development of a series of new techniques (under the name of Probabilistic Model Checking) and tools [PRISM,BK08] capable of quantifying the likelihood for a system (whose behaviors naturally depend on stochastic information) to satisfy some property.  Adding explicitly rich features (e.g., real time) in specifications is also needed. Indeed, in many situations it is not enough to know whether something will or will not happen; rather, one needs to have a precise estimate of the time when some situation will arise. This motivated the creation of a number of new techniques under the name of timed model checking. 
The problem with MC-based approaches is that even though heuristics exist (partial order, symbolic approach, BDDs, etc.), they still suffer from the state-space explosion problem. This is especially the case when the system is obtained as the combination of several subsystems. Moreover, when moving to rich systems such as those with real time features, most of the model checking problems become undecidable.

https://project.inria.fr/plasma-lab/statistical-model-checking/
https://ris.utwente.nl/ws/portalfiles/portal/28200786/A_statistical_model_checker.pdf
https://dl.acm.org/doi/10.1145/3158668
https://dl.acm.org/doi/pdf/10.1145/3158668

Waarom gebruiken we statistisch model checking?
To overcome the above difficulties we propose to work with Statistical Model Checking [KZHHJ09,You05,You06,SVA04,SVA05,SVA05b] an approach that has recently been proposed as an alternative to avoid an exhaustive exploration of the state-space of the model. The core idea of the approach is to conduct some simulations of the system, monitor them, and then use results from the statistic area (including sequential hypothesis testing or Monte Carlo simulation) in order to decide whether the system satisfies the property or not with some degree of confidence. By nature, SMC is a compromise between testing and classical model checking techniques. Simulation-based methods are known to be far less memory and time intensive than exhaustive ones, and are oftentimes the only option. 
https://project.inria.fr/plasma-lab/statistical-model-checking/

Alternatief
Alternatieven voor Uppaal zijn Asynchronous Events,Vesta en MRMC.
%%%%%%%%%%%%%%%%%%%%%%%%%%%%%%%%%%%%%%%%%%%%%%%%%%%%%%%%%%%%%%%%%

\subsubsection{Probleemanalyse}

Na grondige analyse van het Nederlandse sluizenpark is gebleken dat renovatie van een groot aantal sluizen noodzakelijk is.  Uit een eerste verkenning is gebleken  dat het gecombineerd renoveren en automatiseren van het Nederlandsesluizenpark een aanzienlijke verbetering kan opleveren t.a.v. 
Op  het  ministerie  van  infrastructuur  enwaterstaat is helaas onvoldoende kennis van ict en systemen aanwezig om eenen ander uit te voeren 

\subsubsection{Waarom nu}
In  het  kader  van  het  onlangs  afgesloten  klimaatakkoord  heeft  de  Nederlandseoverheid  daarom  besloten  over  te  gaan  tot  een  ingrijpende  renovatie  van  dediverse  sluizen  die  ons  land  rijk  is.     

\subsubsection{Gewenst resultaat }


Wij vragen u een model (of een onderling samenhangend aantal modellen)aan  te  leveren,  opdat  ontwerpen  van  verschillende,  volledig  geautomatiseerdesluizen in de toekomst gerealiseerd kunnen worden. 
Zoals  gesteld  in  de  brief  is  het  de  bedoeling  dat  een  sluis  gemodelleerd  wordten  dat  bewezen  kan  worden  dat  de  te  bouwen  sluis  een  aantal  eigenschappenbezit.  

\subsubsection{Scope}

He gaat om het simuleren van een geautomatiseerde sluis. Wat voor type sluis wordt niet gemeld en ook niet uit welke onderdelen. Belangrijk is dat het model werkt en dat het voldoet aan de eisen die gebaseerd zijn op basis van literatuuronderzoek, observatie, interviews, brainstorming of een andere vorm van requirements elicitation.

\subsubsection{Onderzoeksvragen }

Hoe kan een geautomatiseerde sluis worden gemodeleerd met oog op ontwikkel- en onderhoudskosten,veiligheid, efficientie en capaciteit





\begin{enumerate}

	\item Welke requirements en kwaliteitseisen komen naar voren bij de analyse van een rampenonderzoek
	\item Welke veiligheidseisen er zijn voor sluizen in nederland. 
	\item Hoe kan in uppaal  een model worden getest dat voldoet aan de requirements/eisen volgens het rampenonderzoek?
\end{enumerate}

%%%%%%%%%%%%%%%%%%%%%%%%%%%%%%%%%%%%%%%%%%%%%%%%%%%%%%%%%%%%%%%%%


\subsubsection{Design goals}
Het systeem moet minimaal aan de volgende prestatie eisen voldoen 

\begin{enumerate}
	\item  
	\begin{enumerate}
		\item Requirements gebaseerd op rampenanalyse
	\end{enumerate}
	\item Data
	\begin{enumerate}
		\item Model testbaar in upaal
	\end{enumerate}
	
\end{enumerate}

%%%%%%%%%%%%%%%%%%%%%%%%%%%%%%%%%%%%%%%%%%%%%%%%%%%%%%%%%%%%%%%%%
\subsubsection{Welke aanpak is gekozen en welke studies liggen hieraan ten grondslag?}
https://link.springer.com/article/10.1007/s10626-020-00314-0

\subsubsection{Leeswijzer}
In  de methodologie wordt de lezer uitgelegd met welke methoden de onderzoeksvragen zijn beantwoord. In het hoofdstuk Onderzoek worden alle resultaten behandeld die naar voren zijn gekomen bij het deskresearch. De analyse van de verzamelde data wordt gedaan in het hoofdstuk analyse. Hierin wordt behandeld zoekopdracht naar IoT cloud platforms, feature extractie, prijs-berekening en prijs-feature vergelijking. In het ontwerp komen de uml diagrammen en systeemschetsen naar voren. In de  de hoofdstukken Prototype, IoT cloud en Firmware wordt de implementatie behandeld van het IoT cloud platform in een bestaand project.

%%%%%%%%%%%%%%%%%%%%%%%%%%%%%%%%%%%%%%%%%%%%%%%%%%%%%%%%%%%%%%%%%



%	\input{Methoden} %(verplicht) hoofdverslag
%\hoofdstuk{methoden}