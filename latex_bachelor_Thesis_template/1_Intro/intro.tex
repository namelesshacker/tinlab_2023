\chapter{Inleiding}
\label{chapter:Inleiding}
\thispagestyle{myheadings}



%%%%%%%%%%%%%%%%%%%%%%%%%%%%%%%%%%%%%%%%%%%%%%%%%%%%%%%%%%%%%%%%%


\subsubsection{Algemeen}

Het ministerie van verkeer en Waterstaat wil in het kader van het klimaatakkoord en onderzoek laten uitvoeren naar de staat van het sluizenpark in Nederland. Het onderzoek moet zich richten op het ontwerpen en ontwikkelen van een geautomatiseerd sluismodel dat geschikt is voor een brede toepassing. In het onderzoek moet naar voren komen wat de huidige staat is van de sluizen met oog op veiligheid, efficiëntie, capaciteit, onderhoud, duurzaamheid en automatisering. Het onderzoek geeft aan hoe een volledig model worden opgeleverd opdat ontwerp van verschillend volledig geautomatiseerde sluizen in de toekomst geautomatiseerd kunnen worden.  

\subsubsection{Probleemanalyse}

Na grondige analyse van het Nederlandse sluizenpark is gebleken dat renovatie van een groot aantal sluizen noodzakelijk is.  Uit een eerste verkenning is gebleken  dat het gecombineerd renoveren en automatiseren van het Nederlandsesluizenpark een aanzienlijke verbetering kan opleveren t.a.v. 
Op  het  ministerie  van  infrastructuur  enwaterstaat is helaas onvoldoende kennis van ict en systemen aanwezig om eenen ander uit te voeren 

\subsubsection{Waarom nu}
In  het  kader  van  het  onlangs  afgesloten  klimaatakkoord  heeft  de  Nederlandseoverheid  daarom  besloten  over  te  gaan  tot  een  ingrijpende  renovatie  van  dediverse  sluizen  die  ons  land  rijk  is.     

\subsubsection{Gewenst resultaat }


Wij vragen u een model (of een onderling samenhangend aantal modellen)aan  te  leveren,  opdat  ontwerpen  van  verschillende,  volledig  geautomatiseerdesluizen in de toekomst gerealiseerd kunnen worden. 
Zoals  gesteld  in  de  brief  is  het  de  bedoeling  dat  een  sluis  gemodelleerd  wordten  dat  bewezen  kan  worden  dat  de  te  bouwen  sluis  een  aantal  eigenschappenbezit.  

\subsubsection{Scope}

He gaat om het simuleren van een geautomatiseerde sluis. Wat voor type sluis wordt niet gemeld en ook niet uit welke onderdelen. Belangrijk is dat het model werkt en dat het voldoet aan de eisen die gebaseerd zijn op basis van literatuuronderzoek, observatie, interviews, brainstorming of een andere vorm van requirements elicitation.

\subsubsection{Onderzoeksvragen }






\begin{enumerate}
	\item Uit het onderzoek zal moeten blijken welke veiligheidseisen er zijn voor sluizen in nederland. 
	\item Daarnaast welke factoren een rol spelen in de duurzaamheid van het sluizenpark.  
	\item Hoe wordt de routinecontrole op de sluizen uitgevoerd?  
	\item Welke automatisering is mogelijk met oog op veiligheid, efficientie en capaciteit?  
	\item Welke criteria wegen zwaar in de ontwikkel- en onderhoudskosten van duurzame technologie?
\end{enumerate}


%%%%%%%%%%%%%%%%%%%%%%%%%%%%%%%%%%%%%%%%%%%%%%%%%%%%%%%%%%%%%%%%%



\section{Leeswijzer}
In  de methodologie wordt de lezer uitgelegd met welke methoden de onderzoeksvragen zijn beantwoord. In het hoofdstuk Onderzoek worden alle resultaten behandeld die naar voren zijn gekomen bij het deskresearch. De analyse van de verzamelde data wordt gedaan in het hoofdstuk analyse. Hierin wordt behandeld zoekopdracht naar IoT cloud platforms, feature extractie, prijs-berekening en prijs-feature vergelijking. In het ontwerp komen de uml diagrammen en systeemschetsen naar voren. In de  de hoofdstukken Prototype, IoT cloud en Firmware wordt de implementatie behandeld van het IoT cloud platform in een bestaand project.





%	\input{Methoden} %(verplicht) hoofdverslag
%\hoofdstuk{methoden}