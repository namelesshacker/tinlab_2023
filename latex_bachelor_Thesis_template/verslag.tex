%%%%%%%%%%%%%%%%%%%%%%%%%%%%%%%%%%%%%%%%%%%%%%%%%%%%%%%%%%%%%%%%%%%%%%%%%%%%%
%eventueel dutch in english veranderen, eventueel twoside verwijderen       %
%%%%%%%%%%%%%%%%%%%%%%%%%%%%%%%%%%%%%%%%%%%%%%%%%%%%%%%%%%%%%%%%%%%%%%%%%%%%%
\documentclass[11pt,a4paper,twoside,dutch]{report}%
\usepackage[dutch]{babel}%
\selectlanguage{dutch}%
%%%%%%%%%%%%%%%%%%%%%%%%%%%%%%%%%%%%%%%%%%%%%%%%%%%%%%%%%%%%%%%%%%%%%%%%%%%%%
\newif\ifpublic \newif\iftwocol %nodig voor publicatiemode, niet wijzigen!  %
%%%%%%%%%%%%%%%%%%%%%%%%%%%%%%%%%%%%%%%%%%%%%%%%%%%%%%%%%%%%%%%%%%%%%%%%%%%%%
%publiceerbare versie (geen titelblad, bedanktuiging, evaluatie en bijlagen)%
%%%%%%%%%%%%%%%%%%%%%%%%%%%%%%%%%%%%%%%%%%%%%%%%%%%%%%%%%%%%%%%%%%%%%%%%%%%%%
%\publictrue        %1-kolom publicatiemode (zonder %) of verslagmode (met %)
%\twocoltrue        %2-kolommen publicatiemode
%%%%%%%%%%%%%%%%%%%%%%%%%%%%%%%%%%%%%%%%%%%%%%%%%%%%%%%%%%%%%%%%%%%%%%%%%%%%%
%nodig voor template: packages en macro's                                   %
%%%%%%%%%%%%%%%%%%%%%%%%%%%%%%%%%%%%%%%%%%%%%%%%%%%%%%%%%%%%%%%%%%%%%%%%%%%%% 
\ifpublic\iftwocol\twocolumn\fi\fi  %vlag 1 of 2 kolommen                   %
\usepackage[utf8]{inputenc}         %aanbevolen encoding, kies 1 van de 3
%\usepackage[utf-8]{inputenc}      %sommige Ubuntu en Window distributies
%\usepackage[latin1]{inputenc}     %oudere LaTeX-distributies
\usepackage{modelverslag}           %verplicht voor de template             %
\renewcommand{\redactie}[1]{\relax} %nodig voor redactionele opmerkingen    %
%%%%%%%%%%%%%%%%%%%%%%%%%%%%%%%%%%%%%%%%%%%%%%%%%%%%%%%%%%%%%%%%%%%%%%%%%%%%%
%hieronder komen alle packages die je zelf wil plaatsen of veranderen       %
\usepackage[most]{tcolorbox}
\usepackage{enumitem}
\usepackage{lipsum}

\colorlet{helpful}{lime!70}
\colorlet{harmful}{red!30}
\colorlet{internal}{yellow!20}
\colorlet{external}{cyan!30}
\colorlet{S}{helpful!50!internal}
\colorlet{W}{harmful!50!internal}
\colorlet{O}{helpful!50!external}
\colorlet{T}{harmful!50!external}

\newcommand{\texta}{Helpful}
\newcommand{\textb}{Harmful}
\newcommand{\textcn}{\rotatebox{90}{\parbox[c]{3cm}{\centering Internal origin}}}
\newcommand{\textdn}{\rotatebox{90}{\parbox[c]{3cm}{\centering External origin}}}

\newcommand{\texts}{strength 1\par strength 2}
\newcommand{\textw}{weakness 1\par weakness 2}
\newcommand{\texto}{opportunity 1\par opportunity 2}
\newcommand{\textt}{threat 1\par threat 2}

\tcbset{swotbox/.style={size=normal, boxrule=0pt,
		colback=#1, watermark text=#1, width=.5\linewidth-5mm},
	header/.style={size=normal, boxrule=0pt, width=.5\linewidth-5mm, colback=#1, valign=center, halign=center},
	firstcol/.style={header=#1, width=1cm, boxsep=0mm}
}


%\usepackage{tikzlibrary}{automata,positioning}

\usepackage[all]{xy}
\usepackage{dot2texi}
%\usepackage{tikz}
\usetikzlibrary{cd}
\usetikzlibrary{graphdrawing.trees}
\usetikzlibrary{graphs}
\usetikzlibrary{graphdrawing}
\usetikzlibrary{arrows,automata,positioning}
\usetikzlibrary{automata,arrows,trees,positioning,shapes,calc}
%\usetikzlibrary {arrows.meta,automata,positioning}

%\usegdlibrary{trees}
\usepackage[svgnames, table]{xcolor}
%\usepackage[utf8]{inputenc}
\usepackage[T1]{fontenc}
\usepackage[ngerman]{babel}
%\usepackage[automark,headsepline,plainheadsepline, plainfootsepline, footsepline]{scrlayer-scrpage}
%\pagestyle{scrheadings}

%Tabelle
\usepackage{multirow, makecell, cellspace, bigstrut}
\usepackage{ragged2e}
\usepackage{ltablex}
\keepXColumns
\renewcommand\tabularxcolumn[1]{ >{\arraybackslash}m{#1}}
\usepackage{booktabs}
\newcolumntype{L}[1]{>{\RaggedRight}p{#1}}
\newcolumntype{C}[1]{>{\centering\arraybackslash}p{#1}}
\newlength{\lastcolwd}
\settowidth{\lastcolwd}{Bermerkungen}
\usepackage[inner=2.5cm,outer=2.5cm,top=2cm,bottom=1.0cm,includeheadfoot]{geometry}
\newcommand\mymidrule{\specialrule{\lightrulewidth}{0pt}{\belowrulesep}}
\newcommand\mybottomrule{\specialrule{\heavyrulewidth}{0pt}{\belowrulesep}}
\newcommand\mytoprule{\specialrule{\heavyrulewidth}{0pt}{0pt}}
\usepackage{amsmath}
\setlength\cellspacetoplimit{5pt}
\setlength\cellspacebottomlimit{5pt}
%%%%%%%%%%%%%%%%%%%%%%%%%%%%%%%%%%%%%%%%%%%%%%%%%%%%%%%%%%%%%%%%%%%%%%%%%%%%%
\usepackage{algorithmic}      %aangepaste algoritmen   
\usepackage{listings}         %aangepaste listings (event. in kleuren)
%\usepackage{}
%\usepackage{}

\usepackage[svgnames, table]{xcolor}
\usepackage[utf8]{inputenc}
\usepackage[T1]{fontenc}

\begin{document}
\titelblad                   %argumenten invullen
  {Persoonlijk verslag} %hier titel invullen 
  {Galvin Bartes (0799967)} %hier naam en (stamnummer) van student invullen
  {\TI}                      %hier opleiding \TI, \INF of \BI invullen
  {\today}                   %{\today} concept of {1 april 2011} definitief
  {Dhr. W. Oele}           %eerste docent
  {}                         %tweede docent
% als er meer auteurs zijn, dan wordt de tweede parameter:
% {Onbekend Talent1 (123456)\\Onbekend Talent2 (987654)\\ ...} 
\ifpublic
  {\footnotesize{\samenvatting


Introduction: Whilst every study published in ascientific journal contains an abstract, little researchhas been done on the exact format, content andstyle with which an abstract should be written. Thismakes it difficult for authors to adequately sum-marise their work in an abstract.
Methods: In this study, the authors recruited acohort of medical students who had written at leastone scientific paper. Students were anonymouslysurveyed, on their confidence writing abstracts us-ing an online survey, maintaining confidentiality.However, this method may have been subjected toselection bias, where those who have completed ab-stracts but not written a full scientific paper may beexcluded. Use of online surveys may also contributeto selection bias, based on the fact that subject par-ticipation is voluntary and particular characteristicse.g. access to internet, whether the students viewthe site/email providing access to the questionnaire,time available for completion, etc., may differ perindividual and hence reduce the representativenessof the sample regarding the medical student popu-lation (The Writing Centre & University of NorthCarolina at Chapel Hill, n.d.).
Results:  73 students responded and the studyshowed that 37 % of students surveyed rated theirconfidence writing abstracts as ‘very poor,’ with afurther 42 % rating their confidence as ‘poor.’
Discussion: Based on the author’s results, it isclear that students need more guidance on how towrite abstracts. The authors recommend that allstudents wishing to learn how to write an abstractread the National Student Association for MedicalResearch ‘Anatomy of an Abstract’ article. How-ever, further controlled studies should be done toeliminate biases attributed to methodology in thiscohort study to truly determine whether medicalstudents lack confidence in writing abstracts.References:1. Nulty, D. D. (2018) The adequacyof response rates to online and paper surveys: whatcan be done?Assess Eval High Educ, 33(3), 301-14.doi: 10.1080/02602930701293231

Background: The writing and publication of re-search material by medical students is an area thatoccupies the time and efforts of the students them-selves, but does not yet have a large evidence base.Purpose: Consequently, it is important to under-take research that expands this body of knowledge.
Focus: This review aims to assess the confidenceof medical students in writing up abstracts for theirresearch, to gain a better overall picture of medicalstudents’ feelings about undertaking and writing upresearch.Word count: 81

Informative Abstract
Structured abstract includes the following heads: 
• Objectives: Illustrate the background and purpose of the review in one or two sentences in present tense.
• Material and Methods: Write a few lines to present a general picture of the research methodology of article in past tense.
• Result: Describe outcomes in few sentences. 

Abstract
There are two types of abstracts: one is informative abstract which describes the planned end product and result of the review manuscript or specifies the text structure. Second is descriptive abstract which describes the covered subject without specific details. Present tense will be used in the writing. Usually the length of abstract is 200 to 250 words.


Critical abstract
A critical abstract is generally written about a different au-thor’s work and contains all of the information mentionedabove, but also an element of evaluation or critical appraisalof the study, which may include discussion of the reliabilityand validity of the results (Labaree, 2018). For this purpose,references can be included to provide supporting evidence foryour arguments from relevant literature.The critical abstract includes information regarding thearticle e.g. author, title etc. and then briefly provides theirkey findings/conclusion. The main content of the abstractthen highlights the positives and negatives of the article.Examples of things to consider here could include:
•How relevant is this research question?
•Is the hypothesis clearly stated?
•Type of study/trial/research?
•What is the sample size? Is it large enough to providestatistically significant findings?
•Were the methods used appropriate and justified? Couldthey be improved?
•Is the conclusion valid based on the evidence?
•Are there any conflicts of interests?

Keywords

}}
\else
  \pagenumbering{roman}
  \samenvatting


Introduction: Whilst every study published in ascientific journal contains an abstract, little researchhas been done on the exact format, content andstyle with which an abstract should be written. Thismakes it difficult for authors to adequately sum-marise their work in an abstract.
Methods: In this study, the authors recruited acohort of medical students who had written at leastone scientific paper. Students were anonymouslysurveyed, on their confidence writing abstracts us-ing an online survey, maintaining confidentiality.However, this method may have been subjected toselection bias, where those who have completed ab-stracts but not written a full scientific paper may beexcluded. Use of online surveys may also contributeto selection bias, based on the fact that subject par-ticipation is voluntary and particular characteristicse.g. access to internet, whether the students viewthe site/email providing access to the questionnaire,time available for completion, etc., may differ perindividual and hence reduce the representativenessof the sample regarding the medical student popu-lation (The Writing Centre & University of NorthCarolina at Chapel Hill, n.d.).
Results:  73 students responded and the studyshowed that 37 % of students surveyed rated theirconfidence writing abstracts as ‘very poor,’ with afurther 42 % rating their confidence as ‘poor.’
Discussion: Based on the author’s results, it isclear that students need more guidance on how towrite abstracts. The authors recommend that allstudents wishing to learn how to write an abstractread the National Student Association for MedicalResearch ‘Anatomy of an Abstract’ article. How-ever, further controlled studies should be done toeliminate biases attributed to methodology in thiscohort study to truly determine whether medicalstudents lack confidence in writing abstracts.References:1. Nulty, D. D. (2018) The adequacyof response rates to online and paper surveys: whatcan be done?Assess Eval High Educ, 33(3), 301-14.doi: 10.1080/02602930701293231

Background: The writing and publication of re-search material by medical students is an area thatoccupies the time and efforts of the students them-selves, but does not yet have a large evidence base.Purpose: Consequently, it is important to under-take research that expands this body of knowledge.
Focus: This review aims to assess the confidenceof medical students in writing up abstracts for theirresearch, to gain a better overall picture of medicalstudents’ feelings about undertaking and writing upresearch.Word count: 81

Informative Abstract
Structured abstract includes the following heads: 
• Objectives: Illustrate the background and purpose of the review in one or two sentences in present tense.
• Material and Methods: Write a few lines to present a general picture of the research methodology of article in past tense.
• Result: Describe outcomes in few sentences. 

Abstract
There are two types of abstracts: one is informative abstract which describes the planned end product and result of the review manuscript or specifies the text structure. Second is descriptive abstract which describes the covered subject without specific details. Present tense will be used in the writing. Usually the length of abstract is 200 to 250 words.


Critical abstract
A critical abstract is generally written about a different au-thor’s work and contains all of the information mentionedabove, but also an element of evaluation or critical appraisalof the study, which may include discussion of the reliabilityand validity of the results (Labaree, 2018). For this purpose,references can be included to provide supporting evidence foryour arguments from relevant literature.The critical abstract includes information regarding thearticle e.g. author, title etc. and then briefly provides theirkey findings/conclusion. The main content of the abstractthen highlights the positives and negatives of the article.Examples of things to consider here could include:
•How relevant is this research question?
•Is the hypothesis clearly stated?
•Type of study/trial/research?
•What is the sample size? Is it large enough to providestatistically significant findings?
•Were the methods used appropriate and justified? Couldthey be improved?
•Is the conclusion valid based on the evidence?
•Are there any conflicts of interests?

Keywords

     %(verplicht) samenvatting
  \include{dankbetuiging}  %(optioneel) dankbetuiging
  \tableofcontents         %overzicht hoofdstukken en paragrafen
  \hoofdstuk{trefwoorden}

trefwoorden
volgens de gebruikte thauserus. een thesaurus is een lijst vna goedgekeurde en geaccepteerde vaktermen, de 'controlled descriptors' met de verklaring en met de afgekeurde alternatieve vaktermen    %(optioneel) overzicht afkortingen
  \pagenumbering{arabic}
\fi
\hoofdstuk{Inleiding}
 
 

\subsubsection{Algemeen}

Het ministerie van verkeer en Waterstaat wil in het kader van het klimaatakkoord en onderzoek laten uitvoeren naar de staat van het sluizenpark in Nederland. Het onderzoek moet zich richten op het ontwerpen en ontwikkelen van een geautomatiseerd sluismodel dat geschikt is voor een brede toepassing. In het onderzoek moet naar voren komen wat de huidige staat is van de sluizen met oog op veiligheid, efficiëntie, capaciteit, onderhoud, duurzaamheid en automatisering. Het onderzoek geeft aan hoe een volledig model worden opgeleverd opdat ontwerp van verschillend volledig geautomatiseerde sluizen in de toekomst geautomatiseerd kunnen worden.  

\subsubsection{Recente ontwikkelingen op het gebied van sluisautomatisering}

Het ministerie van verkeer en Waterstaat wil in het kader van het klimaatakkoord en onderzoek laten uitvoeren naar de staat van het sluizenpark in Nederland. Het onderzoek moet zich richten op het ontwerpen en ontwikkelen van een geautomatiseerd sluismodel dat geschikt is voor een brede toepassing. In het onderzoek moet naar voren komen wat de huidige staat is van de sluizen met oog op veiligheid, efficiëntie, capaciteit, onderhoud, duurzaamheid en automatisering. Het onderzoek geeft aan hoe een volledig model worden opgeleverd opdat ontwerp van verschillend volledig geautomatiseerde sluizen in de toekomst geautomatiseerd kunnen worden.  
\subsubsection{Wat is een sluis}

\subsubsection{Wat worrdt er omschreven en wat is er geleerd}

\subsubsection{Wat is uppaal}

Wat is Uppaal
Uppaal is an integrated tool environment for modeling, simulation and verification of real-time systems, developed jointly by Basic Research in Computer Science at Aalborg University in Denmark and the Department of Information Technology at Uppsala University in Sweden. It is appropriate for systems that can be modeled as a collection of non-deterministic processes with finite control structure and real-valued clocks, communicating through channels or shared variables [WPD94, LPW97b]. Typical application areas include real-time controllers and communication protocols in particular, those where timing aspects are critical.


model checking

Wat is statistical model checking?
Dit verwijst naar verschillende technieken dfie worden gebruikt voor de monitoring van een systeem. Daarbij wordt vooral gelet op een specifieke eigenschap. Met de resultaten van de statsitieken wordt de juistheid van een ontwerp beoordeeld. Statistisch model checking wordt onder andere toegepast in systeembiologie, software engineering en industriele toepassingen.
https://www-verimag.imag.fr/Statistical-Model-Checking-814.html?lang=en#:~:text=Statistical%20Model%20Checking%20(SMC)%20is,from%20state%20space%20explosion%20issues.

Model Checking (MC) [BK08,CGP99] is a widely recognized approach to guarantee correctness of a system. The technique relies on algorithms that check whether all executions of a system satisfy some properties stated in a specification logic. If this is the case, then the system is correct, else a bug is reported.
First implementations of model checking suffered from so-called state space explosion problems and could only be applied to small academic models. New techniques build on symbolic data structures and/or heuristics that make them capable of analyzing large-size systems that are part of our daily life
lassical model checking techniques are Boolean (either the system satisfies a property or it does not). Unfortunately such a view is extremely sensitive to changes made in the design and is not able to quantify their impacts (both minor and major changes may reverse the verification outcome).  This view is now obsolete: the designers need a finer analysis that allows to quantify the impacts of any change in the design. This has motivated the development of a series of new techniques (under the name of Probabilistic Model Checking) and tools [PRISM,BK08] capable of quantifying the likelihood for a system (whose behaviors naturally depend on stochastic information) to satisfy some property.  Adding explicitly rich features (e.g., real time) in specifications is also needed. Indeed, in many situations it is not enough to know whether something will or will not happen; rather, one needs to have a precise estimate of the time when some situation will arise. This motivated the creation of a number of new techniques under the name of timed model checking. 
The problem with MC-based approaches is that even though heuristics exist (partial order, symbolic approach, BDDs, etc.), they still suffer from the state-space explosion problem. This is especially the case when the system is obtained as the combination of several subsystems. Moreover, when moving to rich systems such as those with real time features, most of the model checking problems become undecidable.
\footnote{Hello this is unheard}
\cite{inriaStatsMoodCheck}
\cite{ buddeModelChecker}
\cite{AGHASuervey }


Waarom gebruiken we statistisch model checking?
To overcome the above difficulties we propose to work with Statistical Model Checking [KZHHJ09,You05,You06,SVA04,SVA05,SVA05b] an approach that has recently been proposed as an alternative to avoid an exhaustive exploration of the state-space of the model. The core idea of the approach is to conduct some simulations of the system, monitor them, and then use results from the statistic area (including sequential hypothesis testing or Monte Carlo simulation) in order to decide whether the system satisfies the property or not with some degree of confidence. By nature, SMC is a compromise between testing and classical model checking techniques. Simulation-based methods are known to be far less memory and time intensive than exhaustive ones, and are oftentimes the only option. 
https://project.inria.fr/plasma-lab/statistical-model-checking/

Alternatief
Alternatieven voor Uppaal zijn Asynchronous Events,Vesta en MRMC.
%%%%%%%%%%%%%%%%%%%%%%%%%%%%%%%%%%%%%%%%%%%%%%%%%%%%%%%%%%%%%%%%%

\subsubsection{Probleemanalyse}

Na grondige analyse van het Nederlandse sluizenpark is gebleken dat renovatie van een groot aantal sluizen noodzakelijk is.  Uit een eerste verkenning is gebleken  dat het gecombineerd renoveren en automatiseren van het Nederlandsesluizenpark een aanzienlijke verbetering kan opleveren t.a.v. 
Op  het  ministerie  van  infrastructuur  enwaterstaat is helaas onvoldoende kennis van ict en systemen aanwezig om eenen ander uit te voeren 

\subsubsection{Waarom nu}
In  het  kader  van  het  onlangs  afgesloten  klimaatakkoord  heeft  de  Nederlandseoverheid  daarom  besloten  over  te  gaan  tot  een  ingrijpende  renovatie  van  dediverse  sluizen  die  ons  land  rijk  is.     

\subsubsection{Gewenst resultaat }


Wij vragen u een model (of een onderling samenhangend aantal modellen)aan  te  leveren,  opdat  ontwerpen  van  verschillende,  volledig  geautomatiseerdesluizen in de toekomst gerealiseerd kunnen worden. 
Zoals  gesteld  in  de  brief  is  het  de  bedoeling  dat  een  sluis  gemodelleerd  wordten  dat  bewezen  kan  worden  dat  de  te  bouwen  sluis  een  aantal  eigenschappenbezit.  

\subsubsection{Scope}

He gaat om het simuleren van een geautomatiseerde sluis. Wat voor type sluis wordt niet gemeld en ook niet uit welke onderdelen. Belangrijk is dat het model werkt en dat het voldoet aan de eisen die gebaseerd zijn op basis van literatuuronderzoek, observatie, interviews, brainstorming of een andere vorm van requirements elicitation.

\subsubsection{Onderzoeksvragen }

Hoe kan een geautomatiseerde sluis worden gemodeleerd met oog op ontwikkel- en onderhoudskosten,veiligheid, efficientie en capaciteit





\begin{enumerate}
	
	\item Welke requirements en kwaliteitseisen komen naar voren bij de analyse van een rampenonderzoek
	\item Welke veiligheidseisen er zijn voor sluizen in nederland. 
	\item Hoe kan in uppaal  een model worden getest dat voldoet aan de requirements/eisen volgens het rampenonderzoek?
\end{enumerate}

%%%%%%%%%%%%%%%%%%%%%%%%%%%%%%%%%%%%%%%%%%%%%%%%%%%%%%%%%%%%%%%%%


\subsubsection{Design goals}
Het systeem moet minimaal aan de volgende prestatie eisen voldoen 

\begin{enumerate}
	\item  
	\begin{enumerate}
		\item Requirements gebaseerd op rampenanalyse
	\end{enumerate}
	\item Data
	\begin{enumerate}
		\item Model testbaar in upaal
	\end{enumerate}
	
\end{enumerate}

%%%%%%%%%%%%%%%%%%%%%%%%%%%%%%%%%%%%%%%%%%%%%%%%%%%%%%%%%%%%%%%%%
\subsubsection{Welke aanpak is gekozen en welke studies liggen hieraan ten grondslag?}
https://link.springer.com/article/10.1007/s10626-020-00314-0

\subsubsection{Leeswijzer}
In  de methodologie wordt de lezer uitgelegd met welke methoden de onderzoeksvragen zijn beantwoord. In het hoofdstuk Onderzoek worden alle resultaten behandeld die naar voren zijn gekomen bij het deskresearch. De analyse van de verzamelde data wordt gedaan in het hoofdstuk analyse. Hierin wordt behandeld zoekopdracht naar IoT cloud platforms, feature extractie, prijs-berekening en prijs-feature vergelijking. In het ontwerp komen de uml diagrammen en systeemschetsen naar voren. In de  de hoofdstukken Prototype, IoT cloud en Firmware wordt de implementatie behandeld van het IoT cloud platform in een bestaand project.

%%%%%%%%%%%%%%%%%%%%%%%%%%%%%%%%%%%%%%%%%%%%%%%%%%%%%%%%%%%%%%%%%



%	\input{Methoden} %(verplicht) hoofdverslag
%\hoofdstuk{methoden}
    %(verplicht) inleiding
%\hoofdstuk{materiaal}


materiaal
geef type:kwantiteit(aantal steekproeven, hoeveelhjeid per steekproef)


Material and Methods


data sources en references of data. 


research approach

inclusion and exclusion criteria of studies

how many studies have screened and included 

statistical process of meta-analysis.

Instrument development

Sample and data collection

Analytical method

Respondents’ demographic profile

Reliability and validity of research instrument

Result analysis

Examples
Type  

    %(verplicht) inleiding
%\hoofdstuk{methode}


standaardmethoden vermelden met codenummer
bij gewijzigde metho9de/apparaat; de wijziging precies aangeven




Research methodology

Instrument development

Sample and data collection

Analytical method

Respondents’ demographic profile

Reliability and validity of research instrument

Result analysis



\section{Methodologie}
\label{chapter:body}
\thispagestyle{myheadings}
Voor dit rapport is onderzoek gedaan naar sluizen, sluismodellen, rampen, rampenbestrijdingsprocedures en requiremeentsengineering.
%%%%%%%%%%%%%%%%%%%%%%%%%%%%%%%%%%%%%%%%%%%%%%%%%%%%%%%%%%%%%%%%%%%%%%%%%

\subsubsection{Literatuuronderzoek}
\begin{frame}{Literature Review}
	\begin{table}[htbp]
		\footnotesize
		
		\centering
		\begin{tabular}{|c|c|p{2in}|c|c|}\hline
			S.no&Author&Title&Findings&Gap in literature\\\hline
			S.no&Author&wanrooy \textunderscore vab1991a.pdf&Findings&Gap in literature\\\hline
			S.no&Author&wa3300-bezuien2000(1).pdf&Findings&Gap in literature\\\hline
			S.no&Author&Title&Findings&Gap in literature\\\hline
			S.no&Author&Title&Findings&Gap in literature\\\hline
			S.no&Author&rapport-veiligheid-van-op-afstand-bediende-burggen.pd&Findings&Gap in literature\\\hline
			S.no&Author&pronk.pdf&Findings&Gap in literature\\\hline
			S.no&Author&Olieman1987a.pdf&Findings&Gap in literature\\\hline
			S.no&Author&richtlijnen-vaarwegen-2020.pdf&Findings&Gap in literature\\\hline
			S.no&Author&richtlijnen-vaarwegen-2017 \textunderscore tcm21-127359(1).pdf&Findings&Gap in literature\\\hline
			S.no&Author&Olieman1987a.pdf&Findings&Gap in literature\\\hline
			S.no&Author&Meijer1980b.pdf&Findings&Gap in literature\\\hline
			S.no&Author&Meijer1980c.pdf&Findings&Gap in literature\\\hline
			S.no&Author&kst-31200-A-80-b2.pdf&Findings&Gap in literature\\\hline
			S.no&Author&duurzaamheid \textunderscore bij \textunderscore de \textunderscore ontwikkeling \textunderscore van \textunderscore reevesluis.pdf&Findings&Gap in literature\\\hline
			S.no&Author&De \textunderscore deltawerken \textunderscore Cultuurhistorie \textunderscore ontwerpgeschiedenis \textunderscore web-A.pdf&Findings&Gap in literature\\\hline
			S.no&Author&wa3300-Bezuijen2000.pdf&Findings&Gap in literature\\\hline
			S.no&Author&Sander van Alphen Haalbaarheidsstudie naar grote sluisdeuren uitgevoerd in hogesterktebeton.pdf&Findings&Gap in literature\\\hline
			S.no&Author&Dalmeijer1994a.pdf&Findings&Gap in literature\\\hline
			S.no&Author&Dalmeijer1994b.pdf&Findings&Gap in literature\\\hline
			S.no&Author&Dalmeijer1994c.pdf&Findings&Gap in literature\\\hline
			S.no&Author&ceg \textunderscore pruijssers \textunderscore 1982.pdf&Findings&Gap in literature\\\hline
			S.no&Author&Capaciteitsanalyse \textunderscore van \textunderscore de \textunderscore prinses\textunderscore margrietsluis \textunderscore in \textunderscore lemmer \textunderscore - \textunderscore Marc \textunderscore Lamboo.pdf&Findings&Gap in literature\\\hline
			S.no&Author&Boer1979a.pdf&Findings&Gap in literature\\\hline
			S.no&Author&bijlagerapport \textunderscore c \textunderscore - \textunderscore analyse \textunderscore geavanceerd-definitief \textunderscore v1 \textunderscore 0.pdf&Findings&Gap in literature\\\hline
			S.no&Author&Bijl1988a.pdf&Findings&Gap in literature\\\hline
			S.no&Author&Bentum1978a.pdf&Findings&Gap in literature\\\hline
			S.no&Author&Alphen.pdf&Findings&Gap in literature\\\hline
			S.no&Author&Abbenhuis1975a.pdf&Findings&Gap in literature\\\hline
			S.no&Author&Abbenhuis1974a.pdf&Findings&Gap in literature\\\hline
			S.no&Author&https://wiki.woudagemaal.nl/w/index.php/Sluizen&Findings&Gap in literature\\\hline
			S.no&Author&Title&Findings&Gap in literature\\\hline
			
		\end{tabular}
	\end{table}
	
\end{frame}			    %(verplicht) inleiding

 



\hoofdstuk{Theoretisch kader}

In het eerste hoofdstuk is duidelijk geworden wat de onderzoeksvraag is, namelijk ‘Hoe kan een geautomatiseerde sluis worden gemodeleerd met oog op ontwikkel- en onderhoudskosten,veiligheid, efficientie en capaciteit’. Door de toenemende complexiteit van systemen is het gebruik van modellen en de toepassing van timebased model checking  op industriele controle systemen een manier van modelleren van het systeem en de requirements zodat er een bijdagre kan worden geleverd aan de acceptatie van  simulatie-/modeltechniek voor de industrie.(‘https://link.springer.com/article/10.1007/s10626-020-00314-0’, 2020). Of dit ook het geval is bij het modellereren van sluizen is nu de vraag.

De bestudering van rampen aan de hand van het vier-variabelen model biedt maakt het analyseren mogelijk van rampsituaties. Van een aantal rampen is een beschrijving gegeven met datum, plaats en oorzaak. De analyse van de 4-variabelen modellen zal gebruikt worden voor de requirementsdefinitie, ontwerp en ontwikkeling van het sluismodel. 

De verschillende factoren en achtergronden die  samenhangen met het modelleren van een sluis zullen in dit hoofdstuk toegelicht worden. Bovendien worden er hypotheses gevormd die de basis vormen voor debeantwoording van de onderzoeksvraag. 




\paragraph{Wat is uppaal}

Wat is Uppaal
Uppaal is an integrated tool environment for modeling, simulation and verification of real-time systems, developed jointly by Basic Research in Computer Science at Aalborg University in Denmark and the Department of Information Technology at Uppsala University in Sweden. It is appropriate for systems that can be modeled as a collection of non-deterministic processes with finite control structure and real-valued clocks, communicating through channels or shared variables [WPD94, LPW97b]. Typical application areas include real-time controllers and communication protocols in particular, those where timing aspects are critical.


model checking

Wat is statistical model checking?
Dit verwijst naar verschillende technieken dfie worden gebruikt voor de monitoring van een systeem. Daarbij wordt vooral gelet op een specifieke eigenschap. Met de resultaten van de statsitieken wordt de juistheid van een ontwerp beoordeeld. Statistisch model checking wordt onder andere toegepast in systeembiologie, software engineering en industriele toepassingen.
https://www-verimag.imag.fr/Statistical-Model-Checking-814.html?lang=en#:~:text=Statistical%20Model%20Checking%20(SMC)%20is,from%20state%20space%20explosion%20issues.


\cite{inriaStatsMoodCheck}
\cite{ buddeModelChecker}
\cite{AGHASuervey }


Waarom gebruiken we statistisch model checking?
To overcome the above difficulties we propose to work with Statistical Model Checking [KZHHJ09,You05,You06,SVA04,SVA05,SVA05b] an approach that has recently been proposed as an alternative to avoid an exhaustive exploration of the state-space of the model. The core idea of the approach is to conduct some simulations of the system, monitor them, and then use results from the statistic area (including sequential hypothesis testing or Monte Carlo simulation) in order to decide whether the system satisfies the property or not with some degree of confidence. By nature, SMC is a compromise between testing and classical model checking techniques. Simulation-based methods are known to be far less memory and time intensive than exhaustive ones, and are oftentimes the only option. 
https://project.inria.fr/plasma-lab/statistical-model-checking/

Alternatief
Alternatieven voor Uppaal zijn Asynchronous Events,Vesta en MRMC.


\paragraph{MODE CONFUSION }
Mode confusion tredd op als gepbserveerd gedrag van een technisch systeem niet past in het gedragspatroon dat de gebruiker in zijn beeldvorming heeft  en ook niet met voorstellingsvermogen kan bevatten.
\paragraph{Wat is automatiseringsparadox}
Gemak dient de mens. Als er veel energie wordt gestoken in de ontwikkeling van hulmiddelen die taken van werknemers overemen heeft dat tot resultaat dat veel productieprocessen worden geautomatiseerd. De vraag is dan of vanuit mechnisch wereldpunt de robot niet de rol van de mens overneemt en of de mens nog de kwaliteiten heeft om het werk zelf te doen.
\cite{bicker21102016automatiseringsparadox }
\cite{vseautoparadox }
\cite{blogxot21112016slimapparaat }



\paragraph{4 variabelen model}





Het 4 variabelen model kort toegelicht
Monitored variabelen: door sensoren gekwantificeerde fenomenen uit de omgeving, bijv temperatuur

Controlled variabelen: door actuatoren \bestuurde fenomenen uit de omgeving
For example, monitored variables might be the pressure and temperature
inside a nuclear reactor while controlled variables might be visual and audible alarms, as well
as the trip signal that initiates a reactor shutdown; whenever the temperature or pressure reach
abnormal values, the alarms go off and the shutdown procedure is initiated

Input variabelen: data die de software als input gebruikt
Here, IN models the input hardware interface (sensors and analog-to-digital converters) and
relates values of monitored variables to values of input variables in the software. The input variables model the information about the environment that is available to the software. For example,
IN might model a pressure sensor that converts temperature values to analog voltages; these voltages are then converted via an A/D converter to integer values stored in a register accesible to the
software.

Output variabelen: data die de software levert als output
The output hardware interface (digital-to-analog converters and actuators) is modelled
by OUT, which relates values of the output variables of the software to values of controlled variables. An output variable might be, for instance, a boolean variable set by the software with the
understanding that the value true indicates that a reactor shutdown should occur and the value
false indicates the opposite



\paragraph{6 Variable model}
Optitatieve statements omschrijven de omgeving zoals we het willen zien vanwege de machine. 

Indicatieve statements omschrijven de omgeving zoals deze is los van de machine. 

Een requirement is een optitatief statement omdat ten doel heeft om de klantwens uit te drukken in een softwareontwikkel project. 

Domein kennis bestaut uit indicatieve uitspraken die vanuit het oogpunt van software ontwikkeling relevant zijn. 

Een specificatie is een optitatief statement met als doel direct implementeerbaar te zijn en ter verondersteuning van het natreven vande requirements. 

Drie verschillende type domeinkennis: domein eigenschappen, domein hypothesen, en verwachtingen. 

Domein eingenschappen  zijn beschrijvende statementsover een omgeving en zijn feiten.Domein hypotheses  zijn ook beschrijvende uitspraken over een omgeving, maar zijn aannames. 

Verwachtingen zijn ook aannames, maar dat zijn voorschrijvende uitspraken die behaald worden door actoren als personen, sensoren en actuators. 

  
\paragraph{Conceptueel model}



System requirement:
uitspraak over wereld fenomenen (gedeeld of niet) of doelen
die bereikt moeten worden.
met enige regelmaat informeel, niet precies geformuleerd.
Software requirement/specicatie:
uitspraak over gedeelde fenomenen of doelen die de machine
moet bereiken middels de onderdelen waar die machine uit
bestaat of middels de fenomenen waar de machine controle
over heeft.
doorgaans preciezer, meetbaar, exact geformuleerd.


Systemen gaan een zekere interactie aan met hun omgeving:
Sensoren: meten fenomenen uit de omgeving (temperatuur,
druk, licht, geluid, etc.)
actuatoren: veranderen iets in de omgeving (mechanische,
electrisch, pneumatisch, etc.)
Software:
Kan niet direct communiceren met de buitenwereld.
Snapt derhalve niets van de buitenwereld.
Kan alleen maar bestaan in en communiceren met het
systeem.


\paragraph{Requirementsengineering}

Om de juiste requirements te verzamelen en selecteren hebben we meer kennis nodig van de methoden hiervoor gebruikt in het domein van requirementsengineering. Daarom is een literatuurstudie gedaan naar rapporten en artikelen die ons meer informatie over dit onderwerp verschaffen.
 Uitdagingen in requirementsengineering zijn incomplete requirements en specifcates, veranderende requirements en specificates en grote, complexe oftwaresystemen.
 
 Het article the worlds a stage biedt inzicht in de requirementstechnieken voor een ambulance in london. In het artikel gaan de onderzoeks in op de volgende onderwerpen: 
 viewpoints, sociale ascpecten,evolutie, non-functional requirements, conflict resolution, traceability
 
 Goal of this paper is requirement  engineering on London aulance service
 Method of opinions: crew, staff, management, computational, transport, services
 Evolutioon: changes, specification and technology trade
 Environment: company policies, regulation, impact solution on organizational
 Non-functional aspect: communicatio problem, malfunctions, less critical isues: cost, tradeoff beween performance \& user interfaces
 vieuwpoint: is a subset of all system requirements expressible in a given requirements notation regardless of the stakeholders involved
 
 log change
 basic model vieuw
 hypertext vieuw
 data transmission problems
 continued difficulties
 installation problems
 problems caused by mistake
 tracebility requirements[selecting reliable information]
 PRE requirement specification traceability, repository baed approach
 1) compromise specification
 2) representatives
 3) agreement dimensions
 Domain: part of the worl in which the computer system effects will be felt, inclusing its peoples, organizational structure, related legislation, physical location and met only the compyter systems
 
 
 Het artikel "from inconsistencyhandling to non-conanical requirements management: a logical perspective" geeft enkele tips voor het omgaan met inconsistente requirements:
 
 1) identifying non-canonicalrequirements
 2) measuring them
 3) generate caandidate proposals for handling them
 4) choosing acccptable probosals
 5) revising them acccording to the proposals

Het artikel "managing inconsistent specification: reasoning, analysis, action" zoekt een ontologische benadering voor het omgaan met inconsistenties in de requirements specificaties.
Voor de omshrijving van een specificatie kun je gebruik maken van logica. Daarbij kun je onderschei maken in klasieke logica quasi -logica.
Wat ook een rol kan spelen in domain interpretatie. De achtergrond van de gebruikers speelt ook een rol.
Zo is er e=onderscheid te maken in de volgende groepen: users, customers, domain experts, designers,, manufacturers
graphical  textual specification

Basic constraint, legal constraint, cooperation constraint
1) scenatio  definition
2) scenario analysis
3) scenario consolidation


Hoe kan een systeem verder worden ontworpen op een manier dat non-functionele requirements worden geimplementeerd?
Hoe hangt dat ontwerp samen met aanpassingen van het functionele en structurele aspect van het systeem?

block[objects, classes, methods, messages, inheritance]
[goals,agents, alternative, events, actions,existence modalities,agent responsibilities]


Het artikel "representing and using nonfunctional requirements: a process-oriented approach"" gaat in op een het proces van requirements acquisitie. Hierbij in ogenschouw de acquisitie van prestaties, ontwerp en aanpasbaarheid.
product oriented
process oriented


Acquisitie Prestaties
user concern
-Hoe goed werkt het product
-Hoe goed wordt de bron gebruikt?>> Efficiency
-How veilig is het product >> integrity
-Met hoeveel zekerheid is uit  te sluiten dat het werkt >>Reliability
-Hoe goed werkt het product onder zware omstandigheden >> sustainability
-Hoe makkelijk is het product in gebruik >> usability
quality attribute


Acquisitie: Ontwerp
user concern
Hoe valide is het ontwerp
-Is ht ontwerp conform de requirements
-hoe makkelijk is het ontwerp te repareren
-Hoe makkelijk zijn de prestaties te verifieren

quality attribute


Acquisitie: Aanpasbaarheid
user concern
-hoe makkelijk is het om het product aan te passen
- hoe makkelijk is het om het product te updaten en/of uitbreiden>> expendability
- hoe makkelijk is het om een wijziging door te voeren>>flexibility
-hoe makkelijk is het om andere system aan te sluiten >> portability
- hoe makkelijk is het om het product te transporteren >> interoperability
-hoe makkelijk is het om te converteren tot een systeem gebruiksklaar voor communiceren met andere systemen>> reaseability
quality attribute




 \cite{jonkerTreurKlush200informativeAgents}
\cite{boehmBoseLeeRequirementsNegotiations}
\cite{muHungJinLiu2013inconsistencyReqs}
\cite{hunterNuseibeh1996manageSpecs}
\cite{myloloupos1992representingReqs}
\cite{zavePamela4darkCorners}
\cite{zavePAmela1997regEngineering}

%%%%%%%%%%%%%%%%%%%%%%%%%%%%%%%%%%%%%%%%%%%%%%%%%%%%%%%%%%%%%%%%%

what is a good software specification

\cite{fvaandrager2322010Goodmodel}
\cite{onix01102022devopmodel}
\cite{sulemani04012021softwareprocesmodel}
\cite{globalluxsoft18102017softdev}
\cite{wiegers30052022SRS}
\cite{muller06092020goodspecification}
\cite{informit30062008reqmanagement}
\cite{altexsoft15092020writingSRS}


\paragraph{Wat is een sluis}

\paragraph{Recente ontwikkelingen op het gebied van sluisautomatisering}

Het ministerie van verkeer en Waterstaat wil in het kader van het klimaatakkoord en onderzoek laten uitvoeren naar de staat van het sluizenpark in Nederland. Het onderzoek moet zich richten op het ontwerpen en ontwikkelen van een geautomatiseerd sluismodel dat geschikt is voor een brede toepassing. In het onderzoek moet naar voren komen wat de huidige staat is van de sluizen met oog op veiligheid, efficiëntie, capaciteit, onderhoud, duurzaamheid en automatisering. Het onderzoek geeft aan hoe een volledig model worden opgeleverd opdat ontwerp van verschillend volledig geautomatiseerde sluizen in de toekomst geautomatiseerd kunnen worden.  


\paragraph{Studie naar rampen aan de hand van het vier variabelen model}
\newline Voor deze studie is onderzoek gedaan naar verschillende rampen aan de hand van het vier variabelen model.
Elke ramp op deze manier categoriseren  kan ons helpen te bepalen in hoeverre requirements een rol kunnen spelen in de veiligheid van ons model.  Zo is er de bijlmerramp \cite{aviationsafety04101992airplaneCrashBijlmer}
, deze vond plaats op 04/10/1994. Dan nog de  ramp turkisch airlines vlucht 1951 op woensdag 25 februari 2009 \cite{catsr25022009Boeing737AmsterdamCrash}
\cite{zuilen23022019Tijdlijnpoldercrash}
\cite{wikinews04032009techfoutailines1951}
\cite{luchtvaartnieuws21012020boeing737conclusies}
\cite{adformatie280220209communicatiegebreken}
\cite{spinnael25022009onderzoekpolderbaancrash}
\cite{crashTurkishAirlines}
\cite{flightradar24}
\cite{flightstatstracker}. 
%%%%%%%%%%%%%%%%%%%%%%%%%%%%%%%%%%%%%%%%%%%%%%%%%%%%%%%%%%%%%%%%%
\newline \indent
De therac-25 June 1985 and January 1987. 
Medical lineair accelerators accelerate electrons to createhighenergy beams that can destroy tumors with minimal impact on the surrounding healthy tissue.
In the mid-1970s, AECL, developed a radical new "double-pass" concept for electron acceleration. A double passaccelerator needs much less spaceto develop comparableenergy levels because it folds the long  physical mechanismrequired to accelerate the electros, and it is more economic to produce.
Using this double pass concept AECL designed the  Therac-25, a dual mode lineair acelerator that can deliver either photonsat 25 MeVor electrons at various energy levels. Compared with theTerac-20 The Thrac-25 is notably more compact,, more versatile, and arguably easier to use. 
The higejr energy takes advantage of the phenomenon "depth dose": As the energy increases, the depth in the body at which maximum dose buildup occurs alse increases, sparing the tissue above the target area.
First, like the Therac-6 and the Therac-20, the Therac25 is conrolled by a PDP11. The Terac-6and Therac-20 had been designed around machines that already had histories of clinical use without computer control.
The therac-20 has idependent protective circuits for monitoring electron-beam scanning, plus mechanical interlocks for policing the machine and ensuring safe operation.
Finally some software for the machines was interrlatd or reused.
Eleven therac-25 were installed: five in the usand six in canada. Six accidents involving massive oerdoses to patients occured between 1985 and 1987. The machine was recalled in 1987 for extensive design changes, including hardware	 safeguards against errors.
Kennestone Regional Oncology Center 1985
Door rechtzaken waren managegers op de hoogte van de problemen en ongelukken. Maar er werd in het vervolg niet over gerapporteerd.
The treatment prescription printout failure was disabled at the time of the accident , so there was no hardcopyof the treatment data.
Ontario Cancer Foundation in 1985
Since the machine did not suspendand the control display indicated no dose was delivered to the patient, the operator went ahead with a second attempt at trratment by pressing the "P" key, expecting the machine to deliver the proper dose this time. This was standard operating procedure and, described in the "The operating interface" on p 24, Therac 25
oprators had become accustomed to freunt malfunctions that had no untowardconsequences for the patient. Again, the machine shut downin the same manner. The oeprator repeated this process four times after the original attempt- the display showing "no dose" delivered to the patient each time. After th fifth pause, the machine went into treatment suspedn, and a  hospital service technician was called.
The technician found nothing wrong with the machine. This was not an unusual scenario, according to the Therac-26 operator
Manufactureere response
Government and user response
Yakima Valley Memorial Hospital in 1985
Manufactureere response
Government and user response
East Texas Cncer Center, March 1986
Manufactureere response
Government and user response
East Texas Cncer Center, April 1986
Manufactureere response
Government and user response
Yakima Valley Memorial Hospital
Manufactureere response
Government and user response
	\cite{rogaway2004therac25},
\cite{wikiTherac25}, 
\cite{lynch2017theracRaceConditions},	\cite{lim1998theracdisaster}, 
\cite{fabio26102015therac25},	 	\cite{ethicsunwrappedTherac25}, 	\cite{casesHistoryTherac25},	 	\cite{caballero2019Therac25}, 	\cite{rose1994theracFatalDose}, 	\cite{levesonMITTherac25},
\cite{grant1978theracevaluation},	 	\cite{turnerTheracAccidentsInvestigations},	\cite{turner1993TheracAccidentsInvestigations}, 	\cite{wang2017industrialdesignengineering}, 	\cite{levesonturner1993theracpart2},	\cite{porelloTheraccFailure},\cite{theracIncidents}, 
\cite{huffbrown2004casestudyethicatherac}, 
\cite{sebowikimedicalradiation},	\cite{hsia1995testtherac25},	\cite{magsilvaTheracTesting},
\cite{chemeuropetherac25},	\cite{statsenko10102016Therackillerbug},	\cite{therac25casestudy},	\cite{thomas1994theracinLotos},	\cite{twitter2019programmerbehindtherac},	\cite{wikibookstherac}, 
\cite{bozdagTherac25},	\cite{levesonTurnerTheracAbstract}, 	\cite{stackexchange2021therac25code}.
%%%%%%%%%%%%%%%%%%%%%%%%%%%%%%%%%%%%%%%%%%%%%%%%%%%%%%%%%%%%%%%%%
\newline \indent
%Hoe werkt het
tesla autopilot features voor dataverzameling\cite{denneyjdsupraFeds},\cite{gritti24062020tesladataengine}.
% crashes
 De eerste tesla crash is van juni 2016 \url{https://impakter.com/tesla-autopilot-crashes-with-at-least-a-dozen-dead-whos-fault-man-or-machine/#:~:text=The%20first%20known%20death%20reportedly,trailer%20against%20the%20bright%20sky.}. En meerdere zouden volgen.
Een ongeluk in  de VS waarbij 2 inzittenden om het leven kwamen. Een persoon had plaats genomen als bijrijder en de andere persoon als passagier achter de stoel van de bestuurder. Waarschijnlijk was de autopiloot niet ingeschakeld.
\cite{anderson30042021secondteslacrash},\cite{raynal20042021probeTeslaCrash},\cite{firstpress11052021fatalnonautopilot},\cite{cochran18042021nodriverTeslaCrash},\cite{gitlin11052021autopilot},\cite{sommerfield12072021NHTSAmandateresult},\cite{hawkins30062021nhtsarequiresreporting},\cite{wilson19042021teslacrashregulators},\cite{mcfarland22042021selfdrivingrisks}
De situatie en oorzaken zijn bij elke ramp verschillend. 
Een automobilist heeft in een rit van 37 minuten slechts 25 seconden zijn handen aan het suur gehad ondanks de melding "Hands requireed not detected". Hiermee zijn de onderzoekers van de NTSB ervan uitgegaan dat de bestuurder de autopiloot bewschouwde als een volledig autonooom rijsyssteem in plaatst van een veligheidsmechanisme
\cite{oremus21062017fatalTeslaCrash}. Of in 
Mei 2015 als een besuurde foto's van zichzelf maakt in de testla zonder handen aan het stuur of voeten op het pedaal.
\cite{guardian15052021teslacrashHandsOnWheel}
Een faatale crash in 2016 waarbij de bestuurder  e veel vertrouwde op het semi-autonome rijtechnologie op het verkeerde type wegdek.
\cite{Puzzanghera13092017TeslaSharesBlame}
Onderzoek naar een fatale crash op 7 mei 2016 toont aan dat er beperkingen zitten aan de autopilot mode. Om specifiek te zijin is de automatische noodrem niet failsafe, blijkt uit onderzoek.
\cite{jaillet02022017teslaAutopilotLimitations}
\cite{reuters03102019teslaAutoParkingFail}
\cite{dowling23042021}
Op  April 17 2019 een autocrash waarbij het onduidelijk is of de autopiloot aan stond.
\cite{young05112021fatalTeslaReport}. Een auto ongelu waarbij een tesla is betrokken. De bestuurder was waarschijnlijk afgeleid door de games op zijn apple telefoon. De NTSB gaf aan dat het crash-avoidance systeem neit otnworpen is en ook geen crash atnuaor heeft gedetecteerd. Herdoor accelereerde de autopilot  het voertuig. Ook Faalde het systeem in het verschaffen van een crash aleter en werden de noodremmen niet geactiveerd.
\cite{tiungteslasoftwarecrash}
Er is ook een melding van een tesla waarvan de autopilot bots tegen een stilstaande politieauto
\cite{kierstein18032021teslaAutopilotCrashStationary}. Ook uit dit onderzoek blijkt dat er geen gebreken waren en dat het automaische remsysteem neit kapot was. De HNTSA concludeerder dat de bestuurder zelf geen actie ondernam door  bij te sturen of te remmen. In een eerder artikel kwam naar voren dat de tesla een autopilot krijgt die enkel camera's en GPS gebruikt; lidar of een radarsysteem wordt niet toegepast.
\cite{janssen20062017teslacrashdetailflorida}
Enkele fotos van crashes met autonome rijsysstemen \cite{saferoardsCrashesAutonomousvehicles}.
\cite{stephardson18032021revieuwingtesla}
%Onderzoeksrapport naar testla automatic vehicle control system
\cite{habib28062016NHTSATeslaReport},
\cite{darkReading17112020TeslaBackup},
\cite{heilweil26022020teslaAutopilot}
% overzicht
Tesla autopilot crashes met meer crashes en incidenten dan tot dan toe gerapporteerd
\cite{teslaFDSCrash}
De meest voorkomende crashes zijn stationaire objecten bij hoge snelheden, lane incursions from stationary objects, auti=opilot confusion at forks and gores.
\cite{teslaCrashesCauses}
\cite{teslacrashOvervieuw}
\cite{tesladeaths}
% veiligheidsrisico''
De veiligheidsrisicos van de tesla lopen uiteen. Zo zijn er risicos in de machinelearning technologie:
veiigheidsrisico Three Small Stickers in Intersection Can Cause Tesla Autopilot to Swerve Into Wrong Lane
\cite{evan01042019teslaautopilotIntersection},
\cite{lambert31062020q2safetyreport},de autopilot zelf
\cite{templeton06092019HTSBReportTesla}. Een studie door de consumntenbond in de VS toont aan dat hetautopilot systeem van de testla niet failsafe is. Zo zijn de sensoren, gebrukt voor detectie van een bestuurder negatief te beinvloeden.
\cite{dowling23042021autopilottricking} Maar ook andere problemen met de bluetooth 
\cite{wiredBloutoothHackTesla}, touch screen
\cite{preston14012021NHTSATeslaRecall},
Web-based attack crashes Tesla driver interface
\cite{leyden23032020TeslaInterfaceHack}.
Of zelfds de tesla batterij is veiligheidsvraagstuk geworden
\cite{mitchell01072020teslabatterycooling}.
Maar ook was een onderzoeker  was in staat om persoonlijke details van afgedankte voertuigonderdelen  te vekrijgen nadat deze waren afgekeurd vanwege upgrades en reparaties op consumentenvoertuigen.
\cite{stumpff04052020TeslaPersonalData}
Data-opslag in de cloud niet altijd bereikbaar.
\cite{mitchell24022020AIDataTesla}
%Wat er mis zou kunnen gegeaan wordt dru over gespeculeerd online.
%\cite{stackexchange102019teslacarmistake}
dodelijk ongeluk
\cite{fottrell03092018TeslaSecurityChecks},
softwarefout maakt diestal mogelijk
\cite{kirk26112020modelX}
fouten ontdekt in onderzoek
\cite{bbc24022021hyundaiBatteryFireFix},
tesla cloud gehacked
\cite{hawkins22102022}.
%Het AI aloritme vn Tesla
%\cite{rangaiah25022020teslaAI}
%Waarom deeplearning geen zelfrijdende auto's zal voortbrengen
%\cite{bdickson29072020teslalevelfive}
%Een survey naar de tesla gebruikers.
%\cite{randall05112019modelSurvey}
%maatschappelijk probleem
This analysis considers the potential impacts of completely self-driving vehicles on vehicular liability. 
\cite{griemannExaminSelfDriving}
Dan zijn er nog maatschappelijke problemen die de aanpak moeilijker maken.
Er is in de vs in verschillende staten een andere wetgeving
\cite{berry21042021teslacrashtexas}
\cite{hull23072021regulatorsaftercrash}
\cite{wikiTeslaAutopilot}
%oplossingen
Toch zijn er oplossingen en tegenmaatregelen.
tesla gaat advanced driver assistance systems inzetten met behulp van  passive visual, ultrasonic, en radar.
\cite{tasking07062017TeslaAugmentedSafety},\cite{ackerman01072016TeslaImperfect}
Safe system solutions door David Harkey
\cite{Harkey30052019SafeSystemVehicle}
%maatregelen
Voor elke auto uitgerust met een level 2 tot level 5 autonomy wordt nu standaard een rapport van van de crash opgvraagd door de NTSA. Dit in het kader van verder onderzoek waarbij de autoritait kijk naar  ziekenhuisbehandeling, fataliteit, airbag deployment.
\cite{szymkowski29062021nhtsaTeslaCrashReports}. 
%%%%%%%%%%%%%%%%%%%%%%%%%%%%%%%%%%%%%%%%%%%%%%%%%%%%%%%%%%%%%%%%%
\newline \indent De slmramp op  07/06/1989 \cite{espnSLMterugblik},\cite{dennisRosier01052020}
\cite{hassing07062020slmramp},\cite{amsterdamArchiefSLM},\cite{rtvOost06062019nabestaande},
\cite{breda07062021AndroSnel},\cite{andereTijdenSLMCrash},
\cite{aviationReport},\cite{aviationSLMCrashAccidentInvestigation},\cite{mcDonnelDouglasCommissionReportSLMCrash},
\cite{wikiSRFlight764},\cite{nos07062019SLMTerugblik},\cite{dagvantoenSLMCrash},\cite{waterkantNesty07061989},\cite{eduNandlalSRCrash},\cite{oldjetsSRAirways},\cite{cloudberg02012021srflight764},\cite{apnews07061989srplanecrash}.
%%%%%%%%%%%%%%%%%%%%%%%%%%%%%%%%%%%%%%%%%%%%%%%%%%%%%%%%%%%%%%%%%
\newline \indent De schipholbrand op 27/10/2005\cite{schipholbrand27102005video},\cite{schipholbrand27102005video},\cite{onderzoeksraad2610schipholoost},
\cite{schipholbrandvideoargos},\cite{nunl30052023feitenoverzicht},\cite{parlementairemonitorschipholbrand},\cite{videonpoNOVA13112008},\cite{rizoomes01052014schipholbrand},\cite{heuvelkroesschipholbrandcamerabeelden},
\cite{wikiSchipholbrand},\cite{schipholbrand27102005video},\cite{onderzoeksraad2610schipholoost},\cite{schipholbrandvideoargos},\cite{nunl30052023feitenoverzicht},\cite{singeluitgeverijenSchipholbrand},\cite{eenvandaagschipholbrand},\cite{parlementairemonitorschipholbrand},
\cite{videonpoNOVA13112008},\cite{rizoomes01052014schipholbrand},\cite{heuvelkroesschipholbrandcamerabeelden}. 
%%%%%%%%%%%%%%%%%%%%%%%%%%%%%%%%%%%%%%%%%%%%%%%%%%%%%%%%%%%%%%%%%
\newline \indent De explosie tanjin china 12/08/2015. 
Op 12 augustus 2015. Er waren twee explosies bij de Rulthai logistiek  faciliteit zorgde voor de opslag vn  gevaarlijke stoffen. De explosie zorgde voor de vernietiging van 12000 voertuigen, schade aan 17000 huize binnen een traal van 1 km. Er waren 173 doden inclusief brandweermensen.
Een van de explosies zorgde voor  een beving van 2.3 op de schaal van rigter.
De volgende factoren zouden een rol hebben gepeeld:
Een onjuiste afbakening van het opslagmaeriaal
Er was  weinig kennis bij de autoriteiten over  opslagmaterialen. Zo bleek er 7000 ton aan materiaal opgeslagen, dat is ruim 70 keer te maximaal toegestande hoeveelheid. 
Onverenigbaar grondgebruik in de nabije omgeving. Veel woonwijken met nar schatting 6000000 bewoners en 500 lokale bedrijvenin de buurt van de opslag gevaarlijke stoffen.
Opgeslagen materialen  waren: calcium carbine, sodium nitraat, potassium nitraat, amminiak nitraat en cyanide.
Ook is er veel kritiek geweest op de acties van de autoriteiten. Zo was er censuur vanuit de overheid op de journalistiek.
Ook was er naar alle warschijnlijkheid sprake van corruptie. Zo bleek achteraf dat een van de grootste aandeelhouders Dong Shexuang de zoon te zijn van een oud-politiechef in Tanjin haven, genaamd Dong Pijun
De overheid beloofde strengere toezicht en alle bedrijven moeten een risico-inventariatie maken en onderhouden\cite{jiang16042019TanjinExplosion},
\cite{staff31082015tanjinblastunrevealed},\cite{chinafile18082015tanjinexplosion},
\cite{pinghuang2410201TanjinFactreport},\cite{portoTanjinExplosionSight},\cite{imago17082015TanjinApartmentImages},\cite{trager14082015Chemicalblast},\cite{pangeramo27082015TanjinExplosion},\cite{ap06082020ammaniumnitrate},
\cite{morris14082015TanjinIndustryImpact},\cite{milesyu20082015exposingtoxicgovlines},\cite{artemis30032016tanjininsurance},\cite{aidenxiatanjinblast},
\cite{danwangTanjinflexreport},\cite{keyHighlightsTanjin},\cite{hartley13082015videofootage},\cite{odonnel01062017firetanjinblast2015},
\cite{fan15082015newyorkermistrustchina},\cite{yanlidongchinamediaframingTanjin},\cite{evans27092017TnjinInsurance},\cite{jasi26032019chineschemplant},\cite{shiqingTanjinExecutiveSentence},\cite{sophiebeach15082015},\cite{hamzeh05082020BeirutBlast},\cite{chemwatch18082015TanjiinExplosion},
\cite{thehindu15062019chinaExplosion},\cite{santagotimes24032019chinablast},
\cite{klingecorp28042020causedTanjin},\cite{mcgarryExplosions2017},\cite{roswnfeld13082015TanjinReports},
\cite{aria12082015explosionaTanjin},\cite{tremblay11022016chineseInvestigatorsTanjin},\cite{taylor13082015TanjinExplosianAftermath},
\cite{associatedPresss13082013},\cite{un20082015InvestigationTanjin},\cite{france2412082015TnjinExplosion},\cite{npr14082015TanjinCause},\cite{bbc05022016TanjinResponsibles},\cite{CBodeen15082015TanjinExplosion},\cite{reutersTanjinInsurance},\cite{yu082016evaluationTanjin2015},\cite{wiki2015TanjinExplosions},\cite{bbc17082015whathappenedTanjin},
\cite{mortimer19082016taijinexplosioncrater},\cite{internationallabourofficeChmControlTooliit},\cite{euTaxationCustomsICSC},
\cite{iloWHOChemSafetyCards}.
%%%%%%%%%%%%%%%%%%%%%%%%%%%%%%%%%%%%%%%%%%%%%%%%%%%%%%%%%%%%%%%%%
\newline \indent  De ethiopian airlinesop 10/03/2019\cite{caliskan09112013747boeingkalman},\cite{gates18112020boeingcrisis},
\cite{boeing737maxsoftwareprobles},\cite{avetisov19032019boeingmalwarestate},\cite{thompson23112020nationalsecurityboeing},
\cite{wiki737maxgroundings},\cite{campbell02052019boengcrashhumanerrors},
De oorzaak is de MCAS
\cite{hawkins22032019737maxairplanes},\cite{barnett05052019737maxcrisis}, \cite{thomas30082020737safest},\cite{boyle18112020737maxupgrade},\cite{bergstraburgess122019737maxMcasAlgorithm},\cite{737mcas},\cite{german190620217372yaftergrounded},\cite{beningo02052019boeinglessons},\cite{bloomberg26092019failedpred},\cite{afacwaLostSafeguards}, als een single point of failure \cite{uran05042019SPOF}
Angle-of-attack\cite{boeing737maxdisplay},
Behalve de MCAS waren er nog andere failures\cite{fehrm24112020737changes}, en ook deze failures \cite{dohertylindeman15032019737problems}
\cite{travis18042019737maxsoftwaredevop},
%\cite{easa27012021737maxsafereturn},
safety record van de boeing
\cite{touitou11032019737tragedies},
 Oplossingen zijn \cite{caa737modifications}. 
%%%%%%%%%%%%%%%%%%%%%%%%%%%%%%%%%%%%%%%%%%%%%%%%%%%%%%%%%%%%%%%%%
\newline \indent Het mortierongeluk in Mali op 06/04/2016. Aanwezige militair brengt slachtoffer naar de fransen, vervolgens naar de Tongolezen. Maar de kwaliteit van personeel liet te wensen over.
Er werd een Nederlandse arts overgevlogen. De slachtoffers werden overgevlogen naar Gao omvervolgens te worden oergevolgen naar Nederland.
Het ongeluk werd veroorzaakt door een kapot afsluitplaatje in de mortier. De granaat opslag in een niet gekoelde container. Dan was er vocht in de fatale granaat. Zodoende werden er explosieve stoffen gevormd in de granaat.
Tijdens de oefening werden de granaten warm in de zon. De granaat stond in veilie stand kon de explosie niet voorkomen.	\cite{ovvMortierOngevalMaliVideo} 
\cite{bnnvara13062018malirapport}
\cite{eucal11012021malimissieverlengd}
\cite{nos21052014zorgenmalimissie}
\cite{meijnders}
\cite{bnrwebredactie}
\cite{keultjes01062016malimissiecoalitie}
\cite{veenhof18012019}
\cite{isitman06012016militair}
\cite{nporadio11072016filmdemissie}
\cite{parlementairmonitor15122013mortierongeluk}
%%%%%%%%%%%%%%%%%%%%%%%%%%%%%%%%%%%%%%%%%%%%%%%%%%%%%%%%%%%%%%%%%
\newline \indent De ramp tjernobyl 26/04/1986. \cite{INSAVienna1992Chernobyl}
De mislukte veiligheidscontrole op 26 apeil 1986 01.24 uurin de sovjetuni leiddte tot explosies in een van de reactoren in de kerncentrale. De reactoren hadden geen veiligheidomhulling en de reactor bevat grote hoeveelheden brandbaar grafiet.
Door de explosie en de brand kwamen er radioactieve stoffen vrij.het gaat helemaal mis in de kernreactor 4. De warmteproductie nam  toe met een explosie tot gevolg.
31 mensen kwamen om, waaron veel mensen dagen later door stralingsziekte.
\cite{wikiTjernobyl},
\cite{rivmTjernobyl},
\cite{andereTijdenTjernobyl},
\cite{kingskey19042022tjernobyl},
\cite{erikbork26042023reactor4},
\cite{nosTjernobyl30jaarlater},
\cite{knmi04052021tjernobylbosbrand},
\cite{dodonovaKVIRisicoTjernobyl},
\cite{dumarey04062020verhaalTjernobylWaarheid},
\cite{sparkesNewScientistTjernoby},
\cite{kernenergiened26041986chronologiemaatregelen},
\cite{mapszoneReactor},
\cite{kernhistoriek15062021tjernobyl},
\cite{nucleairforumFeitenTjernobyl},
\cite{kernongevalTjernobylFancGov},
\cite{arendswolters062019lessenTjernobyl},\cite{damveld08052020tjernobyl},
\cite{deVriestjernobylHolland},\cite{ing3enieur29042015antistralingskoepel},
\cite{verschuur14012013tjernobylreports},\cite{paperlessarchivesTjernobyl},\cite{vargos082000tjernobylconcerns},\cite{mauroNuclearRiskSociety},\cite{vienna06092005LookingBack}
%%%%%%%%%%%%%%%%%%%%%%%%%%%%%%%%%%%%%%%%%%%%%%%%%%%%%%%%%%%%%%%%%
\newline \indent  Research case: De digitale aanval op de Oekrainese krachtcentrale op 23,december 2015

Op 23,december 2015  vind er een cyber aanval plaats op het elektriciteitsnet van de Oekraine. Dit was de eerste bekende aanval op een elektrisch contole  system.  Dit verslag geeft inzage in een analyse van de Ukraine cyber aanval,
inclusief hoe de actoren zich zelf toegang gavan tot het controle systeem, welke methoden de acoren hebben gebruikt voor reconnaissance en vastleggen van het systeem, een gedetailleerde omshrijving van de aanval op 15 December 2015, en de methoden die gebruikt zijn door de aanvallers om hun sporen uit te wissen en daarmee het het stoppen van schade toebrengen  nog moeilker maken. Daarnaast wordter  een gedetailleerde omschrijving gevevenv an de beveiliging van de SCADA ccontrol systemen gebaeerd op bst practices, inclusief het control network ontwerp, technieken voor whtelisting, monitoring en loggen, en  opleiding van personeel.
\cite{Whitehead2017ukrainepoweroutage}
\cite{noauthor_2022-nm}
\cite{zetter2016GridHack}
\cite{owens21032017ukrainemitigationstrategies}
\cite{cerulus2019FrontlineRussiaAttack}
\cite{grammatikis2019AttackIEC6087505104}
\cite{hidajat2016ScadaSimulator}
\cite{uscert20072021crashmalware}
\cite{zetter12062017malwareanalysis}
\cite{icsRussianHackingCyberWeapon}
\cite{usgovC2M2}
Dit verslag geeft inzage in een analyse van de Ukraine cyber aanval,
inclusief hoe de actoren zich zelf toegang gavan tot het controle systeem, welke methoden de acoren hebben gebruikt voor reconnaissance en vastleggen van het systeem, een gedetailleerde omshrijving van de aanval op 15 December 2015, en de methoden die gebruikt zijn door de aanvallers om hun sporen uit te wissen en daarmee het het stoppen van schade toebrengen  nog moeilker maken. Daarnaast wordter  een gedetailleerde omschrijving gevevenv an de beveiliging van de SCADA ccontrol systemen gebaeerd op bst practices, inclusief het control network ontwerp, technieken voor whtelisting, monitoring en loggen, en  opleiding van personeel.
\cite{Whitehead2017ukrainepoweroutage},\cite{zetter2016GridHack},\cite{boozallen2016lightwentout},\cite{finklejan2016UsBlamesRussianSandworm},\cite{desarnaud2017cyberattacks},\cite{caseli04112016intrusiondetectioncontrolsystem},\cite{rochascadatesting},\cite{hidajat2016ScadaSimulator},\cite{zetter2017moreDangerousMalware}.
Oop 23,december 2015  vind er een cyber aanval plaats op het elektriciteitsnet van de Oekraine. Dit was de eerste bekende aanval op een elektrisch controle  system met corrupte firmware. Daarnaas wordt er een telecom-based denial of service attack met  geautomatieerde systemen om het telefoonverkeer uit te schakelen.
\cite{Whitehead2017ukrainepoweroutage}
Uit onderzoek\cite{zetter2016GridHack} naar de aanval,  uitgevoerd door Oekraiene sen Amerikaanse militairenblijkt  bleek onder meer dat de power grids in sommige gevallen beter waren beveiligd dan de Amerikaanse. Desondanks was de viligheid niet optimaal door onder andere de  hetgegeven dat werknemers op afstand konden inloggen en geen gebruik van 2-stapsverificatie.
Oekraine wijst naar de russen \cite{zetter2016GridHack}, 
\cite{greenberg2017Cyberwartestlab},
\cite{boozallen2016lightwentout},
\cite{finkle08012016russiansandwormhackers},
\cite{zinets15022017ukrainechargesrussia},
\cite{mcelfresh2016cyberattackhowandwhy},
\cite{parkwalstorm11102017russiagridattack}.
{Situatie Oekraiene}
\cite{drago2017CrashOverride},
\cite{slowik2019ReassasUkraine2016Attack}.
{Situatie algemeen}
\cite{cerulus2019FrontlineRussiaAttack},
\cite{desarnaud2017cyberattacks},
\cite{dragos2019TargetedTransStation}.
{Factoren}
\cite{shehod2016gridadvantageus}
{Oorzaak}
\cite{rocha2017cybersecyrityanalysisScada},
\cite{2017crashoverridenostuxnet},
\cite{vijayan2017firstmalwareCausedOutage},
\cite{slowik2019ReassasUkraine2016Attack}.
{Gebruikte materialen}
\cite{2015ukrainegridattack},
\cite{industroyershortfact}
{Uitvoering van de aanval}
\cite{Whitehead2017ukrainepoweroutage},
\cite{boozallen2016lightwentout}.
{Oplossingen}
~\cite{Whitehead2017ukrainepoweroutage}
\cite{Whitehead2017ukrainepoweroutage}
\cite{boozallen2016lightwentout}
{spearfishing}
{blackenergy}
{remote access capabilities}
{serial-to-ethernet communication devices}
{telephony denial of service attacks}
{oplossingen}
Identificeer alle risicos en schrijf een plan foor het managen van de risico's.
Implementeer  effecteve controle  om het riico te managen.
Creeer een diepgaand model dat ervoor zor dat er efectieve en efficiente security controls worden uitgevoerd.
Aangaande de gebeurtenissen in de oekraiene kunnen de volgende security controls worden opgenomen in het securitymodel: Initial access to enterprise network, pivot in interprise network, elevate priviliges, maintainance access, gain access to control system, attack, attack complication, destroy hard drives.
\cite{Whitehead2017ukrainepoweroutage}
{Discussie}
{Verder lezen}
\cite{shahzad2014ScadaProtocolsPollingScenario},
\cite{grammatikis2019AttackIEC6087505104},
\cite{2017win32industroyer},
\cite{yadav2020reviewScadaArchitecture},
\cite{arrizabalaga2020surveyiiotProtocols},\cite{fauri2017EncryptionICS},\cite{resch31102019IEC62351secureCommunication},\cite{levalle2020FuzzingICSProtocols},\cite{blackhatusa2017},\cite{blackhatusa2017},\cite{abb30062017crashoverridenotification},\cite{spinner2018crashoverrideiot},\cite{njccicthreat08102017crashovverrideprofile},\cite{slowikvb2018crashoverride},\cite{crashoverridenetwork},\cite{wikiindustroyer},\cite{icsSecurityRussianHacking},\cite{holappa2017threattoElectricityNetworks}.
%%%%%%%%%%%%%%%%%%%%%%%%%%%%%%%%%%%%%%%%%%%%%%%%%%%%%%%%%%%%%%%%%
\newline \indent Dan zijn er nog andere ongelukken met de stint, de shietpartij op militairencomplex in ossendrecht, stint-ongeluk, de enschedese vuurwerkramp en de molukse treinkaping. Meer recentelijk de coronacrisis.
%%%%%%%%%%%%%%%%%%%%%%%%%%%%%%%%%%%%%%%%%%%%%%%%%%%%%%%%%%%%%%%%%





\paragraph{Safety critical systems}
\cite{winceckCriticalToSafety}
\cite{chambersHazardAnalysisSCS}
\cite{rslater1998SCSAnalysis}
\cite{knightchallengessafetyCritical}
\cite{johnson2006devsafetycritical}
\cite{daucriticalsafetyconsider}
\cite{fallsafedesign}
\cite{arForce2015VerificationExpectations}
\cite{nebulaassessment}
\cite{lalaArchitecturalPrinciples}
\cite{mitNotesSafetyCritical}
\cite{britishColumbia2020GuideSafetyCritical}
\cite{fulvio1993safetycriticalsystems}
\cite{dlrtabid}
\cite{knight2010SafetyCritical}
\cite{creavisafecritical}
\cite{valdes2018SafetybyAutomation}
\cite{2015whensafetymanagementsystemsfail}
\paragraph{Ondeerzoeksresultaten naar sluisbeveiliging}



Verouderde computersystemen zijn door de jaren heen gekoppeld aan netwerken, zodat ze op afstand te besturen zijn. Dit zorgt ervoor dat systemen kwetsbaar zijn voor aanvallen van buitenaf. De beveiliging is in de loop der jaren niet voldoende ontwikkeld om de infrastructuur goed te beveiligen.

Volgens het onderzoek is er de afgelopen jaren wel het nodige geïnvesteerd om de beveiliging op te schroeven, maar deze maatregelen zijn nog onvoldoende doorgevoerd.
https://www.nu.nl/internet/5814282/rekenkamer-waterwerken-niet-goed-beveiligd-tegen-cyberaanvallen.html
\cite{hdsr30092022lichtprojectieswaterliniesluizen}
rapport Digitale dijkverzwaring: cybersecurity en vitale waterwerken 
Crisisdocumentatie is verouderd en er worden geen volwaardige pentesten uitgevoerd. Uit het onderzoek blijkt dat nog niet alle vitale waterwerken rechtstreeks zijn aangesloten op het Security Operations Center (SOC) van Rijkswaterstaat. Hierdoor bestaat het risico dat RWS een cyberaanval niet of te laat detecteert. De minister van Infrastructuur en Waterstaat moet nog stappen zetten om aan de eigen doelstellingen voor cybersecurity te voldoen
De Algemene Rekenkamer beveelt de minister van Infrastructuur en Waterstaat ook aan om het actuele dreigingsniveau te onderzoeken en te besluiten of extra mensen en middelen nodig zijn. Ook is het voor een snelle en adequate reactie op een crisissituatie van essentieel belang dat informatie up-to-date is. Pentesten zouden integraal onderdeel uit moeten maken van de cybersecuritymaatregelen bij vitale waterwerken. Verder zou moeten worden bezien of medewerkers van het SOC beter moeten worden gescreend.

\cite{kramerZeeland}
Sluis Eefde kreeg niet alleen de onderhoudsbeurt, maar werd tevens uitgebreid met een tweede sluiskolk. Zo wil Rijkswaterstaat wachttijden voor de scheepvaart voorko

\cite{gww29032021kantelendesluisdeur}
Om de lokale bemanning, die de oren en ogen waren van de sluizen, te vervangen waren camera’s, communicatielijnen en software nodig. Hoge kwaliteit videobeelden, met echte kleuren en zonder enige vertraging zijn belangrijk voor de operators en zij moeten hierop kunnen vertrouwen. Er zijn verschillende testen gedaan met diverse camera’s en cameraposities om kleurechtheid te kunnen bieden onder alle omstandigheden. Het resultaat was een perfecte kleur op alle 70+ camera’s op iedere locatie.

Vertraging van videobeelden was een cruciale factor in dit project. Het is uiterst belangrijk dat de operator op zijn beeld ziet wat er daadwerkelijk op locatie gebeurt, zonder enige vertraging. Om te laten zien of er eventuele vertraging is, is er een speciale functie gecreëerd. Deze functie laat een rood kruis zien op het scherm wanneer de vertraging meer is dan 500 miliseconden. Zo ziet de operator direct of het beeld wat hij ziet actueel is. 

Een andere functie die voor dit project is gecreëerd, is bij de videobeelden aan te geven van welke kant van de sluis het camerabeeld is. Voor de operators is het belangrijk dat ze weten vanaf welke kant het vaartuig komt en waar deze naartoe vaart. Een simpele oplossing was om een blauw kader te maken om het videobeeld van de ene kant van de sluis en geen kader om het videobeeld van de andere kant. 


\cite{thkwaterwerken}
Het crisismodel kan beter, is de derde deelconclusie van de Algemene Rekenkamer. Er is geen specifiek scenario voor een crisis die wordt veroorzaakt door een cyberaanval. Ook ontbreekt inzicht in de effecten van een cybercrisis op andere sectoren, de zogeheten cascade-effecten. Tevens is de crisisdocumentatie op onderdelen verouderd.

\cite{rekenkamercybersecWater}
Ook maakt cyberveiligheid nog geen volwaardig onderdeel uit van reguliere inspecties.’ De Rekenkamer hamert erop dat alle vitale waterinfrastructuur zo snel mogelijk op het SOC wordt aangesloten. Ook zouden werknemers van Rijkswaterstaat die belangrijke waterkeringen bedienen beter gescreend moeten worden op hun antecedenten. Sollicitanten hoeven nu slechts een Verklaring Omtrent Gedrag te overleggen, maar dat is een heel lichte toets.

\cite{hackerWaterwerk}
deltawerken

\cite{kramerZeeland}
Volgens Rijkswaterstaat is het kostbaar en technisch uitdagend om klassieke automatiseringssystemen te moderniseren en wordt er daarom vooral ingezet op detectie van aanvallen en een adequate reactie daarop.
Uit het onderzoek blijkt dat Rijkswaterstaat de afgelopen jaren zelf van alle tunnels, bruggen, sluizen et cetera heeft vastgesteld welke cyberveiligheidsmaatregelen moeten worden genomen. Een groot deel van die maatregelen (ongeveer 60\%) was begin 2018 ook al uitgevoerd, maar Rijkswaterstaat ziet onvoldoende toe op de uitvoering van het resterend deel en heeft geen actueel overzicht van de overgebleven maatregelen.
De minister heeft een aantal waterwerken die Rijkswaterstaat beheert als vitaal aangewezen. . Uit het onderzoek blijkt dat nog niet alle vitale waterwerken rechtstreeks zijn aangesloten op het Security Operations Center (SOC) van Rijkswaterstaat. De ambitie om eind 2017 bij alle vitale waterwerken cyberaanvallen direct te kunnen detecteren was in het najaar van 2018 daarmee nog niet gerealiseerd. Hierdoor bestaat het risico dat RWS een cyberaanval niet of te laat detecteert.

\cite{cybersecWaterwerk}
Over de cyberbeveiliging van gemeenten en waterschappen wordt al langer geklaagd. Zo meldde EenVandaag al in 2012 dat rioolgemalen en sluizen gemakkelijk van afstand te bedienen waren, onder meer door bijzonder slechte wachtwoorden.

\cite{cybersecWaterschappen}
Rittal doet onderzoek naarop afstand besdienbare sluizen

\cite{cybersecZuidHolland}
Beveiligde VPN
M2M Services levert aan inmiddels 220 gemeenten en waterschappen beveiligde connectiviteitsoplossingen voor het beheer van pompen, riolen en gemalen. Om risico’s op beveiligingsincidenten te voorkomen maken wij gebruik van een VPN oplossing, waarbij de verbinding optimaal beveiligd is middels encryptie en authenticatie.

\cite{waterwerkNED}
Veiligheid op het water én op het land
Gebruik van lampbewaking 

\cite{veiligheidwaterland} 



\paragraph{ethiek}


Ethiek 



persuasive technology 
https://www.humanetech.com/youth/persuasive-technology 
\cite{humanTechpersuasiveTech}
https://www.minddistrict.com/blog/persuasive-technology-new-insights-in-behavioural-change 
https://www.sciencedirect.com/book/9781558606432/persuasive-technology 
https://spectrum.ieee.org/how-persuasive-technology-can-change-your-habits 
\cite{rezenfeld01012018persuasiveTecgHabits}
https://www.frontiersin.org/articles/10.3389/frai.2020.00007/full 
\cite{aldenaini28042020persuasiveTechTrends}
https://psmag.com/environment/captology-fogg-invisible-manipulative-power-persuasive-technology-81301 
\cite{larson14062017persuasivetechmanipulates}
https://www.makeuseof.com/what-is-persuasive-technology/ 
\cite{tanzem22012022persuasivetechchanginglives}
https://lib.ugent.be/catalog/rug01:001235489 
https://cyberpsychology.eu/article/view/12270 
\cite{tikkakuddonenpersuasiveTechnology}
%%%%%%%%%%%%%%%%%%%%%%%%%%%%%%%%%%%%%%%%%%%%%%%%%%%%%%%%%%%%%%%%%
\paragraph{Afbakening van requirements Wet en regelgeving voor sluizen}
Omdat we in deit onderzoek uitgaan van het uitbreiden van bestaande sluizen is er literatuurstudie gedaan naar sluizen. In de archieven van het ministerie van verkeer en waterstaat is er het rapport Design of waterlocks\cite{CivilEngineeringDivision}.
Het programma van requirements kunnen we in ons model niet helemaal overnemen. 
Zo zijn er precondities zaols topgrafie,bestaande watersluizen,waterlevel, wind, morphologie en bodemeigenschappen.

 

Preconditions
Topography
By means of maps (land, water, river, sea, ownership, regional and zoning plans) a detailed description
of the environment should be provided, including any planned changes to existing situations, in so far
as this is of importance to the lock and adjoining lock approaches. Special attention should be paid to
historical, natural and scientific values. The maps should also show sewerage, cables and mains as well
as drainage facilities in the area concerned.
Existing lock (locks)
Water levels (approx.)
Wind
%%%%%%%%%%%%%%%%%%%%%%%%%%%%%%%%%%%%%%%%%%%%%%%%%%%%%%%%%%%%%%%%%
Morphology
Soil characteristics
Functional requirements
Functional requirements regarding navigation
%%%%%%%%%%%%%%%%%%%%%%%%%%%%%%%%%%%%%%%%%%%%%%%%%%%%%%%%%%%%%%%%%
General
Lock approaches
Primarily as part of the traffic management in locking
Stop over harbour
Harbour of refuge
Compulsory harbour
Hazardous substances
Leading jetties
Chamber and heads
The principal dimensions
The design
The facilities and equipment
Functional requirements regarding the water retaining (structure)
%%%%%%%%%%%%%%%%%%%%%%%%%%%%%%%%%%%%%%%%%%%%%%%%%%%%%%%%%%%%%%%%%
\newline \indent Dan zijn er nog de functionele eigenschappen.
 Functional requirements regarding water management
General
Limiting water loss
Separation of salt and fresh water or clean and polluted water
Water intake and discharge
%%%%%%%%%%%%%%%%%%%%%%%%%%%%%%%%%%%%%%%%%%%%%%%%%%%%%%%%%%%%%%%%%
\newline \indent Functional requirements regarding the crossing, dry infrastructure
Roads
Cables and mains
%%%%%%%%%%%%%%%%%%%%%%%%%%%%%%%%%%%%%%%%%%%%%%%%%%%%%%%%%%%%%%%%%
\newline \indent  User requirements
%%%%%%%%%%%%%%%%%%%%%%%%%%%%%%%%%%%%%%%%%%%%%%%%%%%%%%%%%%%%%%%%%
\newline \indent Levels
Locking levels
Situating the lock
Accessibility
Smoothness and safety of dealing with traffic
Design levels
Normative High Water (NHW)
Locking level high water gate
%%%%%%%%%%%%%%%%%%%%%%%%%%%%%%%%%%%%%%%%%%%%%%%%%%%%%%%%%%%%%%%%%
\newline \indent Mogelijke voorkeur voor het scheiden van verschillende soorten vaten
Separation in using line-up area, waiting area and chamber
Separating vessels during locking
Separation of vessels during over night stop
Separation for use of the leading jetty (leidende steiger)
Leading jetty for seagoing vessels
Leading jetty for inland navigation
Leading jetty for recreational navigation
%%%%%%%%%%%%%%%%%%%%%%%%%%%%%%%%%%%%%%%%%%%%%%%%%%%%%%%%%%%%%%%%%
\newline \indent Mooring facilities in chamber and lock approach
Chamber
Lock approaches
Leading jetty
%%%%%%%%%%%%%%%%%%%%%%%%%%%%%%%%%%%%%%%%%%%%%%%%%%%%%%%%%%%%%%%%%
\newline \indent Operating times
%%%%%%%%%%%%%%%%%%%%%%%%%%%%%%%%%%%%%%%%%%%%%%%%%%%%%%%%%%%%%%%%%
\newline \indent Levelling times
%%%%%%%%%%%%%%%%%%%%%%%%%%%%%%%%%%%%%%%%%%%%%%%%%%%%%%%%%%%%%%%%%
\newline \indent Operational management
%%%%%%%%%%%%%%%%%%%%%%%%%%%%%%%%%%%%%%%%%%%%%%%%%%%%%%%%%%%%%%%%%
Process descriptions
%%%%%%%%%%%%%%%%%%%%%%%%%%%%%%%%%%%%%%%%%%%%%%%%%%%%%%%%%%%%%%%%%
Normal locking process
%%%%%%%%%%%%%%%%%%%%%%%%%%%%%%%%%%%%%%%%%%%%%%%%%%%%%%%%%%%%%%%%%
Obstructions
%%%%%%%%%%%%%%%%%%%%%%%%%%%%%%%%%%%%%%%%%%%%%%%%%%%%%%%%%%%%%%%%%
High water retaining structure
%%%%%%%%%%%%%%%%%%%%%%%%%%%%%%%%%%%%%%%%%%%%%%%%%%%%%%%%%%%%%%%%%
Intake/discharge
%%%%%%%%%%%%%%%%%%%%%%%%%%%%%%%%%%%%%%%%%%%%%%%%%%%%%%%%%%%%%%%%%
Salt /freshwater or clean/polluted water
%%%%%%%%%%%%%%%%%%%%%%%%%%%%%%%%%%%%%%%%%%%%%%%%%%%%%%%%%%%%%%%%%
Information for operational management
%%%%%%%%%%%%%%%%%%%%%%%%%%%%%%%%%%%%%%%%%%%%%%%%%%%%%%%%%%%%%%%%%
Procedures and facilities for negative operational situations
%%%%%%%%%%%%%%%%%%%%%%%%%%%%%%%%%%%%%%%%%%%%%%%%%%%%%%%%%%%%%%%%%
Power supply
Levelling%%%%%%%%%%%%%%%%%%%%%%%%%%%%%%%%%%%%%%%%%%%%%%%%%%%%%%%%%%%%%%%%%
Collisions
%%%%%%%%%%%%%%%%%%%%%%%%%%%%%%%%%%%%%%%%%%%%%%%%%%%%%%%%%%%%%%%%%
Too low/too high water levels and inspections
%%%%%%%%%%%%%%%%%%%%%%%%%%%%%%%%%%%%%%%%%%%%%%%%%%%%%%%%%%%%%%%%%
Problems with ice
%%%%%%%%%%%%%%%%%%%%%%%%%%%%%%%%%%%%%%%%%%%%%%%%%%%%%%%%%%%%%%%%%
\newline \indent Operating
Situating the control building
Local control facilities
Means of communication
Choice (partly) automated and self-service
Remote control of locks
%%%%%%%%%%%%%%%%%%%%%%%%%%%%%%%%%%%%%%%%%%%%%%%%%%%%%%%%%%%%%%%%%
\newline \indent Verlichting, signalering en boarding
Verlichting (for details, see Lit. [2.1])
Ship crews and operating personnel must take into account that comfort is decreased during locking that
takes place through the night. Given the decreased visibility and orientation, extra effort is required. This
effort has to be kept as low as possible in order to prevent decreased safety. For this purpose, suitable
and economically sound illumination of the lock complex is essential.
The lighting has to be geared to the ever-increasing use of central control at locks and has to be aimed
at places where activities (manoeuvres, tying and untying, going on land) are executed.
The locations drawing the attention of the individual captain for instance, are the free area, the line-up
and waiting area, the chamber entrance, the chamber, lock grounds, chamber exit and the outlet area
to the unlit waterway. The attention of operating personnel will particularly focus on the vessels in the
line-up and waiting areas, inbound vessels, the chamber, the gates, the lock grounds and the sailing of
outbound vessels.
Given the necessity of illuminating the lock and lock approaches, a number of general minimum conditions
are set. This illumination is compulsory and could be included in the design plan:
• a clear view of the lock complex has to be provided for the benefit of orientation from the water;
• the illumination has to be sufficiently even;
• during arrival and departure dazzling, which is often caused by excessive glare of lock parts because
of cameras etc., should be prevented;
• in the control building the illumination should be adjusted to the outside environment and images
recorded as TV pictures should have such contrast and definition that the operating personnel is given
sufficient information;
• uniformity in the illumination plan for the setup of light towers, height of points of light and light
colour is desired.
In Lit. [2.1], as extension of these conditions, a number of specific recommendations are made that are
of importance to the design.

Scheepsbemanningen en bedienend personeel moeten er rekening mee houden dat het comfort tijdens het schutten afneemt
vindt de hele nacht plaats. Gezien de verminderde zichtbaarheid en oriëntatie is extra inspanning vereist. Dit
inspanning moet zo laag mogelijk worden gehouden om verminderde veiligheid te voorkomen. Voor dit doel geschikt
en economisch verantwoorde verlichting van het sluizencomplex is essentieel.
De verlichting moet zijn afgestemd op het steeds toenemende gebruik van centrale bediening bij sluizen en moet gericht zijn
op plaatsen waar werkzaamheden (manoeuvres, vast- en losmaken, aan land gaan) worden uitgevoerd.
De locaties die bijvoorbeeld de aandacht trekken van de individuele kapitein zijn de vrije ruimte, de opstelling
en wachtruimte, de kolkingang, de kolk, het sluisterrein, de kolkuitgang en het uitloopgebied
naar de onverlichte waterweg. De aandacht van het bedienend personeel zal met name gericht zijn op de schepen in de
opstel- en wachtruimtes, inkomende schepen, de kolk, de deuren, het sluisterrein en het uitvaren
uitgaande schepen.
Gezien de noodzaak van verlichting van de sluis en sluistoegangen gelden een aantal algemene minimumvoorwaarden
spelen zich af. Deze verlichting is verplicht en kan in het inrichtingsplan worden opgenomen:
• er moet vrij zicht zijn op het sluizencomplex ten behoeve van de oriëntatie vanaf het water;
• de verlichting moet voldoende egaal zijn;
• bij aankomst en vertrek verblinding, wat vaak wordt veroorzaakt door overmatige verblinding van sluisdelen doordat
van camera's e.d. moet worden voorkomen;
• in het controlegebouw dient de verlichting afgestemd te zijn op de buitenomgeving en beelden
opgenomen als tv-beelden moeten zo'n contrast en definitie hebben dat het bedienend personeel wordt gegeven
voldoende informatie;
• uniformiteit in het verlichtingsplan voor de opstelling van lichtmasten, hoogte van lichtpunten en lichtpunten
kleur is gewenst.
In Lit. [2.1] In het verlengde van deze voorwaarden worden een aantal specifieke aanbevelingen gedaan die dat wel zijn
belangrijk voor het ontwerp.
%%%%%%%%%%%%%%%%%%%%%%%%%%%%%%%%%%%%%%%%%%%%%%%%%%%%%%%%%%%%%%%%%
Vereist verlichtingsniveau
For the average value of illumination intensity on horizontal surfaces of the above-mentioned lock
parts, 10 lux is adhered to. On vertical surfaces that are more often more striking due to the perpendicular
directional view, a lower value of 3.5 lux can be used.
At a number of critical parts of the lock (both for the captain and the lock master) a larger contrast is
desired and can be achieved by stronger illumination of areas that should be in the light or providing
these with white markings. The latter is preferable. At critical lock parts such as gates and leading
jetties, the vertical illumination strength should be higher: 7 lux. On the chamber and mooring area
where accurate visibility is required, the previously stated values of 10 lux for horizontal and 3.5 lux
for vertical apply. The waiting area and the free area, where illumination is mostly for orientation,
require an illumination level of 5 lux horizontal respectively 3.5 lux vertical.

Voor de gemiddelde waarde van de verlichtingsintensiteit op horizontale oppervlakken van het bovengenoemde slot
onderdelen wordt 10 lux aangehouden. Op verticale vlakken die door de loodlijn vaker opvallender zijn
gericht zicht kan een lagere waarde van 3,5 lux worden gebruikt.
Op een aantal kritische onderdelen van de sluis (zowel voor de gezagvoerder als de sluismeester) is een groter contrast
gewenst en kan worden bereikt door sterkere verlichting van gebieden die in het licht moeten staan of moeten worden voorzien
deze met witte aftekeningen. Dit laatste heeft de voorkeur. Bij kritische sluisdelen zoals poorten en voorloop
aanlegsteigers dient de verticale verlichtingssterkte hoger te zijn: 7 lux. Op de kamer en het ligplaatsgebied
waar nauwkeurig zicht vereist is, de eerder genoemde waarden van 10 lux voor horizontaal en 3,5 lux
voor verticale toepassing. De wachtruimte en de vrije ruimte, waar de verlichting vooral ter oriëntatie is,
vereisen een verlichtingsniveau van 5 lux horizontaal respectievelijk 3,5 lux verticaal.
%%%%%%%%%%%%%%%%%%%%%%%%%%%%%%%%%%%%%%%%%%%%%%%%%%%%%%%%%%%%%%%%%
Omgevingsverlichting en begeleiding
Misleading illumination in the surrounding area can give the captain a wrong picture of the course of
the waterway that provides access to the lock chamber. This can be prevented if the waterway or the
lock complex is illuminated over a sufficient length or by adapting the surrounding illumination to the
illumination of the complex. For visual guidance, differences in illumination strength at crossings
should not exceed a factor 2.
%%%%%%%%%%%%%%%%%%%%%%%%%%%%%%%%%%%%%%%%%%%%%%%%%%%%%%%%%%%%%%%%%
Uniformiteit
For the uniformity (E) of the illumination, a minimum value of Emin/Emax = 0.3 should be adhered
to for both vertical and horizontal areas.
%%%%%%%%%%%%%%%%%%%%%%%%%%%%%%%%%%%%%%%%%%%%%%%%%%%%%%%%%%%%%%%%%
Glare
Unsafe situations due to dazzling should be avoided. The correct combination of armature, lamp and
positioning is of importance.
%%%%%%%%%%%%%%%%%%%%%%%%%%%%%%%%%%%%%%%%%%%%%%%%%%%%%%%%%%%%%%%%%
Kleurherkenning en soort lamp
The colour of the light is one of the factors in the recognition of boards and signalling. Both white
and yellow light can be used.
In the lamp choice of illumination, both high-pressure and low-pressure lamps as well as energy
saving lamps qualify. In the application of low-pressure (monochromatic) sodium (vapour) light,
colour recognition is impossible. If this is the case, separate illumination of traffic signs is recommended.

De kleur van het licht is een van de factoren bij de herkenning van borden en signalering. Beide wit
en geel licht kan worden gebruikt.
Bij de lampkeuze van verlichting, zowel hogedruk- en lagedruklampen als energie
spaarlampen komen in aanmerking. Bij de toepassing van lagedruk (monochromatisch) natrium (damp) licht,
kleurherkenning is onmogelijk. In dat geval is het aan te raden om verkeersborden apart te verlichten.
%%%%%%%%%%%%%%%%%%%%%%%%%%%%%%%%%%%%%%%%%%%%%%%%%%%%%%%%%%%%%%%%%
Marking
White markings are a good and inexpensive tool for obtaining sufficient contrast in the dark while using
little light. Marking vertical surfaces, such as guiding structures and guard walls, to support the visual
guidance of navigation is very effective.

Witte aftekeningen zijn een goed en goedkoop hulpmiddel om tijdens het gebruik voldoende contrast in het donker te krijgen
klein licht. Markering van verticale oppervlakken, zoals geleideconstructies en veiligheidsmuren, ter ondersteuning van het visuele
begeleiding van navigatie is zeer effectief.
%%%%%%%%%%%%%%%%%%%%%%%%%%%%%%%%%%%%%%%%%%%%%%%%%%%%%%%%%%%%%%%%%
Signalling
Signalling should be executed according to the stipulations of the Police Regulations on Inland
Navigation (‘Binnenvaart Politie Reglement’ (BPR))and the Rhine Navigation Police Regulations
(‘Rijnvaart Politie Reglement’ (RPR)), (Lit. [2.4]).Signal indication and lock illumination choices should be
adjusted to terrain illumination of the lock for the benefit of colour recognition; it should have sufficient
attention value.

De seingeving dient te worden uitgevoerd volgens de bepalingen van het Politiereglement Binnenvaart
Scheepvaart (Binnenvaart Politie Reglement (BPR)) en het Rijnvaartpolitiereglement
(‘Rijnvaart Politie Reglement’ (RPR)), (Lit. [2.4]). Keuzes voor signaalindicatie en slotverlichting moeten
aangepast aan terreinverlichting van de sluis ten behoeve van kleurherkenning; het zou voldoende moeten hebben
attentie waarde.
%%%%%%%%%%%%%%%%%%%%%%%%%%%%%%%%%%%%%%%%%%%%%%%%%%%%%%%%%%%%%%%%%
Boarding
Boards should be executed in accordance with the stipulations of the BPR and RPR, (Lit. [2.4]).The colour
recognition could be (substantially) reduced due to the terrain illumination. Sufficient attention should
be paid to adjusting the illumination or to separate board illumination.
%%%%%%%%%%%%%%%%%%%%%%%%%%%%%%%%%%%%%%%%%%%%%%%%%%%%%%%%%%%%%%%%%
Verlichtingsplan
The user requirements for illumination should be incorporated in an illumination design plan.
The chamber depth (distance between low normative water level and the lock coping) and the chamber
width are of great importance. In Lit. [2.1] examples are provided for a number of chamber width
categories (5-13 m, 13-20 m, 20-24 m, larger than 24 m; chamber depth about. 5 m) of the resulting
illumination characteristics (such as illumination strength and uniformity), departing from the relationship
between lock design and the given characteristics of illumination installation (such as positioning
and illumination facilities).
%%%%%%%%%%%%%%%%%%%%%%%%%%%%%%%%%%%%%%%%%%%%%%%%%%%%%%%%%%%%%%%%%
\newline \indent Stroomvoorziening
Emergency power supply is required for vital parts of the installation so that, in case of malfunction,
it can automatically take over the energy supply within minutes. A no-break facility is required for
installation parts that lose data in case of power loss. In addition, emergency lights should be present.

In essence, power is obtained from the public network. In consultation with the local power company,
assessments have to be made about where this is possible and whether the connection contains
sufficient capacity or whether this will have to be adjusted. Of importance is the total capacity required,
voltage variations and frequency of the energy to be supplied. In addition to capacity for lock operation,
the capacity for construction (civil and steel) will have to be determined. It could be taken into consideration
whether the cables for construction could later become part of the supply for the lock.
The lock complex should contain the necessary facilities for high tension, transformers and low-tension
equipment. In addition, room is reserved and facilities provided for cable location lines from the low-tension
area to the various lock parts (cable racks, cable channels, cable shafts, lead-through pipes etc.)
Take into account the other cables and mains required for lock operation as well as those for third parties
(Par. 2.3.4.2). For emergency power supply generators and no-break installations, see Par. 2.4.6.3.

Noodvoorzieningen voor stroomtoevoer i vereist voor bepaalde delen van de installatie, in geval van een storing kan deze binne enkele minuten leveren.

Een no-break facicilieit is vereist voor de ondeerdelen die data verliezen in gevala van sstroomuitval.
Het sluizencomplex moet gaciliteiten hebben voor hoogspanning, transformatoren en laagspanningsapparatuur.
%%%%%%%%%%%%%%%%%%%%%%%%%%%%%%%%%%%%%%%%%%%%%%%%%%%%%%%%%%%%%%%%%
\newline \indent Beschikbaarheid
Introduction
Causes of non-availability
Water levels above and below locking levels
Guidelines on the boundaries of locking levels are provided in Par. 2.4.1.1 (maximum and minimum
locking levels). Overall, this results in non-availability smaller than 2% of the time.
The specific boundaries should be set on economic grounds.
Too much wind, bad visibility

De beschikbaarheid van een sluis kan beinvloed worden door een te hoog  waterlevel boven de sluis.
Dan is er nog de mogelijkheid op te veel wind en slecht zicht.
%%%%%%%%%%%%%%%%%%%%%%%%%%%%%%%%%%%%%%%%%%%%%%%%%%%%%%%%%%%%%%%%%
Storingen aan installaties, bedieningsmechanismen en werking. Er moeten oplossingen komen  zodat ern signalen worden gegeven wanneer een storing zich voordoet, een betre reactie op signalen en  reserveonderdelen.
Based on the previously mentioned economic considerations, requirements will have to be drafted for
the design of the lock or the series of locks for the acceptable risk of failure of these facilities. As an
example, the values applied for the renovation of the ‘Zuider- en de Kleine sluisin IJmuiden’ are stated
(Lit. [2.13]). Not available due to:
• malfunction installations : ≤ 0,5% of the time
• malfunction operating mechanisms : ≤ 0,5% of the time
• malfunction operation : ≤ 0,25% of the time
The number of times that malfunction occurs could also be a determining factor.
Not every malfunction results in complete obstruction. The objective is to limit the duration of the malfunction
as much as possible (alerting, responding, spare parts).
For emergency power supply and no-break installations, please see Par. 2.4.6.3.

%%%%%%%%%%%%%%%%%%%%%%%%%%%%%%%%%%%%%%%%%%%%%%%%%%%%%%%%%%%%%%%%%
Botsingen
For non-availability due to collisions, at best a forecast can be made, based on the information available
for similar locks with a corresponding navigation volume. As an example, the ‘Zuidersluis bij IJmuiden’
(Lit. [2.13]) is mentioned, where the non-availability due to significant damage due to collisions amounted
to 17 hours per annum (about 0.2% of the time). Other locks could provide a different picture.
Within economically acceptable boundaries, the objective will be to limit the collisions and consequences
thereof. The accent is placed on gates (and operating mechanisms), moveable bridges and – to a
lesser degree – on berthing jetties and guide structures.
Measures to decrease risk of collision are, among others:
• good design of approach jetties (Par. 2.3.1.3 and 2.4.2.2);
• positioning of the flooring of moveable bridges – in opened condition – outside the outer walls of the
lock (Par. 2.3.4.1);
• anti-collision structures in front of the gates (Par. 2.4.11.1). This is an expensive facility that will only
be applied in special cases;
• protection of operating mechanism on gates. Preventing collisions with the operating mechanism can
be effected by fitting a tail end to the gate and connecting this to the operating mechanism
(Renovation Oranjesluizen). An extended operating mechanism chamber could also be used so that
the vulnerable cylinder rod cannot be hit in the lock (Middensluis IJmuiden).
Measures to limit the duration of the repairs (obstruction) are, among others, having the spare gates and
spare parts available (Par. 2.5.2 en 2.5.3).
Maatregelen om het aantal botsingen te voorkomen zijn:
Good ontwerp voor aanvaarstijgers.
positionering van de vloer van beweegbare poorten
anti-bots structuren aan de voorkant van de sluisdeuren
bescherming van werkende meschanismen van de slusdeuren

Maintenance
%%%%%%%%%%%%%%%%%%%%%%%%%%%%%%%%%%%%%%%%%%%%%%%%%%%%%%%%%%%%%%%%%
\newline \indent Constructies beschermen tegen schade
%%%%%%%%%%%%%%%%%%%%%%%%%%%%%%%%%%%%%%%%%%%%%%%%%%%%%%%%%%%%%%%%%
Aanrijdbeveiliging voor poorten
Mitre gates and pivot (leaf) gates must be fitted with wood fender on the outside surfaces of the
opened gates to protect the construction from damage caused by inbound and outbound vessels. Wood
fender can also be fitted to other gates in places where they might be hit by vessels.
In special circumstances (for instance Wijk bij Duurstede, Tiel, Belfeld, Panheel, Twenthe-kanaal) trap
constructions are positioned in front of the closed gates. The energy of vessels that do not stop in time
is absorbed here and the construction prevents the gates from being hit (see par. 17.3.3). For this purpose,
cables (cable nets) and friction drums can be used. For the circumstances and setup of these constructions,
we refer to Lit. [2.15]. It does concern expensive constructions for which the investments will
have to be weighed against the risk of failure of the water retaining structure, the navigation interests
etc.
Anti-collision devices protecting lock gates could be economically sound at high-lift locks.

Verstekpoorten en draaipunt. In bijzondere reegvallen staan er valconstructies bij de gesloten poorten voor vaartuigen die niet op tijd stoppen. zodoende wordt de klap opgevangen.
Anti-bots apparaten die de sluisdeuren beschermen  zijn economisch verantwoor bij hoge liftsluizen.
%%%%%%%%%%%%%%%%%%%%%%%%%%%%%%%%%%%%%%%%%%%%%%%%%%%%%%%%%%%%%%%%%
Aanrijdbeveiliging voor beton- en damwandconstructies
Construction surfaces against which vessels moor or along which they shave, have to be as smooth as
possible in order to guide well and limit potential damage (construction and vessel). For inland navigation,
a concrete structure meets the requirements. In the case of other construction materials such as
sheet pile, the flat surface should be made of wooden or synthetic posts and rails wherever possible. This
system can be limited to the day surfaces that vessels meet.
Constructievlakken waar schepen aanmeren of waarlangs ze scheren, moeten zo glad mogelijk zijn
mogelijk om goed te begeleiden en mogelijke schade (constructie en vaartuig) te beperken. Voor de binnenvaart,
een betonconstructie voldoet aan de eisen. In het geval van andere bouwmaterialen zoals
damwand, het vlakke oppervlak dient zoveel mogelijk te bestaan uit houten of kunststof palen en rails. Dit
systeem kan worden beperkt tot de dagoppervlakken die schepen ontmoeten.

Additional facilities are necessary in places where concrete surfaces are interrupted or come to an end
because of expansion joints, gate and ladder recesses. In the case of expansion joints, it will be sufficient
to use (sizeable) bevelled edges, steel corner protection profiles should be applied in recesses. Corner
guards made of tropical hardwood can also be fitted, especially where it concerns rugged navigation
such as tug-pushed dumb barges and sea-going vessels. As protection from hawsers etc, the top of the
wall should be fitted with steel capstone profiles. In locks for large ocean going vessels, floating wooden
frames (the Netherlands) or rubber wheel fenders (Belgium) are used.
Op plaatsen waar betonvlakken worden onderbroken of ophouden, zijn aanvullende voorzieningen nodig
vanwege dilatatievoegen, poort- en ladderuitsparingen. In het geval van dilatatievoegen is dit voldoende
om (flinke) afgeschuinde randen te gebruiken dienen stalen hoekbeschermingsprofielen in uitsparingen te worden aangebracht. Hoek
ook beschermkappen van tropisch hardhout kunnen worden aangebracht, zeker als het om ruige navigatie gaat
zoals sleepboten en zeeschepen. Als bescherming tegen trossen enz., de bovenkant van de
wand dient voorzien te zijn van stalen deksteenprofielen. In sluizen voor grote zeeschepen, drijvend van hout
frames (Nederland) of rubberen wielspatborden (België) worden gebruikt.

The facilities are intended to minimize damage to vessels and constructions, but also to prevent backing
up and friction effects during mooring and unmooring of vessels with large side surfaces, thereby
decreasing the pass through time.

%%%%%%%%%%%%%%%%%%%%%%%%%%%%%%%%%%%%%%%%%%%%%%%%%%%%%%%%%%%%%%%%%
Voorzieningen tegen vandalisme
Lightning protection
%%%%%%%%%%%%%%%%%%%%%%%%%%%%%%%%%%%%%%%%%%%%%%%%%%%%%%%%%%%%%%%%%
Safety
Voorzieningen voor drenkelingen
For rescuing people who accidentally end up in the water, ladders should be fitted to the chamber wall
and to (high) smooth walls in the lock approach. At the upper end, these ladders are equipped with
handgrips. For offering help from the quayside, life-saving devices (life buoy, hooks) should be present
on the lock coping in a clearly visible place. Ladders in the chamber and the lock approach also have an
accessibility function. For locations and distances, also see par. 2.4.13.2 and 2.4.13.3.
Voor het redden van drenkelingen moetn er ladders zijn.
%%%%%%%%%%%%%%%%%%%%%%%%%%%%%%%%%%%%%%%%%%%%%%%%%%%%%%%%%%%%%%%%%
Veiligheidsvoorzieningen
Design and management of safety facilities of personnel will be executed in accordance with Health and
Safety Regulations, construction regulations, labour regulations and safety regulations (CE directives).
A number of facilities are mentioned below.
Railings are attached to the top of gates. If the lock coping is more than 2.5 m above minimum locking
level, fencing is placed behind the bollards. This fencing is always desirable where it concerns recreational
navigation and where tourists are allowed on the lock coping.
In the technical areas, workshops, bridges, control portals, rolling gate casings and the like, where work
is executed and people walk around where there are differences in height in the surrounding area,
railings are provided. From a height difference of 0.60 m or more with the surrounding area, a railing
has to be provided at 1 – 1.10 m. Height differences of more than 12 m require the railing to be placed
at a height of 1.20. Often, additional protection against falling is provided from height differences of
more than 2.5 m such as safety lines, lifelines, harness belts and the like.

Steel ladders should not be in regular use. Straight stairs, a spiral staircase or step ladders should be
installed. Ladders can be used between vertical (90o) and 75o and be equipped with simple round rungs.
The ladder width is between 0.38 and 0.46 m and the step distance is between 0.25 – 0.20 m.
If the ladder connects with the (landing) coping, the distance between the styles of the ladder should be
enlarged to 0.60 and it has to be connected to the railing. If the ladders are higher than 3.60 m, they
have to be provided with a safety cage. This cage has an inside measurement of 0.76 m and starts from
2.40 m above the ground. At ladder heights above 6 m, an intermediate landing is required.

Basement chambers that could possibly flood (for instance those of operating mechanisms of mitre
gates) have to be provided with an exit that can be opened from the inside. In addition, sufficient
natural ventilation will be required as well as plunger pumps.
The area in which the operating mechanisms are working need to be shielded from the environment to
ensure that nobody gets stuck between machine parts. The lock complex should have sufficient and visible
First Aid provisions.

Ontwerp en beheer van veiligheidsfaciliteiten voor personel worden uitgevoerd in overeenstemming met de gezondheids-veiligheidsregelgeving, constructieregelgeing,arbeidsregelgeving en veilgieheidsregeveving. Enkele voorbeelden zijn traliewerk, hekwerk, stalen ladders, kelder kamers en eerste hulp kits
%%%%%%%%%%%%%%%%%%%%%%%%%%%%%%%%%%%%%%%%%%%%%%%%%%%%%%%%%%%%%%%%%
Brand blussen
Toegankelijkheid van sluis en sluistoegangen
Lock Infrastructure
Accessibility of vessels in the lock
Accessibility of vessels in the lock approache
Accessibility of vessels with dangerous goods in the lock approaches
%%%%%%%%%%%%%%%%%%%%%%%%%%%%%%%%%%%%%%%%%%%%%%%%%%%%%%%%%%%%%%%%%
\newline \indent Supplemental client wishes
%%%%%%%%%%%%%%%%%%%%%%%%%%%%%%%%%%%%%%%%%%%%%%%%%%%%%%%%%%%%%%%%%
\newline \indent Eisen aan de levensduur
Ontwerp levensduur sluizencomplex
Steel parts
Electrical installations
Hardware and software
Damwand constructies
Leidende structuren
%%%%%%%%%%%%%%%%%%%%%%%%%%%%%%%%%%%%%%%%%%%%%%%%%%%%%%%%%%%%%%%%%
Maintenance requirements
Maintenance strategy
The maintenance strategy will mainly be based on the requirements regarding the safety of the retaining
structure (par. 2.3.2), the availability for lock operation (par 2.4.10) and the life span (2.4.15). The
external appearance of the structure will also play a role in the strategy (building inspection). With the
exception of the safety requirements, which are fixed, it concerns an assessment between the aggregate
costs of investments and capitalized maintenance, and the interest of obstructions for navigation. An
example is to consider applying 2 horizontal roller-bearing gates per head for a maritime navigation lock
(par. 2.5.2). The optimization of the materials, maintenance choices etc. within the given design life span
of a lock is discussed in par 2.4.15. Environmental requirements necessitate certain maintenance activities
to be executed in closed areas. Providing these facilities on site could be costly and it could be attractive
to have these activities executed by third parties.
Overall, the objective is to incur a minimum of aggregate costs as well as provide the largest service
provision to navigation. The latter includes a limitation of the number and duration of obstructions for
maintenance (par. 2.4.10) and attention for limited passage during maintenance. Please refer to the
modules of ‘Raamwerk Onderhoud van Natte Kunstwerken’ (Lit. [2.20]), which is drafted by the Civil
Engineering Division of the Ministry of Transport and Public Works. At present, the following modules
are available: "Keuze van onderhoud voor een puntdeur", "Damwanden" and "Ducdalven en remmingwerken".
Based on this strategy, maintenance plans, books and schedules will have to be drafted for the various
parts. Supplemental to this, measures and procedures for navigation during maintenance will have to be
drafted.

De onderhoudstrategie wordt bepaald of basis van de vasthoudende structuur, de mogelijkheden voor sluisbediening en de levenscyclus. het uiterlijk van de structuur speelt een belangrijke rol bij de strategie, namelijk gebouw inspectie. Bahalve veiligheidseisen gaat dit over de toetsing van  de totale investeringskosten en noodzakelijk onderhoud, en de behoefte aan belemmering voor nativatie. Het optimaliseren van materialen, onderhoudskeuzen binnen de levenscyclus van de sluis. Omgevingseisen maken het nodg onderhoudsactiviteten uit  te voeren in afgesloten ruimten.
%%%%%%%%%%%%%%%%%%%%%%%%%%%%%%%%%%%%%%%%%%%%%%%%%%%%%%%%%%%%%%%%%
2.5.2 Spare
Reserve poorten
Onderdelen en materialen
%%%%%%%%%%%%%%%%%%%%%%%%%%%%%%%%%%%%%%%%%%%%%%%%%%%%%%%%%%%%%%%%%
Slot openleggen (of niet)
Nowadays, it is no longer usual to lay open the complete lock for maintenance. The reasons are that it
is often too costly (measures required against floating up) and that the main construction of chamber
and heads are maintenance free, the probable exception being wood fenders for sheet pile constructions
and floating frames at sea locks. The latter parts should be easy to replace. Incidental repairs to head
constructions could be executed by divers or in diving bells.
Inspection and maintenance focus on gate supports (sill and side seals), fulcrums, and gate conduction,
in other words, parts that are located in the head. There are two possibilities:
1. Lying open a head, for which stop log weirs or dewatering weirs and rabbets are necessary.
2. Removable pivot-inspection chambers and other local steel dewatering means for the fulcrums,
support and gate condition. This also includes the dewatering stop logs for the gate recesses for lift
and roller-bearing gates.
Gate supports and rabbets are also required for the drainage. These means for water removal are stored
in the near vicinity in a highly accessible place and could possibly be used for several locks.
The choice between two possibilities depends on the inspection and maintenance frequency, the costs
and the duration of the obstruction for navigation. Option 1, in which too much space is laid open is, in
essence, usually only applied at smaller locks.

Tegenwoordig is het niet meer nodig om een complete sluis open te leggen voor onderhoud. De reden is dat dit duur is en dat de hoofd constructie van de kamer en hoofden  vrij zijn van onderhoud afgezien van houden spatborden voor damwandbouwers en zwevende kozijnen.

Poortsteuningen en ponningen zijn nodig voor de drainage
%%%%%%%%%%%%%%%%%%%%%%%%%%%%%%%%%%%%%%%%%%%%%%%%%%%%%%%%%%%%%%%%%
Toegankelijkheid voor het personeel
%%%%%%%%%%%%%%%%%%%%%%%%%%%%%%%%%%%%%%%%%%%%%%%%%%%%%%%%%%%%%%%%%
Monitoring( Toezicht houden)
Monitoring is a permanent measuring and registration system for normative parameters for the condition
of structures, the loads and stresses that they are submitted to and the degree in which corrosion
processes have progressed. Even though the application in construction is still limited, it is necessary to
keep up with the rapid developments. Monitoring is useful, certainly for places of lock structures that are
difficult to inspect (for instance at soil facing side) and for erosion processes that are hardly visible on the
surface (such as chloride penetration).
Monioren betekent het permanent meten en registeratie systeemvoor normatieve parameters voor de conditie van de structuren,ladingen. Het iis belangrijk alle ontwikkelingen in de gaten te houden. Monitoren is nuttig, zeker voor onderdelen van de sluis die moeilijk te inspectiren zijn zoals de bodem en voor erosie processen die moelijk zichtbaar zijn vanaf het oppervlak.
Cathodic protection can be used as a monitoring system at the same time.
Electrical installation, hard- en software
Storage areas and workshops
Environmental requirements in the use phase
Aesthetics
%%%%%%%%%%%%%%%%%%%%%%%%%%%%%%%%%%%%%%%%%%%%%%%%%%%%%%%%%%%%%%%%%
\newline \indent In ons model houden we geen rekening met omgevingseisen zoals de materialen gebruiket voor de bouw, recreatie, bodemvervuiling, grondwaterverlies. Oo is er geen rekening gehouden met verkeer, communicatiekabels onderwater en netspanningskabels.

Environmental requirements with regard to building materials
Recreation
Environmental requirements in the construction phase
Required building site and final grounds
Polluted soil
Groundwater withdrawal
Upkeep/maintenance of road and navigation traffic, cables and mains
Upkeep/maintenance of the water retaining structure
%%%%%%%%%%%%%%%%%%%%%%%%%%%%%%%%%%%%%%%%%%%%%%%%%%%%%%%%%%%%%%%%%
\newline \indent Permits and procedures at the construction of a lock
Construction permits and zoning plan amendments
Demolition permit
Flood Defence Act
Environmental Management Act (M.E.R.)
Act on Earth Removal
Pollution of Surface Waters Act
Groundwater Act permit
Water management Act
Soil Protection Act
Nature Conservation Act
Management of Waterways and Public Works Act (Wet beheer RWS-werken)
Noise Abatement Act
Provincial Road Ordinance
Building Materials (Soil and Surface Waters Protection) Decree
Other permits and exemptions
Standards and guidelines
Standards
Guidelines
%%%%%%%%%%%%%%%%%%%%%%%%%%%%%%%%%%%%%%%%%%%%%%%%%%%%%%%%%%%%%%%%%
\paragraph{Checklist}


\paragraph{Analyse}
\paragraph{Conclusie}
%%%%%%%%%%%%%%%%%%%%%%%%%%%%%%%%%%%%%%%%%%%%%%%%%%%%%%%%%%%%%%%%%

\hoofdstuk{Requirements}
\paragraph{Inleiding}

De meeste sluizen die zich in Nederland bevinden zijn schutsluizen; deze sluizen zijn bedoeld om boten, zowel vrachtschepen als pleziervaart afhangend van de locatie van de sluis, te verwerken. Om deze reden gaan wij deze dus ook verwerken in ons model. Mocht een sluis niet bedoeld zijn om boten te verwerken, dan zou dit model alsnog toegepast kunnen worden opp desbetreffende sluis.
Boten worden toegevoed aan de queue. Hoe dit gebeurt, dat ligt aan de specifieke sluis.  Sinds wij een template maken, hoeven wij geen rekening te hounden met hoe de schepen in de queue komen. Het enige wat wij hoeven te doen, is de data verwerken.





\paragraph{Aandachtspunten}
\begin{enumerate}
	\item Voorrang tussen schepen onderling in de sluis?
	\item Hoe lang mag een schip zich in de sluis bevinden?
\end{enumerate} 




\subparagraph{Afbakening}
\begin{itemize}
	\item Wat doet de sluis niet.
	\item De sluiss houdt geen rekening met links of rechtsrijdend verkeer vanuit de zeevaart
	\item De sluis heeft geen queue met daarin een id gekoppeld aan de sluis.
	\item De waterpomp wordt alleen aan en uitgezet
	\item De waterpomp houdt geen rekening met waterstand
	\item Houdt geen rekening met een schip in de sluis dat is blijven hangen.
\end{itemize}


\begin{enumerate}
	\item Een tweetal sluisdeuren. 
	\item Een sluiskolk waarin de schepen in- enuitvaren
	\item een stoplicht om een signaal af te geven voor invaren en uitvaren.
	\item Een nivelleermachine zorgt ervoor dat het water in de sluis op het gewenste niveau wordt gebracht
	\item Een control-system dat ervoor zorgt dat de opdrachten van de sluisbeheerder (geautomatiseerd) worden uitgevoerd
\end{enumerate}

Een schip komt aanvaren en meld zich aan bij de sluismeester. De sluismeester geeft een signaal aan het controlsystem voor het openen van de sluisdeuren, nadat geccontroleerd is of de nivelleermachine al klaar is. Als er ruimte is voor een invarend schip mag het schip dat zoich heeft aangemeld en toestemming heeft  in de sluis varen. Op het moment dat de sluis vol is gaan de sluisdeuren dicht. Eenmaal afgesloten kan de nivelleermachine beginnen om het water in de sluiskolk op het gewenste waterpeil te brengen. Als dit nivelleerprces is afgerond geeft  het controlsystem daan da de sleusdeuren open kunnen.  Als de sleusdeuren open zijn en het uitvaarsignaal is op groen dan moet het schip in de sluis de sluis uitvaren.

Uit het zojuist genoemnde scenario valt het volgende op te maken.
\begin{enumerate}
	\item Een schip geeft een signaal aan een sluismeester.
	\item Er wordt gekeken of er wel plek is in de sluis .
	\item Er wordt gekeken of de nivelleermachine is afgerond.
	\item Er wordt gekeken wat het niveo van de waterpeil in de sluiskolk is.
	\item Er wordt gekeken of de sluisdeuren gereed zijn voor invarende schepen.
\end{enumerate}

\paragraph{overzicht}

\paragraph{Conclusie}

%%%%%%%%%%%%%%%%%%%%%%%%%%%%%%%%%%%%%%%%%%%%%%%%%%%%%%%%%%%%%%%%%
 
\hoofdstuk{Uppaal model}


\paragraph{Inleiding}






	\subsubsection{De computation tree}

\xymatrix@ur@!R=2pc{%
	*+<1pc>[o][F-]{q_0}  \ar@(l,l)[]^<<<<{start} \ar@/^/[r]^0  \ar@/_/[d]_1 
	& *+<1pc>[o][F-]{q_1} \ar@(ul,ur)[]^{0}  \ar@/^/[d]^1 \\
	*+<1pc>[o][F-]{q_2} \ar@(dr,dl)[]^{1} \ar@/_/[r]_0 
	& *+<1pc>[o][F=]{q_3} \ar@(l,l)[]^>>>>{start}  \ar@(dr,dl)[]^{1} \\
 
 }








\begin{tikzpicture}[>=latex',scale=0.5]
	% set node style
	
	\begin{dot2tex}[dot,tikz,codeonly,styleonly,options=-s -tmath]
		digraph G  {
			node [style="n"];
			p [label="+"];
			t [texlbl="\LaTeX"];
			6
			8
			10-> p;
			6 -> t;
			8 -> t;
			t -> p;
			{rank=same; 10;6;8}
		}
	\end{dot2tex}
	\begin{pgfonlayer}{background}
		\draw[rounded corners,fill=blue!20] (6.north west) -- (8.north east) -- (t.south east)--cycle;
	\end{pgfonlayer}
\end{tikzpicture}


\[\begin{tikzcd}[column sep=1cm]
	ABCDE\arrow[r, leftrightarrow, "\times"{anchor=center},"\text{label}","\text{label}"{below}]\arrow[d] & F\arrow[r]\arrow[d] & G\arrow[rr]\arrow[d] && H\arrow[d]\\
	ABCDEFGH\arrow[r, leftrightarrow, "\times"{anchor=center}]\arrow[d] & II\arrow[r]\arrow[d] & JJ\arrow[rr,"\text{very long label}"]\arrow[d] && KK\arrow[d]\\
	ABCD\arrow[r] & EEE\arrow[r] & FFF\arrow[rr] && GGG
\end{tikzcd}\]




\paragraph{Models}

\subparagraph{Maincontroller}
\[
\begin{tikzcd}%[every arrow/.append style=dash]  uncomment to remev arrowa
	& \tikz{\node[draw,circle]{1}} \ar{d}&  & \tikz{\node[draw,circle]{2}} \ar{d}  \ar{r} &  \ar{r} & \ar{r} & \ar{r} &  \ar{r}& \tikz{\node[draw,circle]{2}}\ar{d} \\
	\tikz{\node[draw,circle]{4}} \ar{d} & \tikz{\node[draw,circle]{2}}  \ar[bend right=15]{l} \ar{d} & & \ar{d} & \tikz{\node[draw,circle]{2}} &\tikz{\node[draw,circle]{2}}&\tikz{\node[draw,circle]{2}}&& \tikz{\node[draw,circle]{2}} \ar{d} &\\
	\tikz{\node[draw,circle]{4}} \ar{u} \ar{r}  & \tikz{\node[draw,circle]{3}} \ar{r}  & \tikz{\node[draw,circle]{5}} \ar{r} &  \tikz{\node[draw,circle]{6}} \ar{r}  \ar{ru} & \tikz{\node[draw,circle]{7}} \ar{r} & \tikz{\node[draw,circle]{7}} \ar[crossing over]{ul}  \ar{r} & \tikz{\node[draw,circle]{8}} \ar{r} & \tikz{\node[draw,circle]{9}} \ar{r} \ar{d} & \tikz{\node[draw,circle]{10}}  \\
	& & & &\tikz{\node[draw,circle]{1}} \ar[crossing over]{ul} \ar{ru}  & & \tikz{\node[draw,circle]{2}}  \ar[bend right=15]{r} & \tikz{\node[draw,circle]{2}} \ar[bend right=15]{l} \ar[bend right=15]{r}  & \tikz{\node[draw,circle]{2}} \ar[bend right=15]{l}
\end{tikzcd}
\]
 

\begin{tikzpicture}[>=latex',shorten >=1pt,node distance=2cm,on grid,auto,scale=0.2]

	\node[state] (q0-e) {$q_0/\epsilon$};
\node[state] (q0-1) [below right=of q0-e] {$q_0'/1$};
\node[state] (q1-0) [above right=of q0-1] {$q_1/0$};
\node[state] (q2-1) [below right=of q1-0] {$q_2/1$};

\node[state] (q3-0) [below left=of q0-1] {$q_3/0$};
\node[state,accepting] (q3-1) [below right=of q0-1] {$q_3'/1$};
\node[state] (0) [ left=of q3-0] {$q_3/0$};
\node[state] (1) [ left=of 0] {$q_3/0$};

\node[state,initial,accepting] (0) [ left=of 1] {$q_3/0$};

\node[state] (3) [ below right=of q2-1] {$2$};
\node[state] (4) [  right=of 3] {$4$};
\node[state] (5) [  right=of 4] {$5$};
\node[state] (6) [  right=of 5] {$6$};

\node[state] (7) [  above=of 1] {$7$};
\node[state] (8) [  above=of 7] {$8$};
\node[state] (9) [  right=of 5] {$9$};

\node[state] (10) [ below  left=of 4] {$10$};
\node[state] (11) [ below  right=of 4] {$11$};

\node[state] (12) [   above=of 5] {$11$};

\node[state] (13) [   above=of 12] {$11$};

\node[state] (14) [ below right  =of q3-0] {$11$};
\node[state] (15) [   below right =of 14] {$15$};
\node[state] (16) [   above right =of 15] {$16$};


\path[->] (q0-e) edge node {a} (q1-0);
\path[->] (q0-e) edge node {b} (q3-0);
\path[->] (q0-1) edge node {a} (q1-0);
\path[->] (q0-1) edge [bend right] node {b} (q3-0);
\path[->] (q1-0) edge node {a} (q3-1);
\path[->] (q1-0) edge node {b} (q2-1);
\path[->] (q2-1) edge node {a} (q0-1);
\path[->] (q2-1) edge node {b} (q3-1);
\path[->] (q3-0) edge node {a} (q3-1);
\path[->] (q3-0) edge [bend right] node {b} (q0-1);
\path[->] (q3-1) edge [loop below] node {a} (q3-1);
\path[->] (q3-1) edge node {b} (q0-1);

\path[->] (4) edge [bend right] node {b} (10);	
\path[->] (10) edge [bend right] node {b} (4);


	
\path[->] (4) edge [bend right] node {b} (11);	
\path[->] (11) edge [bend right] node {b} (4);
	

\end{tikzpicture}

 
\subparagraph{Labeling functions}



\subparagraph{Schip}


\begin{tikzpicture}[>=latex',shorten >=1pt,node distance=2cm,on grid,auto,scale=0.2]
	
	\node[state] (q0-e) {$q_0/\epsilon$};
	\node[state] (q0-1) [below right=of q0-e] {$q_0'/1$};
	\node[state] (q1-0) [above right=of q0-1] {$q_1/0$};
	\node[state] (q2-1) [below right=of q1-0] {$q_2/1$};
	
	\node[state] (q3-0) [below left=of q0-1] {$q_3/0$};
	\node[state,accepting] (q3-1) [below right=of q0-1] {$q_3'/1$};
	\node[state] (0) [ left=of q3-0] {$q_3/0$};
	\node[state] (1) [ left=of 0] {$q_3/0$};
	
	\node[state,initial,accepting] (0) [ left=of 1] {$q_3/0$};
	

	
	\path[->] (q0-e) edge node {a} (q1-0);
	\path[->] (q0-e) edge node {b} (q3-0);
	\path[->] (q0-1) edge node {a} (q1-0);
	\path[->] (q0-1) edge [bend right] node {b} (q3-0);
	\path[->] (q1-0) edge node {a} (q3-1);
	\path[->] (q1-0) edge node {b} (q2-1);
	\path[->] (q2-1) edge node {a} (q0-1);
	\path[->] (q2-1) edge node {b} (q3-1);
	\path[->] (q3-0) edge node {a} (q3-1);
	\path[->] (q3-0) edge [bend right] node {b} (q0-1);
	\path[->] (q3-1) edge [loop below] node {a} (q3-1);
	\path[->] (q3-1) edge node {b} (q0-1);
	

	
	
	
	
\end{tikzpicture}

\subparagraph{Deur}

\begin{tikzpicture}[>=latex',shorten >=1pt,node distance=2cm,on grid,auto,scale=0.2]
	
	\node[state] (q0-e) {$q_0/\epsilon$};
	\node[state] (q0-1) [below right=of q0-e] {$q_0'/1$};
	\node[state] (q1-0) [above right=of q0-1] {$q_1/0$};
	\node[state] (q2-1) [below right=of q1-0] {$q_2/1$};
	
	\node[state] (q3-0) [below left=of q0-1] {$q_3/0$};
	\node[state,initial,accepting] (q3-1) [below right=of q0-1] {$q_3'/1$};

	


	
	
	\path[->] (q0-e) edge node {a} (q1-0);
	\path[->] (q0-e) edge node {b} (q3-0);
	\path[->] (q0-1) edge node {a} (q1-0);
	\path[->] (q0-1) edge [bend right] node {b} (q3-0);
	\path[->] (q1-0) edge node {a} (q3-1);
	\path[->] (q1-0) edge node {b} (q2-1);
	\path[->] (q2-1) edge node {a} (q0-1);
	\path[->] (q2-1) edge node {b} (q3-1);
	\path[->] (q3-0) edge node {a} (q3-1);
	\path[->] (q3-0) edge [bend right] node {b} (q0-1);
	\path[->] (q3-1) edge [loop below] node {a} (q3-1);
	\path[->] (q3-1) edge node {b} (q0-1);
	

	
\end{tikzpicture}

\subparagraph{Stoplicht}


\begin{tikzpicture}[>=latex',shorten >=1pt,node distance=2cm,on grid,auto,scale=0.2]
	
	\node[state] (q0-e) {$q_0/\epsilon$};
	\node[state] (q0-1) [below right=of q0-e] {$q_0'/1$};
	\node[state] (q1-0) [above right=of q0-1] {$q_1/0$};
	\node[state] (q2-1) [below right=of q1-0] {$q_2/1$};
	
	\node[state,initial,accepting] (q3-0) [below left=of q0-1] {$q_3/0$};
	\node[state,accepting] (q3-1) [below right=of q0-1] {$q_3'/1$};



	
	
	\path[->] (q0-e) edge node {a} (q1-0);
	\path[->] (q0-e) edge node {b} (q3-0);
	\path[->] (q0-1) edge node {a} (q1-0);
	\path[->] (q0-1) edge [bend right] node {b} (q3-0);
	\path[->] (q1-0) edge node {a} (q3-1);
	\path[->] (q1-0) edge node {b} (q2-1);
	\path[->] (q2-1) edge node {a} (q0-1);
	\path[->] (q2-1) edge node {b} (q3-1);
	\path[->] (q3-0) edge node {a} (q3-1);
	\path[->] (q3-0) edge [bend right] node {b} (q0-1);
	\path[->] (q3-1) edge [loop below] node {a} (q3-1);
	\path[->] (q3-1) edge node {b} (q0-1);
	

	
	
	
\end{tikzpicture}

\subparagraph{pomp}



\begin{tikzpicture}[>=latex',shorten >=1pt,node distance=2cm,on grid,auto,scale=0.2]
	
	\node[state] (q0-e) {$q_0/\epsilon$};
	\node[state] (q0-1) [below right=of q0-e] {$q_0'/1$};
	\node[state] (q1-0) [above right=of q0-1] {$q_1/0$};
	\node[state] (q2-1) [below right=of q1-0] {$q_2/1$};
	
	\node[state] (q3-0) [below left=of q0-1] {$q_3/0$};
	\node[state,initial,accepting] (q3-1) [below right=of q0-1] {$q_3'/1$};



	
	
	\path[->] (q0-e) edge node {a} (q1-0);
	\path[->] (q0-e) edge node {b} (q3-0);
	\path[->] (q0-1) edge node {a} (q1-0);
	\path[->] (q0-1) edge [bend right] node {b} (q3-0);
	\path[->] (q1-0) edge node {a} (q3-1);
	\path[->] (q1-0) edge node {b} (q2-1);
	\path[->] (q2-1) edge node {a} (q0-1);
	\path[->] (q2-1) edge node {b} (q3-1);
	\path[->] (q3-0) edge node {a} (q3-1);
	\path[->] (q3-0) edge [bend right] node {b} (q0-1);
	\path[->] (q3-1) edge [loop below] node {a} (q3-1);
	\path[->] (q3-1) edge node {b} (q0-1);
	

	
\end{tikzpicture}


\hoofdstuk{Verificatie}
 We moeten aantonen dat een real-time programma voldoet aan de eisen opgesteld en gespecificeerd. De meest gebruikte methode voor het bewij
 
 zen van de correctheid van untimed programma's zijn aangepast voor timed programs.  We hebben nog geen aanpask gevonden voor het gebruik en bewijzen van correct gebruik van clocks.  Een bewijs voor het gebruik van real-time programmas met clocks is gegeven in T.A. Henzinger and P.W. Kopke. Verification methods for the di-
 vergent runs of clock systems
 
 In dit hoofdstuk formaliseren we de requirements ogegeven in de requiremenstlis tin hoofdstuk .. en bewijzen we de correcte toepassing met gebruik van de symbolic model-checker van Uppaal.
 Het systeem is gemodelleerd als een netwerk van meerdere timed automata: controller, sluis, stoplicht, deur, pomp en schip.
 
 Het bewijs vn corret gebruik kan ook worden aangetoond met help van bewijs voor inorrectgebruik
 
 
 Verificatie resultaten
 \paragraph{Het door ons uitgetippelde testpath of scenario}
 
 \paragraph{Timed automata}
 
 
\paragraph{Data variabelen}

\paragraph{Acties}
 
\paragraph{Clock regions}
\cite{clarke2000Modelchecking21}
\cite{clarke2000Modelchecking212}
\cite{clarke2000Modelchecking223}
\cite{clarke2000Modelchecking31}
\cite{clarke2000Modelchecking32}
\cite{clarke2000Modelchecking33}
\cite{clarke2000Modelchecking411}
\cite{clarke2000Modelchecking43}
\cite{clarke2000Modelchecking63}
\cite{clarke2000Modelchecking64}
\cite{clarke2000Modelchecking661}
\cite{clarke2000Modelchecking91}
\cite{clarke2000Modelchecking102}
\cite{clarke2000Modelchecking11}
\cite{clarke2000Modelchecking122}
\cite{clarke2000Modelchecking123}
\cite{clarke2000Modelchecking132}
\cite{clarke2000Modelchecking1321}
\cite{clarke2000Modelchecking152}
\cite{clarke2000Modelchecking171}
\cite{clarke2000Modelchecking172}
\cite{clarke2000Modelchecking173}
\cite{audioSemanticsBengtsson}
\cite{guidingAutomataBberm}
\cite{gearTransitionLindahl1}
\cite{gearTransitionLindahl2}
\cite{martinelliScada}
\cite{IgbalReconstructurintTransition1}
\cite{IgbalReconstructurintTransition2}
\cite{huangVerficationStoch}
\cite{bengtssonUppaalVerification}
\cite{pranaliVerificationWaterLevel}
\cite{alexandreUppaalDefinition}
\cite{behzadEvalQOS}
\cite{behzadVariablesQoS}
\cite{alur}
\cite{alurDenseRealTime}
\cite{alurSystemClok}
\cite{alurModelHybrid}
\cite{rijksoverheidSluizen}
\cite{rijksoverheidSluisStroomschema}

\paragraph{CTL logica}
Alle veiligheid en reachability requirements formeel gespecificeerd in hoofdstuk ... zijn geverifieerd in uppaal met gebruik an A en E state formulae. Deze zijn als volgt:
\newline \\
M, s $\models$ p $\Leftrightarrow$ p $\in$ L(s) \\
M, s $\models$ $\not$ f1 $\Leftrightarrow$ M, s $\nvdash$ f1 \\
M, s $\models$ f1 $\vee$ f2 $\Leftrightarrow$ M,s $\models$ f1 or M,s $\nvdash$ f2 \\
M, s $\models$ f1 $\wedge$ f2 $\Leftrightarrow$  M,s $\models$ f1 and M,s $\nvdash$ f2 \\
M, s $\models$ $\mathrm{E}$ $g_{1}$ $\Leftrightarrow$ there is a path $\pi$  from ~  s ~   such ~  that  ~ M, $\pi$ $\models$ g1 \\
M, s $\models$ p $\Leftrightarrow$ for every path $\pi$  ~ starting from  ~  s, M, $\pi$ $\models$ g1 \\
M, s $\models$ p $\Leftrightarrow$ s is the first state of $\piand$ M, s $\models$ f1 \\
M, s $\models$ $\not$ $g_{1}$ $\Leftrightarrow$ M, $\pi$  $\nvdash$ g1\\
M, s $\models$ p $\Leftrightarrow$  M, $\pi$  $\models$ g1  or  M, $\pi$  M, $\pi$  $\models$ g2\\
M, s $\models$ p $\Leftrightarrow$ M, $\pi$  $\models$ g1  and  M, $\pi$  M, $\pi$  $\models$ g2 \\
M, s $\models$ p $\Leftrightarrow$ M, $\pi^{1}$ $\models$ g1 \\
M, s $\models$ p $\Leftrightarrow$ there exists a k $\ge$ 0, such that  ~ M, $\pi^{k}$  $\models$ g1\\
M, s $\models$ p $\Leftrightarrow$ for all i $\ge$ 0,M,$\pi^{i}$ $\models$ g1 \\
M, s $\models$ g1 $\bugcup$ g2 $\Leftrightarrow$ ~  there  ~ exists  ~ ak  ~ $\ge$  ~ 0 ~  such ~  that  ~ M,  ~ $\pi^{k}$ $\models$ g2\\
and  ~ for  ~ all ~  0  ~ $\le$ j < k, M,$\pi^{j}$ $\models$ g1
M, s $\models$ p $\Leftrightarrow$ for all j $\ge$ 0, if for ~  every  ~ i < j,M,$\pi^{i}$ $\nvdash$ g1 then M,$\pi^{j}$ $\models$ g2\\


%%%%%%%%%%%%%%%%%%%%%%%%%%%%%%%%%%%%%%%%%%%%%%%%%%%%%%%%%%%%%%%%%


 

\hoofdstuk{Conclusie}

Wat hebben alle bovenstaande rampen/ongelukken gemeen? Veiligheid.
Bij de therac waren er diverse problemen: communicatie, doorontwikkeling, controle en toetsing
Was het makkelijk te onderzoeken? Waarom?
Bij de boeing 737 crashes was het probleem van controle en communicatie naar medewerkers
Was het makkelijk te onderzoeken? Waarom?

Uit de evaluatie van de china explosion 2015 tianjin komt naar voren dat communicatie, transparantie en veiligheid niet altijd prioriteit hadden bij de lokale autoriteiten
Was het makkelijk te onderzoeken? Waarom?

Bij de tesla autopilot crashes komen soms onvoldoende onderbouwde ontwerpkeuzes naar voren die niet goed zij  afgewogen tegenover het gedrag van de bestuurder
vlucht 1951
Was het makkelijk te onderzoeken? Waarom?

De ramp in Tsjernobyl toont aan hoe autoriteiten een ramp in de doofpot proberen te stoppen
Was het makkelijk te onderzoeken? Waarom?



Wat heb ik geleerd
Ik heb erg veel geleerd van het veilig opzetten van VPN’s. Een VPN opzettenhad ik namelijk nog nooit gedaan. Het opzetten van SSH en het aanmaken vanVM’s was al bekend. Ook had ik nog nooit met UDP sockets geprogrammeerd.Verder heb ik geleerd hoe ik in de praktijk een VM in een VLAN kan zetten enhoe VLAN’s netwerken van elkaar kunnen scheiden.Het leukste onderdeel van het project, was dat wonderbaarlijk mijn gekozenoplossing elegant werkte. UDP Servers en clients zijn gerealiseerd met minderdan enkele regels logisch scipt. Ik had aan genomen dat het werken met socketsin shell absoluut rampzalig zou uitpakken. Ik ben blij dat het opdracht zo vrijwas, zodat ik experimenteel kon zijn met mijn implementatie.



 %(verplicht) hoofdverslag
\hoofdstuk{Discussie}

discussie
geldigheidsgrenzen van de waarnemingen
betrouwbaarheid van de waarnemingen
waaarde van de waarnemingen
vergelijking van het oude en het nieuwe product/methode/apparaat volgens de genoende criteria. De gewijzigde factor maakt het product/methode/apparaat geheel/half/niet beter




   %(verplicht) geaggregeerde conclusies en aanbevelingen
%\hoofdstuk{Aanbeveling}
aanbeveling
adviezen en richtlijnen om de nieuw verkregen kennis in de praktijk toe te passen



Theoretical implication

Practical implication

Authors' contributions

Data availability statement

Declarations

Footnotes

Contributor Information



\cite{oid}   %(verplicht) geaggregeerde conclusies en aanbevelingen
\ifpublic
  \iflanguage{dutch}{\def\bibname{\normalsize{Bronnen}}}
                    {\def\bibname{\normalsize{References}}}
  {\footnotesize{\input{bronnen}}}   %(verplicht) bronvermeldingen
\else
  \iflanguage{dutch}{\def\bibname{Bronnen}}{\def\bibname{References}}
  \input{bronnen}    %(verplicht) bronvermeldingen
 
  \addcontentsline{toc}{chapter}{\bibname}
  \input{evaluatie}  %(verplicht) reflectie op het leerproces
  \appendix
 % \bijlage{Achtergrond materiaal}

In de bijlagen komen alle gegevens die nodig zijn voor de
onderbouwing, maar die de leesbaarheid van het hoofdverslag verlagen.


\sffamily
\begin{tabularx}{\textwidth}{@{}Sl|X|Sl @{}}
	\mytoprule
	\makecell[lc]{B. Buiiea GmbH \& Co. KG \\ Konstruktion und\\ Entwicklung}
	& Datum der Erstellung: 01.01.17 \par\mbox{}\par Erstellt von: Max Mustermann
	& \makecell[lc]{Aktueller Stand: 02.01.17 \\ Index: 00\\ \mbox{}} \\
	\mymidrule
	\multicolumn{3}{@{}c@{}}{Anforderungsliste} \\
	\addlinespace
	\multicolumn{3}{@{} >{\centering}m{\textwidth}@{}}{Bla Bla Bla Bla Bla} \\
	\midrule
	\multicolumn{3}{@{}c@{}}{Projekt-Nr.: 1234567890} \\
	\multicolumn{3}{@{}c@{}}{Projektname}
\end{tabularx}
\begin{tabularx}{\textwidth}{Sc| Sc |X| X| c | c | >{\RaggedRight\bigstrut}m{\lastcolwd}}
	\specialrule{\lightrulewidth}{-4ex}{0pt}
	\multicolumn{6}{@{}c|@{}}{Anforderungen} & \makecell[lt]{F = Fest \\W = Wunsch}\\
	\specialrule{2pt}{0pt}{0pt}
	\rowcolor{Gainsboro}\makecell[c]{F \\ W} & Nr. & Bezeichnung &
	\bigstrut Werte\par\ Daten \par Anforderungen & Zust. & Status & Bermerkungen \\
	\mybottomrule
	\endfirsthead
	\specialrule{2pt}{0pt}{0pt}
	\rowcolor{Gainsboro}\makecell[c]{F \\ W} & Nr. & Bezeichnung &
	\bigstrut Werte\par\ Daten \par Anforderungen & Zust. & Status & Bermerkungen \\
	\mybottomrule
	\endhead
	\multicolumn{1}{c}{} & \multicolumn{1}{Sc}{1} & \multicolumn{5}{l}{\bfseries Funktionen} \\
	\hline
	F & 1.1 & Hier steht ein Text. Hier steht ein Text. \par Hier steht ein Text. Hier steht ein Text. & Hier steht ein Text. Hier steht ein Text. \par Hier steht ein Text. Hier steht ein Text. & xy & & Hier steht ein Text. Hier steht ein Text. \par Hier steht ein Text. Hier steht ein Text. \\
	\hline
	\multicolumn{1}{c}{} & \multicolumn{1}{Sc}{1} & \multicolumn{5}{l}{\bfseries Funktionen} \\
	\hline
	F & 1.1 & Hier steht ein Text. Hier steht ein Text. \par Hier steht ein Text. Hier steht ein Text. & Hier steht ein Text. Hier steht ein Text. \par Hier steht ein Text. Hier steht ein Text. & xy & & Hier steht ein Text. Hier steht ein Text. \par Hier steht ein Text. Hier steht ein Text. \\
	\hline
	\multicolumn{1}{c}{} & \multicolumn{1}{Sc}{1} & \multicolumn{5}{l}{\bfseries Funktionen} \\
	\hline
	F & 1.1 & Hier steht ein Text. Hier steht ein Text. \par Hier steht ein Text. Hier steht ein Text. & Hier steht ein Text. Hier steht ein Text. \par Hier steht ein Text. Hier steht ein Text. & xy & & Hier steht ein Text. Hier steht ein Text. \par Hier steht ein Text. Hier steht ein Text. \\
	\hline
	\multicolumn{1}{c}{} & \multicolumn{1}{Sc}{1} & \multicolumn{5}{l}{\bfseries Funktionen} \\
	\hline
	F & 1.1 & Hier steht ein Text. Hier steht ein Text. \par Hier steht ein Text. Hier steht ein Text. & Hier steht ein Text. Hier steht ein Text. \par Hier steht ein Text. Hier steht ein Text. & xy & & Hier steht ein Text. Hier steht ein Text. \par Hier steht ein Text. Hier steht ein Text. \\
	\hline \noalign{\penalty-5000}
	\multicolumn{1}{c}{} & \multicolumn{1}{Sc}{1} & \multicolumn{5}{l}{\bfseries Funktionen ! ! ! } \\*
	\hline
	F & 1.1 & Hier steht ein Text. Hier steht ein Text. \par Hier steht ein Text. Hier steht ein Text. & Hier steht ein Text. Hier steht ein Text. \par Hier steht ein Text. Hier steht ein Text. & xy & & Hier steht ein Text. Hier steht ein Text. \par Hier steht ein Text. Hier steht ein Text. \\
	\hline
	\multicolumn{1}{c}{} & \multicolumn{1}{Sc}{1} & \multicolumn{5}{l}{\bfseries Funktionen} \\
	\hline
	F & 1.1 & Hier steht ein Text. Hier steht ein Text. \par Hier steht ein Text. Hier steht ein Text. & Hier steht ein Text. Hier steht ein Text. \par Hier steht ein Text. Hier steht ein Text. & xy & & Hier steht ein Text. Hier steht ein Text. \par Hier steht ein Text. Hier steht ein Text. \\
	\hline
\end{tabularx}


%\bijlage{Onderzoeksgegevens}

%\bijlage{UML diagrammen}

%\bijlage{Gebruikshandleiding}



   %(optioneel) bijlagen
 % 
\newpage
\hoofdstuk{Requirement tracability matrix}
\renewcommand*\theadfont{\bfseries}
\settowidth\rotheadsize{\theadfont Infrastructure}
\renewcommand\theadgape{}
\renewcommand\theadalign{lc}
\renewcommand\rotheadgape{}
\begin{table}
	\centering
	\caption{Caption}
	\label{tab:table1}
	\begin{tabular}{lcccc}
		\toprule
		\thead{Requirements} & \rothead{Accuracy} & \rothead{Coverage} & \rothead{Scalability} & \rothead{Infrastructure}  \\
		\midrule
		Inertial Navigation & $\checkmark$ & $\checkmark$  & $\checkmark$ & \\
		RFID & $\checkmark$ & $\checkmark$  & $\checkmark$ & \\
		Bluetooth & & $\checkmark$  & & $\checkmark$\\
		WLAN & $\checkmark$ & $\checkmark$  & & $\checkmark$ \\
		Infrared & $\checkmark$ & $\checkmark$  & & $\checkmark$\\
		\bottomrule
	\end{tabular}
\end{table}       %(optioneel) bijlagen
  %
\newpage
\hoofdstuk{swot analyse}

\begin{tcbitemize}[raster columns=3, raster rows=3, enhanced, sharp corners, raster equal height=rows, raster force size=false, raster column skip=0pt, raster row skip = 0pt]
	
	%Empty corner and two headers
	\tcbitem[blankest, width=1cm]
	\tcbitem[header = helpful]
	\texta
	\tcbitem[header = harmful]
	\textb
	
	%First row
	\tcbitem[firstcol = internal]
	\textcn
	\tcbitem[swotbox = S]
	\lipsum[2]
	\tcbitem[swotbox = W]
	\lipsum[2]
	
	%Second row
	\tcbitem[firstcol = external]
	\textcn
	\tcbitem[swotbox=O]
	\lipsum[2]
	\tcbitem[swotbox=T]
	\lipsum[2]
\end{tcbitemize}

\newpage
\begin{tcbitemize}[raster columns=3, raster rows=4, enhanced, sharp corners, raster equal height=rows, raster force size=false, raster column skip=0pt, raster row skip = 0pt]
	
	%Empty corner and two headers
	\tcbitem[blankest, width=1cm]
	\tcbitem[header = Strength]
	\texta
	\tcbitem[header = Weakness]
	\textb
	
	%First row
	\tcbitem[firstcol = internal]
	\textcn
	\tcbitem[swotbox = S]
	\lipsum[2]
	\tcbitem[swotbox = W]
	\lipsum[2]
	
	\tcbitem[blankest, width=1cm]
	\tcbitem[header = Opportunity]
	\texta
	\tcbitem[header = threat]
	\textb
	
	%Second row
	\tcbitem[firstcol = external]
	\textcn
	\tcbitem[swotbox=O]
	\lipsum[2]
	\tcbitem[swotbox=T]
	\lipsum[2]
\end{tcbitemize}   %(optioneel) bijlagen
 % 	\section{Research case: De digitale aanval op de Oekrainese krachtcentrale}
Dit verslag geeft inzage in een analyse van de Ukraine cyber aanval,
inclusief hoe de actoren zich zelf toegang gavan tot het controle systeem, welke methoden de acoren hebben gebruikt voor reconnaissance en vastleggen van het systeem, een gedetailleerde omshrijving van de aanval op 15 December 2015, en de methoden die gebruikt zijn door de aanvallers om hun sporen uit te wissen en daarmee het het stoppen van schade toebrengen  nog moeilker maken. Daarnaast wordter  een gedetailleerde omschrijving gevevenv an de beveiliging van de SCADA ccontrol systemen gebaeerd op bst practices, inclusief het control network ontwerp, technieken voor whtelisting, monitoring en loggen, en  opleiding van personeel.


\cite{Whitehead2017ukrainepoweroutage},\cite{zetter2016GridHack},\cite{boozallen2016lightwentout},\cite{finklejan2016UsBlamesRussianSandworm},\cite{desarnaud2017cyberattacks},\cite{caseli04112016intrusiondetectioncontrolsystem},\cite{rochascadatesting},\cite{hidajat2016ScadaSimulator},\cite{zetter2017moreDangerousMalware}.


Oop 23,december 2015  vind er een cyber aanval plaats op het elektriciteitsnet van de Oekraine. Dit was de eerste bekende aanval op een elektrisch controle  system met corrupte firmware. Daarnaas wordt er een telecom-based denial of service attack met  geautomatieerde systemen om het telefoonverkeer uit te schakelen.
\cite{Whitehead2017ukrainepoweroutage}

Uit onderzoek\cite{zetter2016GridHack} naar de aanval,  uitgevoerd door Oekraiene sen Amerikaanse militairenblijkt  bleek onder meer dat de power grids in sommige gevallen beter waren beveiligd dan de Amerikaanse. Desondanks was de viligheid niet optimaal door onder andere de  hetgegeven dat werknemers op afstand konden inloggen en geen gebruik van 2-stapsverificatie.


\subsection{Literaire analyse}

\subsubsection{Motief}
Oekraine wijst naar de russen \cite{zetter2016GridHack}, \cite{greenberg2017Cyberwartestlab},\cite{boozallen2016lightwentout},\cite{finkle08012016russiansandwormhackers},\cite{zinets15022017ukrainechargesrussia},\cite{mcelfresh2016cyberattackhowandwhy},\cite{parkwalstorm11102017russiagridattack}.
\subsubsection{Situatie Oekraiene}
\cite{drago2017CrashOverride},\cite{slowik2019ReassasUkraine2016Attack}.
\subsubsection{Situatie algemeen}
\cite{cerulus2019FrontlineRussiaAttack},\cite{desarnaud2017cyberattacks},\cite{dragos2019TargetedTransStation}.
\subsubsection{Factoren}
\cite{shehod2016gridadvantageus}
\subsubsection{Oorzaak}
\cite{rocha2017cybersecyrityanalysisScada},\cite{2017crashoverridenostuxnet},\cite{vijayan2017firstmalwareCausedOutage},\cite{slowik2019ReassasUkraine2016Attack}.
\subsubsection{Gebruikte materialen}
\cite{2015ukrainegridattack},
\cite{industroyershortfact}
\subsubsection{Uitvoering van de aanval}
\cite{Whitehead2017ukrainepoweroutage},\cite{boozallen2016lightwentout}.
\subsubsection{Oplossingen}
\cite{Whitehead2017ukrainepoweroutage}
\subsubsection{Aanbevelingen}
\subsection{Resultaten}
\subsubsection{De aanval}
1. An initial email spear phishing attack lures recipients
into opening an attached Microsoft® document with a
macro that installs Black Energy 3 (BE3) onto
corporate workstations.
2. BE3 and other tools perform reconnaissance and
enumeration of the network and provide an initial
backdoor for the hackers into the corporate network.
3. As a result of network reconnaissance, the malicious
actors discover and access the oblenergos’ Microsoft
Active Directory® servers that contain corporate user
accounts and credentials.
4. With the harvested credentials, the malicious actors use
an encrypted tunnel from an external network to get
inside the oblenergo network, establishing a presence
on the oblenergo control system networks.
5. Malicious actors discover and access the control center
supervisory control and data acquisition (SCADA)
human-machine interface (HMI) servers and
substations. While a router separates corporate and
SCADA networks, the firewall rules are improperly
configured.
6. On December 23, 2015, at 3:30 p.m., the malicious
actors begin their power outage attacks by entering
operations and SCADA networks through backdoors on
the compromised SCADA workstations. The malicious
actors take control away from HMI operators and then
open breakers.
7. The malicious actors perform several other actions with
the intent of complicating the responses of control
operators and increasing the effort required to return the
system to normal operating conditions. These actions
include:
a. Launching a coordinated Telephony Denial of
Service (TDoS) attack that floods call centers to
prevent legitimate calls from getting through.
b. Disabling the UPSs for the control centers.
c. Corrupting the firmware on a remote terminal unit
(RTU) HMI module and serial-to-Ethernet port
servers.
8. Malicious actors execute KillDisk malware in an
attempt to wipe out the control center HMIs and pivotpoint workstations.

\cite{Whitehead2017ukrainepoweroutage}

\cite{boozallen2016lightwentout}
\subsubsection{spearfishing}
\subsubsection{blackenergy}
\subsubsection{remote access capabilities}
\subsubsection{serial-to-ethernet communication devices}
\subsubsection{telephony denial of service attacks}

\subsection{oplossingen}
Identificeer alle risicos en schrijf een plan foor het managen van de risico's.
Implementeer  effecteve controle  om het riico te managen.
Creeer een diepgaand model dat ervoor zor dat er efectieve en efficiente security controls worden uitgevoerd.
Aangaande de gebeurtenissen in de oekraiene kunnen de volgende security controls worden opgenomen in het securitymodel: Initial access to enterprise network, pivot in interprise network, elevate priviliges, maintainance access, gain access to control system, attack, attack complication, destroy hard drives.
\cite{Whitehead2017ukrainepoweroutage}
\subsection{Discussie}

\subsection{Verder lezen}
\cite{Shahzad2014ScadaProtocolsPollingScenario},\cite{grammatikis2019AttackIEC6087505104},\cite{2017win32industroyer},\cite{yadav2020reviewScadaArchitecture},\cite{arrizabalaga2020surveyiiotProtocols},\cite{fauri2017EncryptionICS},\cite{resch31102019IEC62351secureCommunication},\cite{levalle2020FuzzingICSProtocols},\cite{blackhatusa2017},\cite{blackhatusa2017},\cite{abb30062017crashoverridenotification},\cite{spinner2018crashoverrideiot},\cite{njccicthreat08102017crashovverrideprofile},\cite{slowikvb2018crashoverride},\cite{crashoverridenetwork},\cite{wikiindustroyer},\cite{icsSecurityRussianHacking},\cite{holappa2017threattoElectricityNetworks}.
   %(optioneel) bijlagen
 %  


\subsection{Formele logica}

0004 \\
0031 \\
Modelcheckig boek blz 14 \\
E = {main={}, \\
	deur={}, \\
	stoplicht={}, \\
	sensor={}, \\
	pomp={}, \\
	wachtrij={}, \\
	queue={}, \\
	sluiskolk={} \\
} \\
Q0 = \\
F C Q = sluiskolk_afgesloten \\
Q = E0,...En \\
0064 \\
0077 \\



X: volgende keer \\
F: in de toekomst \\
G: altijd geldig \\
A: voor alle paden \\
E: voor enkele paden \\
U: waar zolang de volgende inclusief geldt \\
R: waar zolang geldt inclusief de startpositie \\

M, s-> f betekent f is geldig in state s in kripke structuur M \\
M, ph -> f is geldig op een pad in kripke structuur M \\
M. s -> FP <-> er bestat een fair pad dat begint bij s en p € L(s) \\
M, s->E(g) er bestaat een fair pad M dat begint van s zodanig dat phi -> F(g) \\
M, s-> F A(g) voor alle fair pads phi beginnend vanad s,  phi -> F(g) \\



Sluis.Draining-->Deuren.laag_open \\
Deuren.laag_open-->Stoplicht.Green \\
E<> (Ship.ship_can_move&&Stoplicht.Green) \\
A[] not (Stoplicht.Green && not \\ (Deuren.hoog_open||Deuren.laag_open||Deuren.stopgaplow1||Deuren.stopgaplow2||Deuren.stopgaphigh1||Deuren.stopgaphigh2)) \\
A[] not \\ ((Deuren.hoog_open||Deuren.laag_open||Deuren.Opening_laag||Deuren.Opening_hoog||Deuren.Closing_hoog||Deuren.Closing_laag) && (Sluis.Draining||Sluis.Filling||Sluis.draining2||Sluis.Filling2)) \\
Sensor.Wait-->Sensor.Wait \\
Stoplicht.Green-->Stoplicht.Green \\
(Deuren.hoog_open||Deuren.laag_open)-->(Deuren.laag_open||Deuren.hoog_open) \\
Deuren.laag_open-->Deuren.Closed \\
Deuren.hoog_open-->Deuren.Closed \\
Deuren.Closed-->Stoplicht.Red \\
Ship.ship_can_move-->Deuren.Closed \\
Deuren.hoog_open-->Stoplicht.Green \\
Ship.ship_can_move-->Stoplicht.Green \\
A[] not (Deuren.laag_open && Deuren.hoog_open) \\
Ship.ship_can_move-->Ship.ship_can_move \\
A[] not (Deuren.laag_open && Sluis.water != Sluis.water_laag) \\
A[] not (Deuren.hoog_open && Sluis.water != Sluis.water_hoog) \\
A[]not deadlock \\




A[] forall (i:id_t) forall (j:id_t) P(i).cs && P(j).cs imply i == j \\
Mutex requirement. \\
P(1).req --> P(1).wait \\
Whenever P(1) requests access to the critical section it will eventually enter the wait state. \\
A[] (sum(i:pid_t) P(i).cs) <= 1 \\
Mutex requirement. \\
A[] forall(i : pid_t) not Task(i).Error \\
Check that the system is schedulable. \\
E<> Gate.Occ \\
Gate can receive (and store in queue) msg's from approaching trains. \\
Train 0 can reach crossing. \\
E<> Train(1).Cross \\
E<> Train(0).Cross and Train(1).Stop \\
E<> Train(0).Cross and (forall (i : id_t) i != 0 imply Train(i).Stop) \\
A[] forall (i : id_t) forall (j : id_t) Train(i).Cross && Train(j).Cross imply i == j
There is never more than one train crossing the bridge (at any time instance). \\
A[] Gate.list[N] == 0 \\
There can never be N elements in the queue (thus the array will not overflow). \\
Train(0).Appr --> Train(0).Cross \\
Train(1).Appr --> Train(1).Cross \\
Train(2).Appr --> Train(2).Cross \\
Train(3).Appr --> Train(3).Cross \\
Train(4).Appr --> Train(4).Cross \\
Train(5).Appr --> Train(5).Cross \\
E<> Gate.Occ \\
Gate can receive (and store in queue) msg's from approaching trains. \\
E<> Train1.Cross \\
Train 1 can reach crossing. \\
E<> Train2.Cross \\
Train 2 can reach crossing. \\
E<> Train1.Cross and Train2.Stop \\
Train 1 can be crossing bridge while Train 2 is waiting to cross. \\
E<> Train1.Cross and Train2.Stop and Train3.Stop and Train4.Stop \\
Train 1 can cross bridge while Train 2, 3 & 4 are waiting to cross. \\
A[] Train1.Cross + Train2.Cross + Train3.Cross + Train4.Cross <= 1 \\
There is never more than one train crossing the bridge (at any time instance). \\
A[] Queue.list[N-1] == 0 \\
There can never be N elements in the queue (thus the array will not overflow) \\
Train1.Appr --> Train1.Cross \\
Whenever a train approaches the bridge, it will eventually cross. \\
Train2.Appr --> Train2.Cross \\
Train3.Appr --> Train3.Cross \\
Train4.Appr --> Train4.Cross \\




%

\cheading{Fake Course Evaluation Summary for \textsc{course
		1234y}}{Sept.\ 2010 --- May 2011}

\begin{longtable}{@{}l rr rr rr rr rr rr}
	% pairs: absolute number (percentage)
	
	\toprule%
	\centering%
	& \multicolumn{2}{c}{{{\bfseries Excellent}}}
	& \multicolumn{2}{c}{{{\bfseries Very Good}}}
	& \multicolumn{2}{c}{{{\bfseries Good}}}
	& \multicolumn{2}{c}{{{\bfseries Average}}}
	& \multicolumn{2}{c}{{{\bfseries Poor}}}
	& \multicolumn{2}{c}{{{\bfseries Very Poor}}} \\
	
	
	\cmidrule[0.4pt](r{0.125em}){1-1}%
	\cmidrule[0.4pt](lr{0.125em}){2-3}%
	\cmidrule[0.4pt](lr{0.125em}){4-5}%
	\cmidrule[0.4pt](lr{0.125em}){6-7}%
	\cmidrule[0.4pt](lr{0.125em}){8-9}%
	\cmidrule[0.4pt](lr{0.125em}){10-11}%
	\cmidrule[0.4pt](l{0.25em}){12-13}%
	% \midrule
	\endhead
	
	
	Some question about the Instructor or Course & 2 & (7.14) & 4 &
	(14.29) & \highest{12} & \highest{(42.86)} & 4
	& (14.29) & 6 & (21.43) & 0 & (0.00) \\
	
	\myrowcolour%
	Some question about the Instructor or Course & 3 & (10.71) &
	\highest{15} & \highest{(53.57)} & 5 & (17.86) & 5 & (17.86) & 0 &
	(0.00) & 0 & (0.00) \\
	
	Some question about the Instructor or Course & 4 & (14.29) & 8 &
	(28.57) & \highest{15}
	& \highest{(53.57)} & 1 & (3.57) & 0 & (0.00) & 0 & (0.00) \\
	
	\myrowcolour%
	Some question about the Instructor or Course & 3 & (10.71) & 8 &
	(28.57) & \highest{10} & \highest{(35.71)}
	& 5 & (17.86) & 2 & (7.14) & 0 & (0.00) \\
	
	Some question about the Instructor or Course & 6 & (21.43) &
	\highest{9} & \highest{(32.14)}
	& 4 & (14.29) & \highest{9} & \highest{(32.14)} & 0 & (0.00) & 0 & (0.00) \\
	
	\myrowcolour%
	Some question about the Instructor or Course & \highest{10} &
	\highest{(35.71)} & \highest{10} & \highest{(35.71)}
	& 3 & (10.71) & 5 & (17.86) & 0 & (0.00) & 0 & (0.00) \\
	
	Some question about the Instructor or Course & \highest{12} &
	\highest{(42.86)} & \highest{12} & \highest{(42.86)} & 3
	& (10.71) & 1 & (3.57) & 0 & (0.00) & 0 & (0.00) \\
	
	\myrowcolour%
	Some question about the Instructor or Course & \highest{12} &
	\highest{(42.86)} & 3 & (10.71) & 7
	& (25.00) & 5 & (17.86) & 1 & (3.57) & 0 & (0.00) \\
	
	Some question about the Instructor or Course & \highest{10} &
	\highest{(35.71)} & 6 & (21.43) & 6 & (21.43) & 6 & (21.43)
	& 1 & (3.57) & 0 & (0.00) \\
	
	\myrowcolour%
	Some question about the Instructor or Course & 5 & (17.86) & 5 &
	(17.86) & \highest{12} & \highest{(42.86)} & 2 & (7.14)
	& 3 & (10.71) & 1 & (3.57)\\
	
	Some question about the Instructor or Course & 3 & (10.71) & 8 &
	(28.57) & \highest{11} & \highest{(39.29)} & 3 & (10.71) & 3 & (10.71)
	& 0 & (0.00) \\
	
	\myrowcolour%
	Some question about the Instructor or Course & \highest{18} &
	\highest{(64.29)}
	& 5 & (17.86) & 3 & (10.71) & 1 & (3.57) & 1 & (3.57) & 0 & (0.00) \\
	
	Some question about the Instructor or Course & \highest{15} &
	\highest{(53.57)}
	& 7 & (25.00) & 2 & (7.14) & 2 & (7.14) & 2 & (7.14) & 0 & (0.00) \\
	
	\myrowcolour%
	Some question about the Instructor or Course & 3 & (10.71) &
	\highest{13} & \highest{(46.43)} & 4 & (14.29) & 6 & (21.43) & 2
	& (7.14) & 0 & (0.00) \\
	
	\bottomrule
	
\end{longtable}













\square ( a_0 \implies (( \lnot a_2 \wedge \lnot a_3 ) \mathcal{U} a_1 ) \vee ( \lnot a_2 \wedge \lnot a_3 ))


$\xymatrix@1{
	A\times B\times C\times D \ar[r]^-{+} &B
}$


\usepackage{logicproof}


\begin{logicproof}{4}
	\forall x \, (P(x) \to Q(x)) & premise \\
	\forall x \, P(x) & premise \\\hspace*{-30pt}
	\begin{subproof}
		\llap{$x_0\quad$} P(x_0) \to Q(x_0) & $\forall x \, \mathrm{e}$ 1 \\
		P(x_0) & $\forall x \, \mathrm{e}$ 2 \\
		Q(x_0) & $\to \mathrm{e}$ 3, 4
	\end{subproof}
	\forall x \, Q(x) & $\forall x \, \mathrm{i}$ 3--5
\end{logicproof}



\begin{align*}
	&p                                                      \\
	&p\pdfliteral{-10 -5 m 0.5 0 1 RG 0.5 w 25 -5 l S }\to q\\
	\therefore\quad &q
\end{align*}


\{a,b\} or \set†{a,b} \\
\langle a,b \rangle or \gens†{a,b} \\


f \colon A \to B \\
f \colon A \into B \\
f \colon A \onto B \\
f \colon A \isom B \\
f \circ g \\
x \mapsto f(x) \\

\begin{align*}
	f \colon \mathbb{R} &\to \mathbb{R} \\
	x &\mapsto x^2
\end{align*}


\mathrm{A} \alpha \\
\mathrm{B} \beta \\
\Gamma \gamma \\
\Delta \delta \\
\mathrm{E} \epsilon \varepsilon \\
\mathrm{Z} \zeta \\
\mathrm{H} \eta \\
\Theta \theta \vartheta \\
\mathrm{I} \iota \\
\mathrm{K} \kappa \\
\Lambda \lambda \\
\mathrm{M} \mu \\
\mathrm{N} \nu \\
\Xi \xi \\
\mathrm{O} \mathrm{o} \\
\Pi \pi \varpi \\
\mathrm{P} \rho \varrho \\
\Sigma \sigma \varsigma \\
\mathrm{T} \tau \\
\Upsilon \upsilon \\
\Phi \phi \varphi \\
\mathrm{X} \chi \\
\Psi \psi \\
\Omega \omega \\



X \implies Y  \\
X \impliedby Y \\
X \iff Y \\
\neg X \\
\sim X \\
X \land Y \\
X \lor Y \\
\forall a \in A \\
\exists b \in B \\
\exists !b \in B \\


A = \{ [elements] : [conditions]\} \\

\mathbb{N} = \{ a \in \mathbb{Z} : a > 0 \} \\

a \in A \\
A \subseteq B \\
A \subset B  \\
A \supseteq B \\
A \supset B  \\
A = B  \\
A \cong B \\
A \cup B \\
A \cap B \\
A-B \\
|A| \\
\{\} = \varnothing \\


\begin{align*}
	f \colon \mathbb{R} &\to \mathbb{R} \\
	x &\mapsto x^2
\end{align*}


\usepackage{tikz}
\usetikzlibrary{cd}
~ 
\usepackage{mathtools,halloweenmath}

\paragraph{Model checking Temporal logics} \\
M, s \models p $\Leftrightarrow$ p \in L(s) \\
M, s \models \not f1 $\Leftrightarrow$ M, s \nvdash f1 \\
M, s \models f1 \vee f2 $\Leftrightarrow$ M,s \models f1 or M,s \nvdash f2 \\
M, s \models f1 \wedge f2 $\Leftrightarrow$  M,s \models f1 and M,s \nvdash f2 \\
M, s \models \mathrm{E} g_{1} $\Leftrightarrow$ there is a path \pi  from ~  s ~   such ~  that  ~ M, \pi \models g1 \\
M, s \models p $\Leftrightarrow$ for every path \pi  ~ starting from s, M, \pi \models g1 \\
M, s \models p $\Leftrightarrow$ s is the first state of \piand M, s \models f1 \\
M, s \models \not g_{1} $\Leftrightarrow$ M, \pi  \nvdash g1\\
M, s \models p $\Leftrightarrow$  M, \pi  \models g1  or  M, \pi  M, \pi  \models g2\\
M, s \models p $\Leftrightarrow$ M, \pi  \models g1  and  M, \pi  M, \pi  \models g2 \\
M, s \models p $\Leftrightarrow$ M, $\pi^{1}$ \models g1 \\
M, s \models p $\Leftrightarrow$ there exists a k \ge 0, such that M, $\pi^{k}$  \models g1\\
M, s \models p $\Leftrightarrow$ for all i \ge 0,M,$\pi^{i}$ \models g1 \\
M, s \models g1 \bugcup g2 $\Leftrightarrow$ there exists ak \ge 0 such that M, $\pi^{k}$ \models g2\\
and for all 0 \le j < k, M,$\pi^{j}$ \models g1
M, s \models p $\Leftrightarrow$ for all j \ge 0, if for every i < j,M,$\pi^{i}$ \nvdash g1 then M,$\pi^{j}$ \models g2\\

\leg \\
\geq \\
\Re \\
\partial \\
\bigup \\
\bigcap \\


\paragraph{Model checking Fairness constraints}
\paragraph{Definition 4.1}

\ell ~ \boldsymbol \ell  g \\

\mathbb{A}  is a tuple = \{ L ,  \ell_{0} ,X ,  Inv , \mathrm{T} , \Sigma \} ~ where: \\
L ~  is ~   a  ~  finite ~   set  ~  of  ~  control ~   states, ~   also ~   called ~   locations, \\
\ell_{0}    \in L  ~  is  ~  the ~   initial  ~  location,  ~  \\
X   ~ is  ~  a ~   finite  ~  set  ~  of  ~  clocks \\
T ~   \subseteq ~   L ~  x ~ C (X) ~  x  ~ \sigma  ~ x  ~ $2^{X}$ ~ x ~ L  ~ is  ~ a  ~ finite ~  set  ~ of ~  tranasitions: ~   e= \langle \ell ,g,a,r,l' \rangle  \in T ~  represents ~   a   ~ trasition  ~  from ~   \ell ~  to  ~  \ell'. ~  g  ~  is  ~  the   ~ guard   ~ of ~   e, ~   r  ~  is  ~  the  ~  set  ~  of  ~  clocks  ~  that  is ~   reset  ~  by ~    ~  e, \\ ~   and   ~   a ~   is \\  ~  the  ~  action  ~  of  ~  e.  ~  We ~   also ~   write  ~  \ell    $$\underrightarrow{g,a,r}  $$ \\
Inv : L \to C (X) \\
\Sigma  ~ is  ~ an  ~ alphabet  ~  of  ~  actions
\paragraph{Definition 4.2}
A timed transition (TTS)is a tuple S =(S,s_{0},$$\underrightarrow{} $$, \Sigma) where S is a (possibly infinite) set of states, s_{0} \in S is the initial state and $$\underrightarrow{} $$   \subseteq S x (\Sigma \cup R) x S is the transition relation. Morever, the relation -> satisfies the three following conditions:
(1)  if s $$\underrightarrow{0} $$ s', then s = s',  (2) is s$$\underrightarrow{d} $$ s' and s' $$\underrightarrow{d'} $$ s"  with d, d' \in R, then s $$\underrightarrow{d+d'} $$ s", and (3) if s $$\underrightarrow{d} $$ ' with d \in R, then for all 0 << d' << d, there exists s" \in S such that s $$\underrightarrow{d'} $$ s"and s" $$\underrightarrow{d-d'} $$ s'
\paragraph{Definition 4.3}
Let \mathrm{A} = (L,\ell,X,Inv, T,\Sigma) be a timed automation. The semantics of A is defined asthe TTS S_{A}=(S,s_{0},$$\underrightarrow{} $$, \Sigma) where:\\
S = L x $R^{X}$  \\
s_{0} = (\ell_{0}, v_{0}) with v_{0}(x)=0 for every x \in X, \\
the transition relation  ,$$\underrightarrow{} $$ is composed of: \\
action transitions (\ell, v)$$\underrightarrow{a} $$ (\ell',v') if and only ifthere exists \ell $$\underrightarrow{g,a,r} $$ \ell' \in  T such that v \models g,v'=[r \leftarrow 0] v and v' \models Inv(\ell').


delay transitions: if d \in R, (\ell,v) ,$$\underrightarrow{d} $$ (\ell,v+d) if and only if v+d \models Inv(\ell $(^{1}$
\url{http://www.iste.co.uk/data/doc_wrkszvritcbv.pdf}


\paragraph{Definition 1}
Let A be a setof actions. A timedtransistion system TTS is a structure L=(S,A,R,s_{0},$\subseteq$, U) where \\
S is a set of states, with the initial state s_{0} \in S; \\
A is a setof labels;\\
$$\underrightarrow{}  $$ \subseteq S x (A x $R^{ \geq 0}$) x S is the transition location; and \\
U \subseteq  $R^{ \geq 0}$ x S is the until predicate \\



\paragraph{Definition 2}
Let L_{i} = (S_{i},A x  $\Re^{\geq0}$,$s^{\i0}$_{0},,$$\underrightarrow{} $$ i, $U^{i}$   ), i \in {1,2}, be two TTS. \\
A timedbisimulation is a relation R $\subseteq$ S1 x S2 with $s^{1}$_{0}R$s^{2}$_{0} satisfying, for all a(d) \in A x $\Re^{\geq0}$, the following transfer properties: \\
if  s_{1}R_{2} and  s_{1}$$\underrightarrow{a(d)} $$_{1} s_{1}', then \exists $s^{'}$_{2} \in S_{2} : s2  $$\underrightarrow{a(d)}  $$ 2$s^{'}$_{2} and $s^{'}$_{1}R$s^{'}$_{2} \\

if  s_{1}R_{2} and s_{1}$$\underrightarrow{a(d)} $$_{2} s_{1}', then \exists $s^{'}$_{1} \in S_{1} : s1  $$\underrightarrow{a(d)}  $$ $s^{'}$_{1} and $s^{'}$_{1}R$s^{'}$_{2}

if   s_{1}R_{2} s_{1}', then  $U^{1}$_{d}(s1)   $\Leftrightarrow$ $U^{2}$_{d}(s2) 

\paragraph{Definition 3}
A timed safety automation is a structure (S,A,C,s_{0},$\subseteq$, $\vartheta$ ,\kappa)
S is a setof states, with the initial state s_{0} \in S; \\
A is a set of actions; \\
C is a set of clocks \\
$$\underrightarrow{}  $$  $\subseteq$ S x A x \phi(C) x S is the set of edges; \\
$\vartheta$ : S $\subseteq$ \Phi(C)is the invariantassignment function
\kappa : S $\subseteq$ \varphi_{fin}(C) is the clocks resetting function
\url{file:///C:/Users/gally/Downloads/228_DArBr96b.pdf}

\paragraph{Model checking Delay transitions}
De lay transitions correspond to the  elapsing time while staying at some location. We write (s,v)  $$\underrightarrow{d}  $$ (s,v+d), where d \in $R^{ \geq +}$) , provided that for every 0$\le$e$\le$d, the invariant I(s) holds for v+e
\paragraph{Model checking Action transitions}
Action transitions correspondto the execution of a transitionfrom T.We write (s,v) $$\underrightarrow{d}  $$ (s',v') where a \in \Sigma, provided that there  is a transition \langle s,a,\phi,\lambda,s' \rangle such  that v satisfies \varphi and v' = v[ \lambda :=0]


% $\xymatrix@1{
	% 	X\ar[r]^a_b & Y & Z\ar[l]^A_B }$
% 
%$\ell$ \\
\Re \\
\patial \\   %(optioneel) bijlagen
  %\include{bijlage6}
 % \section{Overige onderzoeksresultaten}


\subsubsection{explosie in libabon, beirut }
\begin{description}
	\item[Beschrijving]
	\item[Datum en plaats] 
	\item[Oorzaak]
	%Beschrijf wat er mis ging in termen van het vier variabelen model/requirements/specificaties
\end{description}
Op 23 september 2013 voer het vrachtschip de Rhosus onder Moldavische vlag[7] van Batoemi in Georgië naar Beira in Mozambique met 2.750 ton ammoniumnitraat

Gezien het ernstige gevaar van het bewaren van deze goederen in de hangar onder ongeschikte klimatologische omstandigheden, herhalen we ons verzoek aan de marine-instantie om deze goederen onmiddellijk weer te exporteren om de veiligheid van de haven en de mensen die er werken te verzekeren, of om akkoord te gaan om ze te verkopen.
Voorafgaand aan de explosie was er een brand in een opslagplaats. 


\cite{hrw03082021investigateBeirutBlast}

\cite{souaibyElHussein112020Beirutstory}

\cite{ifrc2020chemicalexplosionBeirutPort}




\subsubsection{stint ongeluk}

\begin{description}
	\item[Beschrijving]
	\item[Datum en plaats] 
	\item[Oorzaak]
	%Beschrijf wat er mis ging in termen van het vier variabelen model/requirements/specificaties
\end{description}
Vier kinderen, een bestuurder kwamen om en een vijfde persoon , een kind raakte zwaargewond. Uit odnerzoek van bleek :
Foute torsieveer voor de gashendel werd geleverd
Geen van de drie onderzochte voertuigen haalden de wettelijk vereiste remvertraging
De automatische parkeerrem kan leiden tot gevaarlijke situaties wanneer deze ongewenst geactiveerd wordt tijdens het rijden. 
Het losraken van de nuldraad naar de gashendel leidt volgens TNO tot ongewenst versnellen van het voertuig en een oncontroleerbare situatie voor de bestuurder.
Voor alle drie onderzochte voertuigen geldt dat het ontbreken van een zitplaats leidt tot veiligheidsrisico’s voor remmen en sturen door de grotere kans dat de bestuurder van het voertuig valt. Als de bestuurder van een Stint valt, leidt dit in alle rijsituaties tot een onbeheersbare situatie


\cite{TNOStint}




\subsubsection{vuurwerkramp in enschede }

\cite{boogers092002RampenRegelsRichtlijnen}

Wat waren de afspraken omtrent vuurwerkopslag?
Waarom werden de voorschriften neit nageleefd?




\subsubsection{ecourt in nederlandse rechtspraak}

\begin{description}
	\item[Beschrijving]
	\item[Datum en plaats] 
	\item[Oorzaak]
	%Beschrijf wat er mis ging in termen van het vier variabelen model/requirements/specificaties
\end{description}
niet odnerzocht
https://www.njb.nl/blogs/a-court-with-no-face-and-no-place/ 
\cite{sprongken19032018CourtProcedureDigital}
http://www.e-court.nl/wp-content/uploads/2018/03/Procesreglement-e-Court-2017_20180201.pdf
\cite{PROCESREGLEMENTEcourt}




\paragraph{molukse treinkaping }

\begin{description}
	\item[Beschrijving]
	\item[Datum en plaats] 
	\item[Oorzaak]
	%Beschrijf wat er mis ging in termen van het vier variabelen model/requirements/specificaties
\end{description}
https://www.youtube.com/watch?v=h99Fe9XzzHI 
\cite{molukseTreinkaping}


\subsubsection{Ramp schietpartij militair ossendrecht }

\begin{description}
	\item[Beschrijving]
	\item[Datum en plaats] 
	\item[Oorzaak]
	%Beschrijf wat er mis ging in termen van het vier variabelen model/requirements/specificaties
\end{description}
Een militaire overleid op een schietbaan in ossendracht door onvoldoende begeleiding van cursisten, geen toezicht op de lokatie. E\r was een instructuur in opleiding die niet volledig was mmeegenomen in het poroces en ook was er geen baancommandant aanwezig. Geen van de aanwezig instructeurts had de juiste papieren om de cursisten te begeleiden. De aanwezig instruceur had geen zich op de instructeur in opleiding, evenmin de andere militairen. In de instructiehandleiding ontbreken richtlijnen voor bijzondere schietbanen. Ook was er geen keuring. Door personelstekort is er geen andacht besteed aan documentastie(een slyllabus) hoe en met welke risico’s oefeningnen moeten worden ingericht. Ok werd er vooraf geen veiliheidsanaklyse gedaan. Het gebrek aan lesmateriaal en deskundigen is gemeld binnen de defensieorganisatie maar dit heeft niet geleid tot enige verandering in de situatie.
Op een afgekeurde scheitbaan
Tezicht door een instructeur in opleiding die zelf geen persoonlijke begeleiding heeft gehad tijdens de uitvoering
Belangrijk is dat defensie haar taken kan uitvoeren met personeel dat is getraind in situaties die de risicos van de werkomgeving aan de cursisten kunnen laten zien.
Conclusie
Zonder gekwalificeerde instructuers.
Zonder toezicht
Zonder lesmateriaal
Zonder adequate veiligheidsanalyse
https://www.youtube.com/watch?v=6jmkDClGDHo 
\cite{oVVSchietongevalOssendrecht}
\cite{nos22032016ossendrecht}
\cite{ovv04042016lessenongevalossendrecht}
\cite{quekelboere10052017doodossendrecht}


Wat is de rol van defensie?
Wat is er gedaan om de veligheid van de medewerkers te waarborgen?
Waarom zijn deze regels niet nageleefd?
Wat zijn de gevolgen?
Zijn de acties die naderhand zijn ondernomen wel redelijk naar de slachtoffers, het nationale veiligheisbeeld en de medewerkers?

 %   \hoofdstuk{}






main.py [-h] [--version] [-d] [-c FILE] [-g FILE] [-t FILE] [-o FILE]





32,38,40,44,61,75,78,79,81,82,85,86,201,202,213,306,308,325\




Timeliness
https://www.eecs.yorku.ca/course_archive/2014-15/F/1090/slides/04_TheoremCalculation.pdf
https://math.stackexchange.com/questions/2133202/what-does-the-notation-gamma-vdash-phi-mean-in-mathematical-logic


Assume that e is an action of the system, a Timeliness property for e is defined
to be a property related to the time of occurrence of e [10]. G. Blair, J.-B. Stefani: Open Distributed Processing and Multimedia Addison-
Wesley, Boston, MA, 1997.




De verificatie methode van dit artikel werkt niet omdat we theoretisch gezien niet uitgaan van meerdere schepen die tegelijkinvaren.

Theorem 1, the main
result of the paper, proves that a QTA is a Test Automata [2–4, 16]. Section 4
applies our approach to the verification of throughput in an example of a Video
Player system. Section 5 presents a proof for Theorem 1. The final two sections
discuss some related works and draw a conclusion.


https://www.brics.dk/RS/97/29/BRICS-RS-97-29.pdf

[[♦P kj]] = {s ∈ S | ∃s′
	.hs, s′i ∈ R and s′ ∈ [[j]]}
CTL formulas are based on the following operators:
A (\on every path")
E (\there exists a path")
X (\next time")
G (\globally" or \always")
F (\eventually" or \nally")
U (\until")
R (\release")

PRECONDITIONS
Topography:
Geometry description of the environment including maps of the
expected changes, such as land, water, river, sea and title deeds as
well as regional planning and zoning scheme.
Possible existing lock that could remain operational or has to be
renovated:
Geometry and condition;
Current and anticipated use;
Permitted limitations during the construction/ renovation.
Possible other hydraulic structures nearby:
Geometry and condition;
Current and anticipated use;
Permitted limitations during construction/renovation.
Water levels:
Water levels with exceedance and underrun frequency levels
Water level development (tidal curve etc.) within lock operation
reach
Rising and lowering velocities
Historical water level data during dry and wet seasons
Water flow
River discharges/flood control regime
Water quality (chloride content, "aggressiveness")
Water temperature
Swell and wave data
Wind data (speed, direction, frequencies)
Morphological data (such as bed-load and suspended-load
transport) and forecasts
Soil characteristics
Soil mechanical data (results of field and laboratory sampling)
Geo-hydrological data (such as ground water level rise as a function
of time, groundwater flow, results of pump test in case of groundwater
lowering)
Soil pollution where excavation takes place for lock and lock approach
as well as in the general vicinity in case of possible groundwater
lowering by pumping
FUNCTIONAL REQUIREMENTS
Functional requirements regarding navigation
General:
Design, build and manage the lock complex so that vessels of a
given waterway classification index can pass rapidly and safely:
• Normative vessel (length, width, depth, height)
• Normative combination of shipping vessels
• Normative traffic and fleet composition, taking the spread of
arrival times into account
Lock approaches:
Per lock (chamber) and per side (above and below) a lining up area
is required that is situated as such that moored vessels do not form
an obstacle to departing vessels, while the moored vessels are able
to sail into the lock via leading jetty rapidly.
The size of the lining up area is geared to a complete chamber fill
(for existing locks with small amounts of traffic this is unnecessary).
The width is equal to that of the chamber.
Waiting areas are necessary if it is expected that, on busy days,
lining up areas will not be sufficient. A waiting area is created per
side of the lock complex (a common area if there are several chambers).
At very least, a normative vessel should be able to moor here.
With a view to the stopping and mooring, the free area is given a
length of at least 2.5 times the normative vessel length (inland
navigation) depending on the adjoining waterway.
Is the lock approach also used as stay over harbour, refuge harbour
or compulsory harbour?
Leading jetties:
Approach wall lengths and shapes are a function of navigation (sea,
inland, recreational navigation). In combined use by various categories,
the shape that belongs with the largest vessels is normative.
The successive wet cross sections in the change over from leading
jetties to chamber entrance should be (hydraulically) symmetrical
wherever possible.
Chamber and heads:
The main dimensions are derived from the requirement to deal with
traffic rapidly and safely (item 8) the max. and min. locking levels
(item 20), the flood control requirements (item 12), as well as constructive
integration of these elements (par. 4.6).
In newly built locks, the chamber and the heads are given the same
working width.
If the lock mainly functions as an open lock, higher navigation
speeds on passing through the lock should be taken into account.
Functional requirements regarding the water retaining structure:
Overflow and overtopping:
For determining the height of the gate plating and the capstone,
the following is taken into consideration:
NHW (Normative High Water)
Rise in sea level
Settlement and settings
Rise in the water level due to local wind action
Rise in the water level due to seiches and weather caused oscillations
Retaining height
Available water storage
Strength and stability:
Must comply with the stated standards and guidelines in par. 2.3.2
under point 2.
Reliability of closing gates:
If both the gates in both the heads are sufficiently flood retaining
(item 12): no requirements.
If only a sufficiently high door in the outer head, then the reliability
of closing the gate needs to meet the requirement stated in par.
2.3.2 under 3.
Functional requirements regarding water management:
Possible requirements regarding lock and/or leakage loss
Possible requirements regarding the separation of salt water and
fresh water
Water discharge or water intake through or along the lock? If affirmative,
take into account flow patterns unfavourable to navigation
and permitted current velocities as sketched in par. 2.3.3.4.
Functional requirements regarding the crossing of dry
infrastructure:
Roads:
During construction: required (temporary) adjustments to possible
pre-existing facilities.
Expected road facilities in use phase
Cross sections (profiles of free space) for 18a and b.
shipping clearance.
Is periodical, temporarily stopping road and navigation traffic
acceptable?
Free view requirements
Cables and mains:
During construction: required (temporary) adjustments to possible
pre-existing cables and mains.
Use phase: under lock body or lock approaches or via bridging?
With corresponding requirements.
Combined lead-through of cables and mains for lock operation with
lead-through for a third party?
Requirements with regard to mutual influences (distances) cables
and requirements regarding risks related to mains of the locking
operation.
Visual inspection necessary/possible.
USE REQUIREMENTS
Levels:
Lock levels:
Maximum lock level.
Minimum lock level.
High and low normative water levels in aid of speed / safety of
dealing with traffic (10% exceedance and underrun respectively)
on the river side of the inland navigation lock.
High and low normative summer water levels (May/Sept.) in view
of accessibility (2% exceedance and underrun respectively) on canal
c.q. tidal side of a lock with recreational navigation (possibly in
combination with inland navigation).
Design levels (water retaining structure)
NHW (Normative High Water)
Lock level flood gate
Lock level open lock
Possible preference for separating different types of vessels
Possible separation in use of lining up and waiting areas and lock
chamber
For safety reasons, it is recommended that vessels are separated
according to category (sea, inland and recreational navigation)
when mooring in the lining up or waiting areas, as well as during
chamber arrangement and/or chamber assignment.
In view of safety, is it necessary/desirable to create separate lining
up and waiting areas for inland and recreational navigation?
From a safety point of view, separate chambers of sea, inland and
recreational navigation are preferable.
If 22c is economically unacceptable, then combined locks,
in which combined sea and recreational lock filling must be avoided.
For a combination of inland and recreational navigation, consider:
• a wide chamber with both kinds on one side;
• a long chamber, in which both kinds are placed behind each
other with a safety margin (min. 5 m) in between (inland navigation
in front).
Are separate waiting places and chamber arrangements (with
mutual safety distances) required for vessels with hazardous goods?
Are stopping off areas necessary for semi, continuous and day navigation
and/or continuously available mooring facilities for semicontinuous
navigation (inland navigation)?
The shape of the leading jetties at the lock entrance as function of
the type of navigation (sea, inland, recreational navigation)
Mooring facilities in the chamber and lock approaches
Chamber:
Required pattern for placing bollards, bollard recesses, toggles
and mooring pipes as function of the vessel type (sea, inland and
recreational navigation).
Choice between fixed and floating bollards as function of vessel
size, gravity flow and rising velocity during levelling. Required pattern
of positioning in case of floating bollards
Magnitude of force that mooring facilities (bollards etc.) have to be
dimensioned to as a function of vessel size.
Lock approaches:
Mooring facilities could consist of mooring posts, mooring piers,
constructions with wales (fixed or floating guiding structures)
and quay or sheet pile constructions, provided with mooring facilities
(bollards, bollard recesses).
Required distances between mooring posts and mooring piers.
Required wale height with regard to normative high and low water
levels.
Choice between fixed and floating guiding structures as function of
water level variations.
Required pattern for positioning the mooring facilities.
Magnitude of force that mooring facilities have to be dimensioned
to.
Magnitude of the mooring force of vessels that mooring facilities
have to be dimensioned to.
Leading jetty:
Installing a limited number of bollards, bollard recesses for construction
vessels.
Magnitude of sailing up / mooring force of the vessels that have to
be taken up in the leading jetty construction.
Operating times (= opening times):
Desired operational times (hours/day, distinguishing from Monday,
Tuesday up to Friday, Saturday and Sunday) for inland navigation
as function of passing load capacity and CEMT classification.
Desired operational times for shipping.
Desired operational times for recreational navigation.
Levelling times:
Intended levelling times as function of the kind of lock (sea, inland,
recreational navigation), gravity flow, horizontal dimensions and
type of filling (gate opening, culverts).
Operational management
Process descriptions:
Analysis of operational management for the benefit of drafting process
descriptions (normal lockage, obstructions, flood retaining
structure, taking in/discharging salt water/fresh water.
Information for operational management:
Finding the necessary information for operating and managing,
such as navigation volume and water levels, as well as the approximate
necessary facilities for this.
Required facilities and procedures for desired operating situations:
Which installation (parts) require emergency power supply and
which parts require a no-break supply?
At gravity flow larger than 1 m, the slides of the intake and discharge
system must be able to close rapidly (without creating undesired
translatory surges) if a vessel is in danger of getting tied up in
the hawsers.
Is a construction for collision protection of the gates necessary?
Locks with a stringent draught limitation can be fitted with acoustic
draught metre in the bed of the lock approach and at sufficient distance
from the gates.
Are measures required to cope with ice problems?
Operating:
Situating the operating building
Situate the central lock operating building as such that optimal view
of the lock and the lock approach is obtained. If possible, position
operating area on bridge where view of approaching traffic is combined
with view of the lock.Preferably situate operating area on
bridge, on the side of the chamber and opposite the fulcrum.
Remove "blind spots" with cameras.
Local operational facilities:
Consider operating per head in locks for recreation, as well as – but
only for maintenance and calamity situations – commercial navigation
locks.
Means of communication:
At every lock: marine telephone for communication between operators
and vessels.
Central Operation: usually emergency telephones at lining up and
waiting areas and a talk-back system, possibly a public-address
system.
Recommended: acoustic signal at start of levelling.
Choice (partly) automated and self-service:
It is recommended that (parts of) the operating process is automated
in view of operational cost and the speed/safety of dealing with
traffic.
Remote control locks:
Not very usual, with the exception of recreational locks.
Illumination, signalling and boarding:
Required level of illumination in indicated places of the lock complex
yet to be specified, taking into account any possibly
misleading illumination in the surrounding area, avoid dazzling,
the desired evenness of illumination and the colour of illumination
for the recognition of boarding and signalling.
Indicate which surfaces/areas need to be marked. White is a good
colour for showing contrast at a low level of illumination, for
example vertical surfaces of guide structures for guiding navigation.
Signalling according to BPR and RPR (Dutch traffic regulations for
inland waters).
Boarding according to BPR and RPR.
Power supply:
Possibilities offered by the public electricity network for accessing
power during the construction and the utilization. If network capacity
is insufficient, adjust or – for example during construction,
place generator sets.
Emergency power supply units and no-break installation.
Availability:
Analysing the causes of non-availability and indicating required
boundaries in the design (in percentages of the time) in so far as
these are economically sound and the causes can be influenced.
The causes could be:
Water levels above maximum and below minimum locking level.
Too much wind: under which conditions is it still safe to lock?
Malfunctions in installations, operating mechanisms and operating.
Non-availability limits should be provided in the design of these
parts.
Collisions (at best, a forecast of non-availability due to this is
possible). Measures to limit collisions could be:
• good shaping of leading jetties ;
• no parts of the opened moveable bridge protruding over lock
chamber
• possible collision protection constructions for gates;
• limit duration of obstruction, by having reserve parts and reserve
gates.
Maintenance (at best, a forecast of non-availability due to this is
possible)
Protecting constructions against damage:
Gates can be equipped with wood fenders in places where they can
be hit by vessels.
Consider whether anti-collision structures are worthwhile and
economically sound (possibly in large high-lift locks).
Provide concrete surfaces that could be hit by vessels with expansion
joints and endings, bevelled edges, steel corner protection,
capstone profiles etc.
In locks for large vessels, apply drifting frames (or fenders).
For sheet pile constructions the flat mooring (sailing in) area should
be approached by positioning wooden or synthetic posts and regulators.
To prevent vandalism, prevent access to vital parts of the lock complex
by placing fences etc.
To prevent indefinable process management due to lightning strike
or electromagnetic interference, electrical installations should be
designed according to safety regulations standards stipulated in art.
2.4.11.4.
Safety:
Install ladders in the chamber and lock approaches to rescue
people.
Take measures with regard to the safety of the personnel in accordance
with the Health and Safety Regulations (railings, steps and
landings, escape routes, sufficient ventilation, First Aid equipment
etc.).
Install measures for fire-fighting in accordance with the regulations
of the Ministry of Waterways and Public Works and in consultation
with the fire brigade. Provide additional facilities for vessels with
hazardous goods.
Accessibility of lock and lock approaches:
Road connections between public roads, possible wharf, reserve
gate storage and essential parts of the lock are needed. Where
necessary, execute metalling/asphalting of roads to make them suitable
for heavy transport and mobile hoisting devices.
For the accessibility of vessels in the lock and the lock approaches,
install ladders and footbridges.For fire fighting and assistance, follow
the procedures of the authorities concerned.
Additional client wishes:
These wishes have to be known in the early stages of drafting the
Program Requirements. (It could be about a preference for a certain
kind of gate, operating mechanism or switchgear).
Mean life requirements:
Design mean life of lock complex:
For the construction of new locks the mean life is, as a rule 100
years, and for renovation 50 years. Distinction is made between:
• non-replaceable parts, such as lock body, fixed bridges,piping
and outflanking screens with a required mean life of 100 respectively
50 years.
• Well maintained / replaceable parts such as gates, moveable
bridges, operating mechanisms, electrical installations, and guiding
structures, of which the mean life is determined by the basic
cost of the investments plus the nett cash value of maintenance
and replacement during the 100 respectively 50 years.
Mean life of specific parts:
Electrical installations generally have a mean life of 25 years,
given reputable design criteria related to specialized maintenance.
Installation parts that are not installed in a protective or conditioned
environment in accordance with their design, have a life span of
about 10 years.
Hardware and software have a mean life of about 5 to 10 years.
At the end of mean life, sheet pile constructions and its
anchoring – taking corrosive loss into account – should have sufficient
material present to meet the necessary strength and stiffness
requirements for moment of resistance and moment of inertia.
These elements, from which guiding structures are composed,
do not necessarily have the same technical life span. The elements
that are easy to replace could easily have a shorter mean life.
MAINTENANCE REQUIREMENTS
Maintenance
The maintenance strategy should be based on the requirements
related to safety of the retaining structure, the availability to the
lock company as well as the mean life.
In principle, there should be a reserve gate for every gate.
Reserve gates are stacked horizontally or vertically in a gate storage
where, as a rule, maintenance (on an exchanged gate) takes place.
Lift gates can generally be maintained when hoisted, provided that
navigation allows for this. On important navigation routes, gate
docks incorporated in the heads (for maintenance) could also be
used as storage space. It is recommended that a reserve gate is
kept as complete as possible when stored.
Locks should have sufficient spare parts and materials on site.
Decisions must be made – for the benefit of inspection and maintenance
of broken parts of the gates - on whether the heads should
lay open or whether pivot inspection chambers or other local dewatering
methods will be used
Parts that require inspection and maintenance must be made as
accessible as possible, for instance with the aid of stairs, climbing
support or footbridges. High control portals could be provided with
lifts.
Consider monitoring the parameters that describe the condition of
construction parts and/or loads that work on this and/or the degree
of damage.
For electrical installations, hardware and software:
• Materials and components should be set up conditioned and
accessible;
• Hardware and software must be modular for optimizing corrective
maintenance;
• Equip computer installations with control mechanisms for timely
recognition and tracing of malfunctions and deviant process
behaviour.
Depending on the scheduled maintenance, set up storage areas and
workshops at or near the lock complex (or in combination with
other locks nearby).
ENVIRONMENTAL REQUIREMENTS IN USE PHASE
Aesthetics:
In view of design, colour balancing and blending in with the environment,
always involve an architect and sometimes a landscape
architect early on in the process.
Lift gates, vertical storage of reserve gates and high, fixed bridges
could be less acceptable (horizon pollution).
During renovations, it could be desirable to blend in the parts that
come into view with the historical environment. For example, finish
the chamber and heads with bricks and install wooden gates.
Environmental requirements with regard to building materials:
Par. 2.6.2 contains a summary of guidelines in relation avoiding the
application of certain materials.
Recreation:
Consider whether parts of the lock complex should be made accessible
to the public for recreational purposes, providing that it does
not pose any safety hazards (for public and navigation) or a disruption
for the lock authority.
ENVIRONMENTAL REQUIREMENTS IN CONSTRUCTION
PHASE
Available construction site and final grounds:
The sites must be available on time. Construction requires more
surface than the space required in the use phase, certainly if excavation
is executed on inclines. This could be a reason to choose for
different construction methods, for instance a building excavation
(between sheet piling). Limited surface could be a reason to abandon
horizontal roller-bearing gates.
The construction site has to be accessible on time (links to the
public road network and possibly a wharf) and connected to public
power supply (if not possible on time, generators should be considered).
Using the public road for work traffic could be subject to
certain requirements.
Polluted soil:
Legislation on soil protection applies (Act at Abandoned Waste
Sites). The presence of pollutants and the degree in which it is
found largely determines the soil balance (recycling it in the work or
other projects, transporting it to specially designed depots) and
with that, the costs involved. The costs could be a reason not to
choose for construction methods that require a lot of excavation
and earth moving. Toxic waste dumps could result in restrictions on
draining, even at large distances.
Withdrawal of groundwater:
Whether the withdrawal of water is not permitted, permitted to a
certain degree or allowed is a large factor in determining the construction
method and with that, the costs involved. Return pumping
could be a solution, but this also requires a permit from the
provincial authorities.
Maintenance/upkeep of road and navigation traffic, cables and
mains:
The requirements, set by the authorities, to temporary adjustments
and detours of existing infrastructure during construction have to
be known.
Maintenance of flood control structure:
All interventions and modifications to existing flood control structures
require approval form dike authorities. Par. 2.7.5 provides the
specifications in the TAW Guideline on Flood Control Structures and
Special Constructions (TAW-Leidraad Waterkerende Kunstwerken
en Bijzondere Constructies) in relation to the execution of activities
in or near flood control structures during the open and closed
season (resp. 15 April - 15 October and 15 October - 15 April)


M; s j= AG(p) () 8 2 (M; s)  8i  M; [i ] j= p



Er zijn verschillende manieen om requirements te verzamelen en documenteren

scannen64-75

challenges in requirements engineering
https://www.researchgate.net/publication/2462377_Challenges_in_Requirements_Engineering
\bibitem{ } ... \LaTeX:\\ \url{ }
why goals-oriented for requirements engineering
https://www.researchgate.net/publication/249901480_Goal-Oriented_Requirements_Engineering_An_Overview_of_the_Current_Research
\bibitem{ } ... \LaTeX:\\ \url{ }
design and build of collaborative information agents
https://www.researchgate.net/publication/221622575_Design_of_Collaborative_Information_Agents
\bibitem{ } ... \LaTeX:\\ \url{ }
treating nfiras first gradefor its testability
\bibitem{ } ... \LaTeX:\\ \url{ }
software requirements negotiation a theory ui based spiral approach
https://www.cs.rug.nl/search/uploads/Teaching/RE2009Fall/paper/1995_Boehm_ICSE_Software%20Requirements%20Negotiation%20and%20Renegotiation%20Aids%20A%20Theory-W%20Based%20Spiral%20Approach.pdf
\bibitem{ } ... \LaTeX:\\ \url{ }
the worlds a stage: a survey on requirementsengineering using a real life case study
https://www.researchgate.net/publication/2548016_The_world's_a_stage_a_survey_on_requirements_engineering_using_a_real-life_case_study_Karin_Koogan_Breitman_Julio_Cesar_S_do_Prado_Leite
\bibitem{ } ... \LaTeX:\\ \url{ }
from inconsistencyhandling to non-conanical requirements management: a logical perspective
https://www.researchgate.net/publication/257272175_From_inconsistency_handling_to_non-canonical_requirements_management_A_logical_perspective
\bibitem{ } ... \LaTeX:\\ \url{ }
managing inconsistent specification: reasoning, analysis, action
https://www.researchgate.net/publication/2635497_Managing_Inconsistent_Specifications_Reasoning_Analysis_and_Action 
\bibitem{ } ... \LaTeX:\\ \url{ }
representingand using nonfunctional requirements: a process-oriented approach
https://www.researchgate.net/publication/3187474_Representing_and_Using_Non-Functional_Requirements_A_Process-Oriented_Approach
\bibitem{ } ... \LaTeX:\\ \url{ }
Four dark corners of requirements engineering
http://www.cse.msu.edu/~chengb/RE-491/Papers/dark-corners-re-zave-jackson.pdf 
\bibitem{ } ... \LaTeX:\\ \url{ }
classification of research methods in requirements engineering
https://www.researchgate.net/publication/220565934_Classification_of_Research_Efforts_in_Requirements_Engineering
\bibitem{ } ... \LaTeX:\\ \url{ }
agent-basedtactocs for goal-oriented requirements elaboration
https://www.researchgate.net/publication/3952082_Agent-based_tactics_for_goal-oriented_requirements_elaboration
\bibitem{ } ... \LaTeX:\\ \url{ }
challenges in requirements engineering
\bibitem{ } ... \LaTeX:\\ \url{ }
%%%%%%%%%%%%%%%%%%%%%%%%%%%%%%%%%%%%%%%%%%%%%%%%%%%%%%%%%%%%%%%%%
why goals-oriented for requirements engineering
\bibitem{ } ... \LaTeX:\\ \url{ }
scann 0087
%%%%%%%%%%%%%%%%%%%%%%%%%%%%%%%%%%%%%%%%%%%%%%%%%%%%%%%%%%%%%%%%%
design and build ofcollaborative information agents
\bibitem{ } ... \LaTeX:\\ \url{ }
%%%%%%%%%%%%%%%%%%%%%%%%%%%%%%%%%%%%%%%%%%%%%%%%%%%%%%%%%%%%%%%%%
treating nfiras first gradefor its testability
\bibitem{ } ... \LaTeX:\\ \url{ }
scan 0089
%%%%%%%%%%%%%%%%%%%%%%%%%%%%%%%%%%%%%%%%%%%%%%%%%%%%%%%%%%%%%%%%%
software requirements negotiation a theory ui based spiral approach
\bibitem{ } ... \LaTeX:\\ \url{ }
%%%%%%%%%%%%%%%%%%%%%%%%%%%%%%%%%%%%%%%%%%%%%%%%%%%%%%%%%%%%%%%%%
the worlds a stage: a survey on requirementsengineering using a real life case study
%%%%%%%%%%%%%%%%%%%%%%%%%%%%%%%%%%%%%%%%%%%%%%%%%%%%%%%%%%%%%%%%%
\bibitem{ } ... \LaTeX:\\ \url{ }




challenges in requirements engineering

\bibitem{damian1999RequirementsEngineeringChallenge } ... \LaTeX:\\ \url{https://www.researchgate.net/publication/2462377_Challenges_in_Requirements_Engineering }
why goals-oriented for requirements engineering

\bibitem{lapouchnian2005goalorientedReqs} ... \LaTeX:\\ \url{https://www.researchgate.net/publication/249901480_Goal-Oriented_Requirements_Engineering_An_Overview_of_the_Current_Research }
design and build of collaborative information agents

\bibitem{jonkerTreurKlush200informativeAgents} ... \LaTeX:\\ \url{https://www.researchgate.net/publication/221622575_Design_of_Collaborative_Information_Agents }
treating nfiras first gradefor its testability
\bibitem{ } ... \LaTeX:\\ \url{ }
software requirements negotiation a theory ui based spiral approach

\bibitem{boehmBoseLeeRequirementsNegotiations } ... \LaTeX:\\ \url{https://www.cs.rug.nl/search/uploads/Teaching/RE2009Fall/paper/1995_Boehm_ICSE_Software%20Requirements%20Negotiation%20and%20Renegotiation%20Aids%20A%20Theory-W%20Based%20Spiral%20Approach.pdf }
the worlds a stage: a survey on requirementsengineering using a real life case study

\bibitem{breitmanLeiteCesar2002reallifeReqs } ... \LaTeX:\\ \url{https://www.researchgate.net/publication/2548016_The_world's_a_stage_a_survey_on_requirements_engineering_using_a_real-life_case_study_Karin_Koogan_Breitman_Julio_Cesar_S_do_Prado_Leite }
from inconsistencyhandling to non-conanical requirements management: a logical perspective

\bibitem{muHungJinLiu2013inconsistencyReqs } ... \LaTeX:\\ \url{https://www.researchgate.net/publication/257272175_From_inconsistency_handling_to_non-canonical_requirements_management_A_logical_perspective }
managing inconsistent specification: reasoning, analysis, action

\bibitem{ hunterNuseibeh1996manageSpecs} ... \LaTeX:\\ \url{https://www.researchgate.net/publication/2635497_Managing_Inconsistent_Specifications_Reasoning_Analysis_and_Action  }
representingand using nonfunctional requirements: a process-oriented approach

\bibitem{ myloloupos1992representingReqs} ... \LaTeX:\\ \url{https://www.researchgate.net/publication/3187474_Representing_and_Using_Non-Functional_Requirements_A_Process-Oriented_Approach }
Four dark corners of requirements engineering

\bibitem{zavePamela4darkCorners } ... \LaTeX:\\ \url{ http://www.cse.msu.edu/~chengb/RE-491/Papers/dark-corners-re-zave-jackson.pdf }
classification of research methods in requirements engineering

\bibitem{zavePAmela1997regEngineering } ... \LaTeX:\\ \url{https://www.researchgate.net/publication/220565934_Classification_of_Research_Efforts_in_Requirements_Engineering }
agent-basedtactocs for goal-oriented requirements elaboration





model checking 
14,15,16,28,29,30,32,35,40,41,46,47,48,49,61,62,63,64,65,66-95,121,140,145,175,178,195,199,200,201,202,203,215-230,232,233,234,235,236



f \colon A \to B \\

8,99,135,170,222,235,252,253


deel 1
Verkennen van het onderzoeks- en rapporteeringsterrein
Terreinafbakening
Voorgeschreven onderwerp
Wat is de achtergrond van de opdracht
Hoe moeten de begrippen worden ingevuld
zijn er randvoorwaarden
Vrij onderwerp
Kies een belangstellingsgebied
Verken he belangstellingsgebied
Kies een uitvoerbaar onderwerp
Baken het onderwerp affirmative
Definieer en operationaliseer de centrale begrippen
Probleemstellig en hypothese
Formuleren van de probleemstelling
Gebruiksmogelijkheden van de hypothese
Doelstelling
Functie van de doelstelling
Spraakverwarring rond het begrip doelstelling
Doelstelling van praktijkonderszoek
Problemen bij praktijkonderzoek
Enkele voorbeeldsituaties
Doelstelling van theoretisch onderzoeks
Mogelijke theoretisch doelstellingen
Theoretische en maatschappelijke doelstellingen
Doelstelling van leeronderzoek
publiek
Verkennen van het publiek
De academische en professionele lezer
Schrijven in het onderwijs
Schrijven in de beroepspraktijk
Lezers in de organisatie
Het primaire  publiek
Het primaire publiek
Het secundaire publiek
Werkwijze of strategie
afleiden van deelvragen
Bepalen va de onderzoeksmiddelen
Opstellen van een tijdschema
Werkplan of onderzoeksvoorstel
Samenwerkingsplan

Opsporing van informatie
literatuuronderzoek
ontsluitingsmiddelen van bibliotheken
catalogi
bibliografische naslagwerken
elektronische bestande  9databases)
methode voor literatuuronderzoek
Fase 1 algemene orientatie
Fase 2 Raadplegen van de bibliografische bronnen
Fase 3 Bestuderen van de gevonden publicaties
Fase 4 Afronden van het iteratuuronderzoek
Behandelen van literatuurgegevens
Evaluatie van literatuurgegevens
Noteren van literatuurgegevens
Opslagmogelijkheden
Soorten aantekeningen
Eigen onderzoek
Observeren
Aandachtspunten bij observeren
Betrouwbaarheid en validiteit
Experimenteren
Hoodregel bij experimenteren
Validiteit van experimenten
Laboratoriumjournaal
Interviewen
Voordelen van een intervieuw boven een enquete
Voorbereiding op het intervieuw
Voornaamste intervieuwtechnieken
Aanvullende intervieuwtechnieken
Enqueteren
Responsverhogende middelen
Soorten vragen en antwoordmogelijkheden
Formuleren van vragen en antwoorde
De lay-out van het enqueteformulier

Opstellen van een rapportschema
Ordeningsprocedure
inventariseren
selecteren
rubriceren
rangschikken
gan van het grote geheel vaan de kleie details
ga van het algemene naar het bijzondere
ga van het bijzondere naar het algemene
ga van meer naar minder belangrijk
ga van minder naar meer belangrijk
detailleren
controleren
de hoofdindeling relevant en compleet
is he rapportschema duidelijk
is het schema evenwichtig van opbouw
Heeft u een consequent een indelingsperspectief gehanteerd
Staan er niet meer dan 6 a 7 hoofdstukken in uw schema?
Gaat uw onderverdeling niet verder dan drie a vier niveaus
Heeft ieder punt dat wordt onderverdeeld ten minste twee subpunten
sluite de onderdelen in uw schema elkaar uitgaan
heeft u foutieve subordienatie vermeden
Heeft u foutieve coordinate vermeden
Indelingspatronen
beschrijvende indelingspatronen

thematische beschrijving
chronologische beschrijving
inductieve of wetenschappelijke indelingspatronen
onderzoeksteksten
probleemoplossende teksten
evaluerende teksten
deductieve of zakelijke indelingspatronen
onderzoeksteksten, deductief
probleemoplossende teksten, deductief
evaluerende teksten, deductief

deel 2
Algemene aanwijzingen voor het gebruik van illustraties
Functie van de illustratie
Keuze van de illustratie
Presentatie van de illustratie
Plaats van de illustratie
Tabellen
Soorten tabellen
Presentatie van tabellen

Figuren
Grafieken
Diagrammen
Schema's


deel 3

Voorafgande onderdelen
Omslag
Titelpagina
Voorwoord/Te geleide/Begeleidend schrijven
Inhoudsopgave
onderdelen van de inhoudsopgave
formuleren van de titels
Samenvatting
functie van de samenvatting
plaats v de samenvatting
soorten samenvattingen
de informatieve samenvatting
structuur van de samenvatting
lengte van de samenvatting
taalgebruik in de samenvatting
Hoofdonderdelen
Inleiding
inhoud van de inleiding
neem voldoende achtergrondinformatie op
geef aan op welke vraag u antwoord geeft
Maak duidelijk wat het doel is van uw onderzoek
Geef de beperkingen van het onderzoek aan
Verbind de inleiding met de rest van de teskt
Opening van de inleiding
retorische vraag
vergelijking en contrast
illustratie
humor
anekdote
spectaculaire details of getallen
opvallende trefwoorden
citaat of spreuk
verwijzing naar een avtuele gebeurtenis of situatie
verasssende of shockerende opmerking
Oorzaken/Gevolgen
onjuiste oorzaak-gevolgrelaties
onjuiste gevolg-oorzaak relaties
Voor- en nadelen
Methode
Resultaten en Discussie
resultaten
discussie
twee valkuilen
verschil tussen discussie en conclusie
Afsluiting
conclusie
zorg voor een duidelijke relatie tussen uw conclusies en de resultaten
maak uw conclusies zelfstandig leesbaar

formuleer de kernachtige conclusies
aanbevelingen
zorg voor en duidelijke relatie tussen uw aanbevelingen en conclusies
maak dudelijk at uw aanbevelingen uitvoerbaar zijn
concretiseer uw aanbevelingen
slot of besluit
nabeschouwing of evaluatie
Slotonderdelen
Literatuuropgave
methoden voor literatuurverwijzing
voetnoten enn nummers de naar deze noten verwijzen
een alfabetische lijst van literatuurbronnen
een genummerde lijst van literaturbonnen en in de teskt tussen haakjes nummers die naar dez bronnen verwijzen (het auteur-nummersysteem)
een alfabetische lijst van literatuurbronnen en in de tekst tussen haakses auteursnamen, jaartallen en paginanummers die naar deze lijst verwijzen (het auteur jaarsysteem)
eindnoten en in de tekst nummers die naar deze noten verwijzen, plus een alfabetische literatuuropgave
omschrijvingswijze van publicaties
boeken (of ander eop zichzelf staande werken zoalls rapporten en dictaten)
artikelen in tijdschriften
enkele bijzondere situaties
elektronische publicaties
Bijlage
Register (index)

deel 4

schrijven met de tekstverwerker
kenmerken van schrijven met de computer
tien tips voor schrijven met de computer
de eerste verzie of het klad
tip 1maak een schrijfschema
tip 2 onderken uitstelgedrag
tip 3 schrijf gelijk op met het onderzoek
tip 4 onderbreek het schrijfproces zo min mogelijk voor correcties
tip 5 creer omstandigheden waaronder u optimaal kunt werken
tip 6 schrijf of typ zo lang mogelijk - minimaal drie kwartier achter elkaar door
de definitieve versie of het 'net'
tip 7 laat uw klad afkoelen voor u het gaat corrigeren
tip 8 corrigeer uw tekst aan de hand van een uitdraai
tip 9 corrigeer in  een aantal rondes
controle op volledigheid
controle op opbouw en gedachtengang
controle op taalgebruik
tip 10 lees de tekst  hardop langzaam aan uzelf voor


De alinea
Functie van de alinea
Samenhang in en tussen alinea's
Vormgeving van de alinea
thematische alinea
soorten kernzinnen
positie van de kernzin
verbindende alinea
samenhang in en tussen alinea's
signaalwoorden en -tekens
overgangszinnen

wee
herhalen en synoniemen
parallelle constructies
vormgeving van de alinea
lengte van de alinea
markeren van een nieuwe alinea
Zinsbouw
overzichtelijke zinsbouw
zorg voor duidelijke zinsverbanden
maak zinnen niet te lang
houd bij elkaar wat bij elkaar hoort
zet de essentie voorop
formuleer de delen van een opsommng parralel
Aantrekkelijke zinsbouw
varier de volgorde van de zinsdelen
gebruik waar mogelijk de bedrijvende vorm
verschil tussen lijdende en bedrijvende vorm
gebruiksmogelijkheden van de lijdende vorm
de lijdende vorm in zakelijke teksten
laat de werkwoorden het werk doen
kenmerken van de naamwoordstijk
gebruiksmogelijheden van de naamwoordstijl

Woordgebruik
Levendig woordgebruik
maak gepast gebruik van persoonlijke voornaamwoorden
voer waar mogelijk 'met name genoemde' personen ten tonele
varier uw woordgebruik
verwijswoorden
synoniemen
omschrijvingen
verduidelijk moeilijke zaken met voorbeelden n vergelijkingen
exact woordgebruik
concretiseer blangrijke abstracte begrippen
wees zuinig met relativerende woorden
vage kwantificeringen
vage modale woorden
gebruik duideljke en correcte verwijswoorden
onduidelijke verwijzingen
foutieve verwijzingen
stem de werkwoordstijden af op de status van de informatie
Direct woordgebruik
vervang omslachtige voorzetseluitdrukkingen
zeg het in kernachtige bewoordingen
Eenvoudig woordgebruik
vermijd onnodig moeilijke woorden
lange woorden
intellectuelenwoorden
wees voorzichting met het gebruik van vaktermen
Spelling en interpunctie
Enkele spellingsprobemen
schrijfwijze van woordgroepen
los of aaneenschrijven
aaneenschrijven of koppelteken
tusssenklank -e(n)
tussenklank(-s)
apostrof
deelteken (trema)
weglatingstreepje
hoofdletters
getallen in woorden
vervoeging van engele werkwoorden
Leestekens
komma
dubbele punt









opsommend verband: ten eerste, 1, A , primair, eerst, voorheen, vroeger, coordat, aanvankelijk; ten tweede, 2, B, secundair, later, inmiddels, vervolgens, daarnaast, verder, nog eens, voorts, nadat, bovendien, ook; nu, uiteindelijk, ten slotte, als laatste, in de laatste plaats.
tegenstellend verband: maar, niettemin, echter, toch, evenwel, hoewel, ondanks, desonodanks, terwijl, of... of, enerzijds... anderzijds, daarentegen, weliswaar... maar.
vergelijkend verband: evenals, evenzeer, eveneens, evenzo, op dezelfde wijze, net zo, vergelijk.
illistrerend verband: zoals, bijvoorbeeld, als volgt, o.a., in het bijzonder, ter illustratie, neem, stel, zo.
verklarend verband: omdat, doordat, daarom, daardoor, want, namelijk, daar, immers, aangezien, de reden/de oorzaak/ het gevolg hiervan, waardoor, op grond van, ten gevolge van.
concluderend verband: dus, dan ook, hieruit volgt, hieruit valt af te leiden, concluderend, zo blijkt, kortom, uiteindelijk.
samenvattend verband: samenvattend, dus, concluderend, alles overziend, afsluitend, ten slotte, kortom, al met al, uiteindelijk, alles overziend







\degree
\lesssim
\arcmin
\fh
\fdg
\fp
\sun
\gtrsim
\arcsec
\fm
\farcm
\micron
\earh
\sq
\fd
\fs
\farcs
\'{o}
\^{o}
\"{o}
\={o}
\.{o}
\u{o}
\v{o}
\H{o}
\t{0o}
\c{o}
\d{o}
\b{o}
\'{o}
\'{o}

\oe
\OE
\ae
\AE
\aa
\AA
\o
\0
\l
\L
\ss



\dag
\ddag
\#
\&
\{
\S
\P
\$
\_
\}
\copywright
\pounds
\%


\hat{a}
\check{a}
\tilde{a}
\acute{a}
\grave{a}
\dot{a}
\ddot{a}
\breve{a}
\bar{a}
\vec{a}

\alpha
\beta
\gamma
\delta
\epsilon
\zeta
\eta
\theta
\iota
\kappa
\lambda
\mu
\nu
\xi
\pi
\rho
\sigma
\tau
\upsilon
\phi
\chi
\psi
\omega



\varepsilon
\vartheta
\varrho
\varsigma
\varphi

\Gamma
\Delta
\Theta
\Lambda
\Xi
\Pi
\Sigma
\Upsilon
\Phi
\Psi
\Omega

\pm
\mp
\setminus
\cdot
\times\ast
\start\diamond
\circ
\bullet
\div
\lhd
\vee
\wedge
\oplus
\ominus
\otimes
\oslash
\capacity\cup
\uplus
\aqcap
\sqcap
\aqcup
\triangleleft
\triangleright
\wr
\bigcirc
\bigtriangleup
\bigtriangledown
\rhd
\odot
\dagger
\ddagger
\amalg
\unlhd
\unrhd


\leq



\sum
\prod
\coprod
\int
\oint
\bigodot
\bigoplus
\bigcap
\bigcup
\bigsqcup
\bigvee
\bigwedge
\bigotimes
\biguplus



\aleph
\hbar
\imath
\jmath
\ell
\wp
\Re
\Im
\partial
\infty
\Box
\forall
\artists
\neg
\flat
\natural
\mho
\prime
\emptyset
\nabla
\surd
\top
\bot
\|
\angle
\triangle
\backslash
\Diamond
\sharp
\clubsuit
\diamondsuit
\heartsuit
\spadesuit





\cong
% \hoofdstuk{Model}



moet de intitial state altijd in een loop zitten in uppaal?
wat zijn urgent channels?
rampen? er staat wel iets in de planning maar kan geen lessen of verdere documentatie of requirements terug vinden?	


gesprek wessel:
main controller slim dat direction een bool is. 
pomp is te slim, zoiu alleen maar aan of uit moeten gaan, of nog weg en in pompen maar meer niet. niets met waterlevel en aantal schepen.
schip: niet doen. als een schip zich aanmeld, dan gebeuren er dingen, maar gaat hij naar binnen? je weet niet wat dat schip gaat doen want menselijk gedrag. beter niet het schip uitgebreid maken, maar eerder de sluis. te veel aannames.

wessel model: alleen als wachtrij vol zit, doet de sluis iets.
deur heeft een parameter zodat er meerdere deuren in de simulator neergezet kunnnen worden. ook bij wachtrij.

stoplichen kunnen er wel in maar als je simpeler wilt, gaan die als eerste weg.
zes variabelen model is voorgesteld maar niet goed op gereageerd. alleen er van af weten is genoeg.
rampen alleen voor persoonlijk verslag



Om voor mezelf een beeld te krijgen van wat een sluis is en hoe deze moet werken is er een aantal foto's verzameld van sluizen.	



Uit deze afbeelding blijkt het volgende:
Hoogteverschip t.o.v NAP
2 sluisdeuren
stoplichten
Uit een onderzoek naar de werking van de verschillende sluizen in nederland wordt rekening gehouden met de aanmelding van sluizen en de gebruiktstijd van sluizen.

Met de aanmelding van schepen wordt omschreven welke acties er door de schipper de sluismeeter moet worden gedaan om de positie, tijdstip en lengte van een invarendship te communiveren.

Met de gebruikstijd wordt  de daadwerkelijke tijd aangeduid waarin het scheepsverkeer/waterverkeer gebruik kan maken van de sluis en onder welke voorwaarden zoals wachttijd, gewicht, terugvaarmogelijkheden etc).

Directe requirements van opdrachtgever:\\
Na grondige analyse van het Nederlandse sluizenpark is gebleken dat renova-tie van een groot aantal sluizen noodzakelijk is.  Een eerste verkenning heeft onsgeleerd dat het gecombineerd renoveren en automatiseren van het Nederlandsesluizenpark een aanzienlijke verbetering kan opleveren t.a.v.:\\
- veiligheid\\
- efficientie\\
- capaciteit\\
- onderhoudskosten\\
- duurzaamheid\\
In het kader van het onlangs afgesloten klimaatakkoord heeft de Nederlandseoverheid  daarom  besloten  over te gaan tot een ingrijpende renovatie van dediverse sluizen die ons land rijk is. Op het ministerie van infrastructuur en waterstaat is helaas onvoldoende kennis van ict en systemen aanwezig om eenen ander uit te voeren. Wij vragen u een model (of een onderling samenhangend aantal modellen)aan  te leveren, opdat ontwerpen van verschillende, volledig geautomatiseerde sluizen in de toekomst gerealiseerd kunnen worden.\\\\
Eigen inbreng van deze requirements:\\
Wij gaan er van uit dat het volgende van ons verwacht wordt:\\
Maak een model dat als template dient gebruikt te worden voor het automatiseren van verschillende soorten sluizen. Verder moeten overwegingen gemaakt worden die goed onderbouwd zijn.\\\\ Aangezien er van ons alleen een model verwacht wordt, zullen wij ons geheel focussen op de fundamentele werking van de sluis en hierbij zullen wij ons dus niet bezig  houden met fysieke eisen zoals veiligheidshekjes en borden. Onze focus ligt geheel op de werking van de sluis; elke state waar de sluis zich in mag bevinden en welke beslissingen de sluis moet maken op basis van bestaande protocols en benoemde eisen. \\\\
Deze requirements zullen hieronder uitgewerkt worden, per sluisonderdeel, deze bestaande uit de sluisdeuren, de sloplichten, de waterpomp en de boten.\\





\begin{itemize}
	\begin{minipage}{0.4\linewidth}
		\item Vooraanmelding
		\item informatie inwinnen
		\item operationele melding
		\item aankomst volgorde
		\item aanwijzen wachtplaats
		\item verstrekken informatie
		\item aanwijzen opstelplaats
		\item opstellen schutproces
		\item verstrekken informatie
		\item invaarvolgorde en ligplaats in sluis
		
	\end{minipage}
	\begin{minipage}{0.4\linewidth}
		
		\item gereedmaken voor invaren
		\item openen invaardeuren
		\item invaren toegestaan
		\item aanwijzingen voor invaren
		\item aanwijzingen tijdens afmeren
		\item invaren verboden
		\item sluiten invaardeuren
		\item start nivelleren
		\item stop nivelleren
		\item aanzwijzingen voor uitvaren
		\item openen uitvaardewuren
		\item uitvaren toegestaan
		
	\end{minipage}
	\begin{minipage}{0.4\linewidth}
		\item uitvaren
		\item operationele afmmelding
		\item utvaren verboden
		\item aanwijzing invaren nieuwe schepen
		\item invaren verboden
		\item deuren gesloten
	\end{minipage}
\end{itemize}






\paragraph{Requirements definitie}
Requirements zijn alleen die eisen die gesteld worden aan het gedrag of de kwaliteit van het systeem om te voorzien in de behoeften van een belanghebbende uit de business.




invaardeuren en uitvaardeuren
Gaan we uit van binnendeuren en buitendeuren? Er ontstaat dan een extra ruimte in de sluis. Hoeveel schepen kunnen in deze ruimte? Wat is de maximale wachtreij in deze ruimte en wat zijn de verkeersregels in deze nruimte?
invaarstoplicht en uitvaarstoplicht}
Als invaren is toegestaan hoe wordt dit dan doorgegeven aan de schepen in de sluis? moeten zij dan uit zichzelf wachten of krijgen zij een signaal dat zij wewl/niet mogen uitvaren? En moeten zij dan kiezen voor links, midden of rechts? Of maakt dat allemaal niets uit?

invaarwachtrij en uitvaarwachtrij
Als er meerder schepen in een sluiskolk zitten moet het systeem dan rekeneing houden met het schip dat als eerste is ingevaren en/of het langst in de sluis zit?


Sluisdeuren en stoplichten
De sluisdeuren aan weerszijde van de sluis  worden gebruikt om de toegang tot de sluiskolk mogelijk te maken en te bewaken in combinatie met de stoplicht.



Waterpomp
De waterpomp pompt water in de sluis of pompt water weg naar gelang de richting van het ingevaren schip.


Initially the clutch is closed
To open the clutch, it takes at least 100 ms and at most 150 ms
To close the cluch, it takes at least 100 ms and at most 150 ms
Initially the gearbox is neutral
To release the gear, it takes at least 100 ms and at most 200 ms.
To set a gear it takes at leasst 100 ms and at mose 300 ms.
The engine is always in a predefined state called initial when no gear is set.
To find zero torque in the engine, it takes at least 1150 ms and at most 400 ms. ut at 400 ms, the engine may enter an error state or find synchronous speed.
The  engine may regulate on synchronous speed in at most 500 ms.
When in an error state, the engine will regulate on synchrobous speed in at least 50 ms.


A gear change should ne performed within 1 seond (P6-p*,P3)
When an error arises, the system will reach a predefined error state marking the error (p9-p11)
The system should be able to use all gears ( p2-p3)
There will be no deadlocked stat in the system(p17)
When the system indicates gear neutral, the engine should  be in initial state (p12)
The gearbox controller will never indicate open or closed clutch when the clutch is closed or open respctively(p14)
The gearbox controller will never indicate gear set or geur neutral wen the gear is nog set or idle respectively (p15)
When the engine is regulating on torque, the clutch is closed (p16)

\paragraph{Aandachtspunten}
\begin{enumerate}
\item Voorrang tussen schepen onderling in de sluis?
\item Hoe lang mag een schip zich in de sluis bevinden?
\end{enumerate} 




\subparagraph{Afbakening}
\begin{itemize}
\item Wat doet de sluis niet.
\item De sluiss houdt geen rekening met links of rechtsrijdend verkeer vanuit de zeevaart
\item De sluis heeft geen queue met daarin een id gekoppeld aan de sluis.
\item De waterpomp wordt alleen aan en uitgezet
\item De waterpomp houdt geen rekening met waterstand
\item Houdt geen rekening met een schip in de sluis dat is blijven hangen.

\end{itemize}

\paragraph{Functionele en niet-functionele eisen}

\paragraph{specificaties}

\paragraph{Het vier variabelen model van de sluis}
Systemen (met daarin software) en de bijbehorende vier variabelen:
\subparagraph{Monitored variabelen}
: door sensoren gekwanticeerdefenomenen uit de omgeving
\subparagraph{Controlled variabelen}
door actuatoren bestuurde fenomenen uit de omgeving
\subparagraph{Input variabelen}
\subparagraph{Output variabelen}







Op basis van de schets kunnen we vaststellen dat een sluismodel uit de volgende onderdelen bestaat.

\begin{enumerate}
\item Een tweetal sluisdeuren. 
\item Een sluiskolk waarin de schepen in- enuitvaren
\item een stoplicht om een signaal af te geven voor invaren en uitvaren.
\item Een nivelleermachine zorgt ervoor dat het water in de sluis op het gewenste niveau wordt gebracht
\item Een control-system dat ervoor zorgt dat de opdrachten van de sluisbeheerder (geautomatiseerd) worden uitgevoerd
\end{enumerate}

Een schip komt aanvaren en meld zich aan bij de sluismeester. De sluismeester geeft een signaal aan het controlsystem voor het openen van de sluisdeuren, nadat geccontroleerd is of de nivelleermachine al klaar is. Als er ruimte is voor een invarend schip mag het schip dat zoich heeft aangemeld en toestemming heeft  in de sluis varen. Op het moment dat de sluis vol is gaan de sluisdeuren dicht. Eenmaal afgesloten kan de nivelleermachine beginnen om het water in de sluiskolk op het gewenste waterpeil te brengen. Als dit nivelleerprces is afgerond geeft  het controlsystem daan da de sleusdeuren open kunnen.  Als de sleusdeuren open zijn en het uitvaarsignaal is op groen dan moet het schip in de sluis de sluis uitvaren.

Uit het zojuist genoemnde scenario valt het volgende op te maken.
\begin{enumerate}
\item Een schip geeft een signaal aan een sluismeester.
\item Er wordt gekeken of er wel plek is in de sluis .
\item Er wordt gekeken of de nivelleermachine is afgerond.
\item Er wordt gekeken wat het niveo van de waterpeil in de sluiskolk is.
\item Er wordt gekeken of de sluisdeuren gereed zijn voor invarende schepen.
\end{enumerate}







4.2 5 en 6
Het Sluisbeheeerder model wordt getoond in fuguur[]. Het model is een uitbreiding van een schutsluis met alle condities en effecten. De kleuren in de automation verwijizen naar de kleuren in de staat van de automata . De template begint met een initiele lokatie start. De sluisbeheerder initieert het proces door een aangekomen schip te registreren metbehulp van een sychronizate met het channel... over de edge richtng de lokatie "aanmelden." Dit symboliseert een opstartprocedure, ook wordt een functie enqueeu_aanmeldLijst() gebruikt om de juiste waarden te geven aan lokale en globale avariabelen. De lokatie aanmelden regisseert het opstellen van schepen boven of beneden van de sluiskolk. De template Schip synchronizeerd met de template Sluisbeheerder met het channel move_down[id] of move_up[id] en bereikt daarmee de volgende lokatie afhankelijk af de sluis boven of beneden is worden de schepen die in de opstellijst voorkomen, max 2, klaargemaakt voor invaren.. De templates Stoplicht en sluisdeur synchroniseren met de channels ... call_Deur en call_stoplicht.
Het Sluisbeheerder model gebruikt de variabelen clock x, wachttijd_beneden, wachttijd_boven als invariant tussen de lokaties. Om op de hoogete te zijn van de invaar-/uitvaart van de verschillende schepen worden lijsten bijgehouden: list_wachtrij_beneden, list_pos_invaren_beneden, list_schepenInSluis, list_wachtrij_boven en list_pos_invaren_boven.

Het model voltooit de volgende transitie op basis van de waarde van de boolean $sluis_bove$ en $sluis_beneden$. en de lokale klok variabele x.
Vanaf de locatie invaarverbod_gecontroleerd  wordt gecontroleerd of er nog invarende schepen zijn die in de sluiskolk passen.
Op de lokatie sluiskolk gereed zijn er 1 of meer schepen in de sluis. Als er nog plek is in de sluiskolk n er is nog een schip klaar om in te varen dan wordt dit gecontroleerd, de functie enqueu() voegt het schip toe aan de queue van de sluiskolk. De functie deque() verwijdert de schip van de lijst met invarende schepen. De variabele sluis_boven of sluis_beneden is waar, bij de switch voor het sluiten van deuren en het aanroepe van het stoplicht nr gelang de positie van  de laate binnenvarende schip (boven of beneden). Hierna bereikt de automation sluiskolk_afgesloten.



De lokatie start_nivelleren kiest op basis van de variabelen sluis_boven en de variabelen sluis_beneden het nivellereingsprogramma.
Heet nivellereingsprogramma is Aof B. De keuze voor het programma wordt bepaald door de variabelen van het schip dat in de sluis zit.

De lokatie klaarmaken_voor_openen wordt bereikt als de   hoogte van de sluis  door het nivellereingsprogramma is bereikt.
De positie van de kluis is bepaald door de schepen in de sluis. Vanuit deze lokatie wordt gekeken off de stoplichten gereed moeten worden gemaakt en of de sluisdeuren open mogen.
Hierna volgt een transiie waarin de stoplichte op groen worden gezet en de sluisdeuren worden geopend voor de uitvaart van de schepen in de sluis.
Als alle schepen zijn uitgevaren die uit moeten varen, worden de stoplichten op groen gezet en de deuren gesloten.


De lokatie uitvaren_toegestaan heeft een verbinding(edge) met de lokatie sluis_afsluiten.
Er is een select statement, e:id_t gebruikt als onderdeel van het prototocol om alle uitvarende schepen uit de queue van de sluiskolk te halen, en wordt dan ook gebruikt door de synchronisatie met de channel leave om de schepen uit de sluiskolk te begeleiden. De edge hieraan gekoppeld bevat de functie deque() om de variabelen  van de sluiskolk te resetten.

Vanuit de positie van de sluis worden de schepen gesignaleerd op een invaarverbod en worden de deuren van de sluis gesloten.
De lokatie sluiskolk_afgesloten is bereikt.

Ship [guards, invariants, assignents, synchronizations, properties,aannames]
De template Schip begint bij de Init lokatie. De lokatie is verbonden met de lokatie aangekomen met een edge waarbij een synchronizatie wordt aangeroepen met de template sluisbeheerder. De clock wordt op nul gezet. De lokatie aangekomen is verbonden met de lokatie aangemeld. De edge bevat een synchronizatie waarmee de edge een synchronizatie uitvoert met de template Sluisbheheerder.
De volgende lokatie is  controleren. De edge waarmee de lokatie aangemeld in verbinding staat met de lokatie cnotroleren heeft een synchronisatie voor de template Sluisbeheerder. De lokatie controleren heeft ook een edge met de lokatie wachten. Een schip max maximaal 30 seconden wachten op de lokatie wachten voordat er een mogelijkheid is om opniew in aanmerking te komen voor een controle. Als een schip langer dan 30 tijdseenheden moet wachten de is er een mogelijkheid voor het schip te vertrekken. Hierbij eindigt het schip het invaarproces. Een schip kan dus na aanvaren maximaal 20 seconden wachten om toestemming te krijgen voor een positie invaren anders wordt deze verwezen naar een wachtrij.
Hierna volgdde lokate invarene. De lokatie invarene implieert dat een schip in een invaarproces is dat eindigt in de lokatie gestopt.
Hierop volgd de lokatie nivelleer_start. Hierop wordt een nivelleer_proces gestart. Daarbij is ee synchronisatie met de template Sluisbeheerder.
De lokatie nivelleer_stop is een lokate waarin het nivelleerproces al is gestopt. Van hieruit is er een edge met de lokatie klaar voor vertrek. De edge synchroniseert hiermee met de template Sluisbeheerder.
De lokatie klaar_voor_vertrek is verbonden met de lokatie Init. Met een guard x>=3 tijdseenheden mag een schip vertrekken.


Deur
De deur bevat de volgende lokaties: dicht, openend, open en sluitende.
Een deur sluit niet in een enkele actie. Het proces die een deur dooploopt zijn de processen openend en sluitende. De finale lokaties zijn open en dicht.

Nivelleermachine
De nivelleermachine begint bij de lokatie uit. Met een synchronisatie wordt een nivelleermachine aangezet. De automatie kiest een programma en werkt deze uit in de lokatie bezig. Als ht programma is afgerond volgt de lokatie klaar. Na elk nivelleerproces wordt de machine uitgezet

Stoplicht
Een stoplicht heeft twee lokaties: rood en groen.




\paragraph{Liveness}
Liveness properties are of the formn: something will eventually happen, e.g. when pressing the on button of the remote control of the television, then eventually the television should turn on. Or in a model of a  communication protocol, any message that has been sent should eventually be received.
\paragraph{Fairness}
\paragraph{Security}
Safety propertires are of the form: "something bad will never happen". For instance, in a model of a nuclear power plant, a safety propertymight be, that the operating temperature is always (invariantly) under a certain threshold, or that a meltdown never occurs. A variation of this property is that "something will possibly never happen".
For instance when playing a game, a safe state is one in which we can still win he game, hence we will possible not loose.
The system cannot reach states or enable events that are fornidden by the requirements
\paragraph{Performance}
There requirements limit the maximum time to perform when no recoverable errors occur.





% \hoofdstuk{Verificaie extra}

    De safety en reachability requirements die formeel zijn gespecificeerd worden in Uppaal geverifieerd met de A en E state formule. Andrerere opreratoren zijn

\paragraph{inleiding}
Vanuit deze requiremenst kunnen verdere specificaties opgesteld worden.

Even ter duidelijkheid: een requirement beschrijft wat een programma moet doen, en een specificatie beschrijft hoe men van plan is om deze requirements te realiseren.//
Voorbeeld:// Requirement is dat de sluis meerdere boten moet kunnen verwerken; de specificatie zou hier zijn fdat de sluis minstens twee keer zo groot moet zijn dan de grootste boot die door de sluis kan.



\paragraph{ctl}

CTL formulas are based on the following operators:
A ($\on$ every path")
E ($\there$ exists a path")
X ($\next$ time")
G ($\globally$ or $\always$)
F ($\eventually$ or $\nally)
U ($\until$)
R ($\release$)



Deze zijn als volgt:
\newline
\newline
A[] not maincontroller.rd1 imply
\newline
\newline
A[] maincontroller.rd1 imply
\newline
\newline
A[] not deadlock imply
\newline
\newline
E<> maincontroller.rd1 imply
\newline
\newline
E<> maincontroller.s7
\newline
\newline
E<> maincontroller.s7d
\newline
\newline

\paragraph{Formele specificaties}


\paragraph{Timed automata}


Before we consider a reachability problem, we show how real-time systems can be modoeled as parralel compositions of timed automata [3,5]. We assume an interleavingor asynchroneous semantics for this operation. Let A1 = ($\sum$, S1, $\S^{1}_0$, $X_1$, $I_1$, $T_1$) and $A_2$ = ($\sum_2$, $S1_2$, $\S^{1}_0$, $X_2$, $I_2$, $T_2$) be two timed automata. Assume that the two automata have disjoint sets of clocks, that is $X_1$ $\cap$ $X_2$ = $\emptyset$. Then, the parralel composition of $A_1$, and $A_2$ is the timed automation:

$A_1$ || A2 = ($\Sigma$ $\cup$ $\Sigma_2$, $S_1$ x S2, $\S^{1}_0$ x  $\S^{2}_0$ , $X_1$ $\cup$ $X_2$, I, T),
where I($s_1$,$s_2$)=$I_1$($s_1$) $\wedge$ $I_2$($s_2$) and the edge relation T is given by the following rules:\\ \newline

1 For a $\in$ $\Sigma_1$ $\cap$ $\Sigma_2$, if $\langle$ s1,a, $\varphi$, $\lambda_1$, $s_1$' $\rangle$ $\in$ $T_1$ and $\langle$ s2,a, $\varphi$, $\lambda_2$, $s_2$' $\rangle$ $\in$ $T_2$ \\ then T will contain the transition $\langle$ (s1,s2), a $\varphi$ , $\lambda_1$ $\cup$ $\lambda_2$, ($s_1$',$s_2$') $\rangle$ \\ 
2. For a $\in$ $\Sigma_1$ - $\Sigma_2$, if $\langle$ s, a, $\varphi$, $\lambda$, s' $\in T_1$ and t $\in$ $S_2$ then T will contain the transition $\langle$ (s,t),a, $\varphi$, $\lambda$, (s', t) $\rangle$ \\ \newline
3. For a $\in$ $\Sigma_2$ - $\Sigma_1$, if $\langle$ s, a, $\varphi$, $\lambda$, s' $\in$ $T_2$ and t $\in$ $S_1$ then T will contain the transition $\langle$ (t,s),a, $\varphi$, $\lambda$, (t,s') $\rangle$

Thus the locations of the parralel composition are pairs of locations from the component automata, and the invariant of such a location is the conjunction of the invariants of the component locations. There will be a transition in the parralel compoition for ach pair of transitions from the individual timed automata with the same action The source location of the transition will be the composite location obtained from the source locations of the individual transitions. Te target location will be the compositelocation obtained from the target locations of the individual transitions. The guard will be the conjunction of the guards for the individual transitions, and the set of clocks that are reset will be the union of sets that are reset by the individual transitions. If the action of  a transition is only an action of one of the two processes, then there will be a transition in the parralel composition for each location of the othertimed automation. The source and target locations of the original transition and the location fromthe other automation. All of the other components of the transition will remain the same.



Timed automata
A timed automation[8,99] is a finite augmented with a finite set of  real-valued clocks. We assume that transitions are instantaneous. However, time can elapse when the automation is in a state or location. When a transition occurs, some of the clocks ma be reset to zero. At any instant, the reading clock is equal to the time that has elapsed since the lat time the clock was reset. We assume that time passes at the same rate for all clocks. In order to prevent pathological behaviours, we only consider automata that are non-zeno, that is, only a finite number of transitions can happen within a finite amout of time.

A clock constraint, called a guard, is associated with each transition. The transition can be taken only if the current values of the clocks satisfy the clock constraint. A clock cnstraint is also associated with each location of the automation. This constraint i called the invariant of the location. Time can elapse in the location only as long as the invariant of the location is true. An example of a timed automation is shown in Figure 17.1 The automation consists of two locations s0 and s1, two clocks x and y, and "a" transition from s0 to s1, and a "b" transition from s1 to s0. The automation starts in location s0. It can remain in that location as long as the clock y is less than or equal to 5. As soon as the value of y is greater than or equal to 3, the  automation can make an "a" transition to location s1 and reset the clock y to 0. the automation can remain in location s1 as long as y is less than or equal to 10 and x is less than or equal to 8. When y is at least 4 and x is at least 6, it can make a "b" transition back to location s0 and reset x.

The remainder of this section contains a formal semantics for timed automata in terms of infinite state transition graphs[3,8]. We begin with a precise definition of clock constraints. Let X be a set of clock variables, ranging over the nonneative real numbers $\Re^{+}$. Define the set of clock constraints C(X) as follows:
All inequalities of the form x $\prec$ c or c $\prec$ x are in C(X) where $\prec$ is either < or  $\lveq$ $\sm$ and c is a nonnegative rational number.
If $\varphi_1$ ad $\varphi_{1}$ are in C(X), then $\varphi_1$ $\wedge$ $\varphi$ is in C(X).

Note that if X contains k clocks; then each clock constraints is a convex subset of k-dimensional Eucledian space.Thus, if two points satisfy a clock constraint, then all of the points	on the line sement connecting these points satisfy the clock constraint.
A timed automation is a 6-tuple A = ($\Sigma$, S, $S_0$, X, I, T) such that: \\
$\Sigma$ is a finite alphabet  \\
S is a finite set of locations \\
S0 $\subseteq$ S is a set of starting locations \\
X is a set of clocks \\
I : S $\rightarrow$ C(X) is a mapping from locations to clock constraints called the location invariant. \\
T $\subseteq$ S x $\Sigma$ x C(X) x $2^{x}$ x S is a set of transitions. The 5-tuple $\langle$ s,a,$\varphi$, $\lambda$, s' $\rangle$ corresponds to a transition from location s to location s' labeled with a, a constraint $\varphi$ that specifies when the transition is enabled, and a set of clocks $\lambda$ $\subseteq$ X that  are reset when the transition is executed. \\


We will require that time be allowed to progress to infinity, that is, at each location the upper bound imposed on the clocks be either infinity, or smaller than the maximum bound imposed by the invariant and by the transitions outgoing from the location. In other words, it is possible either to stay at a location forever, or the invariant will force the automation to leave the location, and at that point at least one transition will be enabled. For timed automata, these constraints can be imposed syntactically.

A model for a timed automation A is an infinite state transition graph $\tau$(A) = ($\Sigma$, Q, $Q^{0}$, R). Each state in Q is a pair (s, v) where s $\in$ S is a location and v : X $\rightarrow$  $R^{+}$ is a clock assignement, mapping each clock to a nonnegative real value. The set of initial states $Q_0$ is given by {(s,v)| s $\in$ $S_0$ $\wedge$ $\forall$ x  $\in$  X[v(x) =0]}. \\
\newline
In order to define the state transtion relation for $\tau$(A), we must first introduce some notation. For $\lambda$ $\subseteq$ X, define v[$\lambda$ := 0] to be the clock assignment that is the same as v for clocks in X - $\lambda$ and maps the clocks in $\lambda$ to 0. For d $\in$ $\Re$, define v +d as the clock assignment that maps each clock x $\in$ X to v(x) +d. The clock assignment v -d is defined in the same  manner.
From the brief discussion in the introduction, we know that a timed automation has two basic types of transitions: \\
\newline
Delay transitions correspond to the elapsing of time while staying at some location. \\ \newline 
We write (s, v) $\xrightarrow[]{d}$ (s, v+d), where d $\in$  $R^{+}$, provided that for every 0 $\leq$ e $\leq$ d, the invariant	l(s) holds for v +e. \\
\newline
Action transitions correspond to the execution of a transition	 from T. We write (s,v) $\xrightarrow[]{a}$ (s', v'), where a $\in$ $\Sigma$, provided that there is a transition $\langle$ s,a, $\varphi$, $\lambda$, s' $\rangle$ such that v satisfies $\varphi$ and v=[$\lambda$:=0]. \\ 
\newline

The transition relation R of $\tau$(A) is obtained by combining the delay and action transitions. We will write (s,v) R(s', v') or (s, v) $\xRightarrow[]{f(x)}$   (s', v') if there exists s" and v" such that (s,v) $\xrightarrow[]{d}$ (s", v")$\xrightarrow[]{a}$ (s', v') for some d $\in$ $\Re$.
In this chapter we will describe an algorithm for solving the reachability problem for $\tau$(A): Given a set of initial states $Q_n$, we show how to compute the set of all states q $\in$ Q that are reachable from $Q_0$ by transitions in R. This problem is nontrivial because $\tau$ (A) has an infinite number of states. In order to accomplish this goal, it is necessary to use a finite representation for the infinite state space of $\tau$(A). Developing such representations is the main topic of te following sections.

\paragraph{clock regions}
In the definition	 of timed automata, we allowed the clock constraints that serve as the invariants of locations and the guards of transitions to contain arbitrary rational constants.
We can multiply the constants in each clock constraint by the least common multiple m of the denominators of all the constants to integers. The value of a clock can still be an arbitrary nonnegative real number. Note that applying this transformation can change the clock assignments in the set of reachable states of T(A). Fortunately, this does not cause a mjor problem. Ther reachale states of the original auomation can be obtained from the locations of te transformed automation by applying the inverse transformation, that is, dividing each clock value by m.


Th largest constant in the tranformed in the transformed automation is the product of m and the largest constant in the original automation. Thus, the transformation at worst results in quadratic blowup in the length of the encodings of th lock constraints[3]. This increase in complexity is acceptable, since the transformation simplifies certain operations on clock constraints that will be needed later in the chapter. We will apply this tranformation uniformly to all of th clock constraints that appear in the timed automata the we study. Consequently, in the future we can assume without loss of generality that all constants in clock constraints that we encounter are integers.

In order to obtain a finite representation for the infinite state space of a timed automation, we define clock regions[7,8], which represents sets of clock assignments. If two states, which correspond to the  same location of the timed automation A, agree on the integral parts of all clock constraint in the invariant of a location or in the guard of a transition is satisfied or not. The ordering of the fractional parts of the clock values determines which clock will change its integral part first. This is because clock constraints cn involve only integers, and all clocks increase at the same rate.

For example, let A be a timed automation with two clocks x1 and x2. Let s be a location in A with an outgoing transition e to some other location. Consider two states (s,v) and (s,v') in T(A) that correspond to location s. Suppose that v(x1) = 5.3, v(x2)=7.5, v'(x1)=5.5 and v'(x2) = 7.9. Assume that the guard $\varphi$ associated with e is x1 $\geq$ 8 $\wedge$ x2 $\geq$ 10. It is easy to see that if (s,v) eventually satisfies the guard, then so will (s, v'). \\
\newline

The value of a clock cn get arbitrarily large; however, if the clock is never compared to a constant greater than c, then the value of the clock will have no effect on the computation of A once it exceeds c. Suppose, for instance, that the block x is never compared to a constant greater than 100 in the invariant associated wit a location or in the guard of a transition.

Then, based on the behaviour of A, it is impossible to distinguish between x having the value 101 and x having the value 1001.
Alur,Courcoubetis, and Dill[7,8] show how to formalize this reasoning. For each clock x $\in$ X, let cx, be the largest constant that x is compared with in the invariant of any location or in the guard of any transition. For t $\in$  $\Re^{+}$, let ft(t) be the fractional part of t, and let [t] be the integral part of t. Thus, t = [t] + fr(t). We define an equivalence relation $\cong$ on the set of possible clock assignments as follows: Let v and v' be two clock assignments.
Then v  $\cong$ v' if and only if three conditions are  satisfied: \\
\newline

For all x $\in$ X either v(x) $\geq$ cx, and v'(x) $\geq$ gx or [v(x)] =[v'(x)].
For all x,y $\in$ X such that v(x) $\leq$ cx and v(y) $\leq$ cy, fr(v(x)) $\leq$ fr(v(y)) if and only if fr(v'(x)) 
$\leq$ fr(v'(y))
For all x $\in$ X either v(x) $\leq$ cx,
fr(v(x)) =0 if and only if fr(v'(x)) =0.
it is easy to see that $\cong$ does indeed define anequivalence relation. The equivalence classes of $\cong$ are called regions[7,8].  We will write [v] to denote the region which containsthe clock assignment v.Each region can be represented by specifying \\
\newline

1. for every clock x $\in$ X, once clock constraint from the set {x=c | c=0...., cx} $\cup${c -1 < x < c | c=1,.....cx} $\cup$ {x > cx}
2. for every pairof clocks x, y $\in$ X such that c-1 < x< cand d-1<y< d are clock constraints in the first condition, whether fr(x) is less than, equal to, or greater than fr(y).

Figure 17.7 which is taken from [8], shows the clockregions for a timed automationwith two clocks x and y where cx = 2 and cy =1. In this example, there are a total of 28 regions: 6 corner points, 14 open line segments and 8 open regions.

We will use this observation to show that $\cong$ has finite index and, consequently, that the  number of regions is finite. Our proofof this fact is based on the proof given in [8].





Lemma 43
The number of equivalence classes that $\cong$ induces on C(X) is bounded by
|X|! $\cdot$ $2^{|X|}$ $\cdot$ $\prod$ (2xc +2)
proof
An equivalence class [v] of $\cong$ can be described by a tripple	 of arrays i the following manner. For each block x $\in$ X, the array $\alpha$ tells which of the intervals {[],[]}
contains the value v(x). Thus, the array $\alpha$ represents the cloc assignment v if and only if for each clock x $\in$ X, v(x)$\in$ $\alpha$(x). The number of ways to choose $\alpha$ is $\prod$. \\ 
\newline


Let $X_a$ be th set of clocks with nonzero fractional part. The array $\beta$: $x_a$ $\rightarrow$ {1,....|$A_a$|} is a permutation of $X_a$, which gives the ordering of the fractional parts of the clocks in Xa with respect to $\leq$. Thusm the array $\bta$ represents a clock assignment c if and if for each pair x,y $\in$ $X_a$, if $\beta$(x) < $\beta$(y) then fr(v(x)) $\leq$ fr(v(y)). For a given $\alpha$, the number of ways to choose $\beta$ is bounded by |$X_\alpha$|!  which is bounded by |X|!. \\
\newline

The third component  $\gamma$ is a boolean array indeed by $X_a$ that is used to specify which clocks in $X_a$ have the same fractional part. For each clock c, $\gamma$(x) tells whether or not the fractional part of v(x) equals  the fractional part of its predecessor in the array $\beta$. Thus the array $\gamma$ represents a clock assignment v if and only if for each x \n X, $\gamma$(x) equals 0 exactly when there is a clock $\gamma$ $\in$ $X_alpha$ such that $\beta$(y) = $\beta$(x) +1 and fr(v(x)) equals fr(v(y)). The number of ways of choosing $\gamma$ is bounded by the number of boolean arrays over $X_\alpha$, which is bounded by $2^{|X|}$.
Hence, $\alpha$ encoded the integral parts of he clock assignments, and $\beta$ with $\gamma$ encodes the ordering of their fractioal aprts. It is easy to see that the sets represented bytriples are equivalence of $\cong$ and that every equivalence class is represented by some triple. The bound given in the statement of the lemma is the product of the bounds associated with $\alpha$, $\beta$, and $\gamma$. This completes the proof of the lemma. \\
\newline


The following properties of the equivalence relation $\cong$ are used in later  in this chapter.
Lemma 44
Let v1 and v2 be twoclock assignments1, let $\varphi$ be a clock constraint, and let $\lambda$ $\subseteq$ X be a set of clocks.
1. if v1 $\cong$ v2 and t is a nonnegative integer, then v1 + t $\cong$ v2 +t.
2. if v1 $\cong$ v2, then $\forall$ t1 $\in$ $R^{|+|}$ $\existst_2$ $\in$ $R^{|+|}$[v1 +t1 $\cong$ v2 + t2]
3. if v1 $\cong$ v2, then v1 satisfies $\varphi$ if and only if v2 satisfies $\varphi$
4. If v1 $\cong$ v2, then v1[$\lambda$:=0] $\cong$ v2 [$\lambda$:=0]
\\
\newline

Note that the first property may not hold if t is not an integer. For example, (2.8) $\cong$ (.1, .2),
but (.2, .8) +.3 is not equivalent to (.1,.2) + .3. All of the properties except the second are straightforward to prove and will be left to the reader. A proof if the scond property is sketched below. The proof is not diificultm but it is somewhat tedious. It can be safely skipped when this chapter is read for the first time.

\\
\newline

Proof
Assume that v1 $\cong$ v2. We can assume that t1 > 0 because, otherwise, we can simply choose t2=0. Let X{x1,x2,.....,xn}. We can threat v1 as a vector v1 = $\langle$ a1, ......, an $\rangle$, where $a_i$ is the alue of clock $x_i$ in $v_1$. Similarly, we let v2 = $\langle$ b1, ......, $b_n$ $\rangle$. Since corresponding clocks have the same integer part, we can assume without loss of generality that 0 $\leq$ $a_i$ < 1 and 0 $\leq$ $b_i$ < 1. Also , assume that the clock values are sorrted into increasing order so that $a_1$ $\leq$ $a_2$ $\leq$ ... $\leq$ $a_n$ and b1 $\leq$ $b_2$ $\leq$ .... $\leq$ $b_n$.


case 1
Assume that the largest element in v1 + t1 is less than or equal to 1. This case is trivial. We can easilty choose t2 so that v + t1 $\cong$ v2 + t2

case 2
Assume that  0 $\leq$ t1 < 1. Let the first element of v1 +t1. That is greater than or equal to 1 be $a_k$+t1. Chhoose $\in$ so that $\in$ =0 if $a_k+t_1$ =1 and so that 0< \ni < $b_k$-$b_k-1$ if $a_k$+$t_1$ > 1. Note that $b_k_$-1 < $b_k$ = $b_k$-1, then $a_k$=$a_k$-1 and $a_$+$t_1$ is not the first elment of v1+t1 that is greater than o equal to 1. We will show that v1+t1 $\cong$ v2+(1+ $\in$ - $b_k$). In order to show this we will split the vectors into two parts. Let

L1= $\langle$ $a_1$ + t1, ...., $a_k$-1 + t1    $\rangle$, and
L2= $\langle$  $b_1$ + (1 + $\in$ - $b_k$), ..., $b_k$-1 + (1+ $\in$ - $b_k$)  $\rangle$
In each case it is straightforward to show that

1. all of the elements are positive
2. the elements are sorted in increaing order, and
3. all of the elements are less than 1
Because of these conditios it is easy to see that $L_1$ $\cong$ $L_2$. Similarly, let 

$R_1$= $\langle$ $a_k$ + t1, ...., $a_n$ + t1    $\rangle$, and
$R_2$= $\langle$  $b_k$ + (1 + $\in$ - $b_k$), ..., $b_k$-1 + (1+ $\in$ - $b_k$)  $\rangle$

All of the elements in $R_1$ and $R_2$ are greater than or equal to 1. The fractional parts are given by $R_1$ - 1 and $R_2$ -1, respectively. For these vectors it is straightforward to show that

1. all of the elements are nonnegative
2. the elements are sorted in increasing order, and
3. all of the elements are less than 1

Moreover, an element in one vector is 0 if and only if the corresponding element in the order vector is 0. Thus $R_1$ -1 $\cong$ $R_2$ -1. It follows immediately that $R_1$ $\cong$ $R_2$.
It is not difficult to see that the fractioal parts of $R_2$ precede the fractional parts of $L_2$.
Let i  $\geq$ k and j < k. Then
$b_i$ +  (1 + $\in$ - $b_k$) -1 $\leq$ $b_j$ + (1+ $\in$ - $b_k$).
is equivlent to $b_i$ - $b_j$ $\leq$ 1, which is obviously true. The same relationship holds for the fractioal parts of $R_1$ and $L_1$, that is.
$a_i$ + $t_1$ -1 $\leq$ $a_j$ + $t_1$.

hence , we obtain $R_1$ $\cdot$ $L_1$ $\cong$ $R_2$ $\cdot$ $L_2$, where "$\cdot$" is concatenation of vectors. This shows that for all $t_1$ with 0 $\leq$ t1 < 1, there exits a t2 such that $v_1$ + $t_1$ $\cong$ v2 + $t_2$ and completes the proof of 



case 3
Finally, supppose that $t_1$ $\geq$ 1. Let t1'= $t_1$ - [$t_1$], so that 0 $\leq$ $t_1$ < 1. Find $t_2$ such that $v_1$ + $t_1$ $\cong$ v2 + t2. Then:
v1 + t1 + [t1]  $\cong$ v2 + t2 + [t1].

If we choose t2 = [t1], then we have v1 + t1 $\cong$ v2 + t2 as required. This completes the proof of the second property.

The equivalence relation $\cong$ over clock assignments an be extended to an equvalence relation over the state space of T(A) by requiring that equivalent states have identical locations and equivalent clock assignments: (s,v) $\cong$ (s', v') if and only if s = s' and v $\cong$ v'. The key property of he equivalence reltion $\cong$ is given by the following lemma [5]:


Lemma 45
If v1 $\cong$ v2 and (s, $v_1$)  $\xrightarrow[]{a}$ (s', v'). The transition $\langle$ s, a, $\varphi$, $\lambda$, s' $\rangle$  that takes state (s, v1) to state (s', v1') corresponds to two transiions of the timed automation.

Proof
Assume that v1 $\cong$ v2 and (s, v1)  $\xrightarrow[]{a}$ (s', v'1). The transition $\langle$ s, a, $\varphi$, $\lambda$, s' $\rangle$ that takes state (s, v1) to state (s', v'1) corresponds to two transitions of the timed automation:

a delay transition (s, v1)  $\xrightarrow[]{d1}$ (s, $v_1$ + $d_1$) for some $d_1$ $\geq$ 0, and
an action transition   (s, v1 +d1)  $\xrightarrow[]{a}$ (s', $v_1$') such that $v_1$ + $d_1$ satisfies $\varphi$ and $v'_1$ = ($v_1$ + $d_1$)[$\lambda$ :=0].



Since v1 $\cong$ v2 and v1 satisfies I(s), v2 also satisfies I(s). Furthermore, there exists d2 $\geq$ 0 such that v1 + d1 $\cong$ v2 + d2. Since v1 + d1 satisfies I(s), v2 +d2 also satisfies I(s). Because the clock constraint I(s) is convex and is satisfied by both v2 and v2 + d2, I(s) must be satisfied by v2 + e for all e such that 0 $\leq$ e $\leq$ d2. Consequently, the delay transition (s, v2)  $\xrightarrow[]{d2}$ (s, v2 +d2) is legal.

Since v1 + d1 $\cong$ v2 +d2, both v1+ d1 and v2+ d2 must satisfy the clock constraint for the guard $\varphi$. Thus, the transition $\langle$ s, a, $\varphi$, $\lambda$ , s' $\rangle$ myst also be enabled in the state  *s, $v_2$ + $d_2$
. Let $v'_2$ = ($v_2$ +$d_2$)[$\lambda$ :=0]. Then v'$_2$ is equivalent to $v'_1$. Hence, there is an action transition (s, $v_2$ + $d_2$)  $\xrightarrow[]{a}$  (s', $v'_2$). Combining the delay transition with the action transition, we get (s, $v_2$)  $\xrightarrow[]{a}$ (s', $v'_2$) as required.

As a result of the lemme, we can conostruct a finite state transition raph that is  bisimilaion equivalnt to the infinite state transition graph T(A). The finite state transition graph is called the region graph of A[7,8] and is denoted by R(A). A region is a pair (s, [v]). Since $\cong$  has a finite index, there are only a finite nuber of regions. The states of the region graph are  the regions of A. The construction of R(A) will have the property that whenever (s,v) is a state of T(A), the region (s, [v]) where $s_0$ is an initial state of A and $v_0$ is a clock assignment that assigns 0 to every clock. The transition relation of R(A) is defined so that bisimulation equivalence is guaranteed. There will be a transition labeled with a from the region (s,[v]) to the region (s', [v']) if and only there are assignments $\omega$ $\in$ [v] and $\omega$' $\in$ [v'] such that (s, $\omega$) can make a transition to (s', $\omega$')

We summarize the construction of the region graph R(A) below. Let A = ($\sigma$, S, $S_0$, X, I, T) be a timed automation. Then,
The states of R(A) have the form (s, [v]) where s $\in$ S and [v] i a clock region
The initial states have the form ($s_0$, [v]) where $s_0$ $\in$ $s_0$ and v(x)=0 for all x $\in$ X.
R(A) has a transition ((s,[v]),a, (s',[v'])) if and only if (s, $\omega$)  $\xrightarrow[]{a}$  (s', $\omega$') for some $\omega$ $\in$ [v] and some $\omega$' $\in$ [v'].
We can use Lemma 45 to prove bisimulation equivalene.

Theorem 31
We will show that T(A) and R(A) are bisimilar. Define the bisimulation relation B by (s,v)B(s,[v]). It is easy to see that the initial state ($s_0$, $v_0$) corresponds o the  state ($s_0$, [$v_0$]). Next, we show that for each transtition of T(A), there is a corrresponding transition  of R(A), and vice versa. Suppose first that (s,v)B(s,[v]). Suppose on the other hand that (s,v)B(s,[v]) and that  (s,v)$\xrightarrow[]{a}$(s',[v']). Then there exit $\omega$ $\cong$ v and $\omega$' $\cong$ v' such that (s', v") and (s,v) $\xrightarrow[]{a}$(s', v"). Hence v" $\cong$ $\omega$ $\cong$ v', so [v"] = [v']. By the definition of B, (s', v")
B (s', [v"]), it follows that (s', v")B(s', [v']).


\paragraph{Safety}
Safety Properties are used to verify that something
bad will never happen. Dit kan worden gespecificeerd met de volgende vergelijking

\aqcap\\

\square ( a_0 \implies (( \lnot a_2 \wedge \lnot a_3 ) \mathcal{U} a_1 ) \vee ( \lnot a_2 \wedge \lnot a_3 )) \\

AG(p)
M, s \models AG(p) $\Leftrightarrow$     \forall \pi \in  \sqcap (M,s) \cdot \forall i \cdot M,\pi[i] \models p\\

EG(p)
M, s \models EG(p) $\Leftrightarrow$     \exists \pi \in  \sqcap (M,s) \cdot \forall i \cdot M,\pi[i] \models p\\

AF(p)

EF(p)

AX(p)

EX(p)

A(p \cup q)
M, s \models  A(p \cup q)   $\Leftrightarrow$     \forall \pi \in  \sqcap (M,s) \cdot \exists k \cdot M,\pi[k] \models q \wedge ( \forall i \leq k \cdot M,\pi [i] \models p)\\
E(p \cup q)

A(p \Re q)

E(p \Re q)

\forall x \, (P(x) \to Q(x)) & premise \\
\forall x \, P(x) & premise \\\hspace*{-30pt} \\


P(x_0) & $\forall x \, \mathrm{e}$ 2 \\
Q(x_0) & $\to \mathrm{e}$ 3, 4 \\

\forall x \, Q(x) & $\forall x \, \mathrm{i}$ 3--5 \\







\{a,b\} or \set†{a,b} \\
\langle a,b \rangle or \gens†{a,b} \\


f \colon A \to B \\

f \circ g \\
x \mapsto f(x) \\

\begin{align*}
	f \colon \mathbb{R} &\to \mathbb{R} \\
	x &\mapsto x^2
\end{align*}


\newline \\
M, s $\models$ p $\Leftrightarrow$ p $\in$ L(s) \\
M, s $\models$ $\not$ f1 $\Leftrightarrow$ M, s $\nvdash$ f1 \\
M, s $\models$ f1 $\vee$ f2 $\Leftrightarrow$ M,s $\models$ f1 or M,s $\nvdash$ f2 \\
M, s $\models$ f1 $\wedge$ f2 $\Leftrightarrow$  M,s $\models$ f1 and M,s $\nvdash$ f2 \\
M, s $\models$ $\mathrm{E}$ $g_{1}$ $\Leftrightarrow$ there is a path $\pi$  from ~  s ~   such ~  that  ~ M, $\pi$ $\models$ g1 \\
M, s $\models$ p $\Leftrightarrow$ for every path $\pi$  ~ starting from  ~  s, M, $\pi$ $\models$ g1 \\
M, s $\models$ p $\Leftrightarrow$ s is the first state of $\piand$ M, s $\models$ f1 \\
M, s $\models$ $\not$ $g_{1}$ $\Leftrightarrow$ M, $\pi$  $\nvdash$ g1\\
M, s $\models$ p $\Leftrightarrow$  M, $\pi$  $\models$ g1  or  M, $\pi$  M, $\pi$  $\models$ g2\\
M, s $\models$ p $\Leftrightarrow$ M, $\pi$  $\models$ g1  and  M, $\pi$  M, $\pi$  $\models$ g2 \\
M, s $\models$ p $\Leftrightarrow$ M, $\pi^{1}$ $\models$ g1 \\
M, s $\models$ p $\Leftrightarrow$ there exists a k $\ge$ 0, such that  ~ M, $\pi^{k}$  $\models$ g1\\
M, s $\models$ p $\Leftrightarrow$ for all i $\ge$ 0,M,$\pi^{i}$ $\models$ g1 \\
M, s $\models$ g1 $\bugcup$ g2 $\Leftrightarrow$ ~  there  ~ exists  ~ ak  ~ $\ge$  ~ 0 ~  such ~  that  ~ M,  ~ $\pi^{k}$ $\models$ g2\\
and  ~ for  ~ all ~  0  ~ $\le$ j < k, M,$\pi^{j}$ $\models$ g1
M, s $\models$ p $\Leftrightarrow$ for all j $\ge$ 0, if for ~  every  ~ i < j,M,$\pi^{i}$ $\nvdash$ g1 then M,$\pi^{j}$ $\models$ g2\\


\paragraph{Reachability}
Reachability properties are used to check whether
a given state formula can be satisfied by some
reachable state.

\paragraph{Liveliness}
Liveness properties are used to verify that
something eventually will hold
\paragraph{Security}

\paragraph{Performance}



About transition
A transition is composed of
a unique source location
a unique target location
a guard, i.e. an enabling condition (g := x ∼ c|g ∧ g, where
∼∈ {<, ≤, =, ≥, >}
a label (that can be used for synchronization)
a subset (potentially empty) of clocks to be reset

a clock valuation is a function v: X $\trightarrow$ $R^+$
v[Y:=0] is the valuation obtained from v by resetting clocks from Y: 

\begin{math}
	$v[Y:=0]$=\left\{
	\begin{array}{ll}
		1, & \mbox{0 x $\in$ Y}.\\
		0, & \mbox{otherwise}.
	\end{array}
	\right.
\end{math}


v+d = flow of time (d units)
(v +d)(x) = v(x)+d
v $\implies$ c meansthat valuationv satisfies the constraint c

evaluation of a clock constraint (v $\implies$ g)
v $\implies$ g x  < k iff ν(x) < k
ν |= x ≤ k iff ν(x) ≤ k
ν |= g1 ∧ g2 iff ν |= g1 and ν |= g2

(s', v") and (s,v) $\xrightarrow[]{a}$(s', v").

Action transitions correspond to the execution of a transition	 from T. We write (s,v) $\xrightarrow[]{a}$ (s', v'), where a \in $\Sigma$, provided that there is a transition $\langle$ s,a, $\varphi$, $\lambda$, s' $\rangle$ such that v satisfies $\varphi$ and v=[$\lambda$:=0].

a delay transition (s, v1)  $\xrightarrow[]{$$\delta$(d)}$ (s, v_1 + d_1) for some $d_1$ $\geq$ 0, and
an action transition   (s, v1 +d1)  $\xrightarrow[]{a}$ (s', v_1') such that $v_1$ + $d_1$ satisfies $\varphi$ and v'_1 = (v_1 + d_1)[$\lambda $:=0].


%%%%%%%%%%%%%%%%%%%%%%%%%%%%%%%%%%%%%%%%%%%%%%%%%%%%%%%%%%%%%%%%%
We think of thevariables innV as the present sate variables and the variables in V'as next state variables. Each variable v i V has a corresponding next state variable in V', which we denote y v'. A valuation for the variables in V and V' can be vieuwed as designating an ordered pair of states or a transition, and we can represent setsof these valuations using formulas as above. We refer to a set of pairs of states as a transition relation. If R is a transition relation, then we write R(V,V') to denote a formula that represents it.
In order to write specifications that describe properties of concurrent systems we need to define a set of atomic propositions AP. Atomic propositionswill typically have the form v=d where v $\in$ V and d $\in$ D. A proposition v =d will be truein a state s if s(v)=d. Whenv is a variable over the  boolean domain{True,False}, it is not necessarly to include both v = True and  v = False in AP.We will write v to indicate that s(v)=True and $\neq$ v to indicate that s(v)=False.
We now show how to derive

blz 16

We now show how to derive Kripke M=(S,$S_0$,R,L) from the first order formulas $S_0$ and R that represent the concurrent system.
The set of states is hthe set of all variations	for V
the set of initial states $S_0$ is the set of all valuations $s_0$ for V that satisfy the formula $S_0$
let s and s' be the two states, then R(s,s') holds if R evaluates to True when each v $\in$ V is assigned the value s(v) and each v' $\in$  V' is assigned the value s'(v).
The labeling function L: S $\to$  $\2^{AP}$ is defined so that L(s) is the subsetof all atomic propositions true in s. If v is a variable over the boolean domain, then v $\in$ L(s) indicates that s(v)=True, and v $\notin$  L(s) indicates that s(v)=False.
L: S $\t$o  $\2^{AP}$  is a function that labels each state with the set of atomic propositions true in that state\\

Because we require that the transition relation of a kripke structuer us always total, we must extend the relation R if some state s has no successor. In this case, we modify R so that R(s,s) holds.
To illustrate the notions defined in this section we consider a simple system with variables x and y that range over D={0,1}. Thus, a valuation for the variables x and y is justa pair ($d_1$, $d_2$) $\in$ D x Dwhre $d_1$ is the value for x and $d_2$ is the value for y.

blz 33
Fairness
A fairness constraint an be an arbitraty set of states, usually described by the formula of the logic. if fairness constraints are interpreted as sets of states, then a fair path must contain an element of each fairness constraint infinetely often. If fairness constrants are interpreted	 as CTL formula, then a path is fair if each constraint is true infnetely often along the path. The path quantifiers in the logic are then restricted fair paths.
Formally, a fairkripke structure is a 4-tuple M = (S,R,L,F), where S, L and R are  defined as before and F $\subseteq$  $\2^{S}$  is a set of fairness constraints ( often called Buchi acceptance conditions) Let $\pi$ = $s_0,s_1$ be a path in M. Define 
inf($\pi$) = {s| s=$s_i$ for infinitely many i}.

We say that $\pi$ is fair if and only if for every P $\in$ F, inf(\p) $\cap$ P $\neq$ $\emptyset$. The semantics of CTL* wth respect to a fair kripke structure is very similar to the semantics of CTL* with respect to ordinary kripke structure. We will write M,s $\models_F$ f to indicate that the state formula f is true in state s of the fair Kripke structure M. Similarly, we write M, $\pi$ $\models$ $_f$ g to indicate that the path formula g is true along path $\pi$  in M. Only clauses 1, 5 and 6 in the origial semanticss change.
1. M, s $\models$  $_f$ p  $\Leftrightarrow$ there exists a fair path from s and p $\in$ L(s)
5. M, s $\models$  $_f$ p  $\Leftrightarrow$ there exists a fair path $\pi$ starting from s such that $\pi$ $\models$ $_f$ g1
6. M, s $\models$  $_f$ p  $\Leftrightarrow$ for all fair paths $\pi$ starting from s, $\pi$ $\models$ $_f$ g1

To illustrate the use of fairness, conider again the communication protocol for reliable channels. There is one fairness constraint for each channelthat expresses the reliability of that channel. A possible choice for the fairness constraint associated with channel i is the set of states that satisfy the formula $\neq$ send $\vee$  $receive_i$. Thus, a computation path is fair if and only if for every channel, infinitly often either a message is received. Other notions of fairness are dealt with in[116].
blz 36
ctl model checking

The model checkingproblem is easy to describe. given a kripke structure M =(S,R,L) that represents a finite-state concurrent system and a termporal logic formula f expressing some desired specification, find the set of all states n S that satisfy f: {s $\in$ S | M, s $\models$ f} \\
Let M = (S,R, L) be a kripke structur. Assume that we want to determine which states in S satisfy the CTL formula f. The algorithm will operate by labeling each state s with the set label(s) of subformulas of f which are true in s. Initially, label(s) is just L(s). The algorithm then goes through a series of stages.During the ith stage, subformulas with i-1 nested CTL operators are processed. When a subformula is processed, it is added to the labelig of each state in which it is true. Onze the algorithm terminates, we will have that M, s $models$ f iff f $\in$ label(s)
blz 40
Fairness constraints
In this subsection we show how to extendthe CTL model checking algorithmto handle fairness constraints. Let M = (S,R,L,F) be a fair kripke structure. Let F = {P1, ..., $P_k$} be the setof fairness constraints. We will say that a strongly connnected component C of the graph of M is fair wth respect to F if and only if for each $P_i$ $\in$ F, there is a state $t_i$ $\in$ (C $\cap$  $P_i$). We first give an algorithmfor checking EG $f_1$ with respect to a fair structure. In order to establish the correctness of this algorith, we need a lemma that is analogous to Lemma 1.As before, let M' be obtained from M by deleting from S all of those states at which $f_1$ does not fairly hold. Thus, M'=(S',R',L',F') where $\S^{'}$ = {s $\in$ S | M,s $\models$ F f1}, R' = R|$_S$'xS', L' = L|$_s$;, and F' ={ $P_i$ $\cap$ S' | $P_i$ $\in$ F}.

Lemma 2 M,s $\models$ F EG f1 iff the followingtow conditions are satisfied:
1. s $\in$ S'
2. There exists a path S' that leads from s to somenode t in a nontrivial fair strongly connected component of thr graph (S',R')

In order to determine if M, s $\models$ f p for some p $\in$ AP, we check M,s $\models$ p $\wedge$ fair using the ordinary model-checking procedure.
blz 68
Fairness in model checking with fixpoint

blz 69

blz 70

blz 71
Counterexamples and whitnesses

blz 72


blz 73


blz 74



blz 121
automata theory
blz 141



blz 171
Equivalence and preorders between systems
blz 172

blz 173


blz 174


blz 175


blz 176


blz 177
simulation relations
blz 178

blz 179

blz 180


blz 232
INvariants
blz 233


blz 234


blz 265

blz 266

blz 267


blz 268 parralel compositioon
Before we consider a reachability problem, we show how real-time systems can be modoeled as parralel compositions of timed automata [3,5]. We assume an interleavingor asynchroneous semantics for this operation. Let A1 = ($\sum$, S1, $\S^{1}_0$, $X_1$, $I_1$, $T_1$) and $A_2$ = ($\sum_2$, $S1_2$, $\S^{1}_0$, $X_2$, $I_2$, $T_2$) be two timed automata. Assume that the two automata have disjoint sets of clocks, that is $X_1$ $\cap$ $X_2$ = $\emptyset$. Then, the parralel composition of $A_1$, and $A_2$ is the timed automation:

$A_1$ || A2 = ($\Sigma$ $\cup$ $\Sigma_2$, $S_1$ x S2, $\S^{1}_0$ x  $\S^{2}_0$ , $X_1$ $\cup$ $X_2$, I, T),
where I($s_1$,$s_2$)=$I_1$($s_1$) $\wedge$ $I_2$($s_2$) and the edge relation T is given by the following rules:

1 For a $\in$ $\Sigma_1$ $\cap$ $\Sigma_2$, if $\langle$ s1,a, $\varphi$, $\lambda_1$, $s_1$' $\rangle$ $\in$ $T_1$ and $\langle$ s2,a, $\varphi$, $\lambda_2$, $s_2$' $\rangle$  $T_2$ then T will contain the transition $\langle$ (s1,s2), a $\varphi$ , $\lambda_1$ $\cup$ $\lambda_2$, ($s_1$',$s_2$') $\rangle$
2. For a $\in$ $\Sigma_1$ - $\Sigma_2$, if $\langle$ s, a, $\varphi$, $\lambda$, s' $\in$ $T_1$ and t $\in$ $S_2$ then T will contain the transition $\langle$ (s,t),a, $\varphi$, $\lambda$, (s', t) $\rangle$
3. For a $\in$ $\Sigma_2$ - $\Sigma_1$, if $\langle$ s, a, $\varphi$, $\lambda$, s' $\in$ $T_2$ and t $\in$ $S_1$ then T will contain the transition $\langle$ (t,s),a, $\varphi$, $\lambda$, (t,s') $\rangle$

Thus the locations of the parralel composition are pairs of locations from the component automata, and the invariant of such a location is the conjunction of the invariants of the component locations. There will be a transition in the parralel compoition for ach pair of transitions from the individual timed automata with the same action The source location of the transition will be the composite location obtained from the source locations of the individual transitions. Te target location will be the compositelocation obtained from the target locations of the individual transitions. The guard will be the conjunction of the guards for the individual transitions, and the set of clocks that are reset will be the union of sets that are reset by the individual transitions. If the action of  a transition is only an action of one of the two processes, then there will be a transition in the parralel composition for each location of the othertimed automation. The source and target locations of the original transition and the location fromthe other automation. All of the other components of the transition will remain the same.


blz 269 modelling with timed automata

blz 274 clock regions

blz 280 clock zones

blz 281
\paragraph{Timed automata}
Timed automata
A timed automation[8,99] is a finite augmented with a finite set of  real-valued clocks. We assume that transitions are instantaneous. However, time can elapse when the automation is in a state or location. When a transition occurs, some of the clocks ma be reset to zero. At any instant, the reading clock is equal to the time that has elapsed since the lat time the clock was reset. We assume that time passes at the same rate for all clocks. In order to prevent pathological behaviours, we only consider automata that are non-zeno, that is, only a finite number of transitions can happen within a finite amout of time.

A clock constraint, called a guard, is associated with each transition. The transition can be taken only if the current values of the clocks satisfy the clock constraint. A clock cnstraint is also associated with each location of the automation. This constraint i called the invariant of the location. Time can elapse in the location only as long as the invariant of the location is true. An example of a timed automation is shown in Figure 17.1 The automation consists of two locations s0 and s1, two clocks x and y, and "a" transition from s0 to s1, and a "b" transition from s1 to s0. The automation starts in location s0. It can remain in that location as long as the clock y is less than or equal to 5. As soon as the value of y is greater than or equal to 3, the  automation can make an "a" transition to location s1 and reset the clock y to 0. the automation can remain in location s1 as long as y is less than or equal to 10 and x is less than or equal to 8. When y is at least 4 and x is at least 6, it can make a "b" transition back to location s0 and reset x.

The remainder of this section contains a formal semantics for timed automata in terms of infinite state transition graphs[3,8]. We begin with a precise definition of clock constraints. Let X be a set of clock variables, ranging over the nonneative real numbers $\Re^{+}$. Define the set of clock constraints C(X) as follows:
All inequalities of the form x $\prec$ c or c $\prec$ x are in C(X) where $\prec$ is either < or  $\leq$ $\sm$ and c is a nonnegative rational number.
If $\varphi_1$ ad $\varphi_{1}$ are in C(X), then $\varphi_1$ $\wedge$ $\varphi$ is in C(X).

Note that if X contains k clocks; then each clock constraints is a convex subset of k-dimensional Eucledian space.Thus, if two points satisfy a clock constraint, then all of the points	on the line sement connecting these points satisfy the clock constraint.
A timed automation is a 6-tuple A = ($\Sigma$, S, $S_0$, X, I, T) such that
$\Sigma$ is a finite alphabet
S is a finite set of locations
S0 $\subseteq$ S is a set of starting locations
X is a set of clocks
I : S $\rightarrow$ C(X) is a mapping from locations to clock constraints called the location invariant.
T $\subseteq$ S x $\Sigma$ x C(X) x $2^{x}$ x S is a set of transitions. The 5-tuple $\langle$ s,a,$\varphi$, $\lambda$, s' $\rangle$ corresponds to a transition from location s to location s' labeled with a, a constraint $\varphi$ that specifies when the transition is enabled, and a set of clocks $\lambda$ $\subseteq$ X that  are reset when the transition is executed.


We will require that time be allowed to progress to infinity, that is, at each location the upper bound imposed on the clocks be either infinity, or smaller than the maximum bound imposed by the invariant and by the transitions outgoing from the location. In other words, it is possible either to stay at a location forever, or the invariant will force the automation to leave the location, and at that point at least one transition will be enabled. For timed automata, these constraints can be imposed syntactically.

A model for a timed automation A is an infinite state transition graph $\tau$(A) = ($\Sigma$, Q, $Q^{0}$, R). Each state in Q is a pair (s, v) where s $\in$ S is a location and v : X $\rightarrow$  $R^{+}$ is a clock assignement, mapping each clock to a nonnegative real value. The set of initial states $Q_0$ is given by {(s,v)| s $\in$ $S_0$ $\wedge$ $\forall$ x  $\in$  X[v(x) =0]}.
In order to define the state transtion relation for $\tau$(A), we musr first introduce some notation. For $\lambda$ $\subseteq$ X, define v[$\lambda$ := 0] to be the clock assignment that is the same as v for clocks in X - $\lambda$ and maps the clocks in $\lambda$ to 0. For d $\in$ $\Re$, define v +d as the clock assignment that maps each clock x $\in$ X to v(x) +d. The clock assignment v -> d is defined in the same  manner. \\
From the brief discussion in the introduction, we know that a timed automation has two basic types of transitions:
Delay transitions correspond to the elapsing of time while staying at some location. We write (s, v) $\xrightarrow[]{d}$ (s, v+d), where d $\in$  $R^{+}$, provided that for every 0 $\leq$ e $\leq$ d, the invariant	l(s) holds for v +e. \\
Action transitions correspond to the execution of a transition	 from T. We write (s,v) $\xrightarrow[]{a}$ (s', v'), where a $\in$ $\Sigma$, provided that there is a transition $\langle$ s,a, $\varphi$, $\lambda$, s' $\rangle$ such that v satisfies $\varphi$ and v=[$\lambda$:=0].

The transition relation R of $\tau$(A) is obtained by combining the delay and action transitions. We will write (s,v) R(s', v') or (s, v) $\xRightarrow[]{f(x)}$   (s', v') if there exists s" and v" such that (s,v) $\xrightarrow[]{d}$ (s", v")$\xrightarrow[]{a}$ (s', v') for some d $\in$ $\Re$.
In this chapter we will describe an algorithm for solving the reachability problem for $\tau$(A): Given a set of initial states $Q_n$, we show how to copute the set of all states q $\in$ Q that are reachable from $Q_0$ by transitions in R. This problem is nontrivial because $\tau$ (A) has an infinite number of states. In order to accomplish this goal, it is necessary to use a finite representation for the infinite state space of $\tau$(A). Developing such representations is the main topic of te following sections.

blz 268 parralel composition

blz 274 clock regions
\paragraph{clock regions}
In the definition	 of timed automata, we allowed the clock constraints that serve as the invariants of locations and the guards of transitions to contain arbitrary rational constants.
We can multiply the constants in each clock constraint by the least common multiple m of the denominators of all the constants to integers. The value of a clock can still be an arbitrary nonnegative real number. Note that applying this transformation can change the clock assignments in the set of reachable states of T(A). Fortunately, this does not cause a mjor problem. Ther reachale states of the original auomation can be obtained from the locations of te transformed automation by applying the inverse transformation, that is, dividing each clock value by m.


Th largest constant in the tranformed in the transformed automation is the product of m and the largest constant in the original automation. Thus, the transformation at worst results in quadratic blowup in the length of the encodings of th lock constraints[3]. This increase in complexity is acceptable, since the transformation simplifies certain operations on clock constraints that will be needed later in the chapter. We will apply this tranformation uniformly to all of th clock constraints that appear in the timed automata the we study. Consequently, in the future we can assume without loss of generality that all constants in clock constraints that we encounter are integers.

In order to obtain a finite representation for the infinite state space of a timed automation, we define clock regions[7,8], which represents sets of clock assignments. If two states, which correspond to the  same location of the timed automation A, agree on the integral parts of all clock constraint in the invariant of a location or in the guard of a transition is satisfied or not. The ordering of the fractional parts of the clock values determines which clock will change its integral part first. This is because clock constraints cn involve only integers, and all clocks increase at the same rate.

For example, let A be a timed automation with two clocks x1 and x2. Let s be a location in A with an outgoing transition e to some other location. Consider two states (s,v) and (s,v') in T(A) that correspond to location s. Suppose that v(x1) = 5.3, v(x2)=7.5, v'(x1)=5.5 and v'(x2) = 7.9. Assume that the guard $\varphi$ associated with e is x1 $\geq$ 8 $\wedge$ x2 $\geq$ 10. It is easy to see that if (s,v) eventually satisfies the guard, then so will (s, v').

The value of a clock cn get arbitrarily large; however, if the clock is never compared to a constant greater than c, then the value of the clock will have no effect on the computation of A once it exceeds c. Suppose, for instance, that the block x is never compared to a constant greater than 100 in the invariant associated wit a location or in the guard of a transition.

Then, based on the behaviour of A, it is impossible to distinguish between x having the value 101 and x having the value 1001.
Alur,Courcoubetis, and Dill[7,8] show how to formalize this reasoning. For each clock x $\in$ X, let cx, be the largest constant that x is compared with in the invariant of any location or in the guard of any transition. For t $\in$  $\Re^{+}$, let ft(t) be the fractional part of t, and let [t] be the integral part of t. Thus, t = [t] + fr(t). We define an equivalence relation $\cong$ on the set of possible clock assignments as follows: Let v and v' be two clock assignments.
Then v  $\cong$ v' if and only if three conditions are  satisfied:

For all x $\in$ X either v(x) $\geq$ cx, and v'(x) $\geq$ gx or [v(x)] =[v'(x)].
For all x,y $\in$ X such that v(x) $\leq$ cx and v(y) $\leq$ cy, fr(v(x)) $\leq$ fr(v(y)) if and only if fr(v'(x)) 
$\leq$ fr(v'(y))
For all x $\in$ X either v(x) $\leq$ cx,
fr(v(x)) =0 if and only if fr(v'(x)) =0.
it is easy to see that $\cong$ does indeed define anequivalence relation. The equivalence classes of $\cong$ are called regions[7,8].  We will write [v] to denote the region which containsthe clock assignment v.Each region can be represented by specifying

1. for every clock x $\in$ X, once clock constraint from the set {x=c | c=0...., cx} $\cup${c -1 < x < c | c=1,.....cx} $\cup$ {x > cx}
2. for every pairof clocks x, y $\in$ X such that c-1 < x< cand d-1<y< d are clock constraints in the first condition, whether fr(x) is less than, equal to, or greater than fr(y).

Figure 17.7 which is taken from [8], shows the clockregions for a timed automationwith two clocks x and y where cx = 2 and cy =1. In this example, there are a total of 28 regions: 6 corner points, 14 open line segments and 8 open regions.

We will use this observation to show that $\cong$ has finite index and, consequently, that the  number of regions is finite. Our proofof this fact is based on the proof given in [8].





Lemma 43
The number of equivalence classes that $\cong$ induces on C(X) is bounded by
|X|! $\cdot$ $2^{|X|}$ $\cdot$ $\prod$ (2xc +2)
proof
An equivalence class [v] of $\cong$ can be described by a tripple	 of arrays i the following manner. For each block x $\in$ X, the array $\alpha$ tells which of the intervals {[],[]}
contains the value v(x). Thus, the array $\alpha$ represents the cloc assignment v if and only if for each clock x $\in$ X, v(x)$\in$ $\alpha$(x). The number of ways to choose $\alpha$ is $\prod$.


Let $X_a$ be th set of clocks with nonzero fractional part. The array $\beta$: $x_a$ $\rightarrow$ {1,....|$A_a$|} is a permutation of $X_a$, which gives the ordering of the fractional parts of the clocks in Xa with respect to $\leq$. Thusm the array $\beta$ represents a clock assignment c if and if for each pair x,y $\in$ $X_a$, if $\beta$(x) < $\beta$(y) then fr(v(x)) $\leq$ fr(v(y)). For a given $\alpha$, the number of ways to choose $\beta$ is bounded by |$X_\alpha$|!  which is bounded by |X|!. \\ \newline

The third component  $\gamma$ is a boolean array indeed by $X_a$ that is used to specify which clocks in $X_a$ have the same fractional part. For each clock c, $\gamma$(x) tells whether or not the fractional part of v(x) equals  the fractional part of its predecessor in the array $\beta$. Thus the array $\gamma$ represents a clock assignment v if and only if for each x \n X, $\gamma$(x) equals 0 exactly when there is a clock $\gamma$ $\in$ $X_alpha$ such that $\beta$(y) = $\beta$(x) +1 and fr(v(x)) equals fr(v(y)). The number of ways of choosing $\gamma$ is bounded by the number of boolean arrays over $X_\alpha$, which is bounded by $2^{|X|}$.
Hence, $\alpha$ encoded the integral parts of he clock assignments, and $\beta$ with $\gamma$ encodes the ordering of their fractioal aprts. It is easy to see that the sets represented bytriples are equivalence of $\cong$ and that every equivalence class is represented by some triple. The bound given in the statement of the lemma is the product of the bounds associated with $\alpha$, $\beta$, and $\gamma$. This completes the proof of the lemma. \\ \newline


The following properties of the equivalence relation $\cong$ are used in later  in this chapter.
Lemma 44
Let v1 and v2 be twoclock assignments1, let $\varphi$ be a clock constraint, and let $\lambda$ $\subseteq$ X be a set of clocks.
1. if v1 $\cong$ v2 and t is a nonnegative integer, then v1 + t $\cong$ v2 +t. \\ \newline
2. if v1 $\cong$ v2, then $\forall$ t1 $\in$ $R^{|+|}$ $\existst_2$ $\in$ $R^{|+|}$[v1 +t1 $\cong$ v2 + t2] \\ \newline
3. if v1 $\cong$ v2, then v1 satisfies $\varphi$ if and only if v2 satisfies $\varphi$ \\ \newline
4. If v1 $\cong$ v2, then v1[$\lambda$:=0] $\cong$ v2 [$\lambda$:=0] \\ \newline

Note that the first property may not hold if t is not an integer. For example, (2.8) $\cong$ (.1, .2),
but (.2, .8) +.3 is not equivalent to (.1,.2) + .3. All of the properties except the second are straightforward to prove and will be left to the reader. A proof if the scond property is sketched below. The proof is not diificultm but it is somewhat tedious. It can be safely skipped when this chapter is read for the first time.



Proof
Assume that v1 $\cong$ v2. We can assume that t1 > 0 because, otherwise, we can simply choose t2=0. Let X{x1,x2,.....,xn}. We can threat v1 as a vector v1 = $\langle$ a1, ......, an $\rangle$, where $a_i$ is the alue of clock $x_i$ in $v_1$. Similarly, we let v2 = $\langle$ b1, ......, $b_n$ $\rangle$. Since corresponding clocks have the same integer part, we can assume without loss of generality that 0 $\leq$ $a_i$ < 1 and 0 $\leq$ $b_i$ < 1. Also , assume that the clock values are sorrted into increasing order so that $a_1$ $\leq$ $a_2$ $\leq$ ... $\leq$ $a_n$ and b1 $\leq$ $b_2$ $\leq$ .... $\leq$ $b_n$.


case 1
Assume that the largest element in v1 + t1 is less than or equal to 1. This case is trivial. We can easilty choose t2 so that v + t1 $\cong$ v2 + t2

case 2
Assume that  0 $\leq$ t1 < 1. Let the first element of v1 +t1. That is greater than or equal to 1 be $a_k$+t1. Chhoose $\in$ so that $\in$ =0 if $a_k$+$t_1$ =1 and so that 0< \ni < $b_k$-$b_k$-1 if $a_k$+$t_1$ > 1. Note that $b_k_$-1 < $b_k$ = $b_k$-1, then $a_k$=$a_k$-1 and $a_$+$t_1$ is not the first elment of v1+t1 that is greater than o equal to 1. We will show that v1+t1 $\cong$ v2+(1+ $\in$ - $b_k$). In order to show this we will split the vectors into two parts. Let

L1= $\langle$ $a_1$ + t1, ...., $a_k$-1 + t1    $\rangle$, and
L2= $\langle$  $b_1$ + (1 + $\in$ - $b_k$), ..., $b_k$-1 + (1+ $\in$ - $b_k$)  $\rangle$
In each case it is straightforward to show that

1. all of the elements are positive
2. the elements are sorted in increaing order, and
3. all of the elements are less than 1
Because of these conditios it is easy to see that $L_1$ $\cong$ $L_2$. Similarly, let 

$R_1$= $\langle$ $a_k$ + t1, ...., $a_n$ + t1    $\rangle$, and
$R_2$= $\langle$  $b_k$ + (1 + $\in$ - $b_k$), ..., $b_k$-1 + (1+ $\in$ - $b_k$)  $\rangle$

All of the elements in $R_1$ and $R_2$ are greater than or equal to 1. The fractional parts are given by $R_1$ - 1 and $R_2$ -1, respectively. For these vectors it is straightforward to show that

1. all of the elements are nonnegative
2. the elements are sorted in increasing order, and
3. all of the elements are less than 1

Moreover, an element in one vector is 0 if and only if the corresponding element in the order vector is 0. Thus $R_1$ -1 $\cong$ $R_2$ -1. It follows immediately that $R_1$ $\cong$ $R_2$.
It is not difficult to see that the fractioal parts of $R_2$ precede the fractional parts of $L_2$.
Let i  $\geq$ k and j < k. Then
$b_i$ +  (1 + $\in$ - $b_k$) -1 $\leq$ $b_j$ + (1+ $\in$ - $b_k$).
is equivlent to $b_i$ - $b_j$ $\leq$ 1, which is obviously true. The same relationship holds for the fractioal parts of $R_1$ and $L_1$, that is.
$a_i$ + $t_1$ -1 $\leq$ $a_j$ + $t_1$.

hence , we obtain $R_1$ $\cdot$ $L_1$ $\cong$ $R_2$ $\cdot$ $L_2$, where "$\cdot$" is concatenation of vectors. This shows that for all $t_1$ with 0 $\leq$ t1 < 1, there exits a t2 such that $v_1$ + $vt_1$ $\cong$ v2 + $t_2$ and completes the proof of 



case 3
Finally, supppose that $t_1$ $\geq$ 1. Let t1'= $t_1$ - [$t_1$], so that 0 $\leq$ $t_1$ < 1. Find $t_2$ such that $v_1$ + $t_1$ $\cong$ v2 + t2. Then:
v1 + t1 + [t1]  $\cong$ v2 + t2 + [t1].

If we choose t2 = [t1], then we have v1 + t1 $\cong$ v2 + t2 as required. This completes the proof of the second property.

The equivalence relation $\cong$ over clock assignments an be extended to an equvalence relation over the state space of T(A) by requiring that equivalent states have identical locations and equivalent clock assignments: (s,v) $\cong$ (s', v') if and only if s = s' and v $\cong$ v'. The key property of he equivalence reltion $\cong$ is given by the following lemma [5]:


Lemma 45
If v1 $\cong$ v2 and (s, $v_1$)  $\xrightarrow[]{a}$ (s', v'). The transition $\langle$ s, a, $\varphi$, $\lambda$, s' $\rangle$  that takes state (s, v1) to state (s', v1') corresponds to two transiions of the timed automation.

Proof
Assume that v1 $\cong$ v2 and (s, v1)  $\xrightarrow[]{a}$ (s', v'1). The transition $\langle$ s, a, $\varphi$, $\lambda$, s' $\rangle$ that takes state (s, v1) to state (s', v'1) corresponds to two transitions of the timed automation:

a delay transition (s, v1)  $\xrightarrow[]{d1}$ (s, $v_1$ + $d_1$) for some $d_1$ $\geq$ 0, and
an action transition   (s, v1 +d1)  $\xrightarrow[]{a}$ (s', $v_1$') such that $v_1$ + $d_1$ satisfies $\varphi$ and $v'_1$ = ($v_1$v + $d_1$)[$\lambda$ :=0].



Since v1 $\cong$ v2 and v1 satisfies I(s), v2 also satisfies I(s). Furthermore, there exists d2 $\geq$ 0 such that v1 + d1 $\cong$ v2 + d2. Since v1 + d1 satisfies I(s), v2 +d2 also satisfies I(s). Because the clock constraint I(s) is convex and is satisfied by both v2 and v2 + d2, I(s) must be satisfied by v2 + e for all e such that 0 $\leq$ e $\leq$ d2. Consequently, the delay transition (s, v2)  $\xrightarrow[]{d2}$ (s, v2 +d2) is legal.

Since v1 + d1 $\cong$ v2 +d2, both v1+ d1 and v2+ d2 must satisfy the clock constraint for the guard $\varphi$. Thus, the transition $\langle$ s, a, $\varphi$, $\lambda$ , s' $\rangle$ myst also be enabled in the state  *s, $v_2$ + $d_2$
. Let $v'_2$ = ($v_2$ +$d_2$)[$\lambda$ :=0]. Then $v'_2$ is equivalent to $v'_1$. Hence, there is an action transition (s, $v_2$ + $d_2$)  $\xrightarrow[]{a}$  (s', $v'_2$). Combining the delay transition with the action transition, we get (s, $v_2$)  $\xrightarrow[]{a}$ (s', $v'_2$) as required.

As a result of the lemme, we can conostruct a finite state transition raph that is  bisimilaion equivalnt to the infinite state transition graph T(A). The finite state transition graph is called the region graph of A[7,8] and is denoted by R(A). A region is a pair (s, [v]). Since $\cong$  has a finite index, there are only a finite nuber of regions. The states of the region graph are  the regions of A. The construction of R(A) will have the property that whenever (s,v) is a state of T(A), the region (s, [v]) where $s_0$ is an initial state of A and $v_0$ is a clock assignment that assigns 0 to every clock. The transition relation of R(A) is defined so that bisimulation equivalence is guaranteed. There will be a transition labeled with a from the region (s,[v]) to the region (s', [v']) if and only there are assignments $\omega$ $\in$ [v] and $\omega$' $\in$ [v'] such that (s, $\omega$) can make a transition to (s', $\omega$')

We summarize the construction of the region graph R(A) below. Let A = ($\sigma$, S, $S_0$, X, I, T) be a timed automation. Then,
The states of R(A) have the form (s, [v]) where s $\in$ S and [v] i a clock region
The initial states have the form ($s_0$, [v]) where $s_0$ $\in$ $s_0$ and v(x)=0 for all x $\in$ X.
R(A) has a transition ((s,[v]),a, (s',[v'])) if and only if (s, $\omega$)  $\xrightarrow[]{a}$  (s', $\omega$') for some $\omega$ $\in$ [v] and some $\omega$' $\in$ [v'].
We can use Lemma 45 to prove bisimulation equivalene.

Theorem 31
We will show that T(A) and R(A) are bisimilar. Define the bisimulation relation B by (s,v)B(s,[v]). It is easy to see that the initial state ($s_0$, $v_0$) corresponds o the  state ($s_0$, [$v_0$]). Next, we show that for each transtition of T(A), there is a corrresponding transition  of R(A), and vice versa. Suppose first that (s,v)B(s,[v]). Suppose on the other hand that (s,v)B(s,[v]) and that  (s,v)$\xrightarrow[]{a}$(s',[v']). Then there exit $\omega$ $\cong$ v and $\omega$' $\cong$ v' such that (s', v") and (s,v) $\xrightarrow[]{a}$(s', v"). Hence v" $\cong$ $\omega$ $\cong$ v', so [v"] = [v']. By the definition of B, (s', v")
B (s', [v"]), it follows that (s', v")B(s', [v']).




blz 280 clock zones
blz 281 Intersection

blz 281 Clock reset


blz 281 elapsing of time
In principle, the three oeraions on clock zones described above can be used to construct a finite representationof the transition graph T(A) corresponding to a timed automation.


Real-time System = Discrete System + Clock Variables by Rajeev Alur

blz 2 actions
The state of a system changes over time. We refer to the state changes of a
system as actions. An action is a pair ($\sigma$,$\sigma$ ') of states that consists of a source
state $\sigma$ and a target state $\sigma$ '. Intuitively, if a system is in the source state $\sigma$,
then the action ($\sigma$,$\sigma$ ') takes the system into the target state $\sigma$'. We say that
an action is enabled in its source state and disabled in all other states. Two
actions ($\sigma$,$\sigma$ '1) and ($\sigma$,$\sigma$ '2) are consecutive if the second action is enabled in
the target state of the rst action|i.e., if ($\sigma$ '1=$\sigma$ '2). The action ($\sigma$,$\sigma$ ') is a nul l
action if ($\sigma$=$\sigma$ ')
.

blz 6 clocks and delays

Formally, the action ($\sigma$,$\sigma$ ') is a system action if for all clock variables x, either
$\sigma$ '(x) = $\sigma$(x) or $\sigma$ '(x) = 0; the action ($\sigma$,$\sigma$ ') is a time action - or delay -if there
is a nonnegative real $\delta$ the duration of the delay|such that $\sigma$ ' = ($\sigma$,$\sigma$ '). System
actions have duration 0. Every null action is, by denition, both a system action and a delay of duration 0.



blz 7 Clock constraints
Let ($\sigma$, $\delta$) be a delay, let $\phi$ be a state predicate, and let $\psi$  be an action
predicate. The characteristic function of $\phi$ maps each nonnegative real e < $\delta$ to
1 if $\phi$ is true for $\sigma$ + e, and otherwise to 0; the characteristic function of   maps
e to 1 iff $\psi$   is enabled in $\sigma$ + e. A state or action predicate varies finitely over the
delay ($\sigma$, $\delta$) if its characteristic function has nitely many discontinuities in the
interval (0,$\delta$). Abstractly, we restrict ourselves to state predicates and action
predicates that vary nitely over all delays.


blz 8 Clock-constrained systems
A clock-constrained system S = ($\phi$, $\psi$ ) is a pair that consists of a timed state
predicate 0|the initial condition of S|and a timed action predicate $\psi$ |the
transition condition of S. The timed behavior $\sigma$ is a behavior of the clock-
constrained system S if (1) the initial condition of S is initially true for $\sigma$
and (2) the transition condition of S is invariantly true for $\sigma$. Every clock-
constrained system S denes, then, the set of its divergent behaviors, which is
denoted by [[S]].



blz 9 Clock-constrained programs


blz 10 Delay predicates


blz 11 Real-time systems
A real-time system S = ($\phi$, $\psi$, $\chi$) is a triple that consists of a clock-constrained system ($\phi$, $\psi$) and a delay predicate $\chi$ the environment condition
of S. The timed behavior $\sigma$ is a behavior of the real-time system S if (1) $\sigma$ is a
behavior of the clock-constrained system that underlies S and (2) the environ-
ment condition of S is invariantly true for $\sigma$. Every real-time system S defines,
then, the set of its divergent behaviors, which is denoted by [[S]].

For example, the following real-time system S2 = ($\phi$, $\psi$, $\chi$)
changes the value of m from 0 to 1 at time 3 at the earliest and at time 5 at the
latest:
$\phi$ = (m =0 $\wedge$ x =0)
$\psi$ = (m $\geq$ 3 $\wedge$ m1' =1)
$\chi$ = (m =0 $\wedge$ x < 5 ) $\vee$ (m=1)

blz 12 Real-time executability

blz 13 Real-time programs

blz 15 Sequential real-time processes


blz 17 Concurrent real-time processes


blz 19 Embedded real-time processes

blz 30 Verification of Safety Properties

A safety property is simply a closed set of behaviors.
\begin{equation*}
	\begin{split}
		x &= 1 \\
		y &= 2 \\
		\hline
		x + y &= 3 
	\end{split}
\end{equation*}





 % \hoofdstuk{World a machine samenvating}

\paragraph{World and machine samenvatting}
Waarom zijn wij engineers? Omdat we bruikbare apparaten willen laten functioneren in de wereld waarin we leven. Dat doen we door de machine te beschrijven en deze beschrijving van instructies bieden we aan onze computer opdat deze als de attribuut en gedragingen uitleest zoals wij die hebben omschreven. Dit alles op basis van theoretische funderingen en praktisch inzicht. 

Het doel van een machine is om te worden geinstalleerd en te worden gebruikt. De eisen die we stellen zitten in de omgeving en in de wereld en de machine is slechts de oplossing die we bedenken om aan een eis te voldoen. 

De relatie machine-wereld world gecategoriseerd in: 
Het modelleer aspect: waar een machine de wereld simuleert 

Het interface aspect: waar er fysieke interactie is tussen de machine en de wereld 

Het engineering aspect: waar de machine zich gedraagt als een controlemotor gebruikmakend van de gedragingen van de omgeving in de wereld 

Het probleem aspect: waar de omgeving in de wereld en de omvang van het probleem invloed heeft op de machine en de oplossing 

Het modelleer  of simulatie aspect over een deel van de wereld. Er zijn data,object en proces modellen. Het doel van een model is toegang te geven tot informatie over die wereld. Door het opvangen van statische weergaven en gebeurtenissen kunnen wij deze gebruiken van opgeslagen informatie die we kunnen hergebruiken. Een model kan bruikbare informatie bevatten omdat zowel het model als de wereld warin het model zich bevind gemeenschappelijke omschrijvingen hebben die waar zijn voor zwel het model als voor de wereld. Daarbij moet gesteld worden dat de interpretatie van een model verschilt met een interpretatie van de wereld. 

Omdat zowel de wereld als de machine fysieke realiteiten zijn an niet slechts abstracties, zijn de gemeenschappelijke beschrijvingen slechts een deel van de werkelijheid van beide objecten. For elk object zijn er meerdere beschrijvingen. Toch maken niet alle omschrijvingen deel uit van het getoonde reportoire. Zoals niet alle eigenschappen van een boek; meer dan een auteur, pseudoniemen, een onderdeel van een reeks, een gerevisiteerde versie, worden gereflecteerd in een database.  

Het interface aspect. Een machine kan een probleem in de wereld oplossen als de wereld en de machine phenomena kunnen uitwisselen. Maar de participatie is niet symmetrisch: een status kan als phenomena worden uitgewisseld maar slechts een partij kan er invloed op uitoefenen maar beiden kunnen dezelfde status signaleren. 

Het engineering aspect gaat over requirements, specificaties, en programma’s. Requirements hebben betrekking op phenomena in de wereld. Een programma heeft alleen betrekking tot de machinale phenomena. Het doel van programma’s is om eigenschappen en gedragingen te omschrijven van de machine ten behoeve van de gebruiker. Tussen de requirements en de programma’s zitten de specificaties. Omdat programma’s dan wel beschrijvingen zijn van een gewenste machine, maar dat moeten beschrijvingen zijn van de  machines  die de computers kunnen uitvoeren zodanig dat de computer deze beschrijvingen ook zo kan interpreteren. De engineer moet  de eigenschappen van de wereld kennen en begrijpen en deze eigenschappen manipuleren en laten werken met als doel het dienen van het systeem. 

Het probleem aspect. Het onderscheid tussen specificatie en implementatie. Het probleem zit in de relatie van de machine en de wereld. De machine brengt de oplossing maar het probleem zit in de wereld. Een vertoog over een probleem moet dus gaan over de wereld en over de opvatting die de gebruiker heeft in de wereld. Omdat de wereld veelzijdig is moeten we ervan uit gaan dat er verschillende soorten problemen zijn. Een realistisch probleem wordt dus niet opgelost met een simpele hiërarchische structurele aanpak en een homogene decompositie maar met een paralleele structurele oplossing waar beide kanten van het probleem worden opgelost. 



Ontkenningen 

We hebben als engineers de taak om een machine te bouwen aan de hand van de specificaties opgeleverd door de opdrachtgever. Een engineer heeft niet als taak de fitheid voor een doeleind te onderzoeken, maar wel de haalbaarheid naar een doeleind aan de hand van kennis, tijd, resources, budget en ontwikkelmethodiek. Daaruit komt naar voren dat een engineer zich richt op: elicitation (schetsen van een requirement), description (omschrijving) en analyse van de requirements waaraan het systeem moet voldoen. Vertaalt naar de volgende vragen: Wat is precies de klantwens?  Wat is de precieze omschrijving van het probleem? Voor welke doelen wordt het systeem gebouwd? Welke functies moet het systeem hebben? 

Denial by hacking: obsessief bezig zijn met een systeem omdat het de gebruiker veel macht geeft. Een uitgebreidheid van een systeem zorgt er soms voor dat mensen niet meer geprikkeld zijn na te denken over probleemstellingen, domein beschrijvingen en analyse. 

Denial by a abstraction. Wiskundige benaderingen van werkelijke problemen is  een belangrijke intellectuele strategie om problemen te formuleren. Een software ontwikkelaar moet een probleem kunnen omschrijven in zo min mogelijk woorden, maar de complexiteit ligt in de oplossing. 

Denial by vagueness. De vaagheid van een omschrijving is terug te vinden in: 

Von Neumann’s principe ,Principe van reductionisme ,Shanley principe en het Montaingnes’s principe.
Het Von Neumand principe uitgelegd
Voor een vocabulair  moet een grondslag zijn ontwikkeld waarmee gesproken kan worden over de wereld en de machine. Belangrijke phenomenen moeten geindtifieerd worden, door middel van een grondregel  of ‘herkenningsregel’ moet een fenomeen worden herkend, en vervolgens het fenomeen een formele term geven die gebruikt wordt als duiding van een bepaalde omschrijving. Dan moet voor de formele term een symbool gevonden worden. Samen vormen de grondregel en het symbool een designatie. 

Principe van reductionisme 

Simpelweg het openbreken van termen met een weerlegbare definitie totdat alle begrippen die worden gebruikt om iets te duiden  niet meer te herconstrueren zijn in hun definitie. 

Shanley principe 

Er bestaan volgens dit principe geen scherpe verdelingen in de wereld zoals wetenschappers soms denken. Een strenge opvatting over de wereld waarin een individu geclassificeerd kan worden als een onsamenhangend geheel. Maar dat is slechts een opname van een beeld. De werkelijkheid staat soms toe dat een elementair individueel object in verschillende classificaties verschillende getypeerd kan worden in een andere setting of view. 

Montaignes principe 

De incative mood; gaat over wat we beweren waar te zijn. 

De optitative mood; gaat over wat we willen dat waar is    %(optioneel) bijlagen
%  \include{}   %(optioneel) bijlagen
%\include{Schematics}
%\include{Bill of Material}
%\include{Use-Case flow charts}
%\include{Competences}
%\include{Gannt planning }
%\include{Authors' contributions}
%\include{Data availability statement}
%\include{Declarations}
%\include{Footnotes}
%\include{Contributor Information}
%\include{Conflicts of interest}
\fi
\end{document}       %Einde van het document
