%%%%%%%%%%%%%%%%%%%%%%%%%%%%%%%%%%%%%%%%%%%%%%%%%%%%%%%%%%%%%%%%%%%%%%%%%%%%%
%eventueel dutch in english veranderen, eventueel twoside verwijderen       %
%%%%%%%%%%%%%%%%%%%%%%%%%%%%%%%%%%%%%%%%%%%%%%%%%%%%%%%%%%%%%%%%%%%%%%%%%%%%%
\documentclass[11pt,a4paper,twoside,dutch]{report}%
\usepackage[dutch]{babel}%
\selectlanguage{dutch}%
%%%%%%%%%%%%%%%%%%%%%%%%%%%%%%%%%%%%%%%%%%%%%%%%%%%%%%%%%%%%%%%%%%%%%%%%%%%%%
\newif\ifpublic \newif\iftwocol %nodig voor publicatiemode, niet wijzigen!  %
%%%%%%%%%%%%%%%%%%%%%%%%%%%%%%%%%%%%%%%%%%%%%%%%%%%%%%%%%%%%%%%%%%%%%%%%%%%%%
%publiceerbare versie (geen titelblad, bedanktuiging, evaluatie en bijlagen)%
%%%%%%%%%%%%%%%%%%%%%%%%%%%%%%%%%%%%%%%%%%%%%%%%%%%%%%%%%%%%%%%%%%%%%%%%%%%%%
%\publictrue        %1-kolom publicatiemode (zonder %) of verslagmode (met %)
%\twocoltrue        %2-kolommen publicatiemode
%%%%%%%%%%%%%%%%%%%%%%%%%%%%%%%%%%%%%%%%%%%%%%%%%%%%%%%%%%%%%%%%%%%%%%%%%%%%%
%nodig voor template: packages en macro's                                   %
%%%%%%%%%%%%%%%%%%%%%%%%%%%%%%%%%%%%%%%%%%%%%%%%%%%%%%%%%%%%%%%%%%%%%%%%%%%%% 
\ifpublic\iftwocol\twocolumn\fi\fi  %vlag 1 of 2 kolommen                   %
\usepackage[utf8]{inputenc}         %aanbevolen encoding, kies 1 van de 3
%\usepackage[utf-8]{inputenc}      %sommige Ubuntu en Window distributies
%\usepackage[latin1]{inputenc}     %oudere LaTeX-distributies
\usepackage{modelverslag}           %verplicht voor de template             %
\renewcommand{\redactie}[1]{\relax} %nodig voor redactionele opmerkingen    %
%%%%%%%%%%%%%%%%%%%%%%%%%%%%%%%%%%%%%%%%%%%%%%%%%%%%%%%%%%%%%%%%%%%%%%%%%%%%%
%hieronder komen alle packages die je zelf wil plaatsen of veranderen       %
\usepackage[most]{tcolorbox}
\usepackage{enumitem}
\usepackage{lipsum}

\colorlet{helpful}{lime!70}
\colorlet{harmful}{red!30}
\colorlet{internal}{yellow!20}
\colorlet{external}{cyan!30}
\colorlet{S}{helpful!50!internal}
\colorlet{W}{harmful!50!internal}
\colorlet{O}{helpful!50!external}
\colorlet{T}{harmful!50!external}

\newcommand{\texta}{Helpful}
\newcommand{\textb}{Harmful}
\newcommand{\textcn}{\rotatebox{90}{\parbox[c]{3cm}{\centering Internal origin}}}
\newcommand{\textdn}{\rotatebox{90}{\parbox[c]{3cm}{\centering External origin}}}

\newcommand{\texts}{strength 1\par strength 2}
\newcommand{\textw}{weakness 1\par weakness 2}
\newcommand{\texto}{opportunity 1\par opportunity 2}
\newcommand{\textt}{threat 1\par threat 2}

\tcbset{swotbox/.style={size=normal, boxrule=0pt,
		colback=#1, watermark text=#1, width=.5\linewidth-5mm},
	header/.style={size=normal, boxrule=0pt, width=.5\linewidth-5mm, colback=#1, valign=center, halign=center},
	firstcol/.style={header=#1, width=1cm, boxsep=0mm}
}



\usepackage[svgnames, table]{xcolor}
\usepackage[utf8]{inputenc}
\usepackage[T1]{fontenc}
\usepackage[ngerman]{babel}
%\usepackage[automark,headsepline,plainheadsepline, plainfootsepline, footsepline]{scrlayer-scrpage}
\pagestyle{scrheadings}

%Tabelle
\usepackage{multirow, makecell, cellspace, bigstrut}
\usepackage{ragged2e}
\usepackage{ltablex}
\keepXColumns
\renewcommand\tabularxcolumn[1]{ >{\arraybackslash}m{#1}}
\usepackage{booktabs}
\newcolumntype{L}[1]{>{\RaggedRight}p{#1}}
\newcolumntype{C}[1]{>{\centering\arraybackslash}p{#1}}
\newlength{\lastcolwd}
\settowidth{\lastcolwd}{Bermerkungen}
\usepackage[inner=2.5cm,outer=2.5cm,top=2cm,bottom=1.0cm,includeheadfoot]{geometry}
\newcommand\mymidrule{\specialrule{\lightrulewidth}{0pt}{\belowrulesep}}
\newcommand\mybottomrule{\specialrule{\heavyrulewidth}{0pt}{\belowrulesep}}
\newcommand\mytoprule{\specialrule{\heavyrulewidth}{0pt}{0pt}}
\usepackage{amsmath}
\setlength\cellspacetoplimit{5pt}
\setlength\cellspacebottomlimit{5pt}
%%%%%%%%%%%%%%%%%%%%%%%%%%%%%%%%%%%%%%%%%%%%%%%%%%%%%%%%%%%%%%%%%%%%%%%%%%%%%
\usepackage{algorithmic}      %aangepaste algoritmen   
\usepackage{listings}         %aangepaste listings (event. in kleuren)
%\usepackage{}
%\usepackage{}

\usepackage[svgnames, table]{xcolor}
\usepackage[utf8]{inputenc}
\usepackage[T1]{fontenc}

\begin{document}
\titelblad                   %argumenten invullen
  {Persoonlijk verslag} %hier titel invullen 
  {Galvin Bartes (0799967)} %hier naam en (stamnummer) van student invullen
  {\TI}                      %hier opleiding \TI, \INF of \BI invullen
  {\today}                   %{\today} concept of {1 april 2011} definitief
  {Dhr. W. Oele}           %eerste docent
  {}                         %tweede docent
% als er meer auteurs zijn, dan wordt de tweede parameter:
% {Onbekend Talent1 (123456)\\Onbekend Talent2 (987654)\\ ...} 
\ifpublic
  {\footnotesize{\samenvatting

Het doel van een samenvatting is om potentiële lezers zo snel mogelijk
te overtuigen van de relevantie van het verslag. Als afstudeerverslagen
gepubliceerd worden, is digitaal zoeken noodzakelijk. Daarom worden in
de samenvatting (\emph{abstract}) vaak kenmerkende woorden en
uitspraken opgenomen. Een samenvatting voor een afstudeerverslag mag
niet meer dan een paar honderd woorden bevatten.
}}
\else
  \pagenumbering{roman}
  \samenvatting

Het doel van een samenvatting is om potentiële lezers zo snel mogelijk
te overtuigen van de relevantie van het verslag. Als afstudeerverslagen
gepubliceerd worden, is digitaal zoeken noodzakelijk. Daarom worden in
de samenvatting (\emph{abstract}) vaak kenmerkende woorden en
uitspraken opgenomen. Een samenvatting voor een afstudeerverslag mag
niet meer dan een paar honderd woorden bevatten.
     %(verplicht) samenvatting
  \hoofdstuk{Dankbetuiging}

Wie kan je zoal bedanken? Denk aan de begeleiders en voorbereiders van
je afstudeerproject, familieleden en andere personen die je
geadviseerd of gemotiveerd hebben.  Het is gebruikelijk om dit
voorafgaande aan het verslag te doen. Dit bedanken mag ook in de
inleiding gebeuren. Bijvoorbeeld: Bij het opstellen van dit verslag
heb ik dankbaar gebruik gemaakt van `metathesis' van \emph{Donald Craig
(donald@mun.ca)}.

  %(optioneel) dankbetuiging
  \tableofcontents         %overzicht hoofdstukken en paragrafen
  \hoofdstuk{trefwoorden}

trefwoorden
volgens de gebruikte thauserus. een thesaurus is een lijst vna goedgekeurde en geaccepteerde vaktermen, de 'controlled descriptors' met de verklaring en met de afgekeurde alternatieve vaktermen    %(optioneel) overzicht afkortingen
  \pagenumbering{arabic}
\fi
\hoofdstuk{Inleiding}
 
 

\paragraph{Algemeen}

Het ministerie van verkeer en Waterstaat wil in het kader van het klimaatakkoord en onderzoek laten uitvoeren naar de staat van het sluizenpark in Nederland. Het onderzoek moet zich richten op het ontwerpen en ontwikkelen van een geautomatiseerd sluismodel dat geschikt is voor een brede toepassing. In het onderzoek moet naar voren komen wat de huidige staat is van de sluizen met oog op veiligheid, efficiëntie, capaciteit, onderhoud, duurzaamheid en automatisering. Het onderzoek geeft aan hoe een volledig model worden opgeleverd opdat ontwerp van verschillend volledig geautomatiseerde sluizen in de toekomst geautomatiseerd kunnen worden.  

\paragraph{Recente ontwikkelingen op het gebied van sluisautomatisering}

Het ministerie van verkeer en Waterstaat wil in het kader van het klimaatakkoord en onderzoek laten uitvoeren naar de staat van het sluizenpark in Nederland. Het onderzoek moet zich richten op het ontwerpen en ontwikkelen van een geautomatiseerd sluismodel dat geschikt is voor een brede toepassing. In het onderzoek moet naar voren komen wat de huidige staat is van de sluizen met oog op veiligheid, efficiëntie, capaciteit, onderhoud, duurzaamheid en automatisering. Het onderzoek geeft aan hoe een volledig model worden opgeleverd opdat ontwerp van verschillend volledig geautomatiseerde sluizen in de toekomst geautomatiseerd kunnen worden.  
\paragraph{Wat is een sluis}

\paragraph{Wat worrdt er omschreven en wat is er geleerd}

\paragraph{Wat is uppaal}

Wat is Uppaal
Uppaal is an integrated tool environment for modeling, simulation and verification of real-time systems, developed jointly by Basic Research in Computer Science at Aalborg University in Denmark and the Department of Information Technology at Uppsala University in Sweden. It is appropriate for systems that can be modeled as a collection of non-deterministic processes with finite control structure and real-valued clocks, communicating through channels or shared variables [WPD94, LPW97b]. Typical application areas include real-time controllers and communication protocols in particular, those where timing aspects are critical.


model checking

Wat is statistical model checking?
Dit verwijst naar verschillende technieken dfie worden gebruikt voor de monitoring van een systeem. Daarbij wordt vooral gelet op een specifieke eigenschap. Met de resultaten van de statsitieken wordt de juistheid van een ontwerp beoordeeld. Statistisch model checking wordt onder andere toegepast in systeembiologie, software engineering en industriele toepassingen.
https://www-verimag.imag.fr/Statistical-Model-Checking-814.html?lang=en#:~:text=Statistical%20Model%20Checking%20(SMC)%20is,from%20state%20space%20explosion%20issues.


\cite{inriaStatsMoodCheck}
\cite{ buddeModelChecker}
\cite{AGHASuervey }


Waarom gebruiken we statistisch model checking?
To overcome the above difficulties we propose to work with Statistical Model Checking [KZHHJ09,You05,You06,SVA04,SVA05,SVA05b] an approach that has recently been proposed as an alternative to avoid an exhaustive exploration of the state-space of the model. The core idea of the approach is to conduct some simulations of the system, monitor them, and then use results from the statistic area (including sequential hypothesis testing or Monte Carlo simulation) in order to decide whether the system satisfies the property or not with some degree of confidence. By nature, SMC is a compromise between testing and classical model checking techniques. Simulation-based methods are known to be far less memory and time intensive than exhaustive ones, and are oftentimes the only option. 
https://project.inria.fr/plasma-lab/statistical-model-checking/

Alternatief
Alternatieven voor Uppaal zijn Asynchronous Events,Vesta en MRMC.
%%%%%%%%%%%%%%%%%%%%%%%%%%%%%%%%%%%%%%%%%%%%%%%%%%%%%%%%%%%%%%%%%

\paragraph{Probleemanalyse}

Na grondige analyse van het Nederlandse sluizenpark is gebleken dat renovatie van een groot aantal sluizen noodzakelijk is.  Uit een eerste verkenning is gebleken  dat het gecombineerd renoveren en automatiseren van het Nederlandsesluizenpark een aanzienlijke verbetering kan opleveren t.a.v. 
Op  het  ministerie  van  infrastructuur  enwaterstaat is helaas onvoldoende kennis van ict en systemen aanwezig om eenen ander uit te voeren 

\paragraph{Waarom nu}
In  het  kader  van  het  onlangs  afgesloten  klimaatakkoord  heeft  de  Nederlandseoverheid  daarom  besloten  over  te  gaan  tot  een  ingrijpende  renovatie  van  dediverse  sluizen  die  ons  land  rijk  is.     

\paragraph{Gewenst resultaat }


Wij vragen u een model (of een onderling samenhangend aantal modellen)aan  te  leveren,  opdat  ontwerpen  van  verschillende,  volledig  geautomatiseerdesluizen in de toekomst gerealiseerd kunnen worden. 
Zoals  gesteld  in  de  brief  is  het  de  bedoeling  dat  een  sluis  gemodelleerd  wordten  dat  bewezen  kan  worden  dat  de  te  bouwen  sluis  een  aantal  eigenschappenbezit.  

\paragraph{Scope}

He gaat om het simuleren van een geautomatiseerde sluis. Wat voor type sluis wordt niet gemeld en ook niet uit welke onderdelen. Belangrijk is dat het model werkt en dat het voldoet aan de eisen die gebaseerd zijn op basis van literatuuronderzoek, observatie, interviews, brainstorming of een andere vorm van requirements elicitation.

\paragraph{Onderzoeksvragen }

Hoe kan een geautomatiseerde sluis worden gemodeleerd met oog op ontwikkel- en onderhoudskosten,veiligheid, efficientie en capaciteit





\begin{enumerate}
	
	\item Welke requirements en kwaliteitseisen komen naar voren bij de analyse van een rampenonderzoek
	\item Welke veiligheidseisen er zijn voor sluizen in nederland. 
	\item Hoe kan in uppaal  een model worden getest dat voldoet aan de requirements/eisen volgens het rampenonderzoek?
\end{enumerate}

%%%%%%%%%%%%%%%%%%%%%%%%%%%%%%%%%%%%%%%%%%%%%%%%%%%%%%%%%%%%%%%%%


\paragraph{Design goals}
Het systeem moet minimaal aan de volgende prestatie eisen voldoen 

\begin{enumerate}
	\item   Requirements gebaseerd op rampenanalyse
	\item Model testbaar in upaal
\end{enumerate}

%%%%%%%%%%%%%%%%%%%%%%%%%%%%%%%%%%%%%%%%%%%%%%%%%%%%%%%%%%%%%%%%%
\paragraph{Welke aanpak is gekozen en welke studies liggen hieraan ten grondslag?}
https://link.springer.com/article/10.1007/s10626-020-00314-0

\paragraph{Leeswijzer}
In  de methodologie wordt de lezer uitgelegd met welke methoden de onderzoeksvragen zijn beantwoord. In het hoofdstuk Onderzoek worden alle resultaten behandeld die naar voren zijn gekomen bij het deskresearch. De analyse van de verzamelde data wordt gedaan in het hoofdstuk analyse. Hierin wordt behandeld zoekopdracht naar IoT cloud platforms, feature extractie, prijs-berekening en prijs-feature vergelijking. In het ontwerp komen de uml diagrammen en systeemschetsen naar voren. In de  de hoofdstukken Prototype, IoT cloud en Firmware wordt de implementatie behandeld van het IoT cloud platform in een bestaand project.

%%%%%%%%%%%%%%%%%%%%%%%%%%%%%%%%%%%%%%%%%%%%%%%%%%%%%%%%%%%%%%%%%



%	\input{Methoden} %(verplicht) hoofdverslag
%\hoofdstuk{methoden}
    %(verplicht) inleiding
%\hoofdstuk{materiaal}


materiaal
geef type:kwantiteit(aantal steekproeven, hoeveelhjeid per steekproef)


Material and Methods


data sources en references of data. 


research approach

inclusion and exclusion criteria of studies

how many studies have screened and included 

statistical process of meta-analysis.

Instrument development

Sample and data collection

Analytical method

Respondents’ demographic profile

Reliability and validity of research instrument

Result analysis

Examples
Type  

    %(verplicht) inleiding
%\hoofdstuk{methode}


standaardmethoden vermelden met codenummer
bij gewijzigde metho9de/apparaat; de wijziging precies aangeven




Research methodology

Instrument development

Sample and data collection

Analytical method

Respondents’ demographic profile

Reliability and validity of research instrument

Result analysis



\section{Methodologie}
\label{chapter:body}
\thispagestyle{myheadings}
Voor dit rapport is onderzoek gedaan naar sluizen, sluismodellen, rampen, rampenbestrijdingsprocedures en requiremeentsengineering.
%%%%%%%%%%%%%%%%%%%%%%%%%%%%%%%%%%%%%%%%%%%%%%%%%%%%%%%%%%%%%%%%%%%%%%%%%

\subsubsection{Literatuuronderzoek}
\begin{frame}{Literature Review}
	\begin{table}[htbp]
		\footnotesize
		
		\centering
		\begin{tabular}{|c|c|p{2in}|c|c|}\hline
			S.no&Author&Title&Findings&Gap in literature\\\hline
			S.no&Author&wanrooy \textunderscore vab1991a.pdf&Findings&Gap in literature\\\hline
			S.no&Author&wa3300-bezuien2000(1).pdf&Findings&Gap in literature\\\hline
			S.no&Author&Title&Findings&Gap in literature\\\hline
			S.no&Author&Title&Findings&Gap in literature\\\hline
			S.no&Author&rapport-veiligheid-van-op-afstand-bediende-burggen.pd&Findings&Gap in literature\\\hline
			S.no&Author&pronk.pdf&Findings&Gap in literature\\\hline
			S.no&Author&Olieman1987a.pdf&Findings&Gap in literature\\\hline
			S.no&Author&richtlijnen-vaarwegen-2020.pdf&Findings&Gap in literature\\\hline
			S.no&Author&richtlijnen-vaarwegen-2017 \textunderscore tcm21-127359(1).pdf&Findings&Gap in literature\\\hline
			S.no&Author&Olieman1987a.pdf&Findings&Gap in literature\\\hline
			S.no&Author&Meijer1980b.pdf&Findings&Gap in literature\\\hline
			S.no&Author&Meijer1980c.pdf&Findings&Gap in literature\\\hline
			S.no&Author&kst-31200-A-80-b2.pdf&Findings&Gap in literature\\\hline
			S.no&Author&duurzaamheid \textunderscore bij \textunderscore de \textunderscore ontwikkeling \textunderscore van \textunderscore reevesluis.pdf&Findings&Gap in literature\\\hline
			S.no&Author&De \textunderscore deltawerken \textunderscore Cultuurhistorie \textunderscore ontwerpgeschiedenis \textunderscore web-A.pdf&Findings&Gap in literature\\\hline
			S.no&Author&wa3300-Bezuijen2000.pdf&Findings&Gap in literature\\\hline
			S.no&Author&Sander van Alphen Haalbaarheidsstudie naar grote sluisdeuren uitgevoerd in hogesterktebeton.pdf&Findings&Gap in literature\\\hline
			S.no&Author&Dalmeijer1994a.pdf&Findings&Gap in literature\\\hline
			S.no&Author&Dalmeijer1994b.pdf&Findings&Gap in literature\\\hline
			S.no&Author&Dalmeijer1994c.pdf&Findings&Gap in literature\\\hline
			S.no&Author&ceg \textunderscore pruijssers \textunderscore 1982.pdf&Findings&Gap in literature\\\hline
			S.no&Author&Capaciteitsanalyse \textunderscore van \textunderscore de \textunderscore prinses\textunderscore margrietsluis \textunderscore in \textunderscore lemmer \textunderscore - \textunderscore Marc \textunderscore Lamboo.pdf&Findings&Gap in literature\\\hline
			S.no&Author&Boer1979a.pdf&Findings&Gap in literature\\\hline
			S.no&Author&bijlagerapport \textunderscore c \textunderscore - \textunderscore analyse \textunderscore geavanceerd-definitief \textunderscore v1 \textunderscore 0.pdf&Findings&Gap in literature\\\hline
			S.no&Author&Bijl1988a.pdf&Findings&Gap in literature\\\hline
			S.no&Author&Bentum1978a.pdf&Findings&Gap in literature\\\hline
			S.no&Author&Alphen.pdf&Findings&Gap in literature\\\hline
			S.no&Author&Abbenhuis1975a.pdf&Findings&Gap in literature\\\hline
			S.no&Author&Abbenhuis1974a.pdf&Findings&Gap in literature\\\hline
			S.no&Author&https://wiki.woudagemaal.nl/w/index.php/Sluizen&Findings&Gap in literature\\\hline
			S.no&Author&Title&Findings&Gap in literature\\\hline
			
		\end{tabular}
	\end{table}
	
\end{frame}    %(verplicht) inleiding

 



\hoofdstuk{Theoretisch kader}

In het eerste hoofdstuk is duidelijk geworden wat de onderzoeksvraag is, namelijk ‘Hoe kan een geautomatiseerde sluis worden gemodeleerd met oog op ontwikkel- en onderhoudskosten,veiligheid, efficientie en capaciteit’. Door de toenemende complexiteit van systemen is het gebruik van modellen en de toepassing van timebased model checking  op industriele controle systemen een manier van modelleren van het systeem en de requirements zodat er een bijdagre kan worden geleverd aan de acceptatie van  simulatie-/modeltechniek voor de industrie.(‘https://link.springer.com/article/10.1007/s10626-020-00314-0’, 2020). Of dit ook het geval is bij het modellereren van sluizen is nu de vraag.

De bestudering van rampen aan de hand van het vier-variabelen model biedt maakt het analyseren mogelijk van rampsituaties. Van een aantal rampen is een beschrijving gegeven met datum, plaats en oorzaak. De analyse van de 4-variabelen modellen zal gebruikt worden voor de requirementsdefinitie, ontwerp en ontwikkeling van het sluismodel. 

De verschillende factoren en achtergronden die  samenhangen met het modelleren van een sluis zullen in dit hoofdstuk toegelicht worden. Bovendien worden er hypotheses gevormd die de basis vormen voor debeantwoording van de onderzoeksvraag. 




\paragraph{Wat is uppaal}

Wat is Uppaal
Uppaal is een geïntegreerde toolomgeving voor het modelleren, simuleren en verifiëren van real-time systemen, gezamenlijk ontwikkeld door Basic Research in Computer Science aan de Universiteit van Aalborg in Denemarken en de afdeling Informatietechnologie aan de Universiteit van Uppsala in Zweden. Het is geschikt voor systemen die kunnen worden gemodelleerd als een verzameling niet-deterministische processen met een eindige controlestructuur en klokken met reële waarde, die communiceren via kanalen of gedeelde variabelen [WPD94, LPW97b]. Typische toepassingsgebieden zijn met name real-time controllers en communicatieprotocollen, waarbij timingaspecten van cruciaal belang zijn. \cite{uppaalFeatures}


model checking

Wat is statistical model checking?
Dit verwijst naar verschillende technieken dfie worden gebruikt voor de monitoring van een systeem. Daarbij wordt vooral gelet op een specifieke eigenschap. Met de resultaten van de statsitieken wordt de juistheid van een ontwerp beoordeeld. Statistisch model checking wordt onder andere toegepast in systeembiologie, software engineering en industriele toepassingen.
\cite{verimagStatsModelChecking}.


\cite{inriaStatsMoodCheck}
\cite{ buddeModelChecker}
\cite{AGHASuervey }


Waarom gebruiken we statistisch model checking?
Om de bovenstaande problemen te overwinnen stellen we voor om te werken met Statistical Model Checking [KZHHJ09,You05,You06,SVA04,SVA05,SVA05b], een aanpak die onlangs is voorgesteld als alternatief om een uitputtende verkenning van de toestandsruimte van het model te vermijden. Het kernidee van de aanpak is om een aantal simulaties van het systeem uit te voeren, deze te monitoren en vervolgens de resultaten uit het statistische gebied te gebruiken (inclusief het testen van sequentiële hypothesen of Monte Carlo-simulaties) om te beslissen of het systeem aan de eigenschap voldoet of niet. mate van vertrouwen. Van nature is SMC een compromis tussen testen en klassieke modelcontroletechnieken. Het is bekend dat op simulatie gebaseerde methoden veel minder geheugen- en tijdintensief zijn dan uitputtende methoden, en vaak de enige optie zijn.
\cite{inriaStatsMoodCheck}

Alternatief
Alternatieven voor Uppaal zijn Asynchronous Events,Vesta en MRMC.


\paragraph{MODE CONFUSION }
Mode confusion tredd op als gepbserveerd gedrag van een technisch systeem niet past in het gedragspatroon dat de gebruiker in zijn beeldvorming heeft  en ook niet met voorstellingsvermogen kan bevatten.
\paragraph{Wat is automatiseringsparadox}
Gemak dient de mens. Als er veel energie wordt gestoken in de ontwikkeling van hulmiddelen die taken van werknemers overemen heeft dat tot resultaat dat veel productieprocessen worden geautomatiseerd. De vraag is dan of vanuit mechnisch wereldpunt de robot niet de rol van de mens overneemt en of de mens nog de kwaliteiten heeft om het werk zelf te doen.
\cite{bicker21102016automatiseringsparadox }
\cite{vseautoparadox }
\cite{blogxot21112016slimapparaat }



\paragraph{4 variabelen model}





Het 4 variabelen model kort toegelicht
Monitored variabelen: door sensoren gekwantificeerde fenomenen uit de omgeving, bijv temperatuur

Controlled variabelen: door actuatoren bestuurde fenomenen uit de omgeving
For example, monitored variables might be the pressure and temperature
inside a nuclear reactor while controlled variables might be visual and audible alarms, as well
as the trip signal that initiates a reactor shutdown; whenever the temperature or pressure reach
abnormal values, the alarms go off and the shutdown procedure is initiated

Input variabelen: data die de software als input gebruikt
Here, IN models the input hardware interface (sensors and analog-to-digital converters) and
relates values of monitored variables to values of input variables in the software. The input variables model the information about the environment that is available to the software. For example,
IN might model a pressure sensor that converts temperature values to analog voltages; these voltages are then converted via an A/D converter to integer values stored in a register accesible to the
software.

Output variabelen: data die de software levert als output
The output hardware interface (digital-to-analog converters and actuators) is modelled
by OUT, which relates values of the output variables of the software to values of controlled variables. An output variable might be, for instance, a boolean variable set by the software with the
understanding that the value true indicates that a reactor shutdown should occur and the value
false indicates the opposite




\paragraph{World and machine samenvatting}

\paragraph{6 Variable model}
Optitatieve statements omschrijven de omgeving zoals we het willen zien vanwege de machine. 
Indicatieve statements omschrijven de omgeving zoals deze is los van de machine. 
Een requirement is een optitatief statement omdat ten doel heeft om de klantwens uit te drukken in een softwareontwikkel project. 
Domein kennis bestaut uit indicatieve uitspraken die vanuit het oogpunt van software ontwikkeling relevant zijn. 
Een specificatie is een optitatief statement met als doel direct implementeerbaar te zijn en ter verondersteuning van het natreven vande requirements. 
Drie verschillende type domeinkennis: domein eigenschappen, domein hypothesen, en verwachtingen. 
Domein eingenschappen  zijn beschrijvende statementsover een omgeving en zijn feiten.Domein hypotheses  zijn ook beschrijvende uitspraken over een omgeving, maar zijn aannames. 
Verwachtingen zijn ook aannames, maar dat zijn voorschrijvende uitspraken die behaald worden door actoren als personen, sensoren en actuators. 

  
\paragraph{Conceptueel model}



System requirement:
uitspraak over wereld fenomenen (gedeeld of niet) of doelen
die bereikt moeten worden.
met enige regelmaat informeel, niet precies geformuleerd.
Software requirement/specicatie:
uitspraak over gedeelde fenomenen of doelen die de machine
moet bereiken middels de onderdelen waar die machine uit
bestaat of middels de fenomenen waar de machine controle
over heeft.
doorgaans preciezer, meetbaar, exact geformuleerd.


Systemen gaan een zekere interactie aan met hun omgeving:
Sensoren: meten fenomenen uit de omgeving (temperatuur,
druk, licht, geluid, etc.)
actuatoren: veranderen iets in de omgeving (mechanische,
electrisch, pneumatisch, etc.)
Software:
Kan niet direct communiceren met de buitenwereld.
Snapt derhalve niets van de buitenwereld.
Kan alleen maar bestaan in en communiceren met het
systeem.


\paragraph{Requirementsengineering}

Om de juiste requirements te verzamelen en selecteren hebben we meer kennis nodig van de methoden hiervoor gebruikt in het domein van requirementsengineering. Daarom is een literatuurstudie gedaan naar rapporten en artikelen die ons meer informatie over dit onderwerp verschaffen.
 Uitdagingen in requirementsengineering zijn incomplete requirements en specifcates, veranderende requirements en specificates en grote, complexe oftwaresystemen.
 
 Het article the worlds a stage \cite{breitmanLeiteCesar2002reallifeReqs}biedt inzicht in de requirementstechnieken voor een ambulance in london. In het artikel gaan de onderzoeks in op de volgende onderwerpen: 
 viewpoints, sociale ascpecten,evolutie, non-functional requirements, conflict resolution, traceability
 
 Goal of this paper is requirement  engineering on London aulance service
 Method of opinions: crew, staff, management, computational, transport, services
 Evolutioon: changes, specification and technology trade
 Environment: company policies, regulation, impact solution on organizational
 Non-functional aspect: communicatio problem, malfunctions, less critical isues: cost, tradeoff beween performance \& user interfaces
 vieuwpoint: is a subset of all system requirements expressible in a given requirements notation regardless of the stakeholders involved
 
 log change
 basic model vieuw
 hypertext vieuw
 data transmission problems
 continued difficulties
 installation problems
 problems caused by mistake
 tracebility requirements[selecting reliable information]
 PRE requirement specification traceability, repository baed approach
 1) compromise specification
 2) representatives
 3) agreement dimensions
 Domain: part of the worl in which the computer system effects will be felt, inclusing its peoples, organizational structure, related legislation, physical location and met only the compyter systems
 
 
 
 Het artikel "from inconsistencyhandling to non-conanical requirements management: a logical perspective" \cite{muHungJinLiu2013inconsistencyReqs}geeft enkele tips voor het omgaan met inconsistente requirements:
 
 1) identifying non-canonicalrequirements
 2) measuring them
 3) generate caandidate proposals for handling them
 4) choosing acccptable probosals
 5) revising them acccording to the proposals

Het artikel "managing inconsistent specification: reasoning, analysis, action" \cite{hunterNuseibeh1996manageSpecs} zoekt een ontologische benadering voor het omgaan met inconsistenties in de requirements specificaties.
Voor de omshrijving van een specificatie kun je gebruik maken van logica. Daarbij kun je onderschei maken in klasieke logica quasi -logica.
Wat ook een rol kan spelen in domain interpretatie. De achtergrond van de gebruikers speelt ook een rol.
Zo is er e=onderscheid te maken in de volgende groepen: users, customers, domain experts, designers,, manufacturers
graphical  textual specification

Basic constraint, legal constraint, cooperation constraint
1) scenatio  definition
2) scenario analysis
3) scenario consolidation


Hoe kan een systeem verder worden ontworpen op een manier dat non-functionele requirements worden geimplementeerd?
Hoe hangt dat ontwerp samen met aanpassingen van het functionele en structurele aspect van het systeem?

block[objects, classes, methods, messages, inheritance]
[goals,agents, alternative, events, actions,existence modalities,agent responsibilities]


Het artikel "representing and using nonfunctional requirements: a process-oriented approach" \cite{myloloupos1992representingReqs} gaat in op een het proces van requirements acquisitie. Hierbij in ogenschouw de acquisitie van prestaties, ontwerp en aanpasbaarheid vanuit bijvoorbeeld gebruikersperspectief. Enkele vragen vanuit dit perspectief zijn:
-Hoe goed werkt het product
-Hoe goed wordt de bron gebruikt?>> Efficiency
-How veilig is het product >> integrity
-Met hoeveel zekerheid is uit  te sluiten dat het werkt >>Reliability
-Hoe goed werkt het product onder zware omstandigheden >> sustainability
-Hoe makkelijk is het product in gebruik >> usability
 Ten tweede zijine r vanuit het perspectief van de ontwerper ook andere perspectieven  belangrijk als niet-functionele eis. Een ontwerper stelt bijvoorbeeld de vraag:
Hoe valide is het ontwerp
-Is ht ontwerp conform de requirements
-hoe makkelijk is het ontwerp te repareren
-Hoe makkelijk zijn de prestaties te verifieren
Dan is er nog de aanpasbaarheid.
-hoe makkelijk is het om het product aan te passen
- hoe makkelijk is het om het product te updaten en/of uitbreiden>> expendability
- hoe makkelijk is het om een wijziging door te voeren>>flexibility
-hoe makkelijk is het om andere system aan te sluiten >> portability
- hoe makkelijk is het om het product te transporteren >> interoperability
-hoe makkelijk is het om te converteren tot een systeem gebruiksklaar voor communiceren met andere systemen>> reaseability \cite{myloloupos1992representingReqs}


\cite{jonkerTreurKlush200informativeAgents}
\cite{boehmBoseLeeRequirementsNegotiations}
\cite{muHungJinLiu2013inconsistencyReqs}
\cite{hunterNuseibeh1996manageSpecs}
\cite{myloloupos1992representingReqs}
\cite{zavePamela4darkCorners}
\cite{zavePAmela1997regEngineering}.

%%%%%%%%%%%%%%%%%%%%%%%%%%%%%%%%%%%%%%%%%%%%%%%%%%%%%%%%%%%%%%%%%

what is a good software specification

\cite{fvaandrager2322010Goodmodel}
\cite{onix01102022devopmodel}
\cite{sulemani04012021softwareprocesmodel}
\cite{globalluxsoft18102017softdev}
\cite{wiegers30052022SRS}
\cite{muller06092020goodspecification}
\cite{informit30062008reqmanagement}
\cite{altexsoft15092020writingSRS}


\paragraph{Wat is een sluis}
\cite{woudagemaalSluizen}
\cite{bardetsluizenAmsterdam}
\cite{historischesluizen}
\paragraph{Recente ontwikkelingen op het gebied van sluisautomatisering}

Het ministerie van verkeer en Waterstaat wil in het kader van het klimaatakkoord en onderzoek laten uitvoeren naar de staat van het sluizenpark in Nederland. Het onderzoek moet zich richten op het ontwerpen en ontwikkelen van een geautomatiseerd sluismodel dat geschikt is voor een brede toepassing. In het onderzoek moet naar voren komen wat de huidige staat is van de sluizen met oog op veiligheid, efficiëntie, capaciteit, onderhoud, duurzaamheid en automatisering. Het onderzoek geeft aan hoe een volledig model worden opgeleverd opdat ontwerp van verschillend volledig geautomatiseerde sluizen in de toekomst geautomatiseerd kunnen worden.  


\paragraph{Studie naar rampen aan de hand van het vier variabelen model}
\newline \indent Voor deze studie is onderzoek gedaan naar verschillende rampen aan de hand van het vier variabelen model.
Elke ramp op deze manier categoriseren  kan ons helpen te bepalen in hoeverre requirements een rol kunnen spelen in de veiligheid van ons model.  Zo is er de bijlmerramp \cite{aviationsafety04101992airplaneCrashBijlmer}
, deze vond plaats op 04/10/1994. 
Op de avond van de 4e oktober 1992 ware er bij het toetel van el al fluctuaties in de selheidsregulering, daioproblemen, fluctuaties in de voltage electriciteit van motor 3
%Motor 3 (de binnenste motor aan de rechtervleugel van het vliegtuig) brak af, beschadigde de vleugelkleppen en botste tegen motor 4 die vervolgens ook afbrak.
%De ernst van de situatie werd op Schiphol niet goed ingezien. Dit kwam onder meer doordat lost in de luchtvaart de gebruikelijke term is om het verlies van motorvermogen te melden. Op Schiphol werd er dan ook van uitgegaan dat er twee motoren waren uitgevallen. Dat ze letterlijk verloren waren wist men niet. Gezien het grote aantal handelingen dat de bemanning in een paar minuten moest uitvoeren en de keuzes die de piloot maakte, veronderstelde de parlementaire enquêtecommissie die de ramp later zou onderzoeken dat ook de bemanning waarschijnlijk niet heeft geweten dat beide motoren van de rechtervleugel waren afgebroken. De buitenste motor van een 747 is vanuit de cockpit slechts met moeite zichtbaar en de binnenste motor helemaal niet.
%Op de avond van de 4e oktober 1992 was landingsbaan 06 (de Kaagbaan) in gebruik. De piloot verzocht de luchtverkeersleiding op Schiphol echter een noodlanding te mogen maken op de Buitenveldertbaan (baan 27). Waarom hij juist deze baan koos, is nooit duidelijk geworden. Een keuze voor deze baan lag niet voor de hand; omdat de wind uit het noordoosten kwam, zou het toestel met flinke staartwind moeten landen. Langs de landingsbaan waren enkele grote brandweerwagens van Schiphol geplaatst. Deze zogeheten crashtenders moesten een brand tijdens de landing meteen blussen. Na de crash werd één zwarte doos teruggevonden. De bijbehorende band was in vier stukken gebroken, waardoor de laatste 2 minuten en 45 seconden ervan niet meer te gebruiken waren. De doos werd voor onderzoek naar Washington gestuurd en leverde uiteindelijk onderstaande informatie op.
%Om goed uit te komen voor de landingsbaan vloog het beschadigde toestel eerst nog een rondje boven Amsterdam. Tijdens dit rondje gaf de gezagvoerder de copiloot opdracht de vleugelkleppen (flaps) uit te schuiven. Links schoven de kleppen uit, maar doordat de afgebroken motor 3 de rechtervleugel had beschadigd schoven de kleppen op die vleugel niet uit. Als gevolg hiervan kreeg het toestel links meer draagvermogen dan rechts. De piloot meldde aan de verkeersleiding dat er ook problemen met de flaps waren.
%Aanvankelijk ging het aanvliegen van de Buitenveldertbaan goed. Op het moment dat het vliegtuig daalde tot onder de 1500 voet en snelheid minderde, raakte het echter compleet onbestuurbaar en maakte het een ongecontroleerde, scherpe bocht naar rechts. Over de radio was te horen dat de gezagvoerder zijn copiloot in het Hebreeuws opdracht gaf om alle kleppen in te trekken en het landingsgestel uit te klappen. Vervolgens meldde de copiloot in het Engels aan de luchtverkeersleider dat het toestel zou gaan neerstorten. Uit later onderzoek bleek dat het vliegtuig eerder enkel recht bleef vanwege de hoge snelheid (280 knopen, zijnde 519 km/u). 
Doordat de rechtervleugel beschadigd was, was het moeilijker om het vliegtuig recht te houden. Alleen de hoge snelheid zorgde ervoor dat er nog voldoende draagvermogen was. Toen bij het inzetten van de landing de snelheid verlaagd werd, werd het draagvermogen van de rechtervleugel echter dusdanig gering dat het toestel niet meer onder controle te houden was en een duikvlucht naar rechts maakte.
\cite{aviationsafety04101992airplaneCrashBijlmer}
 \cite{catsr25022009Boeing737AmsterdamCrash}
\cite{zuilen23022019Tijdlijnpoldercrash}
\cite{wikinews04032009techfoutailines1951}
\cite{luchtvaartnieuws21012020boeing737conclusies}
\cite{adformatie280220209communicatiegebreken}
\cite{spinnael25022009onderzoekpolderbaancrash}
\cite{crashTurkishAirlines}
\cite{flightradar24}
\cite{flightstatstracker}. 
%%%%%%%%%%%%%%%%%%%%%%%%%%%%%%%%%%%%%%%%%%%%%%%%%%%%%%%%%%%%%%%%%
\newline \indent Dan nog de  ramp turkisch airlines vlucht 1951 op woensdag 25 februari 2009
25 februar 2009
De automatische reactie van het toestel werd getriggerd door een fout gevoelige radio altimeter waardoor de automatische gashendel de energiemotot op actief stelde.
Inadequaat handelen van de piloten ondanks een defecte hoogtemeter en onvolledige instructies van de luchtverkeersleiding
\cite{catsr25022009Boeing737AmsterdamCrash}
\cite{zuilen23022019Tijdlijnpoldercrash}
\cite{wikinews04032009techfoutailines1951}
\cite{luchtvaartnieuws21012020boeing737conclusies}
\cite{adformatie280220209communicatiegebreken}
\cite{spinnael25022009onderzoekpolderbaancrash}
\cite{crashTurkishAirlines}
\cite{flightradar24}
\cite{flightstatstracker}
%%%%%%%%%%%%%%%%%%%%%%%%%%%%%%%%%%%%%%%%%%%%%%%%%%%%%%%%%%%%%%%%%
\newline \indent De therac-25. In de periode van Juni 1985 and Januarie 1987 zijn er meerdere ongelukken met dodelijka afloop door de implementatie van de Therac-25 bij de behandelig van huidkanker.
De therac-25 is een Medische lineaire versneller. Deze  versnellen elektronen om stralen met hoge energie te creëren die tumoren kunnen vernietigen met minimale impact op het omliggende gezonde weefsel.
Onderzoekers constateren dat er fouten zijn gemaakt tijdens de (her-)implementatie van systemen uit eeerdere productiemodellen.


Yakima Valley Memorial Hospital in 1985
Manufactureere response
Government and user response
East Texas Cncer Center, March 1986
Manufactureere response
Government and user response
East Texas Cncer Center, April 1986
Manufactureere response
Government and user response
Yakima Valley Memorial Hospital

Onderzoekers zijn van mening dat de terkortkokmigen in medische apparatuur niet geheel en altijd te verwijten zijn aan softwareproblemen. Zor is er ook een rol wegglegd voor Manufactureere response
Government and user response.

Medische lineaire versnellers versnellen elektronen om stralen met hoge energie te creëren die tumoren kunnen vernietigen met minimale impact op het omliggende gezonde weefsel.

Manufactureere response
Government and user response
Yakima Valley Memorial Hospital in 1985
Manufactureere response
Government and user response
East Texas Cncer Center, March 1986
Manufactureere response
Government and user response
East Texas Cncer Center, April 1986
Manufactureere response
Government and user response
Yakima Valley Memorial Hospital

Manufactureere response
Government and user response


Veel fouten in safety-critical systeem. In geval van therac spreken we van een systeemongeluk, complexe interacties tusse verschillende componnten en activiteiten. In het artikel woden 6 ongevallen omschreven.
In het eerste geval is het neit helemaal duidelijk wat er is gebeurd.
In het tweede geval is er sprake van onvolmaakte microswitchtes welke	 een ambigu bericht produceert voor de computer.
In het derde geval zijn er open slots in de blocking trays.
In het Vierde geval  heeft de operator verkeerde prescriptie-data ingevoerd. De opertor drukt op return en bevestigd alsnog de invoer. Op een gegeven moment komt er een foutmelding "malfunction 54". De opertor was bekend met de machine en drulte op de knop "p" van proceed.
In het vijfde geval was er een verkeerde invoer voorschift data waardoor verkeerde toets werd gedrukt. Na aanpassen werd de return-toets ingedrukt. Er rad een fout op met de melding "malfunction 54" Na reproductie van de melding bleek dat de ionische amer neit gezouten bleek te zijn
In het zesde geval vergat de operator de files te verwijderen onder de patent. Er werd straling geenten maar de console gaf aan dat er geen dosisratio was gemeten. De operator drukte op de knop "p" om het proces te pauzeren.
Gerelateerde theac-20 problemen
Terwijl de therac 20 afhankelijk was van mechanische vergrendelingen werd er bij de therac-25 software gebruikt

Software problemen zijn onder andere
-slechte software engineering/designing praktijken
-er is een machine gebouw dat afhankelijk is van software voor veilgheidsoperaties
= de fout in de code is niet zo belangrijk als een geheel onveilig ontwer

cleaning the  bendory-magnet variable nistead of at the end of the frame
race conditioning to indicate prescription entry is still in progress


user response
- poor screen refresh subroutines that left trash and erroneous information on the oeprating console
- tape loading problems upon startup whwre dscouraed icluded the use of photom tables to trigger the interlock system in the event of a  load faul instead of a checksum

Gebruikersgroepen klagen over het tekort aan aoftware-evaluaties en 2) een tekort aan hard-copy audit trials om foutmeldinen in beeld te krijgen


%
%
%Medical lineair accelerators accelerate electrons to createhighenergy beams that can destroy tumors with minimal impact on the surrounding healthy tissue.
%In the mid-1970s, AECL, developed a radical new "double-pass" concept for electron acceleration. A double passaccelerator needs much less spaceto develop comparableenergy levels because it folds the long  physical mechanismrequired to accelerate the electros, and it is more economic to produce.
%Using this double pass concept AECL designed the  Therac-25, a dual mode lineair acelerator that can deliver either photonsat 25 MeVor electrons at various energy levels. Compared with theTerac-20 The Thrac-25 is notably more compact,, more versatile, and arguably easier to use. 
%The higejr energy takes advantage of the phenomenon "depth dose": As the energy increases, the depth in the body at which maximum dose buildup occurs alse increases, sparing the tissue above the target area.
%First, like the Therac-6 and the Therac-20, the Therac25 is conrolled by a PDP11. The Terac-6and Therac-20 had been designed around machines that already had histories of clinical use without computer control.
%The therac-20 has idependent protective circuits for monitoring electron-beam scanning, plus mechanical interlocks for policing the machine and ensuring safe operation.
%Finally some software for the machines was interrlatd or reused.
%Eleven therac-25 were installed: five in the usand six in canada. Six accidents involving massive oerdoses to patients occured between 1985 and 1987. The machine was recalled in 1987 for extensive design changes, including hardware	 safeguards against errors.
%Kennestone Regional Oncology Center 1985
%Door rechtzaken waren managegers op de hoogte van de problemen en ongelukken. Maar er werd in het vervolg niet over gerapporteerd.
%The treatment prescription printout failure was disabled at the time of the accident , so there was no hardcopyof the treatment data.
%Ontario Cancer Foundation in 1985
%Since the machine did not suspendand the control display indicated no dose was delivered to the patient, the operator went ahead with a second attempt at trratment by pressing the "P" key, expecting the machine to deliver the proper dose this time. This was standard operating procedure and, described in the "The operating interface" on p 24, Therac 25
%oprators had become accustomed to freunt malfunctions that had no untowardconsequences for the patient. Again, the machine shut downin the same manner. The oeprator repeated this process four times after the original attempt- the display showing "no dose" delivered to the patient each time. After th fifth pause, the machine went into treatment suspedn, and a  hospital service technician was called.
%The technician found nothing wrong with the machine. This was not an unusual scenario, according to the Therac-26 operator
%Manufactureere response
%Government and user response
%Yakima Valley Memorial Hospital in 1985
%Manufactureere response
%Government and user response
%East Texas Cncer Center, March 1986
%Manufactureere response
%Government and user response
%East Texas Cncer Center, April 1986
%Manufactureere response
%Government and user response
%Yakima Valley Memorial Hospital
%Manufactureere response
%Government and user response
%Softwarefout uit zich als hardwarefout de klachtafhandeling geen onderzoek geen second opinion is prioriteit wel 
%gechecked na onderzoek bellen en geen prioriteit aanwezig te zijn alleen importeurs en fabriken mogen fouten 
%in frabrieksinstellingen rapporteren 
%Therac25 Systeem ligt plat veel voorkomende eror stdaardafhandeling om de error te verwerpen resultaat: 
%de patient kreeg overdosis patient overleden onderzoek opgestart, stuatie niet reproduceerbar foutmarkering: 
%gezien als uitzonderlijk, software aanpassing van groote magnitude 5; de oorzaak was waarschijlijk mechanisch 
%maar neit vastgesteld; conceptueel odel niet aangepast probleemclassicificatie door autorititen het probleem 
%en de impact daarvan anar beneden bijgesteld AEFL doe gedeeltelijke aanpassing om hardware na berisping 
%Canadese autoriteit 
%Derde patient overleden door eythema AECL wijst alle doodsoorzaken af AECL beweert dat geen vergeli- 
%jkbare voorvalle bij andere machines of patienten zijn voorgekomen geen vervolgonderzoek vanwege garanties 
%bedrijf gaat uit van geen mogelijke functionele fout 
%vierde patient overleden aan overdodis ontstaan door bug in software onjuiste aanduiding bij de foutmelding 
%verkeerde reactie/invoer ddoor operator communicatie tussen patient en operator werd onvoldoende gemon- 
%itorred ( apparatuur niet aangesloten, en audio monitor kapot) engineer van AECL stelt geen fouten vast 
%Engineer AECl kan fout niet reproduceren Geen communicate tussen bedrijf en uitgezonden technisci over 
%vergelijkbare probleemgevallen 
%vijfde geval malfunction 54 leidt tot overdosis en de dood fout gereproduceerd door operator bedrijf fout 
%was daa entryspeed herpublicatie van de ongevallen en de eerdere ongevallen in de meia apparaat wel nog in 
%gebruik genomen niet handig, waarschuwingsberichten en aanwijzingen voor een bugfix naar de gebruikers door 
%druk van fda is bedrijf op zoek gegaan naar permanente oplossing 
%zesde geval software fout door softwarefout otntstaat lightstruct .. op de patient na onderzoek door AECL 
%blijkt niet alleen hardware de oorzak gebruikers direct geinformeerd oplossing gevonden, media ingeschakeld om transparantie af te dwingen door de gebruikersgroep en de FDA AECL gedwongen functionaliteit aan te passen 
%Engineers hebben meer studie moeten maken van gebruikte technologie en onderhoudbaarheid daarvan 
%sheets
%\cite{rogaway2004therac25}
%~\cite{wikiTherac25}
%reproduceren van de error. IN dit stuk wordt uitgelgd hoe het product werkt en waarom bepaalde beslssingen zijn genomen in de ontwerp/productiefase
%\cite{lynch2017theracRaceConditions}
%kort artikel met daarin een opsomming van alle fouten in het systeem en een korte uitleg
%\cite{lim1998theracdisaster}
%uitgebreid artikel over hoe de fout werd gereproduceerd en de resultaten daaruit voortkwamen. Alsnog werden er na de reproductie fase nog meer fouten gevonden.
%\cite{fabio26102015therac25}
%artikel
%\cite{ethicsunwrappedTherac25}
%onderzoeksartikel waarin de bug wordt uitgelgd: de racecondities, de bytepositie en het testen worden berkitiseerd envenals andere onderdelen van het softwareproces
%onrealistisch testplan. In dit artikel egt de auteur het belang nog eens uit van goede requirements en implementatie, niet de software is waar het probleem ligt
%geschiedenis
%\cite{casesHistoryTherac25}
%artikel
%\cite{caballero2019Therac25}
%computer error. De ongeval en de malfunction nog een keer uitgelegd
%\cite{rose1994theracFatalDose}
%rapport
%\cite{levesonMITTherac25}
%\cite{grant1978theracevaluation}
%onderzoeksartkel
%\cite{turnerTheracAccidentsInvestigations}
%\cite{turner1993TheracAccidentsInvestigations}
%uitgebreid artikel gaat hier ook wat meer over de hardware
%\cite{wang2017industrialdesignengineering}
%artikel waarin in 3 delen de problemaiekwordt blootgesteld
%\cite{levesonturner1993theracpart2}
%case study sheets
%artikel waarin vooral de fabriikant ervan langs krijgt
%\cite{porelloTheraccFailure}
%lessons learned. Vooral de begrippen betrouwbaarheid, welgevalligheid, veilgheid en gebruiksvriendelijkheid
%\cite{theracIncidents}
%root-cause analysis
%case study
%\cite{huffbrown2004casestudyethicatherac}
%case study
%\cite{sebowikimedicalradiation}
%opzetten van systematische acceptaatie test met therac als voorbeeld
%\cite{hsia1995testtherac25}
%artikel waarin een diagnose plaatvindt voor het bedrijf en de ingenieur/ontwerper
%\cite{magsilvaTheracTesting}
%rapport
%oorzaken aangegeven in artikel
%\cite{chemeuropetherac25}
%het onderzoek en enkele ontwerptekeningen en oplossingen
%\cite{statsenko10102016Therackillerbug}
%\cite{therac25casestudy}
%\cite{thomas1994theracinLotos},
%\cite{twitter2019programmerbehindtherac}
%wiki
%\cite{wikibookstherac}
%analyse
%\cite{bozdagTherac25}
%samenvatting
\cite{levesonTurnerTheracAbstract}
%rapport over de fouten die de verschillende partijen hebben gemaakt( overheid, ingenieurs, bedrijf, operators) en de verbeterpunten
%onderzoeksrapport
%slides online over het technisch mankement
%Wat is er gebeurd, nou het volgende:
%Normal radiation treatments: 6,000 rads over a 3 week period, under certain conditions Therac-25 was delivering 60,000 rads during one session.
%En wat ging er mis?
%Paradigm Shift
%Therac-25 replaced expensive hardware safety interlocks with software controls
%Real-time software
%Design
%Race condition caused focusing element to be incorrectly set
%No indication of actual hardware settings
%Error messages appeared the same regardless of how important
%Error messages were difficult to understand
%All errors messages could be manually overridden
%oorzaak-gevolg diagram
%veiligheidsanalyse naar de rapportage van foutmeldingen, de beslissingsmatrix waarmee het programma wordt uitgevoerd en de software-analyse door een consultat
\cite{stackexchange2021therac25code}
\cite{rogaway2004therac25},
\cite{wikiTherac25}, 
\cite{lynch2017theracRaceConditions},	\cite{lim1998theracdisaster}, 
\cite{fabio26102015therac25},	 	\cite{ethicsunwrappedTherac25}, 	\cite{casesHistoryTherac25},	 	\cite{caballero2019Therac25}, 	\cite{rose1994theracFatalDose}, 	\cite{levesonMITTherac25},
\cite{grant1978theracevaluation},	 	\cite{turnerTheracAccidentsInvestigations},	\cite{turner1993TheracAccidentsInvestigations}, 	\cite{wang2017industrialdesignengineering}, 	\cite{levesonturner1993theracpart2},	\cite{porelloTheraccFailure},\cite{theracIncidents}, 
\cite{huffbrown2004casestudyethicatherac}, 
\cite{sebowikimedicalradiation},	\cite{hsia1995testtherac25},	\cite{magsilvaTheracTesting},
\cite{chemeuropetherac25},	\cite{statsenko10102016Therackillerbug},	\cite{therac25casestudy},	\cite{thomas1994theracinLotos},	\cite{twitter2019programmerbehindtherac},	\cite{wikibookstherac}, 
\cite{bozdagTherac25},	\cite{levesonTurnerTheracAbstract}, 	\cite{stackexchange2021therac25code}.
%%%%%%%%%%%%%%%%%%%%%%%%%%%%%%%%%%%%%%%%%%%%%%%%%%%%%%%%%%%%%%%%%
\newline \indent
%Hoe werkt het
tesla autopilot features voor dataverzameling\cite{denneyjdsupraFeds},\cite{gritti24062020tesladataengine}.
% crashes
 De eerste tesla crash is van juni 2016 \cite{impakterTeslaCrash}. En meerdere zouden volgen.
Een ongeluk in  de VS waarbij 2 inzittenden om het leven kwamen. Een persoon had plaats genomen als bijrijder en de andere persoon als passagier achter de stoel van de bestuurder. Waarschijnlijk was de autopiloot niet ingeschakeld.
\cite{anderson30042021secondteslacrash},\cite{raynal20042021probeTeslaCrash},\cite{firstpress11052021fatalnonautopilot},\cite{cochran18042021nodriverTeslaCrash},\cite{gitlin11052021autopilot},\cite{sommerfield12072021NHTSAmandateresult},\cite{hawkins30062021nhtsarequiresreporting},\cite{wilson19042021teslacrashregulators},\cite{mcfarland22042021selfdrivingrisks}
De situatie en oorzaken zijn bij elke ramp verschillend. 
Een automobilist heeft in een rit van 37 minuten slechts 25 seconden zijn handen aan het suur gehad ondanks de melding "Hands requireed not detected". Hiermee zijn de onderzoekers van de NTSB ervan uitgegaan dat de bestuurder de autopiloot bewschouwde als een volledig autonooom rijsyssteem in plaatst van een veligheidsmechanisme
\cite{oremus21062017fatalTeslaCrash}. Of in 
Mei 2015 als een besuurde foto's van zichzelf maakt in de testla zonder handen aan het stuur of voeten op het pedaal.
\cite{guardian15052021teslacrashHandsOnWheel}
Een faatale crash in 2016 waarbij de bestuurder  e veel vertrouwde op het semi-autonome rijtechnologie op het verkeerde type wegdek.
\cite{Puzzanghera13092017TeslaSharesBlame}
Onderzoek naar een fatale crash op 7 mei 2016 toont aan dat er beperkingen zitten aan de autopilot mode. Om specifiek te zijin is de automatische noodrem niet failsafe, blijkt uit onderzoek.
\cite{jaillet02022017teslaAutopilotLimitations}
\cite{reuters03102019teslaAutoParkingFail}
\cite{dowling23042021}
Op  April 17 2019 een autocrash waarbij het onduidelijk is of de autopiloot aan stond.
\cite{young05112021fatalTeslaReport}. Een auto ongelu waarbij een tesla is betrokken. De bestuurder was waarschijnlijk afgeleid door de games op zijn apple telefoon. De NTSB gaf aan dat het crash-avoidance systeem neit otnworpen is en ook geen crash atnuaor heeft gedetecteerd. Herdoor accelereerde de autopilot  het voertuig. Ook Faalde het systeem in het verschaffen van een crash aleter en werden de noodremmen niet geactiveerd.
\cite{tiungteslasoftwarecrash}
Er is ook een melding van een tesla waarvan de autopilot bots tegen een stilstaande politieauto
\cite{kierstein18032021teslaAutopilotCrashStationary}. Ook uit dit onderzoek blijkt dat er geen gebreken waren en dat het automaische remsysteem neit kapot was. De HNTSA concludeerder dat de bestuurder zelf geen actie ondernam door  bij te sturen of te remmen. In een eerder artikel kwam naar voren dat de tesla een autopilot krijgt die enkel camera's en GPS gebruikt; lidar of een radarsysteem wordt niet toegepast.
\cite{janssen20062017teslacrashdetailflorida}
Enkele fotos van crashes met autonome rijsysstemen \cite{saferoardsCrashesAutonomousvehicles}.
\cite{stephardson18032021revieuwingtesla}
%Onderzoeksrapport naar testla automatic vehicle control system
\cite{habib28062016NHTSATeslaReport},
\cite{darkReading17112020TeslaBackup},
\cite{heilweil26022020teslaAutopilot}
% overzicht
Tesla autopilot crashes met meer crashes en incidenten dan tot dan toe gerapporteerd
\cite{teslaFDSCrash}
De meest voorkomende crashes zijn stationaire objecten bij hoge snelheden, lane incursions from stationary objects, auti=opilot confusion at forks and gores.
\cite{teslaCrashesCauses}
\cite{teslacrashOvervieuw}
\cite{tesladeaths}
% veiligheidsrisico''
De veiligheidsrisicos van de tesla lopen uiteen. Zo zijn er risicos in de machinelearning technologie:
veiigheidsrisico Three Small Stickers in Intersection Can Cause Tesla Autopilot to Swerve Into Wrong Lane
\cite{evan01042019teslaautopilotIntersection},
\cite{lambert31062020q2safetyreport},de autopilot zelf
\cite{templeton06092019HTSBReportTesla}. Een studie door de consumntenbond in de VS toont aan dat hetautopilot systeem van de testla niet failsafe is. Zo zijn de sensoren, gebrukt voor detectie van een bestuurder negatief te beinvloeden.
\cite{dowling23042021autopilottricking} Maar ook andere problemen met de bluetooth 
\cite{wiredBloutoothHackTesla}, touch screen
\cite{preston14012021NHTSATeslaRecall},
Web-based attack crashes Tesla driver interface
\cite{leyden23032020TeslaInterfaceHack}.
Of zelfds de tesla batterij is veiligheidsvraagstuk geworden
\cite{mitchell01072020teslabatterycooling}.
Maar ook was een onderzoeker  was in staat om persoonlijke details van afgedankte voertuigonderdelen  te vekrijgen nadat deze waren afgekeurd vanwege upgrades en reparaties op consumentenvoertuigen.
\cite{stumpff04052020TeslaPersonalData}
Data-opslag in de cloud niet altijd bereikbaar.
\cite{mitchell24022020AIDataTesla}
%Wat er mis zou kunnen gegeaan wordt dru over gespeculeerd online.
%\cite{stackexchange102019teslacarmistake}
dodelijk ongeluk
\cite{fottrell03092018TeslaSecurityChecks},
softwarefout maakt diestal mogelijk
\cite{kirk26112020modelX}
fouten ontdekt in onderzoek
\cite{bbc24022021hyundaiBatteryFireFix},
tesla cloud gehacked
\cite{hawkins22102022}.
This analysis considers the potential impacts of completely self-driving vehicles on vehicular liability. 
\cite{griemannExaminSelfDriving}
Dan zijn er nog maatschappelijke problemen die de aanpak moeilijker maken.
Er is in de vs in verschillende staten een andere wetgeving
\cite{berry21042021teslacrashtexas}
\cite{hull23072021regulatorsaftercrash}
\cite{wikiTeslaAutopilot}
%oplossingen
Toch zijn er oplossingen en tegenmaatregelen.
tesla gaat advanced driver assistance systems inzetten met behulp van  passive visual, ultrasonic, en radar.
\cite{tasking07062017TeslaAugmentedSafety},\cite{ackerman01072016TeslaImperfect}
Safe system solutions door David Harkey
\cite{Harkey30052019SafeSystemVehicle}
%maatregelen
Voor elke auto uitgerust met een level 2 tot level 5 autonomy wordt nu standaard een rapport van van de crash opgvraagd door de NTSA. Dit in het kader van verder onderzoek waarbij de autoritait kijk naar  ziekenhuisbehandeling, fataliteit, airbag deployment.
\cite{szymkowski29062021nhtsaTeslaCrashReports}. 
Door een softwarefout zijn er situaties ontstaan waarin het systeem informatie een onvoldoende informatie positie had om de juiste beslissingen te maken. Of dat de informatieverwerking niet juist was.
\cite{teslaFDSCrash}
\cite{teslaCrashesCauses}
\cite{teslacrashOvervieuw}
\cite{tesladeaths}
veiigheidsrisico
\cite{evan01042019teslaautopilotIntersection}
\cite{testVehicleSafetyReport}
veiligheidsrapport mbt autopilot
\cite{lambert31062020q2safetyreport}
consumentenrapport
bluetooth veiligheidsvraagstuk
\cite{wiredBloutoothHackTesla}
veiigheidsvraagstuk vanwege touch screen
\cite{preston14012021NHTSATeslaRecall}
veiligheidsvraagstuk
\cite{cio25112020belgianTeslaHack}
veiligheidsvraagstuk
rapport over autopilot
\cite{templeton06092019HTSBReportTesla}
de invloed van de bestuurder bij tesla ongeluk
veiligheidsvraagstuk
\cite{darkReading17112020TeslaBackup}
veiligheidsvraagstuk
\cite{leyden23032020TeslaInterfaceHack}
veiigheidsvraagstuk
\cite{huddlestonjr03042019ChineseTeslaHack}
veiligheidsvraagstuk
veiligheidsvraagstuk
\cite{heilweil26022020teslaAutopilot}
rapport over ongeluk
veiligheidsvraagstuk
veiligheidsvraagstuk
\cite{blanco04102019NHTSATesla}
veiligheidsvraagstuk
ransomware aanval op tesla
tesla batterij is veiligheidsvraagstuk geworden
\cite{mitchell01072020teslabatterycooling}
ongeluk
\cite{bbc26022020AutopilotCrash}
veiligheidsvraagstuk
veiligheidsvraagstuk
\cite{stumpff04052020TeslaPersonalData}
dodelijk ongeluk
\cite{levin08062018teslaautopilotsafety}
veiligheidsvraagstuk: ransomware
veiligheidsvraagstuk: medewerker in de fout
\cite{cbrook06082021TeslaInsideDataThreft}
\cite{shilling25022021Tesla}
veiligheidsvraagstuk: hackers je systeem laten testen
verdedigen tegenover ransomware
veiligheidsrisico
prijzen omlaag
autopilot
\cite{randall05112019modelSurvey}
malware door een medewerker
dodelijk ongeluk
\cite{fottrell03092018TeslaSecurityChecks}
waarom een tesla stelen bijna onmogelijk is
veiligheidsonderzoek
softwarefout maakt diestal mogelijk
\cite{kirk26112020modelX}
fouten ontdekt in onderzoek
\cite{bbc24022021hyundaiBatteryFireFix}
tesla cloud gehacked
\cite{hawkins22102022}
\cite{gritti24062020tesladataengine}
\cite{bouchard07052019teslaDeepLearning}
\cite{Srikanth2019teslabigdata}
\cite{rangaiah25022020teslaAI}
\cite{marr08012018taslabigdataAI}
\cite{bdickson29072020teslalevelfive}
\cite{dcruz17062022tesladesignthink}
\cite{mcfarland22042021selfdrivingrisks}
\cite{hawkins18032021fedgovinvest}
\cite{berry21042021teslacrashtexas}
\cite{hull23072021regulatorsaftercrash}
\cite{wikiTeslaAutopilot}
\cite{nhtsaAutomatedVehiclesSafety}
\cite{dowling23042021autopilottricking}
\cite{wilson19042021teslacrashregulators}
\cite{seamans22062021aikillerap}
\cite{mitchell24022020AIDataTesla}
\cite{denneyjdsupraFeds}
\cite{siddiqui22102020TeslaCriticism}
\cite{ackerman01072016TeslaImperfect}
\cite{greene04092019misuseautopilot}
\cite{michralli26112019ubserautocarcrsash}
\cite{pitmann21072021wrongfullautodeath}
\cite{stackexchange102019teslacarmistake}
\cite{tasking07062017TeslaAugmentedSafety}
\cite{griemannExaminSelfDriving}
\cite{Harkey30052019SafeSystemVehicle}
tesla crash report
\cite{shepardson18062021TeslaDeaths}
\cite{hawkins30062021nhtsarequiresreporting}
\cite{hawkins10052021autopilotnotavailable}
\cite{szymkowski29062021nhtsaTeslaCrashReports}
\cite{abc1112052021AutopilotNotinTeslaCrash}
\cite{ankel18062021regulatorsinvestigateAutopilot}
\cite{sommerfield12072021NHTSAmandateresult}
\cite{saferoardsCrashesAutonomousvehicles}
\cite{stephardson18032021revieuwingtesla}
\cite{krishner30062021NHTSAreport}
\cite{gitlin11052021autopilot}
\cite{mitchell19012017investigationstop}
\cite{gordon10052021teslaprelimreport}
\cite{shaper07062018}
\cite{cochran18042021nodriverTeslaCrash}
\cite{habib28062016NHTSATeslaReport}
\cite{firstpress11052021fatalnonautopilot}
\cite{raynal20042021probeTeslaCrash}
\cite{tiungteslasoftwarecrash}
\cite{globaltimes08052021guangdongcrash}
\cite{anderson30042021secondteslacrash}
\cite{oremus21062017fatalTeslaCrash}
\cite{guardian15052021teslacrashHandsOnWheel}
\cite{Puzzanghera13092017TeslaSharesBlame}
\cite{jaillet02022017teslaAutopilotLimitations}
\cite{reuters03102019teslaAutoParkingFail}
\cite{dowling23042021}
\cite{young05112021fatalTeslaReport}
\cite{kierstein18032021teslaAutopilotCrashStationary}
\cite{janssen20062017teslacrashdetailflorida}
%%%%%%%%%%%%%%%%%%%%%%%%%%%%%%%%%%%%%%%%%%%%%%%%%%%%%%%%%%%%%%%%%
\newline \indent De slmramp op  07/06/1989.
Toen de Anthony Nesty Zanderij naderde, was het daar, anders dan het weerbericht had voorspeld, mistig. Het zicht was evenwel niet zo slecht dat er niet op zicht kon worden geland. Gezagvoerder Will Rogers besloot echter via het Instrument Landing System (ILS) te landen, hoewel dit niet betrouwbaar was en hij voor zo'n landing ook geen toestemming had. De gezagvoerder brak drie landingspogingen af. Bij de vierde poging negeerde de bemanning de automatische waarschuwing (GPWS) dat het toestel te laag vloog. Het toestel raakte op 25 meter hoogte twee bomen. Het rolde om de lengteas en stortte om 04.27 uur plaatselijke tijd ondersteboven neer.
Uit onderzoek bleek dat de papieren van de bemanning niet in orde ware door nalatigheid in de crew-member screening
Geconcludeerd werd dat de gezagvoerder roekeloos had gehandeld door voor een ILS-landing te kiezen terwijl hij daar geen toestemming voor had, en door onvoldoende op de vlieghoogte te hebben gelet. 
De SLM werd verweten de kwalificaties van de bemanning onvoldoende te hebben gecontroleerd.
Oorzaak: het roekeloos besturen door de kapitein onder de minimum hoogte leidde tot collissie met een boom.
\cite{espnSLMterugblik},\cite{dennisRosier01052020}
\cite{hassing07062020slmramp},\cite{amsterdamArchiefSLM},\cite{rtvOost06062019nabestaande},
\cite{breda07062021AndroSnel},\cite{andereTijdenSLMCrash},
\cite{aviationReport},\cite{aviationSLMCrashAccidentInvestigation},\cite{mcDonnelDouglasCommissionReportSLMCrash},
\cite{wikiSRFlight764},\cite{nos07062019SLMTerugblik},\cite{dagvantoenSLMCrash},\cite{waterkantNesty07061989},\cite{eduNandlalSRCrash},\cite{oldjetsSRAirways},\cite{cloudberg02012021srflight764},\cite{apnews07061989srplanecrash}.
%%%%%%%%%%%%%%%%%%%%%%%%%%%%%%%%%%%%%%%%%%%%%%%%%%%%%%%%%%%%%%%%%
\newline \indent De schipholbrand op 27/10/2005\cite{schipholbrand27102005video},\cite{schipholbrand27102005video},\cite{onderzoeksraad2610schipholoost},
\cite{schipholbrandvideoargos},\cite{nunl30052023feitenoverzicht},\cite{parlementairemonitorschipholbrand},\cite{videonpoNOVA13112008},\cite{rizoomes01052014schipholbrand},\cite{heuvelkroesschipholbrandcamerabeelden},
\cite{wikiSchipholbrand},\cite{schipholbrand27102005video},\cite{onderzoeksraad2610schipholoost},\cite{schipholbrandvideoargos},\cite{nunl30052023feitenoverzicht},\cite{singeluitgeverijenSchipholbrand},\cite{eenvandaagschipholbrand},\cite{parlementairemonitorschipholbrand},
\cite{videonpoNOVA13112008},\cite{rizoomes01052014schipholbrand},\cite{heuvelkroesschipholbrandcamerabeelden}. 
27/10/2005
11 doden onder mirgranten in de cellenclomplexen van schiphol-oost. Doodsoorzaak van de slachtffoffers isverstikking. Het gebrouw voldeed niet aan de eisen voor brandveiligheid, personeel was niet goed getraind voor dergelijke situaties en de hulpverlening kwam door verschillende factoren te laat op gang.
%
%Wat is er gebeurd?
%\cite{schipholbrand27102005video}
%artikel
%\cite{schipholbrand27102005video}
%psychologische gevolgen
%rapport
%\cite{onderzoeksraad2610schipholoost}
%artikel met video
%herdenking
%impact op de persoon
%herdenking
%\cite{schipholbrandvideoargos}
%chronologie
%\cite{nunl30052023feitenoverzicht}
%tijdlijn
%vervolgens van ministers
%beeldanalyse en reconstructie
%\cite{}
%herdenking
%korte samenvatting
%rapport
%artikel
%verwijzing naar het rapport vanuit de politieke oppositie
%beeld vanuit de gevangenisbewaarder
%nationaliteit slachtoffers schipholbrand
%verblijfsvergunning voor de slachtoffers
%gen schadevergoeding voor de verdachte
%verdachte voor de rechter
%geen schadevergoeding voor verdachte
%artikel wat ging er mis bji de schipholbrand
%brand veroorzaakt door een peuk
%smaadschrift
%bewakers worden niet vervolgd
%proces schipholbrand moet over en de brandveilgheid moet worden verbeterd
%de rol van het parlement in de evaluatie
%\cite{parlementairemonitorschipholbrand}
%onderzoeksmemo
%herdenking
%herdenking
%invloed van de ramp op samenleving
%\cite{videonpoNOVA13112008}
%opmerkelijk rapport gestolen in de nasleep
%\cite{rizoomes01052014schipholbrand}
%publicaties
%\cite{heuvelkroesschipholbrandcamerabeelden}
%Wat waren de regels destijds?
%Waren de autoriteiten in staat om op tijd in te grijpen of om erger te voorkomen?
%Wat is er gedaan om de veiligheid van illegalen en gevangenissbewaarders te verbeteren
%Wat is er gebeurd?
%\cite{wikiSchipholbrand},\cite{schipholbrand27102005video}
%psychologische gevolgen
%rapport
%\cite{onderzoeksraad2610schipholoost}
%artikel met video
%herdenking
%impact op de persoon
%herdenking
%\cite{schipholbrandvideoargos}
%chronologie
%\cite{nunl30052023feitenoverzicht}
%tijdlijn
%\cite{singeluitgeverijenSchipholbrand}
%vervolgens van ministers
%beeldanalyse en reconstructie
%\cite{eenvandaagschipholbrand}
%herdenking
%korte samenvatting
%rapport
%artikel
%verwijzing naar het rapport vanuit de politieke oppositie
%beeld vanuit de gevangenisbewaarder
%nationaliteit slachtoffers schipholbrand
%verblijfsvergunning voor de slachtoffers
%gen schadevergoeding voor de verdachte
%verdachte voor de rechter
%geen schadevergoeding voor verdachte
%artikel wat ging er mis bji de schipholbrand
%brand veroorzaakt door een peuk
%smaadschrift
%bewakers worden niet vervolgd
%proces schipholbrand moet over en de brandveilgheid moet worden verbeterd
%de rol van het parlement in de evaluatie
%\cite{parlementairemonitorschipholbrand}
%onderzoeksmemo
%herdenking
%herdenking
%invloed van de ramp op samenleving
%\cite{videonpoNOVA13112008}
%opmerkelijk rapport gestolen in de nasleep
%\cite{rizoomes01052014schipholbrand}
%publicaties
%\cite{heuvelkroesschipholbrandcamerabeelden}
%Wat waren de regels destijds?
%Waren de autoriteiten in staat om op tijd in te grijpen of om erger te voorkomen?
%Wat is er gedaan om de veiligheid van illegalen en gevangenissbewaarders te verbeteren
%%%%%%%%%%%%%%%%%%%%%%%%%%%%%%%%%%%%%%%%%%%%%%%%%%%%%%%%%%%%%%%%%
\newline \indent De explosie tanjin china 12/08/2015. 
Op 12 augustus 2015. Er waren twee explosies bij de Rulthai logistiek  faciliteit zorgde voor de opslag vn  gevaarlijke stoffen. De explosie zorgde voor de vernietiging van 12000 voertuigen, schade aan 17000 huize binnen een traal van 1 km. Er waren 173 doden inclusief brandweermensen.
Een van de explosies zorgde voor  een beving van 2.3 op de schaal van rigter.
De volgende factoren zouden een rol hebben gepeeld:
Een onjuiste afbakening van het opslagmaeriaal
Er was  weinig kennis bij de autoriteiten over  opslagmaterialen. Zo bleek er 7000 ton aan materiaal opgeslagen, dat is ruim 70 keer te maximaal toegestande hoeveelheid. 
Onverenigbaar grondgebruik in de nabije omgeving. Veel woonwijken met nar schatting 6000000 bewoners en 500 lokale bedrijvenin de buurt van de opslag gevaarlijke stoffen.
Opgeslagen materialen  waren: calcium carbine, sodium nitraat, potassium nitraat, amminiak nitraat en cyanide.
Ook is er veel kritiek geweest op de acties van de autoriteiten. Zo was er censuur vanuit de overheid op de journalistiek.
Ook was er naar alle warschijnlijkheid sprake van corruptie. Zo bleek achteraf dat een van de grootste aandeelhouders Dong Shexuang de zoon te zijn van een oud-politiechef in Tanjin haven, genaamd Dong Pijun
De overheid beloofde strengere toezicht en alle bedrijven moeten een risico-inventariatie maken en onderhouden\cite{jiang16042019TanjinExplosion},
\cite{staff31082015tanjinblastunrevealed},\cite{chinafile18082015tanjinexplosion},
\cite{pinghuang2410201TanjinFactreport},\cite{portoTanjinExplosionSight},\cite{imago17082015TanjinApartmentImages},\cite{trager14082015Chemicalblast},\cite{pangeramo27082015TanjinExplosion},\cite{ap06082020ammaniumnitrate},
\cite{morris14082015TanjinIndustryImpact},\cite{milesyu20082015exposingtoxicgovlines},\cite{artemis30032016tanjininsurance},\cite{aidenxiatanjinblast},
\cite{danwangTanjinflexreport},\cite{keyHighlightsTanjin},\cite{hartley13082015videofootage},\cite{odonnel01062017firetanjinblast2015},
\cite{fan15082015newyorkermistrustchina},\cite{yanlidongchinamediaframingTanjin},\cite{evans27092017TnjinInsurance},\cite{jasi26032019chineschemplant},\cite{shiqingTanjinExecutiveSentence},\cite{sophiebeach15082015},\cite{hamzeh05082020BeirutBlast},\cite{chemwatch18082015TanjiinExplosion},
\cite{thehindu15062019chinaExplosion},\cite{santagotimes24032019chinablast},
\cite{klingecorp28042020causedTanjin},\cite{mcgarryExplosions2017},\cite{roswnfeld13082015TanjinReports},
\cite{aria12082015explosionaTanjin},\cite{tremblay11022016chineseInvestigatorsTanjin},\cite{taylor13082015TanjinExplosianAftermath},
\cite{associatedPresss13082013},\cite{un20082015InvestigationTanjin},\cite{france2412082015TnjinExplosion},\cite{npr14082015TanjinCause},\cite{bbc05022016TanjinResponsibles},\cite{CBodeen15082015TanjinExplosion},\cite{reutersTanjinInsurance},\cite{yu082016evaluationTanjin2015},\cite{wiki2015TanjinExplosions},\cite{bbc17082015whathappenedTanjin},
\cite{mortimer19082016taijinexplosioncrater},\cite{internationallabourofficeChmControlTooliit},\cite{euTaxationCustomsICSC},
\cite{iloWHOChemSafetyCards}.
Later bleek uit een onderzoek van de Chinese autoriteiten dat de explosie overeenkwam met de ontploffing van 450 ton TNT.[6] 
De oorzaak van de explosie lag in de spontane zelfontbranding van 207 ton cellulosenitraat dat in containers was opgeslagen op het terminalterrein.[6] 
Verder lag op een tweede locatie nog eens 26 ton van dit explosieve materiaal opgeslagen.
De tweede ontploffing werd versterkt door de opslag van 800 ton kunstmest in de vorm van ammoniumnitraat in de nabijheid.[6]
De opslag van cellulosenitraat is aan strenge regels gebonden. Het moet koel en droog worden opgeslagen. De containers stonden buiten opgesteld in de brandende zon. De temperatuur liep op tot 36 °C en bereikte binnen de containers waarschijnlijk de 65 °C.[6] De verpakking van de cellulosenitraat droogde uit waardoor de ontploffing kon ontstaan. Op het terrein lagen meer gevaarlijke stoffen opgeslagen dan waarvoor vergunningen waren verstrekt.[6] Dit leidde tot een kettingreactie met grote schade tot gevolg. Door de brand en bluswater is in de directe omgeving veel milieuschade opgetreden.
https://www.hindawi.com/journals/joph/2019/1360805/ 
\cite{jiang16042019TanjinExplosion}
verhaal van brandweermannen
\cite{staff31082015tanjinblastunrevealed}
artikel
\cite{chinafile18082015tanjinexplosion}
invloed van social media
\cite{pinghuang2410201TanjinFactreport}
gemaakte fouten
\cite{portoTanjinExplosionSight}
\cite{imago17082015TanjinApartmentImages}
\cite{trager14082015Chemicalblast}
\cite{pangeramo27082015TanjinExplosion}
vergelijking met andere explosies
\cite{ap06082020ammaniumnitrate}
invloed van de ramp op de industrie
\cite{morris14082015TanjinIndustryImpact}
is er sprake van een doofpot
\cite{milesyu20082015exposingtoxicgovlines}
eigendomsverzekering
\cite{artemis30032016tanjininsurance}
\cite{aidenxiatanjinblast}
effecten op de lange termijn
\cite{danwangTanjinflexreport}
\cite{keyHighlightsTanjin}
lessons learned
\cite{hartley13082015videofootage}
\cite{odonnel01062017firetanjinblast2015}
gevolgen voor de industrie
\cite{fan15082015newyorkermistrustchina}
framing vanuit de chinese media
\cite{yanlidongchinamediaframingTanjin}
\cite{evans27092017TnjinInsurance}
niewsartikel
\cite{jasi26032019chineschemplant}
\cite{shiqingTanjinExecutiveSentence}
toegang tot de ramplplek vanuit de okale journalistiek
\cite{sophiebeach15082015}
artikel
\cite{hamzeh05082020BeirutBlast}
\cite{chemwatch18082015TanjiinExplosion}
\cite{thehindu15062019chinaExplosion}
\cite{santagotimes24032019chinablast}
oorzaken
\cite{klingecorp28042020causedTanjin}
case study
\cite{mcgarryExplosions2017}
niewsartikel
\cite{roswnfeld13082015TanjinReports}
chronologische uiteenzetting
\cite{aria12082015explosionaTanjin}
corruptie
mismanagement als oorzaak
autoriteiten publiceren onderoeksrapport
\cite{tremblay11022016chineseInvestigatorsTanjin}
fotos van de rampplek
\cite{taylor13082015TanjinExplosianAftermath}
niuwesartiekel
\cite{associatedPresss13082013}
\cite{un20082015InvestigationTanjin}
\cite{france2412082015TnjinExplosion}
\cite{npr14082015TanjinCause}
123 verantwoordelijken
\cite{bbc05022016TanjinResponsibles}
lang artiekel
\cite{CBodeen15082015TanjinExplosion}
\cite{reutersTanjinInsurance}
\cite{yu082016evaluationTanjin2015}
\cite{wiki2015TanjinExplosions}
\cite{bbc17082015whathappenedTanjin}
\cite{mortimer19082016taijinexplosioncrater}
veiigheidshandhaving
\cite{internationallabourofficeChmControlTooliit}
\cite{euTaxationCustomsICSC}
\cite{iloWHOChemSafetyCards}.
%%%%%%%%%%%%%%%%%%%%%%%%%%%%%%%%%%%%%%%%%%%%%%%%%%%%%%%%%%%%%%%%%
\newline \indent  De ethiopian airlinesop 10/03/2019\cite{caliskan09112013747boeingkalman},\cite{gates18112020boeingcrisis},
\cite{boeing737maxsoftwareprobles},\cite{avetisov19032019boeingmalwarestate},\cite{thompson23112020nationalsecurityboeing},
\cite{wiki737maxgroundings},\cite{campbell02052019boengcrashhumanerrors},
De oorzaak is de MCAS
\cite{hawkins22032019737maxairplanes},\cite{barnett05052019737maxcrisis}, \cite{thomas30082020737safest},\cite{boyle18112020737maxupgrade},\cite{bergstraburgess122019737maxMcasAlgorithm},\cite{737mcas},\cite{german190620217372yaftergrounded},\cite{beningo02052019boeinglessons},\cite{bloomberg26092019failedpred},\cite{afacwaLostSafeguards}, als een single point of failure \cite{uran05042019SPOF}
Angle-of-attack\cite{boeing737maxdisplay},
Behalve de MCAS waren er nog andere failures\cite{fehrm24112020737changes}, en ook deze failures \cite{dohertylindeman15032019737problems}
\cite{travis18042019737maxsoftwaredevop},
%\cite{easa27012021737maxsafereturn},
safety record van de boeing
\cite{touitou11032019737tragedies},
 Oplossingen zijn \cite{caa737modifications}. 
 Ethiopian Airlines Flight 302
In maart 2019 stot vlucht ET302 van Ethiopian airlines neer. De oorzaak ligt bij het mcas flight control system. Dit systeem werd geimplementeerd om kosten te reduceren en opleidingen voor piloten  in te korten. De niweueboeing 737 max model veresite test in volledige flight simulators. Nieuwe faa regels vereisten ondersteuning bij het uitvoeren van enkele manouvres. Tijdens testvluchten uitgevoerd binnen een jaar voor certificatie werd het pitch-up fenomeen geconstateerd waarop het mcas systeem werd aangepast. het werd nu getriggerd door een enkele angle-of-attack sensor.
In eht systeem zaten nu 3 fouten.
- MCAS wordt getriggerd woor enkele sensor zonder vertraging
- Het ontwerp staat toe dat  in situaties waar de angle-of-attack fout is de mCAS wordt geactiveerd
- systeem kreeg onnidig bevoegdheid controle om de neus bij te sturen
- Waarschuwingslicht bij fouten in de angle-of-attack werkte niet door  softwarefout. Het werd ook niet kritisch ebvonden door ethiopian airlines. geplande updates door boeing pas in 2020
- een losstande fout in de microprocessor va de controle computer kan vergelijkbare situaties doen voorkomen zonder dat mcas wordt geactiveerd

Fouten vielen niet op omdat FAA test uitbesteedde aan boeing. Contact tussen de organisaties verliep op management niveau. Boeing instrueerde niet alle piloten ovver mCAs. Het werd gezien als een achtergron relief systeem.

% One minute into the flight, the first officer, acting on the instructions of the captain, reported a "flight control" problem to the control tower.
% Two minutes into the flight, the plane's MCAS system activated, pitching the plane into a dive toward the ground. The pilots struggled to control it and managed to prevent the nose from diving further, but the plane continued to lose altitude.
% The MCAS then activated again, dropping the nose even further down. The pilots then flipped a pair of switches to disable the electrical trim tab system, which also disabled the MCAS software. However, in shutting off the electrical trim system, they also shut off their ability to trim the stabilizer into a neutral position with the electrical switch located on their yokes. The only other possible way to move the stabilizer would be by cranking the wheel by hand, but because the stabilizer was located opposite to the elevator, strong aerodynamic forces were pushing on it.
% As the pilots had inadvertently left the engines on full takeoff power, which caused the plane to accelerate at high speed, there was further pressure on the stabilizer. The pilots' attempts to manually crank the stabilizer back into position failed.
% Three minutes into the flight, with the aircraft continuing to lose altitude and accelerating beyond its safety limits, the captain instructed the first officer to request permission from air traffic control to return to the airport. Permission was granted, and the air traffic controllers diverted other approaching flights. Following instructions from air traffic control, they turned the aircraft to the east, and it rolled to the right. The right wing came to point down as the turn steepened.
% At 8:43, having struggled to keep the plane's nose from diving further by manually pulling the yoke, the captain asked the first officer to help him, and turned the electrical trim tab system back on in the hope that it would allow him to put the stabilizer back into neutral trim. However, in turning the trim system back on, he also reactivated the MCAS system, which pushed the nose further down. The captain and first officer attempted to raise the nose by manually pulling their yokes, but the aircraft continued to plunge toward the ground.
 \cite{caliskan09112013747boeingkalman}
 \cite{gates18112020boeingcrisis}
 \cite{boeing737maxsoftwareprobles}
 \cite{avetisov19032019boeingmalwarestate}
 \cite{thompson23112020nationalsecurityboeing}
 \cite{gates21032019FAAControlSystem}
 \cite{faa18112020boeingreview}
 \cite{wiki737maxgroundings}
 \cite{campbell02052019boengcrashhumanerrors}
 \cite{hawkins22032019737maxairplanes}
 \cite{thomas30082020737safest}
 \cite{boeing737maxdisplay}
 \cite{fehrm24112020737changes}
 \cite{travis18042019737maxsoftwaredevop}
 \cite{barnett05052019737maxcrisis}
 \cite{easa27012021737maxsafereturn}
 \cite{touitou11032019737tragedies}
 \cite{hemmerdinger02022021737maxdeliveries}
 \cite{bielby27022021faaimprovesafety}
 \cite{boyle18112020737maxupgrade}
 \cite{bergstraburgess122019737maxMcasAlgorithm}
 \cite{737mcas}
 \cite{newburger17052019boeingcrisis}
 \cite{arstechnica22012020737problems}
 \cite{german190620217372yaftergrounded}
 \cite{beningo02052019boeinglessons}
 \cite{duran05042019boeingspof}
 \cite{makichuck24012021737fearflying}
 \cite{caa737modifications}
 \cite{oestergaard14122020boeingdeliveries}
 \cite{reenberg787flaws}
 \cite{fitch16092020737backlogrisks}
 \cite{willis27082020737maxfailures}
 \cite{ostrower11062020more737changes}
 \cite{hruska13122019faaknown737crashrate}
 \cite{bloomberg26092019failedpred}
 \cite{whiteman09072020boengcancelstock}
 \cite{leopold09192019boeingreliability}
 \cite{koenig11122019737crashesnofix}
 \cite{dohertylindeman15032019737problems}
 \cite{stodder02102019corruptoversight}
 \cite{afacwaLostSafeguards}
 \cite{swayne18032019profitssafety}
 \cite{freed26022021liftaustraliaban}
 \cite{reed15032019softwareattention}
 \cite{news17032019softwareexplains}
 \cite{legget21122020eu737maxsafe}
 \cite{marketscreener0103221737chinarecertification}
 \cite{euractiv22022021737firegrounds}
 \cite{benny18022019737returnUAE}
 \cite{biersmichel22022021777grounds}
 \cite{reuters23022021777metalfatigue}
%%%%%%%%%%%%%%%%%%%%%%%%%%%%%%%%%%%%%%%%%%%%%%%%%%%%%%%%%%%%%%%%%
\newline \indent Het mortierongeluk in Mali op 06/04/2016. Aanwezige militair brengt slachtoffer naar de fransen, vervolgens naar de Tongolezen. Maar de kwaliteit van personeel liet te wensen over.
Er werd een Nederlandse arts overgevlogen. De slachtoffers werden overgevlogen naar Gao omvervolgens te worden oergevolgen naar Nederland.
Het ongeluk werd veroorzaakt door een kapot afsluitplaatje in de mortier. De granaat opslag in een niet gekoelde container. Dan was er vocht in de fatale granaat. Zodoende werden er explosieve stoffen gevormd in de granaat.
Tijdens de oefening werden de granaten warm in de zon. De granaat stond in veilie stand kon de explosie niet voorkomen.
granaat stond niet op scherp en in afgegaan in veilige stand
Granaat werd opgeslagen in neit gekoelde containers waardoor deze aan te hoge temeperaturen zijn blootgesteld.
Door de comvinatie van vocht en warmte in de granaat zeer gevoelige explosieve stoffen werden gevormd.
Tijdens de oefening was de fatale granaat in de zon.
Het afsluitplaatje in de granaat bleek niet in staat om doorslag in veilige stand te voorkomen waarna de granaat explodeerde.
De moritren zijn aangeschaft bij de amerikanen. gredurende de aanschafperiode zijn procedures en controles op kwaliteit en veiligheid deels nagelaten.
Dit veiligheidsgarantie werd vermeld in het koopcontract.
Conclusie
Koopcontract werd niet goed doorgelezen
Geen controle op kwaliteit en veiligheid
Geen controle op kwaliteit en veiligheid
Zwakke plekken in het ontwerp
Geen controle op kwaliteit en veiligheid
opslag en gebruik in ongunstige condities
De aanwezige medische voorzieningen waren nite volgends de nederlandse militaire richtlijnen
Het ontbreek aan medische toetsing vanuit de defensie organisatie
twijfels die werden geuit binnen de defensieorganisae vonden geen wrrklank
Ok het ongeval tijdens de mortieroefening was voor defensie geen aanleuiding om de medische voorzienignen te evalueren.
De inrichting van veilige medische zorg voor nederlandse militairen in kidal is ondergeschikt gemaakt aan de voortgang van de missie.
\cite{ovvMortierOngevalMaliVideo} 
\cite{bnnvara13062018malirapport}
\cite{eucal11012021malimissieverlengd}
\cite{nos21052014zorgenmalimissie}
\cite{meijnders}
\cite{bnrwebredactie}
\cite{keultjes01062016malimissiecoalitie}
\cite{veenhof18012019}
\cite{isitman06012016militair}
\cite{nporadio11072016filmdemissie}
\cite{parlementairmonitor15122013mortierongeluk}
%%%%%%%%%%%%%%%%%%%%%%%%%%%%%%%%%%%%%%%%%%%%%%%%%%%%%%%%%%%%%%%%%
\newline \indent De ramp tjernobyl 26/04/1986. \cite{INSAVienna1992Chernobyl}
De mislukte veiligheidscontrole op 26 apeil 1986 01.24 uurin de sovjetuni leiddte tot explosies in een van de reactoren in de kerncentrale. De reactoren hadden geen veiligheidomhulling en de reactor bevat grote hoeveelheden brandbaar grafiet.
Door de explosie en de brand kwamen er radioactieve stoffen vrij.het gaat helemaal mis in de kernreactor 4. De warmteproductie nam  toe met een explosie tot gevolg.
31 mensen kwamen om, waaron veel mensen dagen later door stralingsziekte.
 Op 26 april 1986. Techici bij kerncentrale 4 voerden een slcht opgezet/ ontwerpen experiment uit. De  kracht regulering werd uitgeschakeld evenals veiligheidssystemen. 
Een ramp bij een kernreacor in de sovjetunie. Door een bedieningsfout in een testprocedure werd het vermogen van de koelinstallaties negatief beinvloed. Door een ontwerpfout in de noodstopprocedure kon in het systeem niet snel genoeg schakelen om remmende invloed uit te oefenen op het toenemende vermogen van de reactorkernen. Met brand en eksplosie tot gevolg.
\cite{INSAVienna1992Chernobyl}
Tsjernobyl
\cite{wikiTjernobyl}
\cite{rivmTjernobyl}
\cite{andereTijdenTjernobyl}
wat er is gebeurd en hoe het leven verdergaat
\cite{kingskey19042022tjernobyl}
\cite{erikbork26042023reactor4}
\cite{nosTjernobyl30jaarlater}
%Dieren in de omgeving van tjernobyl
%De chronologie
%Echtreme droogte zorgd voor gevaar
%\cite{knmi04052021tjernobylbosbrand}
%\cite{dodonovaKVIRisicoTjernobyl}
%Joernalistiek, entertainment en de waarheid
%\cite{dumarey04062020verhaalTjernobylWaarheid}
%Een onderzoek
%Huidige gevolgen van de explosie van toen
%\cite{sparkesNewScientistTjernoby}
%De ramp, hoe de mensen ermee omgingen en hoe er nu geleef wordt
%evaluatieonderzoek en amatregeen
%\cite{kernenergiened26041986chronologiemaatregelen}
%\cite{mapszoneReactor}
%Invloed van de mens op de omgeving
%Heroplevende splijtingsreacties
%docu van schooltv
%Radioactiviteit bereikt nederland
%documentaire en maatregelen
%\cite{kernhistoriek15062021tjernobyl}
%Het verhaal van een overledende
%Toerisme
%toerisme
%toerisme
%Dieren in de omgevong
%Toevluchtsoord voor vluchtelingen van de oorlog met russische seperatisten
%Ouderen die terugkeerden naar hun woonplaats na de gedwongen verhuizing door de autoriteiten
%De straling neemt weer toe
%Lessen geleerd van tjernobyl
%\cite{nucleairforumFeitenTjernobyl}
%Toerisme
%Bosbrand in tjernobyl
%invloed van de ramp op belgie
%\cite{kernongevalTjernobylFancGov}
%Boek recensie
%Fotos en berekeningen
%ontmanteling en toerisme
%Belangrijke lessen en overeenkomsten
%De journalistieke waarheid van de koude oorlog
%De lessen van
%\cite{arendswolters062019lessenTjernobyl}
%Een toristenattractie maken van tjernobyl
%De radioactieve straling toen en nu
%de 30km zone door de ogen van toeristen
%artikel
%stedentrip
%rapport
%\cite{damveld08052020tjernobyl}
%slapend monster
%docu
%krantenartikel
%hbo serie
%docuserie
%de  nieuwe sacrofaag
%hulp aan slachtoffers
%slapende reactor
%krantenartikel
%\cite{deVriestjernobylHolland}
%hbo serie
%internationale gevolgen
%toerisme
%nieuwe koepel
%media communicatie
%docu
%dieren
%koepel
%koepel
%\cite{ing3enieur29042015antistralingskoepel}
%toerisme
%toeristisch reiperspectief
%toerisme
%niwe koepel
%overschakelen naar duurzaamheid
%docu
%tjernobyl wekt nu duurazme energie
%toerisme
%overeenkomsten tjernobyl en fukushima
%drank en sla uit tjernobyl
%geen efficiente opslag is mogelijk
%wetenschappelijke artikelen
%zaterdag 26 april 1986. Er vind routineonderhoud plaats bij reactor 4, De controle wordt uitegevoerd door de dagploeg. Vnwege een test wordt jhet koelsysteem uitgeschakeld. Door omstandigheden wordt de test uitgesteld en wordt de verantwoordelijkheid overgedragen aan de avondploeg.
%De operator maakt bedieningsfouten waardoot de reactor bijna stil komt te liggen. En vervolgens probeert hij de reactor weer op gang te brengen. ondanks de snelle temperatuurstijging wordt het experiment doorgezet. Dan wordt ook het veiligheidssysteem stilgelgd. Terwijl het koelwater langzaam opwarmt, sluit hij de klep waarlangs de stoom naar de generator stroomt.
%De temperatuur van de reactorstaven neemt daarna snel toe. Terwijl er een oncontroleerbare kettingreactie op gang komt, laat het personeel in paniek de regelstaven zakken om de warmteontwikkeling af te remmen. Het is dan echter al te laat. Door een ontwerpfout loopt het vermogen razendsnel op tot 33.000 megawatt, ruim tien keer hoger dan normaal.
%In een oogwenk verandert al het koelwater in stoom. De ontploffing die daarop volgt, blaast het 2000 ton zware deksel van de reactor af.
%In de ravage vat het gloeiend hete grafiet in de reactor spontaan vlam. De uitslaande brand en een tweede explosie voeren een radioactieve rookwolk tot 8 kilometer hoogte. 
%In een poging het vuur in reactor 4 te doven, storten helikopters vanuit de lucht zand, lood en boorzuur in de reactorkern. Het mag echter niet baten.
%Intussen is de nucleaire brandstof zo heet geworden dat die door de bodem van het reactorvat dreigt te smelten. Als dat gebeurt, kan het bluswater onder het vat in één klap verdampen en dreigt een derde explosie die een groot deel van Europa onbewoonbaar zal maken. Om dit te voorkomen moet het water hoe dan ook worden weggepompt.
%Drie brandweermannen wagen zich daarvoor in de ruimte onder de reactor, blootgesteld aan 300 sievert per uur, 300.000 keer de dosis die een Nederlander jaarlijks maximaal mag oplopen. Ze slagen daarin, maar twee van hen overlijden enkele dagen later aan acute stralingsziekte.
%Hoewel geigertellers de dag na de ramp onrustbarende waarden aangeven, slaat het plaatselijk bestuur geen alarm. De bevolking is het niet gewend om vragen te stellen.
%De volgende dag blijkt er wel degelijk iets ernstigs aan de hand te zijn. In een lange rij bussen worden de 135.000 inwoners op 27 april uit het besmette gebied geëvacueerd, om er nooit meer terug te keren.
%De ramp is dan nog steeds geen wereldnieuws. De Sovjetautoriteiten blijken er niet eens van op de hoogte te zijn – president Gorbatsjov klaagt later dat hij via Zweden aan zijn informatie moest komen.
\cite{verschuur14012013tjernobylreports}
\cite{paperlessarchivesTjernobyl}
\cite{vargos082000tjernobylconcerns}
\cite{mauroNuclearRiskSociety}
\cite{vienna06092005LookingBack}
\cite{wikiTjernobyl},
\cite{rivmTjernobyl},
\cite{andereTijdenTjernobyl},
\cite{kingskey19042022tjernobyl},
\cite{erikbork26042023reactor4},
\cite{nosTjernobyl30jaarlater},
\cite{knmi04052021tjernobylbosbrand},
\cite{dodonovaKVIRisicoTjernobyl},
\cite{dumarey04062020verhaalTjernobylWaarheid},
\cite{sparkesNewScientistTjernoby},
\cite{kernenergiened26041986chronologiemaatregelen},
\cite{mapszoneReactor},
\cite{kernhistoriek15062021tjernobyl},
\cite{nucleairforumFeitenTjernobyl},
\cite{kernongevalTjernobylFancGov},
\cite{arendswolters062019lessenTjernobyl},\cite{damveld08052020tjernobyl},
\cite{deVriestjernobylHolland},\cite{ing3enieur29042015antistralingskoepel},
\cite{verschuur14012013tjernobylreports},\cite{paperlessarchivesTjernobyl},\cite{vargos082000tjernobylconcerns},\cite{mauroNuclearRiskSociety},\cite{vienna06092005LookingBack}
%%%%%%%%%%%%%%%%%%%%%%%%%%%%%%%%%%%%%%%%%%%%%%%%%%%%%%%%%%%%%%%%%
\newline \indent  De digitale aanval op de Oekrainese krachtcentrale op 23,december 2015
Op 23,december 2015  vind er een cyber aanval plaats op het elektriciteitsnet van de Oekraine. Dit was de eerste bekende aanval op een elektrisch contole  system.  Dit verslag geeft inzage in een analyse van de Ukraine cyber aanval,
inclusief hoe de actoren zich zelf toegang gavan tot het controle systeem, welke methoden de acoren hebben gebruikt voor reconnaissance en vastleggen van het systeem, een gedetailleerde omshrijving van de aanval op 15 December 2015, en de methoden die gebruikt zijn door de aanvallers om hun sporen uit te wissen en daarmee het het stoppen van schade toebrengen  nog moeilker maken. Daarnaast wordter  een gedetailleerde omschrijving gevevenv an de beveiliging van de SCADA ccontrol systemen gebaeerd op bst practices, inclusief het control network ontwerp, technieken voor whtelisting, monitoring en loggen, en  opleiding van personeel.
\cite{Whitehead2017ukrainepoweroutage}
\cite{noauthor_2022-nm}
\cite{zetter2016GridHack}
\cite{owens21032017ukrainemitigationstrategies}
\cite{cerulus2019FrontlineRussiaAttack}
\cite{grammatikis2019AttackIEC6087505104}
\cite{hidajat2016ScadaSimulator}
\cite{uscert20072021crashmalware}
\cite{zetter12062017malwareanalysis}
\cite{icsRussianHackingCyberWeapon}
\cite{usgovC2M2}
Dit verslag geeft inzage in een analyse van de Ukraine cyber aanval,
inclusief hoe de actoren zich zelf toegang gavan tot het controle systeem, welke methoden de acoren hebben gebruikt voor reconnaissance en vastleggen van het systeem, een gedetailleerde omshrijving van de aanval op 15 December 2015, en de methoden die gebruikt zijn door de aanvallers om hun sporen uit te wissen en daarmee het het stoppen van schade toebrengen  nog moeilker maken. Daarnaast wordter  een gedetailleerde omschrijving gevevenv an de beveiliging van de SCADA ccontrol systemen gebaeerd op bst practices, inclusief het control network ontwerp, technieken voor whtelisting, monitoring en loggen, en  opleiding van personeel.
\cite{Whitehead2017ukrainepoweroutage},\cite{zetter2016GridHack},\cite{boozallen2016lightwentout},\cite{finklejan2016UsBlamesRussianSandworm},\cite{desarnaud2017cyberattacks},\cite{caseli04112016intrusiondetectioncontrolsystem},\cite{rochascadatesting},\cite{hidajat2016ScadaSimulator},\cite{zetter2017moreDangerousMalware}.
Oop 23,december 2015  vind er een cyber aanval plaats op het elektriciteitsnet van de Oekraine. Dit was de eerste bekende aanval op een elektrisch controle  system met corrupte firmware. Daarnaas wordt er een telecom-based denial of service attack met  geautomatieerde systemen om het telefoonverkeer uit te schakelen.
\cite{Whitehead2017ukrainepoweroutage}
Uit onderzoek\cite{zetter2016GridHack} naar de aanval,  uitgevoerd door Oekraiene sen Amerikaanse militairenblijkt  bleek onder meer dat de power grids in sommige gevallen beter waren beveiligd dan de Amerikaanse. Desondanks was de viligheid niet optimaal door onder andere de  hetgegeven dat werknemers op afstand konden inloggen en geen gebruik van 2-stapsverificatie.
Oekraine wijst naar de russen \cite{zetter2016GridHack}, 
\cite{greenberg2017Cyberwartestlab},
\cite{boozallen2016lightwentout},
\cite{finkle08012016russiansandwormhackers},
\cite{zinets15022017ukrainechargesrussia},
\cite{mcelfresh2016cyberattackhowandwhy},
\cite{parkwalstorm11102017russiagridattack}.
{Situatie Oekraiene}
\cite{drago2017CrashOverride},
\cite{slowik2019ReassasUkraine2016Attack}.
{Situatie algemeen}
\cite{cerulus2019FrontlineRussiaAttack},
\cite{desarnaud2017cyberattacks},
\cite{dragos2019TargetedTransStation}.
{Factoren}
\cite{shehod2016gridadvantageus}
{Oorzaak}
\cite{rocha2017cybersecyrityanalysisScada},
\cite{2017crashoverridenostuxnet},
\cite{vijayan2017firstmalwareCausedOutage},
\cite{slowik2019ReassasUkraine2016Attack}.
{Gebruikte materialen}
\cite{2015ukrainegridattack},
\cite{industroyershortfact}
{Uitvoering van de aanval}
\cite{Whitehead2017ukrainepoweroutage},
\cite{boozallen2016lightwentout}.
{Oplossingen}
~\cite{Whitehead2017ukrainepoweroutage}
\cite{Whitehead2017ukrainepoweroutage}
\cite{boozallen2016lightwentout}
{spearfishing}
{blackenergy}
{remote access capabilities}
{serial-to-ethernet communication devices}
{telephony denial of service attacks}
{oplossingen}
Identificeer alle risicos en schrijf een plan foor het managen van de risico's.
Implementeer  effecteve controle  om het riico te managen.
Creeer een diepgaand model dat ervoor zor dat er efectieve en efficiente security controls worden uitgevoerd.
Aangaande de gebeurtenissen in de oekraiene kunnen de volgende security controls worden opgenomen in het securitymodel: Initial access to enterprise network, pivot in interprise network, elevate priviliges, maintainance access, gain access to control system, attack, attack complication, destroy hard drives.
\cite{Whitehead2017ukrainepoweroutage}
{Discussie}
{Verder lezen}
\cite{shahzad2014ScadaProtocolsPollingScenario},
\cite{grammatikis2019AttackIEC6087505104},
\cite{2017win32industroyer},
\cite{yadav2020reviewScadaArchitecture},
\cite{arrizabalaga2020surveyiiotProtocols},\cite{fauri2017EncryptionICS},\cite{resch31102019IEC62351secureCommunication},\cite{levalle2020FuzzingICSProtocols},\cite{blackhatusa2017},\cite{blackhatusa2017},\cite{abb30062017crashoverridenotification},\cite{spinner2018crashoverrideiot},\cite{njccicthreat08102017crashovverrideprofile},\cite{slowikvb2018crashoverride},\cite{crashoverridenetwork},\cite{wikiindustroyer},\cite{icsSecurityRussianHacking},\cite{holappa2017threattoElectricityNetworks}.
%%%%%%%%%%%%%%%%%%%%%%%%%%%%%%%%%%%%%%%%%%%%%%%%%%%%%%%%%%%%%%%%%

%%%%%%%%%%%%%%%%%%%%%%%%%%%%%%%%%%%%%%%%%%%%%%%%%%%%%%%%%%%%%%%%%
  explosie in libabon, beirut 
Op 23 september 2013 voer het vrachtschip de Rhosus onder Moldavische vlag[7] van Batoemi in Georgië naar Beira in Mozambique met 2.750 ton ammoniumnitraat
Gezien het ernstige gevaar van het bewaren van deze goederen in de hangar onder ongeschikte klimatologische omstandigheden, herhalen we ons verzoek aan de marine-instantie om deze goederen onmiddellijk weer te exporteren om de veiligheid van de haven en de mensen die er werken te verzekeren, of om akkoord te gaan om ze te verkopen.
Voorafgaand aan de explosie was er een brand in een opslagplaats. 
\cite{hrw03082021investigateBeirutBlast}
\cite{souaibyElHussein112020Beirutstory}
\cite{ifrc2020chemicalexplosionBeirutPort}
%%%%%%%%%%%%%%%%%%%%%%%%%%%%%%%%%%%%%%%%%%%%%%%%%%%%%%%%%%%%%%%%%
\newline \indent  stint ongeluk
Vier kinderen, een bestuurder kwamen om en een vijfde persoon , een kind raakte zwaargewond. Uit odnerzoek van bleek :
Foute torsieveer voor de gashendel werd geleverd
Geen van de drie onderzochte voertuigen haalden de wettelijk vereiste remvertraging
De automatische parkeerrem kan leiden tot gevaarlijke situaties wanneer deze ongewenst geactiveerd wordt tijdens het rijden. 
Het losraken van de nuldraad naar de gashendel leidt volgens TNO tot ongewenst versnellen van het voertuig en een oncontroleerbare situatie voor de bestuurder.
Voor alle drie onderzochte voertuigen geldt dat het ontbreken van een zitplaats leidt tot veiligheidsrisico’s voor remmen en sturen door de grotere kans dat de bestuurder van het voertuig valt. Als de bestuurder van een Stint valt, leidt dit in alle rijsituaties tot een onbeheersbare situatie
\cite{TNOStint}
%%%%%%%%%%%%%%%%%%%%%%%%%%%%%%%%%%%%%%%%%%%%%%%%%%%%%%%%%%%%%%%%%
\newline \indent vuurwerkramp in enschede 
\cite{boogers092002RampenRegelsRichtlijnen}
Wat waren de afspraken omtrent vuurwerkopslag?
Waarom werden de voorschriften neit nageleefd?
%%%%%%%%%%%%%%%%%%%%%%%%%%%%%%%%%%%%%%%%%%%%%%%%%%%%%%%%%%%%%%%%%
\newline \indent  ecourt in nederlandse rechtspraak
niet odnerzocht

\cite{sprongken19032018CourtProcedureDigital}

\cite{PROCESREGLEMENTEcourt}
%%%%%%%%%%%%%%%%%%%%%%%%%%%%%%%%%%%%%%%%%%%%%%%%%%%%%%%%%%%%%%%%%
\newline \indent  molukse treinkaping 

\cite{molukseTreinkaping}
%%%%%%%%%%%%%%%%%%%%%%%%%%%%%%%%%%%%%%%%%%%%%%%%%%%%%%%%%%%%%%%%%
\newline \indent Ramp schietpartij militair ossendrecht. Een militaire overleed op een schietbaan in ossendracht door onvoldoende begeleiding van cursisten, geen toezicht op de lokatie. E\r was een instructuur in opleiding die niet volledig was mmeegenomen in het poroces en ook was er geen baancommandant aanwezig. Geen van de aanwezig instructeurts had de juiste papieren om de cursisten te begeleiden. De aanwezig instruceur had geen zich op de instructeur in opleiding, evenmin de andere militairen. In de instructiehandleiding ontbreken richtlijnen voor bijzondere schietbanen. Ook was er geen keuring. Door personelstekort is er geen andacht besteed aan documentastie(een slyllabus) hoe en met welke risico’s oefeningnen moeten worden ingericht. Ok werd er vooraf geen veiliheidsanaklyse gedaan. Het gebrek aan lesmateriaal en deskundigen is gemeld binnen de defensieorganisatie maar dit heeft niet geleid tot enige verandering in de situatie.
Op een afgekeurde scheitbaan
Tezicht door een instructeur in opleiding die zelf geen persoonlijke begeleiding heeft gehad tijdens de uitvoering
Belangrijk is dat defensie haar taken kan uitvoeren met personeel dat is getraind in situaties die de risicos van de werkomgeving aan de cursisten kunnen laten zien.
Conclusie
Zonder gekwalificeerde instructuers.
Zonder toezicht
Zonder lesmateriaal
Zonder adequate veiligheidsanalyse
\cite{ovvVideoOssendrecht}
\cite{oVVSchietongevalOssendrecht}
\cite{nos22032016ossendrecht}
\cite{ovv04042016lessenongevalossendrecht}
\cite{quekelboere10052017doodossendrecht}

%%%%%%%%%%%%%%%%%%%%%%%%%%%%%%%%%%%%%%%%%%%%%%%%%%%%%%%%%%%%%%%%%
\newline \indent Dan zijn er nog andere ongelukken met de stint, de shietpartij op militairencomplex in ossendrecht, stint-ongeluk, de enschedese vuurwerkramp en de molukse treinkaping. Meer recentelijk de coronacrisis.
%%%%%%%%%%%%%%%%%%%%%%%%%%%%%%%%%%%%%%%%%%%%%%%%%%%%%%%%%%%%%%%%%


% 
% 
%\paragraph{ethiek}
%
%
%Ethiek 
%
%
%
%persuasive technology 
%https://www.humanetech.com/youth/persuasive-technology 
%\cite{humanTechpersuasiveTech}
%https://www.minddistrict.com/blog/persuasive-technology-new-insights-in-behavioural-change 
%https://www.sciencedirect.com/book/9781558606432/persuasive-technology 
%https://spectrum.ieee.org/how-persuasive-technology-can-change-your-habits 
%\cite{rezenfeld01012018persuasiveTecgHabits}
%https://www.frontiersin.org/articles/10.3389/frai.2020.00007/full 
%\cite{aldenaini28042020persuasiveTechTrends}
%https://psmag.com/environment/captology-fogg-invisible-manipulative-power-persuasive-technology-81301 
%\cite{larson14062017persuasivetechmanipulates}
%https://www.makeuseof.com/what-is-persuasive-technology/ 
%\cite{tanzem22012022persuasivetechchanginglives}
%https://lib.ugent.be/catalog/rug01:001235489 
%https://cyberpsychology.eu/article/view/12270 
%\cite{tikkakuddonenpersuasiveTechnology}
%

%
%
%\paragraph{Ondeerzoeksresultaten naar sluisbeveiliging}
%
%
%
%Verouderde computersystemen zijn door de jaren heen gekoppeld aan netwerken, zodat ze op afstand te besturen zijn. Dit zorgt ervoor dat systemen kwetsbaar zijn voor aanvallen van buitenaf. De beveiliging is in de loop der jaren niet voldoende ontwikkeld om de infrastructuur goed te beveiligen.
%
%Volgens het onderzoek is er de afgelopen jaren wel het nodige geïnvesteerd om de beveiliging op te schroeven, maar deze maatregelen zijn nog onvoldoende doorgevoerd.
%https://www.nu.nl/internet/5814282/rekenkamer-waterwerken-niet-goed-beveiligd-tegen-cyberaanvallen.html
%\cite{hdsr30092022lichtprojectieswaterliniesluizen}
%rapport Digitale dijkverzwaring: cybersecurity en vitale waterwerken 
%Crisisdocumentatie is verouderd en er worden geen volwaardige pentesten uitgevoerd. Uit het onderzoek blijkt dat nog niet alle vitale waterwerken rechtstreeks zijn aangesloten op het Security Operations Center (SOC) van Rijkswaterstaat. Hierdoor bestaat het risico dat RWS een cyberaanval niet of te laat detecteert. De minister van Infrastructuur en Waterstaat moet nog stappen zetten om aan de eigen doelstellingen voor cybersecurity te voldoen
%De Algemene Rekenkamer beveelt de minister van Infrastructuur en Waterstaat ook aan om het actuele dreigingsniveau te onderzoeken en te besluiten of extra mensen en middelen nodig zijn. Ook is het voor een snelle en adequate reactie op een crisissituatie van essentieel belang dat informatie up-to-date is. Pentesten zouden integraal onderdeel uit moeten maken van de cybersecuritymaatregelen bij vitale waterwerken. Verder zou moeten worden bezien of medewerkers van het SOC beter moeten worden gescreend.
%
% 
%
%
%\cite{thkwaterwerken}
%Het crisismodel kan beter, is de derde deelconclusie van de Algemene Rekenkamer. Er is geen specifiek scenario voor een crisis die wordt veroorzaakt door een cyberaanval. Ook ontbreekt inzicht in de effecten van een cybercrisis op andere sectoren, de zogeheten cascade-effecten. Tevens is de crisisdocumentatie op onderdelen verouderd.
%
%\cite{rekenkamercybersecWater}
%Ook maakt cyberveiligheid nog geen volwaardig onderdeel uit van reguliere inspecties.’ De Rekenkamer hamert erop dat alle vitale waterinfrastructuur zo snel mogelijk op het SOC wordt aangesloten. Ook zouden werknemers van Rijkswaterstaat die belangrijke waterkeringen bedienen beter gescreend moeten worden op hun antecedenten. Sollicitanten hoeven nu slechts een Verklaring Omtrent Gedrag te overleggen, maar dat is een heel lichte toets.
%
%\cite{hackerWaterwerk}
%deltawerken
%
%\cite{kramerZeeland}
%Volgens Rijkswaterstaat is het kostbaar en technisch uitdagend om klassieke automatiseringssystemen te moderniseren en wordt er daarom vooral ingezet op detectie van aanvallen en een adequate reactie daarop.
%Uit het onderzoek blijkt dat Rijkswaterstaat de afgelopen jaren zelf van alle tunnels, bruggen, sluizen et cetera heeft vastgesteld welke cyberveiligheidsmaatregelen moeten worden genomen. Een groot deel van die maatregelen (ongeveer 60\%) was begin 2018 ook al uitgevoerd, maar Rijkswaterstaat ziet onvoldoende toe op de uitvoering van het resterend deel en heeft geen actueel overzicht van de overgebleven maatregelen.
%De minister heeft een aantal waterwerken die Rijkswaterstaat beheert als vitaal aangewezen. . Uit het onderzoek blijkt dat nog niet alle vitale waterwerken rechtstreeks zijn aangesloten op het Security Operations Center (SOC) van Rijkswaterstaat. De ambitie om eind 2017 bij alle vitale waterwerken cyberaanvallen direct te kunnen detecteren was in het najaar van 2018 daarmee nog niet gerealiseerd. Hierdoor bestaat het risico dat RWS een cyberaanval niet of te laat detecteert.
%
%\cite{cybersecWaterwerk}
%Over de cyberbeveiliging van gemeenten en waterschappen wordt al langer geklaagd. Zo meldde EenVandaag al in 2012 dat rioolgemalen en sluizen gemakkelijk van afstand te bedienen waren, onder meer door bijzonder slechte wachtwoorden.
%
%\cite{cybersecWaterschappen}
%Rittal doet onderzoek naarop afstand besdienbare sluizen
%
%\cite{cybersecZuidHolland}
%Beveiligde VPN
%M2M Services levert aan inmiddels 220 gemeenten en waterschappen beveiligde connectiviteitsoplossingen voor het beheer van pompen, riolen en gemalen. Om risico’s op beveiligingsincidenten te voorkomen maken wij gebruik van een VPN oplossing, waarbij de verbinding optimaal beveiligd is middels encryptie en authenticatie.
%
%\cite{waterwerkNED}
%Veiligheid op het water én op het land
%Gebruik van lampbewaking 
%
%\cite{veiligheidwaterland} 

%%%%%%%%%%%%%%%%%%%%%%%%%%%%%%%%%%%%%%%%%%%%%%%%%%%%%%%%%%%%%%%%%




\paragraph{Safety critical systems}

\cite{winceckCriticalToSafety}
\cite{chambersHazardAnalysisSCS}
\cite{rslater1998SCSAnalysis}
%Traditional Systems
%Traditional areas that have been considered the home of safetycritical systems include medical care, commercial aircraft, nuclear
%power, and weapons. Failure in these areas can quickly lead to
%human life being put in danger, loss of equipment, and so on.
%
%Non-traditional Systems
%Emergency 911 service is an example of a critical infrastructure
%application. Other examples are transportation control, banking
%and financial systems, electricity generation and distribution, telecommunications, and the management of water systems
%
%4.1 Technology
%
%https://users.encs.concordia.ca/~ymzhang/courses/reliability/ICSE02Knight.pdf
\cite{knightchallengessafetyCritical}

https://www.dcs.gla.ac.uk/~johnson/teaching/safety/slides/pt2.pdf
\cite{johnson2006devsafetycritical}
\cite{daucriticalsafetyconsider}
\cite{fallsafedesign}
\cite{arForce2015VerificationExpectations}
\cite{nebulaassessment}
\cite{lalaArchitecturalPrinciples}
\cite{mitNotesSafetyCritical}
\cite{britishColumbia2020GuideSafetyCritical}
%1.       The Assembly is aware that the use of computers in safety-related applications is growing, particularly in areas such as control systems of aeroplanes, high-speed trains and nuclear power stations, medical equipment and medical records, anti-lock braking systems for vehicles and machine engineering in general, and last but not least, modern weapons and their guidance systems.
%
%2.       Many recent accidents (for example, plane crashes due to computer failure, malfunctioning robot killing a mechanic, patient dying because of malfunctioning of computer-controlled intravenous drip, rocket launch failure traced to computer error, software piracy etc.) cause public concern and raise the question of the reliability of such systems.
%
%
%How has the problem of safety-critical software arisen? Essentially from an ever-increasing complexity in engineering. One may compare the steam locomotive of 1830 with the APOLLO Moon spacecraft of 1970 as an example. In 1917 WM FARREN designed, supervised the construction of and testflew an aircraft - the CE 1 and with acceptable safety! [2]. Even in 1965 a chief designer would be familiar with all the decisions taken in the design of a complex product such as an aircraft or ship. The management operation was deeply hierarchical [3] , but as systems became more complex and design teams included more and more specialists it became necessary to formalise the interfaces between the specialist groups to gain benefit and yet maintain overall design disciplines. This led to the matrix design management system in the 1970s to cope with design teams 50 times larger than before [4].
%
%A difficulty embodied in tackling the safety related to software in engineered products arises because of software complexity and the mathematical rigour of some parts of it distorts and clouds the fundamental processes of creative engineering design. 
%
%Before discussing safety definitions and integrity a brief mention of design techniques to enhance safety. One way of increasing safety is to develop more reliable components and systems. At the outset, once the general preliminary design is defined there will be a "safety budget" allocating tolerable levels of integrity for every subsystem. Then Reliability Analysis evaluates the probability of failure and Failure Mode Effect and Criticality Analysis deals with the likely results of failure. Once the "life" of a part has been measured then the inspection and maintenance function will act to replace the part with a new one in good time. Another technique is to design an item to "fail-safe" i.e. even if it does fail it does not create a safety risk before the fault can be rectified. This has been extensively used on structures and coping with the development of fatigue cracks. "Fail- operate", "fault tolerant design" and "graceful degradation of systems" are other methods.


\cite{fulvio1993safetycriticalsystems}
\cite{dlrtabid}
\cite{knight2010SafetyCritical}
\cite{creavisafecritical}
\cite{valdes2018SafetybyAutomation}

https://verticalmag.com/features/whensafetymanagementsystemsfail/
\cite{2015whensafetymanagementsystemsfail}
\paragraph{Ondeerzoeksresultaten naar sluisbeveiliging}



Verouderde computersystemen zijn door de jaren heen gekoppeld aan netwerken, zodat ze op afstand te besturen zijn. Dit zorgt ervoor dat systemen kwetsbaar zijn voor aanvallen van buitenaf. De beveiliging is in de loop der jaren niet voldoende ontwikkeld om de infrastructuur goed te beveiligen.

Volgens het onderzoek is er de afgelopen jaren wel het nodige geïnvesteerd om de beveiliging op te schroeven, maar deze maatregelen zijn nog onvoldoende doorgevoerd.

\cite{hdsr30092022lichtprojectieswaterliniesluizen}
rapport Digitale dijkverzwaring: cybersecurity en vitale waterwerken 
Crisisdocumentatie is verouderd en er worden geen volwaardige pentesten uitgevoerd. Uit het onderzoek blijkt dat nog niet alle vitale waterwerken rechtstreeks zijn aangesloten op het Security Operations Center (SOC) van Rijkswaterstaat. Hierdoor bestaat het risico dat RWS een cyberaanval niet of te laat detecteert. De minister van Infrastructuur en Waterstaat moet nog stappen zetten om aan de eigen doelstellingen voor cybersecurity te voldoen
De Algemene Rekenkamer beveelt de minister van Infrastructuur en Waterstaat ook aan om het actuele dreigingsniveau te onderzoeken en te besluiten of extra mensen en middelen nodig zijn. Ook is het voor een snelle en adequate reactie op een crisissituatie van essentieel belang dat informatie up-to-date is. Pentesten zouden integraal onderdeel uit moeten maken van de cybersecuritymaatregelen bij vitale waterwerken. Verder zou moeten worden bezien of medewerkers van het SOC beter moeten worden gescreend
\cite{thkwaterwerken}.Het crisismodel kan beter, is de derde deelconclusie van de Algemene Rekenkamer. Er is geen specifiek scenario voor een crisis die wordt veroorzaakt door een cyberaanval. Ook ontbreekt inzicht in de effecten van een cybercrisis op andere sectoren, de zogeheten cascade-effecten. Tevens is de crisisdocumentatie op onderdelen verouderd\cite{rekenkamercybersecWater}.
Ook maakt cyberveiligheid nog geen volwaardig onderdeel uit van reguliere inspecties.’ De Rekenkamer hamert erop dat alle vitale waterinfrastructuur zo snel mogelijk op het SOC wordt aangesloten. Ook zouden werknemers van Rijkswaterstaat die belangrijke waterkeringen bedienen beter gescreend moeten worden op hun antecedenten. Sollicitanten hoeven nu slechts een Verklaring Omtrent Gedrag te overleggen, maar dat is een heel lichte toets
\cite{hackerWaterwerk}.
Volgens Rijkswaterstaat\cite{kramerZeeland} is het kostbaar en technisch uitdagend om klassieke automatiseringssystemen te moderniseren en wordt er daarom vooral ingezet op detectie van aanvallen en een adequate reactie daarop.
Uit het onderzoek blijkt dat Rijkswaterstaat de afgelopen jaren zelf van alle tunnels, bruggen, sluizen et cetera heeft vastgesteld welke cyberveiligheidsmaatregelen moeten worden genomen. Een groot deel van die maatregelen (ongeveer 60\%) was begin 2018 ook al uitgevoerd, maar Rijkswaterstaat ziet onvoldoende toe op de uitvoering van het resterend deel en heeft geen actueel overzicht van de overgebleven maatregelen.
De minister heeft een aantal waterwerken die Rijkswaterstaat beheert als vitaal aangewezen. . Uit het onderzoek blijkt dat nog niet alle vitale waterwerken rechtstreeks zijn aangesloten op het Security Operations Center (SOC) van Rijkswaterstaat. De ambitie om eind 2017 bij alle vitale waterwerken cyberaanvallen direct te kunnen detecteren was in het najaar van 2018 daarmee nog niet gerealiseerd. Hierdoor bestaat het risico dat RWS een cyberaanval niet of te laat detecteert\cite{cybersecWaterwerk}.
Over de cyberbeveiliging van gemeenten en waterschappen wordt al langer geklaagd. Zo meldde EenVandaag al in 2012 dat rioolgemalen en sluizen gemakkelijk van afstand te bedienen waren, onder meer door bijzonder slechte wachtwoorden
\cite{cybersecWaterschappen}.
Rittal doet onderzoek naarop afstand besdienbare sluizen\cite{cybersecZuidHolland}.
Beveiligde VPN
M2M Services levert aan inmiddels 220 gemeenten en waterschappen beveiligde connectiviteitsoplossingen voor het beheer van pompen, riolen en gemalen. Om risico’s op beveiligingsincidenten te voorkomen maken wij gebruik van een VPN oplossing, waarbij de verbinding optimaal beveiligd is middels encryptie en authenticatie\cite{waterwerkNED}.
Veiligheid op het water én op het land Gebruik van lampbewaking \cite{veiligheidwaterland}. 



%%%%%%%%%%%%%%%%%%%%%%%%%%%%%%%%%%%%%%%%%%%%%%%%%%%%%%%%%%%%%%%%%
\paragraph{Afbakening van requirements Wet en regelgeving voor sluizen}
Omdat we in deit onderzoek uitgaan van het uitbreiden van bestaande sluizen is er literatuurstudie gedaan naar sluizen. In de archieven van het ministerie van verkeer en waterstaat is er het rapport Design of waterlocks\cite{CivilEngineeringDivision}.
Het programma van requirements kunnen we in ons model niet helemaal overnemen. 
Zo zijn er precondities zaols topgrafie,bestaande watersluizen,waterlevel, wind, morphologie en bodemeigenschappen.

 

\paragraph{Analyse}
\paragraph{Conclusie}
%%%%%%%%%%%%%%%%%%%%%%%%%%%%%%%%%%%%%%%%%%%%%%%%%%%%%%%%%%%%%%%%%

%%%%%%%%%%%%%%%%%%%%%%%%%%%%%%%%%%%%%%%%%%%%%%%%%%%%%%%%%%%%%%%%%
 
\hoofdstuk{Uppaal model}


\paragraph{Inleiding}


\begin{center}
	\figuur{scale=0.45,angle=180}{plaatjes/sluispassage.png}{PDFA}{Vaste breedte
		(pdf)}
\end{center}













\hoofdstuk{Verificatie}
 We moeten aantonen dat een real-time programma voldoet aan de eisen opgesteld en gespecificeerd. De meest gebruikte methode voor het bewij
 
 zen van de correctheid van untimed programma's zijn aangepast voor timed programs.  We hebben nog geen aanpask gevonden voor het gebruik en bewijzen van correct gebruik van clocks.  Een bewijs voor het gebruik van real-time programmas met clocks is gegeven in T.A. Henzinger and P.W. Kopke. Verification methods for the di-
 vergent runs of clock systems
 
 In dit hoofdstuk formaliseren we de requirements ogegeven in de requiremenstlis tin hoofdstuk .. en bewijzen we de correcte toepassing met gebruik van de symbolic model-checker van Uppaal.
 Het systeem is gemodelleerd als een netwerk van meerdere timed automata: controller, sluis, stoplicht, deur, pomp en schip.
 
 Het bewijs vn corret gebruik kan ook worden aangetoond met help van bewijs voor inorrectgebruik
 
 
 
\paragraph{Semantiek}
 
 \paragraph{Timed automata}
  Timed automata [4] [57] are hereby to model timed systems. These are finite-state automata
 equipped with clocks used to specify constraints on the amount of time that can elapse
 between two events (blz 46). Timed kripke structures (blz63) (blz 69) (blz 78) blz 99.
 \cite{nourollahi20191215}
 
\paragraph{Data variabelen}
Dat variabelen zijn onder andere: water hoog  en laag, en aanal schepen in de queue.
\paragraph{Acties}
 Acties in het model zijn onder andere: invaren, uitvaren, deuren openen en sluiten, nivelleren
\paragraph{Clock regions}
\cite{clarke2000Modelchecking}
\cite[p.~14]{clarke2000Modelchecking}
\cite[p.~16--20]{clarke2000Modelchecking}
\cite[p.~22-24]{clarke2000Modelchecking}
\cite[p.27--29]{clarke2000Modelchecking}
\cite[p.~30]{clarke2000Modelchecking}
\cite[p.~32]{clarke2000Modelchecking}
\cite[p.~40--41]{clarke2000Modelchecking}
\cite[p.~46--49]{clarke2000Modelchecking}
\cite[p.~68--71]{clarke2000Modelchecking}
\cite[p.~71]{clarke2000Modelchecking}
\cite[p.~79]{clarke2000Modelchecking}
\cite[p.~121]{clarke2000Modelchecking}
\cite[p.~144--145]{clarke2000Modelchecking}
\cite[p.~178]{clarke2000Modelchecking}
\cite[p.~189]{clarke2000Modelchecking}
\cite[p.~197]{clarke2000Modelchecking}
\cite[p.~199--203]{clarke2000Modelchecking}
\cite[p.~215--236]{clarke2000Modelchecking}
\cite[p.~265--268]{clarke2000Modelchecking}
\cite[p.~268]{clarke2000Modelchecking}
\cite[p.~274--281]{clarke2000Modelchecking}


\cite{audioSemanticsBengtsson}
\cite{guidingAutomataBberm}
\cite{gearTransitionLindahl1}
\cite{gearTransitionLindahl2}
\cite{martinelliScada}
\cite{IgbalReconstructurintTransition1}
\cite{IgbalReconstructurintTransition2}
\cite{huangVerficationStoch}
\cite{bengtssonUppaalVerification}
\cite{pranaliVerificationWaterLevel}
\cite{alexandreUppaalDefinition}
\cite{behzadEvalQOS}
\cite{behzadVariablesQoS}
\cite{alur}
\cite{alurDenseRealTime}
\cite{alurSystemClok}
\cite{alurModelHybrid}
\cite{rijksoverheidSluizen}
\cite{rijksoverheidSluisStroomschema}

\paragraph{CTL logica}
Alle veiligheid en reachability requirements formeel gespecificeerd in hoofdstuk ... zijn geverifieerd in uppaal met gebruik an A en E state formulae. Deze zijn als volgt:
$\sim$, $\xi$ , $\cong$,$\overset{\Delta}{=}$ or equal by definition, $\uplus$
\newline \\
Om aan te tonen dat de gedefinieerde specificaties altijd geldig zijn moet de basisi specificatie inductief worden opgelost. \cite{latin06} blz 73,82,83,90,91,92,93,98,104,156,197, 225, 236, 315, 317, 318\cite[p.~318]{realtimeForms}  

%%%%%%%%%%%%%%%%%%%%%%%%%%%%%%%%%%%%%%%%%%%%%%%%%%%%%%%%%%%%%%%%%
M, s $\models$ p $\Leftrightarrow$ p $\in$ L(s) \\
M, s $\models$ $\not$ f1 $\Leftrightarrow$ M, s $\nvdash$ f1 \\
M, s $\models$ f1 $\vee$ f2 $\Leftrightarrow$ M,s $\models$ f1 or M,s $\nvdash$ f2 \\
M, s $\models$ f1 $\wedge$ f2 $\Leftrightarrow$  M,s $\models$ f1 and M,s $\nvdash$ f2 \\
M, s $\models$ $\mathrm{E}$ $g_{1}$ $\Leftrightarrow$ there is a path $\pi$  from ~  s ~   such ~  that  ~ M, $\pi$ $\models$ g1 \\
M, s $\models$ p $\Leftrightarrow$ for every path $\pi$  ~ starting from  ~  s, M, $\pi$ $\models$ g1 \\
M, s $\models$ p $\Leftrightarrow$ s is the first state of $\piand$ M, s $\models$ f1 \\
M, s $\models$ $\not$ $g_{1}$ $\Leftrightarrow$ M, $\pi$  $\nvdash$ g1\\
M, s $\models$ p $\Leftrightarrow$  M, $\pi$  $\models$ g1  or  M, $\pi$  M, $\pi$  $\models$ g2\\
M, s $\models$ p $\Leftrightarrow$ M, $\pi$  $\models$ g1  and  M, $\pi$  M, $\pi$  $\models$ g2 \\
M, s $\models$ p $\Leftrightarrow$ M, $\pi^{1}$ $\models$ g1 \\
M, s $\models$ p $\Leftrightarrow$ there exists a k $\ge$ 0, such that  ~ M, $\pi^{k}$  $\models$ g1\\
M, s $\models$ p $\Leftrightarrow$ for all i $\ge$ 0,M,$\pi^{i}$ $\models$ g1 \\
M, s $\models$ g1 $\bugcup$ g2 $\Leftrightarrow$ ~  there  ~ exists  ~ ak  ~ $\ge$  ~ 0 ~  such ~  that  ~ M,  ~ $\pi^{k}$ $\models$ g2\\
and  ~ for  ~ all ~  0  ~ $\le$ j < k, M,$\pi^{j}$ $\models$ g1
M, s $\models$ p $\Leftrightarrow$ for all j $\ge$ 0, if for ~  every  ~ i < j,M,$\pi^{i}$ $\nvdash$ g1 then M,$\pi^{j}$ $\models$ g2\\
%%%%%%%%%%%%%%%%%%%%%%%%%%%%%%%%%%%%%%%%%%%%%%%%%%%%%%%%%%%%%%%%%
Safety properties
Following L. Lamport, a safety property states that
something bad must never happen. The “bad thing” represents a
critical system state that should never occur, for instance a train
being inside a crossing with the gates open. Taking a Boolean observable C : Time −→ {0, 1}, where C(t) = 1 expresses that at
time t the system is in the critical state, this safety property can be
expressed by the formula:
$\forall$ t \in Time $\dot{}$ $\neg$ C(t)


Here C(t) abbreviates C(t) = 1 and thus ¬C(t) denotes that at time
t the system is not in the critical state. Thus for all time points it
is not the case that the system is in the critical state.
In general, a safety property is characterised as a property that can be falsified in bounded time. In case of (1.1) exhibiting a single
time point t0 with C(t0) suffices to show that (1.1) does not hold.
In the example, a crossing with permanently closed gates is safe,
but it is unacceptable for the waiting cars and pedestrians. Therefore
we need other types of properties.
%%%%%%%%%%%%%%%%%%%%%%%%%%%%%%%%%%%%%%%%%%%%%%%%%%%%%%%%%%%%%%%%%
liveness properties
Safety properties state what may or may not occur,
but do not require that anything ever does happen. Liveness properties state what must occur. The simplest form of a liveness property guarantees that something good eventually does happen. The
“good thing” represents a desirable system state, for instance the
gates being open for the road traffic. Taking a Boolean observable
G : Time −→ {0, 1}, where G(t) = 1 expresses that at time t the
system is in the good state, this liveness property can be expressed
by the formula: 
$\exists$t $\in$ Time $\dot{}$ G(t).
In other words, there exists a time point in which the system is in the
good state. Note that this property cannot be falsified in bounded
time. If for any time point t0 only ¬G(t) has been observed for
t ≤ t0, we cannot complain that (1.2) is violated because eventually
does not say how long it will take for the good state to occur.
Such liveness property is not strong enough in the context of realtime systems. Here one would like to see a time bound when the
good state occurs. This brings us to the next kind of property.
%%%%%%%%%%%%%%%%%%%%%%%%%%%%%%%%%%%%%%%%%%%%%%%%%%%%%%%%%%%%%%%%%
bounded response properties

A bounded response property states that
a desired system reaction to an input occurs within a time interval
[b, e] with lower bound b ∈ Time and upper bound e ∈ Time where
b ≤ e. For example, whenever a pedestrian at a traffic light pushes
the button to cross the road, the light for pedestrians should turn
green within a time interval of, say, [10, 15]. The need for an upper
bound is clear: the pedestrian wants to cross the road within a short
time (and not eventually). However, also a lower bound is needed
because the traffic light must not change from green to red instantaneously, but only after a yellow phase of, say, 10 seconds to allow
cars to slow down gently.
With P(t) representing the pushing of the button at time t and
G(t) representing a green traffic light for the pedestrians at time t,
we can express the desired property by the formula
$\forall$ t1 $\in$ Time $\dot{}$ (P(t1) $\rightarrow$ $\exists$t2 \in [t1 + 10, t1 + 15] $\dot{}$ G(t2))
Note that this property can be falsified in bounded time. When
for some time point t1 with P(t1) we find out that during the time
interval [t1 + 10, t1 + 15] no green light for the pedestrians appeared,
property (1.3) is violated.
%%%%%%%%%%%%%%%%%%%%%%%%%%%%%%%%%%%%%%%%%%%%%%%%%%%%%%%%%%%%%%%%%
Duration properties

A duration property is more subtle. It requires that
for observation intervals [b, e] satisfying a certain condition A(b, e)
the accumulated time in which the system is in a certain critical
state has an upper bound u(b, e). For example, the leak state of a
gas burner, where gas escapes without a flame burning, should occur
at most 5% of the time of a whole day.
To measure the accumulated time t of a critical state C(t) in a
given interval [b, e] we use the integral notion of mathematical calculus:

\[ \int_{b}^{e} C(t) \,dx \]	

Then the duration property can be expressed by a formula:


\[
$\forall$ b,e $\in$ Time $\bullet$ A(b,e) =\int_b^{e}C(t)\,\mathrm{d}t $\leq$  u(b,e)
\]
%%%%%%%%%%%%%%%%%%%%%%%%%%%%%%%%%%%%%%%%%%%%%%%%%%%%%%%%%%%%%%%%%
\paragraph{Andere duration properties}
Queries voor een  time based specificatie in Uppaal worden volgens literatuur \cite{04_giWorkshop2000} gedefinieerd als:

It is at all times possible that a weak sequence A with time interval(s) [x, y]
occurs
It is at all times possible that a weak sequence A with time interval(s) [x, y] does
not occur
It is at all times possible that a strong sequence A with time interval(s) [x, y]
occurs
It is at all times possible that an element of set A occurs within the interval [x, y]
. It is at all times possible that all elements of set A occur simultaneously within the
interval [x, y]
It is at all times possible that all elements of set A occur exclusively within the
interval [x, y]
It is at all times possible that an element of set A never occurs within the
interval [x, y
. It is at all times possible that all elements of set A never occur simultaneously
within the interval [x, y]
. It is at all times possible that all elements of set A never occur exclusively within
the interval [x, y]
 It is at all times true that if a strong sequence A with time interval(s) [x1, y1] occurs
then it must happen within [x2, y2] time unit(s) that an element of set B occurs
It is inevitable that if all elements of set A occur simultaneously within the
interval [x1, y1] then it is possible at some time later that a weak sequence B with
time interval(s) [x2, y2] occurs
. It is at all times true that if all elements of set A always occur simultaneously
within the interval [x1, y1] then it must happen in exactly [z] time unit(s) that all
elements of set B occur simultaneously within the interval [x2, y2]

AG EF_[_x_,_y_] $\vee$
%%%%%%%%%%%%%%%%%%%%%%%%%%%%%%%%%%%%%%%%%%%%%%%%%%%%%%%%%%%%%%%%%

\hoofdstuk{Conclusie}

Wat hebben alle bovenstaande rampen/ongelukken gemeen? Veiligheid.
Bij de therac waren er diverse problemen: communicatie, doorontwikkeling, controle en toetsing
Was het makkelijk te onderzoeken? Waarom?
Bij de boeing 737 crashes was het probleem van controle en communicatie naar medewerkers
Was het makkelijk te onderzoeken? Waarom?

Uit de evaluatie van de china explosion 2015 tianjin komt naar voren dat communicatie, transparantie en veiligheid niet altijd prioriteit hadden bij de lokale autoriteiten
Was het makkelijk te onderzoeken? Waarom?

Bij de tesla autopilot crashes komen soms onvoldoende onderbouwde ontwerpkeuzes naar voren die niet goed zij  afgewogen tegenover het gedrag van de bestuurder
vlucht 1951
Was het makkelijk te onderzoeken? Waarom?

De ramp in Tsjernobyl toont aan hoe autoriteiten een ramp in de doofpot proberen te stoppen
Was het makkelijk te onderzoeken? Waarom?



Wat heb ik geleerd?


Veiligheid is een vaag concept.Bij het onderzoeken van rampen is vaak wet- en regelgeving vastgesteld. Naast de machine is er ook een omgeving die in de veiligheidsanalyse moet worden meegenomen. In het voorkomen van ongelukken zijn opleiding en middelen  vaak doorslaggevend in het doorlopen van alle veiligheidsprocedures.

Moeilijker is om veiligheidsspecificaties vast te leggen in duidelijke taal. Nog moeilijker is deze logisch vast te leggen in predicaten voor een wiskundig model. Met name de tijds- en datavariabelen  vind ik moelijk om in een wiskundig model te plaatsen.

Over het algemeen is deze module een bijdrage aan de algemene kennis van een ontwikkelaar. Je leert op een kritische manier naar je requirementsengineeringproces te kijken en daaruit conclusies te trekken.



 %(verplicht) hoofdverslag
\hoofdstuk{Discussie}

discussie
geldigheidsgrenzen van de waarnemingen
betrouwbaarheid van de waarnemingen
waaarde van de waarnemingen
vergelijking van het oude en het nieuwe product/methode/apparaat volgens de genoende criteria. De gewijzigde factor maakt het product/methode/apparaat geheel/half/niet beter





Preconditions
Topography
By means of maps (land, water, river, sea, ownership, regional and zoning plans) a detailed description
of the environment should be provided, including any planned changes to existing situations, in so far
as this is of importance to the lock and adjoining lock approaches. Special attention should be paid to
historical, natural and scientific values. The maps should also show sewerage, cables and mains as well
as drainage facilities in the area concerned.
Existing lock (locks)
Water levels (approx.)
Wind
%%%%%%%%%%%%%%%%%%%%%%%%%%%%%%%%%%%%%%%%%%%%%%%%%%%%%%%%%%%%%%%%%
Morphology
Soil characteristics
Functional requirements
Functional requirements regarding navigation
%%%%%%%%%%%%%%%%%%%%%%%%%%%%%%%%%%%%%%%%%%%%%%%%%%%%%%%%%%%%%%%%%
General
Lock approaches
Primarily as part of the traffic management in locking
Stop over harbour
Harbour of refuge
Compulsory harbour
Hazardous substances
Leading jetties
Chamber and heads
The principal dimensions
The design
The facilities and equipment
Functional requirements regarding the water retaining (structure)
%%%%%%%%%%%%%%%%%%%%%%%%%%%%%%%%%%%%%%%%%%%%%%%%%%%%%%%%%%%%%%%%%
\newline \indent Dan zijn er nog de functionele eigenschappen.
Functional requirements regarding water management
General
Limiting water loss
Separation of salt and fresh water or clean and polluted water
Water intake and discharge
%%%%%%%%%%%%%%%%%%%%%%%%%%%%%%%%%%%%%%%%%%%%%%%%%%%%%%%%%%%%%%%%%
\newline \indent Functional requirements regarding the crossing, dry infrastructure
Roads
Cables and mains
%%%%%%%%%%%%%%%%%%%%%%%%%%%%%%%%%%%%%%%%%%%%%%%%%%%%%%%%%%%%%%%%%
\newline \indent  User requirements
%%%%%%%%%%%%%%%%%%%%%%%%%%%%%%%%%%%%%%%%%%%%%%%%%%%%%%%%%%%%%%%%%
\newline \indent Levels
Locking levels
Situating the lock
Accessibility
Smoothness and safety of dealing with traffic
Design levels
Normative High Water (NHW)
Locking level high water gate
%%%%%%%%%%%%%%%%%%%%%%%%%%%%%%%%%%%%%%%%%%%%%%%%%%%%%%%%%%%%%%%%%
\newline \indent Mogelijke voorkeur voor het scheiden van verschillende soorten vaten
Separation in using line-up area, waiting area and chamber
Separating vessels during locking
Separation of vessels during over night stop
Separation for use of the leading jetty (leidende steiger)
Leading jetty for seagoing vessels
Leading jetty for inland navigation
Leading jetty for recreational navigation
%%%%%%%%%%%%%%%%%%%%%%%%%%%%%%%%%%%%%%%%%%%%%%%%%%%%%%%%%%%%%%%%%
\newline \indent Mooring facilities in chamber and lock approach
Chamber
Lock approaches
Leading jetty
%%%%%%%%%%%%%%%%%%%%%%%%%%%%%%%%%%%%%%%%%%%%%%%%%%%%%%%%%%%%%%%%%
\newline \indent Operating times
%%%%%%%%%%%%%%%%%%%%%%%%%%%%%%%%%%%%%%%%%%%%%%%%%%%%%%%%%%%%%%%%%
\newline \indent Levelling times
%%%%%%%%%%%%%%%%%%%%%%%%%%%%%%%%%%%%%%%%%%%%%%%%%%%%%%%%%%%%%%%%%
\newline \indent Operational management
%%%%%%%%%%%%%%%%%%%%%%%%%%%%%%%%%%%%%%%%%%%%%%%%%%%%%%%%%%%%%%%%%
Process descriptions
%%%%%%%%%%%%%%%%%%%%%%%%%%%%%%%%%%%%%%%%%%%%%%%%%%%%%%%%%%%%%%%%%
Normal locking process
%%%%%%%%%%%%%%%%%%%%%%%%%%%%%%%%%%%%%%%%%%%%%%%%%%%%%%%%%%%%%%%%%
Obstructions
%%%%%%%%%%%%%%%%%%%%%%%%%%%%%%%%%%%%%%%%%%%%%%%%%%%%%%%%%%%%%%%%%
High water retaining structure
%%%%%%%%%%%%%%%%%%%%%%%%%%%%%%%%%%%%%%%%%%%%%%%%%%%%%%%%%%%%%%%%%
Intake/discharge
%%%%%%%%%%%%%%%%%%%%%%%%%%%%%%%%%%%%%%%%%%%%%%%%%%%%%%%%%%%%%%%%%
Salt /freshwater or clean/polluted water
%%%%%%%%%%%%%%%%%%%%%%%%%%%%%%%%%%%%%%%%%%%%%%%%%%%%%%%%%%%%%%%%%
Information for operational management
%%%%%%%%%%%%%%%%%%%%%%%%%%%%%%%%%%%%%%%%%%%%%%%%%%%%%%%%%%%%%%%%%
Procedures and facilities for negative operational situations
%%%%%%%%%%%%%%%%%%%%%%%%%%%%%%%%%%%%%%%%%%%%%%%%%%%%%%%%%%%%%%%%%
Power supply
Levelling%%%%%%%%%%%%%%%%%%%%%%%%%%%%%%%%%%%%%%%%%%%%%%%%%%%%%%%%%%%%%%%%%
Collisions
%%%%%%%%%%%%%%%%%%%%%%%%%%%%%%%%%%%%%%%%%%%%%%%%%%%%%%%%%%%%%%%%%
Too low/too high water levels and inspections
%%%%%%%%%%%%%%%%%%%%%%%%%%%%%%%%%%%%%%%%%%%%%%%%%%%%%%%%%%%%%%%%%
Problems with ice
%%%%%%%%%%%%%%%%%%%%%%%%%%%%%%%%%%%%%%%%%%%%%%%%%%%%%%%%%%%%%%%%%
\newline \indent Operating
Situating the control building
Local control facilities
Means of communication
Choice (partly) automated and self-service
Remote control of locks
%%%%%%%%%%%%%%%%%%%%%%%%%%%%%%%%%%%%%%%%%%%%%%%%%%%%%%%%%%%%%%%%%
\newline \indent Verlichting, signalering en boarding
Verlichting (for details, see Lit. [2.1])
Ship crews and operating personnel must take into account that comfort is decreased during locking that
takes place through the night. Given the decreased visibility and orientation, extra effort is required. This
effort has to be kept as low as possible in order to prevent decreased safety. For this purpose, suitable
and economically sound illumination of the lock complex is essential.
The lighting has to be geared to the ever-increasing use of central control at locks and has to be aimed
at places where activities (manoeuvres, tying and untying, going on land) are executed.
The locations drawing the attention of the individual captain for instance, are the free area, the line-up
and waiting area, the chamber entrance, the chamber, lock grounds, chamber exit and the outlet area
to the unlit waterway. The attention of operating personnel will particularly focus on the vessels in the
line-up and waiting areas, inbound vessels, the chamber, the gates, the lock grounds and the sailing of
outbound vessels.
Given the necessity of illuminating the lock and lock approaches, a number of general minimum conditions
are set. This illumination is compulsory and could be included in the design plan:
• a clear view of the lock complex has to be provided for the benefit of orientation from the water;
• the illumination has to be sufficiently even;
• during arrival and departure dazzling, which is often caused by excessive glare of lock parts because
of cameras etc., should be prevented;
• in the control building the illumination should be adjusted to the outside environment and images
recorded as TV pictures should have such contrast and definition that the operating personnel is given
sufficient information;
• uniformity in the illumination plan for the setup of light towers, height of points of light and light
colour is desired.
In Lit. [2.1], as extension of these conditions, a number of specific recommendations are made that are
of importance to the design.

Scheepsbemanningen en bedienend personeel moeten er rekening mee houden dat het comfort tijdens het schutten afneemt
vindt de hele nacht plaats. Gezien de verminderde zichtbaarheid en oriëntatie is extra inspanning vereist. Dit
inspanning moet zo laag mogelijk worden gehouden om verminderde veiligheid te voorkomen. Voor dit doel geschikt
en economisch verantwoorde verlichting van het sluizencomplex is essentieel.
De verlichting moet zijn afgestemd op het steeds toenemende gebruik van centrale bediening bij sluizen en moet gericht zijn
op plaatsen waar werkzaamheden (manoeuvres, vast- en losmaken, aan land gaan) worden uitgevoerd.
De locaties die bijvoorbeeld de aandacht trekken van de individuele kapitein zijn de vrije ruimte, de opstelling
en wachtruimte, de kolkingang, de kolk, het sluisterrein, de kolkuitgang en het uitloopgebied
naar de onverlichte waterweg. De aandacht van het bedienend personeel zal met name gericht zijn op de schepen in de
opstel- en wachtruimtes, inkomende schepen, de kolk, de deuren, het sluisterrein en het uitvaren
uitgaande schepen.
Gezien de noodzaak van verlichting van de sluis en sluistoegangen gelden een aantal algemene minimumvoorwaarden
spelen zich af. Deze verlichting is verplicht en kan in het inrichtingsplan worden opgenomen:
• er moet vrij zicht zijn op het sluizencomplex ten behoeve van de oriëntatie vanaf het water;
• de verlichting moet voldoende egaal zijn;
• bij aankomst en vertrek verblinding, wat vaak wordt veroorzaakt door overmatige verblinding van sluisdelen doordat
van camera's e.d. moet worden voorkomen;
• in het controlegebouw dient de verlichting afgestemd te zijn op de buitenomgeving en beelden
opgenomen als tv-beelden moeten zo'n contrast en definitie hebben dat het bedienend personeel wordt gegeven
voldoende informatie;
• uniformiteit in het verlichtingsplan voor de opstelling van lichtmasten, hoogte van lichtpunten en lichtpunten
kleur is gewenst.
In Lit. [2.1] In het verlengde van deze voorwaarden worden een aantal specifieke aanbevelingen gedaan die dat wel zijn
belangrijk voor het ontwerp.
%%%%%%%%%%%%%%%%%%%%%%%%%%%%%%%%%%%%%%%%%%%%%%%%%%%%%%%%%%%%%%%%%
Vereist verlichtingsniveau
For the average value of illumination intensity on horizontal surfaces of the above-mentioned lock
parts, 10 lux is adhered to. On vertical surfaces that are more often more striking due to the perpendicular
directional view, a lower value of 3.5 lux can be used.
At a number of critical parts of the lock (both for the captain and the lock master) a larger contrast is
desired and can be achieved by stronger illumination of areas that should be in the light or providing
these with white markings. The latter is preferable. At critical lock parts such as gates and leading
jetties, the vertical illumination strength should be higher: 7 lux. On the chamber and mooring area
where accurate visibility is required, the previously stated values of 10 lux for horizontal and 3.5 lux
for vertical apply. The waiting area and the free area, where illumination is mostly for orientation,
require an illumination level of 5 lux horizontal respectively 3.5 lux vertical.

Voor de gemiddelde waarde van de verlichtingsintensiteit op horizontale oppervlakken van het bovengenoemde slot
onderdelen wordt 10 lux aangehouden. Op verticale vlakken die door de loodlijn vaker opvallender zijn
gericht zicht kan een lagere waarde van 3,5 lux worden gebruikt.
Op een aantal kritische onderdelen van de sluis (zowel voor de gezagvoerder als de sluismeester) is een groter contrast
gewenst en kan worden bereikt door sterkere verlichting van gebieden die in het licht moeten staan of moeten worden voorzien
deze met witte aftekeningen. Dit laatste heeft de voorkeur. Bij kritische sluisdelen zoals poorten en voorloop
aanlegsteigers dient de verticale verlichtingssterkte hoger te zijn: 7 lux. Op de kamer en het ligplaatsgebied
waar nauwkeurig zicht vereist is, de eerder genoemde waarden van 10 lux voor horizontaal en 3,5 lux
voor verticale toepassing. De wachtruimte en de vrije ruimte, waar de verlichting vooral ter oriëntatie is,
vereisen een verlichtingsniveau van 5 lux horizontaal respectievelijk 3,5 lux verticaal.
%%%%%%%%%%%%%%%%%%%%%%%%%%%%%%%%%%%%%%%%%%%%%%%%%%%%%%%%%%%%%%%%%
Omgevingsverlichting en begeleiding
Misleading illumination in the surrounding area can give the captain a wrong picture of the course of
the waterway that provides access to the lock chamber. This can be prevented if the waterway or the
lock complex is illuminated over a sufficient length or by adapting the surrounding illumination to the
illumination of the complex. For visual guidance, differences in illumination strength at crossings
should not exceed a factor 2.
%%%%%%%%%%%%%%%%%%%%%%%%%%%%%%%%%%%%%%%%%%%%%%%%%%%%%%%%%%%%%%%%%
Uniformiteit
For the uniformity (E) of the illumination, a minimum value of Emin/Emax = 0.3 should be adhered
to for both vertical and horizontal areas.
%%%%%%%%%%%%%%%%%%%%%%%%%%%%%%%%%%%%%%%%%%%%%%%%%%%%%%%%%%%%%%%%%
Glare
Unsafe situations due to dazzling should be avoided. The correct combination of armature, lamp and
positioning is of importance.
%%%%%%%%%%%%%%%%%%%%%%%%%%%%%%%%%%%%%%%%%%%%%%%%%%%%%%%%%%%%%%%%%
Kleurherkenning en soort lamp
The colour of the light is one of the factors in the recognition of boards and signalling. Both white
and yellow light can be used.
In the lamp choice of illumination, both high-pressure and low-pressure lamps as well as energy
saving lamps qualify. In the application of low-pressure (monochromatic) sodium (vapour) light,
colour recognition is impossible. If this is the case, separate illumination of traffic signs is recommended.

De kleur van het licht is een van de factoren bij de herkenning van borden en signalering. Beide wit
en geel licht kan worden gebruikt.
Bij de lampkeuze van verlichting, zowel hogedruk- en lagedruklampen als energie
spaarlampen komen in aanmerking. Bij de toepassing van lagedruk (monochromatisch) natrium (damp) licht,
kleurherkenning is onmogelijk. In dat geval is het aan te raden om verkeersborden apart te verlichten.
%%%%%%%%%%%%%%%%%%%%%%%%%%%%%%%%%%%%%%%%%%%%%%%%%%%%%%%%%%%%%%%%%
Marking
White markings are a good and inexpensive tool for obtaining sufficient contrast in the dark while using
little light. Marking vertical surfaces, such as guiding structures and guard walls, to support the visual
guidance of navigation is very effective.

Witte aftekeningen zijn een goed en goedkoop hulpmiddel om tijdens het gebruik voldoende contrast in het donker te krijgen
klein licht. Markering van verticale oppervlakken, zoals geleideconstructies en veiligheidsmuren, ter ondersteuning van het visuele
begeleiding van navigatie is zeer effectief.
%%%%%%%%%%%%%%%%%%%%%%%%%%%%%%%%%%%%%%%%%%%%%%%%%%%%%%%%%%%%%%%%%
Signalling
Signalling should be executed according to the stipulations of the Police Regulations on Inland
Navigation (‘Binnenvaart Politie Reglement’ (BPR))and the Rhine Navigation Police Regulations
(‘Rijnvaart Politie Reglement’ (RPR)), (Lit. [2.4]).Signal indication and lock illumination choices should be
adjusted to terrain illumination of the lock for the benefit of colour recognition; it should have sufficient
attention value.

De seingeving dient te worden uitgevoerd volgens de bepalingen van het Politiereglement Binnenvaart
Scheepvaart (Binnenvaart Politie Reglement (BPR)) en het Rijnvaartpolitiereglement
(‘Rijnvaart Politie Reglement’ (RPR)), (Lit. [2.4]). Keuzes voor signaalindicatie en slotverlichting moeten
aangepast aan terreinverlichting van de sluis ten behoeve van kleurherkenning; het zou voldoende moeten hebben
attentie waarde.
%%%%%%%%%%%%%%%%%%%%%%%%%%%%%%%%%%%%%%%%%%%%%%%%%%%%%%%%%%%%%%%%%
Boarding
Boards should be executed in accordance with the stipulations of the BPR and RPR, (Lit. [2.4]).The colour
recognition could be (substantially) reduced due to the terrain illumination. Sufficient attention should
be paid to adjusting the illumination or to separate board illumination.
%%%%%%%%%%%%%%%%%%%%%%%%%%%%%%%%%%%%%%%%%%%%%%%%%%%%%%%%%%%%%%%%%
Verlichtingsplan
The user requirements for illumination should be incorporated in an illumination design plan.
The chamber depth (distance between low normative water level and the lock coping) and the chamber
width are of great importance. In Lit. [2.1] examples are provided for a number of chamber width
categories (5-13 m, 13-20 m, 20-24 m, larger than 24 m; chamber depth about. 5 m) of the resulting
illumination characteristics (such as illumination strength and uniformity), departing from the relationship
between lock design and the given characteristics of illumination installation (such as positioning
and illumination facilities).
%%%%%%%%%%%%%%%%%%%%%%%%%%%%%%%%%%%%%%%%%%%%%%%%%%%%%%%%%%%%%%%%%
\newline \indent Stroomvoorziening
Emergency power supply is required for vital parts of the installation so that, in case of malfunction,
it can automatically take over the energy supply within minutes. A no-break facility is required for
installation parts that lose data in case of power loss. In addition, emergency lights should be present.

In essence, power is obtained from the public network. In consultation with the local power company,
assessments have to be made about where this is possible and whether the connection contains
sufficient capacity or whether this will have to be adjusted. Of importance is the total capacity required,
voltage variations and frequency of the energy to be supplied. In addition to capacity for lock operation,
the capacity for construction (civil and steel) will have to be determined. It could be taken into consideration
whether the cables for construction could later become part of the supply for the lock.
The lock complex should contain the necessary facilities for high tension, transformers and low-tension
equipment. In addition, room is reserved and facilities provided for cable location lines from the low-tension
area to the various lock parts (cable racks, cable channels, cable shafts, lead-through pipes etc.)
Take into account the other cables and mains required for lock operation as well as those for third parties
(Par. 2.3.4.2). For emergency power supply generators and no-break installations, see Par. 2.4.6.3.

Noodvoorzieningen voor stroomtoevoer i vereist voor bepaalde delen van de installatie, in geval van een storing kan deze binne enkele minuten leveren.

Een no-break facicilieit is vereist voor de ondeerdelen die data verliezen in gevala van sstroomuitval.
Het sluizencomplex moet gaciliteiten hebben voor hoogspanning, transformatoren en laagspanningsapparatuur.
%%%%%%%%%%%%%%%%%%%%%%%%%%%%%%%%%%%%%%%%%%%%%%%%%%%%%%%%%%%%%%%%%
\newline \indent Beschikbaarheid
Introduction
Causes of non-availability
Water levels above and below locking levels
Guidelines on the boundaries of locking levels are provided in Par. 2.4.1.1 (maximum and minimum
locking levels). Overall, this results in non-availability smaller than 2% of the time.
The specific boundaries should be set on economic grounds.
Too much wind, bad visibility

De beschikbaarheid van een sluis kan beinvloed worden door een te hoog  waterlevel boven de sluis.
Dan is er nog de mogelijkheid op te veel wind en slecht zicht.
%%%%%%%%%%%%%%%%%%%%%%%%%%%%%%%%%%%%%%%%%%%%%%%%%%%%%%%%%%%%%%%%%
Storingen aan installaties, bedieningsmechanismen en werking. Er moeten oplossingen komen  zodat ern signalen worden gegeven wanneer een storing zich voordoet, een betre reactie op signalen en  reserveonderdelen.
Based on the previously mentioned economic considerations, requirements will have to be drafted for
the design of the lock or the series of locks for the acceptable risk of failure of these facilities. As an
example, the values applied for the renovation of the ‘Zuider- en de Kleine sluisin IJmuiden’ are stated
(Lit. [2.13]). Not available due to:
• malfunction installations : ≤ 0,5% of the time
• malfunction operating mechanisms : ≤ 0,5% of the time
• malfunction operation : ≤ 0,25% of the time
The number of times that malfunction occurs could also be a determining factor.
Not every malfunction results in complete obstruction. The objective is to limit the duration of the malfunction
as much as possible (alerting, responding, spare parts).
For emergency power supply and no-break installations, please see Par. 2.4.6.3.

%%%%%%%%%%%%%%%%%%%%%%%%%%%%%%%%%%%%%%%%%%%%%%%%%%%%%%%%%%%%%%%%%
Botsingen
For non-availability due to collisions, at best a forecast can be made, based on the information available
for similar locks with a corresponding navigation volume. As an example, the ‘Zuidersluis bij IJmuiden’
(Lit. [2.13]) is mentioned, where the non-availability due to significant damage due to collisions amounted
to 17 hours per annum (about 0.2% of the time). Other locks could provide a different picture.
Within economically acceptable boundaries, the objective will be to limit the collisions and consequences
thereof. The accent is placed on gates (and operating mechanisms), moveable bridges and – to a
lesser degree – on berthing jetties and guide structures.
Measures to decrease risk of collision are, among others:
• good design of approach jetties (Par. 2.3.1.3 and 2.4.2.2);
• positioning of the flooring of moveable bridges – in opened condition – outside the outer walls of the
lock (Par. 2.3.4.1);
• anti-collision structures in front of the gates (Par. 2.4.11.1). This is an expensive facility that will only
be applied in special cases;
• protection of operating mechanism on gates. Preventing collisions with the operating mechanism can
be effected by fitting a tail end to the gate and connecting this to the operating mechanism
(Renovation Oranjesluizen). An extended operating mechanism chamber could also be used so that
the vulnerable cylinder rod cannot be hit in the lock (Middensluis IJmuiden).
Measures to limit the duration of the repairs (obstruction) are, among others, having the spare gates and
spare parts available (Par. 2.5.2 en 2.5.3).
Maatregelen om het aantal botsingen te voorkomen zijn:
Good ontwerp voor aanvaarstijgers.
positionering van de vloer van beweegbare poorten
anti-bots structuren aan de voorkant van de sluisdeuren
bescherming van werkende meschanismen van de slusdeuren

Maintenance
%%%%%%%%%%%%%%%%%%%%%%%%%%%%%%%%%%%%%%%%%%%%%%%%%%%%%%%%%%%%%%%%%
\newline \indent Constructies beschermen tegen schade
%%%%%%%%%%%%%%%%%%%%%%%%%%%%%%%%%%%%%%%%%%%%%%%%%%%%%%%%%%%%%%%%%
Aanrijdbeveiliging voor poorten
Mitre gates and pivot (leaf) gates must be fitted with wood fender on the outside surfaces of the
opened gates to protect the construction from damage caused by inbound and outbound vessels. Wood
fender can also be fitted to other gates in places where they might be hit by vessels.
In special circumstances (for instance Wijk bij Duurstede, Tiel, Belfeld, Panheel, Twenthe-kanaal) trap
constructions are positioned in front of the closed gates. The energy of vessels that do not stop in time
is absorbed here and the construction prevents the gates from being hit (see par. 17.3.3). For this purpose,
cables (cable nets) and friction drums can be used. For the circumstances and setup of these constructions,
we refer to Lit. [2.15]. It does concern expensive constructions for which the investments will
have to be weighed against the risk of failure of the water retaining structure, the navigation interests
etc.
Anti-collision devices protecting lock gates could be economically sound at high-lift locks.

Verstekpoorten en draaipunt. In bijzondere reegvallen staan er valconstructies bij de gesloten poorten voor vaartuigen die niet op tijd stoppen. zodoende wordt de klap opgevangen.
Anti-bots apparaten die de sluisdeuren beschermen  zijn economisch verantwoor bij hoge liftsluizen.
%%%%%%%%%%%%%%%%%%%%%%%%%%%%%%%%%%%%%%%%%%%%%%%%%%%%%%%%%%%%%%%%%
Aanrijdbeveiliging voor beton- en damwandconstructies
Construction surfaces against which vessels moor or along which they shave, have to be as smooth as
possible in order to guide well and limit potential damage (construction and vessel). For inland navigation,
a concrete structure meets the requirements. In the case of other construction materials such as
sheet pile, the flat surface should be made of wooden or synthetic posts and rails wherever possible. This
system can be limited to the day surfaces that vessels meet.
Constructievlakken waar schepen aanmeren of waarlangs ze scheren, moeten zo glad mogelijk zijn
mogelijk om goed te begeleiden en mogelijke schade (constructie en vaartuig) te beperken. Voor de binnenvaart,
een betonconstructie voldoet aan de eisen. In het geval van andere bouwmaterialen zoals
damwand, het vlakke oppervlak dient zoveel mogelijk te bestaan uit houten of kunststof palen en rails. Dit
systeem kan worden beperkt tot de dagoppervlakken die schepen ontmoeten.

Additional facilities are necessary in places where concrete surfaces are interrupted or come to an end
because of expansion joints, gate and ladder recesses. In the case of expansion joints, it will be sufficient
to use (sizeable) bevelled edges, steel corner protection profiles should be applied in recesses. Corner
guards made of tropical hardwood can also be fitted, especially where it concerns rugged navigation
such as tug-pushed dumb barges and sea-going vessels. As protection from hawsers etc, the top of the
wall should be fitted with steel capstone profiles. In locks for large ocean going vessels, floating wooden
frames (the Netherlands) or rubber wheel fenders (Belgium) are used.
Op plaatsen waar betonvlakken worden onderbroken of ophouden, zijn aanvullende voorzieningen nodig
vanwege dilatatievoegen, poort- en ladderuitsparingen. In het geval van dilatatievoegen is dit voldoende
om (flinke) afgeschuinde randen te gebruiken dienen stalen hoekbeschermingsprofielen in uitsparingen te worden aangebracht. Hoek
ook beschermkappen van tropisch hardhout kunnen worden aangebracht, zeker als het om ruige navigatie gaat
zoals sleepboten en zeeschepen. Als bescherming tegen trossen enz., de bovenkant van de
wand dient voorzien te zijn van stalen deksteenprofielen. In sluizen voor grote zeeschepen, drijvend van hout
frames (Nederland) of rubberen wielspatborden (België) worden gebruikt.

The facilities are intended to minimize damage to vessels and constructions, but also to prevent backing
up and friction effects during mooring and unmooring of vessels with large side surfaces, thereby
decreasing the pass through time.

%%%%%%%%%%%%%%%%%%%%%%%%%%%%%%%%%%%%%%%%%%%%%%%%%%%%%%%%%%%%%%%%%
Voorzieningen tegen vandalisme
Lightning protection
%%%%%%%%%%%%%%%%%%%%%%%%%%%%%%%%%%%%%%%%%%%%%%%%%%%%%%%%%%%%%%%%%
Safety
Voorzieningen voor drenkelingen
For rescuing people who accidentally end up in the water, ladders should be fitted to the chamber wall
and to (high) smooth walls in the lock approach. At the upper end, these ladders are equipped with
handgrips. For offering help from the quayside, life-saving devices (life buoy, hooks) should be present
on the lock coping in a clearly visible place. Ladders in the chamber and the lock approach also have an
accessibility function. For locations and distances, also see par. 2.4.13.2 and 2.4.13.3.
Voor het redden van drenkelingen moetn er ladders zijn.
%%%%%%%%%%%%%%%%%%%%%%%%%%%%%%%%%%%%%%%%%%%%%%%%%%%%%%%%%%%%%%%%%
Veiligheidsvoorzieningen
Design and management of safety facilities of personnel will be executed in accordance with Health and
Safety Regulations, construction regulations, labour regulations and safety regulations (CE directives).
A number of facilities are mentioned below.
Railings are attached to the top of gates. If the lock coping is more than 2.5 m above minimum locking
level, fencing is placed behind the bollards. This fencing is always desirable where it concerns recreational
navigation and where tourists are allowed on the lock coping.
In the technical areas, workshops, bridges, control portals, rolling gate casings and the like, where work
is executed and people walk around where there are differences in height in the surrounding area,
railings are provided. From a height difference of 0.60 m or more with the surrounding area, a railing
has to be provided at 1 – 1.10 m. Height differences of more than 12 m require the railing to be placed
at a height of 1.20. Often, additional protection against falling is provided from height differences of
more than 2.5 m such as safety lines, lifelines, harness belts and the like.

Steel ladders should not be in regular use. Straight stairs, a spiral staircase or step ladders should be
installed. Ladders can be used between vertical (90o) and 75o and be equipped with simple round rungs.
The ladder width is between 0.38 and 0.46 m and the step distance is between 0.25 – 0.20 m.
If the ladder connects with the (landing) coping, the distance between the styles of the ladder should be
enlarged to 0.60 and it has to be connected to the railing. If the ladders are higher than 3.60 m, they
have to be provided with a safety cage. This cage has an inside measurement of 0.76 m and starts from
2.40 m above the ground. At ladder heights above 6 m, an intermediate landing is required.

Basement chambers that could possibly flood (for instance those of operating mechanisms of mitre
gates) have to be provided with an exit that can be opened from the inside. In addition, sufficient
natural ventilation will be required as well as plunger pumps.
The area in which the operating mechanisms are working need to be shielded from the environment to
ensure that nobody gets stuck between machine parts. The lock complex should have sufficient and visible
First Aid provisions.

Ontwerp en beheer van veiligheidsfaciliteiten voor personel worden uitgevoerd in overeenstemming met de gezondheids-veiligheidsregelgeving, constructieregelgeing,arbeidsregelgeving en veilgieheidsregeveving. Enkele voorbeelden zijn traliewerk, hekwerk, stalen ladders, kelder kamers en eerste hulp kits
%%%%%%%%%%%%%%%%%%%%%%%%%%%%%%%%%%%%%%%%%%%%%%%%%%%%%%%%%%%%%%%%%
Brand blussen
Toegankelijkheid van sluis en sluistoegangen
Lock Infrastructure
Accessibility of vessels in the lock
Accessibility of vessels in the lock approache
Accessibility of vessels with dangerous goods in the lock approaches
%%%%%%%%%%%%%%%%%%%%%%%%%%%%%%%%%%%%%%%%%%%%%%%%%%%%%%%%%%%%%%%%%
\newline \indent Supplemental client wishes
%%%%%%%%%%%%%%%%%%%%%%%%%%%%%%%%%%%%%%%%%%%%%%%%%%%%%%%%%%%%%%%%%
\newline \indent Eisen aan de levensduur
Ontwerp levensduur sluizencomplex
Steel parts
Electrical installations
Hardware and software
Damwand constructies
Leidende structuren
%%%%%%%%%%%%%%%%%%%%%%%%%%%%%%%%%%%%%%%%%%%%%%%%%%%%%%%%%%%%%%%%%
Maintenance requirements
Maintenance strategy
The maintenance strategy will mainly be based on the requirements regarding the safety of the retaining
structure (par. 2.3.2), the availability for lock operation (par 2.4.10) and the life span (2.4.15). The
external appearance of the structure will also play a role in the strategy (building inspection). With the
exception of the safety requirements, which are fixed, it concerns an assessment between the aggregate
costs of investments and capitalized maintenance, and the interest of obstructions for navigation. An
example is to consider applying 2 horizontal roller-bearing gates per head for a maritime navigation lock
(par. 2.5.2). The optimization of the materials, maintenance choices etc. within the given design life span
of a lock is discussed in par 2.4.15. Environmental requirements necessitate certain maintenance activities
to be executed in closed areas. Providing these facilities on site could be costly and it could be attractive
to have these activities executed by third parties.
Overall, the objective is to incur a minimum of aggregate costs as well as provide the largest service
provision to navigation. The latter includes a limitation of the number and duration of obstructions for
maintenance (par. 2.4.10) and attention for limited passage during maintenance. Please refer to the
modules of ‘Raamwerk Onderhoud van Natte Kunstwerken’ (Lit. [2.20]), which is drafted by the Civil
Engineering Division of the Ministry of Transport and Public Works. At present, the following modules
are available: "Keuze van onderhoud voor een puntdeur", "Damwanden" and "Ducdalven en remmingwerken".
Based on this strategy, maintenance plans, books and schedules will have to be drafted for the various
parts. Supplemental to this, measures and procedures for navigation during maintenance will have to be
drafted.

De onderhoudstrategie wordt bepaald of basis van de vasthoudende structuur, de mogelijkheden voor sluisbediening en de levenscyclus. het uiterlijk van de structuur speelt een belangrijke rol bij de strategie, namelijk gebouw inspectie. Bahalve veiligheidseisen gaat dit over de toetsing van  de totale investeringskosten en noodzakelijk onderhoud, en de behoefte aan belemmering voor nativatie. Het optimaliseren van materialen, onderhoudskeuzen binnen de levenscyclus van de sluis. Omgevingseisen maken het nodg onderhoudsactiviteten uit  te voeren in afgesloten ruimten.
%%%%%%%%%%%%%%%%%%%%%%%%%%%%%%%%%%%%%%%%%%%%%%%%%%%%%%%%%%%%%%%%%
2.5.2 Spare
Reserve poorten
Onderdelen en materialen
%%%%%%%%%%%%%%%%%%%%%%%%%%%%%%%%%%%%%%%%%%%%%%%%%%%%%%%%%%%%%%%%%
Slot openleggen (of niet)
Nowadays, it is no longer usual to lay open the complete lock for maintenance. The reasons are that it
is often too costly (measures required against floating up) and that the main construction of chamber
and heads are maintenance free, the probable exception being wood fenders for sheet pile constructions
and floating frames at sea locks. The latter parts should be easy to replace. Incidental repairs to head
constructions could be executed by divers or in diving bells.
Inspection and maintenance focus on gate supports (sill and side seals), fulcrums, and gate conduction,
in other words, parts that are located in the head. There are two possibilities:
1. Lying open a head, for which stop log weirs or dewatering weirs and rabbets are necessary.
2. Removable pivot-inspection chambers and other local steel dewatering means for the fulcrums,
support and gate condition. This also includes the dewatering stop logs for the gate recesses for lift
and roller-bearing gates.
Gate supports and rabbets are also required for the drainage. These means for water removal are stored
in the near vicinity in a highly accessible place and could possibly be used for several locks.
The choice between two possibilities depends on the inspection and maintenance frequency, the costs
and the duration of the obstruction for navigation. Option 1, in which too much space is laid open is, in
essence, usually only applied at smaller locks.

Tegenwoordig is het niet meer nodig om een complete sluis open te leggen voor onderhoud. De reden is dat dit duur is en dat de hoofd constructie van de kamer en hoofden  vrij zijn van onderhoud afgezien van houden spatborden voor damwandbouwers en zwevende kozijnen.

Poortsteuningen en ponningen zijn nodig voor de drainage
%%%%%%%%%%%%%%%%%%%%%%%%%%%%%%%%%%%%%%%%%%%%%%%%%%%%%%%%%%%%%%%%%
Toegankelijkheid voor het personeel
%%%%%%%%%%%%%%%%%%%%%%%%%%%%%%%%%%%%%%%%%%%%%%%%%%%%%%%%%%%%%%%%%
Monitoring( Toezicht houden)
Monitoring is a permanent measuring and registration system for normative parameters for the condition
of structures, the loads and stresses that they are submitted to and the degree in which corrosion
processes have progressed. Even though the application in construction is still limited, it is necessary to
keep up with the rapid developments. Monitoring is useful, certainly for places of lock structures that are
difficult to inspect (for instance at soil facing side) and for erosion processes that are hardly visible on the
surface (such as chloride penetration).
Monioren betekent het permanent meten en registeratie systeemvoor normatieve parameters voor de conditie van de structuren,ladingen. Het iis belangrijk alle ontwikkelingen in de gaten te houden. Monitoren is nuttig, zeker voor onderdelen van de sluis die moeilijk te inspectiren zijn zoals de bodem en voor erosie processen die moelijk zichtbaar zijn vanaf het oppervlak.
Cathodic protection can be used as a monitoring system at the same time.
Electrical installation, hard- en software
Storage areas and workshops
Environmental requirements in the use phase
Aesthetics
%%%%%%%%%%%%%%%%%%%%%%%%%%%%%%%%%%%%%%%%%%%%%%%%%%%%%%%%%%%%%%%%%
\newline \indent In ons model houden we geen rekening met omgevingseisen zoals de materialen gebruiket voor de bouw, recreatie, bodemvervuiling, grondwaterverlies. Oo is er geen rekening gehouden met verkeer, communicatiekabels onderwater en netspanningskabels.

Environmental requirements with regard to building materials
Recreation
Environmental requirements in the construction phase
Required building site and final grounds
Polluted soil
Groundwater withdrawal
Upkeep/maintenance of road and navigation traffic, cables and mains
Upkeep/maintenance of the water retaining structure
%%%%%%%%%%%%%%%%%%%%%%%%%%%%%%%%%%%%%%%%%%%%%%%%%%%%%%%%%%%%%%%%%
\newline \indent Permits and procedures at the construction of a lock
Construction permits and zoning plan amendments
Demolition permit
Flood Defence Act
Environmental Management Act (M.E.R.)
Act on Earth Removal
Pollution of Surface Waters Act
Groundwater Act permit
Water management Act
Soil Protection Act
Nature Conservation Act
Management of Waterways and Public Works Act (Wet beheer RWS-werken)
Noise Abatement Act
Provincial Road Ordinance
Building Materials (Soil and Surface Waters Protection) Decree
Other permits and exemptions
Standards and guidelines
Standards
Guidelines
%%%%%%%%%%%%%%%%%%%%%%%%%%%%%%%%%%%%%%%%%%%%%%%%%%%%%%%%%%%%%%%%%
\paragraph{Checklist}
   %(verplicht) geaggregeerde conclusies en aanbevelingen
%\hoofdstuk{Aanbeveling}
aanbeveling
adviezen en richtlijnen om de nieuw verkregen kennis in de praktijk toe te passen



Theoretical implication

Practical implication

Authors' contributions

Data availability statement

Declarations

Footnotes

Contributor Information



\cite{oid}   %(verplicht) geaggregeerde conclusies en aanbevelingen
\ifpublic
  \iflanguage{dutch}{\def\bibname{\normalsize{Bronnen}}}
                    {\def\bibname{\normalsize{References}}}
  {\footnotesize{
\begin{thebibliography}{99}

\bibitem{lam1994} Lamport L.: \emph{\LaTeX: A Document Preparation System}, Addison-Wesley, 1994 

\bibitem{Oos1996} Oostrum van P.: \emph{Handleiding \LaTeX}, Vakgroep
  Informatica, Universiteit Utrecht, 1998,\\ 
  \url{http://people.cs.uu.nl/piet/latexhnd.pdf}

\bibitem{wikibooks} Wikibooks \LaTeX:\\
  \url{http://nl.wikibooks.org/wiki/LaTeX}


\bibitem{oid} Wikibooks \LaTeX:\\
\url{https://www.waterkant.net/suriname/2023/05/29/milieuactivist-sleur-zeer-grote-onwaarheden-staan-in-cyanide-onderzoeksrapport/}

   


%%%%%%%%%%%%%%%%%%%%%%%%%%%%%%%%%%%%%%%%%%%%%%%%%%%%%%%%%%%%%%%%%
%
%
%misc{para,
%	Author = {{IEEE Referencing}},
%	Howpublished = {\url{http://libguides.bhtafe.edu.au/content.php?pid=88814&sid=660920}},
%	Keywords = {paraphrasing, plagariasm, citation, quotation},
%	Lastchecked = {19/04/2017},
%	Title = {{How to Cite\/Quote in Your Assignment}}}
%
%misc{ieee-write,
%	Author = {{IEEE Authorship Series}},
%	Howpublished = {\url{http://ieeeauthorcenter.ieee.org/wp-content/uploads/How-to-Write-for-Technical-Periodicals-and-Conferences.pdf}},
%	Institution = {{IEEE}},
%	Keywords = {writing, paragraph, sentences, ethics},
%	Lastchecked = {18/04/2017},
%	Title = {How to Write for Technical Periodicals \& Conferences}}
%
%
%
%@article{schon2017agile,
%	title={Agile Requirements Engineering: A systematic literature review},
%	author={Sch\"on, Eva-Maria and Thomaschewski, J\"org and Escalona, Maria Jose},
%	journal={Computer Standards \& Interfaces},
%	volume={49},
%	pages={79--91},
%	year={2017},
%	publisher={Elsevier}
%}
%
%@inproceedings{royce1987managing,
%	title={Managing the development of large software systems: concepts and techniques},
%	author={Royce, Winston W},
%	booktitle={Proceedings of the 9th international conference on Software Engineering},
%	pages={328--338},
%	year=1987,
%	organization={IEEE Computer Society Press}
%}
%
%@article{leveson1993investigation,
%	title={An investigation of the Therac-25 accidents},
%	author={Leveson, Nancy G and Turner, Clark S},
%	journal={IEEE computer},
%	volume={26},
%	number={7},
%	pages={18--41},
%	year={1993}
%}
%@book{modelchecking,
%	author = {Clarke,Jr., Edmund M. and Grumberg, Orna and Peled, Doron A.},
%	title = {Model Checking},
%	year = {1999},
%	isbn = {0-262-03270-8},
%	publisher = {MIT Press},
%	address = {Cambridge, MA, USA},
%}
%@inproceedings{nuseibeh2000requirements,
%	title={Requirements engineering: a roadmap},
%	author={Nuseibeh, Bashar and Easterbrook, Steve},
%	booktitle={Proceedings of the Conference on the Future of Software Engineering},
%	pages={35--46},
%	year={2000},
%	organization={ACM}
%}
%
%
%%%%%%%%%%%%%%%%%%%%%%%%%%%%%%%%%%%%%%%%%%%%%%%%%%%%%%%%%%%%%%%%%%
%
%@online{inriaStatsMoodCheck,
%	ALTauthor = {project.inria},
%	ALTeditor = {editor},
%	title = {statistical-model-checking},
%	date = {date},
%	url = {https://project.inria.fr/plasma-lab/statistical-model-checking/},
%}
%
%
%@online{buddeModelChecker,
%	ALTauthor = {Carlos E. Budde1, Pedro R. D’Argenio2,3,4, Arnd Hartmanns1(B) , and Sean Sedwards5},
%	ALTeditor = {editor},
%	title = {A Statistical Model Checker for Nondeterminism and Rare Events},
%	date = {date},
%	url = {https://ris.utwente.nl/ws/portalfiles/portal/28200786/A_statistical_model_checker.pdf},
%}
%
%
%@online{ID,
%	ALTauthor = {GUL AGHA and  },
%	ALTeditor = {KARL PALMSKOG},
%	title = {A Survey of Statistical Model Checking
%	},
%	date = {date},
%	url = {https://dl.acm.org/doi/10.1145/3158668},
%}
%
%
%
%@online{AGHASuervey,
%	ALTauthor = {GUL AGHA and KARL PALMSKOG,},
%	ALTeditor = {editor},
%	title = {A Survey of Statistical Model Checking},
%	date = {date},
%	url = {https://dl.acm.org/doi/pdf/10.1145/3158668},
%}
%%%%%%%%%%%%%%%%%%%%%%%%%%%%%%%%%%%%%%%%%%%%%%%%%%%%%%%%%%%%%%%%%%
%
%
%
%@online{bicker21102016automatiseringsparadox,
%	ALTauthor = {author},
%	ALTeditor = {editor},
%	title = {title},
%	date = {date},
%	url = {https://www.debicker.eu/de-automatiseringsparadox/},
%}
%@online{vseautoparadox,
%	ALTauthor = {author},
%	ALTeditor = {editor},
%	title = {title},
%	date = {date},
%	url = {https://vse.nl/de-paradox-van-de-industriele-automatisering/},
%}
%
%
%
%@online{blogxot21112016slimapparaat,
%	ALTauthor = {author},
%	ALTeditor = {editor},
%	title = {title},
%	date = {date},
%	url = {https://blog.xot.nl/2016/11/21/slimme-apparaten-maken-ons-dom-en-kwetsbaar/index.html},
%}
%
%@online{ID,
%	ALTauthor = {author},
%	ALTeditor = {editor},
%	title = {title},
%	date = {date},
%	url = {https://automatie-pma.com/nieuws/industriele-automatiseringsparadox},
%}
%%%%%%%%%%%%%%%%%%%%%%%%%%%%%%%%%%%%%%%%%%%%%%%%%%%%%%%%%%%%%%%%%%
%
%
%
%
%
%
%
%
%
%@online{aviationsafety04101992airplaneCrashBijlmer,
%	ALTauthor = {author},
%	ALTeditor = {editor},
%	title = {title},
%	date = {date},
%	url = {https://aviation-safety.net/database/record.php?id=19921004-2&lang=nl  },
%}
%
%
%
%@online{boogers092002RampenRegelsRichtlijnen,
%	ALTauthor = {author},
%	ALTeditor = {editor},
%	title = {title},
%	date = {date},
%	url = { https://www.researchgate.net/publication/254815008_Rampen_regels_richtlijnen},
%}
%
%
%@online{catsr25022009Boeing737AmsterdamCrash,
%	ALTauthor = {author},
%	ALTeditor = {editor},
%	title = {title},
%	date = {date},
%	url = {https://catsr.vse.gmu.edu/SYST460/TA1951_AccidentReport.pdf  },
%}
%
%@online{INSAVienna1992Chernobyl,
%	ALTauthor = {author},
%	ALTeditor = {editor},
%	title = {title},
%	date = {date},
%	url = {https://www-pub.iaea.org/MTCD/publications/PDF/Pub913e_web.pdf  },
%}
%
%
%
%
%aviationReport
%
%@online{oVVSchietongevalOssendrecht,
%	ALTauthor = {author},
%	ALTeditor = {editor},
%	title = {title},
%	date = {date},
%	url = {https://www.youtube.com/watch?v=6jmkDClGDHo  },
%}
%
%@online{molukseTreinkaping,
%	ALTauthor = {author},
%	ALTeditor = {editor},
%	title = {title},
%	date = {date},
%	url = {https://www.youtube.com/watch?v=h99Fe9XzzHI  },
%}
%
%@online{jiang16042019TanjinExplosion,
%	ALTauthor = {author},
%	ALTeditor = {editor},
%	title = {title},
%	date = {date},
%	url = {https://www.hindawi.com/journals/joph/2019/1360805/  },
%}
%
%
%@online{hrw03082021investigateBeirutBlast,
%	ALTauthor = {author},
%	ALTeditor = {editor},
%	title = {title},
%	date = {date},
%	url = { 	https://www.hrw.org/report/2021/08/03/they-killed-us-inside/investigation-august-4-beirut-blast },
%}
%
%@online{souaibyElHussein112020Beirutstory,
%	ALTauthor = {author},
%	ALTeditor = {editor},
%	title = {title},
%	date = {date},
%	url = { https://www.researchgate.net/publication/348325979_Beirut_Explosion_the_full_story },
%}
%
%@online{ifrc2020chemicalexplosionBeirutPort,
%	ALTauthor = {author},
%	ALTeditor = {editor},
%	title = {title},
%	date = {date},
%	url = { https://reliefweb.int/sites/reliefweb.int/files/resources/CaseStudy_BeirutExplosion_TechBioHazardsweb.pdf },
%}
%
%@online{caliskan09112013747boeingkalman,
%	ALTauthor = {author},
%	ALTeditor = {editor},
%	title = {title},
%	date = {date},
%	url = {https://www.hindawi.com/journals/ijae/2014/472395/  },
%}
%
%%%%%%%%%%%%%%%%%%%%%%%%%%%%%%%%%%%%%%%%%%%%%%%%%%%%%%%%%%%%%%%%%%
%persuasive technology 
%
%@online{humanTechpersuasiveTech,
%	ALTauthor = {author},
%	ALTeditor = {editor},
%	title = {title},
%	date = {date},
%	url = {https://www.humanetech.com/youth/persuasive-technology  },
%}
%
%
%
%
%@online{rezenfeld01012018persuasiveTecgHabits,
%	ALTauthor = {author},
%	ALTeditor = {editor},
%	title = {title},
%	date = {date},
%	url = {https://spectrum.ieee.org/how-persuasive-technology-can-change-your-habits  },
%}
%
%
%@online{aldenaini28042020persuasiveTechTrends,
%	ALTauthor = {author},
%	ALTeditor = {editor},
%	title = {title},
%	date = {date},
%	url = { https://www.frontiersin.org/articles/10.3389/frai.2020.00007/full },
%}
%
%@online{larson14062017persuasivetechmanipulates,
%	ALTauthor = {author},
%	ALTeditor = {editor},
%	title = {title},
%	date = {date},
%	url = {https://psmag.com/environment/captology-fogg-invisible-manipulative-power-persuasive-technology-81301  },
%}
%
%@online{tanzem22012022persuasivetechchanginglives,
%	ALTauthor = {author},
%	ALTeditor = {editor},
%	title = {title},
%	date = {date},
%	url = {https://www.makeuseof.com/what-is-persuasive-technology/  },
%}
%
%
%
%
%@online{tikkakuddonenpersuasiveTechnology,
%	ALTauthor = {author},
%	ALTeditor = {editor},
%	title = {title},
%	date = {date},
%	url = { https://cyberpsychology.eu/article/view/12270 },
%}
%
%@online{sprongken19032018CourtProcedureDigital,
%	ALTauthor = {author},
%	ALTeditor = {editor},
%	title = {title},
%	date = {date},
%	url = { https://www.njb.nl/blogs/a-court-with-no-face-and-no-place/ },
%}
%
%@online{PROCESREGLEMENTEcourt,
%	ALTauthor = {author},
%	ALTeditor = {editor},
%	title = {title},
%	date = {date},
%	url = { http://www.e-court.nl/wp-content/uploads/2018/03/Procesreglement-e-Court-2017_20180201.pdf},
%}
%
%
%@online{ovvMortierOngevalMaliVideo,
%	ALTauthor = {author},
%	ALTeditor = {editor},
%	title = {title},
%	date = {date},
%	url = {https://www.youtube.com/watch?v=PC2ekl4SaNA  },
%}
%
%
%
%
%@online{vtmGroep29032019waterwerkencyber,
%	ALTauthor = {author},
%	ALTeditor = {editor},
%	title = {title},
%	date = {date},
%	url = { https://www.vtmgroep.nl/blog/waterwerken-in-nederland-onvoldoende-beveiligd-tegen-cyberaanvallen},
%}
%
%
%
%
%
%@online{wesemannVeiligheidLandEnWater,
%	ALTauthor = {author},
%	ALTeditor = {editor},
%	title = {title},
%	date = {date},
%	url = {
%		http://www.wesemann.nl/nl/nieuws-en-pers/274-veiligheid-op-het-water-en-op-het-land.html },
%}
%
%
%
%

%%%%%%%%%%%%%%%%%%%%%%%%%%%%%%%%%%%%%%%%%%%%%%%%%%%%%%%%%%%%%%%%%
\bibitem{ISAC_SANS_Ukraine_DUC_18Mar2016} ... \LaTeX:\\ \url{https://www.nerc.com/_layouts/15/Nerc.404/CustomFileNotFound.aspx?requestUrl=https://www.nerc.com/pa/CI/ESISAC/Documents/E-ISAC_SANS_Ukraine_DUC_18Mar2016.pdf},}
\bibitem{zetter2016GridHack,	ALTauthor = {Kim Zetter},	ALTeditor = {editor},	title = {Inside the Cunning, Unprecedented Hack of Ukraine's Power Grid},	date = {date},	url = {https://www.wired.com/2016/03/inside-cunning-unprecedented-hack-ukraines-power-grid/},}
\bibitem{2015ukrainegridattack,	ALTauthor = {wikipedia},	ALTeditor = {editor},	title = {2015 Ukraine power grid hack},	date = {date},	url = {https://en.wikipedia.org/wiki/2015_Ukraine_power_grid_hack},}
\bibitem{greenberg2017Cyberwartestlab} ... \LaTeX:\\ \url{https://www.wired.com/story/russian-hackers-attack-ukraine/},}
\bibitem{boozallen2016lightwentout,	ALTauthor = {allen},	ALTeditor = {editor},	title = {Ukrain report when the lights went out},	date = {date},	url = {https://www.boozallen.com/content/dam/boozallen/documents/2016/09/ukraine-report-when-the-lights-went-out.pdf},}
\bibitem{finklejan2016UsBlamesRussianSandworm} ... \LaTeX:\\ \url{https://www.reuters.com/article/us-ukraine-cybersecurity-sandworm-idUSKBN0UM00N20160108},}
\bibitem{wired012016ukrainegrid} ... \LaTeX:\\ \url{https://www.wired.com/2016/01/everything-we-know-about-ukraines-power-plant-hack/},}
\bibitem{icsalert20072021cyberattackukraine} ... \LaTeX:\\ \url{https://www.us-cert.gov/ics/alerts/IR-ALERT-H-16-056-01},}
\bibitem{finkle08012016russiansandwormhackers} ... \LaTeX:\\ \url{https://www.reuters.com/article/us-ukraine-cybersecurity-sandworm/u-s-firm-blames-russian-sandworm-hackers-for-ukraine-outage-idUSKBN0UM00N20160108},}
\bibitem{zinets15022017ukrainechargesrussia} ... \LaTeX:\\ \url{https://www.reuters.com/article/us-ukraine-crisis-cyber-idUSKBN15U2CN},}
\bibitem{2014russiansandworm} ... \LaTeX:\\ \url{https://www.wired.com/2014/10/russian-sandworm-hack-isight/},}
\bibitem{mcelfresh2016cyberattackhowandwhy} ... \LaTeX:\\ \url{https://theconversation.com/cyberattack-on-ukraine-grid-heres-how-it-worked-and-perhaps-why-it-was-done-52802},}
\bibitem{desarnaud2017cyberattacks} ... \LaTeX:\\ \url{https://www.ifri.org/sites/default/files/atoms/files/desarnaud_cyber_attacks_energy_infrastructures_2017_2.pdf},}
\bibitem{osti2018historycontrolsystemIncidents} ... \LaTeX:\\ \url{https://www.osti.gov/biblio/1505628},}
\bibitem{Shahzad2014ScadaProtocolsPollingScenario} ... \LaTeX:\\ \url{https://scialert.net/fulltext/?doi=tasr.2014.396.405},}
\bibitem{caseli04112016intrusiondetectioncontrolsystem} ... \LaTeX:\\ \url{https://ris.utwente.nl/ws/files/6028066/3-s2_0-B9780128015957000227.pdf},}
\bibitem{rochascadatesting} ... \LaTeX:\\ \url{https://repositorio-aberto.up.pt/bitstream/10216/119066/2/315683.pdf},}
\bibitem{levalle2020FuzzingICSProtocols} ... \LaTeX:\\ \url{https://dreamlab.net/en/blog/post/fuzzing-ics-protocols/},}
\bibitem{resch31102019IEC62351secureCommunication} ... \LaTeX:\\ \url{http://www.connectivity4ir.co.uk/article/175490/IEC-62351--Secure-communication-in-the-energy-industry.aspx},}
\bibitem{slowik2019ReassasUkraine2016Attack} ... \LaTeX:\\ \url{https://www.dragos.com/wp-content/uploads/CRASHOVERRIDE.pdf},}
\bibitem{arrizabalaga2020surveyiiotProtocols} ... \LaTeX:\\ \url{https://dl.acm.org/doi/fullHtml/10.1145/3381038},}
\bibitem{yadav2020reviewScadaArchitecture} ... \LaTeX:\\ \url{https://arxiv.org/pdf/2001.02925.pdf},}
\bibitem{cis20072021crashoverridemalware} ... \LaTeX:\\ \url{https://www.us-cert.gov/ncas/alerts/TA17-163A},}
\bibitem{holappa2017threattoElectricityNetworks} ... \LaTeX:\\ \url{https://www.nixu.com/fi/node/53},}
\bibitem{shehod2016gridadvantageus} ... \LaTeX:\\ \url{http://web.mit.edu/smadnick/www/wp/2016-22.pdf},}
\bibitem{parkwalstorm11102017russiagridattack} ... \LaTeX:\\ \url{https://jsis.washington.edu/news/cyberattack-critical-infrastructure-russia-ukrainian-power-grid-attacks/},}
\bibitem{drago2017CrashOverride} ... \LaTeX:\\ \url{https://www.dragos.com/wp-content/uploads/CrashOverride-01.pdf},}
\bibitem{wikiindustroyer} ... \LaTeX:\\ \url{https://en.wikipedia.org/wiki/Industroyer},}
\bibitem{crashoverridenetwork} ... \LaTeX:\\ \url{https://en.wikipedia.org/wiki/Crash_Override_Network},}
\bibitem{slowikvb2018crashoverride} ... \LaTeX:\\ \url{https://www.virusbulletin.com/virusbulletin/2019/03/vb2018-paper-anatomy-attack-detecting-and-defeating-crashoverride/},}
\bibitem{njccicthreat08102017crashovverrideprofile} ... \LaTeX:\\ \url{https://www.cyber.nj.gov/threat-center/threat-profiles/ics-malware-variants/crashoverride},}
\bibitem{crashoverrideindustroyermalware} ... \LaTeX:\\ \url{https://www.webopedia.com/TERM/C/crashoverride-industroyer-malware.html},}
\bibitem{incibe23062017crashoverrideback} ... \LaTeX:\\ \url{https://www.incibe-cert.es/en/blog/crashoverride-malware-ics-back-again},}
\bibitem{industroyershortfact} ... \LaTeX:\\ \url{https://rhebo.com/en/service/glossar/industroyer-25114/},}
\bibitem{vijayan2017firstmalwareCausedOutage} ... \LaTeX:\\ \url{https://www.darkreading.com/threat-intelligence/first-malware-designed-solely-for-electric-grids-caused-2016-ukraine-outage/d/d-id/1329114},}
\bibitem{ferrante2017crashoverrideredflag} ... \LaTeX:\\ \url{https://www.powermag.com/why-crashoverride-is-a-red-flag-for-u-s-power-companies/},}
\bibitem{2017cashoverrideindustroyerAlto} ... \LaTeX:\\ \url{https://blog.paloaltonetworks.com/2017/06/crashoverrideindustroyer-protections-palo-alto-networks-customers/},}
\bibitem{spinner2018crashoverrideiot} ... \LaTeX:\\ \url{https://iiot-world.com/ics-security/cybersecurity/five-cybersecurity-experts-about-crashoverride-malware-main-dangers-and-lessons-for-iiot/},}
\bibitem{abb30062017crashoverridenotification} ... \LaTeX:\\ \url{https://search.abb.com/library/Download.aspx?DocumentID=9AKK107045A1003&amp;LanguageCode=en&amp;DocumentPartId=&amp;Action=Launch},}
\bibitem{blackhatusa2017} ... \LaTeX:\\ \url{https://www.blackhat.com/us-17/briefings/schedule/#industroyercrashoverride-zero-things-cool-about-a-threat-group-targeting-the-power-grid-6159},}
\bibitem{humanitcrashalert} ... \LaTeX:\\ \url{https://humanit.asia/ta17-163a/},}
\bibitem{sroberts2017CrashoerrideCronicles} ... \LaTeX:\\ \url{https://medium.com/@sroberts/the-crash-override-chronicles-overall-8389ef178fdf},}
\bibitem{parker2020industrialsystemsattack} ... \LaTeX:\\ \url{https://www.oilandgaseng.com/articles/the-most-infamous-cyber-attacks-on-industrial-systems/},}
\bibitem{kirk2017threatIndustrialControls} ... \LaTeX:\\ \url{https://www.bankinfosecurity.com/power-grid-malware-platform-threatens-industrial-controls-a-9987},}
\bibitem{chalfant2017USElectricGrid} ... \LaTeX:\\ \url{https://thehill.com/policy/cybersecurity/337877-crash-override-malware-heightens-fears-for-us-electric-grid},}
\bibitem{fbiWarningcrashOverride} ... \LaTeX:\\ \url{https://www.inguardians.com/dhs-fbi-warn-of-attacks-against-us-energy-manufacturing-companies-and-employees/},}
\bibitem{davies2017crashoveerideUkraineAttack} ... \LaTeX:\\ \url{https://rethinkresearch.biz/articles/industroyer-crashoverride-malware-behind-ukraine-utility-attack/},}
\bibitem{slowik2018securitySessions} ... \LaTeX:\\ \url{https://electricenergyonline.com/energy/magazine/1104/article/Security-Sessions-Combating-ICS-Threats.htm},}
\bibitem{ENCS2017crashoverridemodules} ... \LaTeX:\\ \url{https://www.smart-energy.com/regional-news/europe-uk/encs-crash-override-virus/},}
\bibitem{brocklehurst2017crashoverridegridtakedown} ... \LaTeX:\\ \url{https://isssource.com/crashoverride-designed-for-grid-takedown/},}
\bibitem{20170613_crashoverride} ... \LaTeX:\\ \url{https://gigazine.net/gsc_news/en/20170613-crashoverride/},}
@onlinec{leetaru2017crashoverridehomefront} ... \LaTeX:\\ \url{https://www.forbes.com/sites/kalevleetaru/2017/06/24/crash-override-and-how-cyberwarfare-is-bringing-conflict-to-the-homefront/#42eb8984277c},}
\bibitem{fti2017redflag} ... \LaTeX:\\ \url{https://fticybersecurity.com/2017-11/crashoverride-red-flag-u-s-power-companies/},}
\bibitem{27001academy2014Whitepaper} ... \LaTeX:\\ \url{https://info.advisera.com/hubfs/27001Academy/27001Academy_FreeDownloads/NL/Checklist_of_ISO_27001_Mandatory_Documentation_NL.pdf},}
\bibitem{2017info_isoiec27019} ... \LaTeX:\\ \url{https://webstore.iec.ch/preview/info_isoiec27019%7Bed1.0%7Den.pdf},}
\bibitem{iec623512023serseries} ... \LaTeX:\\ \url{https://webstore.iec.ch/publication/6912},}
\bibitem{iec623512011Withdrawn} ... \LaTeX:\\ \url{https://webstore.iec.ch/publication/6911},}
\bibitem{IEC62351sheet} ... \LaTeX:\\ \url{https://www.ipcomm.de/protocol/IEC62351/en/sheet.html},}
\bibitem{NISTOIR2014GuidelinesCyverSec} ... \LaTeX:\\ \url{https://nvlpubs.nist.gov/nistpubs/ir/2014/NIST.IR.7628r1.pdf},}
\bibitem{ukraiinnesandwormteam} ... \LaTeX:\\ \url{https://www.fireeye.com/blog/threat-research/2016/01/ukraine-and-sandworm-team.html/cybersecurity-audit-checklist. },}
\bibitem{politicsrussagridukraine} ... \LaTeX:\\ \url{https://edition.cnn.com/2016/02/11/politics/ukraine-power-grid-attack-russia-us/index.html/ukraine-sees-russian-hand-in-cyber-attacks-on-power-grid-idUSKCN0VL18E. },}
\bibitem{Whitehead2017ukrainepoweroutage,	ALTauthor = {David E. Whitehead, Kevin Owens, Dennis Gammel, and Jess Smith},	ALTeditor = {editor},	title = {title},	date = {March 21–23, 2017},	url = {https://na.eventscloud.com/file_uploads/aed4bc20e84d2839b83c18bcba7e2876_Owens1.pdf},}
\bibitem{zetter2017moreDangerousMalware} ... \LaTeX:\\ \url{https://www.vice.com/en/article/zmeyg8/ukraine-power-grid-malware-crashoverride-industroyer},}
\bibitem{icsSecurityRussianHacking} ... \LaTeX:\\ \url{https://www.wallix.com/blog/ics-security-russian-hacking},}
\bibitem{rocha2017cybersecyrityanalysisScada} ... \LaTeX:\\ \url{https://www.sans.org/blog/confirmation-of-a-coordinated-attack-on-the-ukrainian-power-grid/},}
\bibitem{cheah2008testingiec60870} ... \LaTeX:\\ \url{http://citeseerx.ist.psu.edu/viewdoc/download;jsessionid=0513EED48102FDAD1BD940260EF12B11?doi=10.1.1.548.7490&amp;rep=rep1&amp;type=pdf},}
\bibitem{virsec2017DeepDiveIndustroyer} ... \LaTeX:\\ \url{https://virsec.com/virsec-hack-analysis-deep-dive-into-industroyer-aka-crash-override/},}
\bibitem{fauri2017EncryptionICS} ... \LaTeX:\\ \url{https://www.win.tue.nl/~setalle/2017_fauri_encryption.pdf},}
\bibitem{2017IEC61850} ... \LaTeX:\\ \url{http://blog.nettedautomation.com/2017/},}
\bibitem{2017win32industroyer} ... \LaTeX:\\ \url{https://www.welivesecurity.com/wp-content/uploads/2017/06/Win32_Industroyer.pdf},}
\bibitem{dragos2019TargetedTransStation} ... \LaTeX:\\ \url{https://www.cybersecurityintelligence.com/blog/attack-on-ukraines-power-grid-targeted-transmission-stations-4530.html},}
\bibitem{2017crashoverridenostuxnet} ... \LaTeX:\\ \url{https://arstechnica.com/information-technology/2017/06/crash-override-malware-may-sabotage-electric-grids-but-its-no-stuxnet/},}
%%%%%%%%%%%%%%%%%%%%%%%%%%%%%%%%%%%%%%%%%%%%%%%%%%%%%%%%%%%%%%%%%



%
%\chapter{Deelonderzoek naar veiligheidsrisico's en eisen voor sluizen}
%
%Gevonden weblinks in google op 07-04-2023 met zoekopdracht: "veiligheidsrisico's voor sluizen en waterwerken"
%





\bibitem{ID} ... \LaTeX:\\ \url{https://www.tweedekamer.nl/downloads/document?id=80443e97-f17e-499c-b3f2-ad608f32e1aa&title=Rapportage%20Staat%20van%20de%20infra%20RWS%20%28definitief%29.pdf}
\bibitem{ID} ... \LaTeX:\\ \url{https://www.nu.nl/internet/5814282/rekenkamer-waterwerken-niet-goed-beveiligd-tegen-cyberaanvallen.html}
\bibitem{linne21122022onderhousluis} ... \LaTeX:\\ \url{https://www.deltalimburg.nl/article/9824/Onderhoudswerkzaamheden+aan+Sluis+Linne+afgerond}
\bibitem{sluisterneuzenveiligheid} ... \LaTeX:\\ \url{https://nieuwesluisterneuzen.eu/veiligheid}
\bibitem{mrdmarinesupport} ... \LaTeX:\\ \url{https://www.mrdmarinesupport.nl/nl/maritieme-dienstverlening/ondersteuning-veiligheid/}
\bibitem{krabbendam27052021infrasitrrisicobeoordeling} ... \LaTeX:\\ \url{https://www.infrasite.nl/bouwen/2021/05/27/veiligheid-voorop-begin-project-sluis-of-brug-altijd-met-risicobeoordeling/}
\bibitem{bedieningsluisVechterweerdVilsteren} ... \LaTeX:\\ \url{https://www.wdodelta.nl/bediening-schutsluizen-vechterweerd-en-vilsteren}
\bibitem{krabbedam21052021sluisHeel} ... \LaTeX:\\ \url{https://www.infrasite.nl/waterbouw-deltas/2021/05/21/sluis-heel-onder-handen-genomen/}
\bibitem{hdsr30092022lichtprojectieswaterliniesluizen} ... \LaTeX:\\ \url{https://www.hdsr.nl/actueel/nieuws/@154100/lichtprojecties-zetten-waterliniesluizen/}
\bibitem{nos28032019rekenkamerhackwaterwerk} ... \LaTeX:\\ \url{https://nos.nl/artikel/2277937-rekenkamer-hack-aanval-op-waterwerk-niet-altijd-opgemerkt}
\bibitem{sluisLinne08012023} ... \LaTeX:\\ \url{https://varendoejesamen.nl/kenniscentrum/artikel/onderhoud-sluis-linne-afgerond}
\bibitem{gww29032021kantelendesluisdeur} ... \LaTeX:\\ \url{https://www.gww-bouw.nl/artikel/de-eerste-sluis-met-kantelende-sluisdeur/}
\bibitem{tkh03022018sluisbeheerScadaZeeland} ... \LaTeX:\\ \url{https://tkhsecurity.com/nl/waterwerken/}
\bibitem{h2o28032019dreigingsniveaurekenkamer} ... \LaTeX:\\ \url{https://www.h2owaternetwerk.nl/h2o-actueel/rekenkamer-vitale-waterwerken-nog-onvoldoende-beschermd-tegen-cyberaanvallen}
\bibitem{krammersluizencomplex} ... \LaTeX:\\ \url{https://www.magazinesrijkswaterstaat.nl/bereikbaarzeeland/2021/01/krammersluizencomplex-verleden-heden-en-toekomst}



%\chapter{Deelonderzoek wet en regelgeving voor sluizen}
%
%
%
%Gevonden weblinks in google op 07-04-2023 met zoekopdracht: "wet en regelgeving voor sluizen"



\bibitem{ID} ... \LaTeX:\\ \url{https://www.hdsr.nl/publish/pages/86927/sluizen_in_of_bij_een_waterkering_-_uitvoeringsregels.pdf}
\bibitem{ID} ... \LaTeX:\\ \url{https://api1.ibabs.eu/publicdownload.aspx?site=sluis&id=100100292}
\bibitem{ID} ... \LaTeX:\\ \url{https://services.pilz.nl/wp-content/uploads/2021/12/brochure_bruggen_2018.pdf}
\bibitem{ID} ... \LaTeX:\\ \url{https://lokaleregelgeving.overheid.nl/CVDR375606/6}
\bibitem{ID} ... \LaTeX:\\ \url{https://zoek.officielebekendmakingen.nl/stb-2019-27.html}
\bibitem{ID} ... \LaTeX:\\ \url{https://a-quin.nl/nieuws/veiligheid-van-bruggen-sluizen-waarborgen-wie-wat-hoe/}
\bibitem{ID} ... \LaTeX:\\ \url{https://www.gemeentesluis.nl/Bestuur_en_Organisatie/Wetten_Regels_Bekendmakingen}
\bibitem{ID} ... \LaTeX:\\ \url{https://www.overijssel.nl/onderwerpen/verkeer-en-vervoer/varen-in-overijssel/informatie-bedieningstijden-sluizen-en-bruggen-noordwest-overijssel/}
\bibitem{ID} ... \LaTeX:\\ \url{https://www.rijkswaterstaat.nl/water/wetten-regels-en-vergunningen}
\bibitem{ID} ... \LaTeX:\\ \url{https://www.schuttevaer.nl/nieuws/actueel/2022/11/23/binnenvaart-zit-klem-tussen-regels-en-realiteit-kapotte-steigers-en-gesperde-sluizen-dwingen-tot-doorvaren/}
\bibitem{ID} ... \LaTeX:\\ \url{https://repository.officiele-overheidspublicaties.nl/CVDR/CVDR271406/1/html/CVDR271406_1.html}
\bibitem{ID} ... \LaTeX:\\ \url{https://www.zeeland.nl/actueel/bedieningstijden-sluizen-en-bruggen}
\bibitem{ID} ... \LaTeX:\\ \url{https://www.amsterdam.nl/verkeer-vervoer/varen-amsterdam/regels-varen/}
\bibitem{ID} ... \LaTeX:\\ \url{https://www.schielandendekrimpenerwaard.nl/wat-doen-we/regels-en-afspraken-over-beheer-keur-en-leggers/}
\bibitem{ID} ... \LaTeX:\\ \url{http://www.wetboek-online.nl/wet/Wet%20tot%20samenvoeging%20van%20de%20gemeenten%20Aardenburg%20en%20Sluis.html}
\bibitem{ID} ... \LaTeX:\\ \url{https://www.rijnland.net/regels-op-een-rij/richtlijnen-en-akkoorden/alle-regelgeving-van-rijnland/}
\bibitem{ID} ... \LaTeX:\\ \url{https://www.itbb.nl/diensten/advies-ce-markering-europese-richtlijnen/}
\bibitem{ID} ... \LaTeX:\\ \url{https://www.portofamsterdam.com/nl/scheepvaart/zeevaart/regelgeving}
\bibitem{ID} ... \LaTeX:\\ \url{https://www.watersportverbond.nl/nieuws/achterstallig-onderhoud-wachtplaatsen-bruggen-en-sluizen-zuid-holland-zorgelijk/}
\bibitem{ID} ... \LaTeX:\\ \url{https://varendoejesamen.nl/nieuws}
\bibitem{ID} ... \LaTeX:\\ \url{https://www.flevoland.nl/wat-doen-we/flevowegen-vlot-en-veilig-door-flevoland/water/varen-in-flevoland/bediening-bruggen-en-sluizen}
\bibitem{ID} ... \LaTeX:\\ \url{https://eur-lex.europa.eu/legal-content/NL/TXT/PDF/?uri=CELEX:32020L0012&from=DE}
\bibitem{ID} ... \LaTeX:\\ \url{https://www.werkenvoornederland.nl/organisatie/rijkswaterstaat/ict-middelen-maken-om-bruggen-sluizen-en-tunnels-te-besturen}
\bibitem{ID} ... \LaTeX:\\ \url{https://www.lobocom.nl/infra-bruggen-sluizen}
\bibitem{ID} ... \LaTeX:\\ \url{https://waterrecreatienederland.nl/content/uploads/2018/04/richtlijnen-vaarwegen-2017.pdf}
\bibitem{ID} ... \LaTeX:\\ \url{https://www.wetterskipfryslan.nl/melden-en-regelen/vergunningen-wetten-en-regels}
\bibitem{ID} ... \LaTeX:\\ \url{https://www.onlinezeilschool.nl/sluizen/}
\bibitem{ID} ... \LaTeX:\\ \url{https://www.provincie.drenthe.nl/onderwerpen/verkeer-vervoer/vaarwegen/rondje-drenthe/bedieningstijden/}





Rampen
In dit hoofdstuk worden de resultaten van een deskresearch naar verschillende rampen behandeld
Hierbij een verslag naar de oorzaken van de rampen, de werkwijze waarop het product is ontwikkeld, de verwerking van feedback, implementatie en nazorg.
Met behulp van het 4 variabelen model wordt duidelijk gemaakt hoe het systeem is opgezet en wat daarin verkeerd is gegaan.
Het  hoofdstuk wordt afgesloten met een analyse van algemene kenmerken van de verschillende rampen die zijn onderzocht.


Het 4 variabelen model kort toegelicht
Monitored variabelen: door sensoren gekwantificeerde fenomenen uit de omgeving, bijv temperatuur

Controlled variabelen: door actuatoren \bestuurde fenomenen uit de omgeving
For example, monitored variables might be the pressure and temperature
inside a nuclear reactor while controlled variables might be visual and audible alarms, as well
as the trip signal that initiates a reactor shutdown; whenever the temperature or pressure reach
abnormal values, the alarms go off and the shutdown procedure is initiated

Input variabelen: data die de software als input gebruikt
Here, IN models the input hardware interface (sensors and analog-to-digital converters) and
relates values of monitored variables to values of input variables in the software. The input variables model the information about the environment that is available to the software. For example,
IN might model a pressure sensor that converts temperature values to analog voltages; these voltages are then converted via an A/D converter to integer values stored in a register accesible to the
software.

Output variabelen: data die de software levert als output
The output hardware interface (digital-to-analog converters and actuators) is modelled
by OUT, which relates values of the output variables of the software to values of controlled variables. An output variable might be, for instance, a boolean variable set by the software with the
understanding that the value true indicates that a reactor shutdown should occur and the value
false indicates the opposite


Bronnen:
\bibitem{ID} ... \LaTeX:\\ \url{https://www.sciencedirect.com/science/article/pii/S0167642315001033}
\bibitem{ID} ... \LaTeX:\\ \url{https://www.cas.mcmaster.ca/~lawford/papers/AVoCS2013.pdf}
\bibitem{ID} ... \LaTeX:\\ \url{https://core.ac.uk/download/pdf/38891842.pdf}
%%%%%%%%%%%%%%%%%%%%%%%%%%%%%%%%%%%%%%%%%%%%%%%%%%%%%%%%%%%%%%%%%

Therac


sheets
\bibitem{ID} ... \LaTeX:\\ \url{https://web.cs.ucdavis.edu/~rogaway/classes/188/winter04/therac-25.pdf}
\bibitem{ID} ... \LaTeX:\\ \url{https://people.physics.carleton.ca/~drogers/egs_windows_collection/tsld008.htm}
\bibitem{ID} ... \LaTeX:\\ \url{https://en.wikipedia.org/wiki/Therac-25}
\bibitem{ID} ... \LaTeX:\\ \url{https://www.youtube.com/watch?v=-7gVqBY52MY}
reproduceren van de error. IN dit stuk wordt uitgelgd hoe het product werkt en waarom bepaalde beslssingen zijn genomen in de ontwerp/productiefase
\bibitem{ID} ... \LaTeX:\\ \url{https://www.bugsnag.com/blog/bug-day-race-condition-therac-25}
kort artikel met daarin een opsomming van alle fouten in het systeem en een korte uitleg
\bibitem{ID} ... \LaTeX:\\ \url{https://www.bowdoin.edu/~allen/courses/cs260/readings/therac.pdf}
uitgebreid artikel over hoe de fout werd gereproduceerd en de resultaten daaruit voortkwamen. Alsnog werden er na de reproductie fase nog meer fouten gevonden.
\bibitem{ID} ... \LaTeX:\\ \url{https://hackaday.com/2015/10/26/killed-by-a-machine-the-therac-25/}
artikel
\bibitem{ID} ... \LaTeX:\\ \url{https://ethicsunwrapped.utexas.edu/case-study/therac-25}
onderzoeksartikel waarin de bug wordt uitgelgd: de racecondities, de bytepositie en het testen worden berkitiseerd envenals andere onderdelen van het softwareproces
\bibitem{ID} ... \LaTeX:\\ \url{https://thedailywtf.com/articles/the-therac-25-incident}
onrealistisch testplan. In dit artikel egt de auteur het belang nog eens uit van goede requirements en implementatie, niet de software is waar het probleem ligt
\bibitem{ID} ... \LaTeX:\\ \url{https://www.computer.org/csdl/magazine/co/2017/11/mco2017110008/13rRUxAStVR}
onderzoeksrapport
\bibitem{ID} ... \LaTeX:\\ \url{https://web.stanford.edu/class/cs240/old/sp2014/readings/therac-25.pdf}
geschiedenis
\bibitem{ID} ... \LaTeX:\\ \url{http://computingcases.org/case_materials/therac/case_history/Case%20History.html}
artikel
\bibitem{ID} ... \LaTeX:\\ \url{https://medium.com/swlh/software-architecture-therac-25-the-killer-radiation-machine-8a05e0705d5b}
computer error. De ongeval en de malfunction nog een keer uitgelegd
\bibitem{ID} ... \LaTeX:\\ \url{http://www.ccnr.org/fatal_dose.html}
rapport
\bibitem{ID} ... \LaTeX:\\ \url{http://sunnyday.mit.edu/papers/therac.pdf}
\bibitem{ID} ... \LaTeX:\\ \url{https://pubmed.ncbi.nlm.nih.gov/101762/}
onderzoeksartkel
\bibitem{ID} ... \LaTeX:\\ \url{http://www1.cs.columbia.edu/~junfeng/08fa-e6998/sched/readings/therac25.pdf}
\bibitem{ID} ... \LaTeX:\\ \url{https://ieeexplore.ieee.org/document/274940}
uitgebreid artikel gaat hier ook wat meer over de hardware
\bibitem{ID} ... \LaTeX:\\ \url{https://www.linkedin.com/pulse/therac-25-industrial-design-engineering-systems-wang-ph-d-cre-acb/}
artikel waarin in 3 delen de problemaiekwordt blootgesteld
\bibitem{ID} ... \LaTeX:\\ \url{http://www.cse.msu.edu/~cse470/Public/Handouts/Therac/Therac_2.html}
case study sheets
\bibitem{ID} ... \LaTeX:\\ \url{https://www.cs.jhu.edu/~cis/cista/445/Lectures/Therac.pdf}
artikel waarin vooral de fabriikant ervan langs krijgt
\bibitem{ID} ... \LaTeX:\\ \url{http://users.csc.calpoly.edu/~jdalbey/SWE/Papers/THERAC25.html}
lessons learned. Vooral de begrippen betrouwbaarheid, welgevalligheid, veilgheid en gebruiksvriendelijkheid
\bibitem{ID} ... \LaTeX:\\ \url{https://bohr.wlu.ca/cp164/therac/therac25.htm}
root-cause analysis
\bibitem{ID} ... \LaTeX:\\ \url{https://root-cause-analysis.info/2010/08/08/therac-25-radiation-overdoses/}
case study
\bibitem{ID} ... \LaTeX:\\ \url{https://dusk.geo.orst.edu/ethics/papers/Therac.Huff.pdf}
\bibitem{ID} ... \LaTeX:\\ \url{https://embeddedartistry.com/fieldatlas/case-study-therac-25/}
case study
\bibitem{ID} ... \LaTeX:\\ \url{https://www.sebokwiki.org/wiki/Medical_Radiation}
opzetten van systematische acceptaatie test met therac als voorbeeld
\bibitem{ID} ... \LaTeX:\\ \url{https://www.sciencedirect.com/science/article/pii/S1474667017448245}
artikel waarin een diagnose plaatvindt voor het bedrijf en de ingenieur/ontwerper
\bibitem{ID} ... \LaTeX:\\ \url{https://magsilva.pro.br/apps/wiki/testing/Therac_25}
rapport
\bibitem{ID} ... \LaTeX:\\ \url{http://csel.eng.ohio-state.edu/productions/pexis/readings/submod3/therac.pdf}
oorzaken aangegeven in artikel
\bibitem{ID} ... \LaTeX:\\ \url{https://www.chemeurope.com/en/encyclopedia/Therac-25.html}
het onderzoek en enkele ontwerptekeningen en oplossingen
\bibitem{ID} ... \LaTeX:\\ \url{https://pvs-studio.com/en/blog/posts/0438/}
\bibitem{ID} ... \LaTeX:\\ \url{https://www.coursera.org/lecture/software-design-threats-mitigations/therac-25-case-study-VmQPa}
\bibitem{ID} ... \LaTeX:\\ \url{https://www.semanticscholar.org/paper/The-story-of-the-Therac-25-in-LOTOS-Thomas/6c9c6024cf95aadae8b7edf1160e0e4500410eb9}
\bibitem{ID} ... \LaTeX:\\ \url{https://news.ycombinator.com/item?id=21679287}
wiki
\bibitem{ID} ... \LaTeX:\\ \url{https://en.wikibooks.org/wiki/Professionalism/Therac-25}
analyse
\bibitem{ID} ... \LaTeX:\\ \url{https://citeseerx.ist.psu.edu/viewdoc/download?doi=10.1.1.96.369&rep=rep1&type=pdf}
samenvatting
\bibitem{ID} ... \LaTeX:\\ \url{https://onlineethics.org/cases/resources-engineering-and-science-ethics/investigation-therac-25-accidents-abstract}
podcast
\bibitem{ID} ... \LaTeX:\\ \url{https://podcasts.apple.com/gb/podcast/therac-25/id1046978749?i=1000514115050}
enkele conclusies
\bibitem{ID} ... \LaTeX:\\ \url{http://www.cas.mcmaster.ca/~se4d03/therac.html}
rapport over de fouten die de verschillende partijen hebben gemaakt( overheid, ingenieurs, bedrijf, operators) en de verbeterpunten
\bibitem{ID} ... \LaTeX:\\ \url{https://www.cs.colostate.edu/~bieman/CS314/Notes/therac25.pdf}
onderzoeksrapport
\bibitem{ID} ... \LaTeX:\\ \url{https://www.cs.ucf.edu/~dcm/Teaching/COP4600-Fall2010/Literature/Therac25-Leveson.pdf}
slides online over het technisch mankement
Wat is er gebeurd, nou het volgende:
Normal radiation treatments: 6,000 rads over a 3 week period, under certain conditions Therac-25 was delivering 60,000 rads during one session.
En wat ging er mis?
Paradigm Shift

Therac-25 replaced expensive hardware safety interlocks with software controls

Real-time software
Design
Race condition caused focusing element to be incorrectly set
No indication of actual hardware settings
Error messages appeared the same regardless of how important
Error messages were difficult to understand
All errors messages could be manually overridden

\bibitem{ID} ... \LaTeX:\\ \url{https://hci.cs.siue.edu/NSF/Files/Semester/Week13-2/PPT-Text/Slide13.html}
oorzaak-gevolg diagram
\bibitem{ID} ... \LaTeX:\\ \url{https://www.thinkreliability.com/InstructorBlogs/Blog-Therac-25.pdf}
veiligheidsanalyse naar de rapportage van foutmeldingen, de beslissingsmatrix waarmee het programma wordt uitgevoerd en de software-analyse door een consultat
\bibitem{ID} ... \LaTeX:\\ \url{https://www.erenkrantz.com/Geeks/Therac-25/Side_bar_5.html}
\bibitem{ID} ... \LaTeX:\\ \url{https://bandcamp.com/therac-25}
\bibitem{ID} ... \LaTeX:\\ \url{https://sites.nd.edu/brent-marin/2017/09/21/blog-post-8-therac-25-accidents/}
\bibitem{ID} ... \LaTeX:\\ \url{https://europepmc.org/article/med/4035720}
\bibitem{ID} ... \LaTeX:\\ \url{https://itlaw.wikia.org/wiki/Therac-25}
\bibitem{ID} ... \LaTeX:\\ \url{https://www.sitepoint.com/therac-25-bad-software-kills/}
\bibitem{ID} ... \LaTeX:\\ \url{https://sqa.stackexchange.com/questions/9798/asking-for-help-with-this-therac-25-bugged-code-i-dont-understand-the-explanat}
\bibitem{ID} ... \LaTeX:\\ \url{http://www.se.rit.edu/~swen-342/activities/TheracIndividual.html}
\bibitem{ID} ... \LaTeX:\\ \url{https://www.designnews.com/automation-motion-control/engineering-disasters-deadly-zaps-therac-25}

publicaties over de therac:
\bibitem{ID} ... \LaTeX:\\ \url{https://www.cs.colostate.edu/~bieman/CS314/Notes/therac25.pdf}
\bibitem{ID} ... \LaTeX:\\ \url{https://www.cs.ucf.edu/~dcm/Teaching/COP4600-Fall2010/Literature/Therac25-Leveson.pdf}
\bibitem{ID} ... \LaTeX:\\ \url{https://onlineethics.org/cases/resources-engineering-and-science-ethics/investigation-therac-25-accidents-abstract}
\bibitem{ID} ... \LaTeX:\\ \url{https://citeseerx.ist.psu.edu/viewdoc/download?doi=10.1.1.96.369&rep=rep1&type=pdf}
\bibitem{ID} ... \LaTeX:\\ \url{http://csel.eng.ohio-state.edu/productions/pexis/readings/submod3/therac.pdf}
\bibitem{ID} ... \LaTeX:\\ \url{https://dusk.geo.orst.edu/ethics/papers/Therac.Huff.pdf}
\bibitem{ID} ... \LaTeX:\\ \url{https://www.cs.jhu.edu/~cis/cista/445/Lectures/Therac.pdf}
\bibitem{ID} ... \LaTeX:\\ \url{http://www1.cs.columbia.edu/~junfeng/08fa-e6998/sched/readings/therac25.pdf}
\bibitem{ID} ... \LaTeX:\\ \url{http://sunnyday.mit.edu/papers/therac.pdf}
\bibitem{ID} ... \LaTeX:\\ \url{https://web.stanford.edu/class/cs240/old/sp2014/readings/therac-25.pdf}




%%%%%%%%%%%%%%%%%%%%%%%%%%%%%%%%%%%%%%%%%%%%%%%%%%%%%%%%%%%%%%%%%
Krakend zorgssteem door covid-19 in suriname

vaccinatieterkort
communicatie met bevolking
communicatie met binnenland
testen van vaccinaties
besmetting vanuit eht buitenland
isolatie na vakantie en voor toeristen
tekort aan ic-personeel
tekort aan ic-bedden
tekort aan zuurtstof
tekort aan middelen}

Wat blijkt hieruit:
de impact van de crisis wereldwijd
de afhnakelijkheid van landen op goede samenwerking
Nut en noodzaak van regelgeving
Naveling van maatregels
Communicatie over beleid vanuit de overheid naar de burgers
Belang van een verzorgingstaat
Een wetenschappelijke ontwikkeling die kan inspelen op gevoelige 'trends
De impact van een lockdown op de economie
Afschaling van andere noodzakelijke no-covid zorg
De bereikbaarheid van een ziekenhuis
Waar heeft het toe geleid?
\bibitem{ID} ... \LaTeX:\\ \url{https://www.waterkant.net/suriname/2007/02/06/school-in-suriname-gesloten-om-zenuwgasvoorraad/}
\bibitem{ID} ... \LaTeX:\\ \url{https://nl.wikipedia.org/wiki/Nationaal_Co%C3%B6rdinatiecentrum_voor_Rampenbeheersing}
\bibitem{ID} ... \LaTeX:\\ \url{https://www.examenkamer.nl/index.php/27-vca-examens-in-suriname}
%%%%%%%%%%%%%%%%%%%%%%%%%%%%%%%%%%%%%%%%%%%%%%%%%%%%%%%%%%%%%%%%%

Waterramp suriname met cyanice


%%%%%%%%%%%%%%%%%%%%%%%%%%%%%%%%%%%%%%%%%%%%%%%%%%%%%%%%%%%%%%%%%
boeing 737 crashes


algemene vragen
oorzaken
\bibitem{gates18112020boeingcrisis} ... \LaTeX:\\ \url{https://www.seattletimes.com/business/boeing-aerospace/what-led-to-boeings-737-max-crisis-a-qa/}
\bibitem{boeing737maxsoftwareprobles} ... \LaTeX:\\ \url{https://www.schneier.com/blog/archives/2019/04/excellent_analy.html}
fout in de software
\bibitem{avetisov19032019boeingmalwarestate} ... \LaTeX:\\ \url{https://www.forbes.com/sites/georgeavetisov/2019/03/19/malware-at-30000-feet-what-the-737-max-says-about-the-state-of-airplane-software-security/?sh=4d26f7052a9e}
het nationaal veiligheidsbelang
\bibitem{thompson23112020nationalsecurityboeing} ... \LaTeX:\\ \url{https://www.forbes.com/sites/lorenthompson/2020/11/23/five-reasons-return-of-boeings-737-max-to-service-is-important-to-national-security/?sh=2128ea552018}
falend toezicht
\bibitem{gates21032019FAAControlSystem} ... \LaTeX:\\ \url{https://www.seattletimes.com/business/boeing-aerospace/failed-certification-faa-missed-safety-issues-in-the-737-max-system-implicated-in-the-lion-air-crash/}
onderzoeksrapport
\bibitem{faa18112020boeingreview} ... \LaTeX:\\ \url{https://www.faa.gov/foia/electronic_reading_room/boeing_reading_room/media/737_RTS_Summary.pdf}
\bibitem{wiki737maxgroundings} ... \LaTeX:\\ \url{https://en.wikipedia.org/wiki/Boeing_737_MAX_groundings}
veiligheidsrisico's
menselijke fouten
\bibitem{campbell02052019boengcrashhumanerrors} ... \LaTeX:\\ \url{https://www.theverge.com/2019/5/2/18518176/boeing-737-max-crash-problems-human-error-mcas-faa}
overzicht van crashes
\bibitem{hawkins22032019737maxairplanes} ... \LaTeX:\\ \url{https://www.theverge.com/2019/3/22/18275736/boeing-737-max-plane-crashes-grounded-problems-info-details-explained-reasons}
veiligheidsopmerking
\bibitem{thomas30082020737safest} ... \LaTeX:\\ \url{https://www.airlineratings.com/news/boeings-737-max-will-one-safest-aircraft-history/}
aanpassingen
\bibitem{boeing737maxdisplay} ... \LaTeX:\\ \url{https://www.boeing.com/commercial/737max/737-max-software-updates.page}
waarschuwingen//output signalen
\bibitem{fehrm24112020737changes} ... \LaTeX:\\ \url{https://leehamnews.com/2020/11/24/boeing-737-max-changes-beyond-mcas/}
software gerelateerde fouten
\bibitem{travis18042019737maxsoftwaredevop} ... \LaTeX:\\ \url{https://spectrum.ieee.org/aerospace/aviation/how-the-boeing-737-max-disaster-looks-to-a-software-developer}
onderzoeksrapport
de rol van de publieke opinie
\bibitem{barnett05052019737maxcrisis} ... \LaTeX:\\ \url{https://pubsonline.informs.org/do/10.1287/orms.2019.05.05/full/}
onderzoek van europese luchtvaart agentschap
\bibitem{easa27012021737maxsafereturn} ... \LaTeX:\\ \url{https://www.easa.europa.eu/newsroom-and-events/news/easa-declares-boeing-737-max-safe-return-service-europe}
veiligheidsvraagstuk
\bibitem{touitou11032019737tragedies} ... \LaTeX:\\ \url{https://phys.org/news/2019-03-boeing-max-safety-tragedies.html}
artikel over sensoren
\bibitem{hemmerdinger02022021737maxdeliveries} ... \LaTeX:\\ \url{https://www.flightglobal.com/airframers/boeing-delays-737-max-10-deliveries-two-years-to-2023/142245.article}
goedkeuring van europese luchtvaart autoriteiten
advies aan de faa
\bibitem{bielby27022021faaimprovesafety} ... \LaTeX:\\ \url{https://www.hstoday.us/subject-matter-areas/airport-aviation-security/oig-tells-faa-to-improve-safety-oversight-following-boeing-737-max-review/}
\bibitem{boyle18112020737maxupgrade} ... \LaTeX:\\ \url{https://www.geekwire.com/2020/faas-go-ahead-737-maxs-return-flight-kicks-off-massive-software-upgrade/}
\bibitem{bergstraburgess122019737maxMcasAlgorithm} ... \LaTeX:\\ \url{https://www.researchgate.net/publication/338420944_A_Promise_Theoretic_Account_of_the_Boeing_737_Max_MCAS_Algorithm_Affair}
achtergrond informatie
\bibitem{737mcas} ... \LaTeX:\\ \url{http://www.b737.org.uk/mcas.htm}
algemeen vertrouwen
\bibitem{newburger17052019boeingcrisis} ... \LaTeX:\\ \url{https://www.cnbc.com/2019/05/16/what-you-need-to-know-about-boeings-737-max-crisis.html}
toestemming europese autoriteiten
problemen
\bibitem{arstechnica22012020737problems} ... \LaTeX:\\ \url{https://arstechnica.com/information-technology/2020/01/737-max-fix-slips-to-summer-and-thats-just-one-of-boeings-problems/}
uitgebreid artikel over de onderzoeken en het vliegverbod
\bibitem{german190620217372yaftergrounded} ... \LaTeX:\\ \url{https://www.cnet.com/news/boeing-737-max-8-all-about-the-aircraft-flight-ban-and-investigations/}
computers als oorzaak
lessons learned
\bibitem{beningo02052019boeinglessons} ... \LaTeX:\\ \url{https://www.designnews.com/electronics-test/5-lessons-learn-boeing-737-max-fiasco}
\bibitem{duran05042019boeingspof} ... \LaTeX:\\ \url{https://www.eurocontrol.int/publication/effects-network-extra-standby-aircraft-and-boeing-737-max-grounding}
single point of failure
\bibitem{ID} ... \LaTeX:\\ \url{https://dmd.solutions/blog/2019/04/05/how-a-single-point-of-failure-spof-in-the-mcas-software-could-have-caused-the-boeing-737-max-crash-in-ethiopia/}
\bibitem{makichuck24012021737fearflying} ... \LaTeX:\\ \url{https://asiatimes.com/2021/01/boeings-737-max-and-the-fear-of-flying/}
lijst van tehnische aanpassingen
\bibitem{caa737modifications} ... \LaTeX:\\ \url{https://www.caa.co.uk/Consumers/Guide-to-aviation/Boeing-737-MAX/}
\bibitem{oestergaard14122020boeingdeliveries} ... \LaTeX:\\ \url{https://dsm.forecastinternational.com/wordpress/2020/12/14/airbus-and-boeing-report-november-2020-commercial-aircraft-orders-and-deliveries/}
code lek
\bibitem{reenberg787flaws} ... \LaTeX:\\ \url{https://www.wired.com/story/boeing-787-code-leak-security-flaws/}
\bibitem{fitch16092020737backlogrisks} ... \LaTeX:\\ \url{https://www.fitchratings.com/research/corporate-finance/boeing-737-max-return-backlog-risks-remain-16-09-2020}
Cultuurverandering, deregulatie, systeemwijziging of gewoon een kwestie van competentie
\bibitem{willis27082020737maxfailures} ... \LaTeX:\\ \url{https://www.aerospacetestinginternational.com/features/what-broke-the-737-max.html}
extra aanpassingen
\bibitem{ostrower11062020more737changes} ... \LaTeX:\\ \url{https://theaircurrent.com/aviation-safety/boeings-737-max-software-done-but-regulators-plot-more-changes-after-jets-return/}
wat ging er mis een analyse van een ex-iloot
De utoriteiten waren op de hoogte
\bibitem{hruska13122019faaknown737crashrate} ... \LaTeX:\\ \url{https://www.extremetech.com/extreme/303373-the-faa-knew-the-737-max-was-dangerous-and-kept-it-flying-anyway}
kwaliteiten van het alarmsysteem niet goed bekend
\bibitem{bloomberg26092019failedpred} ... \LaTeX:\\ \url{https://time.com/5687473/boeing-737-alarm-system/}
\bibitem{whiteman09072020boengcancelstock} ... \LaTeX:\\ \url{https://www.nasdaq.com/articles/boeing-gets-dealt-another-737-max-cancellation-blow.-what-it-means-for-boeing-stock-2020}
\bibitem{leopold09192019boeingreliability} ... \LaTeX:\\ \url{https://www.eetimes.com/boeing-crashes-highlight-a-worsening-reliability-crisis/}
veiligheidsvraagstuk
\bibitem{koenig11122019737crashesnofix} ... \LaTeX:\\ \url{https://www.latimes.com/business/story/2019-12-11/faa-boeing-737-max-crashes}
probleemanalyse, veiligheidsvraagstuk
\bibitem{dohertylindeman15032019737problems} ... \LaTeX:\\ \url{https://www.politico.com/story/2019/03/15/boeing-737-max-grounding-1223072}
falend toezicht
\bibitem{stodder02102019corruptoversight} ... \LaTeX:\\ \url{https://www.pogo.org/analysis/2019/10/corrupted-oversight-the-faa-boeing-and-the-737-max/}
\bibitem{afacwaLostSafeguards} ... \LaTeX:\\ \url{https://www.afacwa.org/the_inside_story_of_mcas_seattle_times}
doelstellingen en veiligheidsvraagstukken
\bibitem{swayne18032019profitssafety} ... \LaTeX:\\ \url{https://www.marxist.com/737-max-scandal-boeing-putting-profits-before-safety.htm}
\bibitem{freed26022021liftaustraliaban} ... \LaTeX:\\ \url{https://finance.yahoo.com/news/australia-lifts-ban-boeing-737-035817682.html?guccounter=1&guce_referrer=aHR0cHM6Ly93d3cuZ29vZ2xlLmNvbS8&guce_referrer_sig=AQAAAHZCJYy_0A5VS2WiPoCvH4xdrRNkmkdsv5EWJ2RLIz_AS-rxsTty6AF1_HlmJiRyWYqCXDi4p0Xs4isYkNkCq2Pfo-pQ60Xz_IfTNjm4FgoZiBMC4zpZlB6F0fwecrjE_ujAXZzG4xPJnWCd8-G3VLlPTY8h3H31eQ1i8hY9AIyy}
autoriteiten krijgen tik op de vingers
\bibitem{reed15032019softwareattention} ... \LaTeX:\\ \url{https://medium.com/@jpaulreed/the-737max-and-why-software-engineers-should-pay-attention-a041290994bd}
\bibitem{news17032019softwareexplains} ... \LaTeX:\\ \url{https://news.ycombinator.com/item?id=19414775}
\bibitem{legget21122020eu737maxsafe} ... \LaTeX:\\ \url{https://www.bbc.com/news/55366320}
\bibitem{marketscreener0103221737chinarecertification} ... \LaTeX:\\ \url{https://www.marketscreener.com/news/latest/China-studies-Boeing-737-MAX-recertification-wants-safety-concerns-fully-addressed--32569125/}
motor in brand
\bibitem{euractiv22022021737firegrounds} ... \LaTeX:\\ \url{https://www.euractiv.com/section/aviation/news/boeing-grounds-777s-after-engine-fire/}
\bibitem{benny18022019737returnUAE} ... \LaTeX:\\ \url{https://gulfnews.com/business/aviation/uae-airspace-to-see-return-of-boeing-737-max-1.1613627548923}
motor in brand gevlogen
\bibitem{biersmichel22022021777grounds} ... \LaTeX:\\ \url{https://techxplore.com/news/2021-02-boeing-urges-grounding-777s.html}
\bibitem{ID} ... \LaTeX:\\ \url{https://www.politico.eu/article/uk-temporarily-bans-some-boeing-aircraft-after-pratt-whitney-engine-incidents/}
\bibitem{reuters23022021777metalfatigue} ... \LaTeX:\\ \url{https://www.timeslive.co.za/news/world/2021-02-23-damage-to-united-boeing-777-engine-consistent-with-metal-fatigue--ntsb/}
faa was niet kritisch genoeg
\bibitem{ID} ... \LaTeX:\\ \url{https://federalnewsnetwork.com/government-news/2021/02/federal-watchdog-blasts-faa-over-certification-of-boeing-jet/}






%%%%%%%%%%%%%%%%%%%%%%%%%%%%%%%%%%%%%%%%%%%%%%%%%%%%%%%%%%%%%%%%%

china explosion 2015 tianjin

verhaal van brandweermannen
\bibitem{ID} ... \LaTeX:\\ \url{https://slate.com/human-interest/2015/08/chinese-explosion-aftermath-officials-investigate-causes-behind-warehouse-blast-and-death-of-88-firefighters.html}
artikel
\bibitem{ID} ... \LaTeX:\\ \url{https://www.chinafile.com/conversation/tianjin-explosion}
invloed van social media
\bibitem{ID} ... \LaTeX:\\ \url{https://www.economist.com/asia/2015/08/18/a-blast-in-tianjin-sets-off-an-explosion-online}
\bibitem{ID} ... \LaTeX:\\ \url{https://america.cgtn.com/2015/08/12/explosion-reported-in-tianjin-china}
\bibitem{ID} ... \LaTeX:\\ \url{https://factcheck.afp.com/no-photo-was-taken-chinese-city-tianjin-august-2015}
vergelijking van twee rampen
\bibitem{ID} ... \LaTeX:\\ \url{https://airshare.air-inc.com/how-does-the-beirut-explosion-compare-to-tianjin}
overheid en media
\bibitem{ID} ... \LaTeX:\\ \url{https://newbloommag.net/2015/08/17/tianjin-explosion/}
chemische industrie ondeer de loep
\bibitem{ID} ... \LaTeX:\\ \url{https://www.voanews.com/east-asia-pacific/tianjin-blast-puts-spotlight-chemical-industry}
\bibitem{ID} ... \LaTeX:\\ \url{https://abcnews.go.com/International/apocalyptic-aftermath-devastating-images-tianjin-china-explosions/story?id=33057017}
\bibitem{ID} ... \LaTeX:\\ \url{https://www.reachingoutacrossdurham.co.uk/osk/tianjin-explosion-2021}
\bibitem{ID} ... \LaTeX:\\ \url{https://aiche.onlinelibrary.wiley.com/doi/abs/10.1002/prs.11789}
\bibitem{ID} ... \LaTeX:\\ \url{https://www.automotivelogistics.media/thousands-of-cars-destroyed-in-tianjin-port-explosions/13570.article}
\bibitem{ID} ... \LaTeX:\\ \url{https://www.joc.com/port-news/asian-ports/port-tianjin/tianjin-port-explosions-could-be-most-expensive-maritime-disaster_20150826.html}
\bibitem{ID} ... \LaTeX:\\ \url{https://www.bloomberg.com/news/articles/2015-08-12/explosion-in-northern-china-shatters-windows-causes-injuries}
\bibitem{ID} ... \LaTeX:\\ \url{https://unece.org/fileadmin/DAM/env/documents/2016/TEIA/OECD_WGCA_24-27_OCT_2016/Session_3_Zhao_-__Introduction_of_Tianjin_Accident_-_Jinsong_Zhao.pdf}
gemaakte fouten
\bibitem{ID} ... \LaTeX:\\ \url{https://porteconomicsmanagement.org/pemp/contents/part6/port-resilience/site-2015-tianjin-port-explosions/}
\bibitem{ID} ... \LaTeX:\\ \url{https://www.alamy.com/stock-image-tianjin-china-17th-aug-2015-tianjin-explosion-aftermath-blast-site-165334778.html}
\bibitem{ID} ... \LaTeX:\\ \url{https://www.popularmechanics.com/technology/news/a16871/massive-explosions-china-city-of-tianjin/}
\bibitem{ID} ... \LaTeX:\\ \url{https://www.imago-images.com/st/0080815934}
\bibitem{ID} ... \LaTeX:\\ \url{https://www.chemistryworld.com/news/deadly-chemical-blast-at-chinese-port/8857.article}
\bibitem{ID} ... \LaTeX:\\ \url{https://www.process-worldwide.com/tianjin-explosion-from-chemical-perspective-insights-and-backgrounds-a-502381/}
vergelijking met andere explosies
\bibitem{ID} ... \LaTeX:\\ \url{https://apnews.com/article/lebanon-fires-us-news-explosions-middle-east-53f4206a7f1db0812262a15d22e1e58f}
invloed van de ramp op de industrie
\bibitem{ID} ... \LaTeX:\\ \url{https://fortune.com/2015/08/14/tianjin-port-explosion-shipping-delays/}
is er sprake van een doofpot
\bibitem{ID} ... \LaTeX:\\ \url{https://www.washingtontimes.com/news/2015/aug/20/inside-china-tianjin-explosions-cover-up-exposes-b/}
eigendomsverzekering
\bibitem{ID} ... \LaTeX:\\ \url{https://www.artemis.bm/news/tianjin-explosions-property-insurance-loss-could-reach-3-5bn-swiss-re/}
\bibitem{ID} ... \LaTeX:\\ \url{https://www.thechinastory.org/yearbooks/yearbook-2015/forum-the-abyss-%E5%9D%8E/tianjin-explosions/}
effecten op de lange termijn
\bibitem{ID} ... \LaTeX:\\ \url{https://www.flexport.com/blog/tianjin-explosion-effect-on-supply-chains/}
\bibitem{ID} ... \LaTeX:\\ \url{https://www.cicm.org.my/images/articles/CICM-Article-on-Tianjin-Blast-Oct2015.pdf}
lessons learned
\bibitem{ID} ... \LaTeX:\\ \url{https://www.genre.com/knowledge/blog/lessons-from-the-tianjin-explosion-en.html}
\bibitem{ID} ... \LaTeX:\\ \url{https://www.ft.com/content/ad62904c-44ce-11e5-b3b2-1672f710807b}
\bibitem{ID} ... \LaTeX:\\ \url{https://www.huffingtonpost.co.uk/2015/08/13/tianjin-explosion-china-shocking-footage-caught-on-camera_n_7980888.html}
\bibitem{ID} ... \LaTeX:\\ \url{https://www.thatsmags.com/china/post/19189/massive-fire-rocks-tianjin-port}
gevolgen voor de industrie
\bibitem{ID} ... \LaTeX:\\ \url{https://www.everstream.ai/risk-center/special-reports/the-jiangsu-yancheng-explosion/}
\bibitem{ID} ... \LaTeX:\\ \url{https://www.newyorker.com/news/news-desk/after-tianjin-an-outbreak-of-mistrust-in-china}
framing vanuit de chinese media
\bibitem{ID} ... \LaTeX:\\ \url{https://www.neliti.com/publications/101997/the-chinese-media-framing-of-the-2015s-tianjin-explosion}
\bibitem{ID} ... \LaTeX:\\ \url{https://www.reinsurancene.ws/chinese-insurers-settle-1-5-billion-tianjin-blast-claims/}
niewsartikel
\bibitem{ID} ... \LaTeX:\\ \url{https://www.thechemicalengineer.com/news/update-78-confirmed-dead-after-chinese-chemicals-plant-explosion/}
\bibitem{ID} ... \LaTeX:\\ \url{https://www.caixinglobal.com/2016-11-10/chinese-executive-receives-suspended-death-sentence-over-2015-tianjin-warehouse-blast-101006325.html}
toegang tot de ramplplek vanuit de okale journalistiek
\bibitem{ID} ... \LaTeX:\\ \url{https://chinadigitaltimes.net/2015/08/he-xiaoxin-how-far-can-i-go-and-how-much-can-i-do/}
artikel
\bibitem{ID} ... \LaTeX:\\ \url{https://www.wnpr.org/post/china-examines-aftermath-immense-twin-explosions-killed-dozens}
\bibitem{ID} ... \LaTeX:\\ \url{https://theconversation.com/what-is-ammonium-nitrate-the-chemical-that-exploded-in-beirut-143979}
\bibitem{ID} ... \LaTeX:\\ \url{https://chemicalwatch.com/36730/nationwide-inspections-in-china-follow-tianjin-explosion}
\bibitem{ID} ... \LaTeX:\\ \url{https://www.thehindu.com/news/international/investigation-begun-into-china-gas-explosion-as-toll-rises/article34818324.ece}
\bibitem{ID} ... \LaTeX:\\ \url{https://santiagotimes.cl/2019/03/24/64-killed-600-injured-in-china-chemical-plant-blast/}
oorzaken
\bibitem{ID} ... \LaTeX:\\ \url{https://klingecorp.com/blog/what-caused-the-tianjin-explosions/}
case study
\bibitem{ID} ... \LaTeX:\\ \url{https://www.preventionweb.net/educational/view/57235}
niewsartikel
\bibitem{ID} ... \LaTeX:\\ \url{https://www.cnbc.com/2015/08/12/explosion-in-tianjin-china.html}
chronologische uiteenzetting
\bibitem{ID} ... \LaTeX:\\ \url{https://www.aria.developpement-durable.gouv.fr/wp-content/files_mf/A46803_a46803_fiche_impel_006.pdf}
corruptie
\bibitem{ID} ... \LaTeX:\\ \url{https://www.nytimes.com/2015/08/31/world/asia/behind-tianjin-tragedy-a-company-that-flouted-regulations-and-reaped-profits.html}
mismanagement als oorzaak
\bibitem{ID} ... \LaTeX:\\ \url{https://www.nytimes.com/2016/02/06/world/asia/tianjin-explosions-were-result-of-mismanagement-china-finds.html}
\bibitem{ID} ... \LaTeX:\\ \url{https://cen.acs.org/articles/94/web/2016/02/Chinese-Investigators-Identify-Cause-Tianjin.html}
autoriteiten publiceren onderoeksrapport
\bibitem{ID} ... \LaTeX:\\ \url{https://cen.acs.org/articles/94/i7/Chinese-Investigators-Identify-Cause-Tianjin.html}
fotos van de rampplek
\bibitem{ID} ... \LaTeX:\\ \url{https://www.theatlantic.com/photo/2015/08/photos-of-the-aftermath-of-the-massive-explosions-in-tianjin-china/401228/}
\bibitem{ID} ... \LaTeX:\\ \url{https://edition.cnn.com/2015/08/13/asia/china-tianjin-explosions/index.html}
niuwesartiekel}
\bibitem{ID} ... \LaTeX:\\ \url{https://www.cbc.ca/news/world/china-explosion-tianjin-1.3189455}
verantwoordelijke
\bibitem{ID} ... \LaTeX:\\ \url{https://www.thestar.com/news/world/2016/11/09/chinese-executive-gets-death-sentence-over-tianjin-explosion-in-2015.html}
risicobeperking/controle
\bibitem{ID} ... \LaTeX:\\ \url{https://www.swissre.com/en/china/news-insights/articles/analysis-of-tianjin-port-explosion-china.html}
censuur
\bibitem{ID} ... \LaTeX:\\ \url{https://foreignpolicy.com/2015/09/10/censored-china-young-survivor-tianjin-explosion-viral-post/}
censuur
\bibitem{ID} ... \LaTeX:\\ \url{https://qz.com/756872/a-year-after-the-tianjin-blast-public-mourning-and-discussion-about-it-are-still-censored-in-china/}
verschillende artikelen
\bibitem{ID} ... \LaTeX:\\ \url{https://www.scmp.com/topics/tianjin-warehouse-explosion-2015}
\bibitem{ID} ... \LaTeX:\\ \url{https://www.wsj.com/articles/BL-CJB-27664}
\bibitem{ID} ... \LaTeX:\\ \url{https://www.nbcnews.com/news/world/tianjin-explosions-californian-witness-filmed-dramatic-china-blasts-n409701}
\bibitem{ID} ... \LaTeX:\\ \url{https://ui.adsabs.harvard.edu/abs/2016AGUFM.S13D..06P/abstract}
afwikkeling van de ramp
\bibitem{ID} ... \LaTeX:\\ \url{https://chinadialogue.net/en/pollution/9188-back-to-the-blast-zone-one-year-after-the-tianjin-explosion/}
\bibitem{ID} ... \LaTeX:\\ \url{https://www.wired.com/2015/08/chinas-huge-tianjin-explosion-looked-like-space/}
\bibitem{ID} ... \LaTeX:\\ \url{https://www.abc.net.au/news/2015-08-13/explosion-rocks-north-chinese-city-of-tianjin/6693336?nw=0}
ambtenaren onderzocht
\bibitem{ID} ... \LaTeX:\\ \url{https://thediplomat.com/2015/08/23-executives-government-officials-under-investigation-for-role-in-tianjin-explosions/}
\bibitem{ID} ... \LaTeX:\\ \url{http://america.aljazeera.com/articles/2015/8/13/at-least-50-dead-and-hundreds-injured-in-chinese-warehouse-explosion.html}
risico-inschatting
\bibitem{ID} ... \LaTeX:\\ \url{https://www.mdpi.com/2071-1050/12/3/1169/htm}
\bibitem{ID} ... \LaTeX:\\ \url{https://www.mdpi.com/2071-1050/12/3/1169/htm}
\bibitem{ID} ... \LaTeX:\\ \url{https://www.cbsnews.com/news/tianjin-port-china-massive-explosion-hundreds-injured/}
\bibitem{ID} ... \LaTeX:\\ \url{https://www.hkjcdpri.org.hk/download/casestudies/Tianjin_CASE.pdf}
\bibitem{ID} ... \LaTeX:\\ \url{https://time.com/3996168/tianjin-explosion-china-pictures/}
onderzoeksrapport
\bibitem{ID} ... \LaTeX:\\ \url{https://www.hfw.com/Tianjin-Port-explosion-August-2015}
\bibitem{ID} ... \LaTeX:\\ \url{https://news.un.org/en/story/2015/08/506912-following-tianjin-explosion-un-expert-calls-china-ensure-transparent}
\bibitem{ID} ... \LaTeX:\\ \url{https://www.france24.com/en/20150812-huge-explosions-rock-chinese-city-tianjin}
\bibitem{ID} ... \LaTeX:\\ \url{https://choice.npr.org/index.html?origin=https://www.npr.org/2015/08/14/432280627/what-caused-the-warehouse-explosions-in-tianjin-china}
123 verantwoordelijken
\bibitem{ID} ... \LaTeX:\\ \url{https://www.bbc.com/news/world-asia-china-35506311}
\bibitem{ID} ... \LaTeX:\\ \url{https://www.washingtonpost.com/gdpr-consent/?next_url=https%3a%2f%2fwww.washingtonpost.com%2fnews%2fworldviews%2fwp%2f2015%2f08%2f12%2fvideos-show-chinese-city-of-tianjin-rocked-by-enormous-explosion%2f}
lang artiekel
\bibitem{ID} ... \LaTeX:\\ \url{https://www.businessinsider.com/the-chemical-explosion-in-china-killed-more-than-100-people-and-the-devastation-is-unreal-2015-8?international=true&r=US&IR=T}
\bibitem{ID} ... \LaTeX:\\ \url{https://pubmed.ncbi.nlm.nih.gov/27311537/}
\bibitem{ID} ... \LaTeX:\\ \url{https://www.reuters.com/article/us-china-blast-insurance-idUSKCN0QM0N220150817}
\bibitem{ID} ... \LaTeX:\\ \url{https://www.sciencedirect.com/science/article/abs/pii/S0305417916300079}
\bibitem{ID} ... \LaTeX:\\ \url{https://en.wikipedia.org/wiki/2015_Tianjin_explosions}
\bibitem{ID} ... \LaTeX:\\ \url{https://www.bbc.com/news/world-asia-china-33844084}
\bibitem{ID} ... \LaTeX:\\ \url{https://www.independent.co.uk/news/world/asia/tianjin-explosion-photos-china-chemical-factory-accident-crater-revealed-a7199591.html}

veiigheidshandhaving
\bibitem{ID} ... \LaTeX:\\ \url{https://www.ilo.org/legacy/english/protection/safework/ctrl_banding/toolkit/main_guide.pdf}
\bibitem{ID} ... \LaTeX:\\ \url{https://echa.europa.eu/documents/10162/21332507/guide_chemical_safety_sme_en.pdf}
\bibitem{ID} ... \LaTeX:\\ \url{https://ec.europa.eu/taxation_customs/dds2/SAMANCTA/EN/Safety/AppendixD_EN.htm}
\bibitem{ID} ... \LaTeX:\\ \url{https://www.ilo.org/safework/info/publications/WCMS_113134/lang--en/index.htm}


%%%%%%%%%%%%%%%%%%%%%%%%%%%%%%%%%%%%%%%%%%%%%%%%%%%%%%%%%%%%%%%%%
tesla autopilot crashes


veiigheidsrisico
\bibitem{evan01042019teslaautopilotIntersection} ... \LaTeX:\\ \url{https://spectrum.ieee.org/cars-that-think/transportation/self-driving/three-small-stickers-on-road-can-steer-tesla-autopilot-into-oncoming-lane}

\bibitem{testVehicleSafetyReport} ... \LaTeX:\\ \url{https://www.tesla.com/VehicleSafetyReport}
veiligheidsrapport mbt autopilot
\bibitem{lambert31062020q2safetyreport} ... \LaTeX:\\ \url{https://electrek.co/2020/07/31/tesla-q2-2020-safety-report-strong-improvement-autopilot-accidents/}
consumentenrapport
bluetooth veiligheidsvraagstuk
\bibitem{wiredBloutoothHackTesla} ... \LaTeX:\\ \url{https://www.wired.com/story/tesla-model-x-hack-bluetooth/}
veiigheidsvraagstuk vanwege touch screen
\bibitem{preston14012021NHTSATeslaRecall} ... \LaTeX:\\ \url{https://www.consumerreports.org/car-recalls-defects/nhtsa-asks-tesla-to-recall-model-s-model-x-touch-screen-safety-issues/}
veiligheidsvraagstuk
\bibitem{cio25112020belgianTeslaHack} ... \LaTeX:\\ \url{https://cio.economictimes.indiatimes.com/news/digital-security/security-researchers-hack-steal-tesla-model-x-within-minutes/79406553}
veiligheidsvraagstuk
rapport over autopilot
\bibitem{templeton06092019HTSBReportTesla} ... \LaTeX:\\ \url{https://www.forbes.com/sites/bradtempleton/2019/09/06/ntsb-report-on-tesla-autopilot-accident-shows-whats-inside-and-its-not-pretty-for-fsd/?sh=6905e7d4dc55}
de invloed van de bestuurder bij tesla ongeluk
\bibitem{Dickey08012021SuddenTeslaAcceleration} ... \LaTeX:\\ \url{https://techcrunch.com/2021/01/08/nhtsa-tesla-sudden-unintended-acceleration-driver-error/}
veiligheidsvraagstuk
\bibitem{darkReading17112020TeslaBackup} ... \LaTeX:\\ \url{https://www.darkreading.com/threat-intelligence/security-risks-discovered-in-tesla-backup-gateway/d/d-id/1339462}
veiligheidsvraagstuk
\bibitem{leyden23032020TeslaInterfaceHack} ... \LaTeX:\\ \url{https://portswigger.net/daily-swig/web-based-attack-crashes-tesla-driver-interface}
veiigheidsvraagstuk
\bibitem{huddlestonjr03042019ChineseTeslaHack} ... \LaTeX:\\ \url{https://www.cnbc.com/2019/04/03/chinese-hackers-tricked-teslas-autopilot-into-switching-lanes.html}
veiligheidsvraagstuk
veiligheidsvraagstuk
\bibitem{heilweil26022020teslaAutopilot} ... \LaTeX:\\ \url{https://www.vox.com/recode/2020/2/26/21154502/tesla-autopilot-fatal-crashes}
rapport over ongeluk
veiligheidsvraagstuk
veiligheidsvraagstuk
\bibitem{blanco04102019NHTSATesla} ... \LaTeX:\\ \url{https://www.caranddriver.com/news/a29369387/nhtsa-tesla-safety/}
veiligheidsvraagstuk
ransomware aanval op tesla
tesla batterij is veiligheidsvraagstuk geworden
\bibitem{mitchell01072020teslabatterycooling} ... \LaTeX:\\ \url{https://www.latimes.com/business/story/2020-07-01/federal-safety-officials-probe-tesla-battery-cooling-system}
ongeluk
\bibitem{bbc26022020AutopilotCrash} ... \LaTeX:\\ \url{https://www.bbc.com/news/technology-51645566}
veiligheidsvraagstuk
veiligheidsvraagstuk
\bibitem{stumpff04052020TeslaPersonalData} ... \LaTeX:\\ \url{https://www.thedrive.com/news/33272/tesla-discarded-old-car-parts-with-customers-personal-data-passwords-report}
dodelijk ongeluk
\bibitem{levin08062018teslaautopilotsafety} ... \LaTeX:\\ \url{https://www.theguardian.com/technology/2018/jun/07/tesla-fatal-crash-silicon-valley-autopilot-mode-report}
veiligheidsvraagstuk: ransomware
veiligheidsvraagstuk: medewerker in de fout
\bibitem{cbrook06082021TeslaInsideDataThreft} ... \LaTeX:\\ \url{https://digitalguardian.com/blog/tesla-data-theft-case-illustrates-danger-insider-threat}
\bibitem{shilling25022021Tesla} ... \LaTeX:\\ \url{https://jalopnik.com/tesla-is-stopping-some-model-3-production-report-1846353323}
veiligheidsvraagstuk: hackers je systeem laten testen
verdedigen tegenover ransomware
veiligheidsrisico
prijzen omlaag
autopilot
\bibitem{randall05112019modelSurvey} ... \LaTeX:\\ \url{https://www.bloomberg.com/graphics/2019-tesla-model-3-survey/autopilot.html}
malware door een medewerker
\bibitem{ID} ... \LaTeX:\\ \url{https://www.teslarati.com/tesla-employee-fbi-thwarts-russian-cybersecurity-attack/}
dodelijk ongeluk
\bibitem{ID} ... \LaTeX:\\ \url{https://www.marketwatch.com/story/apple-engineer-killed-in-tesla-suv-crash-on-silicon-valley-freeway-was-playing-videogame-ntsb-2020-02-25}
\bibitem{fottrell03092018TeslaSecurityChecks} ... \LaTeX:\\ \url{https://www.marketwatch.com/story/nearly-100-of-teslas-stolen-in-the-us-since-2011-have-been-recovered-2018-08-10}
waarom een tesla stelen bijna onmogelijk is
\bibitem{ID} ... \LaTeX:\\ \url{https://www.welivesecurity.com/2019/03/25/white-hats-hack-tesla-keep/}
veiligheidsonderzoek
\bibitem{ID} ... \LaTeX:\\ \url{https://www.tripwire.com/state-of-security/security-data-protection/tesla-encouraging-good-faith-security-research-in-bug-bounty-program/}
softwarefout maakt diestal mogelijk
\bibitem{kirk26112020modelX} ... \LaTeX:\\ \url{https://www.bankinfosecurity.com/tesla-model-x-stolen-in-minutes-using-software-flaws-a-15462}
fouten ontdekt in onderzoek
\bibitem{ID} ... \LaTeX:\\ \url{https://www.cnet.com/roadshow/news/tesla-ev-appeal-loyalty-study/}
\bibitem{bbc24022021hyundaiBatteryFireFix} ... \LaTeX:\\ \url{https://www.bbc.com/news/technology-56156801}








tesla cloud gehacked
\bibitem{ID} ... \LaTeX:\\ \url{https://arstechnica.com/information-technology/2018/02/tesla-cloud-resources-are-hacked-to-run-cryptocurrency-mining-malware/}
\bibitem{ID} ... \LaTeX:\\ \url{https://www.motortrend.com/news/tesla-model-y-ev-safety-quality-issues-problems/}
\bibitem{ID} ... \LaTeX:\\ \url{https://securityledger.com/2019/04/hackers-remotely-steer-tesla-model-s-using-autopilot-system/}
\bibitem{ID} ... \LaTeX:\\ \url{https://www.pcmag.com/news/report-tesla-suspends-model-3-production-in-california-until-march-7}
\bibitem{ID} ... \LaTeX:\\ \url{https://www.scmp.com/business/money/article/3121173/tesla-conduct-complete-self-inspection-after-chinese-regulators}
\bibitem{ID} ... \LaTeX:\\ \url{https://www.businesswire.com/news/home/20180220005222/en/RedLock-Releases-Cloud-Security-Report-Highlighting-Focus-on-Shared-Responsibilities-Uncovers-Cloud-Related-Exposures-at-Tesla}
\bibitem{ID} ... \LaTeX:\\ \url{https://www.epa.gov/automotive-trends/highlights-automotive-trends-report}
\bibitem{ID} ... \LaTeX:\\ \url{https://www.livemint.com/Companies/o2QLbtJc9EQ7ZcpxqgFbBP/Teslas-reward-for-finding-security-bugs-Model-3.html}
\bibitem{evans11062018} ... \LaTeX:\\ \url{https://revealnews.org/blog/tesla-fired-safety-official-for-reporting-unsafe-conditions-lawsuit-says/}
\bibitem{ID} ... \LaTeX:\\ \url{https://heimdalsecurity.com/blog/security-alert-teslacrypt-4-0-unbreakable-encryption-worse-data-leakage/}
\bibitem{ID} ... \LaTeX:\\ \url{https://www.eweek.com/cloud/tesla-cloud-account-data-breach-revealed-in-redlock-security-report/}
\bibitem{hawkins22102022} ... \LaTeX:\\ \url{https://www.theverge.com/2020/10/21/21527577/tesla-full-self-driving-autopilot-beta-software-update}
\bibitem{ID} ... \LaTeX:\\ \url{file:///C:/Users/gally/Downloads/applsci-10-02749-v2.pdf}
\bibitem{gritti24062020tesladataengine} ... \LaTeX:\\ \url{https://www.braincreators.com/brainpower/insights/teslas-data-engine-and-what-we-should-all-learn-from-it}
\bibitem{ID} ... \LaTeX:\\ \url{https://bernardmarr.com/default.asp?contentID=1251}
\bibitem{ID} ... \LaTeX:\\ \url{https://arstechnica.com/cars/2019/10/how-teslas-latest-acquisition-could-accelerate-autopilot-development/}
\bibitem{bouchard07052019teslaDeepLearning} ... \LaTeX:\\ \url{https://towardsdatascience.com/teslas-deep-learning-at-scale-7eed85b235d3}
\bibitem{ID} ... \LaTeX:\\ \url{file:///C:/Users/gally/Downloads/applsci-10-02749-v2.pdf}
\bibitem{Srikanth2019teslabigdata} ... \LaTeX:\\ \url{https://www.techiexpert.com/how-tesla-is-using-artificial-intelligence-and-big-data/}
\bibitem{rangaiah25022020teslaAI} ... \LaTeX:\\ \url{https://www.analyticssteps.com/blogs/how-tesla-making-use-artificial-intelligence-its-operations}
\bibitem{marr08012018taslabigdataAI} ... \LaTeX:\\ \url{https://www.forbes.com/sites/bernardmarr/2018/01/08/the-amazing-ways-tesla-is-using-artificial-intelligence-and-big-data/?sh=5e396aa24270}
\bibitem{bdickson29072020teslalevelfive = {author},	ALTeditor = {editor},	title = {title},	date = {date},	url = {"https://bdtechtalks.com/2020/07/29/self-driving-tesla-car-deep-learning/}
\bibitem{dcruz17062022tesladesignthink} ... \LaTeX:\\ \url{https://www.mygreatlearning.com/blog/teslas-new-ai-for-self-driving-cars/}
\bibitem{ID} ... \LaTeX:\\ \url{https://www.techiexpert.com/how-tesla-is-using-artificial-intelligence-and-big-data/}





\bibitem{mcfarland22042021selfdrivingrisks} ... \LaTeX:\\ \url{https://www.cnn.com/2021/04/21/tech/tesla-full-self-driving-launch/index.html}
\bibitem{hawkins18032021fedgovinvest} ... \LaTeX:\\ \url{https://www.theverge.com/2021/3/18/22338427/tesla-autopilot-crash-michigan-nhtsa-investigation}
\bibitem{berry21042021teslacrashtexas} ... \LaTeX:\\ \url{https://www.wionews.com/technology/doctor-among-victims-of-lethal-tesla-car-crash-in-texas-378950}
\bibitem{hull23072021regulatorsaftercrash} ... \LaTeX:\\ \url{https://www.bloomberg.com/news/newsletters/2021-06-23/hyperdrive-daily-after-30-tesla-crashes-what-s-a-regulator-to-do}
\bibitem{wikiTeslaAutopilot} ... \LaTeX:\\ \url{https://en.wikipedia.org/wiki/Tesla_Autopilot}
\bibitem{nhtsaAutomatedVehiclesSafety} ... \LaTeX:\\ \url{https://www.nhtsa.gov/technology-innovation/automated-vehicles-safety}
\bibitem{dowling23042021autopilottricking} ... \LaTeX:\\ \url{https://www.caradvice.com.au/947080/elon-musk-responds-to-deadly-texas-tesla-crash-as-consumer-reports-reveals-how-autopilot-can-be-tricked/}
\bibitem{wilson19042021teslacrashregulators} ... \LaTeX:\\ \url{https://usa.streetsblog.org/2021/04/19/regulators-could-have-prevented-fatal-tesla-crash/}

\bibitem{seamans22062021aikillerap} ... \LaTeX:\\ \url{https://www.brookings.edu/research/autonomous-vehicles-as-a-killer-app-for-ai/}
\bibitem{mitchell24022020AIDataTesla} ... \LaTeX:\\ \url{https://www.latimes.com/business/story/2020-02-24/autopilot-data-secrecy}
\bibitem{denneyjdsupraFeds} ... \LaTeX:\\ \url{https://www.jdsupra.com/post/contentViewerEmbed.aspx?fid=9844cae0-aa5a-45a5-988f-7f02fa5709c1}
\bibitem{siddiqui22102020TeslaCriticism} ... \LaTeX:\\ \url{https://www.washingtonpost.com/technology/2020/10/21/tesla-self-driving/}
\bibitem{ackerman01072016TeslaImperfect} ... \LaTeX:\\ \url{https://spectrum.ieee.org/cars-that-think/transportation/self-driving/fatal-tesla-autopilot-crash-reminds-us-that-robots-arent-perfect}
\bibitem{greene04092019misuseautopilot} ... \LaTeX:\\ \url{https://thenextweb.com/news/another-tesla-owner-is-dead-because-of-autopilot}
\bibitem{michralli26112019ubserautocarcrsash} ... \LaTeX:\\ \url{https://towardsdatascience.com/another-self-driving-car-accident-another-ai-development-lesson-b2ce3dbb4444}
\bibitem{pitmann21072021wrongfullautodeath} ... \LaTeX:\\ \url{https://www.theautochannel.com/news/2021/07/21/1024631-is-it-still-wrongful-death-if-car-is-driving-itself.html}
\bibitem{stackexchange102019teslacarmistake} ... \LaTeX:\\ \url{https://ai.stackexchange.com/questions/1488/why-did-a-tesla-car-mistake-a-truck-with-a-bright-sky}
\bibitem{tasking07062017TeslaAugmentedSafety} ... \LaTeX:\\ \url{https://resources.tasking.com/p/benefits-tesla-autopilot-and-how-adas-will-save-lives}
\bibitem{griemannExaminSelfDriving} ... \LaTeX:\\ \url{https://www.jipitec.eu/issues/jipitec-9-3-2018/4806}
\bibitem{Harkey30052019SafeSystemVehicle} ... \LaTeX:\\ \url{https://static.tti.tamu.edu/conferences/traffic-safety19/presentations/lunch/harkey.pdf}
\bibitem{ID} ... \LaTeX:\\ \url{https://thepressfree.com/have-google-and-amazon-backed-the-wrong-technology/}
\bibitem{ID} ... \LaTeX:\\ \url{https://www.irishtimes.com/business/innovation/robotaxis-have-google-and-amazon-backed-the-wrong-technology-1.4626749}
\bibitem{ID} ... \LaTeX:\\ \url{https://www.afr.com/technology/how-teslas-autopilot-got-it-wrong-in-fatal-crash-20160704-gpxsje}
\bibitem{ID} ... \LaTeX:\\ \url{https://economictimes.indiatimes.com/markets/stocks/news/what-me-worry-fed-chiefs-emotional-tone-can-drive-markets-study-suggests/articleshow/84618073.cms}
\bibitem{ID} ... \LaTeX:\\ \url{https://www.ehstoday.com/safety/article/21919260/ntsb-fatal-crash-involving-tesla-autopilot-resulted-from-driver-errors-overreliance-on-automation}
\bibitem{ID} ... \LaTeX:\\ \url{https://www.vanityfair.com/news/2016/07/how-the-media-screwed-up-the-fatal-tesla-accident}


tesla crash report
\bibitem{shepardson18062021TeslaDeaths} ... \LaTeX:\\ \url{https://www.reuters.com/business/autos-transportation/us-safety-agency-says-it-has-opened-probes-into-10-tesla-crash-deaths-since-2016-2021-06-17/}
\bibitem{hawkins30062021nhtsarequiresreporting} ... \LaTeX:\\ \url{https://www.politico.com/news/2021/05/18/ntsb-tesla-owner-was-in-drivers-seat-before-april-texas-crash-489272}
\bibitem{hawkins10052021autopilotnotavailable} ... \LaTeX:\\ \url{https://www.theverge.com/2021/5/10/22429198/tesla-ntsb-texas-crash-driverless-preliminary-report}
\bibitem{szymkowski29062021nhtsaTeslaCrashReports} ... \LaTeX:\\ \url{https://www.cnet.com/roadshow/news/tesla-autopilot-nhtsa-crash-report-self-driving-car-driver-assist-system/}
\bibitem{abc1112052021AutopilotNotinTeslaCrash} ... \LaTeX:\\ \url{https://abc11.com/tesla-crash-battery-fire-national-transportation-safety-board-driverless/10619772/}
\bibitem{ankel18062021regulatorsinvestigateAutopilot} ... \LaTeX:\\ \url{https://www.businessinsider.com/tesla-autopilot-crashes-regulators-open-probes-into-30-report-2021-6?international=true&r=US&IR=T}
\bibitem{sommerfield12072021NHTSAmandateresult} ... \LaTeX:\\ \url{https://driving.ca/column/lorraine/lorraine-explains-what-the-nhtsas-self-driving-car-crash-reporting-mandate-will-find-out}
\bibitem{ID} ... \LaTeX:\\ \url{https://www.teslarati.com/tesla-model-s-crash-texas-ntsb-preliminary-report/}
\bibitem{ID} ... \LaTeX:\\ \url{https://insideevs.com/news/506498/ntsb-report-tesla-texas-crash/}
\bibitem{ID} ... \LaTeX:\\ \url{https://electrek.co/2021/06/03/tesla-tsla-crashes-report-new-orders-in-china-free-falling/}
\bibitem{ID} ... \LaTeX:\\ \url{https://www.newsy.com/stories/ntsb-releases-report-on-fatal-tesla-crash/}
\bibitem{ID} ... \LaTeX:\\ \url{https://www.ndtv.com/world-news/autopilot-not-used-in-april-tesla-crash-says-us-report-2439146}
\bibitem{ID} ... \LaTeX:\\ \url{https://www.autocar.co.nz/autocar-news-app/fatal-driverless-tesla-crash-report-shows-autopilot-not-to-blame}
\bibitem{ID} ... \LaTeX:\\ \url{https://teleperformance-waha.sabacloud.com/Saba/Web_spf/EU2PRD0152/app/dashboard}
\bibitem{ID} ... \LaTeX:\\ \url{https://www.independent.co.uk/news/world/americas/tesla-texas-crash-model-s-autopilot-b1845286.html}
\bibitem{ID} ... \LaTeX:\\ \url{https://www.wired.com/2017/01/probing-teslas-deadly-crash-feds-say-yay-self-driving/}
\bibitem{saferoardsCrashesAutonomousvehicles} ... \LaTeX:\\ \url{https://saferoads.org/wp-content/uploads/2020/03/AV-Crash-List-with-Photos-February-2020.pdf}
\bibitem{ID} ... \LaTeX:\\ \url{https://mashable.com/article/nthsa-tesla-autopilot-model-x-crash-investigation}
\bibitem{stephardson18032021revieuwingtesla} ... \LaTeX:\\ \url{https://www.usnews.com/news/top-news/articles/2021-03-18/us-safety-agency-reviewing-23-tesla-crashes-three-from-recent-weeks}
\bibitem{krishner30062021NHTSAreport} ... \LaTeX:\\ \url{https://chicago.suntimes.com/consumer-affairs/2021/6/30/22557122/nhtsa-automated-driving-crash-reports-tesla-national-highway-traffic-safety-administration}
\bibitem{gitlin11052021autopilot} ... \LaTeX:\\ \url{https://arstechnica.com/cars/2021/05/ntsb-finds-no-reason-to-suspect-autopilot-in-fatal-tesla-crash/}
\bibitem{ID} ... \LaTeX:\\ \url{https://jalopnik.com/the-ntsb-to-partially-blame-teslas-autopilot-in-fatal-c-1803136365}
\bibitem{mitchell19012017investigationstop} ... \LaTeX:\\ \url{https://www.latimes.com/business/autos/la-fi-hy-tesla-autopilot-20170119-story.html}
\bibitem{gordon10052021teslaprelimreport} ... \LaTeX:\\ \url{https://www.vice.com/en/article/z3xxaw/ntsb-releases-preliminary-report-on-tesla-crash-that-killed-two-people}
\bibitem{shaper07062018} ... \LaTeX:\\ \url{https://choice.npr.org/index.html?origin=https://www.npr.org/2018/06/07/618081406/no-driver-input-detected-in-seconds-before-deadly-tesla-crash-ntsb-finds}
\bibitem{cochran18042021nodriverTeslaCrash} ... \LaTeX:\\ \url{https://www.click2houston.com/news/local/2021/04/18/2-men-dead-after-fiery-tesla-crash-in-spring-officials-say/}
\bibitem{habib28062016NHTSATeslaReport} ... \LaTeX:\\ \url{https://static.nhtsa.gov/odi/inv/2016/INCLA-PE16007-7876.pdf}
\bibitem{firstpress11052021fatalnonautopilot} ... \LaTeX:\\ \url{https://www.firstpost.com/tech/news-analysis/tesla-model-s-involved-in-fatal-crash-in-the-us-did-not-use-autopilot-says-ntsb-report-9609911.html}
\bibitem{raynal20042021probeTeslaCrash} ... \LaTeX:\\ \url{https://www.autoweek.com/news/green-cars/a36173804/both-local-police-and-nhtsa-probe-tesla-crash/}
\bibitem{tiungteslasoftwarecrash} ... \LaTeX:\\ \url{https://www.zdnet.com/article/apple-and-tesla-under-fire-over-software-engineers-fatal-autopilot-crash/}
\bibitem{globaltimes08052021guangdongcrash,	ALTauthor = {global times},	ALTeditor = {editor},	title = {Tesla crash in Guangdong sparks new round of safety concerns},	date = {date},	url = {"https://www.globaltimes.cn/page/202105/1222902.shtml}
\bibitem{anderson30042021secondteslacrash,	ALTauthor = {Brad Anderson},	ALTeditor = {editor},	title = {NTSB To Release Report On Deadly Tesla Crash Within A Month},	date = {date},	url = {"https://www.carscoops.com/2021/04/ntsb-to-release-report-on-deadly-tesla-crash-within-a-month/}
	\bibitem{oremus21062017fatalTeslaCrash} ... \LaTeX:\\ \url{https://slate.com/technology/2017/06/a-new-report-on-what-happened-in-the-fatal-tesla-autopilot-crash.html}
	\bibitem{guardian15052021teslacrashHandsOnWheel} ... \LaTeX:\\ \url{https://www.theguardian.com/us-news/2021/may/15/tesla-fatal-california-crash-autopilot}
	\bibitem{Puzzanghera13092017TeslaSharesBlame} ... \LaTeX:\\ \url{https://www.stuff.co.nz/motoring/news/96797272/tesla-shares-some-blame-in-fatal-autopilot-crash-report}
	\bibitem{jaillet02022017teslaAutopilotLimitations} ... \LaTeX:\\ \url{https://www.overdriveonline.com/business/article/14891759/dot-report-on-fatal-2016-tesla-crash-with-tractor-trailer-blames-limitations-of-autopilot-mode}
	\bibitem{reuters03102019teslaAutoParkingFail} ... \LaTeX:\\ \url{https://venturebeat.com/2019/10/03/regulators-investigate-teslas-automated-parking-feature-following-crash-reports/}
	\bibitem{dowling23042021} ... \LaTeX:\\ \url{https://www.caradvice.com.au/947080/elon-musk-responds-to-deadly-texas-tesla-crash-as-consumer-reports-reveals-how-autopilot-can-be-tricked/}
	\bibitem{young05112021fatalTeslaReport} ... \LaTeX:\\ \url{https://www.consumeraffairs.com/news/ntsb-releases-report-on-fatal-tesla-crash-in-texas-051121.html}
	\bibitem{kierstein18032021teslaAutopilotCrashStationary} ... \LaTeX:\\ \url{https://www.motortrend.com/news/tesla-michigan-state-autopilot-crash-report/}
	\bibitem{janssen20062017teslacrashdetailflorida} ... \LaTeX:\\ \url{https://tweakers.net/nieuws/126145/onderzoeksraad-vs-publiceert-technische-details-tesla-crash-in-florida.html}
	\bibitem{ID} ... \LaTeX:\\ \url{https://gizmodo.com/new-documents-reveal-one-driver-s-agony-and-confusion-d-1841720801}
	
	
	\bibitem{ID} ... \LaTeX:\\ \url{https://www.google.com/search?q=tesla+crash+report&rlz=1C1AVUC_enNL953NL953&ei=p3kNYa6sLI_UsAeSoZrwDw&start=100&sa=N&ved=2ahUKEwjum77s_ZzyAhUPKuwKHZKQBv44WhDw0wN6BAgBEEg&biw=1920&bih=933}
	
	
	
	%%%%%%%%%%%%%%%%%%%%%%%%%%%%%%%%%%%%%%%%%%%%%%%%%%%%%%%%%%%%%%%%%
	
	vlucht 1951
	\bibitem{ID} ... \LaTeX:\\ \url{https://nl.wikipedia.org/wiki/Turkish_Airlines-vlucht_1951}
	technisch rapport
	\bibitem{ID} ... \LaTeX:\\ \url{file:///C:/Users/gally/Downloads/rapport_ta_nl_aangepast.pdf}
	beschrijving
	terugblik met overlevenden
	tijdlijn
	\bibitem{zuilen23022019Tijdlijnpoldercrash} ... \LaTeX:\\ \url{https://www.noordhollandsdagblad.nl/cnt/dmf20190221_65390940}
	artikel
	terugblik met overlevenden
	advies raad voor de veiligheid
	de overlevende, de oorzaak, regeling, herdenking, smartengeld
	verhaal van een overlevende
	herdenking
	herdenking
	bemanning deed niets met foutmelding
	parlementaire besluitenlijst
	kamervragen over de onafhankelijkheid van de raad voor veiligheid
	verhaal van een overlevende
	beschrijvend artikel van letsel en gewonden
	\bibitem{ntvg20012010letsel} ... \LaTeX:\\ \url{https://www.ntvg.nl/artikelen/vliegtuigongeval-schiphol-25-02-2009-letsels-en-verdeling-van-gewonden}
	technische fout als oorzaak
	\bibitem{wikinews04032009techfoutailines1951} ... \LaTeX:\\ \url{https://nl.wikinews.org/wiki/Technische_fout_oorzaak_vliegtuigcrash_Turkish_Airlines-vlucht_1951}
	gesprek met pieter van vollenhove voorzitter van de onderzoeksraad voor veiligheid
	onderzoeksraad voor veiligheid is onderdruk gezet
	\bibitem{luchtvaartnieuws21012020boeing737conclusies} ... \LaTeX:\\ \url{https://www.luchtvaartnieuws.nl/nieuws/categorie/72/algemeen/conclusies-crash-tk1951-na-amerikaanse-druk-afgezwakt}
	niuwesartikel
	feitenverloop
	\bibitem{adformatie280220209communicatiegebreken} ... \LaTeX:\\ \url{https://www.adformatie.nl/contentmarketing/communicatie-na-vliegramp-vertoonde-gebreken}
	zwarte doos
	\bibitem{spinnael25022009onderzoekpolderbaancrash} ... \LaTeX:\\ \url{https://flightlevel.be/244/onderzoek-polderbaan-crash-turkish-airlines-1951/}
	\bibitem{crashTurkishAirlines} ... \LaTeX:\\ \url{http://wikimapia.org/11633002/nl/Crash-Turkish-Airlines-vlucht-1951}
	\bibitem{flightradar24} ... \LaTeX:\\ \url{https://www.flightradar24.com/data/flights/tk1951}
	\bibitem{flightstatstracker} ... \LaTeX:\\ \url{https://www.flightstats.com/v2/flight-tracker/TK/1951}
	
	
	
	
	
	
	
	
	
	
	%%%%%%%%%%%%%%%%%%%%%%%%%%%%%%%%%%%%%%%%%%%%%%%%%%%%%%%%%%%%%%%%%
	de mali missie
	
	
	\bibitem{bnnvara13062018malirapport} ... \LaTeX:\\ \url{https://joop.bnnvara.nl/nieuws/rapport-haalbaarheid-en-houdbaarheid-van-mali-missie-twijfelachtig}
	\bibitem{eucal11012021malimissieverlengd} ... \LaTeX:\\ \url{https://www.consilium.europa.eu/nl/press/press-releases/2021/01/11/eucap-sahel-mali-mission-extended-until-31-january-2023-and-mandate-adjusted/}
	\bibitem{nos21052014zorgenmalimissie} ... \LaTeX:\\ \url{https://nos.nl/artikel/650637-kamer-bezorgd-over-mali-missie}
	\bibitem{meijnders} ... \LaTeX:\\ \url{https://www.bnr.nl/nieuws/10015679/koenders-positief-tegenover-verlening-mali-missie}
	\bibitem{bnrwebredactie} ... \LaTeX:\\ \url{https://www.bnr.nl/nieuws/politiek/10345553/kabinet-wil-mali-missie-stoppen-verrassend-besluit}
	\bibitem{keultjes01062016malimissiecoalitie} ... \LaTeX:\\ \url{https://www.ad.nl/nieuws/clash-om-mali-missie-dreigt-binnen-coalitie~a4151d4f/}
	\bibitem{veenhof18012019} ... \LaTeX:\\ \url{https://www.nd.nl/cultuur/boeken/536861/boek-kijkje-bij-de-mali-missie}
	\bibitem{ID} ... \LaTeX:\\ \url{https://www.youtube.com/watch?v=jmZ6uSbpCvg}
	\bibitem{isitman06012016militair} ... \LaTeX:\\ \url{https://www.ewmagazine.nl/nederland/achtergrond/2016/07/twee-nederlanse-militairen-dood-bij-oefening-mali-missie-325226/}
	\bibitem{nporadio11072016filmdemissie} ... \LaTeX:\\ \url{https://www.nporadio1.nl/nieuws/cultuur-media/9e3b076e-5401-4630-bf39-f925213c5b6b/onverwachte-openhartigheid-over-missie-in-mali}
	\bibitem{parlementairmonitor15122013mortierongeluk} ... \LaTeX:\\ \url{https://www.parlementairemonitor.nl/9353000/1/j9vvij5epmj1ey0/vjfm5p0nujzw?ctx=vj2mc67lofnr}
	
	
	sollicitatie
	de bureaucratie
	aankomst
	interview van de burgerbevolking
	steun van de bevolking minuut 15:00
	de organisatie minuut 23:00
	De militaire briefing minuut 34:00
	prioriteit minuut 39:00
	briefing minuut 40:00
	de communicatie met ministerie over inlichten minuut 44:00
	\bibitem{ID} ... \LaTeX:\\ \url{https://www.2doc.nl/documentaires/series/2doc/2016/juli/de-missie.html}
	
	%%%%%%%%%%%%%%%%%%%%%%%%%%%%%%%%%%%%%%%%%%%%%%%%%%%%%%%%%%%%%%%%%
	militair overleden door schietoefening in ossendrecht
	
	
	\bibitem{nos22032016ossendrecht} ... \LaTeX:\\ \url{https://amp.nos.nl/artikel/2094524-militair-omgekomen-bij-schietoefening-ossendrecht.html}
	
	\bibitem{ovv04042016lessenongevalossendrecht} ... \LaTeX:\\ \url{https://www.onderzoeksraad.nl/nl/page/4293/lessen-uit-schietongeval-ossendrecht}
	
	\bibitem{quekelboere10052017doodossendrecht} ... \LaTeX:\\ \url{https://www.bndestem.nl/bergen-op-zoom/dood-van-militair-sander-klap-35-in-ossendrecht-was-ongeluk-militairen-vrijuit-hij-probeerde-zijn-leven-te-redden~afe4c7a0/}
	
	
	Wat is de rol van defensie?
	Wat is er gedaan om de veligheid van de medewerkers te waarborgen?
	Waarom zijn deze regels niet nageleefd?
	Wat zijn de gevolgen?
	Zijn de acties die naderhand zijn ondernomen wel redelijk naar de slachtoffers, het nationale veiligheisbeeld en de medewerkers?
	
	
	
	
	
	
	
	%%%%%%%%%%%%%%%%%%%%%%%%%%%%%%%%%%%%%%%%%%%%%%%%%%%%%%%%%%%%%%%%%
	schipholbrand
	
	Wat is er gebeurd?
	\bibitem{wikiSchipholbrand} ... \LaTeX:\\ \url{https://nl.wikipedia.org/wiki/Schipholbrand}
	artikel
	\bibitem{schipholbrand27102005video} ... \LaTeX:\\ \url{https://www.youtube.com/watch?v=1i-hfEzxFfk}
	psychologische gevolgen
	rapport
	\bibitem{onderzoeksraad2610schipholoost} ... \LaTeX:\\ \url{https://www.onderzoeksraad.nl/nl/page/392/brand-cellencomplex-schiphol-oost-nacht-van-26-op-27-oktober}
	artikel met video
	herdenking
	impact op de persoon
	herdenking
	\bibitem{schipholbrandvideoargos} ... \LaTeX:\\ \url{https://www.vpro.nl/argos/speel~POMS_VPRO_461907~schadevergoeding-voor-ex-verdachte-schipholbrand~.html}
	chronologie
	\bibitem{nunl30052023feitenoverzicht} ... \LaTeX:\\ \url{https://www.nu.nl/binnenland/3355935/feitenoverzicht-schipholbrand-en-rechtszaken.html}
	tijdlijn
	\bibitem{ID} ... \LaTeX:\\ \url{https://www.singeluitgeverijen.nl/isbn/de-schipholbrand/}
	vervolgens van ministers
	beeldanalyse en reconstructie
	\bibitem{ID} ... \LaTeX:\\ \url{https://eenvandaag.avrotros.nl/item/schipholbrand-niet-ontstaan-in-cel-11/}
	herdenking
	korte samenvatting
	rapport
	artikel
	verwijzing naar het rapport vanuit de politieke oppositie
	beeld vanuit de gevangenisbewaarder
	nationaliteit slachtoffers schipholbrand
	verblijfsvergunning voor de slachtoffers
	gen schadevergoeding voor de verdachte
	verdachte voor de rechter
	geen schadevergoeding voor verdachte
	artikel wat ging er mis bji de schipholbrand
	brand veroorzaakt door een peuk
	smaadschrift
	bewakers worden niet vervolgd
	proces schipholbrand moet over en de brandveilgheid moet worden verbeterd
	de rol van het parlement in de evaluatie
	\bibitem{parlementairemonitorschipholbrand} ... \LaTeX:\\ \url{https://www.parlementairemonitor.nl/9353000/1/j9vvij5epmj1ey0/vi3aof7awcxg}
	onderzoeksmemo
	herdenking
	\bibitem{ID} ... \LaTeX:\\ \url{https://archief.ntr.nl/nova/page/detail/uitzendingen/3847/Den%20Haag%20Vandaag_%20herdenking%20Schipholbrand.html}
	herdenking
	invloed van de ramp op samenleving
	\bibitem{videonpoNOVA13112008} ... \LaTeX:\\ \url{https://www.npostart.nl/heropen-onderzoek-schipholbrand/13-11-2008/POMS_NTR_103332}
	opmerkelijk rapport gestolen in de nasleep
	\bibitem{schipperSchipholbrand} ... \LaTeX:\\ \url{https://www.nd.nl/nieuws/nederland/600395/schipholbrand-blijft-schrijnen}
	\bibitem{ID} ... \LaTeX:\\ \url{https://www.ed.nl/economie/om-geen-schadevergoeding-voor-verdachte-schipholbrand~a6c7c51d/63042600/?referrer=https%3A%2F%2Fwww.google.com%2F}
	\bibitem{ID} ... \LaTeX:\\ \url{https://www.groene.nl/artikel/schipholbrand-vereist-debat}
	\bibitem{rizoomes01052014schipholbrand} ... \LaTeX:\\ \url{https://www.rizoomes.nl/brandweer/brand-cellencomplex-schiphol/}
	
	
	
	publicaties
	\bibitem{heuvelkroesschipholbrandcamerabeelden} ... \LaTeX:\\ \url{http://www.msnp.nl/downloads/Onderzoeksmemo%20beeldanalyse%20Schipholbrand%20prot.pdf}
	\bibitem{ID} ... \LaTeX:\\ \url{http://www.dakweb.nl/roofs/2006-10/RH10-P30-31.pdf}
	\bibitem{ID} ... \LaTeX:\\ \url{https://www.delta.tudelft.nl/article/dood-door-zuinigheid}
	\bibitem{ID} ... \LaTeX:\\ \url{https://www.onderzoeksraad.nl/nl/page/392/brand-cellencomplex-schiphol-oost-nacht-van-26-op-27-oktober}
	Wat waren de regels destijds?
	Waren de autoriteiten in staat om op tijd in te grijpen of om erger te voorkomen?
	Wat is er gedaan om de veiligheid van illegalen en gevangenissbewaarders te verbeteren
	
	
	
	%%%%%%%%%%%%%%%%%%%%%%%%%%%%%%%%%%%%%%%%%%%%%%%%%%%%%%%%%%%%%%%%%
	vuurwerkramp enschede
	\bibitem{ID} ... \LaTeX:\\ \url{https://www.youtube.com/watch?v=OMkIsj8FsHw}
	\bibitem{ID} ... \LaTeX:\\ \url{https://depot03.archiefweb.eu/archives/archiefweb/20210703085353/http://www.vuurwerkramp.enschede.nl/publicaties/00005/#.YOAlp-gzaUk}
	
	Wat waren de afspraken omtrent vuurwerkopslag?
	Waarom werden de voorschriften neit nageleefd?
	
	
	%%%%%%%%%%%%%%%%%%%%%%%%%%%%%%%%%%%%%%%%%%%%%%%%%%%%%%%%%%%%%%%%%
	
	
	explosie in beirut
	\bibitem{landryalameddine12112020beiruthelathsystem} ... \LaTeX:\\ \url{https://bmchealthservres.biomedcentral.com/articles/10.1186/s12913-020-05906-y}
	\bibitem{ID} ... \LaTeX:\\ \url{https://news.sky.com/story/beirut-blast-cctv-captures-moment-huge-explosion-devastated-hospital-12047452}
	\bibitem{ID} ... \LaTeX:\\ \url{https://www.unodc.org/unodc/en/frontpage/2020/September/unodc-assists-lebanon-in-reestablishing-container-shipments-in-the-aftermath-of-the-port-of-beirut-explosion.html}
	\bibitem{ID} ... \LaTeX:\\ \url{https://reliefweb.int/sites/reliefweb.int/files/resources/LEB201-Lebanon-Emergency-Response.pdf}
	\bibitem{yadav07082020handlingexplosivesBeirut} ... \LaTeX:\\ \url{https://www.downtoearth.org.in/news/governance/beirut-blast-lessons-time-for-india-to-strengthen-handling-of-explosives-chemicals-72707}
	\bibitem{graham21082020rootsImpactBeirutBlast} ... \LaTeX:\\ \url{https://www.justsecurity.org/72122/the-cost-of-resilience-the-roots-and-impacts-of-the-beirut-blast/}
	\bibitem{ID} ... \LaTeX:\\ \url{https://www.fire-magazine.com/the-port-of-beirut-explosion-a-timely-reminder}
	\bibitem{neusaeter07082020beirutexplosioneval} ... \LaTeX:\\ \url{https://www.ctvnews.ca/sci-tech/mapping-the-beirut-explosion-what-the-impact-would-look-like-in-canadian-cities-1.5053932}
	
	
	secyrity:
	\bibitem{ID} ... \LaTeX:\\ \url{https://permanent.fdlp.gov/gpo45474/AN_advisory.pdf}
	
	
	secyrity:
	\bibitem{ID} ... \LaTeX:\\ \url{https://permanent.fdlp.gov/gpo45474/AN_advisory.pdf}
	
	
	
	
	
	
	
	
	%%%%%%%%%%%%%%%%%%%%%%%%%%%%%%%%%%%%%%%%%%%%%%%%%%%%%%%%%%%%%%%%%
	bijlmerramp
	
	
	
	
	%%%%%%%%%%%%%%%%%%%%%%%%%%%%%%%%%%%%%%%%%%%%%%%%%%%%%%%%%%%%%%%%%
	slmramp
	Wat is er gebeurd?
	\bibitem{ID} ... \LaTeX:\\ \url{https://www.srnieuws.com/suriname/290721/slm-ramp-herdacht/}
	\bibitem{ID} ... \LaTeX:\\ \url{https://werkgroepcaraibischeletteren.nl/documentaire-waarom-nou-jij-over-de-slm-ramp-in-89/}
	\bibitem{ID} ... \LaTeX:\\ \url{https://www.vpro.nl/speel~WO_NTR_15390142~andere-tijden-17-apr-2019-3-09-min-fouten-en-misstanden-leiden-tot-de-slm-ramp~.html}
	\bibitem{ID} ... \LaTeX:\\ \url{https://www.canonvannederland.nl/nl/kalender/06/1989-06-07}
	\bibitem{ID} ... \LaTeX:\\ \url{https://vijfeeuwenmigratie.nl/archief-herdenkingen-slm-ramp}
	\bibitem{ID} ... \LaTeX:\\ \url{https://www.hulpverleningsforum.nl/index.php?topic=84702.0}
	\bibitem{ID} ... \LaTeX:\\ \url{https://www.nporadio1.nl/fragmenten/focus/f792e720-bd85-4c18-8a71-b334d9d5de7e/2019-04-17-slm-ramp-een-paar-cowboys-hebben-achter-de-stuurknuppel-gezeten}
	\bibitem{ID} ... \LaTeX:\\ \url{https://www.waterkant.net/suriname/2017/06/07/herdenking-slm-ramp-28-jaar-geleden-suriname/}
	\bibitem{espnSLMterugblik} ... \LaTeX:\\ \url{https://www.espn.nl/video/clip?id=8744942}
	\bibitem{ID} ... \LaTeX:\\ \url{http://www.themediabrothers.nl/tag/slm-ramp/}
	\bibitem{ID} ... \LaTeX:\\ \url{https://www.rijnmond.nl/nieuws/182546/30-jaar-na-de-SLM-ramp-Ik-mis-mijn-broer-nog-elke-dag}
	\bibitem{dennisRosier01052020} ... \LaTeX:\\ \url{https://www.voetbalkrant.com/nieuws/2020-05-01/het-vergeten-verhaal-van-de-slm-ramp}
	\bibitem{hassing07062020slmramp} ... \LaTeX:\\ \url{https://www.bd.nl/sport/de-slm-ramp-en-het-hartverscheurende-verhaal-van-jerry-en-winnie-haatrecht~ae4ce105/?referrer=https%3A%2F%2Fwww.google.com%2F}
	\bibitem{amsterdamArchiefSLM} ... \LaTeX:\\ \url{https://www.amsterdam.nl/stadsarchief/nieuws/slm-ramp/}
	\bibitem{rtvOost06062019nabestaande} ... \LaTeX:\\ \url{https://www.rtvoost.nl/nieuws/313496/Nabestaande-SLM-ramp-Heb-ik-wel-mijn-broer-en-moeder-begraven}
	\bibitem{breda07062021AndroSnel} ... \LaTeX:\\ \url{https://www.bredavandaag.nl/nieuws/algemeen/337919/nac-herdenkt-andro-knel-slm-ramp-precies-32-jaar-geleden}
	\bibitem{andereTijdenSLMCrash} ... \LaTeX:\\ \url{https://www.anderetijden.nl/aflevering/792/Een-aangekondigde-vliegramp}
	\bibitem{aviationSLMCrashAccidentInvestigation} ... \LaTeX:\\ \url{https://nl.wikipedia.org/wiki/SLM-ramp}
	database
	\bibitem{aviationReport} ... \LaTeX:\\ \url{https://aviation-safety.net/database/record.php?id=19890607-2}
	rapport
	\bibitem{aviationSLMCrashAccidentInvestigation} ... \LaTeX:\\ \url{https://reports.aviation-safety.net/1989/19890607-2_DC86_N1809E.pdf}
	\bibitem{mcDonnelDouglasCommissionReportSLMCrash} ... \LaTeX:\\ \url{https://aviation-safety.net/investigation/cvr/transcripts/cvr_py764.php}
	\bibitem{wikiSRFlight764} ... \LaTeX:\\ \url{https://en.wikipedia.org/wiki/Surinam_Airways_Flight_764}
	\bibitem{ID} ... \LaTeX:\\ \url{https://web.archive.org/web/20050113010822/https://www.ntsb.gov/ntsb/brief.asp?ev_id=34510&key=0}
	\bibitem{nos07062019SLMTerugblik} ... \LaTeX:\\ \url{https://nos.nl/artikel/2287986-slm-vliegramp-van-precies-30-jaar-geleden-trof-ook-nederlands-voetbal}
	\bibitem{dagvantoenSLMCrash} ... \LaTeX:\\ \url{https://www.dagvantoen.nl/vliegtuigcrash-slm-bij-zanderij-meer-dan-170-doden/}
	\bibitem{waterkantNesty07061989} ... \LaTeX:\\ \url{https://www.waterkant.net/suriname/2006/06/07/vliegramp-suriname-op-7-juni-1989-2/}
	uitgebreid engels artikel
	\bibitem{eduNandlalSRCrash} ... \LaTeX:\\ \url{http://www.edufd.nl/planecrash/}
	ntsb investigtion
	\bibitem{oldjetsSRAirways} ... \LaTeX:\\ \url{http://www.oldjets.net/slm-dc-8-crash.html}
	uitgebreid engels artikel
	\bibitem{cloudberg02012021srflight764} ... \LaTeX:\\ \url{https://admiralcloudberg.medium.com/contract-to-kill-the-crash-of-surinam-airways-flight-764-828979c7efe2}
	persbericht
	\bibitem{apnews07061989srplanecrash} ... \LaTeX:\\ \url{https://apnews.com/article/5b240d758ee4c5422381cc7cdc98566b}
	Wat is de rol van de autoriteiten?
	Welke andere betrokkeen? Enw at is hun verantwoordelijkheid
	Hadden de negatieve gevolgen voorkomen kunnen worden?
	Hoe werd er over veiligheid gedacht?
	
	
	
	
	%%%%%%%%%%%%%%%%%%%%%%%%%%%%%%%%%%%%%%%%%%%%%%%%%%%%%%%%%%%%%%%%%
	Tsjernobyl
	\bibitem{ID} ... \LaTeX:\\ \url{https://www.youtube.com/watch?v=Xw3SFOfbR84}
	\bibitem{wikiTjernobyl} ... \LaTeX:\\ \url{https://nl.wikipedia.org/wiki/Kernramp_van_Tsjernobyl}
	\bibitem{rivmTjernobyl} ... \LaTeX:\\ \url{https://www.rivm.nl/straling-en-radioactiviteit/stralingsincidenten-en-kernongevallen/tsjernobyl}
	\bibitem{andereTijdenTjernobyl} ... \LaTeX:\\ \url{https://www.anderetijden.nl/aflevering/599/Tsjernobyl-als-Nederlandse-ramp}
	wat er is gebeurd en hoe het leven verdergaat
	\bibitem{kingskey19042022tjernobyl} ... \LaTeX:\\ \url{https://www.nationalgeographic.nl/het-leven-in-tsjernobyl-gaat-door}
	pernsioenfondsen en de tjernobyl ramp
	In 2021 worden mensen nog steeds blootgesteld blijkt ut een gezamelijk onderzoek van greenpeace en oekraiense wetenschappers
	stijging van de nucliaire activiteit gemeten in tjernobyl
	Het toerisme  aspect
	De chronologie
	\bibitem{erikbork26042023reactor4} ... \LaTeX:\\ \url{https://historianet.nl/maatschappij/rampen/tsjernobyl-atoomhel-bij-reactor-4}
	\bibitem{nosTjernobyl30jaarlater} ... \LaTeX:\\ \url{https://nos.nl/artikel/2101523-de-spookstad-van-tsjernobyl-30-jaar-later}
	Dieren in de omgeving van tjernobyl
	De chronologie
	Echtreme droogte zorgd voor gevaar
	\bibitem{knmi04052021tjernobylbosbrand} ... \LaTeX:\\ \url{https://www.knmi.nl/over-het-knmi/nieuws/35-jaar-na-tsjernobyl-liggen-branden-op-de-loer}
	\bibitem{dodonovaKVIRisicoTjernobyl} ... \LaTeX:\\ \url{https://www.kivi.nl/afdelingen/risicobeheer-en-techniek/columns/kernramp-tsjernobyl-het-dilemma-van-scherbitsky}
	Joernalistiek, entertainment en de waarheid
	\bibitem{dumarey04062020verhaalTjernobylWaarheid} ... \LaTeX:\\ \url{https://www.vrt.be/vrtnws/nl/2020/04/06/in-de-ban-van-tsjernobyl-vooruitblik/}
	Een onderzoek
	
	Huidige gevolgen van de explosie van toen
	\bibitem{sparkesNewScientistTjernoby;} ... \LaTeX:\\ \url{https://www.newscientist.nl/nieuws/steeds-meer-kernreacties-in-ontoegankelijke-ruimte-in-tsjernobyl/}
	De ramp, hoe de mensen ermee omgingen en hoe er nu geleef wordt
	
	evaluatieonderzoek en amatregeen
	\bibitem{kernenergiened26041986chronologiemaatregelen} ... \LaTeX:\\ \url{https://www.kernenergieinnederland.nl/node/308}
	\bibitem{mapszoneReactor} ... \LaTeX:\\ \url{https://www.google.com/maps/d/u/0/viewer?ie=UTF8&hl=nl&t=h&msa=0&ll=51.388923%2C30.099792&spn=0.685583%2C1.645203&z=9&source=embed&mid=1MLcOcMK_WrIJYMuTf0VVuYnMqQI}
	Invloed van de mens op de omgeving
	\bibitem{ID} ... \LaTeX:\\ \url{https://www.animalstoday.nl/mens-schadelijker-natuur-tsjernobyl/}
	Heroplevende splijtingsreacties
	docu van schooltv
	Radioactiviteit bereikt nederland
	documentaire en maatregelen
	\bibitem{kernhistoriek15062021tjernobyl} ... \LaTeX:\\ \url{https://historiek.net/kernramp-van-tsjernobyl-1986/8769/}
	Het verhaal van een overledende
	Toerisme
	toerisme
	toerisme
	Dieren in de omgevong
	Toevluchtsoord voor vluchtelingen van de oorlog met russische seperatisten
	Ouderen die terugkeerden naar hun woonplaats na de gedwongen verhuizing door de autoriteiten
	De straling neemt weer toe
	Lessen geleerd van tjernobyl
	\bibitem{nucleairforumFeitenTjernobyl} ... \LaTeX:\\ \url{https://www.nucleairforum.be/thema/veiligheid-als-prioriteit/tsjernobyl-de-feiten}
	Toerisme
	Bosbrand in tjernobyl
	invloed van de ramp op belgie
	\bibitem{kernongevalTjernobylFancGov} ... \LaTeX:\\ \url{https://fanc.fgov.be/nl/noodsituaties/zware-ongevallen-het-buitenland/1986-kernongeval-tsjernobyl}
	Boek recensie
	Fotos en berekeningen
	ontmanteling en toerisme
	Belangrijke lessen en overeenkomsten
	De journalistieke waarheid van de koude oorlog
	De lessen van
	\bibitem{arendswolters062019lessenTjernobyl} ... \LaTeX:\\ \url{https://magazines.autoriteitnvs.nl/nieuwsbrief-anvs/2019/02/de-lessen-van-tsjernobyl}
	Een toristenattractie maken van tjernobyl
	De radioactieve straling toen en nu
	de 30km zone door de ogen van toeristen
	artikel
	stedentrip
	rapport
	\bibitem{damveld08052020tjernobyl} ... \LaTeX:\\ \url{https://wisenederland.nl/wp-content/uploads/2020/06/TSJERNOBYL.pdf}
	slapend monster
	docu
	krantenartikel
	hbo serie
	docuserie
	de  nieuwe sacrofaag
	hulp aan slachtoffers
	slapende reactor
	krantenartikel
	\bibitem{deVriestjernobylHolland} ... \LaTeX:\\ \url{https://onh.nl/verhaal/besmette-melk-en-radioactieve-spinazie-tsjernobyl-in-holland}
	hbo serie
	internationale gevolgen
	toerisme
	nieuwe koepel
	media communicatie
	docu
	dieren
	\bibitem{ID} ... \LaTeX:\\ \url{https://www.amboanthos.nl/boek/nacht-in-tsjernobyl/}
	koepel
	koepel
	\bibitem{ing3enieur29042015antistralingskoepel} ... \LaTeX:\\ \url{https://www.deingenieur.nl/artikel/nieuwe-antistralingskoepel-tsjernobyl-bijna-af}
	toerisme
	toeristisch reiperspectief
	toerisme
	niwe koepel
	overschakelen naar duurzaamheid
	docu
	tjernobyl wekt nu duurazme energie
	toerisme
	overeenkomsten tjernobyl en fukushima
	drank en sla uit tjernobyl
	geen efficiente opslag is mogelijk
	
	wetenschappelijke artikelen
	
	zaterdag 26 april 1986. Er vind routineonderhoud plaats bij reactor 4, De controle wordt uitegevoerd door de dagploeg. Vnwege een test wordt jhet koelsysteem uitgeschakeld. Door omstandigheden wordt de test uitgesteld en wordt de verantwoordelijkheid overgedragen aan de avondploeg.
	De operator maakt bedieningsfouten waardoot de reactor bijna stil komt te liggen. En vervolgens probeert hij de reactor weer op gang te brengen. ondanks de snelle temperatuurstijging wordt het experiment doorgezet. Dan wordt ook het veiligheidssysteem stilgelgd. Terwijl het koelwater langzaam opwarmt, sluit hij de klep waarlangs de stoom naar de generator stroomt.
	
	De temperatuur van de reactorstaven neemt daarna snel toe. Terwijl er een oncontroleerbare kettingreactie op gang komt, laat het personeel in paniek de regelstaven zakken om de warmteontwikkeling af te remmen. Het is dan echter al te laat. Door een ontwerpfout loopt het vermogen razendsnel op tot 33.000 megawatt, ruim tien keer hoger dan normaal.
	
	In een oogwenk verandert al het koelwater in stoom. De ontploffing die daarop volgt, blaast het 2000 ton zware deksel van de reactor af.}

In de ravage vat het gloeiend hete grafiet in de reactor spontaan vlam. De uitslaande brand en een tweede explosie voeren een radioactieve rookwolk tot 8 kilometer hoogte. 
In een poging het vuur in reactor 4 te doven, storten helikopters vanuit de lucht zand, lood en boorzuur in de reactorkern. Het mag echter niet baten.

Intussen is de nucleaire brandstof zo heet geworden dat die door de bodem van het reactorvat dreigt te smelten. Als dat gebeurt, kan het bluswater onder het vat in één klap verdampen en dreigt een derde explosie die een groot deel van Europa onbewoonbaar zal maken. Om dit te voorkomen moet het water hoe dan ook worden weggepompt.

Drie brandweermannen wagen zich daarvoor in de ruimte onder de reactor, blootgesteld aan 300 sievert per uur, 300.000 keer de dosis die een Nederlander jaarlijks maximaal mag oplopen. Ze slagen daarin, maar twee van hen overlijden enkele dagen later aan acute stralingsziekte.

Hoewel geigertellers de dag na de ramp onrustbarende waarden aangeven, slaat het plaatselijk bestuur geen alarm. De bevolking is het niet gewend om vragen te stellen.

De volgende dag blijkt er wel degelijk iets ernstigs aan de hand te zijn. In een lange rij bussen worden de 135.000 inwoners op 27 april uit het besmette gebied geëvacueerd, om er nooit meer terug te keren.

De ramp is dan nog steeds geen wereldnieuws. De Sovjetautoriteiten blijken er niet eens van op de hoogte te zijn – president Gorbatsjov klaagt later dat hij via Zweden aan zijn informatie moest komen.


\bibitem{verschuur14012013tjernobylreports} ... \LaTeX:\\ \url{http://essay.utwente.nl/63353/1/Verschuur,_W._-_s0123617_(verslag).pdf}
\bibitem{paperlessarchivesTjernobyl} ... \LaTeX:\\ \url{https://www.paperlessarchives.com/chernobyl_nuclear_accident_doc.html}
\bibitem{vargos082000tjernobylconcerns} ... \LaTeX:\\ \url{https://www.pnnl.gov/main/publications/external/technical_reports/pnnl-13294.pdf}
\bibitem{mauroNuclearRiskSociety} ... \LaTeX:\\ \url{http://www.geocities.ws/scannapuerci/demauroinnovation.pdf}
\bibitem{vienna06092005LookingBack} ... \LaTeX:\\ \url{https://www-pub.iaea.org/MTCD/publications/PDF/Pub1312_web.pdf}



%%%%%%%%%%%%%%%%%%%%%%%%%%%%%%%%%%%%%%%%%%%%%%%%%%%%%%%%%%%%%%%%%

MH17




%%%%%%%%%%%%%%%%%%%%%%%%%%%%%%%%%%%%%%%%%%%%%%%%%%%%%%%%%%%%%%%%%
oekraine powergrid
\bibitem{owens21032017ukrainemitigationstrategies} ... \LaTeX:\\ \url{https://na.eventscloud.com/file_uploads/aed4bc20e84d2839b83c18bcba7e2876_Owens1.pdf}
\bibitem{ID} ... \LaTeX:\\ \url{https://www.us-cert.gov/ics/alerts/IR-ALERT-H-16-056-01}

\bibitem{cerulus2019FrontlineRussiaAttack} ... \LaTeX:\\ \url{https://www.politico.eu/article/ukraine-cyber-war-frontline-russia-malware-attacks/}
\bibitem{ID} ... \LaTeX:\\ \url{https://en.wikipedia.org/wiki/December_2015_Ukraine_power_grid_cyberattack}
\bibitem{grammatikis2019AttackIEC6087505104} ... \LaTeX:\\ \url{https://www.researchgate.net/publication/333671061_Attacking_IEC-60870-5-104_SCADA_Systems}
\bibitem{ID} ... \LaTeX:\\ \url{https://ris.utwente.nl/ws/files/6028066/3-s2_0-B9780128015957000227.pdf}
\bibitem{hidajat2016ScadaSimulator} ... \LaTeX:\\ \url{https://www.diva-portal.org/smash/get/diva2:1046339/FULLTEXT01.pdf}
\bibitem{ID} ... \LaTeX:\\ \url{https://www.semanticscholar.org/paper/Cybersecurity-analysis-of-a-SCADA-system-under-and-Rocha/dfa7c12551ebe7b24da8d806e87e946051a57cb9}
\bibitem{ID} ... \LaTeX:\\ \url{https://tutcris.tut.fi/portal/files/16294332/jafary_1534.pdf}
\bibitem{ID} ... \LaTeX:\\ \url{http://blog.nettedautomation.com/2017/}
\bibitem{uscert20072021crashmalware} ... \LaTeX:\\ \url{https://www.us-cert.gov/ncas/alerts/TA17-163A}
\bibitem{zetter12062017malwareanalysis} ... \LaTeX:\\ \url{https://www.vice.com/en_us/article/zmeyg8/ukraine-power-grid-malware-crashoverride-industroyer}
\bibitem{icsRussianHackingCyberWeapon} ... \LaTeX:\\ \url{http://blog.wallix.com/ics-security-russian-hacking}
\bibitem{usgovC2M2} ... \LaTeX:\\ \url{https://www.energy.gov/ceser/activities/cybersecurity-critical-energy-infrastructure/energy-sector-cybersecurity-0}

bediening werking schutsluizen pdf
\bibitem{varendoejesamenVeiligSluisvaren} ... \LaTeX:\\ \url{https://www.varendoejesamen.nl/storage/app/media/downloads/vlot-en-veilig-door-brug-en-sluis-.pdf}
\bibitem{vlaamsewaterwegen012014} ... \LaTeX:\\ \url{http://www.scarphout.be/assets/bedieningstijden2014.pdf}
\bibitem{bardetsluizenAmsterdam} ... \LaTeX:\\ \url{https://www.theobakker.net/pdf/sluizen.pdf}
\bibitem{dvsbedieningsluizenenbruggen} ... \LaTeX:\\ \url{http://www.watersportalmanak.nl/files/File/Brugbediengstijden_watersport.pdf}
\bibitem{crowstappenplanmachinerichtlijnen} ... \LaTeX:\\ \url{https://www.crow.nl/downloads/pdf/verkeer-en-vervoer/verkeersmanagement/verkeersregelinstallaties/stappenplan-machinerichtlijnen_web.aspx}
\bibitem{rijksoverheidrwsonderzoeksrapporten} ... \LaTeX:\\ \url{https://puc.overheid.nl/rijkswaterstaat/doc/PUC_95170_31/}
\bibitem{bedieningstijdensluizenenbruggen} ... \LaTeX:\\ \url{http://wsv.wsvdegors.nl/wp-content/uploads/2017/05/Bedieningstijden_201701.pdf}
\bibitem{sluiscomplexkornwerderzand} ... \LaTeX:\\ \url{https://www.commissiemer.nl/projectdocumenten/00004717.pdf}
\bibitem{varengroningendrenthe} ... \LaTeX:\\ \url{https://tasmanroutes.nl/wp-content/uploads/docs/1900-bedieningstijden-groningen-drenthe.pdf}
\bibitem{wallemuldergetijhoogteverschillen} ... \LaTeX:\\ \url{http://www.vliz.be/docs/groterede/GR21_Zeesluis.pdf}
\bibitem{biemandssluizenLith} ... \LaTeX:\\ \url{https://www.bhic.nl/media/document/file/rien-biemans-sluis-en-stuw-bij-lith.pdf}
\bibitem{weilerburgers11062018zoutindringingschutsluizen} ... \LaTeX:\\ \url{https://www.nattekunstwerkenvandetoekomst.nl/upload/documents/tinymce/KpNK-2017-SKW-01c001-v1-Zoutindringing-door-schutsluizen-overzicht-projecten-en-aanzet-formulering-tbv-netwerkmodellen.pdf}
\bibitem{rwsrichtlijnvaarwegen2011} ... \LaTeX:\\ \url{https://www.arnhemspeil.nl/nap/dok/2011-12-00-rijkswaterstaat-richtlijnen-vaarwegen.pdf}
\bibitem{slotterwisscha02062016swimwaywaddenzee} ... \LaTeX:\\ \url{https://rijkewaddenzee.nl/wp-content/uploads/2016/08/Inventarisatie-toestand-vispasseerbaarheid-zoet-zout-overgangen-Waddenzee-2-6-2016-PRW-rapportage-Definitief.pdf}
\bibitem{visvriendelijksluisbeheer06012014} ... \LaTeX:\\ \url{http://www.nevepaling.nl/files/Image//nederlands/informatiecentrum/2014-definitieve-voorkeursvariantennotitie-visvriendelijk-sluisbeheer-afsluitdijk-en-houtribdijk//2014_definitieve_voorkeursvariantennotitie_visvriendelijk_sluisbeheer_afsluitdijk_en_houtribdijk.pdf}
\bibitem{ID} ... \LaTeX:\\ \url{https://www.ifv.nl/kennisplein/Documents/20120614-BwNL-Handboek-brandbeveiligingsinstallaties.pdf}
\bibitem{aubel112016sluiscomplexHeumen} ... \LaTeX:\\ \url{https://ienc-kennisportaal.nl/wp-content/uploads/2017/01/Objectbeschrijving-Heumen.pdf}
\bibitem{arends052005schutsluisstolwijkschevliet} ... \LaTeX:\\ \url{https://library.wur.nl/edepot/websites/stolwijkersluis/presentatie-data/data/pdf/TUDelft-bouwhistorisch-onderzoek.pdf}
\bibitem{multidomeinbediening} ... \LaTeX:\\ \url{https://www.icentrale.nl/wp-content/uploads/bsk-pdf-manager/2019/01/20170929_Project-2.02-Deliverable-Gehele-werkpakket-2.02.pdf}
\bibitem{historischesluizen} ... \LaTeX:\\ \url{https://www.stowa.nl/sites/default/files/assets/PUBLICATIES/Publicaties%202000-2010/Publicaties%202000-2004/STOWA%202004-XX%20boekenreeks%2020.pdf}
\bibitem{nmmag2017centralebediening} ... \LaTeX:\\ \url{https://www.nm-magazine.nl/pdf/NM_Magazine_2017-3.pdf}
\bibitem{knooppuntvaarwegenFryslanGroningenDrenthe} ... \LaTeX:\\ \url{https://www.varendoejesamen.nl/storage/app/media/knooppunten/knooppuntenboekje_03_Friesland_Groningen_Drenthe.pdf}
\bibitem{afsluitdijk052015} ... \LaTeX:\\ \url{https://deafsluitdijk.nl/wp-content/uploads/2014/05/Plan-project-MER-Afsluitdijk.pdf}
\bibitem{ID} ... \LaTeX:\\ \url{file:///C:/Users/gally/Downloads/vaarroutekaart_provincie_drenthe.pdf}
\bibitem{ID} ... \LaTeX:\\ \url{file:///C:/Users/gally/Downloads/6227_watermanagement_nl_dv.pdf}
\bibitem{ID} ... \LaTeX:\\ \url{file:///C:/Users/gally/Downloads/applsci-11-00092-v3.pdf}
\bibitem{ID} ... \LaTeX:\\ \url{file:///C:/Users/gally/Downloads/Bedieningstijden_sluizen_en_bruggen_2004.pdf.pdf}
\bibitem{ID} ... \LaTeX:\\ \url{file:///C:/Users/gally/Downloads/Bedieningstijden.pdf}
\bibitem{ID} ... \LaTeX:\\ \url{file:///C:/Users/gally/Downloads/bijlagerapport_c_-_analyse_geavanceerd_definitief_v1_0.pdf}
\bibitem{ID} ... \LaTeX:\\ \url{file:///C:/Users/gally/Downloads/BIT-advies+Bediening+op+Afstand,+sluizen+en+bruggen+in+Friesland.pdf}
\bibitem{ID} ... \LaTeX:\\ \url{file:///C:/Users/gally/Downloads/conceptverordeningnautischbeheerzuid-holland.pdf}
\bibitem{ID} ... \LaTeX:\\ \url{file:///C:/Users/gally/Downloads/De_Deltawerken_Cultuurhistorie_ontwerpgeschiedenis_web-A.pdf}
\bibitem{ID} ... \LaTeX:\\ \url{file:///C:/Users/gally/Downloads/duurzaamheid_bij_de_ontwikkeling_van_reevesluis.pdf}
\bibitem{ID} ... \LaTeX:\\ \url{file:///C:/Users/gally/Downloads/gebruikershandleiding-databank-vismigratie.pdf}
\bibitem{ID} ... \LaTeX:\\ \url{file:///C:/Users/gally/Downloads/onderzoek_vispasseerbaarheid_sluizen_zuid_holland_2016_definitief_16-5-20171.pdf}
\bibitem{ID} ... \LaTeX:\\ \url{file:///C:/Users/gally/Downloads/rapport-veiligheid-van-op-afstand-bediende-bruggen.pdf}
\bibitem{ID} ... \LaTeX:\\ \url{file:///C:/Users/gally/Downloads/richtlijnen-vaarwegen-2017_tcm21-127359%20(1).pdf}
\bibitem{ID} ... \LaTeX:\\ \url{file:///C:/Users/gally/Downloads/richtlijnen-vaarwegen-2017_tcm21-127359.pdf}
\bibitem{ID} ... \LaTeX:\\ \url{file:///C:/Users/gally/Downloads/TPE_2342144_20210511223747_PEU_52457169.pdf}


Wat hebben alle bovenstaande rampen/ongelukken gemeen? Veiligheid.
Bij de therac waren er diverse problemen: communicatie, doorontwikkeling, controle en toetsing
Was het makkelijk te onderzoeken? Waarom?
Bij de boeing 737 crashes was het probleem van controle en communicatie naar medewerkers
Was het makkelijk te onderzoeken? Waarom?

Uit de evaluatie van de china explosion 2015 tianjin komt naar voren dat communicatie, transparantie en veiligheid niet altijd prioriteit hadden bij de lokale autoriteiten
Was het makkelijk te onderzoeken? Waarom?

Bij de tesla autopilot crashes komen soms onvoldoende onderbouwde ontwerpkeuzes naar voren die niet goed zij  afgewogen tegenover het gedrag van de bestuurder
vlucht 1951
Was het makkelijk te onderzoeken? Waarom?

De ramp in Tsjernobyl toont aan hoe autoriteiten een ramp in de doofpot proberen te stoppen
Was het makkelijk te onderzoeken? Waarom?



Wat heb ik geleerd
Ik heb erg veel geleerd van het veilig opzetten van VPN’s. Een VPN opzettenhad ik namelijk nog nooit gedaan. Het opzetten van SSH en het aanmaken vanVM’s was al bekend. Ook had ik nog nooit met UDP sockets geprogrammeerd.Verder heb ik geleerd hoe ik in de praktijk een VM in een VLAN kan zetten enhoe VLAN’s netwerken van elkaar kunnen scheiden.Het leukste onderdeel van het project, was dat wonderbaarlijk mijn gekozenoplossing elegant werkte. UDP Servers en clients zijn gerealiseerd met minderdan enkele regels logisch scipt. Ik had aan genomen dat het werken met socketsin shell absoluut rampzalig zou uitpakken. Ik ben blij dat het opdracht zo vrijwas, zodat ik experimenteel kon zijn met mijn implementatie.




\bibitem{ID} ... \LaTeX:\\ \url{https://www.uni-saarland.de/fileadmin/user_upload/Professoren/FreyG/DS_KT_GF_INCOM_May_2012.pdf}
vanaf 2.1 tot en met 5

\bibitem{ID} ... \LaTeX:\\ \url{http://www.lasid.ufba.br/publicacoes/artigos/Integrating+UML+and+UPPAAL+for+Designing,+Specifying+and+Verifying+Component-Based+Real-Time+Systems.pdf}

hf7
Reachability: i.e. some condition an posssibly be satisfied
Safety: i.e. some condition will never occur
Liveness: i.e. some condition wille eventually become true [] eventually or leadsto
hf 8
Het systeem is deadlockvrij
De wachttijd is altijd gelijk aan de invaarttijd _2x de nivlleertijd en de invaartijd van de overkant

\bibitem{ID} ... \LaTeX:\\ \url{https://www.diva-portal.org/smash/get/diva2:495691/FULLTEXT01.pdf}

blz 6 tot en met 10
\bibitem{ID} ... \LaTeX:\\ \url{https://www.cister-labs.pt/docs/formal_verification_of_aadl_models_using_uppaal/1331/view.pdf}

hf 3 geeft een voorbeeld van een template met guard en acies
De volgende automata worden gebruikt met hun lokale variabelen

De volgende globale variabelen

Een lijst met relevante einschappen van een schutsluis:

\bibitem{ID} ... \LaTeX:\\ \url{https://iopscience.iop.org/article/10.1088/1742-6596/1821/1/012031/pdf}

hf 5
deadlock

\bibitem{ID} ... \LaTeX:\\ \url{http://www.es.mdh.se/pdf_publications/2934.pdf}

hf 3 tool support
Modelling in UML
Code generation
Domain Model
Behaviour model
State Hierarchy
Transitions
Trigger methods
Time events
Effects
Requirements
Environment model

hf 4
\bibitem{ID} ... \LaTeX:\\ \url{https://files.ifi.uzh.ch/stiller/CLOSER%202014/WEBIST/WEBIST/Internet%20Technology/Full%20Papers/WEBIST_2014_130_CR.pdf}


\bibitem{ID} ... \LaTeX:\\ \url{https://files.ifi.uzh.ch/stiller/CLOSER%202014/WEBIST/WEBIST/Internet%20Technology/Full%20Papers/WEBIST_2014_130_CR.pdf}

4.2 5 en 6
Het Sluisbeheeerder model wordt getoond in fuguur[]. Het model is een uitbreiding van een schutsluis met alle condities en effecten. De kleuren in de automation verwijizen naar de kleuren in de staat van de automata . De template begint met een initiele lokatie start. De sluisbeheerder initieert het proces door een aangekomen schip te registreren metbehulp van een sychronizate met het channel... over de edge richtng de lokatie "aanmelden." Dit symboliseert een opstartprocedure, ook wordt een functie enqueeu_aanmeldLijst() gebruikt om de juiste waarden te geven aan lokale en globale avariabelen. De lokatie aanmelden regisseert het opstellen van schepen boven of beneden van de sluiskolk. De template Schip synchronizeerd met de template Sluisbeheerder met het channel move_down[id] of move_up[id] en bereikt daarmee de volgende lokatie afhankelijk af de sluis boven of beneden is worden de schepen die in de opstellijst voorkomen, max 2, klaargemaakt voor invaren.. De templates Stoplicht en sluisdeur synchroniseren met de channels ... call_Deur en call_stoplicht.
Het Sluisbeheerder model gebruikt de variabelen clock x, wachttijd_beneden, wachttijd_boven als invariant tussen de lokaties. Om op de hoogete te zijn van de invaar-/uitvaart van de verschillende schepen worden lijsten bijgehouden: list_wachtrij_beneden, list_pos_invaren_beneden, list_schepenInSluis, list_wachtrij_boven en list_pos_invaren_boven.

Het model voltooit de volgende transitie op basis van de waarde van de boolean sluis_bove en sluis_beneden. en de lokale klok variabele x.
Vanaf de locatie invaarverbod_gecontroleerd  wordt gecontroleerd of er nog invarende schepen zijn die in de sluiskolk passen.
Op de lokatie sluiskolk gereed zijn er 1 of meer schepen in de sluis. Als er nog plek is in de sluiskolk n er is nog een schip klaar om in te varen dan wordt dit gecontroleerd, de functie enqueu() voegt het schip toe aan de queue van de sluiskolk. De functie deque() verwijdert de schip van de lijst met invarende schepen. De variabele sluis_boven of sluis_beneden is waar, bij de switch voor het sluiten van deuren en het aanroepe van het stoplicht nr gelang de positie van  de laate binnenvarende schip (boven of beneden). Hierna bereikt de automation sluiskolk_afgesloten.



De lokatie start_nivelleren kiest op basis van de variabelen sluis_boven en de variabelen sluis_beneden het nivellereingsprogramma.
Heet nivellereingsprogramma is Aof B. De keuze voor het programma wordt bepaald door de variabelen van het schip dat in de sluis zit.

De lokatie klaarmaken_voor_openen wordt bereikt als de   hoogte van de sluis  door het nivellereingsprogramma is bereikt.
De positie van de kluis is bepaald door de schepen in de sluis. Vanuit deze lokatie wordt gekeken off de stoplichten gereed moeten worden gemaakt en of de sluisdeuren open mogen.
Hierna volgt een transiie waarin de stoplichte op groen worden gezet en de sluisdeuren worden geopend voor de uitvaart van de schepen in de sluis.
Als alle schepen zijn uitgevaren die uit moeten varen, worden de stoplichten op groen gezet en de deuren gesloten.


De lokatie uitvaren_toegestaan heeft een verbinding(edge) met de lokatie sluis_afsluiten.
Er is een select statement, e:id_t gebruikt als onderdeel van het prototocol om alle uitvarende schepen uit de queue van de sluiskolk te halen, en wordt dan ook gebruikt door de synchronisatie met de channel leave om de schepen uit de sluiskolk te begeleiden. De edge hieraan gekoppeld bevat de functie deque() om de variabelen  van de sluiskolk te resetten.

Vanuit de positie van de sluis worden de schepen gesignaleerd op een invaarverbod en worden de deuren van de sluis gesloten.
De lokatie sluiskolk_afgesloten is bereikt.

Ship [guards, invariants, assignents, synchronizations, properties,aannames]
De template Schip begint bij de Init lokatie. De lokatie is verbonden met de lokatie aangekomen met een edge waarbij een synchronizatie wordt aangeroepen met de template sluisbeheerder. De clock wordt op nul gezet. De lokatie aangekomen is verbonden met de lokatie aangemeld. De edge bevat een synchronizatie waarmee de edge een synchronizatie uitvoert met de template Sluisbheheerder.
De volgende lokatie is  controleren. De edge waarmee de lokatie aangemeld in verbinding staat met de lokatie cnotroleren heeft een synchronisatie voor de template Sluisbeheerder. De lokatie controleren heeft ook een edge met de lokatie wachten. Een schip max maximaal 30 seconden wachten op de lokatie wachten voordat er een mogelijkheid is om opniew in aanmerking te komen voor een controle. Als een schip langer dan 30 tijdseenheden moet wachten de is er een mogelijkheid voor het schip te vertrekken. Hierbij eindigt het schip het invaarproces. Een schip kan dus na aanvaren maximaal 20 seconden wachten om toestemming te krijgen voor een positie invaren anders wordt deze verwezen naar een wachtrij.
Hierna volgdde lokate invarene. De lokatie invarene implieert dat een schip in een invaarproces is dat eindigt in de lokatie gestopt.
Hierop volgd de lokatie nivelleer_start. Hierop wordt een nivelleer_proces gestart. Daarbij is ee synchronisatie met de template Sluisbeheerder.
De lokatie nivelleer_stop is een lokate waarin het nivelleerproces al is gestopt. Van hieruit is er een edge met de lokatie klaar voor vertrek. De edge synchroniseert hiermee met de template Sluisbeheerder.
De lokatie klaar_voor_vertrek is verbonden met de lokatie Init. Met een guard x>=3 tijdseenheden mag een schip vertrekken.


Deur
De deur bevat de volgende lokaties: dicht, openend, open en sluitende.
Een deur sluit niet in een enkele actie. Het proces die een deur dooploopt zijn de processen openend en sluitende. De finale lokaties zijn open en dicht.

Nivelleermachine
De nivelleermachine begint bij de lokatie uit. Met een synchronisatie wordt een nivelleermachine aangezet. De automatie kiest een programma en werkt deze uit in de lokatie bezig. Als ht programma is afgerond volgt de lokatie klaar. Na elk nivelleerproces wordt de machine uitgezet

Stoplicht
Een stoplicht heeft twee lokaties: rood en groen.



%%%%%%%%%%%%%%%%%%%%%%%%%%%%%%%%%%%%%%%%%%%%%%%%%%%%%%%%%%%%%%%%%

Bijlage A performance
\bibitem{kumarUppaalDMAMACProtocol} ... \LaTeX:\\ \url{https://home.hvl.no/ansatte/aaks/articles/2015IKT617.pdf}

test specification
\bibitem{larsenRealtimeUppaalTesting} ... \LaTeX:\\ \url{https://d-nb.info/987511998/34}

sheet 24 tot 65
\bibitem{proenza102008UppaalModelChecker} ... \LaTeX:\\ \url{http://ppedreiras.av.it.pt/resources/empse0809/slides/TheUppaalModelChecker-Julian.pdf}


2.3.4.2
4.7

coffie apparaat

\bibitem{uppaalCoffeeMachine} ... \LaTeX:\\ \url{https://www.comp.nus.edu.sg/~cs5270/Notes/chapt6a.pdf}



what is a good software specification
\bibitem{fvaandrager2322010Goodmodel} ... \LaTeX:\\ \url{http://www.cs.ru.nl/~fvaan/PV/what_is_a_good_model.html#:~:text=A%20good%20model%20has%20a%20clearly%20specified%20purpose%20and%20(ideally,code%20generation%2C%20and%20test%20generation.}

\bibitem{onix01102022devopmodel} ... \LaTeX:\\ \url{https://onix-systems.com/blog/7-basic-software-development-models-which-one-to-choose}
\bibitem{sulemani04012021softwareprocesmodel} ... \LaTeX:\\ \url{https://www.educative.io/blog/software-process-model-types}

\bibitem{globalluxsoft18102017softdev} ... \LaTeX:\\ \url{https://medium.com/globalluxsoft/5-popular-software-development-models-with-their-pros-and-cons-12a486b569dc}
\bibitem{wiegers30052022SRS} ... \LaTeX:\\ \url{https://www.jamasoftware.com/blog/characteristics-of-excellent-requirements/}
\bibitem{muller06092020goodspecification} ... \LaTeX:\\ \url{https://www.gaudisite.nl/ValidationOfRequirementsSlides.pdf}
\bibitem{informit30062008reqmanagement} ... \LaTeX:\\ \url{https://www.informit.com/articles/article.aspx?p=1152528&seqNum=4}
\bibitem{altexsoft15092020writingSRS} ... \LaTeX:\\ \url{https://www.altexsoft.com/blog/software-requirements-specification/}
\bibitem{ID} ... \LaTeX:\\ \url{E:\Backup Mijn Documenten\Hogeschool vakken\TINLab advnced algorithms\tinlab_advanced_algoriths\achtergrondinfo research}
sheet 28 transitorische relaties vertalen van ctl naar ltl
\bibitem{ID} ... \LaTeX:\\ \url{file:///E:/Backup%20Mijn%20Documenten/Hogeschool%20vakken/TINLab%20advnced%20algorithms/tinlab_advanced_algoriths/achtergrondinfo%20research/buchi/lec16_Buchi+LTL.pdf}
\bibitem{ID} ... \LaTeX:\\ \url{file:///E:/Backup%20Mijn%20Documenten/Hogeschool%20vakken/TINLab%20advnced%20algorithms/tinlab_advanced_algoriths/achtergrondinfo%20research/buchi/lect4.pdf}
\bibitem{ID} ... \LaTeX:\\ \url{file:///E:/Backup%20Mijn%20Documenten/Hogeschool%20vakken/TINLab%20advnced%20algorithms/tinlab_advanced_algoriths/achtergrondinfo%20research/buchi/lecture8.pdf}
transitie relaties in LTL sheet 8
\bibitem{ID} ... \LaTeX:\\ \url{file:///E:/Backup%20Mijn%20Documenten/Hogeschool%20vakken/TINLab%20advnced%20algorithms/tinlab_advanced_algoriths/achtergrondinfo%20research/CS%20267%20Automated%20Verification/l2.pdf}
\bibitem{ID} ... \LaTeX:\\ \url{file:///E:/Backup%20Mijn%20Documenten/Hogeschool%20vakken/TINLab%20advnced%20algorithms/tinlab_advanced_algoriths/achtergrondinfo%20research/FORMAL%20METHODS/slide3.pdf}
\bibitem{ID} ... \LaTeX:\\ \url{file:///E:/Backup%20Mijn%20Documenten/Hogeschool%20vakken/TINLab%20advnced%20algorithms/tinlab_advanced_algoriths/achtergrondinfo%20research/FORMAL%20METHODS/slide4.pdf}
hf 4.2
\bibitem{ID} ... \LaTeX:\\ \url{file:///E:/Backup%20Mijn%20Documenten/Hogeschool%20vakken/TINLab%20advnced%20algorithms/tinlab_advanced_algoriths/achtergrondinfo%20research/properties%20ctl/Chapter-4-Formal-Methods-LTL-CTL-TRAFFIC-LIGHT-EXAMPLE-pages-18-24.pdf}


\bibitem{ID} ... \LaTeX:\\ \url{E:\Backup Mijn Documenten\Hogeschool vakken\TINLab advnced algorithms\tinlab_advanced_algoriths\lesmateriaal\modelchecking.pdf}





\bibitem{ID} ... \LaTeX:\\ \url{E:\Backup Mijn Documenten\Hogeschool vakken\TINLab advnced algorithms\tinlab_advanced_algoriths\lesmateriaal}

parallelle compositie
\bibitem{ID} ... \LaTeX:\\ \url{file:///E:/Backup%20Mijn%20Documenten/Hogeschool%20vakken/TINLab%20advnced%20algorithms/tinlab_advanced_algoriths/achtergrondinfo%20research/properties%20ctl/model.pdf}

Urgent locations
Is hetzelfde als het toevoegen van een clock x, met een invariant x<=o op de locatie. Zolang een systeem in een urgente locatie zit mag er geen tijd verstrijken
Bjivoorbeeld als een sluis klaar is engeen schpeen in de sluis. Dan moet er een urgentie zijn dat alle schepen waar mogelijk worden opgesteld voor invaren. Als er geen schepen in de wachtrij en er staan geenschepen klaar om in te varen dn is er misschien urgentie om aan de andere kant schepen op te halen.
Commited locations
Als een of meerdere locaties ingesteld zijn als committed. Een committed state kan niet vertragen  en de volgende transitie moet een transitie zijn waarin de uitgaande edge komt van een committed edge


zeno gedrag: de mogelijkheid dat in een eindige hoeveelheid tijd een oneindig antal handelingen kan worden verricht.
Bijvoorbeeld tijdens het nivelleren
Bij het opstellen van schepen
Bij het laten wachten van schepen
Bij het invaren van schepen
\bibitem{ID} ... \LaTeX:\\ \url{file:///E:/Backup%20Mijn%20Documenten/Hogeschool%20vakken/TINLab%20advnced%20algorithms/tinlab_advanced_algoriths/achtergrondinfo%20research/properties%20ctl/std.pdf}



\bibitem{ID} ... \LaTeX:\\ \url{https://wayback.archive-it.org/9650/20200409062940/http:/p3-raw.greenpeace.org/international/Global/international/publications/nuclear/2016/Nuclear_Scars.pdf}
\bibitem{ID} ... \LaTeX:\\ \url{https://bdtechtalks.com/2020/07/29/self-driving-tesla-car-deep-learning/}



%%%%%%%%%%%%%%%%%%%%%%%%%%%%%%%%%%%%%%%%%%%%%%%%%%%%%%%%%%%%%%%%%
critical safety systems chemicals
\bibitem{esc16022021scsdeveloping} ... \LaTeX:\\ \url{https://esc.uk.net/safety-critical-systems}
\bibitem{oecd2008chemsafeperfindct} ... \LaTeX:\\ \url{https://www.oecd.org/chemicalsafety/chemical-accidents/41269710.pdf}
\bibitem{issa2003chemicalsID} ... \LaTeX:\\ \url{https://safety-work.org/fileadmin/safety-work/articles/Verwechslung_von_Chemikalien/Stoffverwechslung_e.pdf}
\bibitem{sommerville2008CriticalSystems} ... \LaTeX:\\ \url{https://ifs.host.cs.st-andrews.ac.uk/Books/SE9/Web/Dependability/CritSys.html}
\bibitem{identifyinglaboratoryHazards} ... \LaTeX:\\ \url{https://www.acs.org/content/dam/acsorg/about/governance/committees/chemicalsafety/publications/identifying-and-evaluating-hazards-in-research-laboratories.pdf}
\bibitem{ID} ... \LaTeX:\\ \url{https://www.computer.org/csdl/magazine/so/2017/04/mso2017040049/13rRUxCitHw}
\bibitem{ID} ... \LaTeX:\\ \url{https://msquair.files.wordpress.com/2012/06/assca-guiding-philosophic-principles-on-the-design-and-acquisition-of-safety-critical-systems-v1-6.pdf}
\bibitem{ID} ... \LaTeX:\\ \url{https://epsc.be/Documents/PS+Fundamentals/_/EPSC_Process%20Safety%20Fundamentals%20-%20Booklet_March2021.pdf}
\bibitem{winceckCriticalToSafety} ... \LaTeX:\\ \url{https://www.icheme.org/media/8976/xxiv-poster-11.pdf}
\bibitem{chambersHazardAnalysisSCS} ... \LaTeX:\\ \url{https://crpit.scem.westernsydney.edu.au/confpapers/CRPITV55Chambers.pdf}
\bibitem{rslater1998SCSAnalysis} ... \LaTeX:\\ \url{https://users.ece.cmu.edu/~koopman/des_s99/safety_critical/}




%%%%%%%%%%%%%%%%%%%%%%%%%%%%%%%%%%%%%%%%%%%%%%%%%%%%%%%%%%%%%%%%%
critical safety systems airplanes
\bibitem{ID} ... \LaTeX:\\ \url{file:///C:/Users/gally/Downloads/AGARDAG300.pdf}
\bibitem{brat2015verifysafetyflightcritical} ... \LaTeX:\\ \url{https://arxiv.org/abs/1502.02605}
\bibitem{knightchallengessafetyCritical} ... \LaTeX:\\ \url{https://users.encs.concordia.ca/~ymzhang/courses/reliability/ICSE02Knight.pdf}
\bibitem{ID} ... \LaTeX:\\ \url{https://www.jstor.org/stable/44682826}
\bibitem{johnson2006devsafetycritical} ... \LaTeX:\\ \url{http://www.dcs.gla.ac.uk/~johnson/teaching/safety/slides/pt2.pdf}
\bibitem{yeagerSafetyCritical} ... \LaTeX:\\ \url{https://sites.google.com/site/cis115textbook/safety-critical-systems}
\bibitem{ID} ... \LaTeX:\\ \url{https://www.dau.edu/tools/se-brainbook/Pages/Design%20Considerations/Critical-Safety-Item.aspx}
\bibitem{ID} ... \LaTeX:\\ \url{https://mcdpinc.com/safety-critical-systems}
\bibitem{fallsafedesign} ... \LaTeX:\\ \url{https://faculty.up.edu/lulay/MEStudentPage/failsafe.pdf}
\bibitem{2008manualflightsafetyParts} ... \LaTeX:\\ \url{https://www.enidine.com/CorporateSite/media/itt/Resources/Distributors/EndUserDocuments/Suppliers_Documents/QAM03_Rev_E.pdf}
\bibitem{arForce2015VerificationExpectations} ... \LaTeX:\\ \url{https://daytonaero.com/wp-content/uploads/AC-17-01.pdf}
\bibitem{harvardRiskResearchaircraft} ... \LaTeX:\\ \url{https://rmas.fad.harvard.edu/pages/chartered-private-aircraft-0}
\bibitem{ID} ... \LaTeX:\\ \url{https://pubmed.ncbi.nlm.nih.gov/7966484/}
\bibitem{ID} ... \LaTeX:\\ \url{https://nebula.esa.int/content/assessment-methodology-certification-safety-gnc-critical-space-systems}
\bibitem{ID} ... \LaTeX:\\ \url{https://www.aopa.org/training-and-safety/online-learning/safety-spotlights/aircraft-systems}
\bibitem{lalaArchitecturalPrinciples} ... \LaTeX:\\ \url{https://www.cs.unc.edu/~anderson/teach/comp790/papers/safety_critical_arch.pdf}
\bibitem{andersenromanski2011verificationsafetycritical} ... \LaTeX:\\ \url{https://queue.acm.org/detail.cfm?id=2024356}
\bibitem{aviatioLawGeneralDefinitions} ... \LaTeX:\\ \url{https://www.law.cornell.edu/cfr/text/14/1.1}
\bibitem{ntsb2006safetyreportTransportAirplanes} ... \LaTeX:\\ \url{http://libraryonline.erau.edu/online-full-text/ntsb/safety-reports/SR06-02.pdf}
\bibitem{mitNotesSafetyCritical} ... \LaTeX:\\ \url{https://www.cs.uct.ac.za/mit_notes/human_computer_interaction/htmls/ch02s10.html}
\bibitem{ID} ... \LaTeX:\\ \url{https://flightsafety.org/}
\bibitem{kochenfender2020aisafetycritical} ... \LaTeX:\\ \url{https://engineering.stanford.edu/magazine/article/mykel-kochenderfer-ai-and-safety-critical-systems}
\bibitem{humanfactorsAviationSafety} ... \LaTeX:\\ \url{https://www.faasafety.gov/files/gslac/courses/content/258/1097/AMT_Handbook_Addendum_Human_Factors.pdf}
\bibitem{eulegislator2000safetyregulation} ... \LaTeX:\\ \url{https://www.eurocontrol.int/sites/default/files/2019-06/src-doc-1-e1.0.pdf}
\bibitem{ID} ... \LaTeX:\\ \url{http://aerossurance.com/safety-management/critical-maintenance-tasks/}
\bibitem{gao2020aviationcybersecurity} ... \LaTeX:\\ \url{https://www.gao.gov/assets/gao-21-86.pdf}
\bibitem{roleofcodeinsoftware} ... \LaTeX:\\ \url{https://criticalsoftware.com/en/news/coding-the-skies}
\bibitem{airlie2018designparameters} ... \LaTeX:\\ \url{https://aviation.stackexchange.com/questions/46677/what-are-the-design-parameters-for-airliner-safety}
\bibitem{ID} ... \LaTeX:\\ \url{https://www.cantwell.senate.gov/news/press-releases/cantwells-comprehensive-bipartisan-bicameral-aircraft-safety-and-certification-reforms-signed-into-law}
\bibitem{ID} ... \LaTeX:\\ \url{https://www.forbes.com/advisor/travel-rewards/737-max-what-is-safety-anyway/}
\bibitem{humanfactors2010Aviation} ... \LaTeX:\\ \url{https://www.tandfonline.com/doi/full/10.1080/00140130903521587}
\bibitem{ID} ... \LaTeX:\\ \url{https://www.doi.gov/aviation/safety}
\bibitem{courtin2018safetyconsiderations} ... \LaTeX:\\ \url{https://dspace.mit.edu/bitstream/handle/1721.1/118438/ICAT_2018_07_Christoper%20Courtin_Report.pdf?sequence=1&isAllowed=y}
\bibitem{ID} ... \LaTeX:\\ \url{https://www.defence.gov.au/dasp/Docs/Manuals/7001053/eTAMMweb/1049.htm}
\bibitem{prentice2014Failedaviationprogram} ... \LaTeX:\\ \url{https://www.aviationpros.com/aircraft/commercial-airline/article/10239806/staying-legal-another-failed-faa-safety-program}
\bibitem{civilAviationConsulting} ... \LaTeX:\\ \url{https://www.iata.org/en/services/consulting/safety-operations/}
\bibitem{olivercalvardpotocnik2017Aviationautomation} ... \LaTeX:\\ \url{https://hbr.org/2017/09/the-tragic-crash-of-flight-af447-shows-the-unlikely-but-catastrophic-consequences-of-automation}
\bibitem{ID} ... \LaTeX:\\ \url{https://www.infosys.com/industries/communication-services/documents/landing-gear-design-and-development.pdf}
\bibitem{airforce2000SystemSafety} ... \LaTeX:\\ \url{https://www.acqnotes.com/Attachments/AF_System-Safety-HNDBK.pdf}
\bibitem{fed2019SafeSecureSUAS} ... \LaTeX:\\ \url{https://www.transportation.gov/testimony/state-airline-safety-federal-oversight-commercial-aviation}
\bibitem{AirportSafety} ... \LaTeX:\\ \url{https://www.federalregister.gov/documents/2019/02/13/2019-00758/safe-and-secure-operations-of-small-unmanned-aircraft-systems}
\bibitem{ID} ... \LaTeX:\\ \url{https://archive.etsc.eu/documents/safety%20in%20airports.pdf}
\bibitem{passengerSafety} ... \LaTeX:\\ \url{https://journals.sagepub.com/doi/pdf/10.1177/002029400403700202}
\bibitem{ID} ... \LaTeX:\\ \url{https://www.unmannedsystems.ca/wp-content/uploads/2019/01/DRAFT-AC-922-001-RPAS-SAFETY-ASSURANCE.pdf}
\bibitem{ID} ... \LaTeX:\\ \url{https://www.ccsdualsnap.com/pressure-switches-in-aerospace-applications/}
\bibitem{britishColumbia2020GuideSafetyCritical} ... \LaTeX:\\ \url{https://www.egbc.ca/getmedia/78073fda-5a83-4f0f-b12f-0a40dcbbc29d/EGBC-Safety-Critical-Software-V1-0.pdf.aspx}
\bibitem{uscongres2019aircraftresearch} ... \LaTeX:\\ \url{https://readwrite.com/2018/12/21/air-travel-is-far-safer-than-you-think-heres-why/}
\bibitem{ID} ... \LaTeX:\\ \url{https://fas.org/sgp/crs/misc/R45939.pdf}
\bibitem{ariAssociation2018} ... \LaTeX:\\ \url{https://cdn.ymaws.com/www.astna.org/resource/collection/4392B20B-D0DB-4E76-959C-6989214920E9/ASTNA_Safety_Position_Paper_2018_FINAL.pdf}
\bibitem{ID} ... \LaTeX:\\ \url{https://transportation.house.gov/imo/media/doc/2020.09.15%20FINAL%20737%20MAX%20Report%20for%20Public%20Release.pdf}
\bibitem{ID} ... \LaTeX:\\ \url{https://www.transportstyrelsen.se/globalassets/global/luftfart/seminarier_och_information/seminarier-2016/luftvardighet-camo-och-145-verkstader/11b-critical-task-fpl.pdf}
\bibitem{fox2005HelicopterSafety} ... \LaTeX:\\ \url{https://www.h-a-c.ca/IHSS_Helicopter_Safety_History_05.pdf}
\bibitem{fulvio1993safetycriticalsystems} ... \LaTeX:\\ \url{https://assembly.coe.int/nw/xml/XRef/X2H-Xref-ViewHTML.asp?FileID=7144&lang=EN}
\bibitem{rakas2018criticalsystemsFailures} ... \LaTeX:\\ \url{https://www.skybrary.aero/index.php/Cockpit_Automation_-_Advantages_and_Safety_Challenges}
\bibitem{ID} ... \LaTeX:\\ \url{https://ntrs.nasa.gov/citations/20120014507}
\bibitem{cockpitAutomation} ... \LaTeX:\\ \url{https://www.sciencedirect.com/science/article/abs/pii/S092575351730601X}
\bibitem{ID} ... \LaTeX:\\ \url{https://www.semanticscholar.org/paper/Safety-critical-avionics-for-the-777-primary-flight-Yeh/8facf90f4a9051c3ab8ce11e39d0893118268d90}
\bibitem{ID} ... \LaTeX:\\ \url{https://www.easa.europa.eu/faq/19013}
\bibitem{belcaatrovalidateSafetyCritical} ... \LaTeX:\\ \url{https://ntrs.nasa.gov/api/citations/20120014507/downloads/20120014507.pdf}
\bibitem{ID,	civilAviationsAuthority = {author},	ALTeditor = {editor},	title = {title},	date = {date},	url = {"https://publicapps.caa.co.uk/modalapplication.aspx?catid=1&pagetype=65&appid=11&mode=list&type=subcat&id=32}
	\bibitem{ID} ... \LaTeX:\\ \url{https://definitions.uslegal.com/f/flight-safety-critical-aircraft-part/}
	\bibitem{ID} ... \LaTeX:\\ \url{https://nbaa.org/nbaa-aviation-groups-ask-congress-to-prevent-5g-interference-to-critical-safety-systems/}
	\bibitem{ID} ... \LaTeX:\\ \url{https://www.dlr.de/ft/en/desktopdefault.aspx/tabid-1360/1856_read-36215/}
	\bibitem{ID} ... \LaTeX:\\ \url{https://www.fsd.lrg.tum.de/research/safety-critical/}
	\bibitem{knight2010SafetyCritical} ... \LaTeX:\\ \url{https://ieeexplore.ieee.org/document/1007998}
	\bibitem{AAA052005IdentifySCHitems} ... \LaTeX:\\ \url{https://www.faa.gov/about/office_org/headquarters_offices/ast/licenses_permits/media/RLVGuide_01-05_05.pdf}
	\bibitem{devTopics01032020} ... \LaTeX:\\ \url{https://smallbusinessprogramming.com/safety-critical-software-15-things-every-developer-should-know/}
	\bibitem{ID} ... \LaTeX:\\ \url{https://coreavi.com/the-future-of-safety-critical-systems-in-the-emerging-autonomous-world/}
	\bibitem{valdes2018SafetybyAutomation} ... \LaTeX:\\ \url{https://www.intechopen.com/chapters/59838}
	\bibitem{2015whensafetymanagementsystemsfail} ... \LaTeX:\\ \url{https://verticalmag.com/features/whensafetymanagementsystemsfail/}
	
	
	%%%%%%%%%%%%%%%%%%%%%%%%%%%%%%%%%%%%%%%%%%%%%%%%%%%%%%%%%%%%%%%%%
	critical safety systems fireworks
	\bibitem{ID} ... \LaTeX:\\ \url{https://www.hsdl.org/c/firework-safety/}
	\bibitem{ID} ... \LaTeX:\\ \url{https://www.cpsc.gov/Safety-Education/Safety-Education-Centers/Fireworks}
	\bibitem{ID} ... \LaTeX:\\ \url{https://www.seattletimes.com/subscribe/signup-offers/?pw=redirect&subsource=paywall&return=https://www.seattletimes.com/opinion/editorials/firework-safety-even-more-critical-after-heat-wave/}
	\bibitem{ID} ... \LaTeX:\\ \url{https://www.nrcan.gc.ca/sites/www.nrcan.gc.ca/files/mineralsmetals/pdf/mms-smm/expl-expl/20170828-G05-09E_ACC.pdf}
	\bibitem{fireworksinjuries2021duringcovid} ... \LaTeX:\\ \url{https://www.prnewswire.com/news-releases/fireworks-related-injuries-and-deaths-spiked-during-the-covid-19-pandemic-301322243.html}
	\bibitem{osha2017safetymanagementexplosives} ... \LaTeX:\\ \url{https://www.osha.gov/sites/default/files/publications/OSHA3912.pdf}
	\bibitem{ID} ... \LaTeX:\\ \url{https://www.firelinx.com/wp-content/uploads/2021/02/FLX-Issues-in-Firing-System-Safety.pdf}
	\bibitem{ID} ... \LaTeX:\\ \url{http://www.eig2.org.uk/wp-content/uploads/WTOFD-Blue-Guide.pdf}
	\bibitem{hse2014explosiveregulations} ... \LaTeX:\\ \url{https://www.hse.gov.uk/explosives/er2014-fireworks-retail-prem.pdf}
	\bibitem{ID} ... \LaTeX:\\ \url{https://www.firerescue1.com/firefighter-safety/articles/11-fireworks-safety-videos-from-the-serious-to-the-humorous-fHy0M4pT2gjcQ8jA/}
	\bibitem{pirone2016lessonslearnedfireworks} ... \LaTeX:\\ \url{https://www.aidic.it/cet/16/53/044.pdf}
	\bibitem{ID} ... \LaTeX:\\ \url{http://www.alarmascasas.com.mx/sites/default/files/85006-0061%20--%20FireWorks%20Brochure.pdf}
	\bibitem{ID} ... \LaTeX:\\ \url{https://www.ehs.ufl.edu/programs/fire/fireworks/}
	\bibitem{ID} ... \LaTeX:\\ \url{https://www.interlogix.com.au/documents/FireWorks%20Features%20and%20Operation%20(fire%20only).pdf}
	\bibitem{ID} ... \LaTeX:\\ \url{https://townhall.virginia.gov/l/GetFile.cfm?File=C:%5CTownHall%5Cdocroot%5CGuidanceDocs%5C960%5CGDoc_DFP_4448_v1.pdf}
	\bibitem{ID} ... \LaTeX:\\ \url{https://ec.europa.eu/growth/sectors/chemicals/specific-chemicals_en}
	\bibitem{safehandlepyrotechnics} ... \LaTeX:\\ \url{http://www.iiakm.org/ojakm/articles/2015/volume3_3/OJAKM_Volume3_3pp27-36.pdf}
	\bibitem{ID} ... \LaTeX:\\ \url{https://www.bristol.gov.uk/documents/20182/1175006/Fireworks+in+retail+premises/6aa6ee24-5b74-43b4-a1d9-747689b1dbc9}
	\bibitem{ID} ... \LaTeX:\\ \url{https://www.eversys.com.br/imagens/uploads/arqs/bra_arquivos/04-software-gerenciador-fireworks-brochura.pdf}
	\bibitem{ID} ... \LaTeX:\\ \url{http://www.doiserbia.nb.rs/img/doi/0354-9836/2016/0354-98361500050G.pdf}
	\bibitem{costinFireworksEmbedded} ... \LaTeX:\\ \url{http://s3.eurecom.fr/docs/wisec14_Costin.pdf}
	\bibitem{ID} ... \LaTeX:\\ \url{https://www.firetechsystems.com/assets/uploads/2018/09/FireWorks-Brochure.pdf}
	\bibitem{ritz2020firesafety} ... \LaTeX:\\ \url{https://blog.ritzsafety.com/fireworks-safety-tips}
	\bibitem{lawson2018criticalsystems} ... \LaTeX:\\ \url{https://www.engineerlive.com/content/fire-detection-and-protection-through-safety-critical-systems}
	%%%%%%%%%%%%%%%%%%%%%%%%%%%%%%%%%%%%%%%%%%%%%%%%%%%%%%%%%%%%%%%%%
%    
%    algemene vragen
%    oorzaken
%    @online{gates18112020boeingcrisis,	ALTauthor = {author},	ALTeditor = {editor},	title = {title},	date = {date},	url = {"https://www.seattletimes.com/business/boeing-aerospace/what-led-to-boeings-737-max-crisis-a-qa/"},}
%    @online{boeing737maxsoftwareprobles,	ALTauthor = {author},	ALTeditor = {editor},	title = {title},	date = {date},	url = {"https://www.schneier.com/blog/archives/2019/04/excellent_analy.html"},}
%    fout in de software
%    @online{avetisov19032019boeingmalwarestate,	ALTauthor = {author},	ALTeditor = {editor},	title = {title},	date = {date},	url = {"https://www.forbes.com/sites/georgeavetisov/2019/03/19/malware-at-30000-feet-what-the-737-max-says-about-the-state-of-airplane-software-security/?sh=4d26f7052a9e"},}
%    het nationaal veiligheidsbelang
%    @online{thompson23112020nationalsecurityboeing,	ALTauthor = {author},	ALTeditor = {editor},	title = {title},	date = {date},	url = {"https://www.forbes.com/sites/lorenthompson/2020/11/23/five-reasons-return-of-boeings-737-max-to-service-is-important-to-national-security/?sh=2128ea552018"},}
%    falend toezicht
%    @online{gates21032019FAAControlSystem,	ALTauthor = {author},	ALTeditor = {editor},	title = {title},	date = {date},	url = {"https://www.seattletimes.com/business/boeing-aerospace/failed-certification-faa-missed-safety-issues-in-the-737-max-system-implicated-in-the-lion-air-crash/"},}
%    onderzoeksrapport
%    @online{faa18112020boeingreview,	ALTauthor = {author},	ALTeditor = {editor},	title = {title},	date = {date},	url = {"https://www.faa.gov/foia/electronic_reading_room/boeing_reading_room/media/737_RTS_Summary.pdf"},}
%    @online{wiki737maxgroundings,	ALTauthor = {author},	ALTeditor = {editor},	title = {title},	date = {date},	url = {"https://en.wikipedia.org/wiki/Boeing_737_MAX_groundings"},}
%    veiligheidsrisico's
%    menselijke fouten
%    @online{campbell02052019boengcrashhumanerrors,	ALTauthor = {author},	ALTeditor = {editor},	title = {title},	date = {date},	url = {"https://www.theverge.com/2019/5/2/18518176/boeing-737-max-crash-problems-human-error-mcas-faa"},}
%    overzicht van crashes
%    @online{hawkins22032019737maxairplanes,	ALTauthor = {author},	ALTeditor = {editor},	title = {title},	date = {date},	url = {"https://www.theverge.com/2019/3/22/18275736/boeing-737-max-plane-crashes-grounded-problems-info-details-explained-reasons"},}
%    veiligheidsopmerking
%    @online{thomas30082020737safest,	ALTauthor = {author},	ALTeditor = {editor},	title = {title},	date = {date},	url = {"https://www.airlineratings.com/news/boeings-737-max-will-one-safest-aircraft-history/"},}
%    aanpassingen
%    @online{boeing737maxdisplay,	ALTauthor = {author},	ALTeditor = {editor},	title = {title},	date = {date},	url = {"https://www.boeing.com/commercial/737max/737-max-software-updates.page"},}
%    waarschuwingen//output signalen
%    @online{fehrm24112020737changes,	ALTauthor = {author},	ALTeditor = {editor},	title = {title},	date = {date},	url = {"https://leehamnews.com/2020/11/24/boeing-737-max-changes-beyond-mcas/"},}
%    software gerelateerde fouten
%    @online{travis18042019737maxsoftwaredevop,	ALTauthor = {author},	ALTeditor = {editor},	title = {title},	date = {date},	url = {"https://spectrum.ieee.org/aerospace/aviation/how-the-boeing-737-max-disaster-looks-to-a-software-developer"},}
%    onderzoeksrapport
%    de rol van de publieke opinie
%    @online{barnett05052019737maxcrisis,	ALTauthor = {author},	ALTeditor = {editor},	title = {title},	date = {date},	url = {"https://pubsonline.informs.org/do/10.1287/orms.2019.05.05/full/"},}
%    onderzoek van europese luchtvaart agentschap
%    @online{easa27012021737maxsafereturn,	ALTauthor = {author},	ALTeditor = {editor},	title = {title},	date = {date},	url = {"https://www.easa.europa.eu/newsroom-and-events/news/easa-declares-boeing-737-max-safe-return-service-europe"},}
%    veiligheidsvraagstuk
%    @online{touitou11032019737tragedies,	ALTauthor = {author},	ALTeditor = {editor},	title = {title},	date = {date},	url = {"https://phys.org/news/2019-03-boeing-max-safety-tragedies.html"},}
%    artikel over sensoren
%    @online{hemmerdinger02022021737maxdeliveries,	ALTauthor = {author},	ALTeditor = {editor},	title = {title},	date = {date},	url = {"https://www.flightglobal.com/airframers/boeing-delays-737-max-10-deliveries-two-years-to-2023/142245.article"},}
%    goedkeuring van europese luchtvaart autoriteiten
%    advies aan de faa
%    @online{bielby27022021faaimprovesafety,	ALTauthor = {author},	ALTeditor = {editor},	title = {title},	date = {date},	url = {"https://www.hstoday.us/subject-matter-areas/airport-aviation-security/oig-tells-faa-to-improve-safety-oversight-following-boeing-737-max-review/"},}
%    @online{boyle18112020737maxupgrade,	ALTauthor = {author},	ALTeditor = {editor},	title = {title},	date = {date},	url = {"https://www.geekwire.com/2020/faas-go-ahead-737-maxs-return-flight-kicks-off-massive-software-upgrade/"},}
%    @online{bergstraburgess122019737maxMcasAlgorithm,	ALTauthor = {author},	ALTeditor = {editor},	title = {title},	date = {date},	url = {"https://www.researchgate.net/publication/338420944_A_Promise_Theoretic_Account_of_the_Boeing_737_Max_MCAS_Algorithm_Affair"},}
%    achtergrond informatie
%    @online{737mcas,	ALTauthor = {author},	ALTeditor = {editor},	title = {title},	date = {date},	url = {"http://www.b737.org.uk/mcas.htm"},}
%    algemeen vertrouwen
%    @online{newburger17052019boeingcrisis,	ALTauthor = {author},	ALTeditor = {editor},	title = {title},	date = {date},	url = {"https://www.cnbc.com/2019/05/16/what-you-need-to-know-about-boeings-737-max-crisis.html"},}
%    toestemming europese autoriteiten
%    problemen
%    @online{arstechnica22012020737problems,	ALTauthor = {author},	ALTeditor = {editor},	title = {title},	date = {date},	url = {"https://arstechnica.com/information-technology/2020/01/737-max-fix-slips-to-summer-and-thats-just-one-of-boeings-problems/"},}
%    uitgebreid artikel over de onderzoeken en het vliegverbod
%    @online{german190620217372yaftergrounded,	ALTauthor = {author},	ALTeditor = {editor},	title = {title},	date = {date},	url = {"https://www.cnet.com/news/boeing-737-max-8-all-about-the-aircraft-flight-ban-and-investigations/"},}
%    computers als oorzaak
%    lessons learned
%    @online{beningo02052019boeinglessons,	ALTauthor = {author},	ALTeditor = {editor},	title = {title},	date = {date},	url = {"https://www.designnews.com/electronics-test/5-lessons-learn-boeing-737-max-fiasco"},}
%    @online{duran05042019boeingspof,	ALTauthor = {author},	ALTeditor = {editor},	title = {title},	date = {date},	url = {"https://www.eurocontrol.int/publication/effects-network-extra-standby-aircraft-and-boeing-737-max-grounding"},}
%    single point of failure
%    @online{ID,	ALTauthor = {author},	ALTeditor = {editor},	title = {title},	date = {date},	url = {"https://dmd.solutions/blog/2019/04/05/how-a-single-point-of-failure-spof-in-the-mcas-software-could-have-caused-the-boeing-737-max-crash-in-ethiopia/"},}
%    @online{makichuck24012021737fearflying,	ALTauthor = {author},	ALTeditor = {editor},	title = {title},	date = {date},	url = {"https://asiatimes.com/2021/01/boeings-737-max-and-the-fear-of-flying/"},}
%    lijst van tehnische aanpassingen
%    @online{caa737modifications,	ALTauthor = {author},	ALTeditor = {editor},	title = {title},	date = {date},	url = {"https://www.caa.co.uk/Consumers/Guide-to-aviation/Boeing-737-MAX/"},}
%    @online{oestergaard14122020boeingdeliveries,	ALTauthor = {author},	ALTeditor = {editor},	title = {title},	date = {date},	url = {"https://dsm.forecastinternational.com/wordpress/2020/12/14/airbus-and-boeing-report-november-2020-commercial-aircraft-orders-and-deliveries/"},}
%    code lek
%    @online{reenberg787flaws,	ALTauthor = {author},	ALTeditor = {editor},	title = {title},	date = {date},	url = {"https://www.wired.com/story/boeing-787-code-leak-security-flaws/"},}
%    @online{fitch16092020737backlogrisks,	ALTauthor = {author},	ALTeditor = {editor},	title = {title},	date = {date},	url = {"https://www.fitchratings.com/research/corporate-finance/boeing-737-max-return-backlog-risks-remain-16-09-2020"},}
%    Cultuurverandering, deregulatie, systeemwijziging of gewoon een kwestie van competentie
%    @online{willis27082020737maxfailures,	ALTauthor = {author},	ALTeditor = {editor},	title = {title},	date = {date},	url = {"https://www.aerospacetestinginternational.com/features/what-broke-the-737-max.html"},}
%    extra aanpassingen
%    @online{ostrower11062020more737changes,	ALTauthor = {author},	ALTeditor = {editor},	title = {title},	date = {date},	url = {"https://theaircurrent.com/aviation-safety/boeings-737-max-software-done-but-regulators-plot-more-changes-after-jets-return/"},}
%    wat ging er mis een analyse van een ex-iloot
%    De utoriteiten waren op de hoogte
%    @online{hruska13122019faaknown737crashrate,	ALTauthor = {author},	ALTeditor = {editor},	title = {title},	date = {date},	url = {"https://www.extremetech.com/extreme/303373-the-faa-knew-the-737-max-was-dangerous-and-kept-it-flying-anyway"},}
%    kwaliteiten van het alarmsysteem niet goed bekend
%    @online{bloomberg26092019failedpred,	ALTauthor = {author},	ALTeditor = {editor},	title = {title},	date = {date},	url = {"https://time.com/5687473/boeing-737-alarm-system/"},}
%    @online{whiteman09072020boengcancelstock,	ALTauthor = {author},	ALTeditor = {editor},	title = {title},	date = {date},	url = {"https://www.nasdaq.com/articles/boeing-gets-dealt-another-737-max-cancellation-blow.-what-it-means-for-boeing-stock-2020"},}
%    @online{leopold09192019boeingreliability,	ALTauthor = {author},	ALTeditor = {editor},	title = {title},	date = {date},	url = {"https://www.eetimes.com/boeing-crashes-highlight-a-worsening-reliability-crisis/"},}
%    veiligheidsvraagstuk
%    @online{koenig11122019737crashesnofix,	ALTauthor = {author},	ALTeditor = {editor},	title = {title},	date = {date},	url = {"https://www.latimes.com/business/story/2019-12-11/faa-boeing-737-max-crashes"},}
%    probleemanalyse, veiligheidsvraagstuk
%    @online{dohertylindeman15032019737problems,	ALTauthor = {author},	ALTeditor = {editor},	title = {title},	date = {date},	url = {"https://www.politico.com/story/2019/03/15/boeing-737-max-grounding-1223072"},}
%    falend toezicht
%    @online{stodder02102019corruptoversight,	ALTauthor = {author},	ALTeditor = {editor},	title = {title},	date = {date},	url = {"https://www.pogo.org/analysis/2019/10/corrupted-oversight-the-faa-boeing-and-the-737-max/"},}
%    @online{afacwaLostSafeguards,	ALTauthor = {author},	ALTeditor = {editor},	title = {title},	date = {date},	url = {"https://www.afacwa.org/the_inside_story_of_mcas_seattle_times"},}
%    doelstellingen en veiligheidsvraagstukken
%    @online{swayne18032019profitssafety,	ALTauthor = {author},	ALTeditor = {editor},	title = {title},	date = {date},	url = {"https://www.marxist.com/737-max-scandal-boeing-putting-profits-before-safety.htm"},}
%    @online{freed26022021liftaustraliaban,	ALTauthor = {author},	ALTeditor = {editor},	title = {title},	date = {date},	url = {"https://finance.yahoo.com/news/australia-lifts-ban-boeing-737-035817682.html?guccounter=1&guce_referrer=aHR0cHM6Ly93d3cuZ29vZ2xlLmNvbS8&guce_referrer_sig=AQAAAHZCJYy_0A5VS2WiPoCvH4xdrRNkmkdsv5EWJ2RLIz_AS-rxsTty6AF1_HlmJiRyWYqCXDi4p0Xs4isYkNkCq2Pfo-pQ60Xz_IfTNjm4FgoZiBMC4zpZlB6F0fwecrjE_ujAXZzG4xPJnWCd8-G3VLlPTY8h3H31eQ1i8hY9AIyy"},}
%    autoriteiten krijgen tik op de vingers
%    @online{reed15032019softwareattention,	ALTauthor = {author},	ALTeditor = {editor},	title = {title},	date = {date},	url = {"https://medium.com/@jpaulreed/the-737max-and-why-software-engineers-should-pay-attention-a041290994bd"},}
%    @online{news17032019softwareexplains,	ALTauthor = {author},	ALTeditor = {editor},	title = {title},	date = {date},	url = {"https://news.ycombinator.com/item?id=19414775"},}
%    @online{legget21122020eu737maxsafe,	ALTauthor = {author},	ALTeditor = {editor},	title = {title},	date = {date},	url = {"https://www.bbc.com/news/55366320"},}
%    @online{marketscreener0103221737chinarecertification,	ALTauthor = {author},	ALTeditor = {editor},	title = {title},	date = {date},	url = {"https://www.marketscreener.com/news/latest/China-studies-Boeing-737-MAX-recertification-wants-safety-concerns-fully-addressed--32569125/"},}
%    motor in brand
%    @online{euractiv22022021737firegrounds,	ALTauthor = {author},	ALTeditor = {editor},	title = {title},	date = {date},	url = {"https://www.euractiv.com/section/aviation/news/boeing-grounds-777s-after-engine-fire/"},}
%    @online{benny18022019737returnUAE,	ALTauthor = {author},	ALTeditor = {editor},	title = {title},	date = {date},	url = {"https://gulfnews.com/business/aviation/uae-airspace-to-see-return-of-boeing-737-max-1.1613627548923"},}
%    motor in brand gevlogen
%    @online{biersmichel22022021777grounds,	ALTauthor = {author},	ALTeditor = {editor},	title = {title},	date = {date},	url = {"https://techxplore.com/news/2021-02-boeing-urges-grounding-777s.html"},}
%    @online{ID,	ALTauthor = {author},	ALTeditor = {editor},	title = {title},	date = {date},	url = {"https://www.politico.eu/article/uk-temporarily-bans-some-boeing-aircraft-after-pratt-whitney-engine-incidents/"},}
%    @online{reuters23022021777metalfatigue,	ALTauthor = {author},	ALTeditor = {editor},	title = {title},	date = {date},	url = {"https://www.timeslive.co.za/news/world/2021-02-23-damage-to-united-boeing-777-engine-consistent-with-metal-fatigue--ntsb/"},}
%    faa was niet kritisch genoeg
%    @online{ID,	ALTauthor = {author},	ALTeditor = {editor},	title = {title},	date = {date},	url = {"https://federalnewsnetwork.com/government-news/2021/02/federal-watchdog-blasts-faa-over-certification-of-boeing-jet/"},}
%    
%    
%    
%    
%    
%    
%    
%    \cite{bnnvara13062018malirapport}
%    \cite{eucal11012021malimissieverlengd}
%    \cite{nos21052014zorgenmalimissie}
%    \cite{meijnders}
%    \cite{bnrwebredactie}
%    \cite{keultjes01062016malimissiecoalitie}
%    \cite{veenhof18012019}
%    
%    \cite{isitman06012016militair}
%    \cite{nporadio11072016filmdemissie}
%    \cite{parlementairmonitor15122013mortierongeluk}
%    
%    sollicitatie
%    de bureaucratie
%    aankomst
%    interview van de burgerbevolking
%    steun van de bevolking minuut 15:00
%    de organisatie minuut 23:00
%    De militaire briefing minuut 34:00
%    prioriteit minuut 39:00
%    briefing minuut 40:00
%    de communicatie met ministerie over inlichten minuut 44:00
%    @online{DemissieFilm,	ALTauthor = {author},	ALTeditor = {editor},	title = {title},	date = {date},	url = {"https://www.2doc.nl/documentaires/series/2doc/2016/juli/de-missie.html"},}
%    \cite{DemissieFilm}
%    
%    
%    
%    
%    
%    
%    verhaal van brandweermannen
%    
%    \cite{staff31082015tanjinblastunrevealed}
%    artikel
%    
%    \cite{chinafile18082015tanjinexplosion}
%    invloed van social media
%    \bibitem{ID} ... \LaTeX:\\ \url{https://www.economist.com/asia/2015/08/18/a-blast-in-tianjin-sets-off-an-explosion-online}
%    \cite{}
%    \bibitem{ID} ... \LaTeX:\\ \url{https://america.cgtn.com/2015/08/12/explosion-reported-in-tianjin-china}
%    \cite{}
%    \bibitem{ID} ... \LaTeX:\\ \url{https://factcheck.afp.com/no-photo-was-taken-chinese-city-tianjin-august-2015}
%    \cite{}
%    vergelijking van twee rampen
%    \bibitem{ID} ... \LaTeX:\\ \url{https://airshare.air-inc.com/how-does-the-beirut-explosion-compare-to-tianjin}
%    \cite{}
%    overheid en media
%    \bibitem{ID} ... \LaTeX:\\ \url{https://newbloommag.net/2015/08/17/tianjin-explosion/}
%    \cite{}
%    chemische industrie ondeer de loep
%    \bibitem{ID} ... \LaTeX:\\ \url{https://www.voanews.com/east-asia-pacific/tianjin-blast-puts-spotlight-chemical-industry}
%    \cite{}
%    \bibitem{ID} ... \LaTeX:\\ \url{https://abcnews.go.com/International/apocalyptic-aftermath-devastating-images-tianjin-china-explosions/story?id=33057017}
%    \cite{}
%    \bibitem{ID} ... \LaTeX:\\ \url{https://www.reachingoutacrossdurham.co.uk/osk/tianjin-explosion-2021}
%    \cite{}
%    \bibitem{pinghuang2410201TanjinFactreport} ... \LaTeX:\\ \url{https://aiche.onlinelibrary.wiley.com/doi/abs/10.1002/prs.11789}
%    \cite{pinghuang2410201TanjinFactreport}
%    \bibitem{ID} ... \LaTeX:\\ \url{https://www.automotivelogistics.media/thousands-of-cars-destroyed-in-tianjin-port-explosions/13570.article}
%    \cite{}
%    \bibitem{ID} ... \LaTeX:\\ \url{https://www.joc.com/port-news/asian-ports/port-tianjin/tianjin-port-explosions-could-be-most-expensive-maritime-disaster_20150826.html}
%    \cite{}
%    \bibitem{ID} ... \LaTeX:\\ \url{https://www.bloomberg.com/news/articles/2015-08-12/explosion-in-northern-china-shatters-windows-causes-injuries}
%    \cite{}
%    \bibitem{ID} ... \LaTeX:\\ \url{https://unece.org/fileadmin/DAM/env/documents/2016/TEIA/OECD_WGCA_24-27_OCT_2016/Session_3_Zhao_-__Introduction_of_Tianjin_Accident_-_Jinsong_Zhao.pdf}
%    \cite{}
%    gemaakte fouten
%    \bibitem{portoTanjinExplosionSight} ... \LaTeX:\\ \url{https://porteconomicsmanagement.org/pemp/contents/part6/port-resilience/site-2015-tianjin-port-explosions/}
%    \cite{portoTanjinExplosionSight}
%    \bibitem{ID} ... \LaTeX:\\ \url{https://www.alamy.com/stock-image-tianjin-china-17th-aug-2015-tianjin-explosion-aftermath-blast-site-165334778.html}
%    \cite{}
%    \bibitem{ID} ... \LaTeX:\\ \url{https://www.popularmechanics.com/technology/news/a16871/massive-explosions-china-city-of-tianjin/}
%    \cite{}
%    \bibitem{imago17082015TanjinApartmentImages} ... \LaTeX:\\ \url{https://www.imago-images.com/st/0080815934}
%    \cite{imago17082015TanjinApartmentImages}
%    \bibitem{trager14082015Chemicalblast} ... \LaTeX:\\ \url{https://www.chemistryworld.com/news/deadly-chemical-blast-at-chinese-port/8857.article}
%    \cite{trager14082015Chemicalblast}
%    \bibitem{pangeramo27082015TanjinExplosion} ... \LaTeX:\\ \url{https://www.process-worldwide.com/tianjin-explosion-from-chemical-perspective-insights-and-backgrounds-a-502381/}
%    \cite{pangeramo27082015TanjinExplosion}
%    vergelijking met andere explosies
%    \bibitem{ap06082020ammaniumnitrate} ... \LaTeX:\\ \url{https://apnews.com/article/lebanon-fires-us-news-explosions-middle-east-53f4206a7f1db0812262a15d22e1e58f}
%    \cite{ap06082020ammaniumnitrate}
%    invloed van de ramp op de industrie
%    \bibitem{morris14082015TanjinIndustryImpact} ... \LaTeX:\\ \url{https://fortune.com/2015/08/14/tianjin-port-explosion-shipping-delays/}
%    \cite{morris14082015TanjinIndustryImpact}
%    is er sprake van een doofpot
%    \bibitem{milesyu20082015exposingtoxicgovlines} ... \LaTeX:\\ \url{https://www.washingtontimes.com/news/2015/aug/20/inside-china-tianjin-explosions-cover-up-exposes-b/}
%    \cite{milesyu20082015exposingtoxicgovlines}
%    eigendomsverzekering
%    \bibitem{artemis30032016tanjininsurance} ... \LaTeX:\\ \url{https://www.artemis.bm/news/tianjin-explosions-property-insurance-loss-could-reach-3-5bn-swiss-re/}
%    \cite{artemis30032016tanjininsurance}
%    \bibitem{aidenxiatanjinblast} ... \LaTeX:\\ \url{https://www.thechinastory.org/yearbooks/yearbook-2015/forum-the-abyss-%E5%9D%8E/tianjin-explosions/}
%    \cite{aidenxiatanjinblast}
%    effecten op de lange termijn
%    \bibitem{danwangTanjinflexreport} ... \LaTeX:\\ \url{https://www.flexport.com/blog/tianjin-explosion-effect-on-supply-chains/}
%    \cite{danwangTanjinflexreport}
%    \bibitem{keyHighlightsTanjin} ... \LaTeX:\\ \url{https://www.cicm.org.my/images/articles/CICM-Article-on-Tianjin-Blast-Oct2015.pdf}
%    \cite{keyHighlightsTanjin}
%    lessons learned
%    \bibitem{ID} ... \LaTeX:\\ \url{https://www.genre.com/knowledge/blog/lessons-from-the-tianjin-explosion-en.html}
%    \cite{}
%    \bibitem{ID} ... \LaTeX:\\ \url{https://www.ft.com/content/ad62904c-44ce-11e5-b3b2-1672f710807b}
%    \cite{}
%    \bibitem{hartley13082015videofootage} ... \LaTeX:\\ \url{https://www.huffingtonpost.co.uk/2015/08/13/tianjin-explosion-china-shocking-footage-caught-on-camera_n_7980888.html}
%    \cite{hartley13082015videofootage}
%    \bibitem{odonnel01062017firetanjinblast2015} ... \LaTeX:\\ \url{https://www.thatsmags.com/china/post/19189/massive-fire-rocks-tianjin-port}
%    \cite{odonnel01062017firetanjinblast2015}
%    gevolgen voor de industrie
%    \bibitem{ID} ... \LaTeX:\\ \url{https://www.everstream.ai/risk-center/special-reports/the-jiangsu-yancheng-explosion/}
%    \cite{}
%    \bibitem{fan15082015newyorkermistrustchina} ... \LaTeX:\\ \url{https://www.newyorker.com/news/news-desk/after-tianjin-an-outbreak-of-mistrust-in-china}
%    \cite{fan15082015newyorkermistrustchina}
%    framing vanuit de chinese media
%    \bibitem{yanlidongchinamediaframingTanjin} ... \LaTeX:\\ \url{https://www.neliti.com/publications/101997/the-chinese-media-framing-of-the-2015s-tianjin-explosion}
%    \cite{yanlidongchinamediaframingTanjin}
%    \bibitem{evans27092017TnjinInsurance} ... \LaTeX:\\ \url{https://www.reinsurancene.ws/chinese-insurers-settle-1-5-billion-tianjin-blast-claims/}
%    \cite{evans27092017TnjinInsurance}
%    niewsartikel
%    \bibitem{jasi26032019chineschemplant} ... \LaTeX:\\ \url{https://www.thechemicalengineer.com/news/update-78-confirmed-dead-after-chinese-chemicals-plant-explosion/}
%    \cite{jasi26032019chineschemplant}
%    \bibitem{shiqingTanjinExecutiveSentence} ... \LaTeX:\\ \url{https://www.caixinglobal.com/2016-11-10/chinese-executive-receives-suspended-death-sentence-over-2015-tianjin-warehouse-blast-101006325.html}
%    \cite{shiqingTanjinExecutiveSentence}
%    toegang tot de ramplplek vanuit de okale journalistiek
%    \bibitem{sophiebeach15082015} ... \LaTeX:\\ \url{https://chinadigitaltimes.net/2015/08/he-xiaoxin-how-far-can-i-go-and-how-much-can-i-do/}
%    \cite{sophiebeach15082015}
%    artikel
%    \bibitem{ID} ... \LaTeX:\\ \url{https://www.wnpr.org/post/china-examines-aftermath-immense-twin-explosions-killed-dozens}
%    \cite{}
%    \bibitem{hamzeh05082020BeirutBlast} ... \LaTeX:\\ \url{https://theconversation.com/what-is-ammonium-nitrate-the-chemical-that-exploded-in-beirut-143979}
%    \cite{hamzeh05082020BeirutBlast}
%    \bibitem{chemwatch18082015TanjiinExplosion} ... \LaTeX:\\ \url{https://chemicalwatch.com/36730/nationwide-inspections-in-china-follow-tianjin-explosion}
%    \cite{chemwatch18082015TanjiinExplosion}
%    \bibitem{thehindu15062019chinaExplosion} ... \LaTeX:\\ \url{https://www.thehindu.com/news/international/investigation-begun-into-china-gas-explosion-as-toll-rises/article34818324.ece}
%    \cite{thehindu15062019chinaExplosion}
%    \bibitem{santagotimes24032019chinablast} ... \LaTeX:\\ \url{https://santiagotimes.cl/2019/03/24/64-killed-600-injured-in-china-chemical-plant-blast/}
%    \cite{santagotimes24032019chinablast}
%    oorzaken
%    \bibitem{klingecorp28042020causedTanjin} ... \LaTeX:\\ \url{https://klingecorp.com/blog/what-caused-the-tianjin-explosions/}
%    \cite{klingecorp28042020causedTanjin}
%    case study
%    \bibitem{mcgarryExplosions2017 ... \LaTeX:\\ \url{https://www.preventionweb.net/educational/view/57235}
%    	\cite{mcgarryExplosions2017}
%    	niewsartikel
%    	\bibitem{roswnfeld13082015TanjinReports} ... \LaTeX:\\ \url{https://www.cnbc.com/2015/08/12/explosion-in-tianjin-china.html}
%    	\cite{roswnfeld13082015TanjinReports}
%    	chronologische uiteenzetting
%    	\bibitem{aria12082015explosionaTanjin} ... \LaTeX:\\ \url{https://www.aria.developpement-durable.gouv.fr/wp-content/files_mf/A46803_a46803_fiche_impel_006.pdf}
%    	\cite{aria12082015explosionaTanjin}
%    	corruptie
%    	\bibitem{ID} ... \LaTeX:\\ \url{https://www.nytimes.com/2015/08/31/world/asia/behind-tianjin-tragedy-a-company-that-flouted-regulations-and-reaped-profits.html}
%    	\cite{}
%    	mismanagement als oorzaak
%    	\bibitem{ID} ... \LaTeX:\\ \url{https://www.nytimes.com/2016/02/06/world/asia/tianjin-explosions-were-result-of-mismanagement-china-finds.html}
%    	\cite{}
%    	\bibitem{ID} ... \LaTeX:\\ \url{https://cen.acs.org/articles/94/web/2016/02/Chinese-Investigators-Identify-Cause-Tianjin.html}
%    	\cite{}
%    	autoriteiten publiceren onderoeksrapport
%    	\bibitem{tremblay11022016chineseInvestigatorsTanjin} ... \LaTeX:\\ \url{https://cen.acs.org/articles/94/i7/Chinese-Investigators-Identify-Cause-Tianjin.html}
%    	\cite{tremblay11022016chineseInvestigatorsTanjin}
%    	fotos van de rampplek
%    	\bibitem{taylor13082015TanjinExplosianAftermath} ... \LaTeX:\\ \url{https://www.theatlantic.com/photo/2015/08/photos-of-the-aftermath-of-the-massive-explosions-in-tianjin-china/401228/}
%    	\cite{taylor13082015TanjinExplosianAftermath}
%    	\bibitem{ID} ... \LaTeX:\\ \url{https://edition.cnn.com/2015/08/13/asia/china-tianjin-explosions/index.html}
%    	\cite{}
%    	niuwesartiekel}
%    \bibitem{associatedPresss13082013} ... \LaTeX:\\ \url{https://www.cbc.ca/news/world/china-explosion-tianjin-1.3189455}
%    \cite{associatedPresss13082013}
%    verantwoordelijke
%    \bibitem{ID} ... \LaTeX:\\ \url{https://www.thestar.com/news/world/2016/11/09/chinese-executive-gets-death-sentence-over-tianjin-explosion-in-2015.html}
%    \cite{}
%    risicobeperking/controle
%    \bibitem{ID} ... \LaTeX:\\ \url{https://www.swissre.com/en/china/news-insights/articles/analysis-of-tianjin-port-explosion-china.html}
%    \cite{}
%    censuur
%    \bibitem{ID} ... \LaTeX:\\ \url{https://foreignpolicy.com/2015/09/10/censored-china-young-survivor-tianjin-explosion-viral-post/}
%    \cite{}
%    censuur
%    \bibitem{ID} ... \LaTeX:\\ \url{https://qz.com/756872/a-year-after-the-tianjin-blast-public-mourning-and-discussion-about-it-are-still-censored-in-china/}
%    \cite{}
%    verschillende artikelen
%    \bibitem{ID} ... \LaTeX:\\ \url{https://www.scmp.com/topics/tianjin-warehouse-explosion-2015}
%    \cite{}
%    \bibitem{ID} ... \LaTeX:\\ \url{https://www.wsj.com/articles/BL-CJB-27664}
%    \cite{}
%    \bibitem{ID} ... \LaTeX:\\ \url{https://www.nbcnews.com/news/world/tianjin-explosions-californian-witness-filmed-dramatic-china-blasts-n409701}
%    \cite{}
%    \bibitem{ID} ... \LaTeX:\\ \url{https://ui.adsabs.harvard.edu/abs/2016AGUFM.S13D..06P/abstract}
%    afwikkeling van de ramp
%    \cite{}
%    \bibitem{ID} ... \LaTeX:\\ \url{https://chinadialogue.net/en/pollution/9188-back-to-the-blast-zone-one-year-after-the-tianjin-explosion/}
%    \cite{}
%    \bibitem{ID} ... \LaTeX:\\ \url{https://www.wired.com/2015/08/chinas-huge-tianjin-explosion-looked-like-space/}
%    \cite{}
%    \bibitem{ID} ... \LaTeX:\\ \url{https://www.abc.net.au/news/2015-08-13/explosion-rocks-north-chinese-city-of-tianjin/6693336?nw=0}
%    \cite{}
%    ambtenaren onderzocht
%    
%    risico-inschatting
%    \bibitem{ID} ... \LaTeX:\\ \url{https://www.mdpi.com/2071-1050/12/3/1169/htm}
%    \cite{}
%    \bibitem{ID} ... \LaTeX:\\ \url{https://www.mdpi.com/2071-1050/12/3/1169/htm}
%    \cite{}
%    \bibitem{ID} ... \LaTeX:\\ \url{https://www.cbsnews.com/news/tianjin-port-china-massive-explosion-hundreds-injured/}
%    \cite{}
%    \bibitem{ID} ... \LaTeX:\\ \url{https://www.hkjcdpri.org.hk/download/casestudies/Tianjin_CASE.pdf}
%    \cite{}
%    \bibitem{ID} ... \LaTeX:\\ \url{https://time.com/3996168/tianjin-explosion-china-pictures/}
%    \cite{}
%    onderzoeksrapport
%    \bibitem{ID} ... \LaTeX:\\ \url{https://www.hfw.com/Tianjin-Port-explosion-August-2015}
%    \cite{}
%    \bibitem{un20082015InvestigationTanjin} ... \LaTeX:\\ \url{https://news.un.org/en/story/2015/08/506912-following-tianjin-explosion-un-expert-calls-china-ensure-transparent}
%    \cite{un20082015InvestigationTanjin}
%    \bibitem{france2412082015TnjinExplosion} ... \LaTeX:\\ \url{https://www.france24.com/en/20150812-huge-explosions-rock-chinese-city-tianjin}
%    \cite{france2412082015TnjinExplosion}
%    \bibitem{npr14082015TanjinCause} ... \LaTeX:\\ \url{https://choice.npr.org/index.html?origin=https://www.npr.org/2015/08/14/432280627/what-caused-the-warehouse-explosions-in-tianjin-china}
%    \cite{npr14082015TanjinCause}
%    123 verantwoordelijken
%    \bibitem{bbc05022016TanjinResponsibles} ... \LaTeX:\\ \url{https://www.bbc.com/news/world-asia-china-35506311}
%    \cite{bbc05022016TanjinResponsibles}
%    \bibitem{ID} ... \LaTeX:\\ \url{https://www.washingtonpost.com/gdpr-consent/?next_url=https%3a%2f%2fwww.washingtonpost.com%2fnews%2fworldviews%2fwp%2f2015%2f08%2f12%2fvideos-show-chinese-city-of-tianjin-rocked-by-enormous-explosion%2f}
%    \cite{}
%    lang artiekel
%    \bibitem{CBodeen15082015TanjinExplosion} ... \LaTeX:\\ \url{https://www.businessinsider.com/the-chemical-explosion-in-china-killed-more-than-100-people-and-the-devastation-is-unreal-2015-8?international=true&r=US&IR=T}
%    \cite{CBodeen15082015TanjinExplosion}
%    \bibitem{ID} ... \LaTeX:\\ \url{https://pubmed.ncbi.nlm.nih.gov/27311537/}
%    \cite{}
%    \bibitem{reutersTanjinInsurance} ... \LaTeX:\\ \url{https://www.reuters.com/article/us-china-blast-insurance-idUSKCN0QM0N220150817}
%    \cite{reutersTanjinInsurance}
%    \bibitem{yu082016evaluationTanjin2015} ... \LaTeX:\\ \url{https://www.sciencedirect.com/science/article/abs/pii/S0305417916300079}
%    \cite{yu082016evaluationTanjin2015}
%    \bibitem{wiki2015TanjinExplosions} ... \LaTeX:\\ \url{https://en.wikipedia.org/wiki/2015_Tianjin_explosions}
%    \cite{wiki2015TanjinExplosions}
%    \bibitem{bbc17082015whathappenedTanjin} ... \LaTeX:\\ \url{https://www.bbc.com/news/world-asia-china-33844084}
%    \cite{bbc17082015whathappenedTanjin}
%    \bibitem{mortimer19082016taijinexplosioncrater} ... \LaTeX:\\ \url{https://www.independent.co.uk/news/world/asia/tianjin-explosion-photos-china-chemical-factory-accident-crater-revealed-a7199591.html}
%    \cite{mortimer19082016taijinexplosioncrater}
%    veiigheidshandhaving
%    \bibitem{internationallabourofficeChmControlTooliit} ... \LaTeX:\\ \url{https://www.ilo.org/legacy/english/protection/safework/ctrl_banding/toolkit/main_guide.pdf}
%    \cite{internationallabourofficeChmControlTooliit}
%    \bibitem{ID} ... \LaTeX:\\ \url{https://echa.europa.eu/documents/10162/21332507/guide_chemical_safety_sme_en.pdf}
%    \cite{}
%    \bibitem{euTaxationCustomsICSC} ... \LaTeX:\\ \url{https://ec.europa.eu/taxation_customs/dds2/SAMANCTA/EN/Safety/AppendixD_EN.htm}
%    \cite{euTaxationCustomsICSC}
%    \bibitem{iloWHOChemSafetyCards} ... \LaTeX:\\ \url{https://www.ilo.org/safework/info/publications/WCMS_113134/lang--en/index.htm}
%    \cite{iloWHOChemSafetyCards}
%    
%    
%    
%    Wat is er gebeurd?
%    @online{wikiSchipholbrand,	ALTauthor = {author},	ALTeditor = {editor},	title = {title},	date = {date},	url = {"https://nl.wikipedia.org/wiki/Schipholbrand"},}
%    artikel
%    @online{schipholbrand27102005video,	ALTauthor = {author},	ALTeditor = {editor},	title = {title},	date = {date},	url = {"https://www.youtube.com/watch?v=1i-hfEzxFfk"},}
%    psychologische gevolgen
%    rapport
%    @online{onderzoeksraad2610schipholoost,	ALTauthor = {author},	ALTeditor = {editor},	title = {title},	date = {date},	url = {"https://www.onderzoeksraad.nl/nl/page/392/brand-cellencomplex-schiphol-oost-nacht-van-26-op-27-oktober"},}
%    artikel met video
%    herdenking
%    impact op de persoon
%    herdenking
%    @online{schipholbrandvideoargos,	ALTauthor = {author},	ALTeditor = {editor},	title = {title},	date = {date},	url = {"https://www.vpro.nl/argos/speel~POMS_VPRO_461907~schadevergoeding-voor-ex-verdachte-schipholbrand~.html"},}
%    chronologie
%    @online{nunl30052023feitenoverzicht,	ALTauthor = {author},	ALTeditor = {editor},	title = {title},	date = {date},	url = {"https://www.nu.nl/binnenland/3355935/feitenoverzicht-schipholbrand-en-rechtszaken.html"},}
%    tijdlijn
%    @online{ID,	ALTauthor = {author},	ALTeditor = {editor},	title = {title},	date = {date},	url = {"https://www.singeluitgeverijen.nl/isbn/de-schipholbrand/"},}
%    vervolgens van ministers
%    beeldanalyse en reconstructie
%    @online{ID,	ALTauthor = {author},	ALTeditor = {editor},	title = {title},	date = {date},	url = {"https://eenvandaag.avrotros.nl/item/schipholbrand-niet-ontstaan-in-cel-11/"},}
%    herdenking
%    korte samenvatting
%    rapport
%    artikel
%    verwijzing naar het rapport vanuit de politieke oppositie
%    beeld vanuit de gevangenisbewaarder
%    nationaliteit slachtoffers schipholbrand
%    verblijfsvergunning voor de slachtoffers
%    gen schadevergoeding voor de verdachte
%    verdachte voor de rechter
%    geen schadevergoeding voor verdachte
%    artikel wat ging er mis bji de schipholbrand
%    brand veroorzaakt door een peuk
%    smaadschrift
%    bewakers worden niet vervolgd
%    proces schipholbrand moet over en de brandveilgheid moet worden verbeterd
%    de rol van het parlement in de evaluatie
%    @online{parlementairemonitorschipholbrand,	ALTauthor = {author},	ALTeditor = {editor},	title = {title},	date = {date},	url = {"https://www.parlementairemonitor.nl/9353000/1/j9vvij5epmj1ey0/vi3aof7awcxg"},}
%    onderzoeksmemo
%    herdenking
%    @online{ID,	ALTauthor = {author},	ALTeditor = {editor},	title = {title},	date = {date},	url = {"https://archief.ntr.nl/nova/page/detail/uitzendingen/3847/Den%20Haag%20Vandaag_%20herdenking%20Schipholbrand.html"},}
%    herdenking
%    invloed van de ramp op samenleving
%    @online{videonpoNOVA13112008,	ALTauthor = {author},	ALTeditor = {editor},	title = {title},	date = {date},	url = {"https://www.npostart.nl/heropen-onderzoek-schipholbrand/13-11-2008/POMS_NTR_103332"},}
%    opmerkelijk rapport gestolen in de nasleep
%    @online{rizoomes01052014schipholbrand,	ALTauthor = {author},	ALTeditor = {editor},	title = {title},	date = {date},	url = {"https://www.rizoomes.nl/brandweer/brand-cellencomplex-schiphol/"},}
%    
%    
%    
%    publicaties
%    @online{heuvelkroesschipholbrandcamerabeelden,	ALTauthor = {author},	ALTeditor = {editor},	title = {title},	date = {date},	url = {"http://www.msnp.nl/downloads/Onderzoeksmemo%20beeldanalyse%20Schipholbrand%20prot.pdf"},}
%    Wat waren de regels destijds?
%    Waren de autoriteiten in staat om op tijd in te grijpen of om erger te voorkomen?
%    Wat is er gedaan om de veiligheid van illegalen en gevangenissbewaarders te verbeteren
%    
%    
%    
%    algemene vragen
%    oorzaken
%    \bibitem{gates18112020boeingcrisis} ... \LaTeX:\\ \url{https://www.seattletimes.com/business/boeing-aerospace/what-led-to-boeings-737-max-crisis-a-qa/}
%    \cite{gates18112020boeingcrisis}
%    \bibitem{boeing737maxsoftwareprobles} ... \LaTeX:\\ \url{https://www.schneier.com/blog/archives/2019/04/excellent_analy.html}
%    \cite{boeing737maxsoftwareprobles}
%    fout in de software
%    \bibitem{avetisov19032019boeingmalwarestate} ... \LaTeX:\\ \url{https://www.forbes.com/sites/georgeavetisov/2019/03/19/malware-at-30000-feet-what-the-737-max-says-about-the-state-of-airplane-software-security/?sh=4d26f7052a9e}
%    \cite{avetisov19032019boeingmalwarestate}
%    het nationaal veiligheidsbelang
%    \bibitem{thompson23112020nationalsecurityboeing} ... \LaTeX:\\ \url{https://www.forbes.com/sites/lorenthompson/2020/11/23/five-reasons-return-of-boeings-737-max-to-service-is-important-to-national-security/?sh=2128ea552018}
%    \cite{thompson23112020nationalsecurityboeing}
%    falend toezicht
%    \bibitem{gates21032019FAAControlSystem} ... \LaTeX:\\ \url{https://www.seattletimes.com/business/boeing-aerospace/failed-certification-faa-missed-safety-issues-in-the-737-max-system-implicated-in-the-lion-air-crash/}
%    \cite{gates21032019FAAControlSystem}
%    onderzoeksrapport
%    \bibitem{faa18112020boeingreview} ... \LaTeX:\\ \url{https://www.faa.gov/foia/electronic_reading_room/boeing_reading_room/media/737_RTS_Summary.pdf}
%    \cite{faa18112020boeingreview}
%    \bibitem{wiki737maxgroundings} ... \LaTeX:\\ \url{https://en.wikipedia.org/wiki/Boeing_737_MAX_groundings}
%    \cite{wiki737maxgroundings}
%    veiligheidsrisico's
%    menselijke fouten
%    \bibitem{campbell02052019boengcrashhumanerrors} ... \LaTeX:\\ \url{https://www.theverge.com/2019/5/2/18518176/boeing-737-max-crash-problems-human-error-mcas-faa}
%    \cite{campbell02052019boengcrashhumanerrors}
%    overzicht van crashes
%    \bibitem{hawkins22032019737maxairplanes} ... \LaTeX:\\ \url{https://www.theverge.com/2019/3/22/18275736/boeing-737-max-plane-crashes-grounded-problems-info-details-explained-reasons}
%    \cite{hawkins22032019737maxairplanes}
%    veiligheidsopmerking
%    \bibitem{thomas30082020737safest} ... \LaTeX:\\ \url{https://www.airlineratings.com/news/boeings-737-max-will-one-safest-aircraft-history/}
%    \cite{thomas30082020737safest}
%    aanpassingen
%    \bibitem{boeing737maxdisplay} ... \LaTeX:\\ \url{https://www.boeing.com/commercial/737max/737-max-software-updates.page}
%    \cite{boeing737maxdisplay}
%    waarschuwingen//output signalen
%    \bibitem{fehrm24112020737changes} ... \LaTeX:\\ \url{https://leehamnews.com/2020/11/24/boeing-737-max-changes-beyond-mcas/}
%    \cite{fehrm24112020737changes}
%    software gerelateerde fouten
%    \bibitem{travis18042019737maxsoftwaredevop} ... \LaTeX:\\ \url{https://spectrum.ieee.org/aerospace/aviation/how-the-boeing-737-max-disaster-looks-to-a-software-developer}
%    \cite{travis18042019737maxsoftwaredevop}
%    onderzoeksrapport
%    de rol van de publieke opinie
%    \bibitem{barnett05052019737maxcrisis} ... \LaTeX:\\ \url{https://pubsonline.informs.org/do/10.1287/orms.2019.05.05/full/}
%    \cite{barnett05052019737maxcrisis}
%    onderzoek van europese luchtvaart agentschap
%    \bibitem{easa27012021737maxsafereturn} ... \LaTeX:\\ \url{https://www.easa.europa.eu/newsroom-and-events/news/easa-declares-boeing-737-max-safe-return-service-europe}
%    \cite{easa27012021737maxsafereturn}
%    \veiligheidsvraagstuk
%    \bibitem{touitou11032019737tragedies} ... \LaTeX:\\ \url{https://phys.org/news/2019-03-boeing-max-safety-tragedies.html}
%    \cite{touitou11032019737tragedies}
%    artikel over sensoren
%    \bibitem{hemmerdinger02022021737maxdeliveries} ... \LaTeX:\\ \url{https://www.flightglobal.com/airframers/boeing-delays-737-max-10-deliveries-two-years-to-2023/142245.article}
%    \cite{hemmerdinger02022021737maxdeliveries}
%    goedkeuring van europese luchtvaart autoriteiten
%    advies aan de faa
%    \bibitem{bielby27022021faaimprovesafety} ... \LaTeX:\\ \url{https://www.hstoday.us/subject-matter-areas/airport-aviation-security/oig-tells-faa-to-improve-safety-oversight-following-boeing-737-max-review/}
%    \cite{bielby27022021faaimprovesafety}
%    \bibitem{boyle18112020737maxupgrade} ... \LaTeX:\\ \url{https://www.geekwire.com/2020/faas-go-ahead-737-maxs-return-flight-kicks-off-massive-software-upgrade/}
%    \cite{boyle18112020737maxupgrade}
%    \bibitem{bergstraburgess122019737maxMcasAlgorithm} ... \LaTeX:\\ \url{https://www.researchgate.net/publication/338420944_A_Promise_Theoretic_Account_of_the_Boeing_737_Max_MCAS_Algorithm_Affair}
%    \cite{bergstraburgess122019737maxMcasAlgorithm}
%    achtergrond informatie
%    \bibitem{737mcas} ... \LaTeX:\\ \url{http://www.b737.org.uk/mcas.htm}
%    \cite{737mcas}
%    algemeen vertrouwen
%    \bibitem{newburger17052019boeingcrisis} ... \LaTeX:\\ \url{https://www.cnbc.com/2019/05/16/what-you-need-to-know-about-boeings-737-max-crisis.html}
%    \cite{newburger17052019boeingcrisis}
%    toestemming europese autoriteiten
%    problemen
%    \bibitem{arstechnica22012020737problems} ... \LaTeX:\\ \url{https://arstechnica.com/information-technology/2020/01/737-max-fix-slips-to-summer-and-thats-just-one-of-boeings-problems/}
%    \cite{arstechnica22012020737problems}
%    uitgebreid artikel over de onderzoeken en het vliegverbod
%    \bibitem{german190620217372yaftergrounded} ... \LaTeX:\\ \url{https://www.cnet.com/news/boeing-737-max-8-all-about-the-aircraft-flight-ban-and-investigations/}
%    \cite{german190620217372yaftergrounded}
%    computers als oorzaak
%    lessons learned
%    \bibitem{beningo02052019boeinglessons} ... \LaTeX:\\ \url{https://www.designnews.com/electronics-test/5-lessons-learn-boeing-737-max-fiasco}
%    \cite{beningo02052019boeinglessons}
%    \bibitem{duran05042019boeingspof} ... \LaTeX:\\ \url{https://www.eurocontrol.int/publication/effects-network-extra-standby-aircraft-and-boeing-737-max-grounding}
%    \cite{duran05042019boeingspof}
%    single point of failure
%    \bibitem{ID} ... \LaTeX:\\ \url{https://dmd.solutions/blog/2019/04/05/how-a-single-point-of-failure-spof-in-the-mcas-software-could-have-caused-the-boeing-737-max-crash-in-ethiopia/}
%    \cite{}
%    \bibitem{makichuck24012021737fearflying} ... \LaTeX:\\ \url{https://asiatimes.com/2021/01/boeings-737-max-and-the-fear-of-flying/}
%    \cite{makichuck24012021737fearflying}
%    lijst van tehnische aanpassingen
%    \bibitem{caa737modifications} ... \LaTeX:\\ \url{https://www.caa.co.uk/Consumers/Guide-to-aviation/Boeing-737-MAX/}
%    \cite{caa737modifications}
%    \bibitem{oestergaard14122020boeingdeliveries} ... \LaTeX:\\ \url{https://dsm.forecastinternational.com/wordpress/2020/12/14/airbus-and-boeing-report-november-2020-commercial-aircraft-orders-and-deliveries/}
%    \cite{oestergaard14122020boeingdeliveries}
%    code lek
%    \bibitem{reenberg787flaws} ... \LaTeX:\\ \url{https://www.wired.com/story/boeing-787-code-leak-security-flaws/}
%    \cite{reenberg787flaws}
%    \bibitem{fitch16092020737backlogrisks} ... \LaTeX:\\ \url{https://www.fitchratings.com/research/corporate-finance/boeing-737-max-return-backlog-risks-remain-16-09-2020}
%    \cite{fitch16092020737backlogrisks}
%    Cultuurverandering, deregulatie, systeemwijziging of gewoon een kwestie van competentie
%    \bibitem{willis27082020737maxfailures} ... \LaTeX:\\ \url{https://www.aerospacetestinginternational.com/features/what-broke-the-737-max.html}
%    \cite{willis27082020737maxfailures}
%    extra aanpassingen
%    \bibitem{ostrower11062020more737changes} ... \LaTeX:\\ \url{https://theaircurrent.com/aviation-safety/boeings-737-max-software-done-but-regulators-plot-more-changes-after-jets-return/}
%    \cite{ostrower11062020more737changes}
%    wat ging er mis een analyse van een ex-iloot
%    De utoriteiten waren op de hoogte
%    \bibitem{hruska13122019faaknown737crashrate} ... \LaTeX:\\ \url{https://www.extremetech.com/extreme/303373-the-faa-knew-the-737-max-was-dangerous-and-kept-it-flying-anyway}
%    \cite{hruska13122019faaknown737crashrate}
%    kwaliteiten van het alarmsysteem niet goed bekend
%    \bibitem{bloomberg26092019failedpred} ... \LaTeX:\\ \url{https://time.com/5687473/boeing-737-alarm-system/}
%    \cite{bloomberg26092019failedpred}
%    \bibitem{whiteman09072020boengcancelstock} ... \LaTeX:\\ \url{https://www.nasdaq.com/articles/boeing-gets-dealt-another-737-max-cancellation-blow.-what-it-means-for-boeing-stock-2020}
%    \cite{whiteman09072020boengcancelstock}
%    \bibitem{leopold09192019boeingreliability} ... \LaTeX:\\ \url{https://www.eetimes.com/boeing-crashes-highlight-a-worsening-reliability-crisis/}
%    \cite{leopold09192019boeingreliability}
%    veiligheidsvraagstuk
%    \bibitem{koenig11122019737crashesnofix} ... \LaTeX:\\ \url{https://www.latimes.com/business/story/2019-12-11/faa-boeing-737-max-crashes}
%    \cite{koenig11122019737crashesnofix}
%    probleemanalyse, veiligheidsvraagstuk
%    \bibitem{dohertylindeman15032019737problems} ... \LaTeX:\\ \url{https://www.politico.com/story/2019/03/15/boeing-737-max-grounding-1223072}
%    \cite{dohertylindeman15032019737problems}
%    falend toezicht
%    \bibitem{stodder02102019corruptoversight} ... \LaTeX:\\ \url{https://www.pogo.org/analysis/2019/10/corrupted-oversight-the-faa-boeing-and-the-737-max/}
%    \cite{stodder02102019corruptoversight}
%    \bibitem{afacwaLostSafeguards} ... \LaTeX:\\ \url{https://www.afacwa.org/the_inside_story_of_mcas_seattle_times}
%    \cite{afacwaLostSafeguards}
%    doelstellingen en veiligheidsvraagstukken
%    \bibitem{swayne18032019profitssafety} ... \LaTeX:\\ \url{https://www.marxist.com/737-max-scandal-boeing-putting-profits-before-safety.htm}
%    \cite{swayne18032019profitssafety}
%    \bibitem{freed26022021liftaustraliaban} ... \LaTeX:\\ \url{https://finance.yahoo.com/news/australia-lifts-ban-boeing-737-035817682.html?guccounter=1&guce_referrer=aHR0cHM6Ly93d3cuZ29vZ2xlLmNvbS8&guce_referrer_sig=AQAAAHZCJYy_0A5VS2WiPoCvH4xdrRNkmkdsv5EWJ2RLIz_AS-rxsTty6AF1_HlmJiRyWYqCXDi4p0Xs4isYkNkCq2Pfo-pQ60Xz_IfTNjm4FgoZiBMC4zpZlB6F0fwecrjE_ujAXZzG4xPJnWCd8-G3VLlPTY8h3H31eQ1i8hY9AIyy}
%    \cite{freed26022021liftaustraliaban}
%    autoriteiten krijgen tik op de vingers
%    \bibitem{reed15032019softwareattention} ... \LaTeX:\\ \url{https://medium.com/@jpaulreed/the-737max-and-why-software-engineers-should-pay-attention-a041290994bd}
%    \cite{reed15032019softwareattention}
%    \bibitem{news17032019softwareexplains} ... \LaTeX:\\ \url{https://news.ycombinator.com/item?id=19414775}
%    \cite{news17032019softwareexplains}
%    \bibitem{legget21122020eu737maxsafe} ... \LaTeX:\\ \url{https://www.bbc.com/news/55366320}
%    \cite{legget21122020eu737maxsafe}
%    \bibitem{marketscreener0103221737chinarecertification} ... \LaTeX:\\ \url{https://www.marketscreener.com/news/latest/China-studies-Boeing-737-MAX-recertification-wants-safety-concerns-fully-addressed--32569125/}
%    \cite{marketscreener0103221737chinarecertification}
%    motor in brand
%    \bibitem{euractiv22022021737firegrounds} ... \LaTeX:\\ \url{https://www.euractiv.com/section/aviation/news/boeing-grounds-777s-after-engine-fire/}
%    \cite{euractiv22022021737firegrounds}
%    \bibitem{benny18022019737returnUAE} ... \LaTeX:\\ \url{https://gulfnews.com/business/aviation/uae-airspace-to-see-return-of-boeing-737-max-1.1613627548923}
%    \cite{benny18022019737returnUAE}
%    motor in brand gevlogen
%    \bibitem{biersmichel22022021777grounds} ... \LaTeX:\\ \url{https://techxplore.com/news/2021-02-boeing-urges-grounding-777s.html}
%    \cite{biersmichel22022021777grounds}
%    \bibitem{ID} ... \LaTeX:\\ \url{https://www.politico.eu/article/uk-temporarily-bans-some-boeing-aircraft-after-pratt-whitney-engine-incidents/}
%    \cite{}
%    \bibitem{reuters23022021777metalfatigue} ... \LaTeX:\\ \url{https://www.timeslive.co.za/news/world/2021-02-23-damage-to-united-boeing-777-engine-consistent-with-metal-fatigue--ntsb/}
%    \cite{reuters23022021777metalfatigue}
%    faa was niet kritisch genoeg
%    \bibitem{ID} ... \LaTeX:\\ \url{https://federalnewsnetwork.com/government-news/2021/02/federal-watchdog-blasts-faa-over-certification-of-boeing-jet/}
%    \cite{}
%    
%    
%    
%    algemene vragen
%    oorzaken
%    \bibitem{{gates18112020boeingcrisis} ... \LaTeX:\\ \url{https://www.seattletimes.com/business/boeing-aerospace/what-led-to-boeings-737-max-crisis-a-qa/}
%    	
%    	\bibitem{{boeing737maxsoftwareprobles} ... \LaTeX:\\ \url{https://www.schneier.com/blog/archives/2019/04/excellent_analy.html}
%    		fout in de software
%    		\bibitem{{avetisov19032019boeingmalwarestate} ... \LaTeX:\\ \url{https://www.forbes.com/sites/georgeavetisov/2019/03/19/malware-at-30000-feet-what-the-737-max-says-about-the-state-of-airplane-software-security/?sh=4d26f7052a9e}
%    			het nationaal veiligheidsbelang
%    			\bibitem{{thompson23112020nationalsecurityboeing} ... \LaTeX:\\ \url{https://www.forbes.com/sites/lorenthompson/2020/11/23/five-reasons-return-of-boeings-737-max-to-service-is-important-to-national-security/?sh=2128ea552018}
%    				falend toezicht
%    				\bibitem{{gates21032019FAAControlSystem} ... \LaTeX:\\ \url{https://www.seattletimes.com/business/boeing-aerospace/failed-certification-faa-missed-safety-issues-in-the-737-max-system-implicated-in-the-lion-air-crash/}
%    					onderzoeksrapport
%    					\bibitem{{faa18112020boeingreview} ... \LaTeX:\\ \url{https://www.faa.gov/foia/electronic_reading_room/boeing_reading_room/media/737_RTS_Summary.pdf}
%    						\bibitem{{wiki737maxgroundings} ... \LaTeX:\\ \url{https://en.wikipedia.org/wiki/Boeing_737_MAX_groundings}
%    							veiligheidsrisico's
%    							menselijke fouten
%    							\bibitem{{campbell02052019boengcrashhumanerrors} ... \LaTeX:\\ \url{https://www.theverge.com/2019/5/2/18518176/boeing-737-max-crash-problems-human-error-mcas-faa}
%    								overzicht van crashes
%    								\bibitem{{hawkins22032019737maxairplanes} ... \LaTeX:\\ \url{https://www.theverge.com/2019/3/22/18275736/boeing-737-max-plane-crashes-grounded-problems-info-details-explained-reasons}
%    									veiligheidsopmerking
%    									\bibitem{{thomas30082020737safest} ... \LaTeX:\\ \url{https://www.airlineratings.com/news/boeings-737-max-will-one-safest-aircraft-history/}
%    										aanpassingen
%    										\bibitem{{boeing737maxdisplay} ... \LaTeX:\\ \url{https://www.boeing.com/commercial/737max/737-max-software-updates.page}
%    											waarschuwingen//output signalen
%    											\bibitem{{fehrm24112020737changes} ... \LaTeX:\\ \url{https://leehamnews.com/2020/11/24/boeing-737-max-changes-beyond-mcas/}
%    												software gerelateerde fouten
%    												\bibitem{{travis18042019737maxsoftwaredevop} ... \LaTeX:\\ \url{https://spectrum.ieee.org/aerospace/aviation/how-the-boeing-737-max-disaster-looks-to-a-software-developer}
%    													onderzoeksrapport
%    													de rol van de publieke opinie
%    													\bibitem{{barnett05052019737maxcrisis} ... \LaTeX:\\ \url{https://pubsonline.informs.org/do/10.1287/orms.2019.05.05/full/}
%    														onderzoek van europese luchtvaart agentschap
%    														\bibitem{{easa27012021737maxsafereturn} ... \LaTeX:\\ \url{https://www.easa.europa.eu/newsroom-and-events/news/easa-declares-boeing-737-max-safe-return-service-europe}
%    															veiligheidsvraagstuk
%    															\bibitem{{touitou11032019737tragedies} ... \LaTeX:\\ \url{https://phys.org/news/2019-03-boeing-max-safety-tragedies.html}
%    																artikel over sensoren
%    																\bibitem{{hemmerdinger02022021737maxdeliveries} ... \LaTeX:\\ \url{https://www.flightglobal.com/airframers/boeing-delays-737-max-10-deliveries-two-years-to-2023/142245.article}
%    																	goedkeuring van europese luchtvaart autoriteiten
%    																	advies aan de faa
%    																	\bibitem{{bielby27022021faaimprovesafety} ... \LaTeX:\\ \url{https://www.hstoday.us/subject-matter-areas/airport-aviation-security/oig-tells-faa-to-improve-safety-oversight-following-boeing-737-max-review/}
%    																		\bibitem{{boyle18112020737maxupgrade} ... \LaTeX:\\ \url{https://www.geekwire.com/2020/faas-go-ahead-737-maxs-return-flight-kicks-off-massive-software-upgrade/}
%    																			\bibitem{{bergstraburgess122019737maxMcasAlgorithm} ... \LaTeX:\\ \url{https://www.researchgate.net/publication/338420944_A_Promise_Theoretic_Account_of_the_Boeing_737_Max_MCAS_Algorithm_Affair}
%    																				achtergrond informatie
%    																				\bibitem{{737mcas} ... \LaTeX:\\ \url{http://www.b737.org.uk/mcas.htm}
%    																					algemeen vertrouwen
%    																					\bibitem{{newburger17052019boeingcrisis} ... \LaTeX:\\ \url{https://www.cnbc.com/2019/05/16/what-you-need-to-know-about-boeings-737-max-crisis.html}
%    																						toestemming europese autoriteiten
%    																						problemen
%    																						\bibitem{{arstechnica22012020737problems} ... \LaTeX:\\ \url{https://arstechnica.com/information-technology/2020/01/737-max-fix-slips-to-summer-and-thats-just-one-of-boeings-problems/}
%    																							uitgebreid artikel over de onderzoeken en het vliegverbod
%    																							\bibitem{{german190620217372yaftergrounded} ... \LaTeX:\\ \url{https://www.cnet.com/news/boeing-737-max-8-all-about-the-aircraft-flight-ban-and-investigations/}
%    																								computers als oorzaak
%    																								lessons learned
%    																								\bibitem{{beningo02052019boeinglessons} ... \LaTeX:\\ \url{https://www.designnews.com/electronics-test/5-lessons-learn-boeing-737-max-fiasco}
%    																									\bibitem{{duran05042019boeingspof} ... \LaTeX:\\ \url{https://www.eurocontrol.int/publication/effects-network-extra-standby-aircraft-and-boeing-737-max-grounding}
%    																										single point of failure
%    																										\bibitem{{ID} ... \LaTeX:\\ \url{https://dmd.solutions/blog/2019/04/05/how-a-single-point-of-failure-spof-in-the-mcas-software-could-have-caused-the-boeing-737-max-crash-in-ethiopia/}
%    																											\bibitem{{makichuck24012021737fearflying} ... \LaTeX:\\ \url{https://asiatimes.com/2021/01/boeings-737-max-and-the-fear-of-flying/}
%    																												lijst van tehnische aanpassingen
%    																												\bibitem{{caa737modifications} ... \LaTeX:\\ \url{https://www.caa.co.uk/Consumers/Guide-to-aviation/Boeing-737-MAX/}
%    																													\bibitem{{oestergaard14122020boeingdeliveries} ... \LaTeX:\\ \url{https://dsm.forecastinternational.com/wordpress/2020/12/14/airbus-and-boeing-report-november-2020-commercial-aircraft-orders-and-deliveries/}
%    																														code lek
%    																														\bibitem{{reenberg787flaws} ... \LaTeX:\\ \url{https://www.wired.com/story/boeing-787-code-leak-security-flaws/}
%    																															\bibitem{{fitch16092020737backlogrisks} ... \LaTeX:\\ \url{https://www.fitchratings.com/research/corporate-finance/boeing-737-max-return-backlog-risks-remain-16-09-2020}
%    																																Cultuurverandering, deregulatie, systeemwijziging of gewoon een kwestie van competentie
%    																																\bibitem{{willis27082020737maxfailures} ... \LaTeX:\\ \url{https://www.aerospacetestinginternational.com/features/what-broke-the-737-max.html}
%    																																	extra aanpassingen
%    																																	\bibitem{{ostrower11062020more737changes} ... \LaTeX:\\ \url{https://theaircurrent.com/aviation-safety/boeings-737-max-software-done-but-regulators-plot-more-changes-after-jets-return/}
%    																																		wat ging er mis een analyse van een ex-iloot
%    																																		De utoriteiten waren op de hoogte
%    																																		\bibitem{{hruska13122019faaknown737crashrate} ... \LaTeX:\\ \url{https://www.extremetech.com/extreme/303373-the-faa-knew-the-737-max-was-dangerous-and-kept-it-flying-anyway}
%    																																			kwaliteiten van het alarmsysteem niet goed bekend
%    																																			\bibitem{{bloomberg26092019failedpred} ... \LaTeX:\\ \url{https://time.com/5687473/boeing-737-alarm-system/}
%    																																				\bibitem{{whiteman09072020boengcancelstock} ... \LaTeX:\\ \url{https://www.nasdaq.com/articles/boeing-gets-dealt-another-737-max-cancellation-blow.-what-it-means-for-boeing-stock-2020}
%    																																					\bibitem{{leopold09192019boeingreliability} ... \LaTeX:\\ \url{https://www.eetimes.com/boeing-crashes-highlight-a-worsening-reliability-crisis/}
%    																																						veiligheidsvraagstuk
%    																																						\bibitem{{koenig11122019737crashesnofix} ... \LaTeX:\\ \url{https://www.latimes.com/business/story/2019-12-11/faa-boeing-737-max-crashes}
%    																																							probleemanalyse, veiligheidsvraagstuk
%    																																							\bibitem{{dohertylindeman15032019737problems} ... \LaTeX:\\ \url{https://www.politico.com/story/2019/03/15/boeing-737-max-grounding-1223072}
%    																																								falend toezicht
%    																																								\bibitem{{stodder02102019corruptoversight} ... \LaTeX:\\ \url{https://www.pogo.org/analysis/2019/10/corrupted-oversight-the-faa-boeing-and-the-737-max/}
%    																																									\bibitem{{afacwaLostSafeguards} ... \LaTeX:\\ \url{https://www.afacwa.org/the_inside_story_of_mcas_seattle_times}
%    																																										doelstellingen en veiligheidsvraagstukken
%    																																										\bibitem{{swayne18032019profitssafety} ... \LaTeX:\\ \url{https://www.marxist.com/737-max-scandal-boeing-putting-profits-before-safety.htm}
%    																																											\bibitem{{freed26022021liftaustraliaban} ... \LaTeX:\\ \url{https://finance.yahoo.com/news/australia-lifts-ban-boeing-737-035817682.html?guccounter=1&guce_referrer=aHR0cHM6Ly93d3cuZ29vZ2xlLmNvbS8&guce_referrer_sig=AQAAAHZCJYy_0A5VS2WiPoCvH4xdrRNkmkdsv5EWJ2RLIz_AS-rxsTty6AF1_HlmJiRyWYqCXDi4p0Xs4isYkNkCq2Pfo-pQ60Xz_IfTNjm4FgoZiBMC4zpZlB6F0fwecrjE_ujAXZzG4xPJnWCd8-G3VLlPTY8h3H31eQ1i8hY9AIyy}
%    																																												autoriteiten krijgen tik op de vingers
%    																																												\bibitem{{reed15032019softwareattention} ... \LaTeX:\\ \url{https://medium.com/@jpaulreed/the-737max-and-why-software-engineers-should-pay-attention-a041290994bd}
%    																																													\bibitem{{news17032019softwareexplains} ... \LaTeX:\\ \url{https://news.ycombinator.com/item?id=19414775}
%    																																														\bibitem{{legget21122020eu737maxsafe} ... \LaTeX:\\ \url{https://www.bbc.com/news/55366320}
%    																																															\bibitem{{marketscreener0103221737chinarecertification} ... \LaTeX:\\ \url{https://www.marketscreener.com/news/latest/China-studies-Boeing-737-MAX-recertification-wants-safety-concerns-fully-addressed--32569125/}
%    																																																motor in brand
%    																																																\bibitem{{euractiv22022021737firegrounds} ... \LaTeX:\\ \url{https://www.euractiv.com/section/aviation/news/boeing-grounds-777s-after-engine-fire/}
%    																																																	\bibitem{{benny18022019737returnUAE} ... \LaTeX:\\ \url{https://gulfnews.com/business/aviation/uae-airspace-to-see-return-of-boeing-737-max-1.1613627548923}
%    																																																		motor in brand gevlogen
%    																																																		\bibitem{{biersmichel22022021777grounds} ... \LaTeX:\\ \url{https://techxplore.com/news/2021-02-boeing-urges-grounding-777s.html}
%    																																																			\bibitem{{ID} ... \LaTeX:\\ \url{https://www.politico.eu/article/uk-temporarily-bans-some-boeing-aircraft-after-pratt-whitney-engine-incidents/}
%    																																																				\bibitem{{reuters23022021777metalfatigue} ... \LaTeX:\\ \url{https://www.timeslive.co.za/news/world/2021-02-23-damage-to-united-boeing-777-engine-consistent-with-metal-fatigue--ntsb/}
%    																																																					faa was niet kritisch genoeg
%    																																																					\bibitem{{ID} ... \LaTeX:\\ \url{https://federalnewsnetwork.com/government-news/2021/02/federal-watchdog-blasts-faa-over-certification-of-boeing-jet/}
%    																																																						\cite{}
    																																																						
  
\end{thebibliography}

%
%\newpage
%\bibliography{references}
%

\vfill
}}   %(verplicht) bronvermeldingen
\else
  \iflanguage{dutch}{\def\bibname{Bronnen}}{\def\bibname{References}}
  
\begin{thebibliography}{99}

\bibitem{lam1994} Lamport L.: \emph{\LaTeX: A Document Preparation System}, Addison-Wesley, 1994 

\bibitem{Oos1996} Oostrum van P.: \emph{Handleiding \LaTeX}, Vakgroep
  Informatica, Universiteit Utrecht, 1998,\\ 
  \url{http://people.cs.uu.nl/piet/latexhnd.pdf}

\bibitem{wikibooks} Wikibooks \LaTeX:\\
  \url{http://nl.wikibooks.org/wiki/LaTeX}


\bibitem{oid} Wikibooks \LaTeX:\\
\url{https://www.waterkant.net/suriname/2023/05/29/milieuactivist-sleur-zeer-grote-onwaarheden-staan-in-cyanide-onderzoeksrapport/}

   


%%%%%%%%%%%%%%%%%%%%%%%%%%%%%%%%%%%%%%%%%%%%%%%%%%%%%%%%%%%%%%%%%
%
%
%misc{para,
%	Author = {{IEEE Referencing}},
%	Howpublished = {\url{http://libguides.bhtafe.edu.au/content.php?pid=88814&sid=660920}},
%	Keywords = {paraphrasing, plagariasm, citation, quotation},
%	Lastchecked = {19/04/2017},
%	Title = {{How to Cite\/Quote in Your Assignment}}}
%
%misc{ieee-write,
%	Author = {{IEEE Authorship Series}},
%	Howpublished = {\url{http://ieeeauthorcenter.ieee.org/wp-content/uploads/How-to-Write-for-Technical-Periodicals-and-Conferences.pdf}},
%	Institution = {{IEEE}},
%	Keywords = {writing, paragraph, sentences, ethics},
%	Lastchecked = {18/04/2017},
%	Title = {How to Write for Technical Periodicals \& Conferences}}
%
%
%
%@article{schon2017agile,
%	title={Agile Requirements Engineering: A systematic literature review},
%	author={Sch\"on, Eva-Maria and Thomaschewski, J\"org and Escalona, Maria Jose},
%	journal={Computer Standards \& Interfaces},
%	volume={49},
%	pages={79--91},
%	year={2017},
%	publisher={Elsevier}
%}
%
%@inproceedings{royce1987managing,
%	title={Managing the development of large software systems: concepts and techniques},
%	author={Royce, Winston W},
%	booktitle={Proceedings of the 9th international conference on Software Engineering},
%	pages={328--338},
%	year=1987,
%	organization={IEEE Computer Society Press}
%}
%
%@article{leveson1993investigation,
%	title={An investigation of the Therac-25 accidents},
%	author={Leveson, Nancy G and Turner, Clark S},
%	journal={IEEE computer},
%	volume={26},
%	number={7},
%	pages={18--41},
%	year={1993}
%}
%@book{modelchecking,
%	author = {Clarke,Jr., Edmund M. and Grumberg, Orna and Peled, Doron A.},
%	title = {Model Checking},
%	year = {1999},
%	isbn = {0-262-03270-8},
%	publisher = {MIT Press},
%	address = {Cambridge, MA, USA},
%}
%@inproceedings{nuseibeh2000requirements,
%	title={Requirements engineering: a roadmap},
%	author={Nuseibeh, Bashar and Easterbrook, Steve},
%	booktitle={Proceedings of the Conference on the Future of Software Engineering},
%	pages={35--46},
%	year={2000},
%	organization={ACM}
%}
%
%
%%%%%%%%%%%%%%%%%%%%%%%%%%%%%%%%%%%%%%%%%%%%%%%%%%%%%%%%%%%%%%%%%%
%
%@online{inriaStatsMoodCheck,
%	ALTauthor = {project.inria},
%	ALTeditor = {editor},
%	title = {statistical-model-checking},
%	date = {date},
%	url = {https://project.inria.fr/plasma-lab/statistical-model-checking/},
%}
%
%
%@online{buddeModelChecker,
%	ALTauthor = {Carlos E. Budde1, Pedro R. D’Argenio2,3,4, Arnd Hartmanns1(B) , and Sean Sedwards5},
%	ALTeditor = {editor},
%	title = {A Statistical Model Checker for Nondeterminism and Rare Events},
%	date = {date},
%	url = {https://ris.utwente.nl/ws/portalfiles/portal/28200786/A_statistical_model_checker.pdf},
%}
%
%
%@online{ID,
%	ALTauthor = {GUL AGHA and  },
%	ALTeditor = {KARL PALMSKOG},
%	title = {A Survey of Statistical Model Checking
%	},
%	date = {date},
%	url = {https://dl.acm.org/doi/10.1145/3158668},
%}
%
%
%
%@online{AGHASuervey,
%	ALTauthor = {GUL AGHA and KARL PALMSKOG,},
%	ALTeditor = {editor},
%	title = {A Survey of Statistical Model Checking},
%	date = {date},
%	url = {https://dl.acm.org/doi/pdf/10.1145/3158668},
%}
%%%%%%%%%%%%%%%%%%%%%%%%%%%%%%%%%%%%%%%%%%%%%%%%%%%%%%%%%%%%%%%%%%
%
%
%
%@online{bicker21102016automatiseringsparadox,
%	ALTauthor = {author},
%	ALTeditor = {editor},
%	title = {title},
%	date = {date},
%	url = {https://www.debicker.eu/de-automatiseringsparadox/},
%}
%@online{vseautoparadox,
%	ALTauthor = {author},
%	ALTeditor = {editor},
%	title = {title},
%	date = {date},
%	url = {https://vse.nl/de-paradox-van-de-industriele-automatisering/},
%}
%
%
%
%@online{blogxot21112016slimapparaat,
%	ALTauthor = {author},
%	ALTeditor = {editor},
%	title = {title},
%	date = {date},
%	url = {https://blog.xot.nl/2016/11/21/slimme-apparaten-maken-ons-dom-en-kwetsbaar/index.html},
%}
%
%@online{ID,
%	ALTauthor = {author},
%	ALTeditor = {editor},
%	title = {title},
%	date = {date},
%	url = {https://automatie-pma.com/nieuws/industriele-automatiseringsparadox},
%}
%%%%%%%%%%%%%%%%%%%%%%%%%%%%%%%%%%%%%%%%%%%%%%%%%%%%%%%%%%%%%%%%%%
%
%
%
%
%
%
%
%
%
%@online{aviationsafety04101992airplaneCrashBijlmer,
%	ALTauthor = {author},
%	ALTeditor = {editor},
%	title = {title},
%	date = {date},
%	url = {https://aviation-safety.net/database/record.php?id=19921004-2&lang=nl  },
%}
%
%
%
%@online{boogers092002RampenRegelsRichtlijnen,
%	ALTauthor = {author},
%	ALTeditor = {editor},
%	title = {title},
%	date = {date},
%	url = { https://www.researchgate.net/publication/254815008_Rampen_regels_richtlijnen},
%}
%
%
%@online{catsr25022009Boeing737AmsterdamCrash,
%	ALTauthor = {author},
%	ALTeditor = {editor},
%	title = {title},
%	date = {date},
%	url = {https://catsr.vse.gmu.edu/SYST460/TA1951_AccidentReport.pdf  },
%}
%
%@online{INSAVienna1992Chernobyl,
%	ALTauthor = {author},
%	ALTeditor = {editor},
%	title = {title},
%	date = {date},
%	url = {https://www-pub.iaea.org/MTCD/publications/PDF/Pub913e_web.pdf  },
%}
%
%
%
%
%aviationReport
%
%@online{oVVSchietongevalOssendrecht,
%	ALTauthor = {author},
%	ALTeditor = {editor},
%	title = {title},
%	date = {date},
%	url = {https://www.youtube.com/watch?v=6jmkDClGDHo  },
%}
%
%@online{molukseTreinkaping,
%	ALTauthor = {author},
%	ALTeditor = {editor},
%	title = {title},
%	date = {date},
%	url = {https://www.youtube.com/watch?v=h99Fe9XzzHI  },
%}
%
%@online{jiang16042019TanjinExplosion,
%	ALTauthor = {author},
%	ALTeditor = {editor},
%	title = {title},
%	date = {date},
%	url = {https://www.hindawi.com/journals/joph/2019/1360805/  },
%}
%
%
%@online{hrw03082021investigateBeirutBlast,
%	ALTauthor = {author},
%	ALTeditor = {editor},
%	title = {title},
%	date = {date},
%	url = { 	https://www.hrw.org/report/2021/08/03/they-killed-us-inside/investigation-august-4-beirut-blast },
%}
%
%@online{souaibyElHussein112020Beirutstory,
%	ALTauthor = {author},
%	ALTeditor = {editor},
%	title = {title},
%	date = {date},
%	url = { https://www.researchgate.net/publication/348325979_Beirut_Explosion_the_full_story },
%}
%
%@online{ifrc2020chemicalexplosionBeirutPort,
%	ALTauthor = {author},
%	ALTeditor = {editor},
%	title = {title},
%	date = {date},
%	url = { https://reliefweb.int/sites/reliefweb.int/files/resources/CaseStudy_BeirutExplosion_TechBioHazardsweb.pdf },
%}
%
%@online{caliskan09112013747boeingkalman,
%	ALTauthor = {author},
%	ALTeditor = {editor},
%	title = {title},
%	date = {date},
%	url = {https://www.hindawi.com/journals/ijae/2014/472395/  },
%}
%
%%%%%%%%%%%%%%%%%%%%%%%%%%%%%%%%%%%%%%%%%%%%%%%%%%%%%%%%%%%%%%%%%%
%persuasive technology 
%
%@online{humanTechpersuasiveTech,
%	ALTauthor = {author},
%	ALTeditor = {editor},
%	title = {title},
%	date = {date},
%	url = {https://www.humanetech.com/youth/persuasive-technology  },
%}
%
%
%
%
%@online{rezenfeld01012018persuasiveTecgHabits,
%	ALTauthor = {author},
%	ALTeditor = {editor},
%	title = {title},
%	date = {date},
%	url = {https://spectrum.ieee.org/how-persuasive-technology-can-change-your-habits  },
%}
%
%
%@online{aldenaini28042020persuasiveTechTrends,
%	ALTauthor = {author},
%	ALTeditor = {editor},
%	title = {title},
%	date = {date},
%	url = { https://www.frontiersin.org/articles/10.3389/frai.2020.00007/full },
%}
%
%@online{larson14062017persuasivetechmanipulates,
%	ALTauthor = {author},
%	ALTeditor = {editor},
%	title = {title},
%	date = {date},
%	url = {https://psmag.com/environment/captology-fogg-invisible-manipulative-power-persuasive-technology-81301  },
%}
%
%@online{tanzem22012022persuasivetechchanginglives,
%	ALTauthor = {author},
%	ALTeditor = {editor},
%	title = {title},
%	date = {date},
%	url = {https://www.makeuseof.com/what-is-persuasive-technology/  },
%}
%
%
%
%
%@online{tikkakuddonenpersuasiveTechnology,
%	ALTauthor = {author},
%	ALTeditor = {editor},
%	title = {title},
%	date = {date},
%	url = { https://cyberpsychology.eu/article/view/12270 },
%}
%
%@online{sprongken19032018CourtProcedureDigital,
%	ALTauthor = {author},
%	ALTeditor = {editor},
%	title = {title},
%	date = {date},
%	url = { https://www.njb.nl/blogs/a-court-with-no-face-and-no-place/ },
%}
%
%@online{PROCESREGLEMENTEcourt,
%	ALTauthor = {author},
%	ALTeditor = {editor},
%	title = {title},
%	date = {date},
%	url = { http://www.e-court.nl/wp-content/uploads/2018/03/Procesreglement-e-Court-2017_20180201.pdf},
%}
%
%
%@online{ovvMortierOngevalMaliVideo,
%	ALTauthor = {author},
%	ALTeditor = {editor},
%	title = {title},
%	date = {date},
%	url = {https://www.youtube.com/watch?v=PC2ekl4SaNA  },
%}
%
%
%
%
%@online{vtmGroep29032019waterwerkencyber,
%	ALTauthor = {author},
%	ALTeditor = {editor},
%	title = {title},
%	date = {date},
%	url = { https://www.vtmgroep.nl/blog/waterwerken-in-nederland-onvoldoende-beveiligd-tegen-cyberaanvallen},
%}
%
%
%
%
%
%@online{wesemannVeiligheidLandEnWater,
%	ALTauthor = {author},
%	ALTeditor = {editor},
%	title = {title},
%	date = {date},
%	url = {
%		http://www.wesemann.nl/nl/nieuws-en-pers/274-veiligheid-op-het-water-en-op-het-land.html },
%}
%
%
%
%

%%%%%%%%%%%%%%%%%%%%%%%%%%%%%%%%%%%%%%%%%%%%%%%%%%%%%%%%%%%%%%%%%
\bibitem{ISAC_SANS_Ukraine_DUC_18Mar2016} ... \LaTeX:\\ \url{https://www.nerc.com/_layouts/15/Nerc.404/CustomFileNotFound.aspx?requestUrl=https://www.nerc.com/pa/CI/ESISAC/Documents/E-ISAC_SANS_Ukraine_DUC_18Mar2016.pdf},}
\bibitem{zetter2016GridHack,	ALTauthor = {Kim Zetter},	ALTeditor = {editor},	title = {Inside the Cunning, Unprecedented Hack of Ukraine's Power Grid},	date = {date},	url = {https://www.wired.com/2016/03/inside-cunning-unprecedented-hack-ukraines-power-grid/},}
\bibitem{2015ukrainegridattack,	ALTauthor = {wikipedia},	ALTeditor = {editor},	title = {2015 Ukraine power grid hack},	date = {date},	url = {https://en.wikipedia.org/wiki/2015_Ukraine_power_grid_hack},}
\bibitem{greenberg2017Cyberwartestlab} ... \LaTeX:\\ \url{https://www.wired.com/story/russian-hackers-attack-ukraine/},}
\bibitem{boozallen2016lightwentout,	ALTauthor = {allen},	ALTeditor = {editor},	title = {Ukrain report when the lights went out},	date = {date},	url = {https://www.boozallen.com/content/dam/boozallen/documents/2016/09/ukraine-report-when-the-lights-went-out.pdf},}
\bibitem{finklejan2016UsBlamesRussianSandworm} ... \LaTeX:\\ \url{https://www.reuters.com/article/us-ukraine-cybersecurity-sandworm-idUSKBN0UM00N20160108},}
\bibitem{wired012016ukrainegrid} ... \LaTeX:\\ \url{https://www.wired.com/2016/01/everything-we-know-about-ukraines-power-plant-hack/},}
\bibitem{icsalert20072021cyberattackukraine} ... \LaTeX:\\ \url{https://www.us-cert.gov/ics/alerts/IR-ALERT-H-16-056-01},}
\bibitem{finkle08012016russiansandwormhackers} ... \LaTeX:\\ \url{https://www.reuters.com/article/us-ukraine-cybersecurity-sandworm/u-s-firm-blames-russian-sandworm-hackers-for-ukraine-outage-idUSKBN0UM00N20160108},}
\bibitem{zinets15022017ukrainechargesrussia} ... \LaTeX:\\ \url{https://www.reuters.com/article/us-ukraine-crisis-cyber-idUSKBN15U2CN},}
\bibitem{2014russiansandworm} ... \LaTeX:\\ \url{https://www.wired.com/2014/10/russian-sandworm-hack-isight/},}
\bibitem{mcelfresh2016cyberattackhowandwhy} ... \LaTeX:\\ \url{https://theconversation.com/cyberattack-on-ukraine-grid-heres-how-it-worked-and-perhaps-why-it-was-done-52802},}
\bibitem{desarnaud2017cyberattacks} ... \LaTeX:\\ \url{https://www.ifri.org/sites/default/files/atoms/files/desarnaud_cyber_attacks_energy_infrastructures_2017_2.pdf},}
\bibitem{osti2018historycontrolsystemIncidents} ... \LaTeX:\\ \url{https://www.osti.gov/biblio/1505628},}
\bibitem{Shahzad2014ScadaProtocolsPollingScenario} ... \LaTeX:\\ \url{https://scialert.net/fulltext/?doi=tasr.2014.396.405},}
\bibitem{caseli04112016intrusiondetectioncontrolsystem} ... \LaTeX:\\ \url{https://ris.utwente.nl/ws/files/6028066/3-s2_0-B9780128015957000227.pdf},}
\bibitem{rochascadatesting} ... \LaTeX:\\ \url{https://repositorio-aberto.up.pt/bitstream/10216/119066/2/315683.pdf},}
\bibitem{levalle2020FuzzingICSProtocols} ... \LaTeX:\\ \url{https://dreamlab.net/en/blog/post/fuzzing-ics-protocols/},}
\bibitem{resch31102019IEC62351secureCommunication} ... \LaTeX:\\ \url{http://www.connectivity4ir.co.uk/article/175490/IEC-62351--Secure-communication-in-the-energy-industry.aspx},}
\bibitem{slowik2019ReassasUkraine2016Attack} ... \LaTeX:\\ \url{https://www.dragos.com/wp-content/uploads/CRASHOVERRIDE.pdf},}
\bibitem{arrizabalaga2020surveyiiotProtocols} ... \LaTeX:\\ \url{https://dl.acm.org/doi/fullHtml/10.1145/3381038},}
\bibitem{yadav2020reviewScadaArchitecture} ... \LaTeX:\\ \url{https://arxiv.org/pdf/2001.02925.pdf},}
\bibitem{cis20072021crashoverridemalware} ... \LaTeX:\\ \url{https://www.us-cert.gov/ncas/alerts/TA17-163A},}
\bibitem{holappa2017threattoElectricityNetworks} ... \LaTeX:\\ \url{https://www.nixu.com/fi/node/53},}
\bibitem{shehod2016gridadvantageus} ... \LaTeX:\\ \url{http://web.mit.edu/smadnick/www/wp/2016-22.pdf},}
\bibitem{parkwalstorm11102017russiagridattack} ... \LaTeX:\\ \url{https://jsis.washington.edu/news/cyberattack-critical-infrastructure-russia-ukrainian-power-grid-attacks/},}
\bibitem{drago2017CrashOverride} ... \LaTeX:\\ \url{https://www.dragos.com/wp-content/uploads/CrashOverride-01.pdf},}
\bibitem{wikiindustroyer} ... \LaTeX:\\ \url{https://en.wikipedia.org/wiki/Industroyer},}
\bibitem{crashoverridenetwork} ... \LaTeX:\\ \url{https://en.wikipedia.org/wiki/Crash_Override_Network},}
\bibitem{slowikvb2018crashoverride} ... \LaTeX:\\ \url{https://www.virusbulletin.com/virusbulletin/2019/03/vb2018-paper-anatomy-attack-detecting-and-defeating-crashoverride/},}
\bibitem{njccicthreat08102017crashovverrideprofile} ... \LaTeX:\\ \url{https://www.cyber.nj.gov/threat-center/threat-profiles/ics-malware-variants/crashoverride},}
\bibitem{crashoverrideindustroyermalware} ... \LaTeX:\\ \url{https://www.webopedia.com/TERM/C/crashoverride-industroyer-malware.html},}
\bibitem{incibe23062017crashoverrideback} ... \LaTeX:\\ \url{https://www.incibe-cert.es/en/blog/crashoverride-malware-ics-back-again},}
\bibitem{industroyershortfact} ... \LaTeX:\\ \url{https://rhebo.com/en/service/glossar/industroyer-25114/},}
\bibitem{vijayan2017firstmalwareCausedOutage} ... \LaTeX:\\ \url{https://www.darkreading.com/threat-intelligence/first-malware-designed-solely-for-electric-grids-caused-2016-ukraine-outage/d/d-id/1329114},}
\bibitem{ferrante2017crashoverrideredflag} ... \LaTeX:\\ \url{https://www.powermag.com/why-crashoverride-is-a-red-flag-for-u-s-power-companies/},}
\bibitem{2017cashoverrideindustroyerAlto} ... \LaTeX:\\ \url{https://blog.paloaltonetworks.com/2017/06/crashoverrideindustroyer-protections-palo-alto-networks-customers/},}
\bibitem{spinner2018crashoverrideiot} ... \LaTeX:\\ \url{https://iiot-world.com/ics-security/cybersecurity/five-cybersecurity-experts-about-crashoverride-malware-main-dangers-and-lessons-for-iiot/},}
\bibitem{abb30062017crashoverridenotification} ... \LaTeX:\\ \url{https://search.abb.com/library/Download.aspx?DocumentID=9AKK107045A1003&amp;LanguageCode=en&amp;DocumentPartId=&amp;Action=Launch},}
\bibitem{blackhatusa2017} ... \LaTeX:\\ \url{https://www.blackhat.com/us-17/briefings/schedule/#industroyercrashoverride-zero-things-cool-about-a-threat-group-targeting-the-power-grid-6159},}
\bibitem{humanitcrashalert} ... \LaTeX:\\ \url{https://humanit.asia/ta17-163a/},}
\bibitem{sroberts2017CrashoerrideCronicles} ... \LaTeX:\\ \url{https://medium.com/@sroberts/the-crash-override-chronicles-overall-8389ef178fdf},}
\bibitem{parker2020industrialsystemsattack} ... \LaTeX:\\ \url{https://www.oilandgaseng.com/articles/the-most-infamous-cyber-attacks-on-industrial-systems/},}
\bibitem{kirk2017threatIndustrialControls} ... \LaTeX:\\ \url{https://www.bankinfosecurity.com/power-grid-malware-platform-threatens-industrial-controls-a-9987},}
\bibitem{chalfant2017USElectricGrid} ... \LaTeX:\\ \url{https://thehill.com/policy/cybersecurity/337877-crash-override-malware-heightens-fears-for-us-electric-grid},}
\bibitem{fbiWarningcrashOverride} ... \LaTeX:\\ \url{https://www.inguardians.com/dhs-fbi-warn-of-attacks-against-us-energy-manufacturing-companies-and-employees/},}
\bibitem{davies2017crashoveerideUkraineAttack} ... \LaTeX:\\ \url{https://rethinkresearch.biz/articles/industroyer-crashoverride-malware-behind-ukraine-utility-attack/},}
\bibitem{slowik2018securitySessions} ... \LaTeX:\\ \url{https://electricenergyonline.com/energy/magazine/1104/article/Security-Sessions-Combating-ICS-Threats.htm},}
\bibitem{ENCS2017crashoverridemodules} ... \LaTeX:\\ \url{https://www.smart-energy.com/regional-news/europe-uk/encs-crash-override-virus/},}
\bibitem{brocklehurst2017crashoverridegridtakedown} ... \LaTeX:\\ \url{https://isssource.com/crashoverride-designed-for-grid-takedown/},}
\bibitem{20170613_crashoverride} ... \LaTeX:\\ \url{https://gigazine.net/gsc_news/en/20170613-crashoverride/},}
@onlinec{leetaru2017crashoverridehomefront} ... \LaTeX:\\ \url{https://www.forbes.com/sites/kalevleetaru/2017/06/24/crash-override-and-how-cyberwarfare-is-bringing-conflict-to-the-homefront/#42eb8984277c},}
\bibitem{fti2017redflag} ... \LaTeX:\\ \url{https://fticybersecurity.com/2017-11/crashoverride-red-flag-u-s-power-companies/},}
\bibitem{27001academy2014Whitepaper} ... \LaTeX:\\ \url{https://info.advisera.com/hubfs/27001Academy/27001Academy_FreeDownloads/NL/Checklist_of_ISO_27001_Mandatory_Documentation_NL.pdf},}
\bibitem{2017info_isoiec27019} ... \LaTeX:\\ \url{https://webstore.iec.ch/preview/info_isoiec27019%7Bed1.0%7Den.pdf},}
\bibitem{iec623512023serseries} ... \LaTeX:\\ \url{https://webstore.iec.ch/publication/6912},}
\bibitem{iec623512011Withdrawn} ... \LaTeX:\\ \url{https://webstore.iec.ch/publication/6911},}
\bibitem{IEC62351sheet} ... \LaTeX:\\ \url{https://www.ipcomm.de/protocol/IEC62351/en/sheet.html},}
\bibitem{NISTOIR2014GuidelinesCyverSec} ... \LaTeX:\\ \url{https://nvlpubs.nist.gov/nistpubs/ir/2014/NIST.IR.7628r1.pdf},}
\bibitem{ukraiinnesandwormteam} ... \LaTeX:\\ \url{https://www.fireeye.com/blog/threat-research/2016/01/ukraine-and-sandworm-team.html/cybersecurity-audit-checklist. },}
\bibitem{politicsrussagridukraine} ... \LaTeX:\\ \url{https://edition.cnn.com/2016/02/11/politics/ukraine-power-grid-attack-russia-us/index.html/ukraine-sees-russian-hand-in-cyber-attacks-on-power-grid-idUSKCN0VL18E. },}
\bibitem{Whitehead2017ukrainepoweroutage,	ALTauthor = {David E. Whitehead, Kevin Owens, Dennis Gammel, and Jess Smith},	ALTeditor = {editor},	title = {title},	date = {March 21–23, 2017},	url = {https://na.eventscloud.com/file_uploads/aed4bc20e84d2839b83c18bcba7e2876_Owens1.pdf},}
\bibitem{zetter2017moreDangerousMalware} ... \LaTeX:\\ \url{https://www.vice.com/en/article/zmeyg8/ukraine-power-grid-malware-crashoverride-industroyer},}
\bibitem{icsSecurityRussianHacking} ... \LaTeX:\\ \url{https://www.wallix.com/blog/ics-security-russian-hacking},}
\bibitem{rocha2017cybersecyrityanalysisScada} ... \LaTeX:\\ \url{https://www.sans.org/blog/confirmation-of-a-coordinated-attack-on-the-ukrainian-power-grid/},}
\bibitem{cheah2008testingiec60870} ... \LaTeX:\\ \url{http://citeseerx.ist.psu.edu/viewdoc/download;jsessionid=0513EED48102FDAD1BD940260EF12B11?doi=10.1.1.548.7490&amp;rep=rep1&amp;type=pdf},}
\bibitem{virsec2017DeepDiveIndustroyer} ... \LaTeX:\\ \url{https://virsec.com/virsec-hack-analysis-deep-dive-into-industroyer-aka-crash-override/},}
\bibitem{fauri2017EncryptionICS} ... \LaTeX:\\ \url{https://www.win.tue.nl/~setalle/2017_fauri_encryption.pdf},}
\bibitem{2017IEC61850} ... \LaTeX:\\ \url{http://blog.nettedautomation.com/2017/},}
\bibitem{2017win32industroyer} ... \LaTeX:\\ \url{https://www.welivesecurity.com/wp-content/uploads/2017/06/Win32_Industroyer.pdf},}
\bibitem{dragos2019TargetedTransStation} ... \LaTeX:\\ \url{https://www.cybersecurityintelligence.com/blog/attack-on-ukraines-power-grid-targeted-transmission-stations-4530.html},}
\bibitem{2017crashoverridenostuxnet} ... \LaTeX:\\ \url{https://arstechnica.com/information-technology/2017/06/crash-override-malware-may-sabotage-electric-grids-but-its-no-stuxnet/},}
%%%%%%%%%%%%%%%%%%%%%%%%%%%%%%%%%%%%%%%%%%%%%%%%%%%%%%%%%%%%%%%%%



%
%\chapter{Deelonderzoek naar veiligheidsrisico's en eisen voor sluizen}
%
%Gevonden weblinks in google op 07-04-2023 met zoekopdracht: "veiligheidsrisico's voor sluizen en waterwerken"
%





\bibitem{ID} ... \LaTeX:\\ \url{https://www.tweedekamer.nl/downloads/document?id=80443e97-f17e-499c-b3f2-ad608f32e1aa&title=Rapportage%20Staat%20van%20de%20infra%20RWS%20%28definitief%29.pdf}
\bibitem{ID} ... \LaTeX:\\ \url{https://www.nu.nl/internet/5814282/rekenkamer-waterwerken-niet-goed-beveiligd-tegen-cyberaanvallen.html}
\bibitem{linne21122022onderhousluis} ... \LaTeX:\\ \url{https://www.deltalimburg.nl/article/9824/Onderhoudswerkzaamheden+aan+Sluis+Linne+afgerond}
\bibitem{sluisterneuzenveiligheid} ... \LaTeX:\\ \url{https://nieuwesluisterneuzen.eu/veiligheid}
\bibitem{mrdmarinesupport} ... \LaTeX:\\ \url{https://www.mrdmarinesupport.nl/nl/maritieme-dienstverlening/ondersteuning-veiligheid/}
\bibitem{krabbendam27052021infrasitrrisicobeoordeling} ... \LaTeX:\\ \url{https://www.infrasite.nl/bouwen/2021/05/27/veiligheid-voorop-begin-project-sluis-of-brug-altijd-met-risicobeoordeling/}
\bibitem{bedieningsluisVechterweerdVilsteren} ... \LaTeX:\\ \url{https://www.wdodelta.nl/bediening-schutsluizen-vechterweerd-en-vilsteren}
\bibitem{krabbedam21052021sluisHeel} ... \LaTeX:\\ \url{https://www.infrasite.nl/waterbouw-deltas/2021/05/21/sluis-heel-onder-handen-genomen/}
\bibitem{hdsr30092022lichtprojectieswaterliniesluizen} ... \LaTeX:\\ \url{https://www.hdsr.nl/actueel/nieuws/@154100/lichtprojecties-zetten-waterliniesluizen/}
\bibitem{nos28032019rekenkamerhackwaterwerk} ... \LaTeX:\\ \url{https://nos.nl/artikel/2277937-rekenkamer-hack-aanval-op-waterwerk-niet-altijd-opgemerkt}
\bibitem{sluisLinne08012023} ... \LaTeX:\\ \url{https://varendoejesamen.nl/kenniscentrum/artikel/onderhoud-sluis-linne-afgerond}
\bibitem{gww29032021kantelendesluisdeur} ... \LaTeX:\\ \url{https://www.gww-bouw.nl/artikel/de-eerste-sluis-met-kantelende-sluisdeur/}
\bibitem{tkh03022018sluisbeheerScadaZeeland} ... \LaTeX:\\ \url{https://tkhsecurity.com/nl/waterwerken/}
\bibitem{h2o28032019dreigingsniveaurekenkamer} ... \LaTeX:\\ \url{https://www.h2owaternetwerk.nl/h2o-actueel/rekenkamer-vitale-waterwerken-nog-onvoldoende-beschermd-tegen-cyberaanvallen}
\bibitem{krammersluizencomplex} ... \LaTeX:\\ \url{https://www.magazinesrijkswaterstaat.nl/bereikbaarzeeland/2021/01/krammersluizencomplex-verleden-heden-en-toekomst}



%\chapter{Deelonderzoek wet en regelgeving voor sluizen}
%
%
%
%Gevonden weblinks in google op 07-04-2023 met zoekopdracht: "wet en regelgeving voor sluizen"



\bibitem{ID} ... \LaTeX:\\ \url{https://www.hdsr.nl/publish/pages/86927/sluizen_in_of_bij_een_waterkering_-_uitvoeringsregels.pdf}
\bibitem{ID} ... \LaTeX:\\ \url{https://api1.ibabs.eu/publicdownload.aspx?site=sluis&id=100100292}
\bibitem{ID} ... \LaTeX:\\ \url{https://services.pilz.nl/wp-content/uploads/2021/12/brochure_bruggen_2018.pdf}
\bibitem{ID} ... \LaTeX:\\ \url{https://lokaleregelgeving.overheid.nl/CVDR375606/6}
\bibitem{ID} ... \LaTeX:\\ \url{https://zoek.officielebekendmakingen.nl/stb-2019-27.html}
\bibitem{ID} ... \LaTeX:\\ \url{https://a-quin.nl/nieuws/veiligheid-van-bruggen-sluizen-waarborgen-wie-wat-hoe/}
\bibitem{ID} ... \LaTeX:\\ \url{https://www.gemeentesluis.nl/Bestuur_en_Organisatie/Wetten_Regels_Bekendmakingen}
\bibitem{ID} ... \LaTeX:\\ \url{https://www.overijssel.nl/onderwerpen/verkeer-en-vervoer/varen-in-overijssel/informatie-bedieningstijden-sluizen-en-bruggen-noordwest-overijssel/}
\bibitem{ID} ... \LaTeX:\\ \url{https://www.rijkswaterstaat.nl/water/wetten-regels-en-vergunningen}
\bibitem{ID} ... \LaTeX:\\ \url{https://www.schuttevaer.nl/nieuws/actueel/2022/11/23/binnenvaart-zit-klem-tussen-regels-en-realiteit-kapotte-steigers-en-gesperde-sluizen-dwingen-tot-doorvaren/}
\bibitem{ID} ... \LaTeX:\\ \url{https://repository.officiele-overheidspublicaties.nl/CVDR/CVDR271406/1/html/CVDR271406_1.html}
\bibitem{ID} ... \LaTeX:\\ \url{https://www.zeeland.nl/actueel/bedieningstijden-sluizen-en-bruggen}
\bibitem{ID} ... \LaTeX:\\ \url{https://www.amsterdam.nl/verkeer-vervoer/varen-amsterdam/regels-varen/}
\bibitem{ID} ... \LaTeX:\\ \url{https://www.schielandendekrimpenerwaard.nl/wat-doen-we/regels-en-afspraken-over-beheer-keur-en-leggers/}
\bibitem{ID} ... \LaTeX:\\ \url{http://www.wetboek-online.nl/wet/Wet%20tot%20samenvoeging%20van%20de%20gemeenten%20Aardenburg%20en%20Sluis.html}
\bibitem{ID} ... \LaTeX:\\ \url{https://www.rijnland.net/regels-op-een-rij/richtlijnen-en-akkoorden/alle-regelgeving-van-rijnland/}
\bibitem{ID} ... \LaTeX:\\ \url{https://www.itbb.nl/diensten/advies-ce-markering-europese-richtlijnen/}
\bibitem{ID} ... \LaTeX:\\ \url{https://www.portofamsterdam.com/nl/scheepvaart/zeevaart/regelgeving}
\bibitem{ID} ... \LaTeX:\\ \url{https://www.watersportverbond.nl/nieuws/achterstallig-onderhoud-wachtplaatsen-bruggen-en-sluizen-zuid-holland-zorgelijk/}
\bibitem{ID} ... \LaTeX:\\ \url{https://varendoejesamen.nl/nieuws}
\bibitem{ID} ... \LaTeX:\\ \url{https://www.flevoland.nl/wat-doen-we/flevowegen-vlot-en-veilig-door-flevoland/water/varen-in-flevoland/bediening-bruggen-en-sluizen}
\bibitem{ID} ... \LaTeX:\\ \url{https://eur-lex.europa.eu/legal-content/NL/TXT/PDF/?uri=CELEX:32020L0012&from=DE}
\bibitem{ID} ... \LaTeX:\\ \url{https://www.werkenvoornederland.nl/organisatie/rijkswaterstaat/ict-middelen-maken-om-bruggen-sluizen-en-tunnels-te-besturen}
\bibitem{ID} ... \LaTeX:\\ \url{https://www.lobocom.nl/infra-bruggen-sluizen}
\bibitem{ID} ... \LaTeX:\\ \url{https://waterrecreatienederland.nl/content/uploads/2018/04/richtlijnen-vaarwegen-2017.pdf}
\bibitem{ID} ... \LaTeX:\\ \url{https://www.wetterskipfryslan.nl/melden-en-regelen/vergunningen-wetten-en-regels}
\bibitem{ID} ... \LaTeX:\\ \url{https://www.onlinezeilschool.nl/sluizen/}
\bibitem{ID} ... \LaTeX:\\ \url{https://www.provincie.drenthe.nl/onderwerpen/verkeer-vervoer/vaarwegen/rondje-drenthe/bedieningstijden/}





Rampen
In dit hoofdstuk worden de resultaten van een deskresearch naar verschillende rampen behandeld
Hierbij een verslag naar de oorzaken van de rampen, de werkwijze waarop het product is ontwikkeld, de verwerking van feedback, implementatie en nazorg.
Met behulp van het 4 variabelen model wordt duidelijk gemaakt hoe het systeem is opgezet en wat daarin verkeerd is gegaan.
Het  hoofdstuk wordt afgesloten met een analyse van algemene kenmerken van de verschillende rampen die zijn onderzocht.


Het 4 variabelen model kort toegelicht
Monitored variabelen: door sensoren gekwantificeerde fenomenen uit de omgeving, bijv temperatuur

Controlled variabelen: door actuatoren \bestuurde fenomenen uit de omgeving
For example, monitored variables might be the pressure and temperature
inside a nuclear reactor while controlled variables might be visual and audible alarms, as well
as the trip signal that initiates a reactor shutdown; whenever the temperature or pressure reach
abnormal values, the alarms go off and the shutdown procedure is initiated

Input variabelen: data die de software als input gebruikt
Here, IN models the input hardware interface (sensors and analog-to-digital converters) and
relates values of monitored variables to values of input variables in the software. The input variables model the information about the environment that is available to the software. For example,
IN might model a pressure sensor that converts temperature values to analog voltages; these voltages are then converted via an A/D converter to integer values stored in a register accesible to the
software.

Output variabelen: data die de software levert als output
The output hardware interface (digital-to-analog converters and actuators) is modelled
by OUT, which relates values of the output variables of the software to values of controlled variables. An output variable might be, for instance, a boolean variable set by the software with the
understanding that the value true indicates that a reactor shutdown should occur and the value
false indicates the opposite


Bronnen:
\bibitem{ID} ... \LaTeX:\\ \url{https://www.sciencedirect.com/science/article/pii/S0167642315001033}
\bibitem{ID} ... \LaTeX:\\ \url{https://www.cas.mcmaster.ca/~lawford/papers/AVoCS2013.pdf}
\bibitem{ID} ... \LaTeX:\\ \url{https://core.ac.uk/download/pdf/38891842.pdf}
%%%%%%%%%%%%%%%%%%%%%%%%%%%%%%%%%%%%%%%%%%%%%%%%%%%%%%%%%%%%%%%%%

Therac


sheets
\bibitem{ID} ... \LaTeX:\\ \url{https://web.cs.ucdavis.edu/~rogaway/classes/188/winter04/therac-25.pdf}
\bibitem{ID} ... \LaTeX:\\ \url{https://people.physics.carleton.ca/~drogers/egs_windows_collection/tsld008.htm}
\bibitem{ID} ... \LaTeX:\\ \url{https://en.wikipedia.org/wiki/Therac-25}
\bibitem{ID} ... \LaTeX:\\ \url{https://www.youtube.com/watch?v=-7gVqBY52MY}
reproduceren van de error. IN dit stuk wordt uitgelgd hoe het product werkt en waarom bepaalde beslssingen zijn genomen in de ontwerp/productiefase
\bibitem{ID} ... \LaTeX:\\ \url{https://www.bugsnag.com/blog/bug-day-race-condition-therac-25}
kort artikel met daarin een opsomming van alle fouten in het systeem en een korte uitleg
\bibitem{ID} ... \LaTeX:\\ \url{https://www.bowdoin.edu/~allen/courses/cs260/readings/therac.pdf}
uitgebreid artikel over hoe de fout werd gereproduceerd en de resultaten daaruit voortkwamen. Alsnog werden er na de reproductie fase nog meer fouten gevonden.
\bibitem{ID} ... \LaTeX:\\ \url{https://hackaday.com/2015/10/26/killed-by-a-machine-the-therac-25/}
artikel
\bibitem{ID} ... \LaTeX:\\ \url{https://ethicsunwrapped.utexas.edu/case-study/therac-25}
onderzoeksartikel waarin de bug wordt uitgelgd: de racecondities, de bytepositie en het testen worden berkitiseerd envenals andere onderdelen van het softwareproces
\bibitem{ID} ... \LaTeX:\\ \url{https://thedailywtf.com/articles/the-therac-25-incident}
onrealistisch testplan. In dit artikel egt de auteur het belang nog eens uit van goede requirements en implementatie, niet de software is waar het probleem ligt
\bibitem{ID} ... \LaTeX:\\ \url{https://www.computer.org/csdl/magazine/co/2017/11/mco2017110008/13rRUxAStVR}
onderzoeksrapport
\bibitem{ID} ... \LaTeX:\\ \url{https://web.stanford.edu/class/cs240/old/sp2014/readings/therac-25.pdf}
geschiedenis
\bibitem{ID} ... \LaTeX:\\ \url{http://computingcases.org/case_materials/therac/case_history/Case%20History.html}
artikel
\bibitem{ID} ... \LaTeX:\\ \url{https://medium.com/swlh/software-architecture-therac-25-the-killer-radiation-machine-8a05e0705d5b}
computer error. De ongeval en de malfunction nog een keer uitgelegd
\bibitem{ID} ... \LaTeX:\\ \url{http://www.ccnr.org/fatal_dose.html}
rapport
\bibitem{ID} ... \LaTeX:\\ \url{http://sunnyday.mit.edu/papers/therac.pdf}
\bibitem{ID} ... \LaTeX:\\ \url{https://pubmed.ncbi.nlm.nih.gov/101762/}
onderzoeksartkel
\bibitem{ID} ... \LaTeX:\\ \url{http://www1.cs.columbia.edu/~junfeng/08fa-e6998/sched/readings/therac25.pdf}
\bibitem{ID} ... \LaTeX:\\ \url{https://ieeexplore.ieee.org/document/274940}
uitgebreid artikel gaat hier ook wat meer over de hardware
\bibitem{ID} ... \LaTeX:\\ \url{https://www.linkedin.com/pulse/therac-25-industrial-design-engineering-systems-wang-ph-d-cre-acb/}
artikel waarin in 3 delen de problemaiekwordt blootgesteld
\bibitem{ID} ... \LaTeX:\\ \url{http://www.cse.msu.edu/~cse470/Public/Handouts/Therac/Therac_2.html}
case study sheets
\bibitem{ID} ... \LaTeX:\\ \url{https://www.cs.jhu.edu/~cis/cista/445/Lectures/Therac.pdf}
artikel waarin vooral de fabriikant ervan langs krijgt
\bibitem{ID} ... \LaTeX:\\ \url{http://users.csc.calpoly.edu/~jdalbey/SWE/Papers/THERAC25.html}
lessons learned. Vooral de begrippen betrouwbaarheid, welgevalligheid, veilgheid en gebruiksvriendelijkheid
\bibitem{ID} ... \LaTeX:\\ \url{https://bohr.wlu.ca/cp164/therac/therac25.htm}
root-cause analysis
\bibitem{ID} ... \LaTeX:\\ \url{https://root-cause-analysis.info/2010/08/08/therac-25-radiation-overdoses/}
case study
\bibitem{ID} ... \LaTeX:\\ \url{https://dusk.geo.orst.edu/ethics/papers/Therac.Huff.pdf}
\bibitem{ID} ... \LaTeX:\\ \url{https://embeddedartistry.com/fieldatlas/case-study-therac-25/}
case study
\bibitem{ID} ... \LaTeX:\\ \url{https://www.sebokwiki.org/wiki/Medical_Radiation}
opzetten van systematische acceptaatie test met therac als voorbeeld
\bibitem{ID} ... \LaTeX:\\ \url{https://www.sciencedirect.com/science/article/pii/S1474667017448245}
artikel waarin een diagnose plaatvindt voor het bedrijf en de ingenieur/ontwerper
\bibitem{ID} ... \LaTeX:\\ \url{https://magsilva.pro.br/apps/wiki/testing/Therac_25}
rapport
\bibitem{ID} ... \LaTeX:\\ \url{http://csel.eng.ohio-state.edu/productions/pexis/readings/submod3/therac.pdf}
oorzaken aangegeven in artikel
\bibitem{ID} ... \LaTeX:\\ \url{https://www.chemeurope.com/en/encyclopedia/Therac-25.html}
het onderzoek en enkele ontwerptekeningen en oplossingen
\bibitem{ID} ... \LaTeX:\\ \url{https://pvs-studio.com/en/blog/posts/0438/}
\bibitem{ID} ... \LaTeX:\\ \url{https://www.coursera.org/lecture/software-design-threats-mitigations/therac-25-case-study-VmQPa}
\bibitem{ID} ... \LaTeX:\\ \url{https://www.semanticscholar.org/paper/The-story-of-the-Therac-25-in-LOTOS-Thomas/6c9c6024cf95aadae8b7edf1160e0e4500410eb9}
\bibitem{ID} ... \LaTeX:\\ \url{https://news.ycombinator.com/item?id=21679287}
wiki
\bibitem{ID} ... \LaTeX:\\ \url{https://en.wikibooks.org/wiki/Professionalism/Therac-25}
analyse
\bibitem{ID} ... \LaTeX:\\ \url{https://citeseerx.ist.psu.edu/viewdoc/download?doi=10.1.1.96.369&rep=rep1&type=pdf}
samenvatting
\bibitem{ID} ... \LaTeX:\\ \url{https://onlineethics.org/cases/resources-engineering-and-science-ethics/investigation-therac-25-accidents-abstract}
podcast
\bibitem{ID} ... \LaTeX:\\ \url{https://podcasts.apple.com/gb/podcast/therac-25/id1046978749?i=1000514115050}
enkele conclusies
\bibitem{ID} ... \LaTeX:\\ \url{http://www.cas.mcmaster.ca/~se4d03/therac.html}
rapport over de fouten die de verschillende partijen hebben gemaakt( overheid, ingenieurs, bedrijf, operators) en de verbeterpunten
\bibitem{ID} ... \LaTeX:\\ \url{https://www.cs.colostate.edu/~bieman/CS314/Notes/therac25.pdf}
onderzoeksrapport
\bibitem{ID} ... \LaTeX:\\ \url{https://www.cs.ucf.edu/~dcm/Teaching/COP4600-Fall2010/Literature/Therac25-Leveson.pdf}
slides online over het technisch mankement
Wat is er gebeurd, nou het volgende:
Normal radiation treatments: 6,000 rads over a 3 week period, under certain conditions Therac-25 was delivering 60,000 rads during one session.
En wat ging er mis?
Paradigm Shift

Therac-25 replaced expensive hardware safety interlocks with software controls

Real-time software
Design
Race condition caused focusing element to be incorrectly set
No indication of actual hardware settings
Error messages appeared the same regardless of how important
Error messages were difficult to understand
All errors messages could be manually overridden

\bibitem{ID} ... \LaTeX:\\ \url{https://hci.cs.siue.edu/NSF/Files/Semester/Week13-2/PPT-Text/Slide13.html}
oorzaak-gevolg diagram
\bibitem{ID} ... \LaTeX:\\ \url{https://www.thinkreliability.com/InstructorBlogs/Blog-Therac-25.pdf}
veiligheidsanalyse naar de rapportage van foutmeldingen, de beslissingsmatrix waarmee het programma wordt uitgevoerd en de software-analyse door een consultat
\bibitem{ID} ... \LaTeX:\\ \url{https://www.erenkrantz.com/Geeks/Therac-25/Side_bar_5.html}
\bibitem{ID} ... \LaTeX:\\ \url{https://bandcamp.com/therac-25}
\bibitem{ID} ... \LaTeX:\\ \url{https://sites.nd.edu/brent-marin/2017/09/21/blog-post-8-therac-25-accidents/}
\bibitem{ID} ... \LaTeX:\\ \url{https://europepmc.org/article/med/4035720}
\bibitem{ID} ... \LaTeX:\\ \url{https://itlaw.wikia.org/wiki/Therac-25}
\bibitem{ID} ... \LaTeX:\\ \url{https://www.sitepoint.com/therac-25-bad-software-kills/}
\bibitem{ID} ... \LaTeX:\\ \url{https://sqa.stackexchange.com/questions/9798/asking-for-help-with-this-therac-25-bugged-code-i-dont-understand-the-explanat}
\bibitem{ID} ... \LaTeX:\\ \url{http://www.se.rit.edu/~swen-342/activities/TheracIndividual.html}
\bibitem{ID} ... \LaTeX:\\ \url{https://www.designnews.com/automation-motion-control/engineering-disasters-deadly-zaps-therac-25}

publicaties over de therac:
\bibitem{ID} ... \LaTeX:\\ \url{https://www.cs.colostate.edu/~bieman/CS314/Notes/therac25.pdf}
\bibitem{ID} ... \LaTeX:\\ \url{https://www.cs.ucf.edu/~dcm/Teaching/COP4600-Fall2010/Literature/Therac25-Leveson.pdf}
\bibitem{ID} ... \LaTeX:\\ \url{https://onlineethics.org/cases/resources-engineering-and-science-ethics/investigation-therac-25-accidents-abstract}
\bibitem{ID} ... \LaTeX:\\ \url{https://citeseerx.ist.psu.edu/viewdoc/download?doi=10.1.1.96.369&rep=rep1&type=pdf}
\bibitem{ID} ... \LaTeX:\\ \url{http://csel.eng.ohio-state.edu/productions/pexis/readings/submod3/therac.pdf}
\bibitem{ID} ... \LaTeX:\\ \url{https://dusk.geo.orst.edu/ethics/papers/Therac.Huff.pdf}
\bibitem{ID} ... \LaTeX:\\ \url{https://www.cs.jhu.edu/~cis/cista/445/Lectures/Therac.pdf}
\bibitem{ID} ... \LaTeX:\\ \url{http://www1.cs.columbia.edu/~junfeng/08fa-e6998/sched/readings/therac25.pdf}
\bibitem{ID} ... \LaTeX:\\ \url{http://sunnyday.mit.edu/papers/therac.pdf}
\bibitem{ID} ... \LaTeX:\\ \url{https://web.stanford.edu/class/cs240/old/sp2014/readings/therac-25.pdf}




%%%%%%%%%%%%%%%%%%%%%%%%%%%%%%%%%%%%%%%%%%%%%%%%%%%%%%%%%%%%%%%%%
Krakend zorgssteem door covid-19 in suriname

vaccinatieterkort
communicatie met bevolking
communicatie met binnenland
testen van vaccinaties
besmetting vanuit eht buitenland
isolatie na vakantie en voor toeristen
tekort aan ic-personeel
tekort aan ic-bedden
tekort aan zuurtstof
tekort aan middelen}

Wat blijkt hieruit:
de impact van de crisis wereldwijd
de afhnakelijkheid van landen op goede samenwerking
Nut en noodzaak van regelgeving
Naveling van maatregels
Communicatie over beleid vanuit de overheid naar de burgers
Belang van een verzorgingstaat
Een wetenschappelijke ontwikkeling die kan inspelen op gevoelige 'trends
De impact van een lockdown op de economie
Afschaling van andere noodzakelijke no-covid zorg
De bereikbaarheid van een ziekenhuis
Waar heeft het toe geleid?
\bibitem{ID} ... \LaTeX:\\ \url{https://www.waterkant.net/suriname/2007/02/06/school-in-suriname-gesloten-om-zenuwgasvoorraad/}
\bibitem{ID} ... \LaTeX:\\ \url{https://nl.wikipedia.org/wiki/Nationaal_Co%C3%B6rdinatiecentrum_voor_Rampenbeheersing}
\bibitem{ID} ... \LaTeX:\\ \url{https://www.examenkamer.nl/index.php/27-vca-examens-in-suriname}
%%%%%%%%%%%%%%%%%%%%%%%%%%%%%%%%%%%%%%%%%%%%%%%%%%%%%%%%%%%%%%%%%

Waterramp suriname met cyanice


%%%%%%%%%%%%%%%%%%%%%%%%%%%%%%%%%%%%%%%%%%%%%%%%%%%%%%%%%%%%%%%%%
boeing 737 crashes


algemene vragen
oorzaken
\bibitem{gates18112020boeingcrisis} ... \LaTeX:\\ \url{https://www.seattletimes.com/business/boeing-aerospace/what-led-to-boeings-737-max-crisis-a-qa/}
\bibitem{boeing737maxsoftwareprobles} ... \LaTeX:\\ \url{https://www.schneier.com/blog/archives/2019/04/excellent_analy.html}
fout in de software
\bibitem{avetisov19032019boeingmalwarestate} ... \LaTeX:\\ \url{https://www.forbes.com/sites/georgeavetisov/2019/03/19/malware-at-30000-feet-what-the-737-max-says-about-the-state-of-airplane-software-security/?sh=4d26f7052a9e}
het nationaal veiligheidsbelang
\bibitem{thompson23112020nationalsecurityboeing} ... \LaTeX:\\ \url{https://www.forbes.com/sites/lorenthompson/2020/11/23/five-reasons-return-of-boeings-737-max-to-service-is-important-to-national-security/?sh=2128ea552018}
falend toezicht
\bibitem{gates21032019FAAControlSystem} ... \LaTeX:\\ \url{https://www.seattletimes.com/business/boeing-aerospace/failed-certification-faa-missed-safety-issues-in-the-737-max-system-implicated-in-the-lion-air-crash/}
onderzoeksrapport
\bibitem{faa18112020boeingreview} ... \LaTeX:\\ \url{https://www.faa.gov/foia/electronic_reading_room/boeing_reading_room/media/737_RTS_Summary.pdf}
\bibitem{wiki737maxgroundings} ... \LaTeX:\\ \url{https://en.wikipedia.org/wiki/Boeing_737_MAX_groundings}
veiligheidsrisico's
menselijke fouten
\bibitem{campbell02052019boengcrashhumanerrors} ... \LaTeX:\\ \url{https://www.theverge.com/2019/5/2/18518176/boeing-737-max-crash-problems-human-error-mcas-faa}
overzicht van crashes
\bibitem{hawkins22032019737maxairplanes} ... \LaTeX:\\ \url{https://www.theverge.com/2019/3/22/18275736/boeing-737-max-plane-crashes-grounded-problems-info-details-explained-reasons}
veiligheidsopmerking
\bibitem{thomas30082020737safest} ... \LaTeX:\\ \url{https://www.airlineratings.com/news/boeings-737-max-will-one-safest-aircraft-history/}
aanpassingen
\bibitem{boeing737maxdisplay} ... \LaTeX:\\ \url{https://www.boeing.com/commercial/737max/737-max-software-updates.page}
waarschuwingen//output signalen
\bibitem{fehrm24112020737changes} ... \LaTeX:\\ \url{https://leehamnews.com/2020/11/24/boeing-737-max-changes-beyond-mcas/}
software gerelateerde fouten
\bibitem{travis18042019737maxsoftwaredevop} ... \LaTeX:\\ \url{https://spectrum.ieee.org/aerospace/aviation/how-the-boeing-737-max-disaster-looks-to-a-software-developer}
onderzoeksrapport
de rol van de publieke opinie
\bibitem{barnett05052019737maxcrisis} ... \LaTeX:\\ \url{https://pubsonline.informs.org/do/10.1287/orms.2019.05.05/full/}
onderzoek van europese luchtvaart agentschap
\bibitem{easa27012021737maxsafereturn} ... \LaTeX:\\ \url{https://www.easa.europa.eu/newsroom-and-events/news/easa-declares-boeing-737-max-safe-return-service-europe}
veiligheidsvraagstuk
\bibitem{touitou11032019737tragedies} ... \LaTeX:\\ \url{https://phys.org/news/2019-03-boeing-max-safety-tragedies.html}
artikel over sensoren
\bibitem{hemmerdinger02022021737maxdeliveries} ... \LaTeX:\\ \url{https://www.flightglobal.com/airframers/boeing-delays-737-max-10-deliveries-two-years-to-2023/142245.article}
goedkeuring van europese luchtvaart autoriteiten
advies aan de faa
\bibitem{bielby27022021faaimprovesafety} ... \LaTeX:\\ \url{https://www.hstoday.us/subject-matter-areas/airport-aviation-security/oig-tells-faa-to-improve-safety-oversight-following-boeing-737-max-review/}
\bibitem{boyle18112020737maxupgrade} ... \LaTeX:\\ \url{https://www.geekwire.com/2020/faas-go-ahead-737-maxs-return-flight-kicks-off-massive-software-upgrade/}
\bibitem{bergstraburgess122019737maxMcasAlgorithm} ... \LaTeX:\\ \url{https://www.researchgate.net/publication/338420944_A_Promise_Theoretic_Account_of_the_Boeing_737_Max_MCAS_Algorithm_Affair}
achtergrond informatie
\bibitem{737mcas} ... \LaTeX:\\ \url{http://www.b737.org.uk/mcas.htm}
algemeen vertrouwen
\bibitem{newburger17052019boeingcrisis} ... \LaTeX:\\ \url{https://www.cnbc.com/2019/05/16/what-you-need-to-know-about-boeings-737-max-crisis.html}
toestemming europese autoriteiten
problemen
\bibitem{arstechnica22012020737problems} ... \LaTeX:\\ \url{https://arstechnica.com/information-technology/2020/01/737-max-fix-slips-to-summer-and-thats-just-one-of-boeings-problems/}
uitgebreid artikel over de onderzoeken en het vliegverbod
\bibitem{german190620217372yaftergrounded} ... \LaTeX:\\ \url{https://www.cnet.com/news/boeing-737-max-8-all-about-the-aircraft-flight-ban-and-investigations/}
computers als oorzaak
lessons learned
\bibitem{beningo02052019boeinglessons} ... \LaTeX:\\ \url{https://www.designnews.com/electronics-test/5-lessons-learn-boeing-737-max-fiasco}
\bibitem{duran05042019boeingspof} ... \LaTeX:\\ \url{https://www.eurocontrol.int/publication/effects-network-extra-standby-aircraft-and-boeing-737-max-grounding}
single point of failure
\bibitem{ID} ... \LaTeX:\\ \url{https://dmd.solutions/blog/2019/04/05/how-a-single-point-of-failure-spof-in-the-mcas-software-could-have-caused-the-boeing-737-max-crash-in-ethiopia/}
\bibitem{makichuck24012021737fearflying} ... \LaTeX:\\ \url{https://asiatimes.com/2021/01/boeings-737-max-and-the-fear-of-flying/}
lijst van tehnische aanpassingen
\bibitem{caa737modifications} ... \LaTeX:\\ \url{https://www.caa.co.uk/Consumers/Guide-to-aviation/Boeing-737-MAX/}
\bibitem{oestergaard14122020boeingdeliveries} ... \LaTeX:\\ \url{https://dsm.forecastinternational.com/wordpress/2020/12/14/airbus-and-boeing-report-november-2020-commercial-aircraft-orders-and-deliveries/}
code lek
\bibitem{reenberg787flaws} ... \LaTeX:\\ \url{https://www.wired.com/story/boeing-787-code-leak-security-flaws/}
\bibitem{fitch16092020737backlogrisks} ... \LaTeX:\\ \url{https://www.fitchratings.com/research/corporate-finance/boeing-737-max-return-backlog-risks-remain-16-09-2020}
Cultuurverandering, deregulatie, systeemwijziging of gewoon een kwestie van competentie
\bibitem{willis27082020737maxfailures} ... \LaTeX:\\ \url{https://www.aerospacetestinginternational.com/features/what-broke-the-737-max.html}
extra aanpassingen
\bibitem{ostrower11062020more737changes} ... \LaTeX:\\ \url{https://theaircurrent.com/aviation-safety/boeings-737-max-software-done-but-regulators-plot-more-changes-after-jets-return/}
wat ging er mis een analyse van een ex-iloot
De utoriteiten waren op de hoogte
\bibitem{hruska13122019faaknown737crashrate} ... \LaTeX:\\ \url{https://www.extremetech.com/extreme/303373-the-faa-knew-the-737-max-was-dangerous-and-kept-it-flying-anyway}
kwaliteiten van het alarmsysteem niet goed bekend
\bibitem{bloomberg26092019failedpred} ... \LaTeX:\\ \url{https://time.com/5687473/boeing-737-alarm-system/}
\bibitem{whiteman09072020boengcancelstock} ... \LaTeX:\\ \url{https://www.nasdaq.com/articles/boeing-gets-dealt-another-737-max-cancellation-blow.-what-it-means-for-boeing-stock-2020}
\bibitem{leopold09192019boeingreliability} ... \LaTeX:\\ \url{https://www.eetimes.com/boeing-crashes-highlight-a-worsening-reliability-crisis/}
veiligheidsvraagstuk
\bibitem{koenig11122019737crashesnofix} ... \LaTeX:\\ \url{https://www.latimes.com/business/story/2019-12-11/faa-boeing-737-max-crashes}
probleemanalyse, veiligheidsvraagstuk
\bibitem{dohertylindeman15032019737problems} ... \LaTeX:\\ \url{https://www.politico.com/story/2019/03/15/boeing-737-max-grounding-1223072}
falend toezicht
\bibitem{stodder02102019corruptoversight} ... \LaTeX:\\ \url{https://www.pogo.org/analysis/2019/10/corrupted-oversight-the-faa-boeing-and-the-737-max/}
\bibitem{afacwaLostSafeguards} ... \LaTeX:\\ \url{https://www.afacwa.org/the_inside_story_of_mcas_seattle_times}
doelstellingen en veiligheidsvraagstukken
\bibitem{swayne18032019profitssafety} ... \LaTeX:\\ \url{https://www.marxist.com/737-max-scandal-boeing-putting-profits-before-safety.htm}
\bibitem{freed26022021liftaustraliaban} ... \LaTeX:\\ \url{https://finance.yahoo.com/news/australia-lifts-ban-boeing-737-035817682.html?guccounter=1&guce_referrer=aHR0cHM6Ly93d3cuZ29vZ2xlLmNvbS8&guce_referrer_sig=AQAAAHZCJYy_0A5VS2WiPoCvH4xdrRNkmkdsv5EWJ2RLIz_AS-rxsTty6AF1_HlmJiRyWYqCXDi4p0Xs4isYkNkCq2Pfo-pQ60Xz_IfTNjm4FgoZiBMC4zpZlB6F0fwecrjE_ujAXZzG4xPJnWCd8-G3VLlPTY8h3H31eQ1i8hY9AIyy}
autoriteiten krijgen tik op de vingers
\bibitem{reed15032019softwareattention} ... \LaTeX:\\ \url{https://medium.com/@jpaulreed/the-737max-and-why-software-engineers-should-pay-attention-a041290994bd}
\bibitem{news17032019softwareexplains} ... \LaTeX:\\ \url{https://news.ycombinator.com/item?id=19414775}
\bibitem{legget21122020eu737maxsafe} ... \LaTeX:\\ \url{https://www.bbc.com/news/55366320}
\bibitem{marketscreener0103221737chinarecertification} ... \LaTeX:\\ \url{https://www.marketscreener.com/news/latest/China-studies-Boeing-737-MAX-recertification-wants-safety-concerns-fully-addressed--32569125/}
motor in brand
\bibitem{euractiv22022021737firegrounds} ... \LaTeX:\\ \url{https://www.euractiv.com/section/aviation/news/boeing-grounds-777s-after-engine-fire/}
\bibitem{benny18022019737returnUAE} ... \LaTeX:\\ \url{https://gulfnews.com/business/aviation/uae-airspace-to-see-return-of-boeing-737-max-1.1613627548923}
motor in brand gevlogen
\bibitem{biersmichel22022021777grounds} ... \LaTeX:\\ \url{https://techxplore.com/news/2021-02-boeing-urges-grounding-777s.html}
\bibitem{ID} ... \LaTeX:\\ \url{https://www.politico.eu/article/uk-temporarily-bans-some-boeing-aircraft-after-pratt-whitney-engine-incidents/}
\bibitem{reuters23022021777metalfatigue} ... \LaTeX:\\ \url{https://www.timeslive.co.za/news/world/2021-02-23-damage-to-united-boeing-777-engine-consistent-with-metal-fatigue--ntsb/}
faa was niet kritisch genoeg
\bibitem{ID} ... \LaTeX:\\ \url{https://federalnewsnetwork.com/government-news/2021/02/federal-watchdog-blasts-faa-over-certification-of-boeing-jet/}






%%%%%%%%%%%%%%%%%%%%%%%%%%%%%%%%%%%%%%%%%%%%%%%%%%%%%%%%%%%%%%%%%

china explosion 2015 tianjin

verhaal van brandweermannen
\bibitem{ID} ... \LaTeX:\\ \url{https://slate.com/human-interest/2015/08/chinese-explosion-aftermath-officials-investigate-causes-behind-warehouse-blast-and-death-of-88-firefighters.html}
artikel
\bibitem{ID} ... \LaTeX:\\ \url{https://www.chinafile.com/conversation/tianjin-explosion}
invloed van social media
\bibitem{ID} ... \LaTeX:\\ \url{https://www.economist.com/asia/2015/08/18/a-blast-in-tianjin-sets-off-an-explosion-online}
\bibitem{ID} ... \LaTeX:\\ \url{https://america.cgtn.com/2015/08/12/explosion-reported-in-tianjin-china}
\bibitem{ID} ... \LaTeX:\\ \url{https://factcheck.afp.com/no-photo-was-taken-chinese-city-tianjin-august-2015}
vergelijking van twee rampen
\bibitem{ID} ... \LaTeX:\\ \url{https://airshare.air-inc.com/how-does-the-beirut-explosion-compare-to-tianjin}
overheid en media
\bibitem{ID} ... \LaTeX:\\ \url{https://newbloommag.net/2015/08/17/tianjin-explosion/}
chemische industrie ondeer de loep
\bibitem{ID} ... \LaTeX:\\ \url{https://www.voanews.com/east-asia-pacific/tianjin-blast-puts-spotlight-chemical-industry}
\bibitem{ID} ... \LaTeX:\\ \url{https://abcnews.go.com/International/apocalyptic-aftermath-devastating-images-tianjin-china-explosions/story?id=33057017}
\bibitem{ID} ... \LaTeX:\\ \url{https://www.reachingoutacrossdurham.co.uk/osk/tianjin-explosion-2021}
\bibitem{ID} ... \LaTeX:\\ \url{https://aiche.onlinelibrary.wiley.com/doi/abs/10.1002/prs.11789}
\bibitem{ID} ... \LaTeX:\\ \url{https://www.automotivelogistics.media/thousands-of-cars-destroyed-in-tianjin-port-explosions/13570.article}
\bibitem{ID} ... \LaTeX:\\ \url{https://www.joc.com/port-news/asian-ports/port-tianjin/tianjin-port-explosions-could-be-most-expensive-maritime-disaster_20150826.html}
\bibitem{ID} ... \LaTeX:\\ \url{https://www.bloomberg.com/news/articles/2015-08-12/explosion-in-northern-china-shatters-windows-causes-injuries}
\bibitem{ID} ... \LaTeX:\\ \url{https://unece.org/fileadmin/DAM/env/documents/2016/TEIA/OECD_WGCA_24-27_OCT_2016/Session_3_Zhao_-__Introduction_of_Tianjin_Accident_-_Jinsong_Zhao.pdf}
gemaakte fouten
\bibitem{ID} ... \LaTeX:\\ \url{https://porteconomicsmanagement.org/pemp/contents/part6/port-resilience/site-2015-tianjin-port-explosions/}
\bibitem{ID} ... \LaTeX:\\ \url{https://www.alamy.com/stock-image-tianjin-china-17th-aug-2015-tianjin-explosion-aftermath-blast-site-165334778.html}
\bibitem{ID} ... \LaTeX:\\ \url{https://www.popularmechanics.com/technology/news/a16871/massive-explosions-china-city-of-tianjin/}
\bibitem{ID} ... \LaTeX:\\ \url{https://www.imago-images.com/st/0080815934}
\bibitem{ID} ... \LaTeX:\\ \url{https://www.chemistryworld.com/news/deadly-chemical-blast-at-chinese-port/8857.article}
\bibitem{ID} ... \LaTeX:\\ \url{https://www.process-worldwide.com/tianjin-explosion-from-chemical-perspective-insights-and-backgrounds-a-502381/}
vergelijking met andere explosies
\bibitem{ID} ... \LaTeX:\\ \url{https://apnews.com/article/lebanon-fires-us-news-explosions-middle-east-53f4206a7f1db0812262a15d22e1e58f}
invloed van de ramp op de industrie
\bibitem{ID} ... \LaTeX:\\ \url{https://fortune.com/2015/08/14/tianjin-port-explosion-shipping-delays/}
is er sprake van een doofpot
\bibitem{ID} ... \LaTeX:\\ \url{https://www.washingtontimes.com/news/2015/aug/20/inside-china-tianjin-explosions-cover-up-exposes-b/}
eigendomsverzekering
\bibitem{ID} ... \LaTeX:\\ \url{https://www.artemis.bm/news/tianjin-explosions-property-insurance-loss-could-reach-3-5bn-swiss-re/}
\bibitem{ID} ... \LaTeX:\\ \url{https://www.thechinastory.org/yearbooks/yearbook-2015/forum-the-abyss-%E5%9D%8E/tianjin-explosions/}
effecten op de lange termijn
\bibitem{ID} ... \LaTeX:\\ \url{https://www.flexport.com/blog/tianjin-explosion-effect-on-supply-chains/}
\bibitem{ID} ... \LaTeX:\\ \url{https://www.cicm.org.my/images/articles/CICM-Article-on-Tianjin-Blast-Oct2015.pdf}
lessons learned
\bibitem{ID} ... \LaTeX:\\ \url{https://www.genre.com/knowledge/blog/lessons-from-the-tianjin-explosion-en.html}
\bibitem{ID} ... \LaTeX:\\ \url{https://www.ft.com/content/ad62904c-44ce-11e5-b3b2-1672f710807b}
\bibitem{ID} ... \LaTeX:\\ \url{https://www.huffingtonpost.co.uk/2015/08/13/tianjin-explosion-china-shocking-footage-caught-on-camera_n_7980888.html}
\bibitem{ID} ... \LaTeX:\\ \url{https://www.thatsmags.com/china/post/19189/massive-fire-rocks-tianjin-port}
gevolgen voor de industrie
\bibitem{ID} ... \LaTeX:\\ \url{https://www.everstream.ai/risk-center/special-reports/the-jiangsu-yancheng-explosion/}
\bibitem{ID} ... \LaTeX:\\ \url{https://www.newyorker.com/news/news-desk/after-tianjin-an-outbreak-of-mistrust-in-china}
framing vanuit de chinese media
\bibitem{ID} ... \LaTeX:\\ \url{https://www.neliti.com/publications/101997/the-chinese-media-framing-of-the-2015s-tianjin-explosion}
\bibitem{ID} ... \LaTeX:\\ \url{https://www.reinsurancene.ws/chinese-insurers-settle-1-5-billion-tianjin-blast-claims/}
niewsartikel
\bibitem{ID} ... \LaTeX:\\ \url{https://www.thechemicalengineer.com/news/update-78-confirmed-dead-after-chinese-chemicals-plant-explosion/}
\bibitem{ID} ... \LaTeX:\\ \url{https://www.caixinglobal.com/2016-11-10/chinese-executive-receives-suspended-death-sentence-over-2015-tianjin-warehouse-blast-101006325.html}
toegang tot de ramplplek vanuit de okale journalistiek
\bibitem{ID} ... \LaTeX:\\ \url{https://chinadigitaltimes.net/2015/08/he-xiaoxin-how-far-can-i-go-and-how-much-can-i-do/}
artikel
\bibitem{ID} ... \LaTeX:\\ \url{https://www.wnpr.org/post/china-examines-aftermath-immense-twin-explosions-killed-dozens}
\bibitem{ID} ... \LaTeX:\\ \url{https://theconversation.com/what-is-ammonium-nitrate-the-chemical-that-exploded-in-beirut-143979}
\bibitem{ID} ... \LaTeX:\\ \url{https://chemicalwatch.com/36730/nationwide-inspections-in-china-follow-tianjin-explosion}
\bibitem{ID} ... \LaTeX:\\ \url{https://www.thehindu.com/news/international/investigation-begun-into-china-gas-explosion-as-toll-rises/article34818324.ece}
\bibitem{ID} ... \LaTeX:\\ \url{https://santiagotimes.cl/2019/03/24/64-killed-600-injured-in-china-chemical-plant-blast/}
oorzaken
\bibitem{ID} ... \LaTeX:\\ \url{https://klingecorp.com/blog/what-caused-the-tianjin-explosions/}
case study
\bibitem{ID} ... \LaTeX:\\ \url{https://www.preventionweb.net/educational/view/57235}
niewsartikel
\bibitem{ID} ... \LaTeX:\\ \url{https://www.cnbc.com/2015/08/12/explosion-in-tianjin-china.html}
chronologische uiteenzetting
\bibitem{ID} ... \LaTeX:\\ \url{https://www.aria.developpement-durable.gouv.fr/wp-content/files_mf/A46803_a46803_fiche_impel_006.pdf}
corruptie
\bibitem{ID} ... \LaTeX:\\ \url{https://www.nytimes.com/2015/08/31/world/asia/behind-tianjin-tragedy-a-company-that-flouted-regulations-and-reaped-profits.html}
mismanagement als oorzaak
\bibitem{ID} ... \LaTeX:\\ \url{https://www.nytimes.com/2016/02/06/world/asia/tianjin-explosions-were-result-of-mismanagement-china-finds.html}
\bibitem{ID} ... \LaTeX:\\ \url{https://cen.acs.org/articles/94/web/2016/02/Chinese-Investigators-Identify-Cause-Tianjin.html}
autoriteiten publiceren onderoeksrapport
\bibitem{ID} ... \LaTeX:\\ \url{https://cen.acs.org/articles/94/i7/Chinese-Investigators-Identify-Cause-Tianjin.html}
fotos van de rampplek
\bibitem{ID} ... \LaTeX:\\ \url{https://www.theatlantic.com/photo/2015/08/photos-of-the-aftermath-of-the-massive-explosions-in-tianjin-china/401228/}
\bibitem{ID} ... \LaTeX:\\ \url{https://edition.cnn.com/2015/08/13/asia/china-tianjin-explosions/index.html}
niuwesartiekel}
\bibitem{ID} ... \LaTeX:\\ \url{https://www.cbc.ca/news/world/china-explosion-tianjin-1.3189455}
verantwoordelijke
\bibitem{ID} ... \LaTeX:\\ \url{https://www.thestar.com/news/world/2016/11/09/chinese-executive-gets-death-sentence-over-tianjin-explosion-in-2015.html}
risicobeperking/controle
\bibitem{ID} ... \LaTeX:\\ \url{https://www.swissre.com/en/china/news-insights/articles/analysis-of-tianjin-port-explosion-china.html}
censuur
\bibitem{ID} ... \LaTeX:\\ \url{https://foreignpolicy.com/2015/09/10/censored-china-young-survivor-tianjin-explosion-viral-post/}
censuur
\bibitem{ID} ... \LaTeX:\\ \url{https://qz.com/756872/a-year-after-the-tianjin-blast-public-mourning-and-discussion-about-it-are-still-censored-in-china/}
verschillende artikelen
\bibitem{ID} ... \LaTeX:\\ \url{https://www.scmp.com/topics/tianjin-warehouse-explosion-2015}
\bibitem{ID} ... \LaTeX:\\ \url{https://www.wsj.com/articles/BL-CJB-27664}
\bibitem{ID} ... \LaTeX:\\ \url{https://www.nbcnews.com/news/world/tianjin-explosions-californian-witness-filmed-dramatic-china-blasts-n409701}
\bibitem{ID} ... \LaTeX:\\ \url{https://ui.adsabs.harvard.edu/abs/2016AGUFM.S13D..06P/abstract}
afwikkeling van de ramp
\bibitem{ID} ... \LaTeX:\\ \url{https://chinadialogue.net/en/pollution/9188-back-to-the-blast-zone-one-year-after-the-tianjin-explosion/}
\bibitem{ID} ... \LaTeX:\\ \url{https://www.wired.com/2015/08/chinas-huge-tianjin-explosion-looked-like-space/}
\bibitem{ID} ... \LaTeX:\\ \url{https://www.abc.net.au/news/2015-08-13/explosion-rocks-north-chinese-city-of-tianjin/6693336?nw=0}
ambtenaren onderzocht
\bibitem{ID} ... \LaTeX:\\ \url{https://thediplomat.com/2015/08/23-executives-government-officials-under-investigation-for-role-in-tianjin-explosions/}
\bibitem{ID} ... \LaTeX:\\ \url{http://america.aljazeera.com/articles/2015/8/13/at-least-50-dead-and-hundreds-injured-in-chinese-warehouse-explosion.html}
risico-inschatting
\bibitem{ID} ... \LaTeX:\\ \url{https://www.mdpi.com/2071-1050/12/3/1169/htm}
\bibitem{ID} ... \LaTeX:\\ \url{https://www.mdpi.com/2071-1050/12/3/1169/htm}
\bibitem{ID} ... \LaTeX:\\ \url{https://www.cbsnews.com/news/tianjin-port-china-massive-explosion-hundreds-injured/}
\bibitem{ID} ... \LaTeX:\\ \url{https://www.hkjcdpri.org.hk/download/casestudies/Tianjin_CASE.pdf}
\bibitem{ID} ... \LaTeX:\\ \url{https://time.com/3996168/tianjin-explosion-china-pictures/}
onderzoeksrapport
\bibitem{ID} ... \LaTeX:\\ \url{https://www.hfw.com/Tianjin-Port-explosion-August-2015}
\bibitem{ID} ... \LaTeX:\\ \url{https://news.un.org/en/story/2015/08/506912-following-tianjin-explosion-un-expert-calls-china-ensure-transparent}
\bibitem{ID} ... \LaTeX:\\ \url{https://www.france24.com/en/20150812-huge-explosions-rock-chinese-city-tianjin}
\bibitem{ID} ... \LaTeX:\\ \url{https://choice.npr.org/index.html?origin=https://www.npr.org/2015/08/14/432280627/what-caused-the-warehouse-explosions-in-tianjin-china}
123 verantwoordelijken
\bibitem{ID} ... \LaTeX:\\ \url{https://www.bbc.com/news/world-asia-china-35506311}
\bibitem{ID} ... \LaTeX:\\ \url{https://www.washingtonpost.com/gdpr-consent/?next_url=https%3a%2f%2fwww.washingtonpost.com%2fnews%2fworldviews%2fwp%2f2015%2f08%2f12%2fvideos-show-chinese-city-of-tianjin-rocked-by-enormous-explosion%2f}
lang artiekel
\bibitem{ID} ... \LaTeX:\\ \url{https://www.businessinsider.com/the-chemical-explosion-in-china-killed-more-than-100-people-and-the-devastation-is-unreal-2015-8?international=true&r=US&IR=T}
\bibitem{ID} ... \LaTeX:\\ \url{https://pubmed.ncbi.nlm.nih.gov/27311537/}
\bibitem{ID} ... \LaTeX:\\ \url{https://www.reuters.com/article/us-china-blast-insurance-idUSKCN0QM0N220150817}
\bibitem{ID} ... \LaTeX:\\ \url{https://www.sciencedirect.com/science/article/abs/pii/S0305417916300079}
\bibitem{ID} ... \LaTeX:\\ \url{https://en.wikipedia.org/wiki/2015_Tianjin_explosions}
\bibitem{ID} ... \LaTeX:\\ \url{https://www.bbc.com/news/world-asia-china-33844084}
\bibitem{ID} ... \LaTeX:\\ \url{https://www.independent.co.uk/news/world/asia/tianjin-explosion-photos-china-chemical-factory-accident-crater-revealed-a7199591.html}

veiigheidshandhaving
\bibitem{ID} ... \LaTeX:\\ \url{https://www.ilo.org/legacy/english/protection/safework/ctrl_banding/toolkit/main_guide.pdf}
\bibitem{ID} ... \LaTeX:\\ \url{https://echa.europa.eu/documents/10162/21332507/guide_chemical_safety_sme_en.pdf}
\bibitem{ID} ... \LaTeX:\\ \url{https://ec.europa.eu/taxation_customs/dds2/SAMANCTA/EN/Safety/AppendixD_EN.htm}
\bibitem{ID} ... \LaTeX:\\ \url{https://www.ilo.org/safework/info/publications/WCMS_113134/lang--en/index.htm}


%%%%%%%%%%%%%%%%%%%%%%%%%%%%%%%%%%%%%%%%%%%%%%%%%%%%%%%%%%%%%%%%%
tesla autopilot crashes


veiigheidsrisico
\bibitem{evan01042019teslaautopilotIntersection} ... \LaTeX:\\ \url{https://spectrum.ieee.org/cars-that-think/transportation/self-driving/three-small-stickers-on-road-can-steer-tesla-autopilot-into-oncoming-lane}

\bibitem{testVehicleSafetyReport} ... \LaTeX:\\ \url{https://www.tesla.com/VehicleSafetyReport}
veiligheidsrapport mbt autopilot
\bibitem{lambert31062020q2safetyreport} ... \LaTeX:\\ \url{https://electrek.co/2020/07/31/tesla-q2-2020-safety-report-strong-improvement-autopilot-accidents/}
consumentenrapport
bluetooth veiligheidsvraagstuk
\bibitem{wiredBloutoothHackTesla} ... \LaTeX:\\ \url{https://www.wired.com/story/tesla-model-x-hack-bluetooth/}
veiigheidsvraagstuk vanwege touch screen
\bibitem{preston14012021NHTSATeslaRecall} ... \LaTeX:\\ \url{https://www.consumerreports.org/car-recalls-defects/nhtsa-asks-tesla-to-recall-model-s-model-x-touch-screen-safety-issues/}
veiligheidsvraagstuk
\bibitem{cio25112020belgianTeslaHack} ... \LaTeX:\\ \url{https://cio.economictimes.indiatimes.com/news/digital-security/security-researchers-hack-steal-tesla-model-x-within-minutes/79406553}
veiligheidsvraagstuk
rapport over autopilot
\bibitem{templeton06092019HTSBReportTesla} ... \LaTeX:\\ \url{https://www.forbes.com/sites/bradtempleton/2019/09/06/ntsb-report-on-tesla-autopilot-accident-shows-whats-inside-and-its-not-pretty-for-fsd/?sh=6905e7d4dc55}
de invloed van de bestuurder bij tesla ongeluk
\bibitem{Dickey08012021SuddenTeslaAcceleration} ... \LaTeX:\\ \url{https://techcrunch.com/2021/01/08/nhtsa-tesla-sudden-unintended-acceleration-driver-error/}
veiligheidsvraagstuk
\bibitem{darkReading17112020TeslaBackup} ... \LaTeX:\\ \url{https://www.darkreading.com/threat-intelligence/security-risks-discovered-in-tesla-backup-gateway/d/d-id/1339462}
veiligheidsvraagstuk
\bibitem{leyden23032020TeslaInterfaceHack} ... \LaTeX:\\ \url{https://portswigger.net/daily-swig/web-based-attack-crashes-tesla-driver-interface}
veiigheidsvraagstuk
\bibitem{huddlestonjr03042019ChineseTeslaHack} ... \LaTeX:\\ \url{https://www.cnbc.com/2019/04/03/chinese-hackers-tricked-teslas-autopilot-into-switching-lanes.html}
veiligheidsvraagstuk
veiligheidsvraagstuk
\bibitem{heilweil26022020teslaAutopilot} ... \LaTeX:\\ \url{https://www.vox.com/recode/2020/2/26/21154502/tesla-autopilot-fatal-crashes}
rapport over ongeluk
veiligheidsvraagstuk
veiligheidsvraagstuk
\bibitem{blanco04102019NHTSATesla} ... \LaTeX:\\ \url{https://www.caranddriver.com/news/a29369387/nhtsa-tesla-safety/}
veiligheidsvraagstuk
ransomware aanval op tesla
tesla batterij is veiligheidsvraagstuk geworden
\bibitem{mitchell01072020teslabatterycooling} ... \LaTeX:\\ \url{https://www.latimes.com/business/story/2020-07-01/federal-safety-officials-probe-tesla-battery-cooling-system}
ongeluk
\bibitem{bbc26022020AutopilotCrash} ... \LaTeX:\\ \url{https://www.bbc.com/news/technology-51645566}
veiligheidsvraagstuk
veiligheidsvraagstuk
\bibitem{stumpff04052020TeslaPersonalData} ... \LaTeX:\\ \url{https://www.thedrive.com/news/33272/tesla-discarded-old-car-parts-with-customers-personal-data-passwords-report}
dodelijk ongeluk
\bibitem{levin08062018teslaautopilotsafety} ... \LaTeX:\\ \url{https://www.theguardian.com/technology/2018/jun/07/tesla-fatal-crash-silicon-valley-autopilot-mode-report}
veiligheidsvraagstuk: ransomware
veiligheidsvraagstuk: medewerker in de fout
\bibitem{cbrook06082021TeslaInsideDataThreft} ... \LaTeX:\\ \url{https://digitalguardian.com/blog/tesla-data-theft-case-illustrates-danger-insider-threat}
\bibitem{shilling25022021Tesla} ... \LaTeX:\\ \url{https://jalopnik.com/tesla-is-stopping-some-model-3-production-report-1846353323}
veiligheidsvraagstuk: hackers je systeem laten testen
verdedigen tegenover ransomware
veiligheidsrisico
prijzen omlaag
autopilot
\bibitem{randall05112019modelSurvey} ... \LaTeX:\\ \url{https://www.bloomberg.com/graphics/2019-tesla-model-3-survey/autopilot.html}
malware door een medewerker
\bibitem{ID} ... \LaTeX:\\ \url{https://www.teslarati.com/tesla-employee-fbi-thwarts-russian-cybersecurity-attack/}
dodelijk ongeluk
\bibitem{ID} ... \LaTeX:\\ \url{https://www.marketwatch.com/story/apple-engineer-killed-in-tesla-suv-crash-on-silicon-valley-freeway-was-playing-videogame-ntsb-2020-02-25}
\bibitem{fottrell03092018TeslaSecurityChecks} ... \LaTeX:\\ \url{https://www.marketwatch.com/story/nearly-100-of-teslas-stolen-in-the-us-since-2011-have-been-recovered-2018-08-10}
waarom een tesla stelen bijna onmogelijk is
\bibitem{ID} ... \LaTeX:\\ \url{https://www.welivesecurity.com/2019/03/25/white-hats-hack-tesla-keep/}
veiligheidsonderzoek
\bibitem{ID} ... \LaTeX:\\ \url{https://www.tripwire.com/state-of-security/security-data-protection/tesla-encouraging-good-faith-security-research-in-bug-bounty-program/}
softwarefout maakt diestal mogelijk
\bibitem{kirk26112020modelX} ... \LaTeX:\\ \url{https://www.bankinfosecurity.com/tesla-model-x-stolen-in-minutes-using-software-flaws-a-15462}
fouten ontdekt in onderzoek
\bibitem{ID} ... \LaTeX:\\ \url{https://www.cnet.com/roadshow/news/tesla-ev-appeal-loyalty-study/}
\bibitem{bbc24022021hyundaiBatteryFireFix} ... \LaTeX:\\ \url{https://www.bbc.com/news/technology-56156801}








tesla cloud gehacked
\bibitem{ID} ... \LaTeX:\\ \url{https://arstechnica.com/information-technology/2018/02/tesla-cloud-resources-are-hacked-to-run-cryptocurrency-mining-malware/}
\bibitem{ID} ... \LaTeX:\\ \url{https://www.motortrend.com/news/tesla-model-y-ev-safety-quality-issues-problems/}
\bibitem{ID} ... \LaTeX:\\ \url{https://securityledger.com/2019/04/hackers-remotely-steer-tesla-model-s-using-autopilot-system/}
\bibitem{ID} ... \LaTeX:\\ \url{https://www.pcmag.com/news/report-tesla-suspends-model-3-production-in-california-until-march-7}
\bibitem{ID} ... \LaTeX:\\ \url{https://www.scmp.com/business/money/article/3121173/tesla-conduct-complete-self-inspection-after-chinese-regulators}
\bibitem{ID} ... \LaTeX:\\ \url{https://www.businesswire.com/news/home/20180220005222/en/RedLock-Releases-Cloud-Security-Report-Highlighting-Focus-on-Shared-Responsibilities-Uncovers-Cloud-Related-Exposures-at-Tesla}
\bibitem{ID} ... \LaTeX:\\ \url{https://www.epa.gov/automotive-trends/highlights-automotive-trends-report}
\bibitem{ID} ... \LaTeX:\\ \url{https://www.livemint.com/Companies/o2QLbtJc9EQ7ZcpxqgFbBP/Teslas-reward-for-finding-security-bugs-Model-3.html}
\bibitem{evans11062018} ... \LaTeX:\\ \url{https://revealnews.org/blog/tesla-fired-safety-official-for-reporting-unsafe-conditions-lawsuit-says/}
\bibitem{ID} ... \LaTeX:\\ \url{https://heimdalsecurity.com/blog/security-alert-teslacrypt-4-0-unbreakable-encryption-worse-data-leakage/}
\bibitem{ID} ... \LaTeX:\\ \url{https://www.eweek.com/cloud/tesla-cloud-account-data-breach-revealed-in-redlock-security-report/}
\bibitem{hawkins22102022} ... \LaTeX:\\ \url{https://www.theverge.com/2020/10/21/21527577/tesla-full-self-driving-autopilot-beta-software-update}
\bibitem{ID} ... \LaTeX:\\ \url{file:///C:/Users/gally/Downloads/applsci-10-02749-v2.pdf}
\bibitem{gritti24062020tesladataengine} ... \LaTeX:\\ \url{https://www.braincreators.com/brainpower/insights/teslas-data-engine-and-what-we-should-all-learn-from-it}
\bibitem{ID} ... \LaTeX:\\ \url{https://bernardmarr.com/default.asp?contentID=1251}
\bibitem{ID} ... \LaTeX:\\ \url{https://arstechnica.com/cars/2019/10/how-teslas-latest-acquisition-could-accelerate-autopilot-development/}
\bibitem{bouchard07052019teslaDeepLearning} ... \LaTeX:\\ \url{https://towardsdatascience.com/teslas-deep-learning-at-scale-7eed85b235d3}
\bibitem{ID} ... \LaTeX:\\ \url{file:///C:/Users/gally/Downloads/applsci-10-02749-v2.pdf}
\bibitem{Srikanth2019teslabigdata} ... \LaTeX:\\ \url{https://www.techiexpert.com/how-tesla-is-using-artificial-intelligence-and-big-data/}
\bibitem{rangaiah25022020teslaAI} ... \LaTeX:\\ \url{https://www.analyticssteps.com/blogs/how-tesla-making-use-artificial-intelligence-its-operations}
\bibitem{marr08012018taslabigdataAI} ... \LaTeX:\\ \url{https://www.forbes.com/sites/bernardmarr/2018/01/08/the-amazing-ways-tesla-is-using-artificial-intelligence-and-big-data/?sh=5e396aa24270}
\bibitem{bdickson29072020teslalevelfive = {author},	ALTeditor = {editor},	title = {title},	date = {date},	url = {"https://bdtechtalks.com/2020/07/29/self-driving-tesla-car-deep-learning/}
\bibitem{dcruz17062022tesladesignthink} ... \LaTeX:\\ \url{https://www.mygreatlearning.com/blog/teslas-new-ai-for-self-driving-cars/}
\bibitem{ID} ... \LaTeX:\\ \url{https://www.techiexpert.com/how-tesla-is-using-artificial-intelligence-and-big-data/}





\bibitem{mcfarland22042021selfdrivingrisks} ... \LaTeX:\\ \url{https://www.cnn.com/2021/04/21/tech/tesla-full-self-driving-launch/index.html}
\bibitem{hawkins18032021fedgovinvest} ... \LaTeX:\\ \url{https://www.theverge.com/2021/3/18/22338427/tesla-autopilot-crash-michigan-nhtsa-investigation}
\bibitem{berry21042021teslacrashtexas} ... \LaTeX:\\ \url{https://www.wionews.com/technology/doctor-among-victims-of-lethal-tesla-car-crash-in-texas-378950}
\bibitem{hull23072021regulatorsaftercrash} ... \LaTeX:\\ \url{https://www.bloomberg.com/news/newsletters/2021-06-23/hyperdrive-daily-after-30-tesla-crashes-what-s-a-regulator-to-do}
\bibitem{wikiTeslaAutopilot} ... \LaTeX:\\ \url{https://en.wikipedia.org/wiki/Tesla_Autopilot}
\bibitem{nhtsaAutomatedVehiclesSafety} ... \LaTeX:\\ \url{https://www.nhtsa.gov/technology-innovation/automated-vehicles-safety}
\bibitem{dowling23042021autopilottricking} ... \LaTeX:\\ \url{https://www.caradvice.com.au/947080/elon-musk-responds-to-deadly-texas-tesla-crash-as-consumer-reports-reveals-how-autopilot-can-be-tricked/}
\bibitem{wilson19042021teslacrashregulators} ... \LaTeX:\\ \url{https://usa.streetsblog.org/2021/04/19/regulators-could-have-prevented-fatal-tesla-crash/}

\bibitem{seamans22062021aikillerap} ... \LaTeX:\\ \url{https://www.brookings.edu/research/autonomous-vehicles-as-a-killer-app-for-ai/}
\bibitem{mitchell24022020AIDataTesla} ... \LaTeX:\\ \url{https://www.latimes.com/business/story/2020-02-24/autopilot-data-secrecy}
\bibitem{denneyjdsupraFeds} ... \LaTeX:\\ \url{https://www.jdsupra.com/post/contentViewerEmbed.aspx?fid=9844cae0-aa5a-45a5-988f-7f02fa5709c1}
\bibitem{siddiqui22102020TeslaCriticism} ... \LaTeX:\\ \url{https://www.washingtonpost.com/technology/2020/10/21/tesla-self-driving/}
\bibitem{ackerman01072016TeslaImperfect} ... \LaTeX:\\ \url{https://spectrum.ieee.org/cars-that-think/transportation/self-driving/fatal-tesla-autopilot-crash-reminds-us-that-robots-arent-perfect}
\bibitem{greene04092019misuseautopilot} ... \LaTeX:\\ \url{https://thenextweb.com/news/another-tesla-owner-is-dead-because-of-autopilot}
\bibitem{michralli26112019ubserautocarcrsash} ... \LaTeX:\\ \url{https://towardsdatascience.com/another-self-driving-car-accident-another-ai-development-lesson-b2ce3dbb4444}
\bibitem{pitmann21072021wrongfullautodeath} ... \LaTeX:\\ \url{https://www.theautochannel.com/news/2021/07/21/1024631-is-it-still-wrongful-death-if-car-is-driving-itself.html}
\bibitem{stackexchange102019teslacarmistake} ... \LaTeX:\\ \url{https://ai.stackexchange.com/questions/1488/why-did-a-tesla-car-mistake-a-truck-with-a-bright-sky}
\bibitem{tasking07062017TeslaAugmentedSafety} ... \LaTeX:\\ \url{https://resources.tasking.com/p/benefits-tesla-autopilot-and-how-adas-will-save-lives}
\bibitem{griemannExaminSelfDriving} ... \LaTeX:\\ \url{https://www.jipitec.eu/issues/jipitec-9-3-2018/4806}
\bibitem{Harkey30052019SafeSystemVehicle} ... \LaTeX:\\ \url{https://static.tti.tamu.edu/conferences/traffic-safety19/presentations/lunch/harkey.pdf}
\bibitem{ID} ... \LaTeX:\\ \url{https://thepressfree.com/have-google-and-amazon-backed-the-wrong-technology/}
\bibitem{ID} ... \LaTeX:\\ \url{https://www.irishtimes.com/business/innovation/robotaxis-have-google-and-amazon-backed-the-wrong-technology-1.4626749}
\bibitem{ID} ... \LaTeX:\\ \url{https://www.afr.com/technology/how-teslas-autopilot-got-it-wrong-in-fatal-crash-20160704-gpxsje}
\bibitem{ID} ... \LaTeX:\\ \url{https://economictimes.indiatimes.com/markets/stocks/news/what-me-worry-fed-chiefs-emotional-tone-can-drive-markets-study-suggests/articleshow/84618073.cms}
\bibitem{ID} ... \LaTeX:\\ \url{https://www.ehstoday.com/safety/article/21919260/ntsb-fatal-crash-involving-tesla-autopilot-resulted-from-driver-errors-overreliance-on-automation}
\bibitem{ID} ... \LaTeX:\\ \url{https://www.vanityfair.com/news/2016/07/how-the-media-screwed-up-the-fatal-tesla-accident}


tesla crash report
\bibitem{shepardson18062021TeslaDeaths} ... \LaTeX:\\ \url{https://www.reuters.com/business/autos-transportation/us-safety-agency-says-it-has-opened-probes-into-10-tesla-crash-deaths-since-2016-2021-06-17/}
\bibitem{hawkins30062021nhtsarequiresreporting} ... \LaTeX:\\ \url{https://www.politico.com/news/2021/05/18/ntsb-tesla-owner-was-in-drivers-seat-before-april-texas-crash-489272}
\bibitem{hawkins10052021autopilotnotavailable} ... \LaTeX:\\ \url{https://www.theverge.com/2021/5/10/22429198/tesla-ntsb-texas-crash-driverless-preliminary-report}
\bibitem{szymkowski29062021nhtsaTeslaCrashReports} ... \LaTeX:\\ \url{https://www.cnet.com/roadshow/news/tesla-autopilot-nhtsa-crash-report-self-driving-car-driver-assist-system/}
\bibitem{abc1112052021AutopilotNotinTeslaCrash} ... \LaTeX:\\ \url{https://abc11.com/tesla-crash-battery-fire-national-transportation-safety-board-driverless/10619772/}
\bibitem{ankel18062021regulatorsinvestigateAutopilot} ... \LaTeX:\\ \url{https://www.businessinsider.com/tesla-autopilot-crashes-regulators-open-probes-into-30-report-2021-6?international=true&r=US&IR=T}
\bibitem{sommerfield12072021NHTSAmandateresult} ... \LaTeX:\\ \url{https://driving.ca/column/lorraine/lorraine-explains-what-the-nhtsas-self-driving-car-crash-reporting-mandate-will-find-out}
\bibitem{ID} ... \LaTeX:\\ \url{https://www.teslarati.com/tesla-model-s-crash-texas-ntsb-preliminary-report/}
\bibitem{ID} ... \LaTeX:\\ \url{https://insideevs.com/news/506498/ntsb-report-tesla-texas-crash/}
\bibitem{ID} ... \LaTeX:\\ \url{https://electrek.co/2021/06/03/tesla-tsla-crashes-report-new-orders-in-china-free-falling/}
\bibitem{ID} ... \LaTeX:\\ \url{https://www.newsy.com/stories/ntsb-releases-report-on-fatal-tesla-crash/}
\bibitem{ID} ... \LaTeX:\\ \url{https://www.ndtv.com/world-news/autopilot-not-used-in-april-tesla-crash-says-us-report-2439146}
\bibitem{ID} ... \LaTeX:\\ \url{https://www.autocar.co.nz/autocar-news-app/fatal-driverless-tesla-crash-report-shows-autopilot-not-to-blame}
\bibitem{ID} ... \LaTeX:\\ \url{https://teleperformance-waha.sabacloud.com/Saba/Web_spf/EU2PRD0152/app/dashboard}
\bibitem{ID} ... \LaTeX:\\ \url{https://www.independent.co.uk/news/world/americas/tesla-texas-crash-model-s-autopilot-b1845286.html}
\bibitem{ID} ... \LaTeX:\\ \url{https://www.wired.com/2017/01/probing-teslas-deadly-crash-feds-say-yay-self-driving/}
\bibitem{saferoardsCrashesAutonomousvehicles} ... \LaTeX:\\ \url{https://saferoads.org/wp-content/uploads/2020/03/AV-Crash-List-with-Photos-February-2020.pdf}
\bibitem{ID} ... \LaTeX:\\ \url{https://mashable.com/article/nthsa-tesla-autopilot-model-x-crash-investigation}
\bibitem{stephardson18032021revieuwingtesla} ... \LaTeX:\\ \url{https://www.usnews.com/news/top-news/articles/2021-03-18/us-safety-agency-reviewing-23-tesla-crashes-three-from-recent-weeks}
\bibitem{krishner30062021NHTSAreport} ... \LaTeX:\\ \url{https://chicago.suntimes.com/consumer-affairs/2021/6/30/22557122/nhtsa-automated-driving-crash-reports-tesla-national-highway-traffic-safety-administration}
\bibitem{gitlin11052021autopilot} ... \LaTeX:\\ \url{https://arstechnica.com/cars/2021/05/ntsb-finds-no-reason-to-suspect-autopilot-in-fatal-tesla-crash/}
\bibitem{ID} ... \LaTeX:\\ \url{https://jalopnik.com/the-ntsb-to-partially-blame-teslas-autopilot-in-fatal-c-1803136365}
\bibitem{mitchell19012017investigationstop} ... \LaTeX:\\ \url{https://www.latimes.com/business/autos/la-fi-hy-tesla-autopilot-20170119-story.html}
\bibitem{gordon10052021teslaprelimreport} ... \LaTeX:\\ \url{https://www.vice.com/en/article/z3xxaw/ntsb-releases-preliminary-report-on-tesla-crash-that-killed-two-people}
\bibitem{shaper07062018} ... \LaTeX:\\ \url{https://choice.npr.org/index.html?origin=https://www.npr.org/2018/06/07/618081406/no-driver-input-detected-in-seconds-before-deadly-tesla-crash-ntsb-finds}
\bibitem{cochran18042021nodriverTeslaCrash} ... \LaTeX:\\ \url{https://www.click2houston.com/news/local/2021/04/18/2-men-dead-after-fiery-tesla-crash-in-spring-officials-say/}
\bibitem{habib28062016NHTSATeslaReport} ... \LaTeX:\\ \url{https://static.nhtsa.gov/odi/inv/2016/INCLA-PE16007-7876.pdf}
\bibitem{firstpress11052021fatalnonautopilot} ... \LaTeX:\\ \url{https://www.firstpost.com/tech/news-analysis/tesla-model-s-involved-in-fatal-crash-in-the-us-did-not-use-autopilot-says-ntsb-report-9609911.html}
\bibitem{raynal20042021probeTeslaCrash} ... \LaTeX:\\ \url{https://www.autoweek.com/news/green-cars/a36173804/both-local-police-and-nhtsa-probe-tesla-crash/}
\bibitem{tiungteslasoftwarecrash} ... \LaTeX:\\ \url{https://www.zdnet.com/article/apple-and-tesla-under-fire-over-software-engineers-fatal-autopilot-crash/}
\bibitem{globaltimes08052021guangdongcrash,	ALTauthor = {global times},	ALTeditor = {editor},	title = {Tesla crash in Guangdong sparks new round of safety concerns},	date = {date},	url = {"https://www.globaltimes.cn/page/202105/1222902.shtml}
\bibitem{anderson30042021secondteslacrash,	ALTauthor = {Brad Anderson},	ALTeditor = {editor},	title = {NTSB To Release Report On Deadly Tesla Crash Within A Month},	date = {date},	url = {"https://www.carscoops.com/2021/04/ntsb-to-release-report-on-deadly-tesla-crash-within-a-month/}
	\bibitem{oremus21062017fatalTeslaCrash} ... \LaTeX:\\ \url{https://slate.com/technology/2017/06/a-new-report-on-what-happened-in-the-fatal-tesla-autopilot-crash.html}
	\bibitem{guardian15052021teslacrashHandsOnWheel} ... \LaTeX:\\ \url{https://www.theguardian.com/us-news/2021/may/15/tesla-fatal-california-crash-autopilot}
	\bibitem{Puzzanghera13092017TeslaSharesBlame} ... \LaTeX:\\ \url{https://www.stuff.co.nz/motoring/news/96797272/tesla-shares-some-blame-in-fatal-autopilot-crash-report}
	\bibitem{jaillet02022017teslaAutopilotLimitations} ... \LaTeX:\\ \url{https://www.overdriveonline.com/business/article/14891759/dot-report-on-fatal-2016-tesla-crash-with-tractor-trailer-blames-limitations-of-autopilot-mode}
	\bibitem{reuters03102019teslaAutoParkingFail} ... \LaTeX:\\ \url{https://venturebeat.com/2019/10/03/regulators-investigate-teslas-automated-parking-feature-following-crash-reports/}
	\bibitem{dowling23042021} ... \LaTeX:\\ \url{https://www.caradvice.com.au/947080/elon-musk-responds-to-deadly-texas-tesla-crash-as-consumer-reports-reveals-how-autopilot-can-be-tricked/}
	\bibitem{young05112021fatalTeslaReport} ... \LaTeX:\\ \url{https://www.consumeraffairs.com/news/ntsb-releases-report-on-fatal-tesla-crash-in-texas-051121.html}
	\bibitem{kierstein18032021teslaAutopilotCrashStationary} ... \LaTeX:\\ \url{https://www.motortrend.com/news/tesla-michigan-state-autopilot-crash-report/}
	\bibitem{janssen20062017teslacrashdetailflorida} ... \LaTeX:\\ \url{https://tweakers.net/nieuws/126145/onderzoeksraad-vs-publiceert-technische-details-tesla-crash-in-florida.html}
	\bibitem{ID} ... \LaTeX:\\ \url{https://gizmodo.com/new-documents-reveal-one-driver-s-agony-and-confusion-d-1841720801}
	
	
	\bibitem{ID} ... \LaTeX:\\ \url{https://www.google.com/search?q=tesla+crash+report&rlz=1C1AVUC_enNL953NL953&ei=p3kNYa6sLI_UsAeSoZrwDw&start=100&sa=N&ved=2ahUKEwjum77s_ZzyAhUPKuwKHZKQBv44WhDw0wN6BAgBEEg&biw=1920&bih=933}
	
	
	
	%%%%%%%%%%%%%%%%%%%%%%%%%%%%%%%%%%%%%%%%%%%%%%%%%%%%%%%%%%%%%%%%%
	
	vlucht 1951
	\bibitem{ID} ... \LaTeX:\\ \url{https://nl.wikipedia.org/wiki/Turkish_Airlines-vlucht_1951}
	technisch rapport
	\bibitem{ID} ... \LaTeX:\\ \url{file:///C:/Users/gally/Downloads/rapport_ta_nl_aangepast.pdf}
	beschrijving
	terugblik met overlevenden
	tijdlijn
	\bibitem{zuilen23022019Tijdlijnpoldercrash} ... \LaTeX:\\ \url{https://www.noordhollandsdagblad.nl/cnt/dmf20190221_65390940}
	artikel
	terugblik met overlevenden
	advies raad voor de veiligheid
	de overlevende, de oorzaak, regeling, herdenking, smartengeld
	verhaal van een overlevende
	herdenking
	herdenking
	bemanning deed niets met foutmelding
	parlementaire besluitenlijst
	kamervragen over de onafhankelijkheid van de raad voor veiligheid
	verhaal van een overlevende
	beschrijvend artikel van letsel en gewonden
	\bibitem{ntvg20012010letsel} ... \LaTeX:\\ \url{https://www.ntvg.nl/artikelen/vliegtuigongeval-schiphol-25-02-2009-letsels-en-verdeling-van-gewonden}
	technische fout als oorzaak
	\bibitem{wikinews04032009techfoutailines1951} ... \LaTeX:\\ \url{https://nl.wikinews.org/wiki/Technische_fout_oorzaak_vliegtuigcrash_Turkish_Airlines-vlucht_1951}
	gesprek met pieter van vollenhove voorzitter van de onderzoeksraad voor veiligheid
	onderzoeksraad voor veiligheid is onderdruk gezet
	\bibitem{luchtvaartnieuws21012020boeing737conclusies} ... \LaTeX:\\ \url{https://www.luchtvaartnieuws.nl/nieuws/categorie/72/algemeen/conclusies-crash-tk1951-na-amerikaanse-druk-afgezwakt}
	niuwesartikel
	feitenverloop
	\bibitem{adformatie280220209communicatiegebreken} ... \LaTeX:\\ \url{https://www.adformatie.nl/contentmarketing/communicatie-na-vliegramp-vertoonde-gebreken}
	zwarte doos
	\bibitem{spinnael25022009onderzoekpolderbaancrash} ... \LaTeX:\\ \url{https://flightlevel.be/244/onderzoek-polderbaan-crash-turkish-airlines-1951/}
	\bibitem{crashTurkishAirlines} ... \LaTeX:\\ \url{http://wikimapia.org/11633002/nl/Crash-Turkish-Airlines-vlucht-1951}
	\bibitem{flightradar24} ... \LaTeX:\\ \url{https://www.flightradar24.com/data/flights/tk1951}
	\bibitem{flightstatstracker} ... \LaTeX:\\ \url{https://www.flightstats.com/v2/flight-tracker/TK/1951}
	
	
	
	
	
	
	
	
	
	
	%%%%%%%%%%%%%%%%%%%%%%%%%%%%%%%%%%%%%%%%%%%%%%%%%%%%%%%%%%%%%%%%%
	de mali missie
	
	
	\bibitem{bnnvara13062018malirapport} ... \LaTeX:\\ \url{https://joop.bnnvara.nl/nieuws/rapport-haalbaarheid-en-houdbaarheid-van-mali-missie-twijfelachtig}
	\bibitem{eucal11012021malimissieverlengd} ... \LaTeX:\\ \url{https://www.consilium.europa.eu/nl/press/press-releases/2021/01/11/eucap-sahel-mali-mission-extended-until-31-january-2023-and-mandate-adjusted/}
	\bibitem{nos21052014zorgenmalimissie} ... \LaTeX:\\ \url{https://nos.nl/artikel/650637-kamer-bezorgd-over-mali-missie}
	\bibitem{meijnders} ... \LaTeX:\\ \url{https://www.bnr.nl/nieuws/10015679/koenders-positief-tegenover-verlening-mali-missie}
	\bibitem{bnrwebredactie} ... \LaTeX:\\ \url{https://www.bnr.nl/nieuws/politiek/10345553/kabinet-wil-mali-missie-stoppen-verrassend-besluit}
	\bibitem{keultjes01062016malimissiecoalitie} ... \LaTeX:\\ \url{https://www.ad.nl/nieuws/clash-om-mali-missie-dreigt-binnen-coalitie~a4151d4f/}
	\bibitem{veenhof18012019} ... \LaTeX:\\ \url{https://www.nd.nl/cultuur/boeken/536861/boek-kijkje-bij-de-mali-missie}
	\bibitem{ID} ... \LaTeX:\\ \url{https://www.youtube.com/watch?v=jmZ6uSbpCvg}
	\bibitem{isitman06012016militair} ... \LaTeX:\\ \url{https://www.ewmagazine.nl/nederland/achtergrond/2016/07/twee-nederlanse-militairen-dood-bij-oefening-mali-missie-325226/}
	\bibitem{nporadio11072016filmdemissie} ... \LaTeX:\\ \url{https://www.nporadio1.nl/nieuws/cultuur-media/9e3b076e-5401-4630-bf39-f925213c5b6b/onverwachte-openhartigheid-over-missie-in-mali}
	\bibitem{parlementairmonitor15122013mortierongeluk} ... \LaTeX:\\ \url{https://www.parlementairemonitor.nl/9353000/1/j9vvij5epmj1ey0/vjfm5p0nujzw?ctx=vj2mc67lofnr}
	
	
	sollicitatie
	de bureaucratie
	aankomst
	interview van de burgerbevolking
	steun van de bevolking minuut 15:00
	de organisatie minuut 23:00
	De militaire briefing minuut 34:00
	prioriteit minuut 39:00
	briefing minuut 40:00
	de communicatie met ministerie over inlichten minuut 44:00
	\bibitem{ID} ... \LaTeX:\\ \url{https://www.2doc.nl/documentaires/series/2doc/2016/juli/de-missie.html}
	
	%%%%%%%%%%%%%%%%%%%%%%%%%%%%%%%%%%%%%%%%%%%%%%%%%%%%%%%%%%%%%%%%%
	militair overleden door schietoefening in ossendrecht
	
	
	\bibitem{nos22032016ossendrecht} ... \LaTeX:\\ \url{https://amp.nos.nl/artikel/2094524-militair-omgekomen-bij-schietoefening-ossendrecht.html}
	
	\bibitem{ovv04042016lessenongevalossendrecht} ... \LaTeX:\\ \url{https://www.onderzoeksraad.nl/nl/page/4293/lessen-uit-schietongeval-ossendrecht}
	
	\bibitem{quekelboere10052017doodossendrecht} ... \LaTeX:\\ \url{https://www.bndestem.nl/bergen-op-zoom/dood-van-militair-sander-klap-35-in-ossendrecht-was-ongeluk-militairen-vrijuit-hij-probeerde-zijn-leven-te-redden~afe4c7a0/}
	
	
	Wat is de rol van defensie?
	Wat is er gedaan om de veligheid van de medewerkers te waarborgen?
	Waarom zijn deze regels niet nageleefd?
	Wat zijn de gevolgen?
	Zijn de acties die naderhand zijn ondernomen wel redelijk naar de slachtoffers, het nationale veiligheisbeeld en de medewerkers?
	
	
	
	
	
	
	
	%%%%%%%%%%%%%%%%%%%%%%%%%%%%%%%%%%%%%%%%%%%%%%%%%%%%%%%%%%%%%%%%%
	schipholbrand
	
	Wat is er gebeurd?
	\bibitem{wikiSchipholbrand} ... \LaTeX:\\ \url{https://nl.wikipedia.org/wiki/Schipholbrand}
	artikel
	\bibitem{schipholbrand27102005video} ... \LaTeX:\\ \url{https://www.youtube.com/watch?v=1i-hfEzxFfk}
	psychologische gevolgen
	rapport
	\bibitem{onderzoeksraad2610schipholoost} ... \LaTeX:\\ \url{https://www.onderzoeksraad.nl/nl/page/392/brand-cellencomplex-schiphol-oost-nacht-van-26-op-27-oktober}
	artikel met video
	herdenking
	impact op de persoon
	herdenking
	\bibitem{schipholbrandvideoargos} ... \LaTeX:\\ \url{https://www.vpro.nl/argos/speel~POMS_VPRO_461907~schadevergoeding-voor-ex-verdachte-schipholbrand~.html}
	chronologie
	\bibitem{nunl30052023feitenoverzicht} ... \LaTeX:\\ \url{https://www.nu.nl/binnenland/3355935/feitenoverzicht-schipholbrand-en-rechtszaken.html}
	tijdlijn
	\bibitem{ID} ... \LaTeX:\\ \url{https://www.singeluitgeverijen.nl/isbn/de-schipholbrand/}
	vervolgens van ministers
	beeldanalyse en reconstructie
	\bibitem{ID} ... \LaTeX:\\ \url{https://eenvandaag.avrotros.nl/item/schipholbrand-niet-ontstaan-in-cel-11/}
	herdenking
	korte samenvatting
	rapport
	artikel
	verwijzing naar het rapport vanuit de politieke oppositie
	beeld vanuit de gevangenisbewaarder
	nationaliteit slachtoffers schipholbrand
	verblijfsvergunning voor de slachtoffers
	gen schadevergoeding voor de verdachte
	verdachte voor de rechter
	geen schadevergoeding voor verdachte
	artikel wat ging er mis bji de schipholbrand
	brand veroorzaakt door een peuk
	smaadschrift
	bewakers worden niet vervolgd
	proces schipholbrand moet over en de brandveilgheid moet worden verbeterd
	de rol van het parlement in de evaluatie
	\bibitem{parlementairemonitorschipholbrand} ... \LaTeX:\\ \url{https://www.parlementairemonitor.nl/9353000/1/j9vvij5epmj1ey0/vi3aof7awcxg}
	onderzoeksmemo
	herdenking
	\bibitem{ID} ... \LaTeX:\\ \url{https://archief.ntr.nl/nova/page/detail/uitzendingen/3847/Den%20Haag%20Vandaag_%20herdenking%20Schipholbrand.html}
	herdenking
	invloed van de ramp op samenleving
	\bibitem{videonpoNOVA13112008} ... \LaTeX:\\ \url{https://www.npostart.nl/heropen-onderzoek-schipholbrand/13-11-2008/POMS_NTR_103332}
	opmerkelijk rapport gestolen in de nasleep
	\bibitem{schipperSchipholbrand} ... \LaTeX:\\ \url{https://www.nd.nl/nieuws/nederland/600395/schipholbrand-blijft-schrijnen}
	\bibitem{ID} ... \LaTeX:\\ \url{https://www.ed.nl/economie/om-geen-schadevergoeding-voor-verdachte-schipholbrand~a6c7c51d/63042600/?referrer=https%3A%2F%2Fwww.google.com%2F}
	\bibitem{ID} ... \LaTeX:\\ \url{https://www.groene.nl/artikel/schipholbrand-vereist-debat}
	\bibitem{rizoomes01052014schipholbrand} ... \LaTeX:\\ \url{https://www.rizoomes.nl/brandweer/brand-cellencomplex-schiphol/}
	
	
	
	publicaties
	\bibitem{heuvelkroesschipholbrandcamerabeelden} ... \LaTeX:\\ \url{http://www.msnp.nl/downloads/Onderzoeksmemo%20beeldanalyse%20Schipholbrand%20prot.pdf}
	\bibitem{ID} ... \LaTeX:\\ \url{http://www.dakweb.nl/roofs/2006-10/RH10-P30-31.pdf}
	\bibitem{ID} ... \LaTeX:\\ \url{https://www.delta.tudelft.nl/article/dood-door-zuinigheid}
	\bibitem{ID} ... \LaTeX:\\ \url{https://www.onderzoeksraad.nl/nl/page/392/brand-cellencomplex-schiphol-oost-nacht-van-26-op-27-oktober}
	Wat waren de regels destijds?
	Waren de autoriteiten in staat om op tijd in te grijpen of om erger te voorkomen?
	Wat is er gedaan om de veiligheid van illegalen en gevangenissbewaarders te verbeteren
	
	
	
	%%%%%%%%%%%%%%%%%%%%%%%%%%%%%%%%%%%%%%%%%%%%%%%%%%%%%%%%%%%%%%%%%
	vuurwerkramp enschede
	\bibitem{ID} ... \LaTeX:\\ \url{https://www.youtube.com/watch?v=OMkIsj8FsHw}
	\bibitem{ID} ... \LaTeX:\\ \url{https://depot03.archiefweb.eu/archives/archiefweb/20210703085353/http://www.vuurwerkramp.enschede.nl/publicaties/00005/#.YOAlp-gzaUk}
	
	Wat waren de afspraken omtrent vuurwerkopslag?
	Waarom werden de voorschriften neit nageleefd?
	
	
	%%%%%%%%%%%%%%%%%%%%%%%%%%%%%%%%%%%%%%%%%%%%%%%%%%%%%%%%%%%%%%%%%
	
	
	explosie in beirut
	\bibitem{landryalameddine12112020beiruthelathsystem} ... \LaTeX:\\ \url{https://bmchealthservres.biomedcentral.com/articles/10.1186/s12913-020-05906-y}
	\bibitem{ID} ... \LaTeX:\\ \url{https://news.sky.com/story/beirut-blast-cctv-captures-moment-huge-explosion-devastated-hospital-12047452}
	\bibitem{ID} ... \LaTeX:\\ \url{https://www.unodc.org/unodc/en/frontpage/2020/September/unodc-assists-lebanon-in-reestablishing-container-shipments-in-the-aftermath-of-the-port-of-beirut-explosion.html}
	\bibitem{ID} ... \LaTeX:\\ \url{https://reliefweb.int/sites/reliefweb.int/files/resources/LEB201-Lebanon-Emergency-Response.pdf}
	\bibitem{yadav07082020handlingexplosivesBeirut} ... \LaTeX:\\ \url{https://www.downtoearth.org.in/news/governance/beirut-blast-lessons-time-for-india-to-strengthen-handling-of-explosives-chemicals-72707}
	\bibitem{graham21082020rootsImpactBeirutBlast} ... \LaTeX:\\ \url{https://www.justsecurity.org/72122/the-cost-of-resilience-the-roots-and-impacts-of-the-beirut-blast/}
	\bibitem{ID} ... \LaTeX:\\ \url{https://www.fire-magazine.com/the-port-of-beirut-explosion-a-timely-reminder}
	\bibitem{neusaeter07082020beirutexplosioneval} ... \LaTeX:\\ \url{https://www.ctvnews.ca/sci-tech/mapping-the-beirut-explosion-what-the-impact-would-look-like-in-canadian-cities-1.5053932}
	
	
	secyrity:
	\bibitem{ID} ... \LaTeX:\\ \url{https://permanent.fdlp.gov/gpo45474/AN_advisory.pdf}
	
	
	secyrity:
	\bibitem{ID} ... \LaTeX:\\ \url{https://permanent.fdlp.gov/gpo45474/AN_advisory.pdf}
	
	
	
	
	
	
	
	
	%%%%%%%%%%%%%%%%%%%%%%%%%%%%%%%%%%%%%%%%%%%%%%%%%%%%%%%%%%%%%%%%%
	bijlmerramp
	
	
	
	
	%%%%%%%%%%%%%%%%%%%%%%%%%%%%%%%%%%%%%%%%%%%%%%%%%%%%%%%%%%%%%%%%%
	slmramp
	Wat is er gebeurd?
	\bibitem{ID} ... \LaTeX:\\ \url{https://www.srnieuws.com/suriname/290721/slm-ramp-herdacht/}
	\bibitem{ID} ... \LaTeX:\\ \url{https://werkgroepcaraibischeletteren.nl/documentaire-waarom-nou-jij-over-de-slm-ramp-in-89/}
	\bibitem{ID} ... \LaTeX:\\ \url{https://www.vpro.nl/speel~WO_NTR_15390142~andere-tijden-17-apr-2019-3-09-min-fouten-en-misstanden-leiden-tot-de-slm-ramp~.html}
	\bibitem{ID} ... \LaTeX:\\ \url{https://www.canonvannederland.nl/nl/kalender/06/1989-06-07}
	\bibitem{ID} ... \LaTeX:\\ \url{https://vijfeeuwenmigratie.nl/archief-herdenkingen-slm-ramp}
	\bibitem{ID} ... \LaTeX:\\ \url{https://www.hulpverleningsforum.nl/index.php?topic=84702.0}
	\bibitem{ID} ... \LaTeX:\\ \url{https://www.nporadio1.nl/fragmenten/focus/f792e720-bd85-4c18-8a71-b334d9d5de7e/2019-04-17-slm-ramp-een-paar-cowboys-hebben-achter-de-stuurknuppel-gezeten}
	\bibitem{ID} ... \LaTeX:\\ \url{https://www.waterkant.net/suriname/2017/06/07/herdenking-slm-ramp-28-jaar-geleden-suriname/}
	\bibitem{espnSLMterugblik} ... \LaTeX:\\ \url{https://www.espn.nl/video/clip?id=8744942}
	\bibitem{ID} ... \LaTeX:\\ \url{http://www.themediabrothers.nl/tag/slm-ramp/}
	\bibitem{ID} ... \LaTeX:\\ \url{https://www.rijnmond.nl/nieuws/182546/30-jaar-na-de-SLM-ramp-Ik-mis-mijn-broer-nog-elke-dag}
	\bibitem{dennisRosier01052020} ... \LaTeX:\\ \url{https://www.voetbalkrant.com/nieuws/2020-05-01/het-vergeten-verhaal-van-de-slm-ramp}
	\bibitem{hassing07062020slmramp} ... \LaTeX:\\ \url{https://www.bd.nl/sport/de-slm-ramp-en-het-hartverscheurende-verhaal-van-jerry-en-winnie-haatrecht~ae4ce105/?referrer=https%3A%2F%2Fwww.google.com%2F}
	\bibitem{amsterdamArchiefSLM} ... \LaTeX:\\ \url{https://www.amsterdam.nl/stadsarchief/nieuws/slm-ramp/}
	\bibitem{rtvOost06062019nabestaande} ... \LaTeX:\\ \url{https://www.rtvoost.nl/nieuws/313496/Nabestaande-SLM-ramp-Heb-ik-wel-mijn-broer-en-moeder-begraven}
	\bibitem{breda07062021AndroSnel} ... \LaTeX:\\ \url{https://www.bredavandaag.nl/nieuws/algemeen/337919/nac-herdenkt-andro-knel-slm-ramp-precies-32-jaar-geleden}
	\bibitem{andereTijdenSLMCrash} ... \LaTeX:\\ \url{https://www.anderetijden.nl/aflevering/792/Een-aangekondigde-vliegramp}
	\bibitem{aviationSLMCrashAccidentInvestigation} ... \LaTeX:\\ \url{https://nl.wikipedia.org/wiki/SLM-ramp}
	database
	\bibitem{aviationReport} ... \LaTeX:\\ \url{https://aviation-safety.net/database/record.php?id=19890607-2}
	rapport
	\bibitem{aviationSLMCrashAccidentInvestigation} ... \LaTeX:\\ \url{https://reports.aviation-safety.net/1989/19890607-2_DC86_N1809E.pdf}
	\bibitem{mcDonnelDouglasCommissionReportSLMCrash} ... \LaTeX:\\ \url{https://aviation-safety.net/investigation/cvr/transcripts/cvr_py764.php}
	\bibitem{wikiSRFlight764} ... \LaTeX:\\ \url{https://en.wikipedia.org/wiki/Surinam_Airways_Flight_764}
	\bibitem{ID} ... \LaTeX:\\ \url{https://web.archive.org/web/20050113010822/https://www.ntsb.gov/ntsb/brief.asp?ev_id=34510&key=0}
	\bibitem{nos07062019SLMTerugblik} ... \LaTeX:\\ \url{https://nos.nl/artikel/2287986-slm-vliegramp-van-precies-30-jaar-geleden-trof-ook-nederlands-voetbal}
	\bibitem{dagvantoenSLMCrash} ... \LaTeX:\\ \url{https://www.dagvantoen.nl/vliegtuigcrash-slm-bij-zanderij-meer-dan-170-doden/}
	\bibitem{waterkantNesty07061989} ... \LaTeX:\\ \url{https://www.waterkant.net/suriname/2006/06/07/vliegramp-suriname-op-7-juni-1989-2/}
	uitgebreid engels artikel
	\bibitem{eduNandlalSRCrash} ... \LaTeX:\\ \url{http://www.edufd.nl/planecrash/}
	ntsb investigtion
	\bibitem{oldjetsSRAirways} ... \LaTeX:\\ \url{http://www.oldjets.net/slm-dc-8-crash.html}
	uitgebreid engels artikel
	\bibitem{cloudberg02012021srflight764} ... \LaTeX:\\ \url{https://admiralcloudberg.medium.com/contract-to-kill-the-crash-of-surinam-airways-flight-764-828979c7efe2}
	persbericht
	\bibitem{apnews07061989srplanecrash} ... \LaTeX:\\ \url{https://apnews.com/article/5b240d758ee4c5422381cc7cdc98566b}
	Wat is de rol van de autoriteiten?
	Welke andere betrokkeen? Enw at is hun verantwoordelijkheid
	Hadden de negatieve gevolgen voorkomen kunnen worden?
	Hoe werd er over veiligheid gedacht?
	
	
	
	
	%%%%%%%%%%%%%%%%%%%%%%%%%%%%%%%%%%%%%%%%%%%%%%%%%%%%%%%%%%%%%%%%%
	Tsjernobyl
	\bibitem{ID} ... \LaTeX:\\ \url{https://www.youtube.com/watch?v=Xw3SFOfbR84}
	\bibitem{wikiTjernobyl} ... \LaTeX:\\ \url{https://nl.wikipedia.org/wiki/Kernramp_van_Tsjernobyl}
	\bibitem{rivmTjernobyl} ... \LaTeX:\\ \url{https://www.rivm.nl/straling-en-radioactiviteit/stralingsincidenten-en-kernongevallen/tsjernobyl}
	\bibitem{andereTijdenTjernobyl} ... \LaTeX:\\ \url{https://www.anderetijden.nl/aflevering/599/Tsjernobyl-als-Nederlandse-ramp}
	wat er is gebeurd en hoe het leven verdergaat
	\bibitem{kingskey19042022tjernobyl} ... \LaTeX:\\ \url{https://www.nationalgeographic.nl/het-leven-in-tsjernobyl-gaat-door}
	pernsioenfondsen en de tjernobyl ramp
	In 2021 worden mensen nog steeds blootgesteld blijkt ut een gezamelijk onderzoek van greenpeace en oekraiense wetenschappers
	stijging van de nucliaire activiteit gemeten in tjernobyl
	Het toerisme  aspect
	De chronologie
	\bibitem{erikbork26042023reactor4} ... \LaTeX:\\ \url{https://historianet.nl/maatschappij/rampen/tsjernobyl-atoomhel-bij-reactor-4}
	\bibitem{nosTjernobyl30jaarlater} ... \LaTeX:\\ \url{https://nos.nl/artikel/2101523-de-spookstad-van-tsjernobyl-30-jaar-later}
	Dieren in de omgeving van tjernobyl
	De chronologie
	Echtreme droogte zorgd voor gevaar
	\bibitem{knmi04052021tjernobylbosbrand} ... \LaTeX:\\ \url{https://www.knmi.nl/over-het-knmi/nieuws/35-jaar-na-tsjernobyl-liggen-branden-op-de-loer}
	\bibitem{dodonovaKVIRisicoTjernobyl} ... \LaTeX:\\ \url{https://www.kivi.nl/afdelingen/risicobeheer-en-techniek/columns/kernramp-tsjernobyl-het-dilemma-van-scherbitsky}
	Joernalistiek, entertainment en de waarheid
	\bibitem{dumarey04062020verhaalTjernobylWaarheid} ... \LaTeX:\\ \url{https://www.vrt.be/vrtnws/nl/2020/04/06/in-de-ban-van-tsjernobyl-vooruitblik/}
	Een onderzoek
	
	Huidige gevolgen van de explosie van toen
	\bibitem{sparkesNewScientistTjernoby;} ... \LaTeX:\\ \url{https://www.newscientist.nl/nieuws/steeds-meer-kernreacties-in-ontoegankelijke-ruimte-in-tsjernobyl/}
	De ramp, hoe de mensen ermee omgingen en hoe er nu geleef wordt
	
	evaluatieonderzoek en amatregeen
	\bibitem{kernenergiened26041986chronologiemaatregelen} ... \LaTeX:\\ \url{https://www.kernenergieinnederland.nl/node/308}
	\bibitem{mapszoneReactor} ... \LaTeX:\\ \url{https://www.google.com/maps/d/u/0/viewer?ie=UTF8&hl=nl&t=h&msa=0&ll=51.388923%2C30.099792&spn=0.685583%2C1.645203&z=9&source=embed&mid=1MLcOcMK_WrIJYMuTf0VVuYnMqQI}
	Invloed van de mens op de omgeving
	\bibitem{ID} ... \LaTeX:\\ \url{https://www.animalstoday.nl/mens-schadelijker-natuur-tsjernobyl/}
	Heroplevende splijtingsreacties
	docu van schooltv
	Radioactiviteit bereikt nederland
	documentaire en maatregelen
	\bibitem{kernhistoriek15062021tjernobyl} ... \LaTeX:\\ \url{https://historiek.net/kernramp-van-tsjernobyl-1986/8769/}
	Het verhaal van een overledende
	Toerisme
	toerisme
	toerisme
	Dieren in de omgevong
	Toevluchtsoord voor vluchtelingen van de oorlog met russische seperatisten
	Ouderen die terugkeerden naar hun woonplaats na de gedwongen verhuizing door de autoriteiten
	De straling neemt weer toe
	Lessen geleerd van tjernobyl
	\bibitem{nucleairforumFeitenTjernobyl} ... \LaTeX:\\ \url{https://www.nucleairforum.be/thema/veiligheid-als-prioriteit/tsjernobyl-de-feiten}
	Toerisme
	Bosbrand in tjernobyl
	invloed van de ramp op belgie
	\bibitem{kernongevalTjernobylFancGov} ... \LaTeX:\\ \url{https://fanc.fgov.be/nl/noodsituaties/zware-ongevallen-het-buitenland/1986-kernongeval-tsjernobyl}
	Boek recensie
	Fotos en berekeningen
	ontmanteling en toerisme
	Belangrijke lessen en overeenkomsten
	De journalistieke waarheid van de koude oorlog
	De lessen van
	\bibitem{arendswolters062019lessenTjernobyl} ... \LaTeX:\\ \url{https://magazines.autoriteitnvs.nl/nieuwsbrief-anvs/2019/02/de-lessen-van-tsjernobyl}
	Een toristenattractie maken van tjernobyl
	De radioactieve straling toen en nu
	de 30km zone door de ogen van toeristen
	artikel
	stedentrip
	rapport
	\bibitem{damveld08052020tjernobyl} ... \LaTeX:\\ \url{https://wisenederland.nl/wp-content/uploads/2020/06/TSJERNOBYL.pdf}
	slapend monster
	docu
	krantenartikel
	hbo serie
	docuserie
	de  nieuwe sacrofaag
	hulp aan slachtoffers
	slapende reactor
	krantenartikel
	\bibitem{deVriestjernobylHolland} ... \LaTeX:\\ \url{https://onh.nl/verhaal/besmette-melk-en-radioactieve-spinazie-tsjernobyl-in-holland}
	hbo serie
	internationale gevolgen
	toerisme
	nieuwe koepel
	media communicatie
	docu
	dieren
	\bibitem{ID} ... \LaTeX:\\ \url{https://www.amboanthos.nl/boek/nacht-in-tsjernobyl/}
	koepel
	koepel
	\bibitem{ing3enieur29042015antistralingskoepel} ... \LaTeX:\\ \url{https://www.deingenieur.nl/artikel/nieuwe-antistralingskoepel-tsjernobyl-bijna-af}
	toerisme
	toeristisch reiperspectief
	toerisme
	niwe koepel
	overschakelen naar duurzaamheid
	docu
	tjernobyl wekt nu duurazme energie
	toerisme
	overeenkomsten tjernobyl en fukushima
	drank en sla uit tjernobyl
	geen efficiente opslag is mogelijk
	
	wetenschappelijke artikelen
	
	zaterdag 26 april 1986. Er vind routineonderhoud plaats bij reactor 4, De controle wordt uitegevoerd door de dagploeg. Vnwege een test wordt jhet koelsysteem uitgeschakeld. Door omstandigheden wordt de test uitgesteld en wordt de verantwoordelijkheid overgedragen aan de avondploeg.
	De operator maakt bedieningsfouten waardoot de reactor bijna stil komt te liggen. En vervolgens probeert hij de reactor weer op gang te brengen. ondanks de snelle temperatuurstijging wordt het experiment doorgezet. Dan wordt ook het veiligheidssysteem stilgelgd. Terwijl het koelwater langzaam opwarmt, sluit hij de klep waarlangs de stoom naar de generator stroomt.
	
	De temperatuur van de reactorstaven neemt daarna snel toe. Terwijl er een oncontroleerbare kettingreactie op gang komt, laat het personeel in paniek de regelstaven zakken om de warmteontwikkeling af te remmen. Het is dan echter al te laat. Door een ontwerpfout loopt het vermogen razendsnel op tot 33.000 megawatt, ruim tien keer hoger dan normaal.
	
	In een oogwenk verandert al het koelwater in stoom. De ontploffing die daarop volgt, blaast het 2000 ton zware deksel van de reactor af.}

In de ravage vat het gloeiend hete grafiet in de reactor spontaan vlam. De uitslaande brand en een tweede explosie voeren een radioactieve rookwolk tot 8 kilometer hoogte. 
In een poging het vuur in reactor 4 te doven, storten helikopters vanuit de lucht zand, lood en boorzuur in de reactorkern. Het mag echter niet baten.

Intussen is de nucleaire brandstof zo heet geworden dat die door de bodem van het reactorvat dreigt te smelten. Als dat gebeurt, kan het bluswater onder het vat in één klap verdampen en dreigt een derde explosie die een groot deel van Europa onbewoonbaar zal maken. Om dit te voorkomen moet het water hoe dan ook worden weggepompt.

Drie brandweermannen wagen zich daarvoor in de ruimte onder de reactor, blootgesteld aan 300 sievert per uur, 300.000 keer de dosis die een Nederlander jaarlijks maximaal mag oplopen. Ze slagen daarin, maar twee van hen overlijden enkele dagen later aan acute stralingsziekte.

Hoewel geigertellers de dag na de ramp onrustbarende waarden aangeven, slaat het plaatselijk bestuur geen alarm. De bevolking is het niet gewend om vragen te stellen.

De volgende dag blijkt er wel degelijk iets ernstigs aan de hand te zijn. In een lange rij bussen worden de 135.000 inwoners op 27 april uit het besmette gebied geëvacueerd, om er nooit meer terug te keren.

De ramp is dan nog steeds geen wereldnieuws. De Sovjetautoriteiten blijken er niet eens van op de hoogte te zijn – president Gorbatsjov klaagt later dat hij via Zweden aan zijn informatie moest komen.


\bibitem{verschuur14012013tjernobylreports} ... \LaTeX:\\ \url{http://essay.utwente.nl/63353/1/Verschuur,_W._-_s0123617_(verslag).pdf}
\bibitem{paperlessarchivesTjernobyl} ... \LaTeX:\\ \url{https://www.paperlessarchives.com/chernobyl_nuclear_accident_doc.html}
\bibitem{vargos082000tjernobylconcerns} ... \LaTeX:\\ \url{https://www.pnnl.gov/main/publications/external/technical_reports/pnnl-13294.pdf}
\bibitem{mauroNuclearRiskSociety} ... \LaTeX:\\ \url{http://www.geocities.ws/scannapuerci/demauroinnovation.pdf}
\bibitem{vienna06092005LookingBack} ... \LaTeX:\\ \url{https://www-pub.iaea.org/MTCD/publications/PDF/Pub1312_web.pdf}



%%%%%%%%%%%%%%%%%%%%%%%%%%%%%%%%%%%%%%%%%%%%%%%%%%%%%%%%%%%%%%%%%

MH17




%%%%%%%%%%%%%%%%%%%%%%%%%%%%%%%%%%%%%%%%%%%%%%%%%%%%%%%%%%%%%%%%%
oekraine powergrid
\bibitem{owens21032017ukrainemitigationstrategies} ... \LaTeX:\\ \url{https://na.eventscloud.com/file_uploads/aed4bc20e84d2839b83c18bcba7e2876_Owens1.pdf}
\bibitem{ID} ... \LaTeX:\\ \url{https://www.us-cert.gov/ics/alerts/IR-ALERT-H-16-056-01}

\bibitem{cerulus2019FrontlineRussiaAttack} ... \LaTeX:\\ \url{https://www.politico.eu/article/ukraine-cyber-war-frontline-russia-malware-attacks/}
\bibitem{ID} ... \LaTeX:\\ \url{https://en.wikipedia.org/wiki/December_2015_Ukraine_power_grid_cyberattack}
\bibitem{grammatikis2019AttackIEC6087505104} ... \LaTeX:\\ \url{https://www.researchgate.net/publication/333671061_Attacking_IEC-60870-5-104_SCADA_Systems}
\bibitem{ID} ... \LaTeX:\\ \url{https://ris.utwente.nl/ws/files/6028066/3-s2_0-B9780128015957000227.pdf}
\bibitem{hidajat2016ScadaSimulator} ... \LaTeX:\\ \url{https://www.diva-portal.org/smash/get/diva2:1046339/FULLTEXT01.pdf}
\bibitem{ID} ... \LaTeX:\\ \url{https://www.semanticscholar.org/paper/Cybersecurity-analysis-of-a-SCADA-system-under-and-Rocha/dfa7c12551ebe7b24da8d806e87e946051a57cb9}
\bibitem{ID} ... \LaTeX:\\ \url{https://tutcris.tut.fi/portal/files/16294332/jafary_1534.pdf}
\bibitem{ID} ... \LaTeX:\\ \url{http://blog.nettedautomation.com/2017/}
\bibitem{uscert20072021crashmalware} ... \LaTeX:\\ \url{https://www.us-cert.gov/ncas/alerts/TA17-163A}
\bibitem{zetter12062017malwareanalysis} ... \LaTeX:\\ \url{https://www.vice.com/en_us/article/zmeyg8/ukraine-power-grid-malware-crashoverride-industroyer}
\bibitem{icsRussianHackingCyberWeapon} ... \LaTeX:\\ \url{http://blog.wallix.com/ics-security-russian-hacking}
\bibitem{usgovC2M2} ... \LaTeX:\\ \url{https://www.energy.gov/ceser/activities/cybersecurity-critical-energy-infrastructure/energy-sector-cybersecurity-0}

bediening werking schutsluizen pdf
\bibitem{varendoejesamenVeiligSluisvaren} ... \LaTeX:\\ \url{https://www.varendoejesamen.nl/storage/app/media/downloads/vlot-en-veilig-door-brug-en-sluis-.pdf}
\bibitem{vlaamsewaterwegen012014} ... \LaTeX:\\ \url{http://www.scarphout.be/assets/bedieningstijden2014.pdf}
\bibitem{bardetsluizenAmsterdam} ... \LaTeX:\\ \url{https://www.theobakker.net/pdf/sluizen.pdf}
\bibitem{dvsbedieningsluizenenbruggen} ... \LaTeX:\\ \url{http://www.watersportalmanak.nl/files/File/Brugbediengstijden_watersport.pdf}
\bibitem{crowstappenplanmachinerichtlijnen} ... \LaTeX:\\ \url{https://www.crow.nl/downloads/pdf/verkeer-en-vervoer/verkeersmanagement/verkeersregelinstallaties/stappenplan-machinerichtlijnen_web.aspx}
\bibitem{rijksoverheidrwsonderzoeksrapporten} ... \LaTeX:\\ \url{https://puc.overheid.nl/rijkswaterstaat/doc/PUC_95170_31/}
\bibitem{bedieningstijdensluizenenbruggen} ... \LaTeX:\\ \url{http://wsv.wsvdegors.nl/wp-content/uploads/2017/05/Bedieningstijden_201701.pdf}
\bibitem{sluiscomplexkornwerderzand} ... \LaTeX:\\ \url{https://www.commissiemer.nl/projectdocumenten/00004717.pdf}
\bibitem{varengroningendrenthe} ... \LaTeX:\\ \url{https://tasmanroutes.nl/wp-content/uploads/docs/1900-bedieningstijden-groningen-drenthe.pdf}
\bibitem{wallemuldergetijhoogteverschillen} ... \LaTeX:\\ \url{http://www.vliz.be/docs/groterede/GR21_Zeesluis.pdf}
\bibitem{biemandssluizenLith} ... \LaTeX:\\ \url{https://www.bhic.nl/media/document/file/rien-biemans-sluis-en-stuw-bij-lith.pdf}
\bibitem{weilerburgers11062018zoutindringingschutsluizen} ... \LaTeX:\\ \url{https://www.nattekunstwerkenvandetoekomst.nl/upload/documents/tinymce/KpNK-2017-SKW-01c001-v1-Zoutindringing-door-schutsluizen-overzicht-projecten-en-aanzet-formulering-tbv-netwerkmodellen.pdf}
\bibitem{rwsrichtlijnvaarwegen2011} ... \LaTeX:\\ \url{https://www.arnhemspeil.nl/nap/dok/2011-12-00-rijkswaterstaat-richtlijnen-vaarwegen.pdf}
\bibitem{slotterwisscha02062016swimwaywaddenzee} ... \LaTeX:\\ \url{https://rijkewaddenzee.nl/wp-content/uploads/2016/08/Inventarisatie-toestand-vispasseerbaarheid-zoet-zout-overgangen-Waddenzee-2-6-2016-PRW-rapportage-Definitief.pdf}
\bibitem{visvriendelijksluisbeheer06012014} ... \LaTeX:\\ \url{http://www.nevepaling.nl/files/Image//nederlands/informatiecentrum/2014-definitieve-voorkeursvariantennotitie-visvriendelijk-sluisbeheer-afsluitdijk-en-houtribdijk//2014_definitieve_voorkeursvariantennotitie_visvriendelijk_sluisbeheer_afsluitdijk_en_houtribdijk.pdf}
\bibitem{ID} ... \LaTeX:\\ \url{https://www.ifv.nl/kennisplein/Documents/20120614-BwNL-Handboek-brandbeveiligingsinstallaties.pdf}
\bibitem{aubel112016sluiscomplexHeumen} ... \LaTeX:\\ \url{https://ienc-kennisportaal.nl/wp-content/uploads/2017/01/Objectbeschrijving-Heumen.pdf}
\bibitem{arends052005schutsluisstolwijkschevliet} ... \LaTeX:\\ \url{https://library.wur.nl/edepot/websites/stolwijkersluis/presentatie-data/data/pdf/TUDelft-bouwhistorisch-onderzoek.pdf}
\bibitem{multidomeinbediening} ... \LaTeX:\\ \url{https://www.icentrale.nl/wp-content/uploads/bsk-pdf-manager/2019/01/20170929_Project-2.02-Deliverable-Gehele-werkpakket-2.02.pdf}
\bibitem{historischesluizen} ... \LaTeX:\\ \url{https://www.stowa.nl/sites/default/files/assets/PUBLICATIES/Publicaties%202000-2010/Publicaties%202000-2004/STOWA%202004-XX%20boekenreeks%2020.pdf}
\bibitem{nmmag2017centralebediening} ... \LaTeX:\\ \url{https://www.nm-magazine.nl/pdf/NM_Magazine_2017-3.pdf}
\bibitem{knooppuntvaarwegenFryslanGroningenDrenthe} ... \LaTeX:\\ \url{https://www.varendoejesamen.nl/storage/app/media/knooppunten/knooppuntenboekje_03_Friesland_Groningen_Drenthe.pdf}
\bibitem{afsluitdijk052015} ... \LaTeX:\\ \url{https://deafsluitdijk.nl/wp-content/uploads/2014/05/Plan-project-MER-Afsluitdijk.pdf}
\bibitem{ID} ... \LaTeX:\\ \url{file:///C:/Users/gally/Downloads/vaarroutekaart_provincie_drenthe.pdf}
\bibitem{ID} ... \LaTeX:\\ \url{file:///C:/Users/gally/Downloads/6227_watermanagement_nl_dv.pdf}
\bibitem{ID} ... \LaTeX:\\ \url{file:///C:/Users/gally/Downloads/applsci-11-00092-v3.pdf}
\bibitem{ID} ... \LaTeX:\\ \url{file:///C:/Users/gally/Downloads/Bedieningstijden_sluizen_en_bruggen_2004.pdf.pdf}
\bibitem{ID} ... \LaTeX:\\ \url{file:///C:/Users/gally/Downloads/Bedieningstijden.pdf}
\bibitem{ID} ... \LaTeX:\\ \url{file:///C:/Users/gally/Downloads/bijlagerapport_c_-_analyse_geavanceerd_definitief_v1_0.pdf}
\bibitem{ID} ... \LaTeX:\\ \url{file:///C:/Users/gally/Downloads/BIT-advies+Bediening+op+Afstand,+sluizen+en+bruggen+in+Friesland.pdf}
\bibitem{ID} ... \LaTeX:\\ \url{file:///C:/Users/gally/Downloads/conceptverordeningnautischbeheerzuid-holland.pdf}
\bibitem{ID} ... \LaTeX:\\ \url{file:///C:/Users/gally/Downloads/De_Deltawerken_Cultuurhistorie_ontwerpgeschiedenis_web-A.pdf}
\bibitem{ID} ... \LaTeX:\\ \url{file:///C:/Users/gally/Downloads/duurzaamheid_bij_de_ontwikkeling_van_reevesluis.pdf}
\bibitem{ID} ... \LaTeX:\\ \url{file:///C:/Users/gally/Downloads/gebruikershandleiding-databank-vismigratie.pdf}
\bibitem{ID} ... \LaTeX:\\ \url{file:///C:/Users/gally/Downloads/onderzoek_vispasseerbaarheid_sluizen_zuid_holland_2016_definitief_16-5-20171.pdf}
\bibitem{ID} ... \LaTeX:\\ \url{file:///C:/Users/gally/Downloads/rapport-veiligheid-van-op-afstand-bediende-bruggen.pdf}
\bibitem{ID} ... \LaTeX:\\ \url{file:///C:/Users/gally/Downloads/richtlijnen-vaarwegen-2017_tcm21-127359%20(1).pdf}
\bibitem{ID} ... \LaTeX:\\ \url{file:///C:/Users/gally/Downloads/richtlijnen-vaarwegen-2017_tcm21-127359.pdf}
\bibitem{ID} ... \LaTeX:\\ \url{file:///C:/Users/gally/Downloads/TPE_2342144_20210511223747_PEU_52457169.pdf}


Wat hebben alle bovenstaande rampen/ongelukken gemeen? Veiligheid.
Bij de therac waren er diverse problemen: communicatie, doorontwikkeling, controle en toetsing
Was het makkelijk te onderzoeken? Waarom?
Bij de boeing 737 crashes was het probleem van controle en communicatie naar medewerkers
Was het makkelijk te onderzoeken? Waarom?

Uit de evaluatie van de china explosion 2015 tianjin komt naar voren dat communicatie, transparantie en veiligheid niet altijd prioriteit hadden bij de lokale autoriteiten
Was het makkelijk te onderzoeken? Waarom?

Bij de tesla autopilot crashes komen soms onvoldoende onderbouwde ontwerpkeuzes naar voren die niet goed zij  afgewogen tegenover het gedrag van de bestuurder
vlucht 1951
Was het makkelijk te onderzoeken? Waarom?

De ramp in Tsjernobyl toont aan hoe autoriteiten een ramp in de doofpot proberen te stoppen
Was het makkelijk te onderzoeken? Waarom?



Wat heb ik geleerd
Ik heb erg veel geleerd van het veilig opzetten van VPN’s. Een VPN opzettenhad ik namelijk nog nooit gedaan. Het opzetten van SSH en het aanmaken vanVM’s was al bekend. Ook had ik nog nooit met UDP sockets geprogrammeerd.Verder heb ik geleerd hoe ik in de praktijk een VM in een VLAN kan zetten enhoe VLAN’s netwerken van elkaar kunnen scheiden.Het leukste onderdeel van het project, was dat wonderbaarlijk mijn gekozenoplossing elegant werkte. UDP Servers en clients zijn gerealiseerd met minderdan enkele regels logisch scipt. Ik had aan genomen dat het werken met socketsin shell absoluut rampzalig zou uitpakken. Ik ben blij dat het opdracht zo vrijwas, zodat ik experimenteel kon zijn met mijn implementatie.




\bibitem{ID} ... \LaTeX:\\ \url{https://www.uni-saarland.de/fileadmin/user_upload/Professoren/FreyG/DS_KT_GF_INCOM_May_2012.pdf}
vanaf 2.1 tot en met 5

\bibitem{ID} ... \LaTeX:\\ \url{http://www.lasid.ufba.br/publicacoes/artigos/Integrating+UML+and+UPPAAL+for+Designing,+Specifying+and+Verifying+Component-Based+Real-Time+Systems.pdf}

hf7
Reachability: i.e. some condition an posssibly be satisfied
Safety: i.e. some condition will never occur
Liveness: i.e. some condition wille eventually become true [] eventually or leadsto
hf 8
Het systeem is deadlockvrij
De wachttijd is altijd gelijk aan de invaarttijd _2x de nivlleertijd en de invaartijd van de overkant

\bibitem{ID} ... \LaTeX:\\ \url{https://www.diva-portal.org/smash/get/diva2:495691/FULLTEXT01.pdf}

blz 6 tot en met 10
\bibitem{ID} ... \LaTeX:\\ \url{https://www.cister-labs.pt/docs/formal_verification_of_aadl_models_using_uppaal/1331/view.pdf}

hf 3 geeft een voorbeeld van een template met guard en acies
De volgende automata worden gebruikt met hun lokale variabelen

De volgende globale variabelen

Een lijst met relevante einschappen van een schutsluis:

\bibitem{ID} ... \LaTeX:\\ \url{https://iopscience.iop.org/article/10.1088/1742-6596/1821/1/012031/pdf}

hf 5
deadlock

\bibitem{ID} ... \LaTeX:\\ \url{http://www.es.mdh.se/pdf_publications/2934.pdf}

hf 3 tool support
Modelling in UML
Code generation
Domain Model
Behaviour model
State Hierarchy
Transitions
Trigger methods
Time events
Effects
Requirements
Environment model

hf 4
\bibitem{ID} ... \LaTeX:\\ \url{https://files.ifi.uzh.ch/stiller/CLOSER%202014/WEBIST/WEBIST/Internet%20Technology/Full%20Papers/WEBIST_2014_130_CR.pdf}


\bibitem{ID} ... \LaTeX:\\ \url{https://files.ifi.uzh.ch/stiller/CLOSER%202014/WEBIST/WEBIST/Internet%20Technology/Full%20Papers/WEBIST_2014_130_CR.pdf}

4.2 5 en 6
Het Sluisbeheeerder model wordt getoond in fuguur[]. Het model is een uitbreiding van een schutsluis met alle condities en effecten. De kleuren in de automation verwijizen naar de kleuren in de staat van de automata . De template begint met een initiele lokatie start. De sluisbeheerder initieert het proces door een aangekomen schip te registreren metbehulp van een sychronizate met het channel... over de edge richtng de lokatie "aanmelden." Dit symboliseert een opstartprocedure, ook wordt een functie enqueeu_aanmeldLijst() gebruikt om de juiste waarden te geven aan lokale en globale avariabelen. De lokatie aanmelden regisseert het opstellen van schepen boven of beneden van de sluiskolk. De template Schip synchronizeerd met de template Sluisbeheerder met het channel move_down[id] of move_up[id] en bereikt daarmee de volgende lokatie afhankelijk af de sluis boven of beneden is worden de schepen die in de opstellijst voorkomen, max 2, klaargemaakt voor invaren.. De templates Stoplicht en sluisdeur synchroniseren met de channels ... call_Deur en call_stoplicht.
Het Sluisbeheerder model gebruikt de variabelen clock x, wachttijd_beneden, wachttijd_boven als invariant tussen de lokaties. Om op de hoogete te zijn van de invaar-/uitvaart van de verschillende schepen worden lijsten bijgehouden: list_wachtrij_beneden, list_pos_invaren_beneden, list_schepenInSluis, list_wachtrij_boven en list_pos_invaren_boven.

Het model voltooit de volgende transitie op basis van de waarde van de boolean sluis_bove en sluis_beneden. en de lokale klok variabele x.
Vanaf de locatie invaarverbod_gecontroleerd  wordt gecontroleerd of er nog invarende schepen zijn die in de sluiskolk passen.
Op de lokatie sluiskolk gereed zijn er 1 of meer schepen in de sluis. Als er nog plek is in de sluiskolk n er is nog een schip klaar om in te varen dan wordt dit gecontroleerd, de functie enqueu() voegt het schip toe aan de queue van de sluiskolk. De functie deque() verwijdert de schip van de lijst met invarende schepen. De variabele sluis_boven of sluis_beneden is waar, bij de switch voor het sluiten van deuren en het aanroepe van het stoplicht nr gelang de positie van  de laate binnenvarende schip (boven of beneden). Hierna bereikt de automation sluiskolk_afgesloten.



De lokatie start_nivelleren kiest op basis van de variabelen sluis_boven en de variabelen sluis_beneden het nivellereingsprogramma.
Heet nivellereingsprogramma is Aof B. De keuze voor het programma wordt bepaald door de variabelen van het schip dat in de sluis zit.

De lokatie klaarmaken_voor_openen wordt bereikt als de   hoogte van de sluis  door het nivellereingsprogramma is bereikt.
De positie van de kluis is bepaald door de schepen in de sluis. Vanuit deze lokatie wordt gekeken off de stoplichten gereed moeten worden gemaakt en of de sluisdeuren open mogen.
Hierna volgt een transiie waarin de stoplichte op groen worden gezet en de sluisdeuren worden geopend voor de uitvaart van de schepen in de sluis.
Als alle schepen zijn uitgevaren die uit moeten varen, worden de stoplichten op groen gezet en de deuren gesloten.


De lokatie uitvaren_toegestaan heeft een verbinding(edge) met de lokatie sluis_afsluiten.
Er is een select statement, e:id_t gebruikt als onderdeel van het prototocol om alle uitvarende schepen uit de queue van de sluiskolk te halen, en wordt dan ook gebruikt door de synchronisatie met de channel leave om de schepen uit de sluiskolk te begeleiden. De edge hieraan gekoppeld bevat de functie deque() om de variabelen  van de sluiskolk te resetten.

Vanuit de positie van de sluis worden de schepen gesignaleerd op een invaarverbod en worden de deuren van de sluis gesloten.
De lokatie sluiskolk_afgesloten is bereikt.

Ship [guards, invariants, assignents, synchronizations, properties,aannames]
De template Schip begint bij de Init lokatie. De lokatie is verbonden met de lokatie aangekomen met een edge waarbij een synchronizatie wordt aangeroepen met de template sluisbeheerder. De clock wordt op nul gezet. De lokatie aangekomen is verbonden met de lokatie aangemeld. De edge bevat een synchronizatie waarmee de edge een synchronizatie uitvoert met de template Sluisbheheerder.
De volgende lokatie is  controleren. De edge waarmee de lokatie aangemeld in verbinding staat met de lokatie cnotroleren heeft een synchronisatie voor de template Sluisbeheerder. De lokatie controleren heeft ook een edge met de lokatie wachten. Een schip max maximaal 30 seconden wachten op de lokatie wachten voordat er een mogelijkheid is om opniew in aanmerking te komen voor een controle. Als een schip langer dan 30 tijdseenheden moet wachten de is er een mogelijkheid voor het schip te vertrekken. Hierbij eindigt het schip het invaarproces. Een schip kan dus na aanvaren maximaal 20 seconden wachten om toestemming te krijgen voor een positie invaren anders wordt deze verwezen naar een wachtrij.
Hierna volgdde lokate invarene. De lokatie invarene implieert dat een schip in een invaarproces is dat eindigt in de lokatie gestopt.
Hierop volgd de lokatie nivelleer_start. Hierop wordt een nivelleer_proces gestart. Daarbij is ee synchronisatie met de template Sluisbeheerder.
De lokatie nivelleer_stop is een lokate waarin het nivelleerproces al is gestopt. Van hieruit is er een edge met de lokatie klaar voor vertrek. De edge synchroniseert hiermee met de template Sluisbeheerder.
De lokatie klaar_voor_vertrek is verbonden met de lokatie Init. Met een guard x>=3 tijdseenheden mag een schip vertrekken.


Deur
De deur bevat de volgende lokaties: dicht, openend, open en sluitende.
Een deur sluit niet in een enkele actie. Het proces die een deur dooploopt zijn de processen openend en sluitende. De finale lokaties zijn open en dicht.

Nivelleermachine
De nivelleermachine begint bij de lokatie uit. Met een synchronisatie wordt een nivelleermachine aangezet. De automatie kiest een programma en werkt deze uit in de lokatie bezig. Als ht programma is afgerond volgt de lokatie klaar. Na elk nivelleerproces wordt de machine uitgezet

Stoplicht
Een stoplicht heeft twee lokaties: rood en groen.



%%%%%%%%%%%%%%%%%%%%%%%%%%%%%%%%%%%%%%%%%%%%%%%%%%%%%%%%%%%%%%%%%

Bijlage A performance
\bibitem{kumarUppaalDMAMACProtocol} ... \LaTeX:\\ \url{https://home.hvl.no/ansatte/aaks/articles/2015IKT617.pdf}

test specification
\bibitem{larsenRealtimeUppaalTesting} ... \LaTeX:\\ \url{https://d-nb.info/987511998/34}

sheet 24 tot 65
\bibitem{proenza102008UppaalModelChecker} ... \LaTeX:\\ \url{http://ppedreiras.av.it.pt/resources/empse0809/slides/TheUppaalModelChecker-Julian.pdf}


2.3.4.2
4.7

coffie apparaat

\bibitem{uppaalCoffeeMachine} ... \LaTeX:\\ \url{https://www.comp.nus.edu.sg/~cs5270/Notes/chapt6a.pdf}



what is a good software specification
\bibitem{fvaandrager2322010Goodmodel} ... \LaTeX:\\ \url{http://www.cs.ru.nl/~fvaan/PV/what_is_a_good_model.html#:~:text=A%20good%20model%20has%20a%20clearly%20specified%20purpose%20and%20(ideally,code%20generation%2C%20and%20test%20generation.}

\bibitem{onix01102022devopmodel} ... \LaTeX:\\ \url{https://onix-systems.com/blog/7-basic-software-development-models-which-one-to-choose}
\bibitem{sulemani04012021softwareprocesmodel} ... \LaTeX:\\ \url{https://www.educative.io/blog/software-process-model-types}

\bibitem{globalluxsoft18102017softdev} ... \LaTeX:\\ \url{https://medium.com/globalluxsoft/5-popular-software-development-models-with-their-pros-and-cons-12a486b569dc}
\bibitem{wiegers30052022SRS} ... \LaTeX:\\ \url{https://www.jamasoftware.com/blog/characteristics-of-excellent-requirements/}
\bibitem{muller06092020goodspecification} ... \LaTeX:\\ \url{https://www.gaudisite.nl/ValidationOfRequirementsSlides.pdf}
\bibitem{informit30062008reqmanagement} ... \LaTeX:\\ \url{https://www.informit.com/articles/article.aspx?p=1152528&seqNum=4}
\bibitem{altexsoft15092020writingSRS} ... \LaTeX:\\ \url{https://www.altexsoft.com/blog/software-requirements-specification/}
\bibitem{ID} ... \LaTeX:\\ \url{E:\Backup Mijn Documenten\Hogeschool vakken\TINLab advnced algorithms\tinlab_advanced_algoriths\achtergrondinfo research}
sheet 28 transitorische relaties vertalen van ctl naar ltl
\bibitem{ID} ... \LaTeX:\\ \url{file:///E:/Backup%20Mijn%20Documenten/Hogeschool%20vakken/TINLab%20advnced%20algorithms/tinlab_advanced_algoriths/achtergrondinfo%20research/buchi/lec16_Buchi+LTL.pdf}
\bibitem{ID} ... \LaTeX:\\ \url{file:///E:/Backup%20Mijn%20Documenten/Hogeschool%20vakken/TINLab%20advnced%20algorithms/tinlab_advanced_algoriths/achtergrondinfo%20research/buchi/lect4.pdf}
\bibitem{ID} ... \LaTeX:\\ \url{file:///E:/Backup%20Mijn%20Documenten/Hogeschool%20vakken/TINLab%20advnced%20algorithms/tinlab_advanced_algoriths/achtergrondinfo%20research/buchi/lecture8.pdf}
transitie relaties in LTL sheet 8
\bibitem{ID} ... \LaTeX:\\ \url{file:///E:/Backup%20Mijn%20Documenten/Hogeschool%20vakken/TINLab%20advnced%20algorithms/tinlab_advanced_algoriths/achtergrondinfo%20research/CS%20267%20Automated%20Verification/l2.pdf}
\bibitem{ID} ... \LaTeX:\\ \url{file:///E:/Backup%20Mijn%20Documenten/Hogeschool%20vakken/TINLab%20advnced%20algorithms/tinlab_advanced_algoriths/achtergrondinfo%20research/FORMAL%20METHODS/slide3.pdf}
\bibitem{ID} ... \LaTeX:\\ \url{file:///E:/Backup%20Mijn%20Documenten/Hogeschool%20vakken/TINLab%20advnced%20algorithms/tinlab_advanced_algoriths/achtergrondinfo%20research/FORMAL%20METHODS/slide4.pdf}
hf 4.2
\bibitem{ID} ... \LaTeX:\\ \url{file:///E:/Backup%20Mijn%20Documenten/Hogeschool%20vakken/TINLab%20advnced%20algorithms/tinlab_advanced_algoriths/achtergrondinfo%20research/properties%20ctl/Chapter-4-Formal-Methods-LTL-CTL-TRAFFIC-LIGHT-EXAMPLE-pages-18-24.pdf}


\bibitem{ID} ... \LaTeX:\\ \url{E:\Backup Mijn Documenten\Hogeschool vakken\TINLab advnced algorithms\tinlab_advanced_algoriths\lesmateriaal\modelchecking.pdf}





\bibitem{ID} ... \LaTeX:\\ \url{E:\Backup Mijn Documenten\Hogeschool vakken\TINLab advnced algorithms\tinlab_advanced_algoriths\lesmateriaal}

parallelle compositie
\bibitem{ID} ... \LaTeX:\\ \url{file:///E:/Backup%20Mijn%20Documenten/Hogeschool%20vakken/TINLab%20advnced%20algorithms/tinlab_advanced_algoriths/achtergrondinfo%20research/properties%20ctl/model.pdf}

Urgent locations
Is hetzelfde als het toevoegen van een clock x, met een invariant x<=o op de locatie. Zolang een systeem in een urgente locatie zit mag er geen tijd verstrijken
Bjivoorbeeld als een sluis klaar is engeen schpeen in de sluis. Dan moet er een urgentie zijn dat alle schepen waar mogelijk worden opgesteld voor invaren. Als er geen schepen in de wachtrij en er staan geenschepen klaar om in te varen dn is er misschien urgentie om aan de andere kant schepen op te halen.
Commited locations
Als een of meerdere locaties ingesteld zijn als committed. Een committed state kan niet vertragen  en de volgende transitie moet een transitie zijn waarin de uitgaande edge komt van een committed edge


zeno gedrag: de mogelijkheid dat in een eindige hoeveelheid tijd een oneindig antal handelingen kan worden verricht.
Bijvoorbeeld tijdens het nivelleren
Bij het opstellen van schepen
Bij het laten wachten van schepen
Bij het invaren van schepen
\bibitem{ID} ... \LaTeX:\\ \url{file:///E:/Backup%20Mijn%20Documenten/Hogeschool%20vakken/TINLab%20advnced%20algorithms/tinlab_advanced_algoriths/achtergrondinfo%20research/properties%20ctl/std.pdf}



\bibitem{ID} ... \LaTeX:\\ \url{https://wayback.archive-it.org/9650/20200409062940/http:/p3-raw.greenpeace.org/international/Global/international/publications/nuclear/2016/Nuclear_Scars.pdf}
\bibitem{ID} ... \LaTeX:\\ \url{https://bdtechtalks.com/2020/07/29/self-driving-tesla-car-deep-learning/}



%%%%%%%%%%%%%%%%%%%%%%%%%%%%%%%%%%%%%%%%%%%%%%%%%%%%%%%%%%%%%%%%%
critical safety systems chemicals
\bibitem{esc16022021scsdeveloping} ... \LaTeX:\\ \url{https://esc.uk.net/safety-critical-systems}
\bibitem{oecd2008chemsafeperfindct} ... \LaTeX:\\ \url{https://www.oecd.org/chemicalsafety/chemical-accidents/41269710.pdf}
\bibitem{issa2003chemicalsID} ... \LaTeX:\\ \url{https://safety-work.org/fileadmin/safety-work/articles/Verwechslung_von_Chemikalien/Stoffverwechslung_e.pdf}
\bibitem{sommerville2008CriticalSystems} ... \LaTeX:\\ \url{https://ifs.host.cs.st-andrews.ac.uk/Books/SE9/Web/Dependability/CritSys.html}
\bibitem{identifyinglaboratoryHazards} ... \LaTeX:\\ \url{https://www.acs.org/content/dam/acsorg/about/governance/committees/chemicalsafety/publications/identifying-and-evaluating-hazards-in-research-laboratories.pdf}
\bibitem{ID} ... \LaTeX:\\ \url{https://www.computer.org/csdl/magazine/so/2017/04/mso2017040049/13rRUxCitHw}
\bibitem{ID} ... \LaTeX:\\ \url{https://msquair.files.wordpress.com/2012/06/assca-guiding-philosophic-principles-on-the-design-and-acquisition-of-safety-critical-systems-v1-6.pdf}
\bibitem{ID} ... \LaTeX:\\ \url{https://epsc.be/Documents/PS+Fundamentals/_/EPSC_Process%20Safety%20Fundamentals%20-%20Booklet_March2021.pdf}
\bibitem{winceckCriticalToSafety} ... \LaTeX:\\ \url{https://www.icheme.org/media/8976/xxiv-poster-11.pdf}
\bibitem{chambersHazardAnalysisSCS} ... \LaTeX:\\ \url{https://crpit.scem.westernsydney.edu.au/confpapers/CRPITV55Chambers.pdf}
\bibitem{rslater1998SCSAnalysis} ... \LaTeX:\\ \url{https://users.ece.cmu.edu/~koopman/des_s99/safety_critical/}




%%%%%%%%%%%%%%%%%%%%%%%%%%%%%%%%%%%%%%%%%%%%%%%%%%%%%%%%%%%%%%%%%
critical safety systems airplanes
\bibitem{ID} ... \LaTeX:\\ \url{file:///C:/Users/gally/Downloads/AGARDAG300.pdf}
\bibitem{brat2015verifysafetyflightcritical} ... \LaTeX:\\ \url{https://arxiv.org/abs/1502.02605}
\bibitem{knightchallengessafetyCritical} ... \LaTeX:\\ \url{https://users.encs.concordia.ca/~ymzhang/courses/reliability/ICSE02Knight.pdf}
\bibitem{ID} ... \LaTeX:\\ \url{https://www.jstor.org/stable/44682826}
\bibitem{johnson2006devsafetycritical} ... \LaTeX:\\ \url{http://www.dcs.gla.ac.uk/~johnson/teaching/safety/slides/pt2.pdf}
\bibitem{yeagerSafetyCritical} ... \LaTeX:\\ \url{https://sites.google.com/site/cis115textbook/safety-critical-systems}
\bibitem{ID} ... \LaTeX:\\ \url{https://www.dau.edu/tools/se-brainbook/Pages/Design%20Considerations/Critical-Safety-Item.aspx}
\bibitem{ID} ... \LaTeX:\\ \url{https://mcdpinc.com/safety-critical-systems}
\bibitem{fallsafedesign} ... \LaTeX:\\ \url{https://faculty.up.edu/lulay/MEStudentPage/failsafe.pdf}
\bibitem{2008manualflightsafetyParts} ... \LaTeX:\\ \url{https://www.enidine.com/CorporateSite/media/itt/Resources/Distributors/EndUserDocuments/Suppliers_Documents/QAM03_Rev_E.pdf}
\bibitem{arForce2015VerificationExpectations} ... \LaTeX:\\ \url{https://daytonaero.com/wp-content/uploads/AC-17-01.pdf}
\bibitem{harvardRiskResearchaircraft} ... \LaTeX:\\ \url{https://rmas.fad.harvard.edu/pages/chartered-private-aircraft-0}
\bibitem{ID} ... \LaTeX:\\ \url{https://pubmed.ncbi.nlm.nih.gov/7966484/}
\bibitem{ID} ... \LaTeX:\\ \url{https://nebula.esa.int/content/assessment-methodology-certification-safety-gnc-critical-space-systems}
\bibitem{ID} ... \LaTeX:\\ \url{https://www.aopa.org/training-and-safety/online-learning/safety-spotlights/aircraft-systems}
\bibitem{lalaArchitecturalPrinciples} ... \LaTeX:\\ \url{https://www.cs.unc.edu/~anderson/teach/comp790/papers/safety_critical_arch.pdf}
\bibitem{andersenromanski2011verificationsafetycritical} ... \LaTeX:\\ \url{https://queue.acm.org/detail.cfm?id=2024356}
\bibitem{aviatioLawGeneralDefinitions} ... \LaTeX:\\ \url{https://www.law.cornell.edu/cfr/text/14/1.1}
\bibitem{ntsb2006safetyreportTransportAirplanes} ... \LaTeX:\\ \url{http://libraryonline.erau.edu/online-full-text/ntsb/safety-reports/SR06-02.pdf}
\bibitem{mitNotesSafetyCritical} ... \LaTeX:\\ \url{https://www.cs.uct.ac.za/mit_notes/human_computer_interaction/htmls/ch02s10.html}
\bibitem{ID} ... \LaTeX:\\ \url{https://flightsafety.org/}
\bibitem{kochenfender2020aisafetycritical} ... \LaTeX:\\ \url{https://engineering.stanford.edu/magazine/article/mykel-kochenderfer-ai-and-safety-critical-systems}
\bibitem{humanfactorsAviationSafety} ... \LaTeX:\\ \url{https://www.faasafety.gov/files/gslac/courses/content/258/1097/AMT_Handbook_Addendum_Human_Factors.pdf}
\bibitem{eulegislator2000safetyregulation} ... \LaTeX:\\ \url{https://www.eurocontrol.int/sites/default/files/2019-06/src-doc-1-e1.0.pdf}
\bibitem{ID} ... \LaTeX:\\ \url{http://aerossurance.com/safety-management/critical-maintenance-tasks/}
\bibitem{gao2020aviationcybersecurity} ... \LaTeX:\\ \url{https://www.gao.gov/assets/gao-21-86.pdf}
\bibitem{roleofcodeinsoftware} ... \LaTeX:\\ \url{https://criticalsoftware.com/en/news/coding-the-skies}
\bibitem{airlie2018designparameters} ... \LaTeX:\\ \url{https://aviation.stackexchange.com/questions/46677/what-are-the-design-parameters-for-airliner-safety}
\bibitem{ID} ... \LaTeX:\\ \url{https://www.cantwell.senate.gov/news/press-releases/cantwells-comprehensive-bipartisan-bicameral-aircraft-safety-and-certification-reforms-signed-into-law}
\bibitem{ID} ... \LaTeX:\\ \url{https://www.forbes.com/advisor/travel-rewards/737-max-what-is-safety-anyway/}
\bibitem{humanfactors2010Aviation} ... \LaTeX:\\ \url{https://www.tandfonline.com/doi/full/10.1080/00140130903521587}
\bibitem{ID} ... \LaTeX:\\ \url{https://www.doi.gov/aviation/safety}
\bibitem{courtin2018safetyconsiderations} ... \LaTeX:\\ \url{https://dspace.mit.edu/bitstream/handle/1721.1/118438/ICAT_2018_07_Christoper%20Courtin_Report.pdf?sequence=1&isAllowed=y}
\bibitem{ID} ... \LaTeX:\\ \url{https://www.defence.gov.au/dasp/Docs/Manuals/7001053/eTAMMweb/1049.htm}
\bibitem{prentice2014Failedaviationprogram} ... \LaTeX:\\ \url{https://www.aviationpros.com/aircraft/commercial-airline/article/10239806/staying-legal-another-failed-faa-safety-program}
\bibitem{civilAviationConsulting} ... \LaTeX:\\ \url{https://www.iata.org/en/services/consulting/safety-operations/}
\bibitem{olivercalvardpotocnik2017Aviationautomation} ... \LaTeX:\\ \url{https://hbr.org/2017/09/the-tragic-crash-of-flight-af447-shows-the-unlikely-but-catastrophic-consequences-of-automation}
\bibitem{ID} ... \LaTeX:\\ \url{https://www.infosys.com/industries/communication-services/documents/landing-gear-design-and-development.pdf}
\bibitem{airforce2000SystemSafety} ... \LaTeX:\\ \url{https://www.acqnotes.com/Attachments/AF_System-Safety-HNDBK.pdf}
\bibitem{fed2019SafeSecureSUAS} ... \LaTeX:\\ \url{https://www.transportation.gov/testimony/state-airline-safety-federal-oversight-commercial-aviation}
\bibitem{AirportSafety} ... \LaTeX:\\ \url{https://www.federalregister.gov/documents/2019/02/13/2019-00758/safe-and-secure-operations-of-small-unmanned-aircraft-systems}
\bibitem{ID} ... \LaTeX:\\ \url{https://archive.etsc.eu/documents/safety%20in%20airports.pdf}
\bibitem{passengerSafety} ... \LaTeX:\\ \url{https://journals.sagepub.com/doi/pdf/10.1177/002029400403700202}
\bibitem{ID} ... \LaTeX:\\ \url{https://www.unmannedsystems.ca/wp-content/uploads/2019/01/DRAFT-AC-922-001-RPAS-SAFETY-ASSURANCE.pdf}
\bibitem{ID} ... \LaTeX:\\ \url{https://www.ccsdualsnap.com/pressure-switches-in-aerospace-applications/}
\bibitem{britishColumbia2020GuideSafetyCritical} ... \LaTeX:\\ \url{https://www.egbc.ca/getmedia/78073fda-5a83-4f0f-b12f-0a40dcbbc29d/EGBC-Safety-Critical-Software-V1-0.pdf.aspx}
\bibitem{uscongres2019aircraftresearch} ... \LaTeX:\\ \url{https://readwrite.com/2018/12/21/air-travel-is-far-safer-than-you-think-heres-why/}
\bibitem{ID} ... \LaTeX:\\ \url{https://fas.org/sgp/crs/misc/R45939.pdf}
\bibitem{ariAssociation2018} ... \LaTeX:\\ \url{https://cdn.ymaws.com/www.astna.org/resource/collection/4392B20B-D0DB-4E76-959C-6989214920E9/ASTNA_Safety_Position_Paper_2018_FINAL.pdf}
\bibitem{ID} ... \LaTeX:\\ \url{https://transportation.house.gov/imo/media/doc/2020.09.15%20FINAL%20737%20MAX%20Report%20for%20Public%20Release.pdf}
\bibitem{ID} ... \LaTeX:\\ \url{https://www.transportstyrelsen.se/globalassets/global/luftfart/seminarier_och_information/seminarier-2016/luftvardighet-camo-och-145-verkstader/11b-critical-task-fpl.pdf}
\bibitem{fox2005HelicopterSafety} ... \LaTeX:\\ \url{https://www.h-a-c.ca/IHSS_Helicopter_Safety_History_05.pdf}
\bibitem{fulvio1993safetycriticalsystems} ... \LaTeX:\\ \url{https://assembly.coe.int/nw/xml/XRef/X2H-Xref-ViewHTML.asp?FileID=7144&lang=EN}
\bibitem{rakas2018criticalsystemsFailures} ... \LaTeX:\\ \url{https://www.skybrary.aero/index.php/Cockpit_Automation_-_Advantages_and_Safety_Challenges}
\bibitem{ID} ... \LaTeX:\\ \url{https://ntrs.nasa.gov/citations/20120014507}
\bibitem{cockpitAutomation} ... \LaTeX:\\ \url{https://www.sciencedirect.com/science/article/abs/pii/S092575351730601X}
\bibitem{ID} ... \LaTeX:\\ \url{https://www.semanticscholar.org/paper/Safety-critical-avionics-for-the-777-primary-flight-Yeh/8facf90f4a9051c3ab8ce11e39d0893118268d90}
\bibitem{ID} ... \LaTeX:\\ \url{https://www.easa.europa.eu/faq/19013}
\bibitem{belcaatrovalidateSafetyCritical} ... \LaTeX:\\ \url{https://ntrs.nasa.gov/api/citations/20120014507/downloads/20120014507.pdf}
\bibitem{ID,	civilAviationsAuthority = {author},	ALTeditor = {editor},	title = {title},	date = {date},	url = {"https://publicapps.caa.co.uk/modalapplication.aspx?catid=1&pagetype=65&appid=11&mode=list&type=subcat&id=32}
	\bibitem{ID} ... \LaTeX:\\ \url{https://definitions.uslegal.com/f/flight-safety-critical-aircraft-part/}
	\bibitem{ID} ... \LaTeX:\\ \url{https://nbaa.org/nbaa-aviation-groups-ask-congress-to-prevent-5g-interference-to-critical-safety-systems/}
	\bibitem{ID} ... \LaTeX:\\ \url{https://www.dlr.de/ft/en/desktopdefault.aspx/tabid-1360/1856_read-36215/}
	\bibitem{ID} ... \LaTeX:\\ \url{https://www.fsd.lrg.tum.de/research/safety-critical/}
	\bibitem{knight2010SafetyCritical} ... \LaTeX:\\ \url{https://ieeexplore.ieee.org/document/1007998}
	\bibitem{AAA052005IdentifySCHitems} ... \LaTeX:\\ \url{https://www.faa.gov/about/office_org/headquarters_offices/ast/licenses_permits/media/RLVGuide_01-05_05.pdf}
	\bibitem{devTopics01032020} ... \LaTeX:\\ \url{https://smallbusinessprogramming.com/safety-critical-software-15-things-every-developer-should-know/}
	\bibitem{ID} ... \LaTeX:\\ \url{https://coreavi.com/the-future-of-safety-critical-systems-in-the-emerging-autonomous-world/}
	\bibitem{valdes2018SafetybyAutomation} ... \LaTeX:\\ \url{https://www.intechopen.com/chapters/59838}
	\bibitem{2015whensafetymanagementsystemsfail} ... \LaTeX:\\ \url{https://verticalmag.com/features/whensafetymanagementsystemsfail/}
	
	
	%%%%%%%%%%%%%%%%%%%%%%%%%%%%%%%%%%%%%%%%%%%%%%%%%%%%%%%%%%%%%%%%%
	critical safety systems fireworks
	\bibitem{ID} ... \LaTeX:\\ \url{https://www.hsdl.org/c/firework-safety/}
	\bibitem{ID} ... \LaTeX:\\ \url{https://www.cpsc.gov/Safety-Education/Safety-Education-Centers/Fireworks}
	\bibitem{ID} ... \LaTeX:\\ \url{https://www.seattletimes.com/subscribe/signup-offers/?pw=redirect&subsource=paywall&return=https://www.seattletimes.com/opinion/editorials/firework-safety-even-more-critical-after-heat-wave/}
	\bibitem{ID} ... \LaTeX:\\ \url{https://www.nrcan.gc.ca/sites/www.nrcan.gc.ca/files/mineralsmetals/pdf/mms-smm/expl-expl/20170828-G05-09E_ACC.pdf}
	\bibitem{fireworksinjuries2021duringcovid} ... \LaTeX:\\ \url{https://www.prnewswire.com/news-releases/fireworks-related-injuries-and-deaths-spiked-during-the-covid-19-pandemic-301322243.html}
	\bibitem{osha2017safetymanagementexplosives} ... \LaTeX:\\ \url{https://www.osha.gov/sites/default/files/publications/OSHA3912.pdf}
	\bibitem{ID} ... \LaTeX:\\ \url{https://www.firelinx.com/wp-content/uploads/2021/02/FLX-Issues-in-Firing-System-Safety.pdf}
	\bibitem{ID} ... \LaTeX:\\ \url{http://www.eig2.org.uk/wp-content/uploads/WTOFD-Blue-Guide.pdf}
	\bibitem{hse2014explosiveregulations} ... \LaTeX:\\ \url{https://www.hse.gov.uk/explosives/er2014-fireworks-retail-prem.pdf}
	\bibitem{ID} ... \LaTeX:\\ \url{https://www.firerescue1.com/firefighter-safety/articles/11-fireworks-safety-videos-from-the-serious-to-the-humorous-fHy0M4pT2gjcQ8jA/}
	\bibitem{pirone2016lessonslearnedfireworks} ... \LaTeX:\\ \url{https://www.aidic.it/cet/16/53/044.pdf}
	\bibitem{ID} ... \LaTeX:\\ \url{http://www.alarmascasas.com.mx/sites/default/files/85006-0061%20--%20FireWorks%20Brochure.pdf}
	\bibitem{ID} ... \LaTeX:\\ \url{https://www.ehs.ufl.edu/programs/fire/fireworks/}
	\bibitem{ID} ... \LaTeX:\\ \url{https://www.interlogix.com.au/documents/FireWorks%20Features%20and%20Operation%20(fire%20only).pdf}
	\bibitem{ID} ... \LaTeX:\\ \url{https://townhall.virginia.gov/l/GetFile.cfm?File=C:%5CTownHall%5Cdocroot%5CGuidanceDocs%5C960%5CGDoc_DFP_4448_v1.pdf}
	\bibitem{ID} ... \LaTeX:\\ \url{https://ec.europa.eu/growth/sectors/chemicals/specific-chemicals_en}
	\bibitem{safehandlepyrotechnics} ... \LaTeX:\\ \url{http://www.iiakm.org/ojakm/articles/2015/volume3_3/OJAKM_Volume3_3pp27-36.pdf}
	\bibitem{ID} ... \LaTeX:\\ \url{https://www.bristol.gov.uk/documents/20182/1175006/Fireworks+in+retail+premises/6aa6ee24-5b74-43b4-a1d9-747689b1dbc9}
	\bibitem{ID} ... \LaTeX:\\ \url{https://www.eversys.com.br/imagens/uploads/arqs/bra_arquivos/04-software-gerenciador-fireworks-brochura.pdf}
	\bibitem{ID} ... \LaTeX:\\ \url{http://www.doiserbia.nb.rs/img/doi/0354-9836/2016/0354-98361500050G.pdf}
	\bibitem{costinFireworksEmbedded} ... \LaTeX:\\ \url{http://s3.eurecom.fr/docs/wisec14_Costin.pdf}
	\bibitem{ID} ... \LaTeX:\\ \url{https://www.firetechsystems.com/assets/uploads/2018/09/FireWorks-Brochure.pdf}
	\bibitem{ritz2020firesafety} ... \LaTeX:\\ \url{https://blog.ritzsafety.com/fireworks-safety-tips}
	\bibitem{lawson2018criticalsystems} ... \LaTeX:\\ \url{https://www.engineerlive.com/content/fire-detection-and-protection-through-safety-critical-systems}
	%%%%%%%%%%%%%%%%%%%%%%%%%%%%%%%%%%%%%%%%%%%%%%%%%%%%%%%%%%%%%%%%%
%    
%    algemene vragen
%    oorzaken
%    @online{gates18112020boeingcrisis,	ALTauthor = {author},	ALTeditor = {editor},	title = {title},	date = {date},	url = {"https://www.seattletimes.com/business/boeing-aerospace/what-led-to-boeings-737-max-crisis-a-qa/"},}
%    @online{boeing737maxsoftwareprobles,	ALTauthor = {author},	ALTeditor = {editor},	title = {title},	date = {date},	url = {"https://www.schneier.com/blog/archives/2019/04/excellent_analy.html"},}
%    fout in de software
%    @online{avetisov19032019boeingmalwarestate,	ALTauthor = {author},	ALTeditor = {editor},	title = {title},	date = {date},	url = {"https://www.forbes.com/sites/georgeavetisov/2019/03/19/malware-at-30000-feet-what-the-737-max-says-about-the-state-of-airplane-software-security/?sh=4d26f7052a9e"},}
%    het nationaal veiligheidsbelang
%    @online{thompson23112020nationalsecurityboeing,	ALTauthor = {author},	ALTeditor = {editor},	title = {title},	date = {date},	url = {"https://www.forbes.com/sites/lorenthompson/2020/11/23/five-reasons-return-of-boeings-737-max-to-service-is-important-to-national-security/?sh=2128ea552018"},}
%    falend toezicht
%    @online{gates21032019FAAControlSystem,	ALTauthor = {author},	ALTeditor = {editor},	title = {title},	date = {date},	url = {"https://www.seattletimes.com/business/boeing-aerospace/failed-certification-faa-missed-safety-issues-in-the-737-max-system-implicated-in-the-lion-air-crash/"},}
%    onderzoeksrapport
%    @online{faa18112020boeingreview,	ALTauthor = {author},	ALTeditor = {editor},	title = {title},	date = {date},	url = {"https://www.faa.gov/foia/electronic_reading_room/boeing_reading_room/media/737_RTS_Summary.pdf"},}
%    @online{wiki737maxgroundings,	ALTauthor = {author},	ALTeditor = {editor},	title = {title},	date = {date},	url = {"https://en.wikipedia.org/wiki/Boeing_737_MAX_groundings"},}
%    veiligheidsrisico's
%    menselijke fouten
%    @online{campbell02052019boengcrashhumanerrors,	ALTauthor = {author},	ALTeditor = {editor},	title = {title},	date = {date},	url = {"https://www.theverge.com/2019/5/2/18518176/boeing-737-max-crash-problems-human-error-mcas-faa"},}
%    overzicht van crashes
%    @online{hawkins22032019737maxairplanes,	ALTauthor = {author},	ALTeditor = {editor},	title = {title},	date = {date},	url = {"https://www.theverge.com/2019/3/22/18275736/boeing-737-max-plane-crashes-grounded-problems-info-details-explained-reasons"},}
%    veiligheidsopmerking
%    @online{thomas30082020737safest,	ALTauthor = {author},	ALTeditor = {editor},	title = {title},	date = {date},	url = {"https://www.airlineratings.com/news/boeings-737-max-will-one-safest-aircraft-history/"},}
%    aanpassingen
%    @online{boeing737maxdisplay,	ALTauthor = {author},	ALTeditor = {editor},	title = {title},	date = {date},	url = {"https://www.boeing.com/commercial/737max/737-max-software-updates.page"},}
%    waarschuwingen//output signalen
%    @online{fehrm24112020737changes,	ALTauthor = {author},	ALTeditor = {editor},	title = {title},	date = {date},	url = {"https://leehamnews.com/2020/11/24/boeing-737-max-changes-beyond-mcas/"},}
%    software gerelateerde fouten
%    @online{travis18042019737maxsoftwaredevop,	ALTauthor = {author},	ALTeditor = {editor},	title = {title},	date = {date},	url = {"https://spectrum.ieee.org/aerospace/aviation/how-the-boeing-737-max-disaster-looks-to-a-software-developer"},}
%    onderzoeksrapport
%    de rol van de publieke opinie
%    @online{barnett05052019737maxcrisis,	ALTauthor = {author},	ALTeditor = {editor},	title = {title},	date = {date},	url = {"https://pubsonline.informs.org/do/10.1287/orms.2019.05.05/full/"},}
%    onderzoek van europese luchtvaart agentschap
%    @online{easa27012021737maxsafereturn,	ALTauthor = {author},	ALTeditor = {editor},	title = {title},	date = {date},	url = {"https://www.easa.europa.eu/newsroom-and-events/news/easa-declares-boeing-737-max-safe-return-service-europe"},}
%    veiligheidsvraagstuk
%    @online{touitou11032019737tragedies,	ALTauthor = {author},	ALTeditor = {editor},	title = {title},	date = {date},	url = {"https://phys.org/news/2019-03-boeing-max-safety-tragedies.html"},}
%    artikel over sensoren
%    @online{hemmerdinger02022021737maxdeliveries,	ALTauthor = {author},	ALTeditor = {editor},	title = {title},	date = {date},	url = {"https://www.flightglobal.com/airframers/boeing-delays-737-max-10-deliveries-two-years-to-2023/142245.article"},}
%    goedkeuring van europese luchtvaart autoriteiten
%    advies aan de faa
%    @online{bielby27022021faaimprovesafety,	ALTauthor = {author},	ALTeditor = {editor},	title = {title},	date = {date},	url = {"https://www.hstoday.us/subject-matter-areas/airport-aviation-security/oig-tells-faa-to-improve-safety-oversight-following-boeing-737-max-review/"},}
%    @online{boyle18112020737maxupgrade,	ALTauthor = {author},	ALTeditor = {editor},	title = {title},	date = {date},	url = {"https://www.geekwire.com/2020/faas-go-ahead-737-maxs-return-flight-kicks-off-massive-software-upgrade/"},}
%    @online{bergstraburgess122019737maxMcasAlgorithm,	ALTauthor = {author},	ALTeditor = {editor},	title = {title},	date = {date},	url = {"https://www.researchgate.net/publication/338420944_A_Promise_Theoretic_Account_of_the_Boeing_737_Max_MCAS_Algorithm_Affair"},}
%    achtergrond informatie
%    @online{737mcas,	ALTauthor = {author},	ALTeditor = {editor},	title = {title},	date = {date},	url = {"http://www.b737.org.uk/mcas.htm"},}
%    algemeen vertrouwen
%    @online{newburger17052019boeingcrisis,	ALTauthor = {author},	ALTeditor = {editor},	title = {title},	date = {date},	url = {"https://www.cnbc.com/2019/05/16/what-you-need-to-know-about-boeings-737-max-crisis.html"},}
%    toestemming europese autoriteiten
%    problemen
%    @online{arstechnica22012020737problems,	ALTauthor = {author},	ALTeditor = {editor},	title = {title},	date = {date},	url = {"https://arstechnica.com/information-technology/2020/01/737-max-fix-slips-to-summer-and-thats-just-one-of-boeings-problems/"},}
%    uitgebreid artikel over de onderzoeken en het vliegverbod
%    @online{german190620217372yaftergrounded,	ALTauthor = {author},	ALTeditor = {editor},	title = {title},	date = {date},	url = {"https://www.cnet.com/news/boeing-737-max-8-all-about-the-aircraft-flight-ban-and-investigations/"},}
%    computers als oorzaak
%    lessons learned
%    @online{beningo02052019boeinglessons,	ALTauthor = {author},	ALTeditor = {editor},	title = {title},	date = {date},	url = {"https://www.designnews.com/electronics-test/5-lessons-learn-boeing-737-max-fiasco"},}
%    @online{duran05042019boeingspof,	ALTauthor = {author},	ALTeditor = {editor},	title = {title},	date = {date},	url = {"https://www.eurocontrol.int/publication/effects-network-extra-standby-aircraft-and-boeing-737-max-grounding"},}
%    single point of failure
%    @online{ID,	ALTauthor = {author},	ALTeditor = {editor},	title = {title},	date = {date},	url = {"https://dmd.solutions/blog/2019/04/05/how-a-single-point-of-failure-spof-in-the-mcas-software-could-have-caused-the-boeing-737-max-crash-in-ethiopia/"},}
%    @online{makichuck24012021737fearflying,	ALTauthor = {author},	ALTeditor = {editor},	title = {title},	date = {date},	url = {"https://asiatimes.com/2021/01/boeings-737-max-and-the-fear-of-flying/"},}
%    lijst van tehnische aanpassingen
%    @online{caa737modifications,	ALTauthor = {author},	ALTeditor = {editor},	title = {title},	date = {date},	url = {"https://www.caa.co.uk/Consumers/Guide-to-aviation/Boeing-737-MAX/"},}
%    @online{oestergaard14122020boeingdeliveries,	ALTauthor = {author},	ALTeditor = {editor},	title = {title},	date = {date},	url = {"https://dsm.forecastinternational.com/wordpress/2020/12/14/airbus-and-boeing-report-november-2020-commercial-aircraft-orders-and-deliveries/"},}
%    code lek
%    @online{reenberg787flaws,	ALTauthor = {author},	ALTeditor = {editor},	title = {title},	date = {date},	url = {"https://www.wired.com/story/boeing-787-code-leak-security-flaws/"},}
%    @online{fitch16092020737backlogrisks,	ALTauthor = {author},	ALTeditor = {editor},	title = {title},	date = {date},	url = {"https://www.fitchratings.com/research/corporate-finance/boeing-737-max-return-backlog-risks-remain-16-09-2020"},}
%    Cultuurverandering, deregulatie, systeemwijziging of gewoon een kwestie van competentie
%    @online{willis27082020737maxfailures,	ALTauthor = {author},	ALTeditor = {editor},	title = {title},	date = {date},	url = {"https://www.aerospacetestinginternational.com/features/what-broke-the-737-max.html"},}
%    extra aanpassingen
%    @online{ostrower11062020more737changes,	ALTauthor = {author},	ALTeditor = {editor},	title = {title},	date = {date},	url = {"https://theaircurrent.com/aviation-safety/boeings-737-max-software-done-but-regulators-plot-more-changes-after-jets-return/"},}
%    wat ging er mis een analyse van een ex-iloot
%    De utoriteiten waren op de hoogte
%    @online{hruska13122019faaknown737crashrate,	ALTauthor = {author},	ALTeditor = {editor},	title = {title},	date = {date},	url = {"https://www.extremetech.com/extreme/303373-the-faa-knew-the-737-max-was-dangerous-and-kept-it-flying-anyway"},}
%    kwaliteiten van het alarmsysteem niet goed bekend
%    @online{bloomberg26092019failedpred,	ALTauthor = {author},	ALTeditor = {editor},	title = {title},	date = {date},	url = {"https://time.com/5687473/boeing-737-alarm-system/"},}
%    @online{whiteman09072020boengcancelstock,	ALTauthor = {author},	ALTeditor = {editor},	title = {title},	date = {date},	url = {"https://www.nasdaq.com/articles/boeing-gets-dealt-another-737-max-cancellation-blow.-what-it-means-for-boeing-stock-2020"},}
%    @online{leopold09192019boeingreliability,	ALTauthor = {author},	ALTeditor = {editor},	title = {title},	date = {date},	url = {"https://www.eetimes.com/boeing-crashes-highlight-a-worsening-reliability-crisis/"},}
%    veiligheidsvraagstuk
%    @online{koenig11122019737crashesnofix,	ALTauthor = {author},	ALTeditor = {editor},	title = {title},	date = {date},	url = {"https://www.latimes.com/business/story/2019-12-11/faa-boeing-737-max-crashes"},}
%    probleemanalyse, veiligheidsvraagstuk
%    @online{dohertylindeman15032019737problems,	ALTauthor = {author},	ALTeditor = {editor},	title = {title},	date = {date},	url = {"https://www.politico.com/story/2019/03/15/boeing-737-max-grounding-1223072"},}
%    falend toezicht
%    @online{stodder02102019corruptoversight,	ALTauthor = {author},	ALTeditor = {editor},	title = {title},	date = {date},	url = {"https://www.pogo.org/analysis/2019/10/corrupted-oversight-the-faa-boeing-and-the-737-max/"},}
%    @online{afacwaLostSafeguards,	ALTauthor = {author},	ALTeditor = {editor},	title = {title},	date = {date},	url = {"https://www.afacwa.org/the_inside_story_of_mcas_seattle_times"},}
%    doelstellingen en veiligheidsvraagstukken
%    @online{swayne18032019profitssafety,	ALTauthor = {author},	ALTeditor = {editor},	title = {title},	date = {date},	url = {"https://www.marxist.com/737-max-scandal-boeing-putting-profits-before-safety.htm"},}
%    @online{freed26022021liftaustraliaban,	ALTauthor = {author},	ALTeditor = {editor},	title = {title},	date = {date},	url = {"https://finance.yahoo.com/news/australia-lifts-ban-boeing-737-035817682.html?guccounter=1&guce_referrer=aHR0cHM6Ly93d3cuZ29vZ2xlLmNvbS8&guce_referrer_sig=AQAAAHZCJYy_0A5VS2WiPoCvH4xdrRNkmkdsv5EWJ2RLIz_AS-rxsTty6AF1_HlmJiRyWYqCXDi4p0Xs4isYkNkCq2Pfo-pQ60Xz_IfTNjm4FgoZiBMC4zpZlB6F0fwecrjE_ujAXZzG4xPJnWCd8-G3VLlPTY8h3H31eQ1i8hY9AIyy"},}
%    autoriteiten krijgen tik op de vingers
%    @online{reed15032019softwareattention,	ALTauthor = {author},	ALTeditor = {editor},	title = {title},	date = {date},	url = {"https://medium.com/@jpaulreed/the-737max-and-why-software-engineers-should-pay-attention-a041290994bd"},}
%    @online{news17032019softwareexplains,	ALTauthor = {author},	ALTeditor = {editor},	title = {title},	date = {date},	url = {"https://news.ycombinator.com/item?id=19414775"},}
%    @online{legget21122020eu737maxsafe,	ALTauthor = {author},	ALTeditor = {editor},	title = {title},	date = {date},	url = {"https://www.bbc.com/news/55366320"},}
%    @online{marketscreener0103221737chinarecertification,	ALTauthor = {author},	ALTeditor = {editor},	title = {title},	date = {date},	url = {"https://www.marketscreener.com/news/latest/China-studies-Boeing-737-MAX-recertification-wants-safety-concerns-fully-addressed--32569125/"},}
%    motor in brand
%    @online{euractiv22022021737firegrounds,	ALTauthor = {author},	ALTeditor = {editor},	title = {title},	date = {date},	url = {"https://www.euractiv.com/section/aviation/news/boeing-grounds-777s-after-engine-fire/"},}
%    @online{benny18022019737returnUAE,	ALTauthor = {author},	ALTeditor = {editor},	title = {title},	date = {date},	url = {"https://gulfnews.com/business/aviation/uae-airspace-to-see-return-of-boeing-737-max-1.1613627548923"},}
%    motor in brand gevlogen
%    @online{biersmichel22022021777grounds,	ALTauthor = {author},	ALTeditor = {editor},	title = {title},	date = {date},	url = {"https://techxplore.com/news/2021-02-boeing-urges-grounding-777s.html"},}
%    @online{ID,	ALTauthor = {author},	ALTeditor = {editor},	title = {title},	date = {date},	url = {"https://www.politico.eu/article/uk-temporarily-bans-some-boeing-aircraft-after-pratt-whitney-engine-incidents/"},}
%    @online{reuters23022021777metalfatigue,	ALTauthor = {author},	ALTeditor = {editor},	title = {title},	date = {date},	url = {"https://www.timeslive.co.za/news/world/2021-02-23-damage-to-united-boeing-777-engine-consistent-with-metal-fatigue--ntsb/"},}
%    faa was niet kritisch genoeg
%    @online{ID,	ALTauthor = {author},	ALTeditor = {editor},	title = {title},	date = {date},	url = {"https://federalnewsnetwork.com/government-news/2021/02/federal-watchdog-blasts-faa-over-certification-of-boeing-jet/"},}
%    
%    
%    
%    
%    
%    
%    
%    \cite{bnnvara13062018malirapport}
%    \cite{eucal11012021malimissieverlengd}
%    \cite{nos21052014zorgenmalimissie}
%    \cite{meijnders}
%    \cite{bnrwebredactie}
%    \cite{keultjes01062016malimissiecoalitie}
%    \cite{veenhof18012019}
%    
%    \cite{isitman06012016militair}
%    \cite{nporadio11072016filmdemissie}
%    \cite{parlementairmonitor15122013mortierongeluk}
%    
%    sollicitatie
%    de bureaucratie
%    aankomst
%    interview van de burgerbevolking
%    steun van de bevolking minuut 15:00
%    de organisatie minuut 23:00
%    De militaire briefing minuut 34:00
%    prioriteit minuut 39:00
%    briefing minuut 40:00
%    de communicatie met ministerie over inlichten minuut 44:00
%    @online{DemissieFilm,	ALTauthor = {author},	ALTeditor = {editor},	title = {title},	date = {date},	url = {"https://www.2doc.nl/documentaires/series/2doc/2016/juli/de-missie.html"},}
%    \cite{DemissieFilm}
%    
%    
%    
%    
%    
%    
%    verhaal van brandweermannen
%    
%    \cite{staff31082015tanjinblastunrevealed}
%    artikel
%    
%    \cite{chinafile18082015tanjinexplosion}
%    invloed van social media
%    \bibitem{ID} ... \LaTeX:\\ \url{https://www.economist.com/asia/2015/08/18/a-blast-in-tianjin-sets-off-an-explosion-online}
%    \cite{}
%    \bibitem{ID} ... \LaTeX:\\ \url{https://america.cgtn.com/2015/08/12/explosion-reported-in-tianjin-china}
%    \cite{}
%    \bibitem{ID} ... \LaTeX:\\ \url{https://factcheck.afp.com/no-photo-was-taken-chinese-city-tianjin-august-2015}
%    \cite{}
%    vergelijking van twee rampen
%    \bibitem{ID} ... \LaTeX:\\ \url{https://airshare.air-inc.com/how-does-the-beirut-explosion-compare-to-tianjin}
%    \cite{}
%    overheid en media
%    \bibitem{ID} ... \LaTeX:\\ \url{https://newbloommag.net/2015/08/17/tianjin-explosion/}
%    \cite{}
%    chemische industrie ondeer de loep
%    \bibitem{ID} ... \LaTeX:\\ \url{https://www.voanews.com/east-asia-pacific/tianjin-blast-puts-spotlight-chemical-industry}
%    \cite{}
%    \bibitem{ID} ... \LaTeX:\\ \url{https://abcnews.go.com/International/apocalyptic-aftermath-devastating-images-tianjin-china-explosions/story?id=33057017}
%    \cite{}
%    \bibitem{ID} ... \LaTeX:\\ \url{https://www.reachingoutacrossdurham.co.uk/osk/tianjin-explosion-2021}
%    \cite{}
%    \bibitem{pinghuang2410201TanjinFactreport} ... \LaTeX:\\ \url{https://aiche.onlinelibrary.wiley.com/doi/abs/10.1002/prs.11789}
%    \cite{pinghuang2410201TanjinFactreport}
%    \bibitem{ID} ... \LaTeX:\\ \url{https://www.automotivelogistics.media/thousands-of-cars-destroyed-in-tianjin-port-explosions/13570.article}
%    \cite{}
%    \bibitem{ID} ... \LaTeX:\\ \url{https://www.joc.com/port-news/asian-ports/port-tianjin/tianjin-port-explosions-could-be-most-expensive-maritime-disaster_20150826.html}
%    \cite{}
%    \bibitem{ID} ... \LaTeX:\\ \url{https://www.bloomberg.com/news/articles/2015-08-12/explosion-in-northern-china-shatters-windows-causes-injuries}
%    \cite{}
%    \bibitem{ID} ... \LaTeX:\\ \url{https://unece.org/fileadmin/DAM/env/documents/2016/TEIA/OECD_WGCA_24-27_OCT_2016/Session_3_Zhao_-__Introduction_of_Tianjin_Accident_-_Jinsong_Zhao.pdf}
%    \cite{}
%    gemaakte fouten
%    \bibitem{portoTanjinExplosionSight} ... \LaTeX:\\ \url{https://porteconomicsmanagement.org/pemp/contents/part6/port-resilience/site-2015-tianjin-port-explosions/}
%    \cite{portoTanjinExplosionSight}
%    \bibitem{ID} ... \LaTeX:\\ \url{https://www.alamy.com/stock-image-tianjin-china-17th-aug-2015-tianjin-explosion-aftermath-blast-site-165334778.html}
%    \cite{}
%    \bibitem{ID} ... \LaTeX:\\ \url{https://www.popularmechanics.com/technology/news/a16871/massive-explosions-china-city-of-tianjin/}
%    \cite{}
%    \bibitem{imago17082015TanjinApartmentImages} ... \LaTeX:\\ \url{https://www.imago-images.com/st/0080815934}
%    \cite{imago17082015TanjinApartmentImages}
%    \bibitem{trager14082015Chemicalblast} ... \LaTeX:\\ \url{https://www.chemistryworld.com/news/deadly-chemical-blast-at-chinese-port/8857.article}
%    \cite{trager14082015Chemicalblast}
%    \bibitem{pangeramo27082015TanjinExplosion} ... \LaTeX:\\ \url{https://www.process-worldwide.com/tianjin-explosion-from-chemical-perspective-insights-and-backgrounds-a-502381/}
%    \cite{pangeramo27082015TanjinExplosion}
%    vergelijking met andere explosies
%    \bibitem{ap06082020ammaniumnitrate} ... \LaTeX:\\ \url{https://apnews.com/article/lebanon-fires-us-news-explosions-middle-east-53f4206a7f1db0812262a15d22e1e58f}
%    \cite{ap06082020ammaniumnitrate}
%    invloed van de ramp op de industrie
%    \bibitem{morris14082015TanjinIndustryImpact} ... \LaTeX:\\ \url{https://fortune.com/2015/08/14/tianjin-port-explosion-shipping-delays/}
%    \cite{morris14082015TanjinIndustryImpact}
%    is er sprake van een doofpot
%    \bibitem{milesyu20082015exposingtoxicgovlines} ... \LaTeX:\\ \url{https://www.washingtontimes.com/news/2015/aug/20/inside-china-tianjin-explosions-cover-up-exposes-b/}
%    \cite{milesyu20082015exposingtoxicgovlines}
%    eigendomsverzekering
%    \bibitem{artemis30032016tanjininsurance} ... \LaTeX:\\ \url{https://www.artemis.bm/news/tianjin-explosions-property-insurance-loss-could-reach-3-5bn-swiss-re/}
%    \cite{artemis30032016tanjininsurance}
%    \bibitem{aidenxiatanjinblast} ... \LaTeX:\\ \url{https://www.thechinastory.org/yearbooks/yearbook-2015/forum-the-abyss-%E5%9D%8E/tianjin-explosions/}
%    \cite{aidenxiatanjinblast}
%    effecten op de lange termijn
%    \bibitem{danwangTanjinflexreport} ... \LaTeX:\\ \url{https://www.flexport.com/blog/tianjin-explosion-effect-on-supply-chains/}
%    \cite{danwangTanjinflexreport}
%    \bibitem{keyHighlightsTanjin} ... \LaTeX:\\ \url{https://www.cicm.org.my/images/articles/CICM-Article-on-Tianjin-Blast-Oct2015.pdf}
%    \cite{keyHighlightsTanjin}
%    lessons learned
%    \bibitem{ID} ... \LaTeX:\\ \url{https://www.genre.com/knowledge/blog/lessons-from-the-tianjin-explosion-en.html}
%    \cite{}
%    \bibitem{ID} ... \LaTeX:\\ \url{https://www.ft.com/content/ad62904c-44ce-11e5-b3b2-1672f710807b}
%    \cite{}
%    \bibitem{hartley13082015videofootage} ... \LaTeX:\\ \url{https://www.huffingtonpost.co.uk/2015/08/13/tianjin-explosion-china-shocking-footage-caught-on-camera_n_7980888.html}
%    \cite{hartley13082015videofootage}
%    \bibitem{odonnel01062017firetanjinblast2015} ... \LaTeX:\\ \url{https://www.thatsmags.com/china/post/19189/massive-fire-rocks-tianjin-port}
%    \cite{odonnel01062017firetanjinblast2015}
%    gevolgen voor de industrie
%    \bibitem{ID} ... \LaTeX:\\ \url{https://www.everstream.ai/risk-center/special-reports/the-jiangsu-yancheng-explosion/}
%    \cite{}
%    \bibitem{fan15082015newyorkermistrustchina} ... \LaTeX:\\ \url{https://www.newyorker.com/news/news-desk/after-tianjin-an-outbreak-of-mistrust-in-china}
%    \cite{fan15082015newyorkermistrustchina}
%    framing vanuit de chinese media
%    \bibitem{yanlidongchinamediaframingTanjin} ... \LaTeX:\\ \url{https://www.neliti.com/publications/101997/the-chinese-media-framing-of-the-2015s-tianjin-explosion}
%    \cite{yanlidongchinamediaframingTanjin}
%    \bibitem{evans27092017TnjinInsurance} ... \LaTeX:\\ \url{https://www.reinsurancene.ws/chinese-insurers-settle-1-5-billion-tianjin-blast-claims/}
%    \cite{evans27092017TnjinInsurance}
%    niewsartikel
%    \bibitem{jasi26032019chineschemplant} ... \LaTeX:\\ \url{https://www.thechemicalengineer.com/news/update-78-confirmed-dead-after-chinese-chemicals-plant-explosion/}
%    \cite{jasi26032019chineschemplant}
%    \bibitem{shiqingTanjinExecutiveSentence} ... \LaTeX:\\ \url{https://www.caixinglobal.com/2016-11-10/chinese-executive-receives-suspended-death-sentence-over-2015-tianjin-warehouse-blast-101006325.html}
%    \cite{shiqingTanjinExecutiveSentence}
%    toegang tot de ramplplek vanuit de okale journalistiek
%    \bibitem{sophiebeach15082015} ... \LaTeX:\\ \url{https://chinadigitaltimes.net/2015/08/he-xiaoxin-how-far-can-i-go-and-how-much-can-i-do/}
%    \cite{sophiebeach15082015}
%    artikel
%    \bibitem{ID} ... \LaTeX:\\ \url{https://www.wnpr.org/post/china-examines-aftermath-immense-twin-explosions-killed-dozens}
%    \cite{}
%    \bibitem{hamzeh05082020BeirutBlast} ... \LaTeX:\\ \url{https://theconversation.com/what-is-ammonium-nitrate-the-chemical-that-exploded-in-beirut-143979}
%    \cite{hamzeh05082020BeirutBlast}
%    \bibitem{chemwatch18082015TanjiinExplosion} ... \LaTeX:\\ \url{https://chemicalwatch.com/36730/nationwide-inspections-in-china-follow-tianjin-explosion}
%    \cite{chemwatch18082015TanjiinExplosion}
%    \bibitem{thehindu15062019chinaExplosion} ... \LaTeX:\\ \url{https://www.thehindu.com/news/international/investigation-begun-into-china-gas-explosion-as-toll-rises/article34818324.ece}
%    \cite{thehindu15062019chinaExplosion}
%    \bibitem{santagotimes24032019chinablast} ... \LaTeX:\\ \url{https://santiagotimes.cl/2019/03/24/64-killed-600-injured-in-china-chemical-plant-blast/}
%    \cite{santagotimes24032019chinablast}
%    oorzaken
%    \bibitem{klingecorp28042020causedTanjin} ... \LaTeX:\\ \url{https://klingecorp.com/blog/what-caused-the-tianjin-explosions/}
%    \cite{klingecorp28042020causedTanjin}
%    case study
%    \bibitem{mcgarryExplosions2017 ... \LaTeX:\\ \url{https://www.preventionweb.net/educational/view/57235}
%    	\cite{mcgarryExplosions2017}
%    	niewsartikel
%    	\bibitem{roswnfeld13082015TanjinReports} ... \LaTeX:\\ \url{https://www.cnbc.com/2015/08/12/explosion-in-tianjin-china.html}
%    	\cite{roswnfeld13082015TanjinReports}
%    	chronologische uiteenzetting
%    	\bibitem{aria12082015explosionaTanjin} ... \LaTeX:\\ \url{https://www.aria.developpement-durable.gouv.fr/wp-content/files_mf/A46803_a46803_fiche_impel_006.pdf}
%    	\cite{aria12082015explosionaTanjin}
%    	corruptie
%    	\bibitem{ID} ... \LaTeX:\\ \url{https://www.nytimes.com/2015/08/31/world/asia/behind-tianjin-tragedy-a-company-that-flouted-regulations-and-reaped-profits.html}
%    	\cite{}
%    	mismanagement als oorzaak
%    	\bibitem{ID} ... \LaTeX:\\ \url{https://www.nytimes.com/2016/02/06/world/asia/tianjin-explosions-were-result-of-mismanagement-china-finds.html}
%    	\cite{}
%    	\bibitem{ID} ... \LaTeX:\\ \url{https://cen.acs.org/articles/94/web/2016/02/Chinese-Investigators-Identify-Cause-Tianjin.html}
%    	\cite{}
%    	autoriteiten publiceren onderoeksrapport
%    	\bibitem{tremblay11022016chineseInvestigatorsTanjin} ... \LaTeX:\\ \url{https://cen.acs.org/articles/94/i7/Chinese-Investigators-Identify-Cause-Tianjin.html}
%    	\cite{tremblay11022016chineseInvestigatorsTanjin}
%    	fotos van de rampplek
%    	\bibitem{taylor13082015TanjinExplosianAftermath} ... \LaTeX:\\ \url{https://www.theatlantic.com/photo/2015/08/photos-of-the-aftermath-of-the-massive-explosions-in-tianjin-china/401228/}
%    	\cite{taylor13082015TanjinExplosianAftermath}
%    	\bibitem{ID} ... \LaTeX:\\ \url{https://edition.cnn.com/2015/08/13/asia/china-tianjin-explosions/index.html}
%    	\cite{}
%    	niuwesartiekel}
%    \bibitem{associatedPresss13082013} ... \LaTeX:\\ \url{https://www.cbc.ca/news/world/china-explosion-tianjin-1.3189455}
%    \cite{associatedPresss13082013}
%    verantwoordelijke
%    \bibitem{ID} ... \LaTeX:\\ \url{https://www.thestar.com/news/world/2016/11/09/chinese-executive-gets-death-sentence-over-tianjin-explosion-in-2015.html}
%    \cite{}
%    risicobeperking/controle
%    \bibitem{ID} ... \LaTeX:\\ \url{https://www.swissre.com/en/china/news-insights/articles/analysis-of-tianjin-port-explosion-china.html}
%    \cite{}
%    censuur
%    \bibitem{ID} ... \LaTeX:\\ \url{https://foreignpolicy.com/2015/09/10/censored-china-young-survivor-tianjin-explosion-viral-post/}
%    \cite{}
%    censuur
%    \bibitem{ID} ... \LaTeX:\\ \url{https://qz.com/756872/a-year-after-the-tianjin-blast-public-mourning-and-discussion-about-it-are-still-censored-in-china/}
%    \cite{}
%    verschillende artikelen
%    \bibitem{ID} ... \LaTeX:\\ \url{https://www.scmp.com/topics/tianjin-warehouse-explosion-2015}
%    \cite{}
%    \bibitem{ID} ... \LaTeX:\\ \url{https://www.wsj.com/articles/BL-CJB-27664}
%    \cite{}
%    \bibitem{ID} ... \LaTeX:\\ \url{https://www.nbcnews.com/news/world/tianjin-explosions-californian-witness-filmed-dramatic-china-blasts-n409701}
%    \cite{}
%    \bibitem{ID} ... \LaTeX:\\ \url{https://ui.adsabs.harvard.edu/abs/2016AGUFM.S13D..06P/abstract}
%    afwikkeling van de ramp
%    \cite{}
%    \bibitem{ID} ... \LaTeX:\\ \url{https://chinadialogue.net/en/pollution/9188-back-to-the-blast-zone-one-year-after-the-tianjin-explosion/}
%    \cite{}
%    \bibitem{ID} ... \LaTeX:\\ \url{https://www.wired.com/2015/08/chinas-huge-tianjin-explosion-looked-like-space/}
%    \cite{}
%    \bibitem{ID} ... \LaTeX:\\ \url{https://www.abc.net.au/news/2015-08-13/explosion-rocks-north-chinese-city-of-tianjin/6693336?nw=0}
%    \cite{}
%    ambtenaren onderzocht
%    
%    risico-inschatting
%    \bibitem{ID} ... \LaTeX:\\ \url{https://www.mdpi.com/2071-1050/12/3/1169/htm}
%    \cite{}
%    \bibitem{ID} ... \LaTeX:\\ \url{https://www.mdpi.com/2071-1050/12/3/1169/htm}
%    \cite{}
%    \bibitem{ID} ... \LaTeX:\\ \url{https://www.cbsnews.com/news/tianjin-port-china-massive-explosion-hundreds-injured/}
%    \cite{}
%    \bibitem{ID} ... \LaTeX:\\ \url{https://www.hkjcdpri.org.hk/download/casestudies/Tianjin_CASE.pdf}
%    \cite{}
%    \bibitem{ID} ... \LaTeX:\\ \url{https://time.com/3996168/tianjin-explosion-china-pictures/}
%    \cite{}
%    onderzoeksrapport
%    \bibitem{ID} ... \LaTeX:\\ \url{https://www.hfw.com/Tianjin-Port-explosion-August-2015}
%    \cite{}
%    \bibitem{un20082015InvestigationTanjin} ... \LaTeX:\\ \url{https://news.un.org/en/story/2015/08/506912-following-tianjin-explosion-un-expert-calls-china-ensure-transparent}
%    \cite{un20082015InvestigationTanjin}
%    \bibitem{france2412082015TnjinExplosion} ... \LaTeX:\\ \url{https://www.france24.com/en/20150812-huge-explosions-rock-chinese-city-tianjin}
%    \cite{france2412082015TnjinExplosion}
%    \bibitem{npr14082015TanjinCause} ... \LaTeX:\\ \url{https://choice.npr.org/index.html?origin=https://www.npr.org/2015/08/14/432280627/what-caused-the-warehouse-explosions-in-tianjin-china}
%    \cite{npr14082015TanjinCause}
%    123 verantwoordelijken
%    \bibitem{bbc05022016TanjinResponsibles} ... \LaTeX:\\ \url{https://www.bbc.com/news/world-asia-china-35506311}
%    \cite{bbc05022016TanjinResponsibles}
%    \bibitem{ID} ... \LaTeX:\\ \url{https://www.washingtonpost.com/gdpr-consent/?next_url=https%3a%2f%2fwww.washingtonpost.com%2fnews%2fworldviews%2fwp%2f2015%2f08%2f12%2fvideos-show-chinese-city-of-tianjin-rocked-by-enormous-explosion%2f}
%    \cite{}
%    lang artiekel
%    \bibitem{CBodeen15082015TanjinExplosion} ... \LaTeX:\\ \url{https://www.businessinsider.com/the-chemical-explosion-in-china-killed-more-than-100-people-and-the-devastation-is-unreal-2015-8?international=true&r=US&IR=T}
%    \cite{CBodeen15082015TanjinExplosion}
%    \bibitem{ID} ... \LaTeX:\\ \url{https://pubmed.ncbi.nlm.nih.gov/27311537/}
%    \cite{}
%    \bibitem{reutersTanjinInsurance} ... \LaTeX:\\ \url{https://www.reuters.com/article/us-china-blast-insurance-idUSKCN0QM0N220150817}
%    \cite{reutersTanjinInsurance}
%    \bibitem{yu082016evaluationTanjin2015} ... \LaTeX:\\ \url{https://www.sciencedirect.com/science/article/abs/pii/S0305417916300079}
%    \cite{yu082016evaluationTanjin2015}
%    \bibitem{wiki2015TanjinExplosions} ... \LaTeX:\\ \url{https://en.wikipedia.org/wiki/2015_Tianjin_explosions}
%    \cite{wiki2015TanjinExplosions}
%    \bibitem{bbc17082015whathappenedTanjin} ... \LaTeX:\\ \url{https://www.bbc.com/news/world-asia-china-33844084}
%    \cite{bbc17082015whathappenedTanjin}
%    \bibitem{mortimer19082016taijinexplosioncrater} ... \LaTeX:\\ \url{https://www.independent.co.uk/news/world/asia/tianjin-explosion-photos-china-chemical-factory-accident-crater-revealed-a7199591.html}
%    \cite{mortimer19082016taijinexplosioncrater}
%    veiigheidshandhaving
%    \bibitem{internationallabourofficeChmControlTooliit} ... \LaTeX:\\ \url{https://www.ilo.org/legacy/english/protection/safework/ctrl_banding/toolkit/main_guide.pdf}
%    \cite{internationallabourofficeChmControlTooliit}
%    \bibitem{ID} ... \LaTeX:\\ \url{https://echa.europa.eu/documents/10162/21332507/guide_chemical_safety_sme_en.pdf}
%    \cite{}
%    \bibitem{euTaxationCustomsICSC} ... \LaTeX:\\ \url{https://ec.europa.eu/taxation_customs/dds2/SAMANCTA/EN/Safety/AppendixD_EN.htm}
%    \cite{euTaxationCustomsICSC}
%    \bibitem{iloWHOChemSafetyCards} ... \LaTeX:\\ \url{https://www.ilo.org/safework/info/publications/WCMS_113134/lang--en/index.htm}
%    \cite{iloWHOChemSafetyCards}
%    
%    
%    
%    Wat is er gebeurd?
%    @online{wikiSchipholbrand,	ALTauthor = {author},	ALTeditor = {editor},	title = {title},	date = {date},	url = {"https://nl.wikipedia.org/wiki/Schipholbrand"},}
%    artikel
%    @online{schipholbrand27102005video,	ALTauthor = {author},	ALTeditor = {editor},	title = {title},	date = {date},	url = {"https://www.youtube.com/watch?v=1i-hfEzxFfk"},}
%    psychologische gevolgen
%    rapport
%    @online{onderzoeksraad2610schipholoost,	ALTauthor = {author},	ALTeditor = {editor},	title = {title},	date = {date},	url = {"https://www.onderzoeksraad.nl/nl/page/392/brand-cellencomplex-schiphol-oost-nacht-van-26-op-27-oktober"},}
%    artikel met video
%    herdenking
%    impact op de persoon
%    herdenking
%    @online{schipholbrandvideoargos,	ALTauthor = {author},	ALTeditor = {editor},	title = {title},	date = {date},	url = {"https://www.vpro.nl/argos/speel~POMS_VPRO_461907~schadevergoeding-voor-ex-verdachte-schipholbrand~.html"},}
%    chronologie
%    @online{nunl30052023feitenoverzicht,	ALTauthor = {author},	ALTeditor = {editor},	title = {title},	date = {date},	url = {"https://www.nu.nl/binnenland/3355935/feitenoverzicht-schipholbrand-en-rechtszaken.html"},}
%    tijdlijn
%    @online{ID,	ALTauthor = {author},	ALTeditor = {editor},	title = {title},	date = {date},	url = {"https://www.singeluitgeverijen.nl/isbn/de-schipholbrand/"},}
%    vervolgens van ministers
%    beeldanalyse en reconstructie
%    @online{ID,	ALTauthor = {author},	ALTeditor = {editor},	title = {title},	date = {date},	url = {"https://eenvandaag.avrotros.nl/item/schipholbrand-niet-ontstaan-in-cel-11/"},}
%    herdenking
%    korte samenvatting
%    rapport
%    artikel
%    verwijzing naar het rapport vanuit de politieke oppositie
%    beeld vanuit de gevangenisbewaarder
%    nationaliteit slachtoffers schipholbrand
%    verblijfsvergunning voor de slachtoffers
%    gen schadevergoeding voor de verdachte
%    verdachte voor de rechter
%    geen schadevergoeding voor verdachte
%    artikel wat ging er mis bji de schipholbrand
%    brand veroorzaakt door een peuk
%    smaadschrift
%    bewakers worden niet vervolgd
%    proces schipholbrand moet over en de brandveilgheid moet worden verbeterd
%    de rol van het parlement in de evaluatie
%    @online{parlementairemonitorschipholbrand,	ALTauthor = {author},	ALTeditor = {editor},	title = {title},	date = {date},	url = {"https://www.parlementairemonitor.nl/9353000/1/j9vvij5epmj1ey0/vi3aof7awcxg"},}
%    onderzoeksmemo
%    herdenking
%    @online{ID,	ALTauthor = {author},	ALTeditor = {editor},	title = {title},	date = {date},	url = {"https://archief.ntr.nl/nova/page/detail/uitzendingen/3847/Den%20Haag%20Vandaag_%20herdenking%20Schipholbrand.html"},}
%    herdenking
%    invloed van de ramp op samenleving
%    @online{videonpoNOVA13112008,	ALTauthor = {author},	ALTeditor = {editor},	title = {title},	date = {date},	url = {"https://www.npostart.nl/heropen-onderzoek-schipholbrand/13-11-2008/POMS_NTR_103332"},}
%    opmerkelijk rapport gestolen in de nasleep
%    @online{rizoomes01052014schipholbrand,	ALTauthor = {author},	ALTeditor = {editor},	title = {title},	date = {date},	url = {"https://www.rizoomes.nl/brandweer/brand-cellencomplex-schiphol/"},}
%    
%    
%    
%    publicaties
%    @online{heuvelkroesschipholbrandcamerabeelden,	ALTauthor = {author},	ALTeditor = {editor},	title = {title},	date = {date},	url = {"http://www.msnp.nl/downloads/Onderzoeksmemo%20beeldanalyse%20Schipholbrand%20prot.pdf"},}
%    Wat waren de regels destijds?
%    Waren de autoriteiten in staat om op tijd in te grijpen of om erger te voorkomen?
%    Wat is er gedaan om de veiligheid van illegalen en gevangenissbewaarders te verbeteren
%    
%    
%    
%    algemene vragen
%    oorzaken
%    \bibitem{gates18112020boeingcrisis} ... \LaTeX:\\ \url{https://www.seattletimes.com/business/boeing-aerospace/what-led-to-boeings-737-max-crisis-a-qa/}
%    \cite{gates18112020boeingcrisis}
%    \bibitem{boeing737maxsoftwareprobles} ... \LaTeX:\\ \url{https://www.schneier.com/blog/archives/2019/04/excellent_analy.html}
%    \cite{boeing737maxsoftwareprobles}
%    fout in de software
%    \bibitem{avetisov19032019boeingmalwarestate} ... \LaTeX:\\ \url{https://www.forbes.com/sites/georgeavetisov/2019/03/19/malware-at-30000-feet-what-the-737-max-says-about-the-state-of-airplane-software-security/?sh=4d26f7052a9e}
%    \cite{avetisov19032019boeingmalwarestate}
%    het nationaal veiligheidsbelang
%    \bibitem{thompson23112020nationalsecurityboeing} ... \LaTeX:\\ \url{https://www.forbes.com/sites/lorenthompson/2020/11/23/five-reasons-return-of-boeings-737-max-to-service-is-important-to-national-security/?sh=2128ea552018}
%    \cite{thompson23112020nationalsecurityboeing}
%    falend toezicht
%    \bibitem{gates21032019FAAControlSystem} ... \LaTeX:\\ \url{https://www.seattletimes.com/business/boeing-aerospace/failed-certification-faa-missed-safety-issues-in-the-737-max-system-implicated-in-the-lion-air-crash/}
%    \cite{gates21032019FAAControlSystem}
%    onderzoeksrapport
%    \bibitem{faa18112020boeingreview} ... \LaTeX:\\ \url{https://www.faa.gov/foia/electronic_reading_room/boeing_reading_room/media/737_RTS_Summary.pdf}
%    \cite{faa18112020boeingreview}
%    \bibitem{wiki737maxgroundings} ... \LaTeX:\\ \url{https://en.wikipedia.org/wiki/Boeing_737_MAX_groundings}
%    \cite{wiki737maxgroundings}
%    veiligheidsrisico's
%    menselijke fouten
%    \bibitem{campbell02052019boengcrashhumanerrors} ... \LaTeX:\\ \url{https://www.theverge.com/2019/5/2/18518176/boeing-737-max-crash-problems-human-error-mcas-faa}
%    \cite{campbell02052019boengcrashhumanerrors}
%    overzicht van crashes
%    \bibitem{hawkins22032019737maxairplanes} ... \LaTeX:\\ \url{https://www.theverge.com/2019/3/22/18275736/boeing-737-max-plane-crashes-grounded-problems-info-details-explained-reasons}
%    \cite{hawkins22032019737maxairplanes}
%    veiligheidsopmerking
%    \bibitem{thomas30082020737safest} ... \LaTeX:\\ \url{https://www.airlineratings.com/news/boeings-737-max-will-one-safest-aircraft-history/}
%    \cite{thomas30082020737safest}
%    aanpassingen
%    \bibitem{boeing737maxdisplay} ... \LaTeX:\\ \url{https://www.boeing.com/commercial/737max/737-max-software-updates.page}
%    \cite{boeing737maxdisplay}
%    waarschuwingen//output signalen
%    \bibitem{fehrm24112020737changes} ... \LaTeX:\\ \url{https://leehamnews.com/2020/11/24/boeing-737-max-changes-beyond-mcas/}
%    \cite{fehrm24112020737changes}
%    software gerelateerde fouten
%    \bibitem{travis18042019737maxsoftwaredevop} ... \LaTeX:\\ \url{https://spectrum.ieee.org/aerospace/aviation/how-the-boeing-737-max-disaster-looks-to-a-software-developer}
%    \cite{travis18042019737maxsoftwaredevop}
%    onderzoeksrapport
%    de rol van de publieke opinie
%    \bibitem{barnett05052019737maxcrisis} ... \LaTeX:\\ \url{https://pubsonline.informs.org/do/10.1287/orms.2019.05.05/full/}
%    \cite{barnett05052019737maxcrisis}
%    onderzoek van europese luchtvaart agentschap
%    \bibitem{easa27012021737maxsafereturn} ... \LaTeX:\\ \url{https://www.easa.europa.eu/newsroom-and-events/news/easa-declares-boeing-737-max-safe-return-service-europe}
%    \cite{easa27012021737maxsafereturn}
%    \veiligheidsvraagstuk
%    \bibitem{touitou11032019737tragedies} ... \LaTeX:\\ \url{https://phys.org/news/2019-03-boeing-max-safety-tragedies.html}
%    \cite{touitou11032019737tragedies}
%    artikel over sensoren
%    \bibitem{hemmerdinger02022021737maxdeliveries} ... \LaTeX:\\ \url{https://www.flightglobal.com/airframers/boeing-delays-737-max-10-deliveries-two-years-to-2023/142245.article}
%    \cite{hemmerdinger02022021737maxdeliveries}
%    goedkeuring van europese luchtvaart autoriteiten
%    advies aan de faa
%    \bibitem{bielby27022021faaimprovesafety} ... \LaTeX:\\ \url{https://www.hstoday.us/subject-matter-areas/airport-aviation-security/oig-tells-faa-to-improve-safety-oversight-following-boeing-737-max-review/}
%    \cite{bielby27022021faaimprovesafety}
%    \bibitem{boyle18112020737maxupgrade} ... \LaTeX:\\ \url{https://www.geekwire.com/2020/faas-go-ahead-737-maxs-return-flight-kicks-off-massive-software-upgrade/}
%    \cite{boyle18112020737maxupgrade}
%    \bibitem{bergstraburgess122019737maxMcasAlgorithm} ... \LaTeX:\\ \url{https://www.researchgate.net/publication/338420944_A_Promise_Theoretic_Account_of_the_Boeing_737_Max_MCAS_Algorithm_Affair}
%    \cite{bergstraburgess122019737maxMcasAlgorithm}
%    achtergrond informatie
%    \bibitem{737mcas} ... \LaTeX:\\ \url{http://www.b737.org.uk/mcas.htm}
%    \cite{737mcas}
%    algemeen vertrouwen
%    \bibitem{newburger17052019boeingcrisis} ... \LaTeX:\\ \url{https://www.cnbc.com/2019/05/16/what-you-need-to-know-about-boeings-737-max-crisis.html}
%    \cite{newburger17052019boeingcrisis}
%    toestemming europese autoriteiten
%    problemen
%    \bibitem{arstechnica22012020737problems} ... \LaTeX:\\ \url{https://arstechnica.com/information-technology/2020/01/737-max-fix-slips-to-summer-and-thats-just-one-of-boeings-problems/}
%    \cite{arstechnica22012020737problems}
%    uitgebreid artikel over de onderzoeken en het vliegverbod
%    \bibitem{german190620217372yaftergrounded} ... \LaTeX:\\ \url{https://www.cnet.com/news/boeing-737-max-8-all-about-the-aircraft-flight-ban-and-investigations/}
%    \cite{german190620217372yaftergrounded}
%    computers als oorzaak
%    lessons learned
%    \bibitem{beningo02052019boeinglessons} ... \LaTeX:\\ \url{https://www.designnews.com/electronics-test/5-lessons-learn-boeing-737-max-fiasco}
%    \cite{beningo02052019boeinglessons}
%    \bibitem{duran05042019boeingspof} ... \LaTeX:\\ \url{https://www.eurocontrol.int/publication/effects-network-extra-standby-aircraft-and-boeing-737-max-grounding}
%    \cite{duran05042019boeingspof}
%    single point of failure
%    \bibitem{ID} ... \LaTeX:\\ \url{https://dmd.solutions/blog/2019/04/05/how-a-single-point-of-failure-spof-in-the-mcas-software-could-have-caused-the-boeing-737-max-crash-in-ethiopia/}
%    \cite{}
%    \bibitem{makichuck24012021737fearflying} ... \LaTeX:\\ \url{https://asiatimes.com/2021/01/boeings-737-max-and-the-fear-of-flying/}
%    \cite{makichuck24012021737fearflying}
%    lijst van tehnische aanpassingen
%    \bibitem{caa737modifications} ... \LaTeX:\\ \url{https://www.caa.co.uk/Consumers/Guide-to-aviation/Boeing-737-MAX/}
%    \cite{caa737modifications}
%    \bibitem{oestergaard14122020boeingdeliveries} ... \LaTeX:\\ \url{https://dsm.forecastinternational.com/wordpress/2020/12/14/airbus-and-boeing-report-november-2020-commercial-aircraft-orders-and-deliveries/}
%    \cite{oestergaard14122020boeingdeliveries}
%    code lek
%    \bibitem{reenberg787flaws} ... \LaTeX:\\ \url{https://www.wired.com/story/boeing-787-code-leak-security-flaws/}
%    \cite{reenberg787flaws}
%    \bibitem{fitch16092020737backlogrisks} ... \LaTeX:\\ \url{https://www.fitchratings.com/research/corporate-finance/boeing-737-max-return-backlog-risks-remain-16-09-2020}
%    \cite{fitch16092020737backlogrisks}
%    Cultuurverandering, deregulatie, systeemwijziging of gewoon een kwestie van competentie
%    \bibitem{willis27082020737maxfailures} ... \LaTeX:\\ \url{https://www.aerospacetestinginternational.com/features/what-broke-the-737-max.html}
%    \cite{willis27082020737maxfailures}
%    extra aanpassingen
%    \bibitem{ostrower11062020more737changes} ... \LaTeX:\\ \url{https://theaircurrent.com/aviation-safety/boeings-737-max-software-done-but-regulators-plot-more-changes-after-jets-return/}
%    \cite{ostrower11062020more737changes}
%    wat ging er mis een analyse van een ex-iloot
%    De utoriteiten waren op de hoogte
%    \bibitem{hruska13122019faaknown737crashrate} ... \LaTeX:\\ \url{https://www.extremetech.com/extreme/303373-the-faa-knew-the-737-max-was-dangerous-and-kept-it-flying-anyway}
%    \cite{hruska13122019faaknown737crashrate}
%    kwaliteiten van het alarmsysteem niet goed bekend
%    \bibitem{bloomberg26092019failedpred} ... \LaTeX:\\ \url{https://time.com/5687473/boeing-737-alarm-system/}
%    \cite{bloomberg26092019failedpred}
%    \bibitem{whiteman09072020boengcancelstock} ... \LaTeX:\\ \url{https://www.nasdaq.com/articles/boeing-gets-dealt-another-737-max-cancellation-blow.-what-it-means-for-boeing-stock-2020}
%    \cite{whiteman09072020boengcancelstock}
%    \bibitem{leopold09192019boeingreliability} ... \LaTeX:\\ \url{https://www.eetimes.com/boeing-crashes-highlight-a-worsening-reliability-crisis/}
%    \cite{leopold09192019boeingreliability}
%    veiligheidsvraagstuk
%    \bibitem{koenig11122019737crashesnofix} ... \LaTeX:\\ \url{https://www.latimes.com/business/story/2019-12-11/faa-boeing-737-max-crashes}
%    \cite{koenig11122019737crashesnofix}
%    probleemanalyse, veiligheidsvraagstuk
%    \bibitem{dohertylindeman15032019737problems} ... \LaTeX:\\ \url{https://www.politico.com/story/2019/03/15/boeing-737-max-grounding-1223072}
%    \cite{dohertylindeman15032019737problems}
%    falend toezicht
%    \bibitem{stodder02102019corruptoversight} ... \LaTeX:\\ \url{https://www.pogo.org/analysis/2019/10/corrupted-oversight-the-faa-boeing-and-the-737-max/}
%    \cite{stodder02102019corruptoversight}
%    \bibitem{afacwaLostSafeguards} ... \LaTeX:\\ \url{https://www.afacwa.org/the_inside_story_of_mcas_seattle_times}
%    \cite{afacwaLostSafeguards}
%    doelstellingen en veiligheidsvraagstukken
%    \bibitem{swayne18032019profitssafety} ... \LaTeX:\\ \url{https://www.marxist.com/737-max-scandal-boeing-putting-profits-before-safety.htm}
%    \cite{swayne18032019profitssafety}
%    \bibitem{freed26022021liftaustraliaban} ... \LaTeX:\\ \url{https://finance.yahoo.com/news/australia-lifts-ban-boeing-737-035817682.html?guccounter=1&guce_referrer=aHR0cHM6Ly93d3cuZ29vZ2xlLmNvbS8&guce_referrer_sig=AQAAAHZCJYy_0A5VS2WiPoCvH4xdrRNkmkdsv5EWJ2RLIz_AS-rxsTty6AF1_HlmJiRyWYqCXDi4p0Xs4isYkNkCq2Pfo-pQ60Xz_IfTNjm4FgoZiBMC4zpZlB6F0fwecrjE_ujAXZzG4xPJnWCd8-G3VLlPTY8h3H31eQ1i8hY9AIyy}
%    \cite{freed26022021liftaustraliaban}
%    autoriteiten krijgen tik op de vingers
%    \bibitem{reed15032019softwareattention} ... \LaTeX:\\ \url{https://medium.com/@jpaulreed/the-737max-and-why-software-engineers-should-pay-attention-a041290994bd}
%    \cite{reed15032019softwareattention}
%    \bibitem{news17032019softwareexplains} ... \LaTeX:\\ \url{https://news.ycombinator.com/item?id=19414775}
%    \cite{news17032019softwareexplains}
%    \bibitem{legget21122020eu737maxsafe} ... \LaTeX:\\ \url{https://www.bbc.com/news/55366320}
%    \cite{legget21122020eu737maxsafe}
%    \bibitem{marketscreener0103221737chinarecertification} ... \LaTeX:\\ \url{https://www.marketscreener.com/news/latest/China-studies-Boeing-737-MAX-recertification-wants-safety-concerns-fully-addressed--32569125/}
%    \cite{marketscreener0103221737chinarecertification}
%    motor in brand
%    \bibitem{euractiv22022021737firegrounds} ... \LaTeX:\\ \url{https://www.euractiv.com/section/aviation/news/boeing-grounds-777s-after-engine-fire/}
%    \cite{euractiv22022021737firegrounds}
%    \bibitem{benny18022019737returnUAE} ... \LaTeX:\\ \url{https://gulfnews.com/business/aviation/uae-airspace-to-see-return-of-boeing-737-max-1.1613627548923}
%    \cite{benny18022019737returnUAE}
%    motor in brand gevlogen
%    \bibitem{biersmichel22022021777grounds} ... \LaTeX:\\ \url{https://techxplore.com/news/2021-02-boeing-urges-grounding-777s.html}
%    \cite{biersmichel22022021777grounds}
%    \bibitem{ID} ... \LaTeX:\\ \url{https://www.politico.eu/article/uk-temporarily-bans-some-boeing-aircraft-after-pratt-whitney-engine-incidents/}
%    \cite{}
%    \bibitem{reuters23022021777metalfatigue} ... \LaTeX:\\ \url{https://www.timeslive.co.za/news/world/2021-02-23-damage-to-united-boeing-777-engine-consistent-with-metal-fatigue--ntsb/}
%    \cite{reuters23022021777metalfatigue}
%    faa was niet kritisch genoeg
%    \bibitem{ID} ... \LaTeX:\\ \url{https://federalnewsnetwork.com/government-news/2021/02/federal-watchdog-blasts-faa-over-certification-of-boeing-jet/}
%    \cite{}
%    
%    
%    
%    algemene vragen
%    oorzaken
%    \bibitem{{gates18112020boeingcrisis} ... \LaTeX:\\ \url{https://www.seattletimes.com/business/boeing-aerospace/what-led-to-boeings-737-max-crisis-a-qa/}
%    	
%    	\bibitem{{boeing737maxsoftwareprobles} ... \LaTeX:\\ \url{https://www.schneier.com/blog/archives/2019/04/excellent_analy.html}
%    		fout in de software
%    		\bibitem{{avetisov19032019boeingmalwarestate} ... \LaTeX:\\ \url{https://www.forbes.com/sites/georgeavetisov/2019/03/19/malware-at-30000-feet-what-the-737-max-says-about-the-state-of-airplane-software-security/?sh=4d26f7052a9e}
%    			het nationaal veiligheidsbelang
%    			\bibitem{{thompson23112020nationalsecurityboeing} ... \LaTeX:\\ \url{https://www.forbes.com/sites/lorenthompson/2020/11/23/five-reasons-return-of-boeings-737-max-to-service-is-important-to-national-security/?sh=2128ea552018}
%    				falend toezicht
%    				\bibitem{{gates21032019FAAControlSystem} ... \LaTeX:\\ \url{https://www.seattletimes.com/business/boeing-aerospace/failed-certification-faa-missed-safety-issues-in-the-737-max-system-implicated-in-the-lion-air-crash/}
%    					onderzoeksrapport
%    					\bibitem{{faa18112020boeingreview} ... \LaTeX:\\ \url{https://www.faa.gov/foia/electronic_reading_room/boeing_reading_room/media/737_RTS_Summary.pdf}
%    						\bibitem{{wiki737maxgroundings} ... \LaTeX:\\ \url{https://en.wikipedia.org/wiki/Boeing_737_MAX_groundings}
%    							veiligheidsrisico's
%    							menselijke fouten
%    							\bibitem{{campbell02052019boengcrashhumanerrors} ... \LaTeX:\\ \url{https://www.theverge.com/2019/5/2/18518176/boeing-737-max-crash-problems-human-error-mcas-faa}
%    								overzicht van crashes
%    								\bibitem{{hawkins22032019737maxairplanes} ... \LaTeX:\\ \url{https://www.theverge.com/2019/3/22/18275736/boeing-737-max-plane-crashes-grounded-problems-info-details-explained-reasons}
%    									veiligheidsopmerking
%    									\bibitem{{thomas30082020737safest} ... \LaTeX:\\ \url{https://www.airlineratings.com/news/boeings-737-max-will-one-safest-aircraft-history/}
%    										aanpassingen
%    										\bibitem{{boeing737maxdisplay} ... \LaTeX:\\ \url{https://www.boeing.com/commercial/737max/737-max-software-updates.page}
%    											waarschuwingen//output signalen
%    											\bibitem{{fehrm24112020737changes} ... \LaTeX:\\ \url{https://leehamnews.com/2020/11/24/boeing-737-max-changes-beyond-mcas/}
%    												software gerelateerde fouten
%    												\bibitem{{travis18042019737maxsoftwaredevop} ... \LaTeX:\\ \url{https://spectrum.ieee.org/aerospace/aviation/how-the-boeing-737-max-disaster-looks-to-a-software-developer}
%    													onderzoeksrapport
%    													de rol van de publieke opinie
%    													\bibitem{{barnett05052019737maxcrisis} ... \LaTeX:\\ \url{https://pubsonline.informs.org/do/10.1287/orms.2019.05.05/full/}
%    														onderzoek van europese luchtvaart agentschap
%    														\bibitem{{easa27012021737maxsafereturn} ... \LaTeX:\\ \url{https://www.easa.europa.eu/newsroom-and-events/news/easa-declares-boeing-737-max-safe-return-service-europe}
%    															veiligheidsvraagstuk
%    															\bibitem{{touitou11032019737tragedies} ... \LaTeX:\\ \url{https://phys.org/news/2019-03-boeing-max-safety-tragedies.html}
%    																artikel over sensoren
%    																\bibitem{{hemmerdinger02022021737maxdeliveries} ... \LaTeX:\\ \url{https://www.flightglobal.com/airframers/boeing-delays-737-max-10-deliveries-two-years-to-2023/142245.article}
%    																	goedkeuring van europese luchtvaart autoriteiten
%    																	advies aan de faa
%    																	\bibitem{{bielby27022021faaimprovesafety} ... \LaTeX:\\ \url{https://www.hstoday.us/subject-matter-areas/airport-aviation-security/oig-tells-faa-to-improve-safety-oversight-following-boeing-737-max-review/}
%    																		\bibitem{{boyle18112020737maxupgrade} ... \LaTeX:\\ \url{https://www.geekwire.com/2020/faas-go-ahead-737-maxs-return-flight-kicks-off-massive-software-upgrade/}
%    																			\bibitem{{bergstraburgess122019737maxMcasAlgorithm} ... \LaTeX:\\ \url{https://www.researchgate.net/publication/338420944_A_Promise_Theoretic_Account_of_the_Boeing_737_Max_MCAS_Algorithm_Affair}
%    																				achtergrond informatie
%    																				\bibitem{{737mcas} ... \LaTeX:\\ \url{http://www.b737.org.uk/mcas.htm}
%    																					algemeen vertrouwen
%    																					\bibitem{{newburger17052019boeingcrisis} ... \LaTeX:\\ \url{https://www.cnbc.com/2019/05/16/what-you-need-to-know-about-boeings-737-max-crisis.html}
%    																						toestemming europese autoriteiten
%    																						problemen
%    																						\bibitem{{arstechnica22012020737problems} ... \LaTeX:\\ \url{https://arstechnica.com/information-technology/2020/01/737-max-fix-slips-to-summer-and-thats-just-one-of-boeings-problems/}
%    																							uitgebreid artikel over de onderzoeken en het vliegverbod
%    																							\bibitem{{german190620217372yaftergrounded} ... \LaTeX:\\ \url{https://www.cnet.com/news/boeing-737-max-8-all-about-the-aircraft-flight-ban-and-investigations/}
%    																								computers als oorzaak
%    																								lessons learned
%    																								\bibitem{{beningo02052019boeinglessons} ... \LaTeX:\\ \url{https://www.designnews.com/electronics-test/5-lessons-learn-boeing-737-max-fiasco}
%    																									\bibitem{{duran05042019boeingspof} ... \LaTeX:\\ \url{https://www.eurocontrol.int/publication/effects-network-extra-standby-aircraft-and-boeing-737-max-grounding}
%    																										single point of failure
%    																										\bibitem{{ID} ... \LaTeX:\\ \url{https://dmd.solutions/blog/2019/04/05/how-a-single-point-of-failure-spof-in-the-mcas-software-could-have-caused-the-boeing-737-max-crash-in-ethiopia/}
%    																											\bibitem{{makichuck24012021737fearflying} ... \LaTeX:\\ \url{https://asiatimes.com/2021/01/boeings-737-max-and-the-fear-of-flying/}
%    																												lijst van tehnische aanpassingen
%    																												\bibitem{{caa737modifications} ... \LaTeX:\\ \url{https://www.caa.co.uk/Consumers/Guide-to-aviation/Boeing-737-MAX/}
%    																													\bibitem{{oestergaard14122020boeingdeliveries} ... \LaTeX:\\ \url{https://dsm.forecastinternational.com/wordpress/2020/12/14/airbus-and-boeing-report-november-2020-commercial-aircraft-orders-and-deliveries/}
%    																														code lek
%    																														\bibitem{{reenberg787flaws} ... \LaTeX:\\ \url{https://www.wired.com/story/boeing-787-code-leak-security-flaws/}
%    																															\bibitem{{fitch16092020737backlogrisks} ... \LaTeX:\\ \url{https://www.fitchratings.com/research/corporate-finance/boeing-737-max-return-backlog-risks-remain-16-09-2020}
%    																																Cultuurverandering, deregulatie, systeemwijziging of gewoon een kwestie van competentie
%    																																\bibitem{{willis27082020737maxfailures} ... \LaTeX:\\ \url{https://www.aerospacetestinginternational.com/features/what-broke-the-737-max.html}
%    																																	extra aanpassingen
%    																																	\bibitem{{ostrower11062020more737changes} ... \LaTeX:\\ \url{https://theaircurrent.com/aviation-safety/boeings-737-max-software-done-but-regulators-plot-more-changes-after-jets-return/}
%    																																		wat ging er mis een analyse van een ex-iloot
%    																																		De utoriteiten waren op de hoogte
%    																																		\bibitem{{hruska13122019faaknown737crashrate} ... \LaTeX:\\ \url{https://www.extremetech.com/extreme/303373-the-faa-knew-the-737-max-was-dangerous-and-kept-it-flying-anyway}
%    																																			kwaliteiten van het alarmsysteem niet goed bekend
%    																																			\bibitem{{bloomberg26092019failedpred} ... \LaTeX:\\ \url{https://time.com/5687473/boeing-737-alarm-system/}
%    																																				\bibitem{{whiteman09072020boengcancelstock} ... \LaTeX:\\ \url{https://www.nasdaq.com/articles/boeing-gets-dealt-another-737-max-cancellation-blow.-what-it-means-for-boeing-stock-2020}
%    																																					\bibitem{{leopold09192019boeingreliability} ... \LaTeX:\\ \url{https://www.eetimes.com/boeing-crashes-highlight-a-worsening-reliability-crisis/}
%    																																						veiligheidsvraagstuk
%    																																						\bibitem{{koenig11122019737crashesnofix} ... \LaTeX:\\ \url{https://www.latimes.com/business/story/2019-12-11/faa-boeing-737-max-crashes}
%    																																							probleemanalyse, veiligheidsvraagstuk
%    																																							\bibitem{{dohertylindeman15032019737problems} ... \LaTeX:\\ \url{https://www.politico.com/story/2019/03/15/boeing-737-max-grounding-1223072}
%    																																								falend toezicht
%    																																								\bibitem{{stodder02102019corruptoversight} ... \LaTeX:\\ \url{https://www.pogo.org/analysis/2019/10/corrupted-oversight-the-faa-boeing-and-the-737-max/}
%    																																									\bibitem{{afacwaLostSafeguards} ... \LaTeX:\\ \url{https://www.afacwa.org/the_inside_story_of_mcas_seattle_times}
%    																																										doelstellingen en veiligheidsvraagstukken
%    																																										\bibitem{{swayne18032019profitssafety} ... \LaTeX:\\ \url{https://www.marxist.com/737-max-scandal-boeing-putting-profits-before-safety.htm}
%    																																											\bibitem{{freed26022021liftaustraliaban} ... \LaTeX:\\ \url{https://finance.yahoo.com/news/australia-lifts-ban-boeing-737-035817682.html?guccounter=1&guce_referrer=aHR0cHM6Ly93d3cuZ29vZ2xlLmNvbS8&guce_referrer_sig=AQAAAHZCJYy_0A5VS2WiPoCvH4xdrRNkmkdsv5EWJ2RLIz_AS-rxsTty6AF1_HlmJiRyWYqCXDi4p0Xs4isYkNkCq2Pfo-pQ60Xz_IfTNjm4FgoZiBMC4zpZlB6F0fwecrjE_ujAXZzG4xPJnWCd8-G3VLlPTY8h3H31eQ1i8hY9AIyy}
%    																																												autoriteiten krijgen tik op de vingers
%    																																												\bibitem{{reed15032019softwareattention} ... \LaTeX:\\ \url{https://medium.com/@jpaulreed/the-737max-and-why-software-engineers-should-pay-attention-a041290994bd}
%    																																													\bibitem{{news17032019softwareexplains} ... \LaTeX:\\ \url{https://news.ycombinator.com/item?id=19414775}
%    																																														\bibitem{{legget21122020eu737maxsafe} ... \LaTeX:\\ \url{https://www.bbc.com/news/55366320}
%    																																															\bibitem{{marketscreener0103221737chinarecertification} ... \LaTeX:\\ \url{https://www.marketscreener.com/news/latest/China-studies-Boeing-737-MAX-recertification-wants-safety-concerns-fully-addressed--32569125/}
%    																																																motor in brand
%    																																																\bibitem{{euractiv22022021737firegrounds} ... \LaTeX:\\ \url{https://www.euractiv.com/section/aviation/news/boeing-grounds-777s-after-engine-fire/}
%    																																																	\bibitem{{benny18022019737returnUAE} ... \LaTeX:\\ \url{https://gulfnews.com/business/aviation/uae-airspace-to-see-return-of-boeing-737-max-1.1613627548923}
%    																																																		motor in brand gevlogen
%    																																																		\bibitem{{biersmichel22022021777grounds} ... \LaTeX:\\ \url{https://techxplore.com/news/2021-02-boeing-urges-grounding-777s.html}
%    																																																			\bibitem{{ID} ... \LaTeX:\\ \url{https://www.politico.eu/article/uk-temporarily-bans-some-boeing-aircraft-after-pratt-whitney-engine-incidents/}
%    																																																				\bibitem{{reuters23022021777metalfatigue} ... \LaTeX:\\ \url{https://www.timeslive.co.za/news/world/2021-02-23-damage-to-united-boeing-777-engine-consistent-with-metal-fatigue--ntsb/}
%    																																																					faa was niet kritisch genoeg
%    																																																					\bibitem{{ID} ... \LaTeX:\\ \url{https://federalnewsnetwork.com/government-news/2021/02/federal-watchdog-blasts-faa-over-certification-of-boeing-jet/}
%    																																																						\cite{}
    																																																						
  
\end{thebibliography}

%
%\newpage
%\bibliography{references}
%

\vfill
    %(verplicht) bronvermeldingen
 
  \addcontentsline{toc}{chapter}{\bibname}
  \hoofdstuk{Persoonlijk verslag}

In de evaluatie reflecteer je over je eigen afstudeerproces. Daarbij
moet je vooral letten op de leereffecten. Welke competenties had je
nodig? Welke competenties kwam je tekort en moest je zelf verwerven?
Waren dit algemene of specifieke competenties?  Voldeden de
beroepscompetenties aan de standaard van het \emph{HBO-I} (analyseren,
adviseren, ontwerpen, realiseren en beheren)?  Vielen de algemene
competenties in de vijf categorieën van de \emph{Dublin
Descriptoren}\footnote{Dublin Descriptoren zijn eisen aan de
competenties voor de bachelor en master studies aan universiteiten en
hogescholen in Europa.} zoals het verkrijgen van kennis en inzicht,
het toepassen van kennis en inzicht, het maken van onderbouwde keuzen
(oordeelsvorming), het communiceren (schriftelijk en mondeling) en het
verkrijgen van leervaardigheden?

  %(verplicht) reflectie op het leerproces
  \appendix
  \bijlage{Achtergrond materiaal}

In de bijlagen komen alle gegevens die nodig zijn voor de
onderbouwing, maar die de leesbaarheid van het hoofdverslag verlagen.


\sffamily
\begin{tabularx}{\textwidth}{@{}Sl|X|Sl @{}}
	\mytoprule
	\makecell[lc]{B. Buiiea GmbH \& Co. KG \\ Konstruktion und\\ Entwicklung}
	& Datum der Erstellung: 01.01.17 \par\mbox{}\par Erstellt von: Max Mustermann
	& \makecell[lc]{Aktueller Stand: 02.01.17 \\ Index: 00\\ \mbox{}} \\
	\mymidrule
	\multicolumn{3}{@{}c@{}}{Anforderungsliste} \\
	\addlinespace
	\multicolumn{3}{@{} >{\centering}m{\textwidth}@{}}{Bla Bla Bla Bla Bla} \\
	\midrule
	\multicolumn{3}{@{}c@{}}{Projekt-Nr.: 1234567890} \\
	\multicolumn{3}{@{}c@{}}{Projektname}
\end{tabularx}
\begin{tabularx}{\textwidth}{Sc| Sc |X| X| c | c | >{\RaggedRight\bigstrut}m{\lastcolwd}}
	\specialrule{\lightrulewidth}{-4ex}{0pt}
	\multicolumn{6}{@{}c|@{}}{Anforderungen} & \makecell[lt]{F = Fest \\W = Wunsch}\\
	\specialrule{2pt}{0pt}{0pt}
	\rowcolor{Gainsboro}\makecell[c]{F \\ W} & Nr. & Bezeichnung &
	\bigstrut Werte\par\ Daten \par Anforderungen & Zust. & Status & Bermerkungen \\
	\mybottomrule
	\endfirsthead
	\specialrule{2pt}{0pt}{0pt}
	\rowcolor{Gainsboro}\makecell[c]{F \\ W} & Nr. & Bezeichnung &
	\bigstrut Werte\par\ Daten \par Anforderungen & Zust. & Status & Bermerkungen \\
	\mybottomrule
	\endhead
	\multicolumn{1}{c}{} & \multicolumn{1}{Sc}{1} & \multicolumn{5}{l}{\bfseries Funktionen} \\
	\hline
	F & 1.1 & Hier steht ein Text. Hier steht ein Text. \par Hier steht ein Text. Hier steht ein Text. & Hier steht ein Text. Hier steht ein Text. \par Hier steht ein Text. Hier steht ein Text. & xy & & Hier steht ein Text. Hier steht ein Text. \par Hier steht ein Text. Hier steht ein Text. \\
	\hline
	\multicolumn{1}{c}{} & \multicolumn{1}{Sc}{1} & \multicolumn{5}{l}{\bfseries Funktionen} \\
	\hline
	F & 1.1 & Hier steht ein Text. Hier steht ein Text. \par Hier steht ein Text. Hier steht ein Text. & Hier steht ein Text. Hier steht ein Text. \par Hier steht ein Text. Hier steht ein Text. & xy & & Hier steht ein Text. Hier steht ein Text. \par Hier steht ein Text. Hier steht ein Text. \\
	\hline
	\multicolumn{1}{c}{} & \multicolumn{1}{Sc}{1} & \multicolumn{5}{l}{\bfseries Funktionen} \\
	\hline
	F & 1.1 & Hier steht ein Text. Hier steht ein Text. \par Hier steht ein Text. Hier steht ein Text. & Hier steht ein Text. Hier steht ein Text. \par Hier steht ein Text. Hier steht ein Text. & xy & & Hier steht ein Text. Hier steht ein Text. \par Hier steht ein Text. Hier steht ein Text. \\
	\hline
	\multicolumn{1}{c}{} & \multicolumn{1}{Sc}{1} & \multicolumn{5}{l}{\bfseries Funktionen} \\
	\hline
	F & 1.1 & Hier steht ein Text. Hier steht ein Text. \par Hier steht ein Text. Hier steht ein Text. & Hier steht ein Text. Hier steht ein Text. \par Hier steht ein Text. Hier steht ein Text. & xy & & Hier steht ein Text. Hier steht ein Text. \par Hier steht ein Text. Hier steht ein Text. \\
	\hline \noalign{\penalty-5000}
	\multicolumn{1}{c}{} & \multicolumn{1}{Sc}{1} & \multicolumn{5}{l}{\bfseries Funktionen ! ! ! } \\*
	\hline
	F & 1.1 & Hier steht ein Text. Hier steht ein Text. \par Hier steht ein Text. Hier steht ein Text. & Hier steht ein Text. Hier steht ein Text. \par Hier steht ein Text. Hier steht ein Text. & xy & & Hier steht ein Text. Hier steht ein Text. \par Hier steht ein Text. Hier steht ein Text. \\
	\hline
	\multicolumn{1}{c}{} & \multicolumn{1}{Sc}{1} & \multicolumn{5}{l}{\bfseries Funktionen} \\
	\hline
	F & 1.1 & Hier steht ein Text. Hier steht ein Text. \par Hier steht ein Text. Hier steht ein Text. & Hier steht ein Text. Hier steht ein Text. \par Hier steht ein Text. Hier steht ein Text. & xy & & Hier steht ein Text. Hier steht ein Text. \par Hier steht ein Text. Hier steht ein Text. \\
	\hline
\end{tabularx}


%\bijlage{Onderzoeksgegevens}

%\bijlage{UML diagrammen}

%\bijlage{Gebruikshandleiding}



   %(optioneel) bijlagen
  
\newpage
\hoofdstuk{Requirement tracability matrix}
\renewcommand*\theadfont{\bfseries}
\settowidth\rotheadsize{\theadfont Infrastructure}
\renewcommand\theadgape{}
\renewcommand\theadalign{lc}
\renewcommand\rotheadgape{}
\begin{table}
	\centering
	\caption{Caption}
	\label{tab:table1}
	\begin{tabular}{lcccc}
		\toprule
		\thead{Requirements} & \rothead{Accuracy} & \rothead{Coverage} & \rothead{Scalability} & \rothead{Infrastructure}  \\
		\midrule
		Inertial Navigation & $\checkmark$ & $\checkmark$  & $\checkmark$ & \\
		RFID & $\checkmark$ & $\checkmark$  & $\checkmark$ & \\
		Bluetooth & & $\checkmark$  & & $\checkmark$\\
		WLAN & $\checkmark$ & $\checkmark$  & & $\checkmark$ \\
		Infrared & $\checkmark$ & $\checkmark$  & & $\checkmark$\\
		\bottomrule
	\end{tabular}
\end{table}       %(optioneel) bijlagen
  
\newpage
\hoofdstuk{swot analyse}

\begin{tcbitemize}[raster columns=3, raster rows=3, enhanced, sharp corners, raster equal height=rows, raster force size=false, raster column skip=0pt, raster row skip = 0pt]
	
	%Empty corner and two headers
	\tcbitem[blankest, width=1cm]
	\tcbitem[header = helpful]
	\texta
	\tcbitem[header = harmful]
	\textb
	
	%First row
	\tcbitem[firstcol = internal]
	\textcn
	\tcbitem[swotbox = S]
	\lipsum[2]
	\tcbitem[swotbox = W]
	\lipsum[2]
	
	%Second row
	\tcbitem[firstcol = external]
	\textcn
	\tcbitem[swotbox=O]
	\lipsum[2]
	\tcbitem[swotbox=T]
	\lipsum[2]
\end{tcbitemize}

\newpage
\begin{tcbitemize}[raster columns=3, raster rows=4, enhanced, sharp corners, raster equal height=rows, raster force size=false, raster column skip=0pt, raster row skip = 0pt]
	
	%Empty corner and two headers
	\tcbitem[blankest, width=1cm]
	\tcbitem[header = Strength]
	\texta
	\tcbitem[header = Weakness]
	\textb
	
	%First row
	\tcbitem[firstcol = internal]
	\textcn
	\tcbitem[swotbox = S]
	\lipsum[2]
	\tcbitem[swotbox = W]
	\lipsum[2]
	
	\tcbitem[blankest, width=1cm]
	\tcbitem[header = Opportunity]
	\texta
	\tcbitem[header = threat]
	\textb
	
	%Second row
	\tcbitem[firstcol = external]
	\textcn
	\tcbitem[swotbox=O]
	\lipsum[2]
	\tcbitem[swotbox=T]
	\lipsum[2]
\end{tcbitemize}   %(optioneel) bijlagen
  	\section{Research case: De digitale aanval op de Oekrainese krachtcentrale}
Dit verslag geeft inzage in een analyse van de Ukraine cyber aanval,
inclusief hoe de actoren zich zelf toegang gavan tot het controle systeem, welke methoden de acoren hebben gebruikt voor reconnaissance en vastleggen van het systeem, een gedetailleerde omshrijving van de aanval op 15 December 2015, en de methoden die gebruikt zijn door de aanvallers om hun sporen uit te wissen en daarmee het het stoppen van schade toebrengen  nog moeilker maken. Daarnaast wordter  een gedetailleerde omschrijving gevevenv an de beveiliging van de SCADA ccontrol systemen gebaeerd op bst practices, inclusief het control network ontwerp, technieken voor whtelisting, monitoring en loggen, en  opleiding van personeel.

https://na.eventscloud.com/file_uploads/aed4bc20e84d2839b83c18bcba7e2876_Owens1.pdf
\cite{Whitehead2017ukrainepoweroutage}
https://www.wired.com/2016/03/inside-cunning-unprecedented-hack-ukraines-power-grid/
\cite{zetter2016GridHack}
https://www.boozallen.com/content/dam/boozallen/documents/2016/09/ukraine-report-when-the-lights-went-out.pdf
\cite{boozallen2016lightwentout}
https://www.reuters.com/article/us-ukraine-cybersecurity-sandworm-idUSKBN0UM00N20160108
\cite{finklejan2016UsBlamesRussianSandworm}
https://www.mandiant.com/resources/blog/ukraine-and-sandworm-team
\cite{}
https://www.ifri.org/sites/default/files/atoms/files/desarnaud_cyber_attacks_energy_infrastructures_2017_2.pdf
\cite{desarnaud2017cyberattacks}
https://ris.utwente.nl/ws/files/6028066/3-s2_0-B9780128015957000227.pdf
\cite{caseli04112016intrusiondetectioncontrolsystem}
https://repositorio-aberto.up.pt/bitstream/10216/119066/2/315683.pdf
\cite{rochascadatesting}
https://www.diva-portal.org/smash/get/diva2:1046339/FULLTEXT01.pdf
\cite{hidajat2016ScadaSimulator}
https://www.vice.com/en/article/zmeyg8/ukraine-power-grid-malware-crashoverride-industroyer
\cite{zetter2017moreDangerousMalware}


Oop 23,december 2015  vind er een cyber aanval plaats op het elektriciteitsnet van de Oekraine. Dit was de eerste bekende aanval op een elektrisch controle  system met corrupte firmware. Daarnaas wordt er een telecom-based denial of service attack met  geautomatieerde systemen om het telefoonverkeer uit te schakelen.
\cite{Whitehead2017ukrainepoweroutage}

Uit onderzoek\cite{zetter2016GridHack} naar de aanval,  uitgevoerd door Oekraiene sen Amerikaanse militairenblijkt  bleek onder meer dat de power grids in sommige gevallen beter waren beveiligd dan de Amerikaanse. Desondanks was de viligheid niet optimaal door onder andere de  hetgegeven dat werknemers op afstand konden inloggen en geen gebruik van 2-stapsverificatie.


\subsection{Literaire analyse}

\subsubsection{Motief}
Oekraine wijst naar de russen \cite{zetter2016GridHack}

https://www.wired.com/story/russian-hackers-attack-ukraine/
\cite{greenberg2017Cyberwartestlab}
https://www.boozallen.com/content/dam/boozallen/documents/2016/09/ukraine-report-when-the-lights-went-out.pdf
\cite{boozallen2016lightwentout}
https://www.reuters.com/article/us-ukraine-cybersecurity-sandworm/u-s-firm-blames-russian-sandworm-hackers-for-ukraine-outage-idUSKBN0UM00N20160108
\cite{finkle08012016russiansandwormhackers}
https://www.reuters.com/article/us-ukraine-crisis-cyber-idUSKBN15U2CN
\cite{zinets15022017ukrainechargesrussia}
https://theconversation.com/cyberattack-on-ukraine-grid-heres-how-it-worked-and-perhaps-why-it-was-done-52802
\cite{mcelfresh2016cyberattackhowandwhy}
https://jsis.washington.edu/news/cyberattack-critical-infrastructure-russia-ukrainian-power-grid-attacks/
\cite{parkwalstorm11102017russiagridattack}
\subsubsection{Situatie Oekraiene}
https://www.dragos.com/wp-content/uploads/CrashOverride-01.pdf
\cite{drago2017CrashOverride}
https://www.dragos.com/wp-content/uploads/CRASHOVERRIDE.pdf
\cite{slowik2019ReassasUkraine2016Attack}
\subsubsection{Situatie algemeen}
https://www.politico.eu/article/ukraine-cyber-war-frontline-russia-malware-attacks/
\cite{cerulus2019FrontlineRussiaAttack}
https://www.ifri.org/sites/default/files/atoms/files/desarnaud_cyber_attacks_energy_infrastructures_2017_2.pdf
\cite{desarnaud2017cyberattacks}
https://www.cybersecurityintelligence.com/blog/attack-on-ukraines-power-grid-targeted-transmission-stations-4530.html
\cite{dragos2019TargetedTransStation}
\subsubsection{Factoren}
http://web.mit.edu/smadnick/www/wp/2016-22.pdf
\cite{shehod2016gridadvantageus}
\subsubsection{Oorzaak}
https://www.sans.org/blog/confirmation-of-a-coordinated-attack-on-the-ukrainian-power-grid/
\cite{rocha2017cybersecyrityanalysisScada}
https://arstechnica.com/information-technology/2017/06/crash-override-malware-may-sabotage-electric-grids-but-its-no-stuxnet/
\cite{2017crashoverridenostuxnet}
https://www.darkreading.com/threat-intelligence/first-malware-designed-solely-for-electric-grids-caused-2016-ukraine-outage
\cite{vijayan2017firstmalwareCausedOutage}
https://www.dragos.com/wp-content/uploads/CRASHOVERRIDE.pdf
\cite{slowik2019ReassasUkraine2016Attack}
\subsubsection{Gebruikte materialen}
https://en.wikipedia.org/wiki/2015_Ukraine_power_grid_hack
\cite{2015ukrainegridattack}
https://www.cisa.gov/news-events/alerts/2017/06/12/crashoverride-malware
\cite{}
https://rhebo.com/en/service/glossar/industroyer-25114/
\cite{industroyershortfact}

\subsubsection{Uitvoering van de aanval}
https://na.eventscloud.com/file_uploads/aed4bc20e84d2839b83c18bcba7e2876_Owens1.pdf
\cite{Whitehead2017ukrainepoweroutage}
https://www.boozallen.com/content/dam/boozallen/documents/2016/09/ukraine-report-when-the-lights-went-out.pdf
\cite{boozallen2016lightwentout}
\subsubsection{Oplossingen}
https://na.eventscloud.com/file_uploads/aed4bc20e84d2839b83c18bcba7e2876_Owens1.pdf
\cite{Whitehead2017ukrainepoweroutage}
https://www.cisa.gov/news-events/ics-alerts/ir-alert-h-16-056-01
\cite{}
\subsubsection{Aanbevelingen}

\subsection{Resultaten}
\subsubsection{De aanval}
1. An initial email spear phishing attack lures recipients
into opening an attached Microsoft® document with a
macro that installs Black Energy 3 (BE3) onto
corporate workstations.
2. BE3 and other tools perform reconnaissance and
enumeration of the network and provide an initial
backdoor for the hackers into the corporate network.
3. As a result of network reconnaissance, the malicious
actors discover and access the oblenergos’ Microsoft
Active Directory® servers that contain corporate user
accounts and credentials.
4. With the harvested credentials, the malicious actors use
an encrypted tunnel from an external network to get
inside the oblenergo network, establishing a presence
on the oblenergo control system networks.
5. Malicious actors discover and access the control center
supervisory control and data acquisition (SCADA)
human-machine interface (HMI) servers and
substations. While a router separates corporate and
SCADA networks, the firewall rules are improperly
configured.
6. On December 23, 2015, at 3:30 p.m., the malicious
actors begin their power outage attacks by entering
operations and SCADA networks through backdoors on
the compromised SCADA workstations. The malicious
actors take control away from HMI operators and then
open breakers.
7. The malicious actors perform several other actions with
the intent of complicating the responses of control
operators and increasing the effort required to return the
system to normal operating conditions. These actions
include:
a. Launching a coordinated Telephony Denial of
Service (TDoS) attack that floods call centers to
prevent legitimate calls from getting through.
b. Disabling the UPSs for the control centers.
c. Corrupting the firmware on a remote terminal unit
(RTU) HMI module and serial-to-Ethernet port
servers.
8. Malicious actors execute KillDisk malware in an
attempt to wipe out the control center HMIs and pivotpoint workstations.
https://na.eventscloud.com/file_uploads/aed4bc20e84d2839b83c18bcba7e2876_Owens1.pdf
\cite{Whitehead2017ukrainepoweroutage}
https://www.boozallen.com/content/dam/boozallen/documents/2016/09/ukraine-report-when-the-lights-went-out.pdf
\cite{boozallen2016lightwentout}
\subsubsection{spearfishing}
\subsubsection{blackenergy}
\subsubsection{remote access capabilities}
\subsubsection{serial-to-ethernet communication devices}
\subsubsection{telephony denial of service attacks}

\subsection{oplossingen}
Identificeer alle risicos en schrijf een plan foor het managen van de risico's.
Implementeer  effecteve controle  om het riico te managen.
Creeer een diepgaand model dat ervoor zor dat er efectieve en efficiente security controls worden uitgevoerd.
Aangaande de gebeurtenissen in de oekraiene kunnen de volgende security controls worden opgenomen in het securitymodel: Initial access to enterprise network, pivot in interprise network, elevate priviliges, maintainance access, gain access to control system, attack, attack complication, destroy hard drives.
\cite{Whitehead2017ukrainepoweroutage}

\subsection{Discussie}

\subsection{Verder lezen}
https://citeseerx.ist.psu.edu/viewdoc/download;jsessionid=0513EED48102FDAD1BD940260EF12B11?doi=10.1.1.548.7490&amp;rep=rep1&amp;type=pdf
\cite{}
https://scialert.net/fulltext/?doi=tasr.2014.396.405
\cite{Shahzad2014ScadaProtocolsPollingScenario}
https://www.researchgate.net/publication/333671061_Attacking_IEC-60870-5-104_SCADA_Systems
\cite{grammatikis2019AttackIEC6087505104}
https://www.welivesecurity.com/wp-content/uploads/2017/06/Win32_Industroyer.pdf
\cite{2017win32industroyer}
https://blog.nettedautomation.com/2017/
\cite{}
https://arxiv.org/pdf/2001.02925.pdf
\cite{yadav2020reviewScadaArchitecture}
https://dl.acm.org/doi/fullHtml/10.1145/3381038
\cite{arrizabalaga2020surveyiiotProtocols}
https://www.win.tue.nl/~setalle/2017_fauri_encryption.pdf
\cite{fauri2017EncryptionICS}
http://www.connectivity4ir.co.uk/article/175490/IEC-62351--Secure-communication-in-the-energy-industry.aspx
\cite{resch31102019IEC62351secureCommunication}
https://www.virsec.com/resources/blog/virsec-hack-analysis-deep-dive-into-industroyer-aka-crash-override
\cite{}
https://dreamlab.net/en/blog/post/fuzzing-ics-protocols/
\cite{levalle2020FuzzingICSProtocols}
https://www.blackhat.com/docs/us-17/wednesday/us-17-Staggs-Adventures-In-Attacking-Wind-Farm-Control-Networks.pdf
\cite{blackhatusa2017}
https://blog.checkpoint.com/research/crashoverride/
\cite{}
https://www.blackhat.com/us-17/briefings/schedule/#industroyercrashoverride-zero-things-cool-about-a-threat-group-targeting-the-power-grid-6159
\cite{blackhatusa2017}
https://search.abb.com/library/Download.aspx?DocumentID=9AKK107045A1003&amp;LanguageCode=en&amp;DocumentPartId=&amp;Action=Launch
\cite{abb30062017crashoverridenotification}
https://iiot-world.com/ics-security/cybersecurity/five-cybersecurity-experts-about-crashoverride-malware-main-dangers-and-lessons-for-iiot/
\cite{spinner2018crashoverrideiot}
https://www.csoonline.com/article/3200828/crash-override-malware-that-took-down-a-power-grid-may-have-been-a-test-run.html
\cite{}
https://www.paloaltonetworks.com/blog/2017/06/crashoverrideindustroyer-protections-palo-alto-networks-customers/
\cite{}
https://www.webopedia.com/definitions/crashoverride-industroyer-malware/
\cite{}
https://www.cyber.nj.gov/threat-center/threat-profiles/ics-malware-variants/crashoverride
\cite{njccicthreat08102017crashovverrideprofile}
https://www.nixu.com/blog/crashoverride-threat-electricity-networks
\cite{}
https://www.virusbulletin.com/virusbulletin/2019/03/vb2018-paper-anatomy-attack-detecting-and-defeating-crashoverride/
\cite{slowikvb2018crashoverride}
https://en.wikipedia.org/wiki/Crash_Override_Network
\cite{crashoverridenetwork}
https://en.wikipedia.org/wiki/Industroyer
\cite{wikiindustroyer}
https://www.dragos.com/resource/crashoverride-analyzing-the-malware-that-attacks-power-grids/
\cite{}
https://www.wallix.com/blog/ics-security-russian-hacking
\cite{icsSecurityRussianHacking}
https://www.nixu.com/fi/node/53
\cite{holappa2017threattoElectricityNetworks}
https://control.com/forums/threads/comparison-between-iec60870-5-103-and-modbus-rtu.20317/
\cite{}   %(optioneel) bijlagen
 %  


\subsection{Formele logica}

0004 \\
0031 \\
Modelcheckig boek blz 14 \\
E = {main={}, \\
	deur={}, \\
	stoplicht={}, \\
	sensor={}, \\
	pomp={}, \\
	wachtrij={}, \\
	queue={}, \\
	sluiskolk={} \\
} \\
Q0 = \\
F C Q = sluiskolk_afgesloten \\
Q = E0,...En \\
0064 \\
0077 \\



X: volgende keer \\
F: in de toekomst \\
G: altijd geldig \\
A: voor alle paden \\
E: voor enkele paden \\
U: waar zolang de volgende inclusief geldt \\
R: waar zolang geldt inclusief de startpositie \\

M, s-> f betekent f is geldig in state s in kripke structuur M \\
M, ph -> f is geldig op een pad in kripke structuur M \\
M. s -> FP <-> er bestat een fair pad dat begint bij s en p € L(s) \\
M, s->E(g) er bestaat een fair pad M dat begint van s zodanig dat phi -> F(g) \\
M, s-> F A(g) voor alle fair pads phi beginnend vanad s,  phi -> F(g) \\



Sluis.Draining-->Deuren.laag_open \\
Deuren.laag_open-->Stoplicht.Green \\
E<> (Ship.ship_can_move&&Stoplicht.Green) \\
A[] not (Stoplicht.Green && not \\ (Deuren.hoog_open||Deuren.laag_open||Deuren.stopgaplow1||Deuren.stopgaplow2||Deuren.stopgaphigh1||Deuren.stopgaphigh2)) \\
A[] not \\ ((Deuren.hoog_open||Deuren.laag_open||Deuren.Opening_laag||Deuren.Opening_hoog||Deuren.Closing_hoog||Deuren.Closing_laag) && (Sluis.Draining||Sluis.Filling||Sluis.draining2||Sluis.Filling2)) \\
Sensor.Wait-->Sensor.Wait \\
Stoplicht.Green-->Stoplicht.Green \\
(Deuren.hoog_open||Deuren.laag_open)-->(Deuren.laag_open||Deuren.hoog_open) \\
Deuren.laag_open-->Deuren.Closed \\
Deuren.hoog_open-->Deuren.Closed \\
Deuren.Closed-->Stoplicht.Red \\
Ship.ship_can_move-->Deuren.Closed \\
Deuren.hoog_open-->Stoplicht.Green \\
Ship.ship_can_move-->Stoplicht.Green \\
A[] not (Deuren.laag_open && Deuren.hoog_open) \\
Ship.ship_can_move-->Ship.ship_can_move \\
A[] not (Deuren.laag_open && Sluis.water != Sluis.water_laag) \\
A[] not (Deuren.hoog_open && Sluis.water != Sluis.water_hoog) \\
A[]not deadlock \\




A[] forall (i:id_t) forall (j:id_t) P(i).cs && P(j).cs imply i == j \\
Mutex requirement. \\
P(1).req --> P(1).wait \\
Whenever P(1) requests access to the critical section it will eventually enter the wait state. \\
A[] (sum(i:pid_t) P(i).cs) <= 1 \\
Mutex requirement. \\
A[] forall(i : pid_t) not Task(i).Error \\
Check that the system is schedulable. \\
E<> Gate.Occ \\
Gate can receive (and store in queue) msg's from approaching trains. \\
Train 0 can reach crossing. \\
E<> Train(1).Cross \\
E<> Train(0).Cross and Train(1).Stop \\
E<> Train(0).Cross and (forall (i : id_t) i != 0 imply Train(i).Stop) \\
A[] forall (i : id_t) forall (j : id_t) Train(i).Cross && Train(j).Cross imply i == j
There is never more than one train crossing the bridge (at any time instance). \\
A[] Gate.list[N] == 0 \\
There can never be N elements in the queue (thus the array will not overflow). \\
Train(0).Appr --> Train(0).Cross \\
Train(1).Appr --> Train(1).Cross \\
Train(2).Appr --> Train(2).Cross \\
Train(3).Appr --> Train(3).Cross \\
Train(4).Appr --> Train(4).Cross \\
Train(5).Appr --> Train(5).Cross \\
E<> Gate.Occ \\
Gate can receive (and store in queue) msg's from approaching trains. \\
E<> Train1.Cross \\
Train 1 can reach crossing. \\
E<> Train2.Cross \\
Train 2 can reach crossing. \\
E<> Train1.Cross and Train2.Stop \\
Train 1 can be crossing bridge while Train 2 is waiting to cross. \\
E<> Train1.Cross and Train2.Stop and Train3.Stop and Train4.Stop \\
Train 1 can cross bridge while Train 2, 3 & 4 are waiting to cross. \\
A[] Train1.Cross + Train2.Cross + Train3.Cross + Train4.Cross <= 1 \\
There is never more than one train crossing the bridge (at any time instance). \\
A[] Queue.list[N-1] == 0 \\
There can never be N elements in the queue (thus the array will not overflow) \\
Train1.Appr --> Train1.Cross \\
Whenever a train approaches the bridge, it will eventually cross. \\
Train2.Appr --> Train2.Cross \\
Train3.Appr --> Train3.Cross \\
Train4.Appr --> Train4.Cross \\




%

\cheading{Fake Course Evaluation Summary for \textsc{course
		1234y}}{Sept.\ 2010 --- May 2011}

\begin{longtable}{@{}l rr rr rr rr rr rr}
	% pairs: absolute number (percentage)
	
	\toprule%
	\centering%
	& \multicolumn{2}{c}{{{\bfseries Excellent}}}
	& \multicolumn{2}{c}{{{\bfseries Very Good}}}
	& \multicolumn{2}{c}{{{\bfseries Good}}}
	& \multicolumn{2}{c}{{{\bfseries Average}}}
	& \multicolumn{2}{c}{{{\bfseries Poor}}}
	& \multicolumn{2}{c}{{{\bfseries Very Poor}}} \\
	
	
	\cmidrule[0.4pt](r{0.125em}){1-1}%
	\cmidrule[0.4pt](lr{0.125em}){2-3}%
	\cmidrule[0.4pt](lr{0.125em}){4-5}%
	\cmidrule[0.4pt](lr{0.125em}){6-7}%
	\cmidrule[0.4pt](lr{0.125em}){8-9}%
	\cmidrule[0.4pt](lr{0.125em}){10-11}%
	\cmidrule[0.4pt](l{0.25em}){12-13}%
	% \midrule
	\endhead
	
	
	Some question about the Instructor or Course & 2 & (7.14) & 4 &
	(14.29) & \highest{12} & \highest{(42.86)} & 4
	& (14.29) & 6 & (21.43) & 0 & (0.00) \\
	
	\myrowcolour%
	Some question about the Instructor or Course & 3 & (10.71) &
	\highest{15} & \highest{(53.57)} & 5 & (17.86) & 5 & (17.86) & 0 &
	(0.00) & 0 & (0.00) \\
	
	Some question about the Instructor or Course & 4 & (14.29) & 8 &
	(28.57) & \highest{15}
	& \highest{(53.57)} & 1 & (3.57) & 0 & (0.00) & 0 & (0.00) \\
	
	\myrowcolour%
	Some question about the Instructor or Course & 3 & (10.71) & 8 &
	(28.57) & \highest{10} & \highest{(35.71)}
	& 5 & (17.86) & 2 & (7.14) & 0 & (0.00) \\
	
	Some question about the Instructor or Course & 6 & (21.43) &
	\highest{9} & \highest{(32.14)}
	& 4 & (14.29) & \highest{9} & \highest{(32.14)} & 0 & (0.00) & 0 & (0.00) \\
	
	\myrowcolour%
	Some question about the Instructor or Course & \highest{10} &
	\highest{(35.71)} & \highest{10} & \highest{(35.71)}
	& 3 & (10.71) & 5 & (17.86) & 0 & (0.00) & 0 & (0.00) \\
	
	Some question about the Instructor or Course & \highest{12} &
	\highest{(42.86)} & \highest{12} & \highest{(42.86)} & 3
	& (10.71) & 1 & (3.57) & 0 & (0.00) & 0 & (0.00) \\
	
	\myrowcolour%
	Some question about the Instructor or Course & \highest{12} &
	\highest{(42.86)} & 3 & (10.71) & 7
	& (25.00) & 5 & (17.86) & 1 & (3.57) & 0 & (0.00) \\
	
	Some question about the Instructor or Course & \highest{10} &
	\highest{(35.71)} & 6 & (21.43) & 6 & (21.43) & 6 & (21.43)
	& 1 & (3.57) & 0 & (0.00) \\
	
	\myrowcolour%
	Some question about the Instructor or Course & 5 & (17.86) & 5 &
	(17.86) & \highest{12} & \highest{(42.86)} & 2 & (7.14)
	& 3 & (10.71) & 1 & (3.57)\\
	
	Some question about the Instructor or Course & 3 & (10.71) & 8 &
	(28.57) & \highest{11} & \highest{(39.29)} & 3 & (10.71) & 3 & (10.71)
	& 0 & (0.00) \\
	
	\myrowcolour%
	Some question about the Instructor or Course & \highest{18} &
	\highest{(64.29)}
	& 5 & (17.86) & 3 & (10.71) & 1 & (3.57) & 1 & (3.57) & 0 & (0.00) \\
	
	Some question about the Instructor or Course & \highest{15} &
	\highest{(53.57)}
	& 7 & (25.00) & 2 & (7.14) & 2 & (7.14) & 2 & (7.14) & 0 & (0.00) \\
	
	\myrowcolour%
	Some question about the Instructor or Course & 3 & (10.71) &
	\highest{13} & \highest{(46.43)} & 4 & (14.29) & 6 & (21.43) & 2
	& (7.14) & 0 & (0.00) \\
	
	\bottomrule
	
\end{longtable}













\square ( a_0 \implies (( \lnot a_2 \wedge \lnot a_3 ) \mathcal{U} a_1 ) \vee ( \lnot a_2 \wedge \lnot a_3 ))


$\xymatrix@1{
	A\times B\times C\times D \ar[r]^-{+} &B
}$


\usepackage{logicproof}


\begin{logicproof}{4}
	\forall x \, (P(x) \to Q(x)) & premise \\
	\forall x \, P(x) & premise \\\hspace*{-30pt}
	\begin{subproof}
		\llap{$x_0\quad$} P(x_0) \to Q(x_0) & $\forall x \, \mathrm{e}$ 1 \\
		P(x_0) & $\forall x \, \mathrm{e}$ 2 \\
		Q(x_0) & $\to \mathrm{e}$ 3, 4
	\end{subproof}
	\forall x \, Q(x) & $\forall x \, \mathrm{i}$ 3--5
\end{logicproof}



\begin{align*}
	&p                                                      \\
	&p\pdfliteral{-10 -5 m 0.5 0 1 RG 0.5 w 25 -5 l S }\to q\\
	\therefore\quad &q
\end{align*}


\{a,b\} or \set†{a,b} \\
\langle a,b \rangle or \gens†{a,b} \\


f \colon A \to B \\
f \colon A \into B \\
f \colon A \onto B \\
f \colon A \isom B \\
f \circ g \\
x \mapsto f(x) \\

\begin{align*}
	f \colon \mathbb{R} &\to \mathbb{R} \\
	x &\mapsto x^2
\end{align*}


\mathrm{A} \alpha \\
\mathrm{B} \beta \\
\Gamma \gamma \\
\Delta \delta \\
\mathrm{E} \epsilon \varepsilon \\
\mathrm{Z} \zeta \\
\mathrm{H} \eta \\
\Theta \theta \vartheta \\
\mathrm{I} \iota \\
\mathrm{K} \kappa \\
\Lambda \lambda \\
\mathrm{M} \mu \\
\mathrm{N} \nu \\
\Xi \xi \\
\mathrm{O} \mathrm{o} \\
\Pi \pi \varpi \\
\mathrm{P} \rho \varrho \\
\Sigma \sigma \varsigma \\
\mathrm{T} \tau \\
\Upsilon \upsilon \\
\Phi \phi \varphi \\
\mathrm{X} \chi \\
\Psi \psi \\
\Omega \omega \\



X \implies Y  \\
X \impliedby Y \\
X \iff Y \\
\neg X \\
\sim X \\
X \land Y \\
X \lor Y \\
\forall a \in A \\
\exists b \in B \\
\exists !b \in B \\


A = \{ [elements] : [conditions]\} \\

\mathbb{N} = \{ a \in \mathbb{Z} : a > 0 \} \\

a \in A \\
A \subseteq B \\
A \subset B  \\
A \supseteq B \\
A \supset B  \\
A = B  \\
A \cong B \\
A \cup B \\
A \cap B \\
A-B \\
|A| \\
\{\} = \varnothing \\


\begin{align*}
	f \colon \mathbb{R} &\to \mathbb{R} \\
	x &\mapsto x^2
\end{align*}


\usepackage{tikz}
\usetikzlibrary{cd}
~ 
\usepackage{mathtools,halloweenmath}

\paragraph{Model checking Temporal logics} \\
M, s \models p $\Leftrightarrow$ p \in L(s) \\
M, s \models \not f1 $\Leftrightarrow$ M, s \nvdash f1 \\
M, s \models f1 \vee f2 $\Leftrightarrow$ M,s \models f1 or M,s \nvdash f2 \\
M, s \models f1 \wedge f2 $\Leftrightarrow$  M,s \models f1 and M,s \nvdash f2 \\
M, s \models \mathrm{E} g_{1} $\Leftrightarrow$ there is a path \pi  from ~  s ~   such ~  that  ~ M, \pi \models g1 \\
M, s \models p $\Leftrightarrow$ for every path \pi  ~ starting from s, M, \pi \models g1 \\
M, s \models p $\Leftrightarrow$ s is the first state of \piand M, s \models f1 \\
M, s \models \not g_{1} $\Leftrightarrow$ M, \pi  \nvdash g1\\
M, s \models p $\Leftrightarrow$  M, \pi  \models g1  or  M, \pi  M, \pi  \models g2\\
M, s \models p $\Leftrightarrow$ M, \pi  \models g1  and  M, \pi  M, \pi  \models g2 \\
M, s \models p $\Leftrightarrow$ M, $\pi^{1}$ \models g1 \\
M, s \models p $\Leftrightarrow$ there exists a k \ge 0, such that M, $\pi^{k}$  \models g1\\
M, s \models p $\Leftrightarrow$ for all i \ge 0,M,$\pi^{i}$ \models g1 \\
M, s \models g1 \bugcup g2 $\Leftrightarrow$ there exists ak \ge 0 such that M, $\pi^{k}$ \models g2\\
and for all 0 \le j < k, M,$\pi^{j}$ \models g1
M, s \models p $\Leftrightarrow$ for all j \ge 0, if for every i < j,M,$\pi^{i}$ \nvdash g1 then M,$\pi^{j}$ \models g2\\

\leg \\
\geq \\
\Re \\
\partial \\
\bigup \\
\bigcap \\


\paragraph{Model checking Fairness constraints}
\paragraph{Definition 4.1}

\ell ~ \boldsymbol \ell  g \\

\mathbb{A}  is a tuple = \{ L ,  \ell_{0} ,X ,  Inv , \mathrm{T} , \Sigma \} ~ where: \\
L ~  is ~   a  ~  finite ~   set  ~  of  ~  control ~   states, ~   also ~   called ~   locations, \\
\ell_{0}    \in L  ~  is  ~  the ~   initial  ~  location,  ~  \\
X   ~ is  ~  a ~   finite  ~  set  ~  of  ~  clocks \\
T ~   \subseteq ~   L ~  x ~ C (X) ~  x  ~ \sigma  ~ x  ~ $2^{X}$ ~ x ~ L  ~ is  ~ a  ~ finite ~  set  ~ of ~  tranasitions: ~   e= \langle \ell ,g,a,r,l' \rangle  \in T ~  represents ~   a   ~ trasition  ~  from ~   \ell ~  to  ~  \ell'. ~  g  ~  is  ~  the   ~ guard   ~ of ~   e, ~   r  ~  is  ~  the  ~  set  ~  of  ~  clocks  ~  that  is ~   reset  ~  by ~    ~  e, \\ ~   and   ~   a ~   is \\  ~  the  ~  action  ~  of  ~  e.  ~  We ~   also ~   write  ~  \ell    $$\underrightarrow{g,a,r}  $$ \\
Inv : L \to C (X) \\
\Sigma  ~ is  ~ an  ~ alphabet  ~  of  ~  actions
\paragraph{Definition 4.2}
A timed transition (TTS)is a tuple S =(S,s_{0},$$\underrightarrow{} $$, \Sigma) where S is a (possibly infinite) set of states, s_{0} \in S is the initial state and $$\underrightarrow{} $$   \subseteq S x (\Sigma \cup R) x S is the transition relation. Morever, the relation -> satisfies the three following conditions:
(1)  if s $$\underrightarrow{0} $$ s', then s = s',  (2) is s$$\underrightarrow{d} $$ s' and s' $$\underrightarrow{d'} $$ s"  with d, d' \in R, then s $$\underrightarrow{d+d'} $$ s", and (3) if s $$\underrightarrow{d} $$ ' with d \in R, then for all 0 << d' << d, there exists s" \in S such that s $$\underrightarrow{d'} $$ s"and s" $$\underrightarrow{d-d'} $$ s'
\paragraph{Definition 4.3}
Let \mathrm{A} = (L,\ell,X,Inv, T,\Sigma) be a timed automation. The semantics of A is defined asthe TTS S_{A}=(S,s_{0},$$\underrightarrow{} $$, \Sigma) where:\\
S = L x $R^{X}$  \\
s_{0} = (\ell_{0}, v_{0}) with v_{0}(x)=0 for every x \in X, \\
the transition relation  ,$$\underrightarrow{} $$ is composed of: \\
action transitions (\ell, v)$$\underrightarrow{a} $$ (\ell',v') if and only ifthere exists \ell $$\underrightarrow{g,a,r} $$ \ell' \in  T such that v \models g,v'=[r \leftarrow 0] v and v' \models Inv(\ell').


delay transitions: if d \in R, (\ell,v) ,$$\underrightarrow{d} $$ (\ell,v+d) if and only if v+d \models Inv(\ell $(^{1}$
\url{http://www.iste.co.uk/data/doc_wrkszvritcbv.pdf}


\paragraph{Definition 1}
Let A be a setof actions. A timedtransistion system TTS is a structure L=(S,A,R,s_{0},$\subseteq$, U) where \\
S is a set of states, with the initial state s_{0} \in S; \\
A is a setof labels;\\
$$\underrightarrow{}  $$ \subseteq S x (A x $R^{ \geq 0}$) x S is the transition location; and \\
U \subseteq  $R^{ \geq 0}$ x S is the until predicate \\



\paragraph{Definition 2}
Let L_{i} = (S_{i},A x  $\Re^{\geq0}$,$s^{\i0}$_{0},,$$\underrightarrow{} $$ i, $U^{i}$   ), i \in {1,2}, be two TTS. \\
A timedbisimulation is a relation R $\subseteq$ S1 x S2 with $s^{1}$_{0}R$s^{2}$_{0} satisfying, for all a(d) \in A x $\Re^{\geq0}$, the following transfer properties: \\
if  s_{1}R_{2} and  s_{1}$$\underrightarrow{a(d)} $$_{1} s_{1}', then \exists $s^{'}$_{2} \in S_{2} : s2  $$\underrightarrow{a(d)}  $$ 2$s^{'}$_{2} and $s^{'}$_{1}R$s^{'}$_{2} \\

if  s_{1}R_{2} and s_{1}$$\underrightarrow{a(d)} $$_{2} s_{1}', then \exists $s^{'}$_{1} \in S_{1} : s1  $$\underrightarrow{a(d)}  $$ $s^{'}$_{1} and $s^{'}$_{1}R$s^{'}$_{2}

if   s_{1}R_{2} s_{1}', then  $U^{1}$_{d}(s1)   $\Leftrightarrow$ $U^{2}$_{d}(s2) 

\paragraph{Definition 3}
A timed safety automation is a structure (S,A,C,s_{0},$\subseteq$, $\vartheta$ ,\kappa)
S is a setof states, with the initial state s_{0} \in S; \\
A is a set of actions; \\
C is a set of clocks \\
$$\underrightarrow{}  $$  $\subseteq$ S x A x \phi(C) x S is the set of edges; \\
$\vartheta$ : S $\subseteq$ \Phi(C)is the invariantassignment function
\kappa : S $\subseteq$ \varphi_{fin}(C) is the clocks resetting function
\url{file:///C:/Users/gally/Downloads/228_DArBr96b.pdf}

\paragraph{Model checking Delay transitions}
De lay transitions correspond to the  elapsing time while staying at some location. We write (s,v)  $$\underrightarrow{d}  $$ (s,v+d), where d \in $R^{ \geq +}$) , provided that for every 0$\le$e$\le$d, the invariant I(s) holds for v+e
\paragraph{Model checking Action transitions}
Action transitions correspondto the execution of a transitionfrom T.We write (s,v) $$\underrightarrow{d}  $$ (s',v') where a \in \Sigma, provided that there  is a transition \langle s,a,\phi,\lambda,s' \rangle such  that v satisfies \varphi and v' = v[ \lambda :=0]


% $\xymatrix@1{
	% 	X\ar[r]^a_b & Y & Z\ar[l]^A_B }$
% 
%$\ell$ \\
\Re \\
\patial \\   %(optioneel) bijlagen
  


\begin{tcbitemize}[raster columns=3, raster rows=4, enhanced, sharp corners, raster equal height=rows, raster force size=false, raster column skip=0pt, raster row skip = 0pt]
	
	%Empty corner and two headers
	\tcbitem[blankest, width=1cm]
	\tcbitem[header = Strength]
	\texta
	\tcbitem[header = Weakness]
	\textb
	
	%First row
	\tcbitem[firstcol = internal]
	\textcn
	\tcbitem[swotbox = S]
	\lipsum[2]
	\tcbitem[swotbox = W]
	\lipsum[2]
	
	\tcbitem[blankest, width=1cm]
	\tcbitem[header = Opportunity]
	\texta
	\tcbitem[header = threat]
	\textb
	
	%Second row
	\tcbitem[firstcol = external]
	\textcn
	\tcbitem[swotbox=O]
	\lipsum[2]
	\tcbitem[swotbox=T]
	\lipsum[2]
\end{tcbitemize}

\newpage
\paragraaf{Werken met \LaTeX{}}

Het is niet verplicht om met \LaTeX{} te werken. Men mag ook gebruik
maken van andere tekstverwerkers zoals \emph{MS-Word}, Wel is het
verplicht het afstudeerverslag \LaTeX{}-geformateerd in te leveren en
van de \LaTeX{}-template \verb!modelverslag.sty! gebruik te
maken.

De \LaTeX{}-template bevat enkele macro's voor het opstellen van een
hoofdstuk (\verb!\hoofdstuk!), een paragraaf (\verb!\paragraaf!), een
afbeelding (\verb!\figuur!). De overige \LaTeX{} macro's en omgevingen
blijven bruikbaar. Bijvoorbeeld de \verb!tabular!-omgeving om tabellen
te maken:

\begin{lstlisting}{language=[LaTeX]TeX}
	\begin{tabular}{formaat}
		... 
	\end{tabular}
\end{lstlisting}

\begin{center}
	\begin{tabular}{|l||r|}
		\hline
		\multicolumn{2}{|c|}{Afmetingen ($1\mbox{ pt}=0,351\mbox{ mm}$)}\\
		\hline
		paper width     & \the\paperwidth\\
		text width      & \the\textwidth\\
		column width    & \the\columnwidth\\
		column seperate & \the\columnsep\\
		oddside margin  & \the\oddsidemargin\\
		evenside margin & \the\evensidemargin\\
		paper height    & \the\paperheight\\
		text height     & \the\textheight\\
		top margin      & \the\topmargin\\
		\hline
	\end{tabular}
\end{center}

Een nadeel van tabellen dat ze vaak te groot zijn voor de
twocolumn-mode. Het zou mooi zijn als ze ingedrukt kunnen
worden. Bovendien is deze tabel niet-zwevend, hij wordt geplaatst
tussen de tekstdelen waar hij is ingevoerd. Dit kan bezwaarlijk zijn
bij pagina-overgangen. In dat geval kan je beter gebruikmaken van
zwevende tabellen (en figuren) die door \LaTeX{} zelf op een geschikte
plaats worden gezet. Wel moet aan een zwevende tabel een label en een
onderschrift gekoppeld worden om er naar te kunnen verwijzen. Voor een
zwevende horizontale tabel met label en onderschrift wordt in de
`template' de \verb!tabel!-omgeving aangeboden:\\

\begin{Aanpassen}
	\begin{verbatim}
		\begin{tabel}[afm]{formaat}{label}{onderschrift}
			...
		\end{tabel}
	\end{verbatim}
\end{Aanpassen}


De \verb!tabel!-omgeving plaatst `zwevende' tabellen in verslag- en
publicatie-mode. Het eerste argument is een optioneel \verb![afm]!
argument met de defaultwaarde \verb!\normalsize! voor de afmeting van
de karakters. De mogelijke waarden voor de afmeting zijn -- van groot
tot klein -- de volgende macro's: (\verb!\huge!, \verb!\LARGE!,
\verb!\Large!, \verb!\large!, \verb!\small!, \verb!\footnotesize!,
\verb!\scriptsize! en \verb!\tiny!).

Bovendien zijn de standaard \verb!tabular! kolomformaten
\verb!r,l,c,|,||,p{lengte}! uit de tabelomgeving uitgebreid met
kolomformaten \verb!\R, \C, \L!  voor variabele celinhoud zoals het
plaatsen van meerdere regels per cel.

Een verticale tabel is mogelijk met de omgeving (\verb!TABEL!)  met
dezelfde kolomformaten mogelijkheden.  In \LaTeX{} zijn de tabellen,
vooral in de \verb!twocolumn!-mode erg lastig. Bijvoorbeeld in de
tabellen~\ref{tab:vbt} en \ref{tab:vbx} zijn twee verschillende
uitwerkingen van de tabelomgevingen:

\begin{footnotesize}
	\begin{tabel}[\Large]{|r|l|}{vbt}{Vaste cellen, variabele breedte}
		\hline
		7C0 & hexadecimal \\
		3700 & octal \\ \cline{2-2}
		11111000000 & binary \\
		\hline \hline
		1984 & decimal \\
		\hline
	\end{tabel}
\end{footnotesize}

%Absolute breedte in mm, cm etc. (geschikt voor één mode):
%\begin{tabel}{|>\R p{2cm}|>\L p{4cm}|}{vbx}{OpenGL libraries}
%Relatieve breedte 20% en 65% van de kolombreedte (geschikt voor alle moden).
%Procentwaarden van 0 ... 99, voor 100% neem \columnwidth zelf.
\begin{tabel}{|>\R p{\Procent{20}}|>\L
		p{\Procent{65}}|}{vbx}{Variabele cellen, variabele breedte}
	\hline
	OpenGL core library & OpenGL32 voor MS-Windows en GL voor
	de meeste X-Window systemen\\
	\hline
	OpenGL Utility Library & GLU\\
	\hline
	Koppeling met het platform & GLX voor X-Window en WGL voor MS-Windows\\
	\hline 
	OpenGL Library Utility Toolkit & GLUT, bibliotheek voor
	het openen van windows, invoer van muis en toetsenbord, menus,
	event-driven in- en uitvoer\\
	\hline
\end{tabel}

Plaats afbeeldingen alleen in het hoofdverslag als ze de tekst
ondersteunen en de leesbaarheid niet verlagen.  In de tekst kan naar
afbeeldingen worden verwezen met de macro \verb!\ref{fig:label}!.

In \LaTeX{}\cite{lam1994} geschreven verslagen zijn op diverse manieren
afbeeldingen\cite{Oos1996} te plaatsen. Een van die manieren is gebruik te
maken van de macro \verb!\figuur! in de \verb!modelverslag!-package'.

`Vector graphics' figuren van het `pdf-', `eps-' en `svg-'
formaat\footnote{Een pdf-bestand kan zowel vector-graphics als
	bitmap-graphics bevatten.} met een ingewikkelde `bounding box' zijn
moeilijk op de juiste schaal te brengen. Vaak moet dat met uitproberen
bepaald worden. Het plaatsen van figuren met absolute afmetingen of
een vaste `scale' factor, kan leiden tot minder soepele oplossingen
zoals figuur~\ref{fig:PDFA}. Deze figuur heeft naast een rotatie
(\verb!angle=270!)  een vaste scale-factor (\verb!scale=0.45!) die
alleen geschikt is voor de `twocolumn-mode'.

\begin{center}
	\figuur{scale=0.45,angle=270}{plaatjes/agp.pdf}{PDFA}{Vaste breedte
		(pdf)}
\end{center}

In plaats van \verb!scale=x! kan je beter de relatieve afmeting
\verb!width=\Procent{y}! gebruiken. De waarde $y$ wordt in de
verslag-mode met uitproberen gevonden, zie figuur~\ref{fig:PDFR}.

\figuur{width=\Procent{40},angle=270}{plaatjes/agp.pdf}{PDFR}{Variabele
	breedte (pdf)}

Het afmetingsprobleem is iets gemakkelijker op te lossen met `bitmap
graphics' van het `jpg-', `gif-' en `png-' formaat omdat de figuren al
van te voren geschaald kunnen worden als de `bounding box' bij het
inlezen bekend is. De breedte (\verb!width!) kan als percentage van de
kolombreedte (\verb!width=\Procent{0 ... 99}!) worden opgegeven zoals
dat bij figuur~\ref{fig:PNGR} gedaan is. Voor een 100\% waarde neemt
men \verb!width=\columnwidth! De afmeting wordt automatisch
aangepast aan de nieuwe kolombreedte.


\begin{center}
	\figuur{width=\columnwidth}{plaatjes/agp.png}{PNGR}{Variabele breedte (png)}
\end{center}


De macro \verb!\PROCENT{0...99}! is nodig voor de macro's \verb!Tabel!
en \verb!Figuur!. Deze laatste twee macro's maken het mogelijk dat
tabellen en afbeeldingen in de twocolumn-mode passen met behoud van
hun originele afmeting en detaillering (zie
figuur \ref{fig:FIXED}). De parameters van deze macro's komen overeen
met de parameters van de macro's \verb!tabel! en \verb!figuur!.

\Figuur{width=\PROCENT{40},angle=270}{plaatjes/agp.pdf}{FIXED}{Vaste breedte ook in twocolumn-mode (pdf)}

In het algemeen heeft vector-graphics een betere kwaliteit van de
weergave dan bitmap-graphics.


\paragraaf{Bijzondere tekens en afbreekproblemen}

Bijzondere tekens zoals de á, à, ä, é, è, ë, ï, ü, ç \ldots worden
probleemloos door \LaTeX{} geaccepteerd als normale utf8
karakters. Voor de uitzonderingen bestaan macro's zoals het
euro-symbool \euro{} waarvoor de macro \verb!\euro! nodig is. In
wiskundige formules kan je gebruik maken van de macro \verb!\eurom!.


In de two-columnmode zijn regels soms te lang als er gebruik gemaakt
is van \verb!verb! of \verb!verbatim! of woorden die niet goed worden
afgebroken. In dat laatste geval kan je in zo'n woord een afbreekpunt
introduceren met de twee tekens \verb!\-!. Een regel kan gecontroleerd
afgebroken door van te voren onzichtbare knikpunten te plaatsen met de
\verb!\Knak! macro. De volgende regel moet in in tegenstelling met de
twocolumnmode in de verslagmode ongeknakt worden weergegeven:

\begin{Aanpassen}
	\begin{verbatim}
		... aaaaaaa\Knak{}aaaaaaa ...
	\end{verbatim}
\end{Aanpassen}


aaaaaaaaaaaaaaaaaaaaaaaaaaaaaaaaaaaaaaa\Knak{}aaaaaaaaaaaaaaaaaaaaaaaaaaaaaaaaaaaaaaa.

Voor regels waarbij de structuur niet gebroken mag worden, is de
\verb!\Knak!-methode ongeschikt, bijvoorbeeld bij scripts en
broncode. Daarentegen zorgt de \verb!Aanpassen!-omgeving ervoor dat in
de twocolumn-mode de regels met behoud van de originele structuur
worden weergegeven. Daarvoor wordt een kleinere letterafmeting
gebruikt (default de \verb!\scriptsize!). Deze omgeving werkt alleen
met niet al te lange regels. Bij zeer lange regels moet de
letterafmeting zeer klein worden waardoor de leesbaarheid in het
gedrang komt. In dat geval moet naar een andere oplossing gezocht
worden zoals het opnemen van de probleemregels (broncode en scripts)
in de bijlagen.

\begin{Aanpassen}[\tiny]
	aaaaaaaaaaaaaaaaaaaaaaaaaaaaaaaaaaaaaaaaaaaaaaaaaaaaaaaaaaaaaaaaaaaaaaaaaaaaaa.
\end{Aanpassen}

Hoewel het gebruik van opsommingen (\verb!\item!), letterlijke citaten
\verb!quotation! en kaders (\verb!\fbox!) in de twocolumn-mode tot
problemen kunnen leiden, zijn ze beperkt toegestaan. Bijvoorbeeld voor
de kaders rond de teksten kan je beter gebruik maken van de
\verb!tabular!-omgeving (of de \verb!tabel!-omgeving als je geen last
wil hebben van pagina-overgangen), dan voor de standaard
\verb!\fbox!-methode. De kolom van deze omkaderde tabel moeten dan wel
een relatieve afmetingsverhouding de \verb!\columnwidth! krijgen.

\begin{Aanpassen}
	\begin{verbatim}
		\begin{center}
			\begin{tabular}{|>\C p{\Procent{80}}|}
				\hline
				Afbreekproblemen ...
				\hline
			\end{tabular}
		\end{center}
	\end{verbatim}
\end{Aanpassen}

\begin{center}
	\begin{tabular}{|>\C p{\Procent{80}}|}
		\hline
		~\\
		Afbreek- en andere opmaakproblemen pak je als laatste aan,
		dus bij je definitieve verslag!\\
		~\\
		Tabellen, figuren en listingen in het hoofdverslag tot het
		noodzakelijke beperken.\\
		~\\
		\hline
	\end{tabular}
\end{center}


\paragraaf{Algoritmen en broncode\cite{wikibooks}}

Als je algoritmen met een mooie layout wilt hebben, dan zou je het
\verb!algorithmic!-pakket kunnen gebruiken. Met dit pakket kan je het
algoritme op een logische manier opbouwen met pseudotaal. Het bestand
`verslag.tex' bevat al de pakketten \verb!algorithmic! en
\verb!listings! die voor dit verslag nodig zijn. Als je zelf packages
wil toevoegen of verwijderen (afblijven van
\verb!\usepackage{moduleverslag}!)  dan moet dat in de preambule
`verslag.tex'.

\begin{Aanpassen}
	\begin{verbatim}
		\usepackage{algorithmic}
	\end{verbatim}
\end{Aanpassen}

Een algoritme moet je maken binnen een algorithmic-omgeving, een
voorbeeld:

\begin{Aanpassen}[\small]
	\begin{algorithmic}
		\IF {$i\geq maxval$} 
		\STATE $i\gets 0$
		\ELSE
		\IF {$i+k\leq maxval$}
		\STATE $i\gets i+k$
		\ENDIF
		\ENDIF 
	\end{algorithmic}
\end{Aanpassen}


Broncode kan je in een \verb!verbatim!-omgeving opnemen. De
broncoderegels zien er net zo uit zoals je ze ingetypt hebt.  Het
\verb!listings!-pakket is geavanceerder dan de
\verb!verbatim!-omgeving.

\begin{Aanpassen}
	\begin{verbatim}
		\usepackage{listings}
	\end{verbatim}
\end{Aanpassen}

Merk even op dat alle commando's van het \verb!listings!-pakket
beginnen met \verb!lst!, dit conform de lppl-licentie.

De broncode zelf zet je in een \verb!listings!-omgeving, net zoals bij
de \verb!verbatim!-omgeving, om broncode te zetten gebruik je het
\verb!\lstinline!-commando op dezelfde manier als het
\verb!\verb!-commando. Je kunt ook broncode van een extern document laden met het commando:

\begin{Aanpassen}
	\begin{verbatim}
		\lstinputlisting{pathname}
	\end{verbatim}
\end{Aanpassen}

Het argument `pathname' is de relatieve of absolute locatie van het
bronbestand, de map(pen) gecombineerd met de bestandsnaam. Als je
broncode van een bronbestand laadt, ben je zeker dat de broncode in je
\LaTeX{}-document altijd actueel is en hou je het \LaTeX{}-document
overzichtelijk. Als de broncode niet in dezelfde map of een submap van
het \LaTeX{}-document staat of je gebruikt absolute `pathnames', dan
is het mogelijk dat het verslag niet op andere computers gecompileerd
kan worden. Bij het inleveren van je afstudeerverslag in
\LaTeX{}-formaat zal je hiermee rekening moeten houden.


Alle opties in het \verb!listings!-pakket hebben eenzelfde structuur
\verb!sleutel=waarde!. Als je alleen 'Java' gebruikt hebt, dan kan je
deze taal voor je volledig document na de regel
\verb!\usepackage{listings}! in preambule `verslag.tex' definiëren met
\verb!\lstset{language=java}!

\lstset{language=java}

\begin{Aanpassen}
	\begin{lstlisting}
		public class HelloWorld {
			public static void main(String[] args) {
				System.out.println("Hello, world!");
			}
		}
	\end{lstlisting}
\end{Aanpassen}


De sleutel is hier dus \verb!language! en de waarde die je aan de
sleutel geeft is \verb!java!. Alles wat je als opties binnen de
\verb!\lstset!-macro zet kan je per \verb!listings!-omgeving apart
definiëren. Bijvoorbeeld html-broncode met
\verb!\begin{lstlisting}[language=html]!:
	
	\begin{Aanpassen}
		\begin{lstlisting}[language=html]
			<html>
			<head>
			<title>Hello</title>
			</head>
			<body>Hello</body>
			</html>
		\end{lstlisting}
	\end{Aanpassen}
	
	

  \section{Overige onderzoeksresultaten}


\subsubsection{explosie in libabon, beirut }
\begin{description}
	\item[Beschrijving]
	\item[Datum en plaats] 
	\item[Oorzaak]
	%Beschrijf wat er mis ging in termen van het vier variabelen model/requirements/specificaties
\end{description}
Op 23 september 2013 voer het vrachtschip de Rhosus onder Moldavische vlag[7] van Batoemi in Georgië naar Beira in Mozambique met 2.750 ton ammoniumnitraat

Gezien het ernstige gevaar van het bewaren van deze goederen in de hangar onder ongeschikte klimatologische omstandigheden, herhalen we ons verzoek aan de marine-instantie om deze goederen onmiddellijk weer te exporteren om de veiligheid van de haven en de mensen die er werken te verzekeren, of om akkoord te gaan om ze te verkopen.
Voorafgaand aan de explosie was er een brand in een opslagplaats. 


\cite{hrw03082021investigateBeirutBlast}

\cite{souaibyElHussein112020Beirutstory}

\cite{ifrc2020chemicalexplosionBeirutPort}




\subsubsection{stint ongeluk}

\begin{description}
	\item[Beschrijving]
	\item[Datum en plaats] 
	\item[Oorzaak]
	%Beschrijf wat er mis ging in termen van het vier variabelen model/requirements/specificaties
\end{description}
Vier kinderen, een bestuurder kwamen om en een vijfde persoon , een kind raakte zwaargewond. Uit odnerzoek van bleek :
Foute torsieveer voor de gashendel werd geleverd
Geen van de drie onderzochte voertuigen haalden de wettelijk vereiste remvertraging
De automatische parkeerrem kan leiden tot gevaarlijke situaties wanneer deze ongewenst geactiveerd wordt tijdens het rijden. 
Het losraken van de nuldraad naar de gashendel leidt volgens TNO tot ongewenst versnellen van het voertuig en een oncontroleerbare situatie voor de bestuurder.
Voor alle drie onderzochte voertuigen geldt dat het ontbreken van een zitplaats leidt tot veiligheidsrisico’s voor remmen en sturen door de grotere kans dat de bestuurder van het voertuig valt. Als de bestuurder van een Stint valt, leidt dit in alle rijsituaties tot een onbeheersbare situatie


\cite{TNOStint}




\subsubsection{vuurwerkramp in enschede }

\cite{boogers092002RampenRegelsRichtlijnen}

Wat waren de afspraken omtrent vuurwerkopslag?
Waarom werden de voorschriften neit nageleefd?




\subsubsection{ecourt in nederlandse rechtspraak}

\begin{description}
	\item[Beschrijving]
	\item[Datum en plaats] 
	\item[Oorzaak]
	%Beschrijf wat er mis ging in termen van het vier variabelen model/requirements/specificaties
\end{description}
niet odnerzocht
https://www.njb.nl/blogs/a-court-with-no-face-and-no-place/ 
\cite{sprongken19032018CourtProcedureDigital}
http://www.e-court.nl/wp-content/uploads/2018/03/Procesreglement-e-Court-2017_20180201.pdf
\cite{PROCESREGLEMENTEcourt}




\paragraph{molukse treinkaping }

\begin{description}
	\item[Beschrijving]
	\item[Datum en plaats] 
	\item[Oorzaak]
	%Beschrijf wat er mis ging in termen van het vier variabelen model/requirements/specificaties
\end{description}
https://www.youtube.com/watch?v=h99Fe9XzzHI 
\cite{molukseTreinkaping}


\subsubsection{Ramp schietpartij militair ossendrecht }

\begin{description}
	\item[Beschrijving]
	\item[Datum en plaats] 
	\item[Oorzaak]
	%Beschrijf wat er mis ging in termen van het vier variabelen model/requirements/specificaties
\end{description}
Een militaire overleid op een schietbaan in ossendracht door onvoldoende begeleiding van cursisten, geen toezicht op de lokatie. E\r was een instructuur in opleiding die niet volledig was mmeegenomen in het poroces en ook was er geen baancommandant aanwezig. Geen van de aanwezig instructeurts had de juiste papieren om de cursisten te begeleiden. De aanwezig instruceur had geen zich op de instructeur in opleiding, evenmin de andere militairen. In de instructiehandleiding ontbreken richtlijnen voor bijzondere schietbanen. Ook was er geen keuring. Door personelstekort is er geen andacht besteed aan documentastie(een slyllabus) hoe en met welke risico’s oefeningnen moeten worden ingericht. Ok werd er vooraf geen veiliheidsanaklyse gedaan. Het gebrek aan lesmateriaal en deskundigen is gemeld binnen de defensieorganisatie maar dit heeft niet geleid tot enige verandering in de situatie.
Op een afgekeurde scheitbaan
Tezicht door een instructeur in opleiding die zelf geen persoonlijke begeleiding heeft gehad tijdens de uitvoering
Belangrijk is dat defensie haar taken kan uitvoeren met personeel dat is getraind in situaties die de risicos van de werkomgeving aan de cursisten kunnen laten zien.
Conclusie
Zonder gekwalificeerde instructuers.
Zonder toezicht
Zonder lesmateriaal
Zonder adequate veiligheidsanalyse
https://www.youtube.com/watch?v=6jmkDClGDHo 
\cite{oVVSchietongevalOssendrecht}
\cite{nos22032016ossendrecht}
\cite{ovv04042016lessenongevalossendrecht}
\cite{quekelboere10052017doodossendrecht}


Wat is de rol van defensie?
Wat is er gedaan om de veligheid van de medewerkers te waarborgen?
Waarom zijn deze regels niet nageleefd?
Wat zijn de gevolgen?
Zijn de acties die naderhand zijn ondernomen wel redelijk naar de slachtoffers, het nationale veiligheisbeeld en de medewerkers?

 %   \hoofdstuk{}






main.py [-h] [--version] [-d] [-c FILE] [-g FILE] [-t FILE] [-o FILE]





32,38,40,44,61,75,78,79,81,82,85,86,201,202,213,306,308,325\




Timeliness
https://www.eecs.yorku.ca/course_archive/2014-15/F/1090/slides/04_TheoremCalculation.pdf
https://math.stackexchange.com/questions/2133202/what-does-the-notation-gamma-vdash-phi-mean-in-mathematical-logic


Assume that e is an action of the system, a Timeliness property for e is defined
to be a property related to the time of occurrence of e [10]. G. Blair, J.-B. Stefani: Open Distributed Processing and Multimedia Addison-
Wesley, Boston, MA, 1997.




De verificatie methode van dit artikel werkt niet omdat we theoretisch gezien niet uitgaan van meerdere schepen die tegelijkinvaren.

Theorem 1, the main
result of the paper, proves that a QTA is a Test Automata [2–4, 16]. Section 4
applies our approach to the verification of throughput in an example of a Video
Player system. Section 5 presents a proof for Theorem 1. The final two sections
discuss some related works and draw a conclusion.


https://www.brics.dk/RS/97/29/BRICS-RS-97-29.pdf

[[♦P kj]] = {s ∈ S | ∃s′
	.hs, s′i ∈ R and s′ ∈ [[j]]}
CTL formulas are based on the following operators:
A (\on every path")
E (\there exists a path")
X (\next time")
G (\globally" or \always")
F (\eventually" or \nally")
U (\until")
R (\release")

PRECONDITIONS
Topography:
Geometry description of the environment including maps of the
expected changes, such as land, water, river, sea and title deeds as
well as regional planning and zoning scheme.
Possible existing lock that could remain operational or has to be
renovated:
Geometry and condition;
Current and anticipated use;
Permitted limitations during the construction/ renovation.
Possible other hydraulic structures nearby:
Geometry and condition;
Current and anticipated use;
Permitted limitations during construction/renovation.
Water levels:
Water levels with exceedance and underrun frequency levels
Water level development (tidal curve etc.) within lock operation
reach
Rising and lowering velocities
Historical water level data during dry and wet seasons
Water flow
River discharges/flood control regime
Water quality (chloride content, "aggressiveness")
Water temperature
Swell and wave data
Wind data (speed, direction, frequencies)
Morphological data (such as bed-load and suspended-load
transport) and forecasts
Soil characteristics
Soil mechanical data (results of field and laboratory sampling)
Geo-hydrological data (such as ground water level rise as a function
of time, groundwater flow, results of pump test in case of groundwater
lowering)
Soil pollution where excavation takes place for lock and lock approach
as well as in the general vicinity in case of possible groundwater
lowering by pumping
FUNCTIONAL REQUIREMENTS
Functional requirements regarding navigation
General:
Design, build and manage the lock complex so that vessels of a
given waterway classification index can pass rapidly and safely:
• Normative vessel (length, width, depth, height)
• Normative combination of shipping vessels
• Normative traffic and fleet composition, taking the spread of
arrival times into account
Lock approaches:
Per lock (chamber) and per side (above and below) a lining up area
is required that is situated as such that moored vessels do not form
an obstacle to departing vessels, while the moored vessels are able
to sail into the lock via leading jetty rapidly.
The size of the lining up area is geared to a complete chamber fill
(for existing locks with small amounts of traffic this is unnecessary).
The width is equal to that of the chamber.
Waiting areas are necessary if it is expected that, on busy days,
lining up areas will not be sufficient. A waiting area is created per
side of the lock complex (a common area if there are several chambers).
At very least, a normative vessel should be able to moor here.
With a view to the stopping and mooring, the free area is given a
length of at least 2.5 times the normative vessel length (inland
navigation) depending on the adjoining waterway.
Is the lock approach also used as stay over harbour, refuge harbour
or compulsory harbour?
Leading jetties:
Approach wall lengths and shapes are a function of navigation (sea,
inland, recreational navigation). In combined use by various categories,
the shape that belongs with the largest vessels is normative.
The successive wet cross sections in the change over from leading
jetties to chamber entrance should be (hydraulically) symmetrical
wherever possible.
Chamber and heads:
The main dimensions are derived from the requirement to deal with
traffic rapidly and safely (item 8) the max. and min. locking levels
(item 20), the flood control requirements (item 12), as well as constructive
integration of these elements (par. 4.6).
In newly built locks, the chamber and the heads are given the same
working width.
If the lock mainly functions as an open lock, higher navigation
speeds on passing through the lock should be taken into account.
Functional requirements regarding the water retaining structure:
Overflow and overtopping:
For determining the height of the gate plating and the capstone,
the following is taken into consideration:
NHW (Normative High Water)
Rise in sea level
Settlement and settings
Rise in the water level due to local wind action
Rise in the water level due to seiches and weather caused oscillations
Retaining height
Available water storage
Strength and stability:
Must comply with the stated standards and guidelines in par. 2.3.2
under point 2.
Reliability of closing gates:
If both the gates in both the heads are sufficiently flood retaining
(item 12): no requirements.
If only a sufficiently high door in the outer head, then the reliability
of closing the gate needs to meet the requirement stated in par.
2.3.2 under 3.
Functional requirements regarding water management:
Possible requirements regarding lock and/or leakage loss
Possible requirements regarding the separation of salt water and
fresh water
Water discharge or water intake through or along the lock? If affirmative,
take into account flow patterns unfavourable to navigation
and permitted current velocities as sketched in par. 2.3.3.4.
Functional requirements regarding the crossing of dry
infrastructure:
Roads:
During construction: required (temporary) adjustments to possible
pre-existing facilities.
Expected road facilities in use phase
Cross sections (profiles of free space) for 18a and b.
shipping clearance.
Is periodical, temporarily stopping road and navigation traffic
acceptable?
Free view requirements
Cables and mains:
During construction: required (temporary) adjustments to possible
pre-existing cables and mains.
Use phase: under lock body or lock approaches or via bridging?
With corresponding requirements.
Combined lead-through of cables and mains for lock operation with
lead-through for a third party?
Requirements with regard to mutual influences (distances) cables
and requirements regarding risks related to mains of the locking
operation.
Visual inspection necessary/possible.
USE REQUIREMENTS
Levels:
Lock levels:
Maximum lock level.
Minimum lock level.
High and low normative water levels in aid of speed / safety of
dealing with traffic (10% exceedance and underrun respectively)
on the river side of the inland navigation lock.
High and low normative summer water levels (May/Sept.) in view
of accessibility (2% exceedance and underrun respectively) on canal
c.q. tidal side of a lock with recreational navigation (possibly in
combination with inland navigation).
Design levels (water retaining structure)
NHW (Normative High Water)
Lock level flood gate
Lock level open lock
Possible preference for separating different types of vessels
Possible separation in use of lining up and waiting areas and lock
chamber
For safety reasons, it is recommended that vessels are separated
according to category (sea, inland and recreational navigation)
when mooring in the lining up or waiting areas, as well as during
chamber arrangement and/or chamber assignment.
In view of safety, is it necessary/desirable to create separate lining
up and waiting areas for inland and recreational navigation?
From a safety point of view, separate chambers of sea, inland and
recreational navigation are preferable.
If 22c is economically unacceptable, then combined locks,
in which combined sea and recreational lock filling must be avoided.
For a combination of inland and recreational navigation, consider:
• a wide chamber with both kinds on one side;
• a long chamber, in which both kinds are placed behind each
other with a safety margin (min. 5 m) in between (inland navigation
in front).
Are separate waiting places and chamber arrangements (with
mutual safety distances) required for vessels with hazardous goods?
Are stopping off areas necessary for semi, continuous and day navigation
and/or continuously available mooring facilities for semicontinuous
navigation (inland navigation)?
The shape of the leading jetties at the lock entrance as function of
the type of navigation (sea, inland, recreational navigation)
Mooring facilities in the chamber and lock approaches
Chamber:
Required pattern for placing bollards, bollard recesses, toggles
and mooring pipes as function of the vessel type (sea, inland and
recreational navigation).
Choice between fixed and floating bollards as function of vessel
size, gravity flow and rising velocity during levelling. Required pattern
of positioning in case of floating bollards
Magnitude of force that mooring facilities (bollards etc.) have to be
dimensioned to as a function of vessel size.
Lock approaches:
Mooring facilities could consist of mooring posts, mooring piers,
constructions with wales (fixed or floating guiding structures)
and quay or sheet pile constructions, provided with mooring facilities
(bollards, bollard recesses).
Required distances between mooring posts and mooring piers.
Required wale height with regard to normative high and low water
levels.
Choice between fixed and floating guiding structures as function of
water level variations.
Required pattern for positioning the mooring facilities.
Magnitude of force that mooring facilities have to be dimensioned
to.
Magnitude of the mooring force of vessels that mooring facilities
have to be dimensioned to.
Leading jetty:
Installing a limited number of bollards, bollard recesses for construction
vessels.
Magnitude of sailing up / mooring force of the vessels that have to
be taken up in the leading jetty construction.
Operating times (= opening times):
Desired operational times (hours/day, distinguishing from Monday,
Tuesday up to Friday, Saturday and Sunday) for inland navigation
as function of passing load capacity and CEMT classification.
Desired operational times for shipping.
Desired operational times for recreational navigation.
Levelling times:
Intended levelling times as function of the kind of lock (sea, inland,
recreational navigation), gravity flow, horizontal dimensions and
type of filling (gate opening, culverts).
Operational management
Process descriptions:
Analysis of operational management for the benefit of drafting process
descriptions (normal lockage, obstructions, flood retaining
structure, taking in/discharging salt water/fresh water.
Information for operational management:
Finding the necessary information for operating and managing,
such as navigation volume and water levels, as well as the approximate
necessary facilities for this.
Required facilities and procedures for desired operating situations:
Which installation (parts) require emergency power supply and
which parts require a no-break supply?
At gravity flow larger than 1 m, the slides of the intake and discharge
system must be able to close rapidly (without creating undesired
translatory surges) if a vessel is in danger of getting tied up in
the hawsers.
Is a construction for collision protection of the gates necessary?
Locks with a stringent draught limitation can be fitted with acoustic
draught metre in the bed of the lock approach and at sufficient distance
from the gates.
Are measures required to cope with ice problems?
Operating:
Situating the operating building
Situate the central lock operating building as such that optimal view
of the lock and the lock approach is obtained. If possible, position
operating area on bridge where view of approaching traffic is combined
with view of the lock.Preferably situate operating area on
bridge, on the side of the chamber and opposite the fulcrum.
Remove "blind spots" with cameras.
Local operational facilities:
Consider operating per head in locks for recreation, as well as – but
only for maintenance and calamity situations – commercial navigation
locks.
Means of communication:
At every lock: marine telephone for communication between operators
and vessels.
Central Operation: usually emergency telephones at lining up and
waiting areas and a talk-back system, possibly a public-address
system.
Recommended: acoustic signal at start of levelling.
Choice (partly) automated and self-service:
It is recommended that (parts of) the operating process is automated
in view of operational cost and the speed/safety of dealing with
traffic.
Remote control locks:
Not very usual, with the exception of recreational locks.
Illumination, signalling and boarding:
Required level of illumination in indicated places of the lock complex
yet to be specified, taking into account any possibly
misleading illumination in the surrounding area, avoid dazzling,
the desired evenness of illumination and the colour of illumination
for the recognition of boarding and signalling.
Indicate which surfaces/areas need to be marked. White is a good
colour for showing contrast at a low level of illumination, for
example vertical surfaces of guide structures for guiding navigation.
Signalling according to BPR and RPR (Dutch traffic regulations for
inland waters).
Boarding according to BPR and RPR.
Power supply:
Possibilities offered by the public electricity network for accessing
power during the construction and the utilization. If network capacity
is insufficient, adjust or – for example during construction,
place generator sets.
Emergency power supply units and no-break installation.
Availability:
Analysing the causes of non-availability and indicating required
boundaries in the design (in percentages of the time) in so far as
these are economically sound and the causes can be influenced.
The causes could be:
Water levels above maximum and below minimum locking level.
Too much wind: under which conditions is it still safe to lock?
Malfunctions in installations, operating mechanisms and operating.
Non-availability limits should be provided in the design of these
parts.
Collisions (at best, a forecast of non-availability due to this is
possible). Measures to limit collisions could be:
• good shaping of leading jetties ;
• no parts of the opened moveable bridge protruding over lock
chamber
• possible collision protection constructions for gates;
• limit duration of obstruction, by having reserve parts and reserve
gates.
Maintenance (at best, a forecast of non-availability due to this is
possible)
Protecting constructions against damage:
Gates can be equipped with wood fenders in places where they can
be hit by vessels.
Consider whether anti-collision structures are worthwhile and
economically sound (possibly in large high-lift locks).
Provide concrete surfaces that could be hit by vessels with expansion
joints and endings, bevelled edges, steel corner protection,
capstone profiles etc.
In locks for large vessels, apply drifting frames (or fenders).
For sheet pile constructions the flat mooring (sailing in) area should
be approached by positioning wooden or synthetic posts and regulators.
To prevent vandalism, prevent access to vital parts of the lock complex
by placing fences etc.
To prevent indefinable process management due to lightning strike
or electromagnetic interference, electrical installations should be
designed according to safety regulations standards stipulated in art.
2.4.11.4.
Safety:
Install ladders in the chamber and lock approaches to rescue
people.
Take measures with regard to the safety of the personnel in accordance
with the Health and Safety Regulations (railings, steps and
landings, escape routes, sufficient ventilation, First Aid equipment
etc.).
Install measures for fire-fighting in accordance with the regulations
of the Ministry of Waterways and Public Works and in consultation
with the fire brigade. Provide additional facilities for vessels with
hazardous goods.
Accessibility of lock and lock approaches:
Road connections between public roads, possible wharf, reserve
gate storage and essential parts of the lock are needed. Where
necessary, execute metalling/asphalting of roads to make them suitable
for heavy transport and mobile hoisting devices.
For the accessibility of vessels in the lock and the lock approaches,
install ladders and footbridges.For fire fighting and assistance, follow
the procedures of the authorities concerned.
Additional client wishes:
These wishes have to be known in the early stages of drafting the
Program Requirements. (It could be about a preference for a certain
kind of gate, operating mechanism or switchgear).
Mean life requirements:
Design mean life of lock complex:
For the construction of new locks the mean life is, as a rule 100
years, and for renovation 50 years. Distinction is made between:
• non-replaceable parts, such as lock body, fixed bridges,piping
and outflanking screens with a required mean life of 100 respectively
50 years.
• Well maintained / replaceable parts such as gates, moveable
bridges, operating mechanisms, electrical installations, and guiding
structures, of which the mean life is determined by the basic
cost of the investments plus the nett cash value of maintenance
and replacement during the 100 respectively 50 years.
Mean life of specific parts:
Electrical installations generally have a mean life of 25 years,
given reputable design criteria related to specialized maintenance.
Installation parts that are not installed in a protective or conditioned
environment in accordance with their design, have a life span of
about 10 years.
Hardware and software have a mean life of about 5 to 10 years.
At the end of mean life, sheet pile constructions and its
anchoring – taking corrosive loss into account – should have sufficient
material present to meet the necessary strength and stiffness
requirements for moment of resistance and moment of inertia.
These elements, from which guiding structures are composed,
do not necessarily have the same technical life span. The elements
that are easy to replace could easily have a shorter mean life.
MAINTENANCE REQUIREMENTS
Maintenance
The maintenance strategy should be based on the requirements
related to safety of the retaining structure, the availability to the
lock company as well as the mean life.
In principle, there should be a reserve gate for every gate.
Reserve gates are stacked horizontally or vertically in a gate storage
where, as a rule, maintenance (on an exchanged gate) takes place.
Lift gates can generally be maintained when hoisted, provided that
navigation allows for this. On important navigation routes, gate
docks incorporated in the heads (for maintenance) could also be
used as storage space. It is recommended that a reserve gate is
kept as complete as possible when stored.
Locks should have sufficient spare parts and materials on site.
Decisions must be made – for the benefit of inspection and maintenance
of broken parts of the gates - on whether the heads should
lay open or whether pivot inspection chambers or other local dewatering
methods will be used
Parts that require inspection and maintenance must be made as
accessible as possible, for instance with the aid of stairs, climbing
support or footbridges. High control portals could be provided with
lifts.
Consider monitoring the parameters that describe the condition of
construction parts and/or loads that work on this and/or the degree
of damage.
For electrical installations, hardware and software:
• Materials and components should be set up conditioned and
accessible;
• Hardware and software must be modular for optimizing corrective
maintenance;
• Equip computer installations with control mechanisms for timely
recognition and tracing of malfunctions and deviant process
behaviour.
Depending on the scheduled maintenance, set up storage areas and
workshops at or near the lock complex (or in combination with
other locks nearby).
ENVIRONMENTAL REQUIREMENTS IN USE PHASE
Aesthetics:
In view of design, colour balancing and blending in with the environment,
always involve an architect and sometimes a landscape
architect early on in the process.
Lift gates, vertical storage of reserve gates and high, fixed bridges
could be less acceptable (horizon pollution).
During renovations, it could be desirable to blend in the parts that
come into view with the historical environment. For example, finish
the chamber and heads with bricks and install wooden gates.
Environmental requirements with regard to building materials:
Par. 2.6.2 contains a summary of guidelines in relation avoiding the
application of certain materials.
Recreation:
Consider whether parts of the lock complex should be made accessible
to the public for recreational purposes, providing that it does
not pose any safety hazards (for public and navigation) or a disruption
for the lock authority.
ENVIRONMENTAL REQUIREMENTS IN CONSTRUCTION
PHASE
Available construction site and final grounds:
The sites must be available on time. Construction requires more
surface than the space required in the use phase, certainly if excavation
is executed on inclines. This could be a reason to choose for
different construction methods, for instance a building excavation
(between sheet piling). Limited surface could be a reason to abandon
horizontal roller-bearing gates.
The construction site has to be accessible on time (links to the
public road network and possibly a wharf) and connected to public
power supply (if not possible on time, generators should be considered).
Using the public road for work traffic could be subject to
certain requirements.
Polluted soil:
Legislation on soil protection applies (Act at Abandoned Waste
Sites). The presence of pollutants and the degree in which it is
found largely determines the soil balance (recycling it in the work or
other projects, transporting it to specially designed depots) and
with that, the costs involved. The costs could be a reason not to
choose for construction methods that require a lot of excavation
and earth moving. Toxic waste dumps could result in restrictions on
draining, even at large distances.
Withdrawal of groundwater:
Whether the withdrawal of water is not permitted, permitted to a
certain degree or allowed is a large factor in determining the construction
method and with that, the costs involved. Return pumping
could be a solution, but this also requires a permit from the
provincial authorities.
Maintenance/upkeep of road and navigation traffic, cables and
mains:
The requirements, set by the authorities, to temporary adjustments
and detours of existing infrastructure during construction have to
be known.
Maintenance of flood control structure:
All interventions and modifications to existing flood control structures
require approval form dike authorities. Par. 2.7.5 provides the
specifications in the TAW Guideline on Flood Control Structures and
Special Constructions (TAW-Leidraad Waterkerende Kunstwerken
en Bijzondere Constructies) in relation to the execution of activities
in or near flood control structures during the open and closed
season (resp. 15 April - 15 October and 15 October - 15 April)


M; s j= AG(p) () 8 2 (M; s)  8i  M; [i ] j= p



Er zijn verschillende manieen om requirements te verzamelen en documenteren

scannen64-75

challenges in requirements engineering
https://www.researchgate.net/publication/2462377_Challenges_in_Requirements_Engineering
\bibitem{ } ... \LaTeX:\\ \url{ }
why goals-oriented for requirements engineering
https://www.researchgate.net/publication/249901480_Goal-Oriented_Requirements_Engineering_An_Overview_of_the_Current_Research
\bibitem{ } ... \LaTeX:\\ \url{ }
design and build of collaborative information agents
https://www.researchgate.net/publication/221622575_Design_of_Collaborative_Information_Agents
\bibitem{ } ... \LaTeX:\\ \url{ }
treating nfiras first gradefor its testability
\bibitem{ } ... \LaTeX:\\ \url{ }
software requirements negotiation a theory ui based spiral approach
https://www.cs.rug.nl/search/uploads/Teaching/RE2009Fall/paper/1995_Boehm_ICSE_Software%20Requirements%20Negotiation%20and%20Renegotiation%20Aids%20A%20Theory-W%20Based%20Spiral%20Approach.pdf
\bibitem{ } ... \LaTeX:\\ \url{ }
the worlds a stage: a survey on requirementsengineering using a real life case study
https://www.researchgate.net/publication/2548016_The_world's_a_stage_a_survey_on_requirements_engineering_using_a_real-life_case_study_Karin_Koogan_Breitman_Julio_Cesar_S_do_Prado_Leite
\bibitem{ } ... \LaTeX:\\ \url{ }
from inconsistencyhandling to non-conanical requirements management: a logical perspective
https://www.researchgate.net/publication/257272175_From_inconsistency_handling_to_non-canonical_requirements_management_A_logical_perspective
\bibitem{ } ... \LaTeX:\\ \url{ }
managing inconsistent specification: reasoning, analysis, action
https://www.researchgate.net/publication/2635497_Managing_Inconsistent_Specifications_Reasoning_Analysis_and_Action 
\bibitem{ } ... \LaTeX:\\ \url{ }
representingand using nonfunctional requirements: a process-oriented approach
https://www.researchgate.net/publication/3187474_Representing_and_Using_Non-Functional_Requirements_A_Process-Oriented_Approach
\bibitem{ } ... \LaTeX:\\ \url{ }
Four dark corners of requirements engineering
http://www.cse.msu.edu/~chengb/RE-491/Papers/dark-corners-re-zave-jackson.pdf 
\bibitem{ } ... \LaTeX:\\ \url{ }
classification of research methods in requirements engineering
https://www.researchgate.net/publication/220565934_Classification_of_Research_Efforts_in_Requirements_Engineering
\bibitem{ } ... \LaTeX:\\ \url{ }
agent-basedtactocs for goal-oriented requirements elaboration
https://www.researchgate.net/publication/3952082_Agent-based_tactics_for_goal-oriented_requirements_elaboration
\bibitem{ } ... \LaTeX:\\ \url{ }
challenges in requirements engineering
\bibitem{ } ... \LaTeX:\\ \url{ }
%%%%%%%%%%%%%%%%%%%%%%%%%%%%%%%%%%%%%%%%%%%%%%%%%%%%%%%%%%%%%%%%%
why goals-oriented for requirements engineering
\bibitem{ } ... \LaTeX:\\ \url{ }
scann 0087
%%%%%%%%%%%%%%%%%%%%%%%%%%%%%%%%%%%%%%%%%%%%%%%%%%%%%%%%%%%%%%%%%
design and build ofcollaborative information agents
\bibitem{ } ... \LaTeX:\\ \url{ }
%%%%%%%%%%%%%%%%%%%%%%%%%%%%%%%%%%%%%%%%%%%%%%%%%%%%%%%%%%%%%%%%%
treating nfiras first gradefor its testability
\bibitem{ } ... \LaTeX:\\ \url{ }
scan 0089
%%%%%%%%%%%%%%%%%%%%%%%%%%%%%%%%%%%%%%%%%%%%%%%%%%%%%%%%%%%%%%%%%
software requirements negotiation a theory ui based spiral approach
\bibitem{ } ... \LaTeX:\\ \url{ }
%%%%%%%%%%%%%%%%%%%%%%%%%%%%%%%%%%%%%%%%%%%%%%%%%%%%%%%%%%%%%%%%%
the worlds a stage: a survey on requirementsengineering using a real life case study
%%%%%%%%%%%%%%%%%%%%%%%%%%%%%%%%%%%%%%%%%%%%%%%%%%%%%%%%%%%%%%%%%
\bibitem{ } ... \LaTeX:\\ \url{ }




challenges in requirements engineering

\bibitem{damian1999RequirementsEngineeringChallenge } ... \LaTeX:\\ \url{https://www.researchgate.net/publication/2462377_Challenges_in_Requirements_Engineering }
why goals-oriented for requirements engineering

\bibitem{lapouchnian2005goalorientedReqs} ... \LaTeX:\\ \url{https://www.researchgate.net/publication/249901480_Goal-Oriented_Requirements_Engineering_An_Overview_of_the_Current_Research }
design and build of collaborative information agents

\bibitem{jonkerTreurKlush200informativeAgents} ... \LaTeX:\\ \url{https://www.researchgate.net/publication/221622575_Design_of_Collaborative_Information_Agents }
treating nfiras first gradefor its testability
\bibitem{ } ... \LaTeX:\\ \url{ }
software requirements negotiation a theory ui based spiral approach

\bibitem{boehmBoseLeeRequirementsNegotiations } ... \LaTeX:\\ \url{https://www.cs.rug.nl/search/uploads/Teaching/RE2009Fall/paper/1995_Boehm_ICSE_Software%20Requirements%20Negotiation%20and%20Renegotiation%20Aids%20A%20Theory-W%20Based%20Spiral%20Approach.pdf }
the worlds a stage: a survey on requirementsengineering using a real life case study

\bibitem{breitmanLeiteCesar2002reallifeReqs } ... \LaTeX:\\ \url{https://www.researchgate.net/publication/2548016_The_world's_a_stage_a_survey_on_requirements_engineering_using_a_real-life_case_study_Karin_Koogan_Breitman_Julio_Cesar_S_do_Prado_Leite }
from inconsistencyhandling to non-conanical requirements management: a logical perspective

\bibitem{muHungJinLiu2013inconsistencyReqs } ... \LaTeX:\\ \url{https://www.researchgate.net/publication/257272175_From_inconsistency_handling_to_non-canonical_requirements_management_A_logical_perspective }
managing inconsistent specification: reasoning, analysis, action

\bibitem{ hunterNuseibeh1996manageSpecs} ... \LaTeX:\\ \url{https://www.researchgate.net/publication/2635497_Managing_Inconsistent_Specifications_Reasoning_Analysis_and_Action  }
representingand using nonfunctional requirements: a process-oriented approach

\bibitem{ myloloupos1992representingReqs} ... \LaTeX:\\ \url{https://www.researchgate.net/publication/3187474_Representing_and_Using_Non-Functional_Requirements_A_Process-Oriented_Approach }
Four dark corners of requirements engineering

\bibitem{zavePamela4darkCorners } ... \LaTeX:\\ \url{ http://www.cse.msu.edu/~chengb/RE-491/Papers/dark-corners-re-zave-jackson.pdf }
classification of research methods in requirements engineering

\bibitem{zavePAmela1997regEngineering } ... \LaTeX:\\ \url{https://www.researchgate.net/publication/220565934_Classification_of_Research_Efforts_in_Requirements_Engineering }
agent-basedtactocs for goal-oriented requirements elaboration





model checking 
14,15,16,28,29,30,32,35,40,41,46,47,48,49,61,62,63,64,65,66-95,121,140,145,175,178,195,199,200,201,202,203,215-230,232,233,234,235,236



f \colon A \to B \\

8,99,135,170,222,235,252,253


deel 1
Verkennen van het onderzoeks- en rapporteeringsterrein
Terreinafbakening
Voorgeschreven onderwerp
Wat is de achtergrond van de opdracht
Hoe moeten de begrippen worden ingevuld
zijn er randvoorwaarden
Vrij onderwerp
Kies een belangstellingsgebied
Verken he belangstellingsgebied
Kies een uitvoerbaar onderwerp
Baken het onderwerp affirmative
Definieer en operationaliseer de centrale begrippen
Probleemstellig en hypothese
Formuleren van de probleemstelling
Gebruiksmogelijkheden van de hypothese
Doelstelling
Functie van de doelstelling
Spraakverwarring rond het begrip doelstelling
Doelstelling van praktijkonderszoek
Problemen bij praktijkonderzoek
Enkele voorbeeldsituaties
Doelstelling van theoretisch onderzoeks
Mogelijke theoretisch doelstellingen
Theoretische en maatschappelijke doelstellingen
Doelstelling van leeronderzoek
publiek
Verkennen van het publiek
De academische en professionele lezer
Schrijven in het onderwijs
Schrijven in de beroepspraktijk
Lezers in de organisatie
Het primaire  publiek
Het primaire publiek
Het secundaire publiek
Werkwijze of strategie
afleiden van deelvragen
Bepalen va de onderzoeksmiddelen
Opstellen van een tijdschema
Werkplan of onderzoeksvoorstel
Samenwerkingsplan

Opsporing van informatie
literatuuronderzoek
ontsluitingsmiddelen van bibliotheken
catalogi
bibliografische naslagwerken
elektronische bestande  9databases)
methode voor literatuuronderzoek
Fase 1 algemene orientatie
Fase 2 Raadplegen van de bibliografische bronnen
Fase 3 Bestuderen van de gevonden publicaties
Fase 4 Afronden van het iteratuuronderzoek
Behandelen van literatuurgegevens
Evaluatie van literatuurgegevens
Noteren van literatuurgegevens
Opslagmogelijkheden
Soorten aantekeningen
Eigen onderzoek
Observeren
Aandachtspunten bij observeren
Betrouwbaarheid en validiteit
Experimenteren
Hoodregel bij experimenteren
Validiteit van experimenten
Laboratoriumjournaal
Interviewen
Voordelen van een intervieuw boven een enquete
Voorbereiding op het intervieuw
Voornaamste intervieuwtechnieken
Aanvullende intervieuwtechnieken
Enqueteren
Responsverhogende middelen
Soorten vragen en antwoordmogelijkheden
Formuleren van vragen en antwoorde
De lay-out van het enqueteformulier

Opstellen van een rapportschema
Ordeningsprocedure
inventariseren
selecteren
rubriceren
rangschikken
gan van het grote geheel vaan de kleie details
ga van het algemene naar het bijzondere
ga van het bijzondere naar het algemene
ga van meer naar minder belangrijk
ga van minder naar meer belangrijk
detailleren
controleren
de hoofdindeling relevant en compleet
is he rapportschema duidelijk
is het schema evenwichtig van opbouw
Heeft u een consequent een indelingsperspectief gehanteerd
Staan er niet meer dan 6 a 7 hoofdstukken in uw schema?
Gaat uw onderverdeling niet verder dan drie a vier niveaus
Heeft ieder punt dat wordt onderverdeeld ten minste twee subpunten
sluite de onderdelen in uw schema elkaar uitgaan
heeft u foutieve subordienatie vermeden
Heeft u foutieve coordinate vermeden
Indelingspatronen
beschrijvende indelingspatronen

thematische beschrijving
chronologische beschrijving
inductieve of wetenschappelijke indelingspatronen
onderzoeksteksten
probleemoplossende teksten
evaluerende teksten
deductieve of zakelijke indelingspatronen
onderzoeksteksten, deductief
probleemoplossende teksten, deductief
evaluerende teksten, deductief

deel 2
Algemene aanwijzingen voor het gebruik van illustraties
Functie van de illustratie
Keuze van de illustratie
Presentatie van de illustratie
Plaats van de illustratie
Tabellen
Soorten tabellen
Presentatie van tabellen

Figuren
Grafieken
Diagrammen
Schema's


deel 3

Voorafgande onderdelen
Omslag
Titelpagina
Voorwoord/Te geleide/Begeleidend schrijven
Inhoudsopgave
onderdelen van de inhoudsopgave
formuleren van de titels
Samenvatting
functie van de samenvatting
plaats v de samenvatting
soorten samenvattingen
de informatieve samenvatting
structuur van de samenvatting
lengte van de samenvatting
taalgebruik in de samenvatting
Hoofdonderdelen
Inleiding
inhoud van de inleiding
neem voldoende achtergrondinformatie op
geef aan op welke vraag u antwoord geeft
Maak duidelijk wat het doel is van uw onderzoek
Geef de beperkingen van het onderzoek aan
Verbind de inleiding met de rest van de teskt
Opening van de inleiding
retorische vraag
vergelijking en contrast
illustratie
humor
anekdote
spectaculaire details of getallen
opvallende trefwoorden
citaat of spreuk
verwijzing naar een avtuele gebeurtenis of situatie
verasssende of shockerende opmerking
Oorzaken/Gevolgen
onjuiste oorzaak-gevolgrelaties
onjuiste gevolg-oorzaak relaties
Voor- en nadelen
Methode
Resultaten en Discussie
resultaten
discussie
twee valkuilen
verschil tussen discussie en conclusie
Afsluiting
conclusie
zorg voor een duidelijke relatie tussen uw conclusies en de resultaten
maak uw conclusies zelfstandig leesbaar

formuleer de kernachtige conclusies
aanbevelingen
zorg voor en duidelijke relatie tussen uw aanbevelingen en conclusies
maak dudelijk at uw aanbevelingen uitvoerbaar zijn
concretiseer uw aanbevelingen
slot of besluit
nabeschouwing of evaluatie
Slotonderdelen
Literatuuropgave
methoden voor literatuurverwijzing
voetnoten enn nummers de naar deze noten verwijzen
een alfabetische lijst van literatuurbronnen
een genummerde lijst van literaturbonnen en in de teskt tussen haakjes nummers die naar dez bronnen verwijzen (het auteur-nummersysteem)
een alfabetische lijst van literatuurbronnen en in de tekst tussen haakses auteursnamen, jaartallen en paginanummers die naar deze lijst verwijzen (het auteur jaarsysteem)
eindnoten en in de tekst nummers die naar deze noten verwijzen, plus een alfabetische literatuuropgave
omschrijvingswijze van publicaties
boeken (of ander eop zichzelf staande werken zoalls rapporten en dictaten)
artikelen in tijdschriften
enkele bijzondere situaties
elektronische publicaties
Bijlage
Register (index)

deel 4

schrijven met de tekstverwerker
kenmerken van schrijven met de computer
tien tips voor schrijven met de computer
de eerste verzie of het klad
tip 1maak een schrijfschema
tip 2 onderken uitstelgedrag
tip 3 schrijf gelijk op met het onderzoek
tip 4 onderbreek het schrijfproces zo min mogelijk voor correcties
tip 5 creer omstandigheden waaronder u optimaal kunt werken
tip 6 schrijf of typ zo lang mogelijk - minimaal drie kwartier achter elkaar door
de definitieve versie of het 'net'
tip 7 laat uw klad afkoelen voor u het gaat corrigeren
tip 8 corrigeer uw tekst aan de hand van een uitdraai
tip 9 corrigeer in  een aantal rondes
controle op volledigheid
controle op opbouw en gedachtengang
controle op taalgebruik
tip 10 lees de tekst  hardop langzaam aan uzelf voor


De alinea
Functie van de alinea
Samenhang in en tussen alinea's
Vormgeving van de alinea
thematische alinea
soorten kernzinnen
positie van de kernzin
verbindende alinea
samenhang in en tussen alinea's
signaalwoorden en -tekens
overgangszinnen

wee
herhalen en synoniemen
parallelle constructies
vormgeving van de alinea
lengte van de alinea
markeren van een nieuwe alinea
Zinsbouw
overzichtelijke zinsbouw
zorg voor duidelijke zinsverbanden
maak zinnen niet te lang
houd bij elkaar wat bij elkaar hoort
zet de essentie voorop
formuleer de delen van een opsommng parralel
Aantrekkelijke zinsbouw
varier de volgorde van de zinsdelen
gebruik waar mogelijk de bedrijvende vorm
verschil tussen lijdende en bedrijvende vorm
gebruiksmogelijkheden van de lijdende vorm
de lijdende vorm in zakelijke teksten
laat de werkwoorden het werk doen
kenmerken van de naamwoordstijk
gebruiksmogelijheden van de naamwoordstijl

Woordgebruik
Levendig woordgebruik
maak gepast gebruik van persoonlijke voornaamwoorden
voer waar mogelijk 'met name genoemde' personen ten tonele
varier uw woordgebruik
verwijswoorden
synoniemen
omschrijvingen
verduidelijk moeilijke zaken met voorbeelden n vergelijkingen
exact woordgebruik
concretiseer blangrijke abstracte begrippen
wees zuinig met relativerende woorden
vage kwantificeringen
vage modale woorden
gebruik duideljke en correcte verwijswoorden
onduidelijke verwijzingen
foutieve verwijzingen
stem de werkwoordstijden af op de status van de informatie
Direct woordgebruik
vervang omslachtige voorzetseluitdrukkingen
zeg het in kernachtige bewoordingen
Eenvoudig woordgebruik
vermijd onnodig moeilijke woorden
lange woorden
intellectuelenwoorden
wees voorzichting met het gebruik van vaktermen
Spelling en interpunctie
Enkele spellingsprobemen
schrijfwijze van woordgroepen
los of aaneenschrijven
aaneenschrijven of koppelteken
tusssenklank -e(n)
tussenklank(-s)
apostrof
deelteken (trema)
weglatingstreepje
hoofdletters
getallen in woorden
vervoeging van engele werkwoorden
Leestekens
komma
dubbele punt









opsommend verband: ten eerste, 1, A , primair, eerst, voorheen, vroeger, coordat, aanvankelijk; ten tweede, 2, B, secundair, later, inmiddels, vervolgens, daarnaast, verder, nog eens, voorts, nadat, bovendien, ook; nu, uiteindelijk, ten slotte, als laatste, in de laatste plaats.
tegenstellend verband: maar, niettemin, echter, toch, evenwel, hoewel, ondanks, desonodanks, terwijl, of... of, enerzijds... anderzijds, daarentegen, weliswaar... maar.
vergelijkend verband: evenals, evenzeer, eveneens, evenzo, op dezelfde wijze, net zo, vergelijk.
illistrerend verband: zoals, bijvoorbeeld, als volgt, o.a., in het bijzonder, ter illustratie, neem, stel, zo.
verklarend verband: omdat, doordat, daarom, daardoor, want, namelijk, daar, immers, aangezien, de reden/de oorzaak/ het gevolg hiervan, waardoor, op grond van, ten gevolge van.
concluderend verband: dus, dan ook, hieruit volgt, hieruit valt af te leiden, concluderend, zo blijkt, kortom, uiteindelijk.
samenvattend verband: samenvattend, dus, concluderend, alles overziend, afsluitend, ten slotte, kortom, al met al, uiteindelijk, alles overziend







\degree
\lesssim
\arcmin
\fh
\fdg
\fp
\sun
\gtrsim
\arcsec
\fm
\farcm
\micron
\earh
\sq
\fd
\fs
\farcs
\'{o}
\^{o}
\"{o}
\={o}
\.{o}
\u{o}
\v{o}
\H{o}
\t{0o}
\c{o}
\d{o}
\b{o}
\'{o}
\'{o}

\oe
\OE
\ae
\AE
\aa
\AA
\o
\0
\l
\L
\ss



\dag
\ddag
\#
\&
\{
\S
\P
\$
\_
\}
\copywright
\pounds
\%


\hat{a}
\check{a}
\tilde{a}
\acute{a}
\grave{a}
\dot{a}
\ddot{a}
\breve{a}
\bar{a}
\vec{a}

\alpha
\beta
\gamma
\delta
\epsilon
\zeta
\eta
\theta
\iota
\kappa
\lambda
\mu
\nu
\xi
\pi
\rho
\sigma
\tau
\upsilon
\phi
\chi
\psi
\omega



\varepsilon
\vartheta
\varrho
\varsigma
\varphi

\Gamma
\Delta
\Theta
\Lambda
\Xi
\Pi
\Sigma
\Upsilon
\Phi
\Psi
\Omega

\pm
\mp
\setminus
\cdot
\times\ast
\start\diamond
\circ
\bullet
\div
\lhd
\vee
\wedge
\oplus
\ominus
\otimes
\oslash
\capacity\cup
\uplus
\aqcap
\sqcap
\aqcup
\triangleleft
\triangleright
\wr
\bigcirc
\bigtriangleup
\bigtriangledown
\rhd
\odot
\dagger
\ddagger
\amalg
\unlhd
\unrhd


\leq



\sum
\prod
\coprod
\int
\oint
\bigodot
\bigoplus
\bigcap
\bigcup
\bigsqcup
\bigvee
\bigwedge
\bigotimes
\biguplus



\aleph
\hbar
\imath
\jmath
\ell
\wp
\Re
\Im
\partial
\infty
\Box
\forall
\artists
\neg
\flat
\natural
\mho
\prime
\emptyset
\nabla
\surd
\top
\bot
\|
\angle
\triangle
\backslash
\Diamond
\sharp
\clubsuit
\diamondsuit
\heartsuit
\spadesuit





\cong
%  \include{}   %(optioneel) bijlagen
%  \include{}   %(optioneel) bijlagen
\fi
\end{document}       %Einde van het document
