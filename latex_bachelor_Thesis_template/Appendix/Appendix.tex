\chapter{Uppaal source}
\label{appendix}
\thispagestyle{myheadings}



\begin{verbatim}
	// Place global declarations here.
	
	// knop 
	chan klik;
	
	// alle lampen tegelijk
	//broadcast chan klik;
	
	// max aantal lampen
	const int MAX=4;
	
	//declarate voor een lamp
	typedef int[0,MAX-1] id_l;
	
	// lichtintensiteit in lu
	typedef int[0,10] lumen;
	
	lumen bright;
	
	
	
	// locale variabelen voor de templatet lamp: geef de template de volgende parameter: id_l lampnr
	
	void functie()
	{
		
	}
	
	lumen sterkte;
	
	
	// wacht 5 tijdseenheden
	clock x;
	
	// een constraint op een bepaalde variabele
	bool isForMe()
	{
		//	return false;
		
		if(lampnr < 2 && bright<6) return true;
		else if(lampnr>1 && bright>=6) return true;
		else return false;
		
	}
	
	
	//verschillende tijdseenheden voor even en oneven lampnummers
	
	
	
	// Place local declaraties voor de knop
	
	
	
	clock x;
	
\end{verbatim}
\newpage

\section{\\ Appendix B: Model eerste deelname aan cursus 2020}
% the \\ insures the section title is centered below the phrase: Appendix B

Text of Appendix B is Here



\begin{verbatim}
	
	Queries
	Sluis.Draining-->Deuren.laag_open
	Deuren.laag_open-->Stoplicht.Green
	E<> (Ship.ship_can_move&&Stoplicht.Green)
	A[] not (Stoplicht.Green && not (Deuren.hoog_open||Deuren.laag_open||Deuren.stopgaplow1||Deuren.stopgaplow2||Deuren.stopgaphigh1||Deuren.stopgaphigh2))
	A[] not ((Deuren.hoog_open||Deuren.laag_open||Deuren.Opening_laag||Deuren.Opening_hoog||Deuren.Closing_hoog||Deuren.Closing_laag) && (Sluis.Draining||Sluis.Filling||Sluis.draining2||Sluis.Filling2))
	Sensor.Wait-->Sensor.Wait
	Stoplicht.Green-->Stoplicht.Green
	(Deuren.hoog_open||Deuren.laag_open)-->(Deuren.laag_open||Deuren.hoog_open)
	Deuren.laag_open-->Deuren.Closed
	Deuren.hoog_open-->Deuren.Closed
	Deuren.Closed-->Stoplicht.Red
	Ship.ship_can_move-->Deuren.Closed
	Deuren.hoog_open-->Stoplicht.Green
	Ship.ship_can_move-->Stoplicht.Green
	A[] not (Deuren.laag_open && Deuren.hoog_open)
	Ship.ship_can_move-->Ship.ship_can_move
	A[] not (Deuren.laag_open && Sluis.water != Sluis.water_laag)
	A[] not (Deuren.hoog_open && Sluis.water != Sluis.water_hoog)
	A[]not deadlock
	
	Project declaraties
	//Declarations
	
	chan boot_hoog;
	chan boot_laag;
	chan changedoor_low;
	chan changedoor_high;
	chan ship_moves;
	chan ship_abletomove;
	chan changelight;
	
	\\Sluis declaraties
	const int water_laag=0;
	const int water_hoog=10;
	const int water_median=(water_hoog+water_laag)/2;
	int[water_laag,water_hoog] water=water_median;
	clock x;
	\\Stoplicht declaraties
	
	\\Ship declaraties
	clock x;
	\\Sensor declaraties
	
	\\Deuren declaraties
	bool stoplicht_hoog=false;
	bool stoplicht_laag=false;
	clock x;
	
	
	\\System declaraties
	system Deuren,Sensor,Sluis,Ship,Stoplicht;
	
	
	
	
\end{verbatim}

\newpage
\section{\\ Appendix B: Model herkansing tweede deelname aan cursus 2020}
% the \\ insures the section title is centered below the phrase: Appendix B

Text of Appendix B is Here

\begin{verbatim}
	// Place global declarations here.
	/*
	Project working
	
	AtArrival
	StoplightRed
	DoorOpen
	StoplightGreen
	Startmove
	Sensor
	SchipEntered
	Doorclosed
	StoplightRed
	---------
	Nivelleer started
	NIvelleer stopped
	Waterlevel equilibrium
	-----------
	AtLeaving
	Stoplightred
	Dooropened
	Stoplightgreen
	StartMove
	Sensor
	SchipHasLeft
	Doorclosed
	StoplightRed
	
	
	Uitleg
	Als het schip boven is, dan is waterlvel gelijk aan hoog, filling valve is dicht, lower gates zijn gesloten, uppergates zijn open,empty valve is dicht. 
	Schip is in waterlock, waterlevel is hoog, filling valve is dicht, lower gates gesloten, upper gates gesloten, empty valve is open. 
	Schip is dan laag, waterlevel gelijk aan laag, filling valve is dicht, lowergates zijn open, uppergates zijn dicht, empty valve is dicht.
	AtArrivalHigh
	
	AtArrivalLow
	Als schip beneden is dan is waterlevel gelijk aan laag, filling valve is dicht, lower gates zijn open, upper gates zijn dicht, empty valve is open. 
	Schip is in water lock, waterlevel is laag, flilling valve is open, lower gates zijn gesloten, upper gates zij gesloten, empty valve is dicht,. 
	Schip is dan hoog, waterlevel is gelijk aan hoog, filling valve is dicht, uppergates zijn open, lowergates zijn dicht, filling valve is dicht
	
	
	*/
	
	const int N = 2;         // # trains
	typedef int[0,N-1] id_t;
	
	chan        appr[N], stop[N], leave[N];
	urgent chan go[N];
	
	// waterniveau in in meter van 0 tot 10
	typedef int[3,10] waterniveau;
	
	waterniveau level;
	
	
	//doors
	chan lower_gate;
	chan upper_gate;
	//filling
	chan emptying_valve;
	chan filling_valve;
	bool nivelleer_sessie_bezig;
	// water level
	chan high_water_level;
	chan low_water_level;
	//sluices
	chan signal_sluis_low[N];
	chan signal_sluis_high[N];
	//
	chan move[N];
	//
	chan groen;
	chan rood;
	
	clock central;
	
	\\geef de schip parameter const id_t id
	// Place local declarations here.
	clock schip_clock;
	
	\\sensor declaraties
	clock x;
	
	
	
	\\sluis declaraties
	
	const int water_laag=3;
	const int water_hoog=10;
	const int water_median=(water_hoog+water_laag)/2;
	int[water_laag,water_hoog] water=water_median;
	// level wordt gelijk gezet met temp
	// temp is gelijk aan waterniveau
	clock sluis_clock;
	id_t list[N+1];
	int[0,N] len;
	bool contentHigh, contentLow;
	// Put an element at the end of the queue
	void enqueue(id_t element)
	{
		list[len++] = element;
	}
	
	// Remove the front element of the queue
	void dequeue()
	{
		int i = 0;
		len -= 1;
		while (i < len)
		{
			list[i] = list[i + 1];
			i++;
		}
		list[i] = 0;
	}
	
	// Returns the front element of the queue
	id_t front()
	{
		return list[0];
	}
	
	// Returns the last element of the queue
	id_t tail()
	{
		return list[len - 1];
	}
	
	
	
	\\stoplicht declaraties
	clock stoplicht_clock;
	
	
	
	\\pomp declaraties
	
	const int water_laag=3;
	const int water_hoog=10;
	const int water_median=(water_hoog+water_laag)/2;
	int[water_laag,water_hoog] water=water_median;
	clock pomp_clock;
	// waterniveau van de sensor voor de sluis is gelijk aan level
	waterniveau depth;
	
	
	
	// een constraint op een bepaalde variabele
	bool isForLow()
	{
		//	return false;
		
		if( level>=3) return true;
		else return false;
		
	}
	
	bool isForHigh()
	{
		//	return false;
		
		if( level>=6) return true;
		//else if(level>=6) return true;
		
		else return false;
		
	}
	
	
	
	//verschillende tijdseenheden voor even en oneven lampnummers
	
	
	
	
	
\end{verbatim}






\chapter{Deelonderzoek naar veiligheidsrisico's en eisen voor sluizzen}

Gevonden weblinks in google op 07-04-2023 met zoekopdracht: "veiligheidsrisico's voor sluizen en waterwerken"
https://www.tweedekamer.nl/downloads/document?id=80443e97-f17e-499c-b3f2-ad608f32e1aa&title=Rapportage%20Staat%20van%20de%20infra%20RWS%20%28definitief%29.pdf
https://www.nu.nl/internet/5814282/rekenkamer-waterwerken-niet-goed-beveiligd-tegen-cyberaanvallen.html
https://www.deltalimburg.nl/article/9824/Onderhoudswerkzaamheden+aan+Sluis+Linne+afgerond
https://nieuwesluisterneuzen.eu/veiligheid
https://www.mrdmarinesupport.nl/nl/maritieme-dienstverlening/ondersteuning-veiligheid/
https://www.infrasite.nl/bouwen/2021/05/27/veiligheid-voorop-begin-project-sluis-of-brug-altijd-met-risicobeoordeling/
https://www.wdodelta.nl/bediening-schutsluizen-vechterweerd-en-vilsteren
https://www.infrasite.nl/waterbouw-deltas/2021/05/21/sluis-heel-onder-handen-genomen/
https://www.hdsr.nl/actueel/nieuws/@154100/lichtprojecties-zetten-waterliniesluizen/
https://nos.nl/artikel/2277937-rekenkamer-hack-aanval-op-waterwerk-niet-altijd-opgemerkt
https://magazine.vhbinfra.nl/1-4/zeesluis-ijmuiden/
https://varendoejesamen.nl/kenniscentrum/artikel/onderhoud-sluis-linne-afgerond
https://www.noorderzijlvest.nl/_flysystem/media/vragen-webinar-30-8-2021-en-inloop-2-9-2021-nieuwe-waterwerken-zoutkamp.pdf
https://eenvandaag.avrotros.nl/item/sluizen-en-gemalen-kunnen-eenvoudig-worden-gehackt/
https://www.gww-bouw.nl/artikel/de-eerste-sluis-met-kantelende-sluisdeur/
https://tkhsecurity.com/nl/waterwerken/
https://www.h2owaternetwerk.nl/h2o-actueel/rekenkamer-vitale-waterwerken-nog-onvoldoende-beschermd-tegen-cyberaanvallen
https://anteagroup.nl/diensten/beweegbare-en-vaste-kunstwerken/sluizen
https://balmbv.nl/sluizen-in-tiel-voor-decennia-beschermd/
https://www.volkskrant.nl/nieuws-achtergrond/hacker-dringt-door-in-controlekamer-waterwerk-cyberterrorist-kan-ons-land-onder-water-zetten~b6fcbc3c/?referrer=https%3A%2F%2Fwww.google.com%2F
https://www.magazinesrijkswaterstaat.nl/bereikbaarzeeland/2021/01/krammersluizencomplex-verleden-heden-en-toekomst
https://www.noord-holland.nl/Actueel/Archief/2022/September_2022/Hollandse_Waterlinies_in_de_schijnwerpers
https://www.watersport-tv.nl/nw-31400-7-3715235/nieuws/cybersecurity_vitale_waterwerken_niet_waterdicht.html
https://www.tonverheijen.nl/artikelen/bruggen-sluizen-en-tunnels-kunnen-ook-gehackt-worden
https://nuactueel.noordhoff.nl/hackers-zetten-zo-de-sluizen-open-hv-bb/
https://binnenvaartkrant.nl/waterwerken-zijn-kwetsbaar-voor-cyberaanvallen
https://www.rtlnieuws.nl/nieuws/nederland/artikel/3758966/cyberbeveiliging-waterschappen-hapert-sluizen-kunnen-worden
https://expert.rittal.nl/wp-content/uploads/2017/05/Referentieverhaal-Provincie-Zuid-Holland.pdf
https://www.vtmgroep.nl/blog/waterwerken-in-nederland-onvoldoende-beveiligd-tegen-cyberaanvallen
https://www.securitymanagement.nl/waterwerken-nog-altijd-te-hacken/
https://www.heijmans.nl/nl/verhalen/waterwerken/
https://www.tno.nl/nl/duurzaam/veilige-duurzame-leefomgeving/infrastructuur/natte-infrastructuur/
https://www.computable.nl/artikel/nieuws/security/6634379/250449/waterwerken-slecht-beveiligd-tegen-hackers.html
https://www.cobouw.nl/271317/tunnels-bruggen-en-sluizen-onvoldoende-beschermd-tegen-hackers
https://www.zuid-holland.nl/onderwerpen/verkeer-vervoer/vaarwegen/
http://www.wesemann.nl/nl/nieuws-en-pers/274-veiligheid-op-het-water-en-op-het-land.html
https://www.rekenkamer.nl/actueel/nieuws/2019/03/28/cybersecurity-vitale-waterwerken-niet-waterdicht
https://www.rijkswaterstaat.nl/over-ons/onze-organisatie/vervanging-en-renovatie
https://www.arbo-online.nl/22896/werkzaamheden-langs-het-water


\chapter{Deelonderzoek wet en regelgeving voor sluizen}



Gevonden weblinks in google op 07-04-2023 met zoekopdracht: "wet en regelgeving voor sluizen"
https://www.hdsr.nl/publish/pages/86927/sluizen_in_of_bij_een_waterkering_-_uitvoeringsregels.pdf
https://api1.ibabs.eu/publicdownload.aspx?site=sluis&id=100100292
https://services.pilz.nl/wp-content/uploads/2021/12/brochure_bruggen_2018.pdf
https://lokaleregelgeving.overheid.nl/CVDR375606/6
https://zoek.officielebekendmakingen.nl/stb-2019-27.html
https://a-quin.nl/nieuws/veiligheid-van-bruggen-sluizen-waarborgen-wie-wat-hoe/
https://www.gemeentesluis.nl/Bestuur_en_Organisatie/Wetten_Regels_Bekendmakingen
https://www.overijssel.nl/onderwerpen/verkeer-en-vervoer/varen-in-overijssel/informatie-bedieningstijden-sluizen-en-bruggen-noordwest-overijssel/
https://www.rijkswaterstaat.nl/water/wetten-regels-en-vergunningen
https://www.schuttevaer.nl/nieuws/actueel/2022/11/23/binnenvaart-zit-klem-tussen-regels-en-realiteit-kapotte-steigers-en-gesperde-sluizen-dwingen-tot-doorvaren/
https://repository.officiele-overheidspublicaties.nl/CVDR/CVDR271406/1/html/CVDR271406_1.html
https://www.zeeland.nl/actueel/bedieningstijden-sluizen-en-bruggen
https://www.amsterdam.nl/verkeer-vervoer/varen-amsterdam/regels-varen/
https://www.schielandendekrimpenerwaard.nl/wat-doen-we/regels-en-afspraken-over-beheer-keur-en-leggers/
http://www.wetboek-online.nl/wet/Wet%20tot%20samenvoeging%20van%20de%20gemeenten%20Aardenburg%20en%20Sluis.html
https://www.rijnland.net/regels-op-een-rij/richtlijnen-en-akkoorden/alle-regelgeving-van-rijnland/
https://www.itbb.nl/diensten/advies-ce-markering-europese-richtlijnen/
https://www.portofamsterdam.com/nl/scheepvaart/zeevaart/regelgeving
https://www.watersportverbond.nl/nieuws/achterstallig-onderhoud-wachtplaatsen-bruggen-en-sluizen-zuid-holland-zorgelijk/
https://varendoejesamen.nl/nieuws
https://www.flevoland.nl/wat-doen-we/flevowegen-vlot-en-veilig-door-flevoland/water/varen-in-flevoland/bediening-bruggen-en-sluizen
https://eur-lex.europa.eu/legal-content/NL/TXT/PDF/?uri=CELEX:32020L0012&from=DE
https://www.werkenvoornederland.nl/organisatie/rijkswaterstaat/ict-middelen-maken-om-bruggen-sluizen-en-tunnels-te-besturen
https://www.lobocom.nl/infra-bruggen-sluizen
https://waterrecreatienederland.nl/content/uploads/2018/04/richtlijnen-vaarwegen-2017.pdf
https://www.wetterskipfryslan.nl/melden-en-regelen/vergunningen-wetten-en-regels
https://www.onlinezeilschool.nl/sluizen/
https://www.provincie.drenthe.nl/onderwerpen/verkeer-vervoer/vaarwegen/rondje-drenthe/bedieningstijden/



\usepackage{pgf}
\newcommand\setform{\pgfqkeys{/form }}
\setform{field1/.store in=\fieldi,
	field2/.store in=\fieldii,
}


%\addtolength{\oddsidemargin}{-.875in}
%\addtolength{\evensidemargin}{-.875in}
%\addtolength{\textwidth}{1.75in}

%\addtolength{\topmargin}{-.875in}
%\addtolength{\textheight}{1.75in}


\newcommand\myform{%
	\fboxrule=0.4pt
	
	
	\fbox{\begin{minipage}{\textwidth}
			\fbox{\begin{minipage}[t][3cm][t]{0.25\textwidth}
					Naam vergadering
			\end{minipage}}%
			\fbox{\begin{minipage}[t][3cm][t]{0.25\textwidth}
					Datum en plats
			\end{minipage}}%
			\fbox{\begin{minipage}[t][3cm][t]{0.44\textwidth}
					Namen aanwezigen
			\end{minipage}}
			\fbox{\begin{minipage}[t][1cm][t]{0.98\textwidth}
					Opening en goedkeuring
			\end{minipage}}
			
	\end{minipage}}
	
	\fbox{\begin{minipage}{\textwidth}
			
			\fbox{\begin{minipage}[t][3cm][t]{0.98\textwidth}
					Ingekomen stukken en rondvraag
			\end{minipage}}
	\end{minipage}}
	
	\fbox{\begin{minipage}{\textwidth}
			\fbox{\begin{minipage}[t][10cm][t]{0.98\textwidth}
					Sluiting
			\end{minipage}}%
			
			
	\end{minipage}}
}


\newpage








\begin{center}  
	\begin{tabular}{ | l | l | l | p{5cm} |} % you can change the dimension according to the spacing requirements  
		\hline  
		\multicolumn{4}{|l|}{Actielijst} \\ \hline  
		Onderwerp & Besluit & Wie &Gereed \\ \hline  
		Orange & Fruit & Vitamin C & It is fruit, which is full of nutrients and low in calories. They can promote clear, healthy skin and also lowers the risk for many diseases. It reduces cholesterol and also helps in building a healthy immune system.\\ \hline  
		
		Cauliflower & vegetable & B-Vitamins & It is the vegetable, which is high in fiber and B-Vitamins. It also provides antioxidants, which help in fighting or protect against cancer. It enhances digestion and has many other nutrients.\\ \hline  
		
	\end{tabular}  
\end{center}  


\chapter{Queries}
\begin{verbatim}
	
	Queries
	Sluis.Draining-->Deuren.laag_open
	Deuren.laag_open-->Stoplicht.Green
	E<> (Ship.ship_can_move&&Stoplicht.Green)
	A[] not (Stoplicht.Green && not (Deuren.hoog_open||Deuren.laag_open||Deuren.stopgaplow1||Deuren.stopgaplow2||Deuren.stopgaphigh1||Deuren.stopgaphigh2))
	A[] not ((Deuren.hoog_open||Deuren.laag_open||Deuren.Opening_laag||Deuren.Opening_hoog||Deuren.Closing_hoog||Deuren.Closing_laag) && (Sluis.Draining||Sluis.Filling||Sluis.draining2||Sluis.Filling2))
	Sensor.Wait-->Sensor.Wait
	Stoplicht.Green-->Stoplicht.Green
	(Deuren.hoog_open||Deuren.laag_open)-->(Deuren.laag_open||Deuren.hoog_open)
	Deuren.laag_open-->Deuren.Closed
	Deuren.hoog_open-->Deuren.Closed
	Deuren.Closed-->Stoplicht.Red
	Ship.ship_can_move-->Deuren.Closed
	Deuren.hoog_open-->Stoplicht.Green
	Ship.ship_can_move-->Stoplicht.Green
	A[] not (Deuren.laag_open && Deuren.hoog_open)
	Ship.ship_can_move-->Ship.ship_can_move
	A[] not (Deuren.laag_open && Sluis.water != Sluis.water_laag)
	A[] not (Deuren.hoog_open && Sluis.water != Sluis.water_hoog)
	A[]not deadlock
	
	Project declaraties
	//Declarations
	
	chan boot_hoog;
	chan boot_laag;
	chan changedoor_low;
	chan changedoor_high;
	chan ship_moves;
	chan ship_abletomove;
	chan changelight;
	
	\\Sluis declaraties
	const int water_laag=0;
	const int water_hoog=10;
	const int water_median=(water_hoog+water_laag)/2;
	int[water_laag,water_hoog] water=water_median;
	clock x;
	\\Stoplicht declaraties
	
	\\Ship declaraties
	clock x;
	\\Sensor declaraties
	
	\\Deuren declaraties
	bool stoplicht_hoog=false;
	bool stoplicht_laag=false;
	clock x;
	
	
	\\System declaraties
	system Deuren,Sensor,Sluis,Ship,Stoplicht;
	
	
	Uitleg
	Als het schip boven is, dan is waterlvel gelijk aan hoog, filling valve is dicht, lower gates zijn gesloten, uppergates zijn open,empty valve is dicht. 
	Schip is in waterlock, waterlevel is hoog, filling valve is dicht, lower gates gesloten, upper gates gesloten, empty valve is open. 
	Schip is dan laag, waterlevel gelijk aan laag, filling valve is dicht, lowergates zijn open, uppergates zijn dicht, empty valve is dicht.
	AtArrivalHigh
	
	AtArrivalLow
	Als schip beneden is dan is waterlevel gelijk aan laag, filling valve is dicht, lower gates zijn open, upper gates zijn dicht, empty valve is open. 
	Schip is in water lock, waterlevel is laag, flilling valve is open, lower gates zijn gesloten, upper gates zij gesloten, empty valve is dicht,. 
	Schip is dan hoog, waterlevel is gelijk aan hoog, filling valve is dicht, uppergates zijn open, lowergates zijn dicht, filling valve is dicht
	
	
	
\end{verbatim}


\chapter{Testresultaten}



\subsubsection{onderdeleel van de test}



\begin{tabular}{*{15}{|l|l|l|l|l|l|l|}} \hline
	\multicolumn{7}{|l|}{project name}                                                               \\ \hline
	\multicolumn{4}{|l|}{Test case ID}   &\multicolumn{3}{|l|}{Test designed by}                           \\ \hline
	\multicolumn{4}{|l|}{test priority (low/medium/high)}   &\multicolumn{3}{|l|}{Test design date}                           \\ \hline
	\multicolumn{4}{|l|}{Module name}   &\multicolumn{3}{|l|}{Test executed by}                           \\ \hline
	\multicolumn{4}{|l|}{Test title}   &\multicolumn{3}{|l|}{Test execution date}                           \\ \hline
	\multicolumn{4}{|l|}{Description}   &\multicolumn{3}{|l|}{ }                           \\ \hline 		
	\multicolumn{7}{|l|}{ }   																\\ \hline
	\multicolumn{7}{|l|}{Pre condition}                                                               \\ \hline
	\multicolumn{7}{|l|}{Dependencies}                                                               \\ \hline
	\multicolumn{7}{|l|}{ }   															\\ \hline
	Step  &  Test steps & Test data & expected result &Acual result &(pass or fail)&notes  \\ \hline
	
\end{tabular}





\chapter{Reflectie}

Ik heb erg veel geleerd van het analyseren van de vershillende requirements en specificaties en het opzetten van een model in Uppaal. Een dergelijk model opzetten had ik namelijk nog nooit gedaan. Het uitvoeren van onderzoek heb ik eerder gedaan. Ook de toetsing van het model met behulp van proposities heb ik nog nooit gedaan. Verder heb ik de kennis die had van programmeren/ design pattersn gebruikt om de verschillende templates in mijn Uppaal model van elkaar te onderscheiden. Het leukste onderdeel van het project vond ik hoe mijn templatemodel deadlockvrij werkte. Voor de verificatie van het model heb ik veel achtergrondinformatie opgezet, en het is mooi om te zien dat je met enkele duidelijke zinnen kan aantonen of een propositie geldig is of niet.  Verder had ik moeite met het opstellen van de juiste veiligheidseisen bij het model. Ik had aangenomen dat ik het project niet zou halen omdat ik de opdracht niet in teamverband heb uitgevoerd. Ik ben toch blij dat ik een concept heb opgeleverd dat ik kan toetsen aan de doormijzef opgestelde eisen en dat ik met mijn huidige kennis de proposities uit de requirements kan toetsen.


\usepackage{pgf}
\newcommand\setform{\pgfqkeys{/form }}
\setform{field1/.store in=\fieldi,
	field2/.store in=\fieldii,
}


%\addtolength{\oddsidemargin}{-.875in}
%\addtolength{\evensidemargin}{-.875in}
%\addtolength{\textwidth}{1.75in}

%\addtolength{\topmargin}{-.875in}
%\addtolength{\textheight}{1.75in}


\newcommand\myform{%
	\fboxrule=0.4pt
	
	
	\fbox{\begin{minipage}{\textwidth}
			\fbox{\begin{minipage}[t][3cm][t]{0.25\textwidth}
					Betrokken partij
			\end{minipage}}%
			\fbox{\begin{minipage}[t][3cm][t]{0.25\textwidth}
					Verantwoordelijk
			\end{minipage}}%
			\fbox{\begin{minipage}[t][3cm][t]{0.44\textwidth}
					datetimestamp here
			\end{minipage}}
			\fbox{\begin{minipage}[t][1cm][t]{0.98\textwidth}
					Korte notitie 
			\end{minipage}}
			
	\end{minipage}}
	
	\fbox{\begin{minipage}{\textwidth}
			
			\fbox{\begin{minipage}[t][3cm][t]{0.98\textwidth}
					test
			\end{minipage}}
	\end{minipage}}
	
	\fbox{\begin{minipage}{\textwidth}
			\fbox{\begin{minipage}[t][10cm][t]{0.98\textwidth}
					Foto incident
			\end{minipage}}%
			
			
	\end{minipage}}
}



\setform{field1 = G. Wales,
	field2 = Mathematics}
\myform



