\chapter{Uppaal source}
\label{appendix}
\thispagestyle{myheadings}



\begin{verbatim}
	// Place global declarations here.
	
	// knop 
	chan klik;
	
	// alle lampen tegelijk
	//broadcast chan klik;
	
	// max aantal lampen
	const int MAX=4;
	
	//declarate voor een lamp
	typedef int[0,MAX-1] id_l;
	
	// lichtintensiteit in lu
	typedef int[0,10] lumen;
	
	lumen bright;
	
	
	
	// locale variabelen voor de templatet lamp: geef de template de volgende parameter: id_l lampnr
	
	void functie()
	{
		
	}
	
	lumen sterkte;
	
	
	// wacht 5 tijdseenheden
	clock x;
	
	// een constraint op een bepaalde variabele
	bool isForMe()
	{
		//	return false;
		
		if(lampnr < 2 && bright<6) return true;
		else if(lampnr>1 && bright>=6) return true;
		else return false;
		
	}
	
	
	//verschillende tijdseenheden voor even en oneven lampnummers
	
	
	
	// Place local declaraties voor de knop
	
	
	
	clock x;
	
\end{verbatim}
\newpage

\section{\\ Appendix B: Model eerste deelname aan cursus 2020}
% the \\ insures the section title is centered below the phrase: Appendix B

Text of Appendix B is Here



\begin{verbatim}
	
	Queries
	Sluis.Draining-->Deuren.laag_open
	Deuren.laag_open-->Stoplicht.Green
	E<> (Ship.ship_can_move&&Stoplicht.Green)
	A[] not (Stoplicht.Green && not (Deuren.hoog_open||Deuren.laag_open||Deuren.stopgaplow1||Deuren.stopgaplow2||Deuren.stopgaphigh1||Deuren.stopgaphigh2))
	A[] not ((Deuren.hoog_open||Deuren.laag_open||Deuren.Opening_laag||Deuren.Opening_hoog||Deuren.Closing_hoog||Deuren.Closing_laag) && (Sluis.Draining||Sluis.Filling||Sluis.draining2||Sluis.Filling2))
	Sensor.Wait-->Sensor.Wait
	Stoplicht.Green-->Stoplicht.Green
	(Deuren.hoog_open||Deuren.laag_open)-->(Deuren.laag_open||Deuren.hoog_open)
	Deuren.laag_open-->Deuren.Closed
	Deuren.hoog_open-->Deuren.Closed
	Deuren.Closed-->Stoplicht.Red
	Ship.ship_can_move-->Deuren.Closed
	Deuren.hoog_open-->Stoplicht.Green
	Ship.ship_can_move-->Stoplicht.Green
	A[] not (Deuren.laag_open && Deuren.hoog_open)
	Ship.ship_can_move-->Ship.ship_can_move
	A[] not (Deuren.laag_open && Sluis.water != Sluis.water_laag)
	A[] not (Deuren.hoog_open && Sluis.water != Sluis.water_hoog)
	A[]not deadlock
	
	Project declaraties
	//Declarations
	
	chan boot_hoog;
	chan boot_laag;
	chan changedoor_low;
	chan changedoor_high;
	chan ship_moves;
	chan ship_abletomove;
	chan changelight;
	
	\\Sluis declaraties
	const int water_laag=0;
	const int water_hoog=10;
	const int water_median=(water_hoog+water_laag)/2;
	int[water_laag,water_hoog] water=water_median;
	clock x;
	\\Stoplicht declaraties
	
	\\Ship declaraties
	clock x;
	\\Sensor declaraties
	
	\\Deuren declaraties
	bool stoplicht_hoog=false;
	bool stoplicht_laag=false;
	clock x;
	
	
	\\System declaraties
	system Deuren,Sensor,Sluis,Ship,Stoplicht;
	
	
	
	
\end{verbatim}

\newpage
\section{\\ Appendix B: Model herkansing tweede deelname aan cursus 2020}
% the \\ insures the section title is centered below the phrase: Appendix B

Text of Appendix B is Here

\begin{verbatim}
	// Place global declarations here.
	/*
	Project working
	
	AtArrival
	StoplightRed
	DoorOpen
	StoplightGreen
	Startmove
	Sensor
	SchipEntered
	Doorclosed
	StoplightRed
	---------
	Nivelleer started
	NIvelleer stopped
	Waterlevel equilibrium
	-----------
	AtLeaving
	Stoplightred
	Dooropened
	Stoplightgreen
	StartMove
	Sensor
	SchipHasLeft
	Doorclosed
	StoplightRed
	
	
	Uitleg
	Als het schip boven is, dan is waterlvel gelijk aan hoog, filling valve is dicht, lower gates zijn gesloten, uppergates zijn open,empty valve is dicht. 
	Schip is in waterlock, waterlevel is hoog, filling valve is dicht, lower gates gesloten, upper gates gesloten, empty valve is open. 
	Schip is dan laag, waterlevel gelijk aan laag, filling valve is dicht, lowergates zijn open, uppergates zijn dicht, empty valve is dicht.
	AtArrivalHigh
	
	AtArrivalLow
	Als schip beneden is dan is waterlevel gelijk aan laag, filling valve is dicht, lower gates zijn open, upper gates zijn dicht, empty valve is open. 
	Schip is in water lock, waterlevel is laag, flilling valve is open, lower gates zijn gesloten, upper gates zij gesloten, empty valve is dicht,. 
	Schip is dan hoog, waterlevel is gelijk aan hoog, filling valve is dicht, uppergates zijn open, lowergates zijn dicht, filling valve is dicht
	
	
	*/
	
	const int N = 2;         // # trains
	typedef int[0,N-1] id_t;
	
	chan        appr[N], stop[N], leave[N];
	urgent chan go[N];
	
	// waterniveau in in meter van 0 tot 10
	typedef int[3,10] waterniveau;
	
	waterniveau level;
	
	
	//doors
	chan lower_gate;
	chan upper_gate;
	//filling
	chan emptying_valve;
	chan filling_valve;
	bool nivelleer_sessie_bezig;
	// water level
	chan high_water_level;
	chan low_water_level;
	//sluices
	chan signal_sluis_low[N];
	chan signal_sluis_high[N];
	//
	chan move[N];
	//
	chan groen;
	chan rood;
	
	clock central;
	
	\\geef de schip parameter const id_t id
	// Place local declarations here.
	clock schip_clock;
	
	\\sensor declaraties
	clock x;
	
	
	
	\\sluis declaraties
	
	const int water_laag=3;
	const int water_hoog=10;
	const int water_median=(water_hoog+water_laag)/2;
	int[water_laag,water_hoog] water=water_median;
	// level wordt gelijk gezet met temp
	// temp is gelijk aan waterniveau
	clock sluis_clock;
	id_t list[N+1];
	int[0,N] len;
	bool contentHigh, contentLow;
	// Put an element at the end of the queue
	void enqueue(id_t element)
	{
		list[len++] = element;
	}
	
	// Remove the front element of the queue
	void dequeue()
	{
		int i = 0;
		len -= 1;
		while (i < len)
		{
			list[i] = list[i + 1];
			i++;
		}
		list[i] = 0;
	}
	
	// Returns the front element of the queue
	id_t front()
	{
		return list[0];
	}
	
	// Returns the last element of the queue
	id_t tail()
	{
		return list[len - 1];
	}
	
	
	
	\\stoplicht declaraties
	clock stoplicht_clock;
	
	
	
	\\pomp declaraties
	
	const int water_laag=3;
	const int water_hoog=10;
	const int water_median=(water_hoog+water_laag)/2;
	int[water_laag,water_hoog] water=water_median;
	clock pomp_clock;
	// waterniveau van de sensor voor de sluis is gelijk aan level
	waterniveau depth;
	
	
	
	// een constraint op een bepaalde variabele
	bool isForLow()
	{
		//	return false;
		
		if( level>=3) return true;
		else return false;
		
	}
	
	bool isForHigh()
	{
		//	return false;
		
		if( level>=6) return true;
		//else if(level>=6) return true;
		
		else return false;
		
	}
	
	
	
	//verschillende tijdseenheden voor even en oneven lampnummers
	
	
	
	
	
\end{verbatim}






\chapter{Deelonderzoek naar veiligheidsrisico's en eisen voor sluizzen}

Gevonden weblinks in google op 07-04-2023 met zoekopdracht: "veiligheidsrisico's voor sluizen en waterwerken"
https://www.tweedekamer.nl/downloads/document?id=80443e97-f17e-499c-b3f2-ad608f32e1aa&title=Rapportage%20Staat%20van%20de%20infra%20RWS%20%28definitief%29.pdf
https://www.nu.nl/internet/5814282/rekenkamer-waterwerken-niet-goed-beveiligd-tegen-cyberaanvallen.html
https://www.deltalimburg.nl/article/9824/Onderhoudswerkzaamheden+aan+Sluis+Linne+afgerond
https://nieuwesluisterneuzen.eu/veiligheid
https://www.mrdmarinesupport.nl/nl/maritieme-dienstverlening/ondersteuning-veiligheid/
https://www.infrasite.nl/bouwen/2021/05/27/veiligheid-voorop-begin-project-sluis-of-brug-altijd-met-risicobeoordeling/
https://www.wdodelta.nl/bediening-schutsluizen-vechterweerd-en-vilsteren
https://www.infrasite.nl/waterbouw-deltas/2021/05/21/sluis-heel-onder-handen-genomen/
https://www.hdsr.nl/actueel/nieuws/@154100/lichtprojecties-zetten-waterliniesluizen/
https://nos.nl/artikel/2277937-rekenkamer-hack-aanval-op-waterwerk-niet-altijd-opgemerkt
https://magazine.vhbinfra.nl/1-4/zeesluis-ijmuiden/
https://varendoejesamen.nl/kenniscentrum/artikel/onderhoud-sluis-linne-afgerond
https://www.noorderzijlvest.nl/_flysystem/media/vragen-webinar-30-8-2021-en-inloop-2-9-2021-nieuwe-waterwerken-zoutkamp.pdf
https://eenvandaag.avrotros.nl/item/sluizen-en-gemalen-kunnen-eenvoudig-worden-gehackt/
https://www.gww-bouw.nl/artikel/de-eerste-sluis-met-kantelende-sluisdeur/
https://tkhsecurity.com/nl/waterwerken/
https://www.h2owaternetwerk.nl/h2o-actueel/rekenkamer-vitale-waterwerken-nog-onvoldoende-beschermd-tegen-cyberaanvallen
https://anteagroup.nl/diensten/beweegbare-en-vaste-kunstwerken/sluizen
https://balmbv.nl/sluizen-in-tiel-voor-decennia-beschermd/
https://www.volkskrant.nl/nieuws-achtergrond/hacker-dringt-door-in-controlekamer-waterwerk-cyberterrorist-kan-ons-land-onder-water-zetten~b6fcbc3c/?referrer=https%3A%2F%2Fwww.google.com%2F
https://www.magazinesrijkswaterstaat.nl/bereikbaarzeeland/2021/01/krammersluizencomplex-verleden-heden-en-toekomst
https://www.noord-holland.nl/Actueel/Archief/2022/September_2022/Hollandse_Waterlinies_in_de_schijnwerpers
https://www.watersport-tv.nl/nw-31400-7-3715235/nieuws/cybersecurity_vitale_waterwerken_niet_waterdicht.html
https://www.tonverheijen.nl/artikelen/bruggen-sluizen-en-tunnels-kunnen-ook-gehackt-worden
https://nuactueel.noordhoff.nl/hackers-zetten-zo-de-sluizen-open-hv-bb/
https://binnenvaartkrant.nl/waterwerken-zijn-kwetsbaar-voor-cyberaanvallen
https://www.rtlnieuws.nl/nieuws/nederland/artikel/3758966/cyberbeveiliging-waterschappen-hapert-sluizen-kunnen-worden
https://expert.rittal.nl/wp-content/uploads/2017/05/Referentieverhaal-Provincie-Zuid-Holland.pdf
https://www.vtmgroep.nl/blog/waterwerken-in-nederland-onvoldoende-beveiligd-tegen-cyberaanvallen
https://www.securitymanagement.nl/waterwerken-nog-altijd-te-hacken/
https://www.heijmans.nl/nl/verhalen/waterwerken/
https://www.tno.nl/nl/duurzaam/veilige-duurzame-leefomgeving/infrastructuur/natte-infrastructuur/
https://www.computable.nl/artikel/nieuws/security/6634379/250449/waterwerken-slecht-beveiligd-tegen-hackers.html
https://www.cobouw.nl/271317/tunnels-bruggen-en-sluizen-onvoldoende-beschermd-tegen-hackers
https://www.zuid-holland.nl/onderwerpen/verkeer-vervoer/vaarwegen/
http://www.wesemann.nl/nl/nieuws-en-pers/274-veiligheid-op-het-water-en-op-het-land.html
https://www.rekenkamer.nl/actueel/nieuws/2019/03/28/cybersecurity-vitale-waterwerken-niet-waterdicht
https://www.rijkswaterstaat.nl/over-ons/onze-organisatie/vervanging-en-renovatie
https://www.arbo-online.nl/22896/werkzaamheden-langs-het-water


\chapter{Deelonderzoek wet en regelgeving voor sluizen}



Gevonden weblinks in google op 07-04-2023 met zoekopdracht: "wet en regelgeving voor sluizen"
https://www.hdsr.nl/publish/pages/86927/sluizen_in_of_bij_een_waterkering_-_uitvoeringsregels.pdf
https://api1.ibabs.eu/publicdownload.aspx?site=sluis&id=100100292
https://services.pilz.nl/wp-content/uploads/2021/12/brochure_bruggen_2018.pdf
https://lokaleregelgeving.overheid.nl/CVDR375606/6
https://zoek.officielebekendmakingen.nl/stb-2019-27.html
https://a-quin.nl/nieuws/veiligheid-van-bruggen-sluizen-waarborgen-wie-wat-hoe/
https://www.gemeentesluis.nl/Bestuur_en_Organisatie/Wetten_Regels_Bekendmakingen
https://www.overijssel.nl/onderwerpen/verkeer-en-vervoer/varen-in-overijssel/informatie-bedieningstijden-sluizen-en-bruggen-noordwest-overijssel/
https://www.rijkswaterstaat.nl/water/wetten-regels-en-vergunningen
https://www.schuttevaer.nl/nieuws/actueel/2022/11/23/binnenvaart-zit-klem-tussen-regels-en-realiteit-kapotte-steigers-en-gesperde-sluizen-dwingen-tot-doorvaren/
https://repository.officiele-overheidspublicaties.nl/CVDR/CVDR271406/1/html/CVDR271406_1.html
https://www.zeeland.nl/actueel/bedieningstijden-sluizen-en-bruggen
https://www.amsterdam.nl/verkeer-vervoer/varen-amsterdam/regels-varen/
https://www.schielandendekrimpenerwaard.nl/wat-doen-we/regels-en-afspraken-over-beheer-keur-en-leggers/
http://www.wetboek-online.nl/wet/Wet%20tot%20samenvoeging%20van%20de%20gemeenten%20Aardenburg%20en%20Sluis.html
https://www.rijnland.net/regels-op-een-rij/richtlijnen-en-akkoorden/alle-regelgeving-van-rijnland/
https://www.itbb.nl/diensten/advies-ce-markering-europese-richtlijnen/
https://www.portofamsterdam.com/nl/scheepvaart/zeevaart/regelgeving
https://www.watersportverbond.nl/nieuws/achterstallig-onderhoud-wachtplaatsen-bruggen-en-sluizen-zuid-holland-zorgelijk/
https://varendoejesamen.nl/nieuws
https://www.flevoland.nl/wat-doen-we/flevowegen-vlot-en-veilig-door-flevoland/water/varen-in-flevoland/bediening-bruggen-en-sluizen
https://eur-lex.europa.eu/legal-content/NL/TXT/PDF/?uri=CELEX:32020L0012&from=DE
https://www.werkenvoornederland.nl/organisatie/rijkswaterstaat/ict-middelen-maken-om-bruggen-sluizen-en-tunnels-te-besturen
https://www.lobocom.nl/infra-bruggen-sluizen
https://waterrecreatienederland.nl/content/uploads/2018/04/richtlijnen-vaarwegen-2017.pdf
https://www.wetterskipfryslan.nl/melden-en-regelen/vergunningen-wetten-en-regels
https://www.onlinezeilschool.nl/sluizen/
https://www.provincie.drenthe.nl/onderwerpen/verkeer-vervoer/vaarwegen/rondje-drenthe/bedieningstijden/



\usepackage{pgf}
\newcommand\setform{\pgfqkeys{/form }}
\setform{field1/.store in=\fieldi,
	field2/.store in=\fieldii,
}


%\addtolength{\oddsidemargin}{-.875in}
%\addtolength{\evensidemargin}{-.875in}
%\addtolength{\textwidth}{1.75in}

%\addtolength{\topmargin}{-.875in}
%\addtolength{\textheight}{1.75in}


\newcommand\myform{%
	\fboxrule=0.4pt
	
	
	\fbox{\begin{minipage}{\textwidth}
			\fbox{\begin{minipage}[t][3cm][t]{0.25\textwidth}
					Naam vergadering
			\end{minipage}}%
			\fbox{\begin{minipage}[t][3cm][t]{0.25\textwidth}
					Datum en plats
			\end{minipage}}%
			\fbox{\begin{minipage}[t][3cm][t]{0.44\textwidth}
					Namen aanwezigen
			\end{minipage}}
			\fbox{\begin{minipage}[t][1cm][t]{0.98\textwidth}
					Opening en goedkeuring
			\end{minipage}}
			
	\end{minipage}}
	
	\fbox{\begin{minipage}{\textwidth}
			
			\fbox{\begin{minipage}[t][3cm][t]{0.98\textwidth}
					Ingekomen stukken en rondvraag
			\end{minipage}}
	\end{minipage}}
	
	\fbox{\begin{minipage}{\textwidth}
			\fbox{\begin{minipage}[t][10cm][t]{0.98\textwidth}
					Sluiting
			\end{minipage}}%
			
			
	\end{minipage}}
}


\newpage








\begin{center}  
	\begin{tabular}{ | l | l | l | p{5cm} |} % you can change the dimension according to the spacing requirements  
		\hline  
		\multicolumn{4}{|l|}{Actielijst} \\ \hline  
		Onderwerp & Besluit & Wie &Gereed \\ \hline  
		Orange & Fruit & Vitamin C & It is fruit, which is full of nutrients and low in calories. They can promote clear, healthy skin and also lowers the risk for many diseases. It reduces cholesterol and also helps in building a healthy immune system.\\ \hline  
		
		Cauliflower & vegetable & B-Vitamins & It is the vegetable, which is high in fiber and B-Vitamins. It also provides antioxidants, which help in fighting or protect against cancer. It enhances digestion and has many other nutrients.\\ \hline  
		
	\end{tabular}  
\end{center}  


\chapter{Queries}
\begin{verbatim}
	
	Queries
	Sluis.Draining-->Deuren.laag_open
	Deuren.laag_open-->Stoplicht.Green
	E<> (Ship.ship_can_move&&Stoplicht.Green)
	A[] not (Stoplicht.Green && not (Deuren.hoog_open||Deuren.laag_open||Deuren.stopgaplow1||Deuren.stopgaplow2||Deuren.stopgaphigh1||Deuren.stopgaphigh2))
	A[] not ((Deuren.hoog_open||Deuren.laag_open||Deuren.Opening_laag||Deuren.Opening_hoog||Deuren.Closing_hoog||Deuren.Closing_laag) && (Sluis.Draining||Sluis.Filling||Sluis.draining2||Sluis.Filling2))
	Sensor.Wait-->Sensor.Wait
	Stoplicht.Green-->Stoplicht.Green
	(Deuren.hoog_open||Deuren.laag_open)-->(Deuren.laag_open||Deuren.hoog_open)
	Deuren.laag_open-->Deuren.Closed
	Deuren.hoog_open-->Deuren.Closed
	Deuren.Closed-->Stoplicht.Red
	Ship.ship_can_move-->Deuren.Closed
	Deuren.hoog_open-->Stoplicht.Green
	Ship.ship_can_move-->Stoplicht.Green
	A[] not (Deuren.laag_open && Deuren.hoog_open)
	Ship.ship_can_move-->Ship.ship_can_move
	A[] not (Deuren.laag_open && Sluis.water != Sluis.water_laag)
	A[] not (Deuren.hoog_open && Sluis.water != Sluis.water_hoog)
	A[]not deadlock
	
	Project declaraties
	//Declarations
	
	chan boot_hoog;
	chan boot_laag;
	chan changedoor_low;
	chan changedoor_high;
	chan ship_moves;
	chan ship_abletomove;
	chan changelight;
	
	\\Sluis declaraties
	const int water_laag=0;
	const int water_hoog=10;
	const int water_median=(water_hoog+water_laag)/2;
	int[water_laag,water_hoog] water=water_median;
	clock x;
	\\Stoplicht declaraties
	
	\\Ship declaraties
	clock x;
	\\Sensor declaraties
	
	\\Deuren declaraties
	bool stoplicht_hoog=false;
	bool stoplicht_laag=false;
	clock x;
	
	
	\\System declaraties
	system Deuren,Sensor,Sluis,Ship,Stoplicht;
	
	
	Uitleg
	Als het schip boven is, dan is waterlvel gelijk aan hoog, filling valve is dicht, lower gates zijn gesloten, uppergates zijn open,empty valve is dicht. 
	Schip is in waterlock, waterlevel is hoog, filling valve is dicht, lower gates gesloten, upper gates gesloten, empty valve is open. 
	Schip is dan laag, waterlevel gelijk aan laag, filling valve is dicht, lowergates zijn open, uppergates zijn dicht, empty valve is dicht.
	AtArrivalHigh
	
	AtArrivalLow
	Als schip beneden is dan is waterlevel gelijk aan laag, filling valve is dicht, lower gates zijn open, upper gates zijn dicht, empty valve is open. 
	Schip is in water lock, waterlevel is laag, flilling valve is open, lower gates zijn gesloten, upper gates zij gesloten, empty valve is dicht,. 
	Schip is dan hoog, waterlevel is gelijk aan hoog, filling valve is dicht, uppergates zijn open, lowergates zijn dicht, filling valve is dicht
	
	
	
\end{verbatim}


\chapter{Testresultaten}



\subsubsection{onderdeleel van de test}



\begin{tabular}{*{15}{|l|l|l|l|l|l|l|}} \hline
	\multicolumn{7}{|l|}{project name}                                                               \\ \hline
	\multicolumn{4}{|l|}{Test case ID}   &\multicolumn{3}{|l|}{Test designed by}                           \\ \hline
	\multicolumn{4}{|l|}{test priority (low/medium/high)}   &\multicolumn{3}{|l|}{Test design date}                           \\ \hline
	\multicolumn{4}{|l|}{Module name}   &\multicolumn{3}{|l|}{Test executed by}                           \\ \hline
	\multicolumn{4}{|l|}{Test title}   &\multicolumn{3}{|l|}{Test execution date}                           \\ \hline
	\multicolumn{4}{|l|}{Description}   &\multicolumn{3}{|l|}{ }                           \\ \hline 		
	\multicolumn{7}{|l|}{ }   																\\ \hline
	\multicolumn{7}{|l|}{Pre condition}                                                               \\ \hline
	\multicolumn{7}{|l|}{Dependencies}                                                               \\ \hline
	\multicolumn{7}{|l|}{ }   															\\ \hline
	Step  &  Test steps & Test data & expected result &Acual result &(pass or fail)&notes  \\ \hline
	
\end{tabular}





\chapter{Reflectie}

\section{Inleiding}

Ik heb erg veel geleerd van het analyseren van de vershillende requirements en specificaties en het opzetten van een model in Uppaal. Een dergelijk model opzetten had ik namelijk nog nooit gedaan. Het uitvoeren van onderzoek heb ik eerder gedaan. Ook de toetsing van het model met behulp van proposities heb ik nog nooit gedaan. Verder heb ik de kennis die had van programmeren/ design pattersn gebruikt om de verschillende templates in mijn Uppaal model van elkaar te onderscheiden. Het leukste onderdeel van het project vond ik hoe mijn templatemodel deadlockvrij werkte. Voor de verificatie van het model heb ik veel achtergrondinformatie opgezet, en het is mooi om te zien dat je met enkele duidelijke zinnen kan aantonen of een propositie geldig is of niet.  Verder had ik moeite met het opstellen van de juiste veiligheidseisen bij het model. Ik had aangenomen dat ik het project niet zou halen omdat ik de opdracht niet in teamverband heb uitgevoerd. Ik ben toch blij dat ik een concept heb opgeleverd dat ik kan toetsen aan de doormijzef opgestelde eisen en dat ik met mijn huidige kennis de proposities uit de requirements kan toetsen.


\section{Onderzoeksresultaten}




\subsubsection{bijlmerramp}
Motor 3 (de binnenste motor aan de rechtervleugel van het vliegtuig) brak af, beschadigde de vleugelkleppen en botste tegen motor 4 die vervolgens ook afbrak.
De ernst van de situatie werd op Schiphol niet goed ingezien. Dit kwam onder meer doordat lost in de luchtvaart de gebruikelijke term is om het verlies van motorvermogen te melden. Op Schiphol werd er dan ook van uitgegaan dat er twee motoren waren uitgevallen. Dat ze letterlijk verloren waren wist men niet. Gezien het grote aantal handelingen dat de bemanning in een paar minuten moest uitvoeren en de keuzes die de piloot maakte, veronderstelde de parlementaire enquêtecommissie die de ramp later zou onderzoeken dat ook de bemanning waarschijnlijk niet heeft geweten dat beide motoren van de rechtervleugel waren afgebroken. De buitenste motor van een 747 is vanuit de cockpit slechts met moeite zichtbaar en de binnenste motor helemaal niet.

Op de avond van de 4e oktober 1992 was landingsbaan 06 (de Kaagbaan) in gebruik. De piloot verzocht de luchtverkeersleiding op Schiphol echter een noodlanding te mogen maken op de Buitenveldertbaan (baan 27). Waarom hij juist deze baan koos, is nooit duidelijk geworden. Een keuze voor deze baan lag niet voor de hand; omdat de wind uit het noordoosten kwam, zou het toestel met flinke staartwind moeten landen. Langs de landingsbaan waren enkele grote brandweerwagens van Schiphol geplaatst. Deze zogeheten crashtenders moesten een brand tijdens de landing meteen blussen. Na de crash werd één zwarte doos teruggevonden. De bijbehorende band was in vier stukken gebroken, waardoor de laatste 2 minuten en 45 seconden ervan niet meer te gebruiken waren. De doos werd voor onderzoek naar Washington gestuurd en leverde uiteindelijk onderstaande informatie op.
Om goed uit te komen voor de landingsbaan vloog het beschadigde toestel eerst nog een rondje boven Amsterdam. Tijdens dit rondje gaf de gezagvoerder de copiloot opdracht de vleugelkleppen (flaps) uit te schuiven. Links schoven de kleppen uit, maar doordat de afgebroken motor 3 de rechtervleugel had beschadigd schoven de kleppen op die vleugel niet uit. Als gevolg hiervan kreeg het toestel links meer draagvermogen dan rechts. De piloot meldde aan de verkeersleiding dat er ook problemen met de flaps waren.
Aanvankelijk ging het aanvliegen van de Buitenveldertbaan goed. Op het moment dat het vliegtuig daalde tot onder de 1500 voet en snelheid minderde, raakte het echter compleet onbestuurbaar en maakte het een ongecontroleerde, scherpe bocht naar rechts. Over de radio was te horen dat de gezagvoerder zijn copiloot in het Hebreeuws opdracht gaf om alle kleppen in te trekken en het landingsgestel uit te klappen. Vervolgens meldde de copiloot in het Engels aan de luchtverkeersleider dat het toestel zou gaan neerstorten. Uit later onderzoek bleek dat het vliegtuig eerder enkel recht bleef vanwege de hoge snelheid (280 knopen, zijnde 519 km/u). Doordat de rechtervleugel beschadigd was, was het moeilijker om het vliegtuig recht te houden. Alleen de hoge snelheid zorgde ervoor dat er nog voldoende draagvermogen was. Toen bij het inzetten van de landing de snelheid verlaagd werd, werd het draagvermogen van de rechtervleugel echter dusdanig gering dat het toestel niet meer onder controle te houden was en een duikvlucht naar rechts maakte.

https://aviation-safety.net/database/record.php?id=19921004-2&lang=nl 
\subsubsection{vuurwerkramp in enschede }
https://www.enschede.nl/inhoud/commissie-oosting 
https://www.politie.nl/binaries/content/assets/politie/wob/00-landelijk/vuurwerkramp-enschede/bijlagen-rapport-vuurwerkramp-enschede.pdf 
https://www.researchgate.net/publication/254815008_Rampen_regels_richtlijnen 


\subsubsection{ramp turkisch airlines}
Inadequaat handelen van de piloten ondanks een defecte hoogtemeter en onvolledige instructies van de luchtverkeersleiding/
https://catsr.vse.gmu.edu/SYST460/TA1951_AccidentReport.pdf 

Wat ging er allemaal mis bij de bovengenoemde rampen en ongelukken....... 

Wat hebben deze rampten te maken met de requirements en specificaties van deze odpracht? 



\subsubsection{tjernobyl}
Een ramp bij een kernreacor in de sovjetunie. Door een bedieningsfout in een testprocedure werd het vermogen van de koelinstallaties negatief beinvloed. Door een ontwerpfout in de noodstopprocedure kon in het systeem niet snel genoeg schakelen om remmende invloed uit te oefenen op het toenemende vermogen van de reactorkernen. Met brand en eksplosie tot gevolg.
https://www-pub.iaea.org/MTCD/publications/PDF/Pub913e_web.pdf 



\subsubsection{therac-25}
Softwarefout uit zich als hardwarefout de klachtafhandeling geen onderzoek geen second opinion is prioriteit wel 
gechecked na onderzoek bellen en geen prioriteit aanwezig te zijn alleen importeurs en fabriken mogen fouten 
in frabrieksinstellingen rapporteren 
Therac25 Systeem ligt plat veel voorkomende eror stdaardafhandeling om de error te verwerpen resultaat: 
de patient kreeg overdosis patient overleden onderzoek opgestart, stuatie niet reproduceerbar foutmarkering: 
gezien als uitzonderlijk, software aanpassing van groote magnitude 5; de oorzaak was waarschijlijk mechanisch 
maar neit vastgesteld; conceptueel odel niet aangepast probleemclassicificatie door autorititen het probleem 
en de impact daarvan anar beneden bijgesteld AEFL doe gedeeltelijke aanpassing om hardware na berisping 
Canadese autoriteit 
Derde patient overleden door eythema AECL wijst alle doodsoorzaken af AECL beweert dat geen vergeli- 
jkbare voorvalle bij andere machines of patienten zijn voorgekomen geen vervolgonderzoek vanwege garanties 
bedrijf gaat uit van geen mogelijke functionele fout 
vierde patient overleden aan overdodis ontstaan door bug in software onjuiste aanduiding bij de foutmelding 
verkeerde reactie/invoer ddoor operator communicatie tussen patient en operator werd onvoldoende gemon- 
itorred ( apparatuur niet aangesloten, en audio monitor kapot) engineer van AECL stelt geen fouten vast 
Engineer AECl kan fout niet reproduceren Geen communicate tussen bedrijf en uitgezonden technisci over 
vergelijkbare probleemgevallen 
vijfde geval malfunction 54 leidt tot overdosis en de dood fout gereproduceerd door operator bedrijf fout 
was daa entryspeed herpublicatie van de ongevallen en de eerdere ongevallen in de meia apparaat wel nog in 
gebruik genomen niet handig, waarschuwingsberichten en aanwijzingen voor een bugfix naar de gebruikers door 
druk van fda is bedrijf op zoek gegaan naar permanente oplossing 
zesde geval software fout door softwarefout otntstaat lightstruct .. op de patient na onderzoek door AECL 
blijkt niet alleen hardware de oorzak gebruikers direct geinformeerd oplossing gevonden, media ingeschakeld om 

transparantie af te dwingen door de gebruikersgroep en de FDA AECL gedwongen functionaliteit aan te passen 
Engineers hebben meer studie moeten maken van gebruikte technologie en onderhoudbaarheid daarvan 



\section{tesla crash report}
Door een softwarefout zijn er situaties ontstaan waarin het systeem informatie een onvoldoende informatie positie had om de juiste beslissingen te maken. Of dat de informatieverwerking niet juist was.




\subsubsection{stint ongeluk}
Vier kinderen, een bestuurder kwamen om en een vijfde persoon , een kind raakte zwaargewond. Uit odnerzoek van bleek :
Foute torsieveer voor de gashendel werd geleverd
Geen van de drie onderzochte voertuigen haalden de wettelijk vereiste remvertraging
De automatische parkeerrem kan leiden tot gevaarlijke situaties wanneer deze ongewenst geactiveerd wordt tijdens het rijden. 
Het losraken van de nuldraad naar de gashendel leidt volgens TNO tot ongewenst versnellen van het voertuig en een oncontroleerbare situatie voor de bestuurder.
Voor alle drie onderzochte voertuigen geldt dat het ontbreken van een zitplaats leidt tot veiligheidsrisico’s voor remmen en sturen door de grotere kans dat de bestuurder van het voertuig valt. Als de bestuurder van een Stint valt, leidt dit in alle rijsituaties tot een onbeheersbare situatie


https://repository.tno.nl/islandora/object/uuid%3Acdef48df-da49-46b6-8678-5c62a88a0090 




\subsubsection{slmramp}
Toen de Anthony Nesty Zanderij naderde, was het daar, anders dan het weerbericht had voorspeld, mistig. Het zicht was evenwel niet zo slecht dat er niet op zicht kon worden geland. Gezagvoerder Will Rogers besloot echter via het Instrument Landing System (ILS) te landen, hoewel dit niet betrouwbaar was en hij voor zo'n landing ook geen toestemming had. De gezagvoerder brak drie landingspogingen af. Bij de vierde poging negeerde de bemanning de automatische waarschuwing (GPWS) dat het toestel te laag vloog. Het toestel raakte op 25 meter hoogte twee bomen. Het rolde om de lengteas en stortte om 04.27 uur plaatselijke tijd ondersteboven neer.

Uit onderzoek bleek dat de papieren van de bemanning niet in orde waren. 
Geconcludeerd werd dat de gezagvoerder roekeloos had gehandeld door voor een ILS-landing te kiezen terwijl hij daar geen toestemming voor had, en door onvoldoende op de vlieghoogte te hebben gelet. 
De SLM werd verweten de kwalificaties van de bemanning onvoldoende te hebben gecontroleerd.

https://aviation-safety.net/investigation/cvr/transcripts/cvr_py764.php 
https://aviation-safety.net/database/record.php?id=19890607-2 



\subsubsection{schipholbrand}
Om een goed verhaal op te stellen, moet vooraf aan enkele voorwaarden
worden voldaan. De eerste voorwaarde is de geschiktheid van het
afstudeerproject. Als een afstudeerproject niet tot keuzes leidt, kan
men zich afvragen of dat wel een echte afstudeeropdracht is. Een
afstudeerproject zonder onderzoeksaspecten is ook verdacht. Daarnaast
moet een afstudeerproject passen in het profiel van een opleiding om
beoordeelbaar te zijn. De andere voorwaarde voor goed een verhaal is
de registratie van werkzaamheden tijdens het a


\subsubsection{Ramp schietpartij militair ossendrecht }
Een militaire overleid op een schietbaan in ossendracht door onvoldoende begeleiding van cursisten, geen toezicht op de lokatie. E\r was een instructuur in opleiding die niet volledig was mmeegenomen in het poroces en ook was er geen baancommandant aanwezig. Geen van de aanwezig instructeurts had de juiste papieren om de cursisten te begeleiden. De aanwezig instruceur had geen zich op de instructeur in opleiding, evenmin de andere militairen. In de instructiehandleiding ontbreken richtlijnen voor bijzondere schietbanen. Ook was er geen keuring. Door personelstekort is er geen andacht besteed aan documentastie(een slyllabus) hoe en met welke risico’s oefeningnen moeten worden ingericht. Ok werd er vooraf geen veiliheidsanaklyse gedaan. Het gebrek aan lesmateriaal en deskundigen is gemeld binnen de defensieorganisatie maar dit heeft niet geleid tot enige verandering in de situatie.
Op een afgekeurde scheitbaan
Tezicht door een instructeur in opleiding die zelf geen persoonlijke begeleiding heeft gehad tijdens de uitvoering
Belangrijk is dat defensie haar taken kan uitvoeren met personeel dat is getraind in situaties die de risicos van de werkomgeving aan de cursisten kunnen laten zien.
Conclusie
Zonder gekwalificeerde instructuers.
Zonder toezicht
Zonder lesmateriaal
Zonder adequate veiligheidsanalyse
https://www.youtube.com/watch?v=6jmkDClGDHo 




\subsubsection{molukse treinkaping }
https://www.youtube.com/watch?v=h99Fe9XzzHI 

\subsubsection{explosie tanjin china 
}

Later bleek uit een onderzoek van de Chinese autoriteiten dat de explosie overeenkwam met de ontploffing van 450 ton TNT.[6] 
De oorzaak van de explosie lag in de spontane zelfontbranding van 207 ton cellulosenitraat dat in containers was opgeslagen op het terminalterrein.[6] 
Verder lag op een tweede locatie nog eens 26 ton van dit explosieve materiaal opgeslagen.
De tweede ontploffing werd versterkt door de opslag van 800 ton kunstmest in de vorm van ammoniumnitraat in de nabijheid.[6]
De opslag van cellulosenitraat is aan strenge regels gebonden. Het moet koel en droog worden opgeslagen. De containers stonden buiten opgesteld in de brandende zon. De temperatuur liep op tot 36 °C en bereikte binnen de containers waarschijnlijk de 65 °C.[6] De verpakking van de cellulosenitraat droogde uit waardoor de ontploffing kon ontstaan. Op het terrein lagen meer gevaarlijke stoffen opgeslagen dan waarvoor vergunningen waren verstrekt.[6] Dit leidde tot een kettingreactie met grote schade tot gevolg. Door de brand en bluswater is in de directe omgeving veel milieuschade opgetreden.


https://www.hindawi.com/journals/joph/2019/1360805/ 



\subsubsection{explosie in libabon, beirut 
}

Op 23 september 2013 voer het vrachtschip de Rhosus onder Moldavische vlag[7] van Batoemi in Georgië naar Beira in Mozambique met 2.750 ton ammoniumnitraat

Gezien het ernstige gevaar van het bewaren van deze goederen in de hangar onder ongeschikte klimatologische omstandigheden, herhalen we ons verzoek aan de marine-instantie om deze goederen onmiddellijk weer te exporteren om de veiligheid van de haven en de mensen die er werken te verzekeren, of om akkoord te gaan om ze te verkopen.
Voorafgaand aan de explosie was er een brand in een opslagplaats. 

https://www.hrw.org/report/2021/08/03/they-killed-us-inside/investigation-august-4-beirut-blast 
https://www.researchgate.net/publication/348325979_Beirut_Explosion_the_full_story 
https://reliefweb.int/sites/reliefweb.int/files/resources/CaseStudy_BeirutExplosion_TechBioHazardsweb.pdf 


\subsubsection{ethiopian airlines}
Ethiopian Airlines Flight 302
Door problemen met de flight control
One minute into the flight, the first officer, acting on the instructions of the captain, reported a "flight control" problem to the control tower.
Two minutes into the flight, the plane's MCAS system activated, pitching the plane into a dive toward the ground. The pilots struggled to control it and managed to prevent the nose from diving further, but the plane continued to lose altitude.
The MCAS then activated again, dropping the nose even further down. The pilots then flipped a pair of switches to disable the electrical trim tab system, which also disabled the MCAS software. However, in shutting off the electrical trim system, they also shut off their ability to trim the stabilizer into a neutral position with the electrical switch located on their yokes. The only other possible way to move the stabilizer would be by cranking the wheel by hand, but because the stabilizer was located opposite to the elevator, strong aerodynamic forces were pushing on it.
As the pilots had inadvertently left the engines on full takeoff power, which caused the plane to accelerate at high speed, there was further pressure on the stabilizer. The pilots' attempts to manually crank the stabilizer back into position failed.
Three minutes into the flight, with the aircraft continuing to lose altitude and accelerating beyond its safety limits, the captain instructed the first officer to request permission from air traffic control to return to the airport. Permission was granted, and the air traffic controllers diverted other approaching flights. Following instructions from air traffic control, they turned the aircraft to the east, and it rolled to the right. The right wing came to point down as the turn steepened.
At 8:43, having struggled to keep the plane's nose from diving further by manually pulling the yoke, the captain asked the first officer to help him, and turned the electrical trim tab system back on in the hope that it would allow him to put the stabilizer back into neutral trim. However, in turning the trim system back on, he also reactivated the MCAS system, which pushed the nose further down. The captain and first officer attempted to raise the nose by manually pulling their yokes, but the aircraft continued to plunge toward the ground.

https://www.hindawi.com/journals/ijae/2014/472395/ 


\subsection{ethiek}


Ethiek 



persuasive technology 
https://www.humanetech.com/youth/persuasive-technology 
https://www.minddistrict.com/blog/persuasive-technology-new-insights-in-behavioural-change 
https://www.sciencedirect.com/book/9781558606432/persuasive-technology 
https://spectrum.ieee.org/how-persuasive-technology-can-change-your-habits 
https://www.frontiersin.org/articles/10.3389/frai.2020.00007/full 
https://psmag.com/environment/captology-fogg-invisible-manipulative-power-persuasive-technology-81301 
https://www.makeuseof.com/what-is-persuasive-technology/ 
https://lib.ugent.be/catalog/rug01:001235489 
https://cyberpsychology.eu/article/view/12270 



\section{ecourt in nederlandse rechtspraak}
niet odnerzocht
https://www.njb.nl/blogs/a-court-with-no-face-and-no-place/ 
http://www.e-court.nl/wp-content/uploads/2018/03/Procesreglement-e-Court-2017_20180201.pdf

\subsubsection{ cyber aanval op Oekraïene }
Om een goed verhaal op te stellen, moet vooraf aan enkele voorwaarden
worden voldaan. De eerste voorwaarde is de geschiktheid van het
afstudeerproject. Als een afstudeerproject niet tot keuzes leidt, kan
men zich afvragen of dat wel een echte afstudeeropdracht is. Een
afstudeerproject zonder onderzoeksaspecten is ook verdacht. Daarnaast
moet een afstudeerproject passen in het profiel van een opleiding om
beoordeelbaar te zijn. De andere voorwaarde voor goed een verhaal is
de registratie van werkzaamheden tijdens het a


\subsubsection{Mali}
Een granaat explodeerd in een mortier
De medische zorg na het ongeval was neit voldoende


De algemeen militair verpleegkundige gaf aan het slachtoffer nar het vn-hospitaal in kidal te brengen
De chaauffeur van de bushmaster kende de locatie niet  en bracht het slachtoffer naar een door frane militairren bemand hospitaal mmet minder mediswche faciliteiten
Hierna alsnog overgebracht naar het vn-hospitaal.
Dit verlieop  neit door nederlandse maatstaven.
pas toen een nederlandse arts arrivveerde werd door de Tongolese artsen een buikoperatie uitgevoerd.
Dit gebrurde zonder adequate anesthesie.
Na de operatie werde de gewonde militair overgelogen naar nederland. En later naar nederland.


granaat stond niet op scherp en in afgegaan in veilige stand
Granaat werd opgeslagen in neit gekoelde containers waardoor deze aan te hoge temeperaturen zijn blootgesteld.
Door de comvinatie van vocht en warmte in de granaat zeer gevoelige explosieve stoffen werden gevormd.
Tijdens de oefening was de fatale granaat in de zon.
Het afsluitplaatje in de granaat bleek niet in staat om doorslag in veilige stand te voorkomen waarna de granaat explodeerde.
De moritren zijn aangeschaft bij de amerikanen. gredurende de aanschafperiode zijn procedures en controles op kwaliteit en veiligheid deels nagelaten.
Dit veiligheidsgarantie werd vermeld in het koopcontract.
Conclusie
Koopcontract werd niet goed doorgelezen
Geen controle op kwaliteit en veiligheid
Geen controle op kwaliteit en veiligheid
Zwakke plekken in het ontwerp
Geen controle op kwaliteit en veiligheid
opslag en gebruik in ongunstige condities

De aanwezige medische voorzieningen waren nite volgends de nederlandse militaire richtlijnen
Het ontbreek aan medische toetsing vanuit de defensie organisatie
twijfels die werden geuit binnen de defensieorganisae vonden geen wrrklank
Ok het ongeval tijdens de mortieroefening was voor defensie geen aanleuiding om de medische voorzienignen te evalueren.
De inrichting van veilige medische zorg voor nederlandse militairen in kidal is ondergeschikt gemaakt aan de voortgang van de missie.


https://www.youtube.com/watch?v=PC2ekl4SaNA 

\section{Modelleren}


\section{Ethiek}


\section{Conclusie}

 
\chapter{Security}

\section{Inleiding}

\section{Resultaat}


De cyber aanval op Oekraïene
Samenvatting
probleemstelling
aanleiding
theoretisch kader
methode
resultaten/argumenten
conclusie

Inleiding

Aanleiding
Doel van dit verslag is inzage te geven in de informatieverzameling en begrip van een complex scada
systeem. De lezer krijgt inzage in de achtergrond achter cyberaanval, de gebruikte technieken en
een opsomming van oplossingen en methoden voor beveiligingsvraagstukken.

Doelstelling
Doe onderzoek naar een real-life voorbeeld van een aanval op een ICS,
vooral kijkend naar de technische details.
Zorg dat je verslag bevat hoe de aanval voorkomen had kunnen worden en
hoe het opgelost is.

probleemomschrijving
probleemstelling/onderzoeksvraag Wat zijn aanvallen op ICS? Hoe kunnen deze worden voorkomen
en worden opgelost?
hoofdvraag
Welke beveiligingstechnieken en werkwijzen kunnen ICS aanvallen voorkomen?
deelvragen
Wat is veiligheid?
Wat zijn de technieken en werkwijzen die worden gebruikt?
Wat zijn de kwetsbaarheden?
Hoe wordt de defensie tegen de aanvallen gerealiseerd?

Wat zijn de veronderstellingen
Wat zijn de hoofdzaken en bijzaken
Wat zijn relevante en irrelevante aspecten
Welke redelijke criteria zijn er om te bepalen of iets een hackaanval is
Welke redelijke criteria zijn er voor het beoordelen van de juiste maatregel
Welke bewijzen of tegenvoorbeelden
Wat zijn de alternatieven

Wat zijn de meningen
Welke vragen kunnen gesteld worden op basis van beweringen en conclusies, definities en bewijzen,
meningen en overtuigingn
Welke tegenstellingen zijn waargenomen
Welke standpunten opvattingen en overtuigingen kunnen worden uitgelegd, verdedigd of aangepast
Zijn alle beweringen zorgvuldig en weloverwogen aanvaard of verworpen?
Welke kennis moet worden verworven om tot een oordeel te komen?



Situatie oekraiene
Wat is er gebeurd
Waar is het gebeurd
Waarom is het gebeurd
Hoe is het gebeurd
Wanneer is het gebeurd
Welke beveiliging
Waarom is de beveiliging zo geregeld
Waarom zijn maatregelen niet eerder genomen
Waarom is het probleem in deze regio
Waarom speelt het bedrijf bij deze bedrijven



Waarom wordt er niet meer tijd, kennis, research en ontwikkeling besteed
Wie hebben er last van
Wat is de oplossing
Met welke technologie werken bedrijven aan een oplossing
Welke oplossing worden gegeven in de literatuur
Waarom werken deze oplossingen uit de literatuur en praktijk niet?
afbakening
voorlopige ooraken en gevolg
randvoorwaarden
opzet van het artikel
methode hazard risk assessment




https://ics.sans.org/media/E-ISAC_SANS_Ukraine_DUC_5.pdf
https://na.eventscloud.com/file_uploads/aed4bc20e84d2839b83c18bcba7e2876_Owens1.pdf
https://www.wired.com/2016/03/inside-cunning-unprecedented-hack-ukraines-power-grid/

http://web.mit.edu/smadnick/www/wp/2016-22.pdf

https://en.wikipedia.org/wiki/December_2015_Ukraine_power_grid_cyberattack

https://www.wired.com/story/russian-hackers-attack-ukraine/

https://www.linkedin.com/notifications/

https://www.boozallen.com/content/dam/boozallen/documents/2016/09/ukraine-report-when-the-
lights-went-out.pdf

https://www.reuters.com/article/us-ukraine-cybersecurity-sandworm-idUSKBN0UM00N20160108

https://www.wired.com/2016/01/everything-we-know-about-ukraines-power-plant-hack/

https://www.fireeye.com/blog/threat-research/2016/01/ukraine-and-sandworm-team.html
https://www.us-cert.gov/ics/alerts/IR-ALERT-H-16-056-01
https://www.sans.org/blog/confirmation-of-a-coordinated-attack-on-the-ukrainian-power-grid/
https://www.reuters.com/article/us-ukraine-cybersecurity-sandworm/u-s-firm-blames-russian-
sandworm-hackers-for-ukraine-outage-idUSKBN0UM00N20160108
https://www.reuters.com/article/us-ukraine-crisis-cyber-idUSKBN15U2CN
https://www.wired.com/2014/10/russian-sandworm-hack-isight/
https://blog.trendmicro.com/trendlabs-security-intelligence/sandworm-to-blacken-the-scada-
connection/
https://blog.trendmicro.com/trendlabs-security-intelligence/killdisk-and-blackenergy-are-not-just-
energy-sector-threats/
https://www.nerc.com/pa/CI/ESISAC/Documents/E-ISAC_SANS_Ukraine_DUC_18Mar2016.pdf
https://www.politico.eu/article/ukraine-cyber-war-frontline-russia-malware-attacks/
https://theconversation.com/cyberattack-on-ukraine-grid-heres-how-it-worked-and-perhaps-why-it-
was-done-52802
https://www.ifri.org/sites/default/files/atoms/files/desarnaud_cyber_attacks_energy_infrastructure
s_2017_2.pdf
https://ec.europa.eu/energy/sites/ener/files/evaluation_of_risks_of_cyber-
incidents_and_on_costs_of_preventing_cyber-incidents_in_the_energy_sector.pdf
https://ec.europa.eu/energy/sites/ener/files/evaluation_of_risks_of_cyber-
incidents_and_on_costs_of_preventing_cyber-incidents_in_the_energy_sector.pdf
https://www.wired.com/2016/03/inside-cunning-unprecedented-hack-ukraines-power-grid/
https://www.osti.gov/servlets/purl/1505628
https://jsis.washington.edu/news/cyberattack-critical-infrastructure-russia-ukrainian-power-grid-
attacks/
IEC_60870-5
http://citeseerx.ist.psu.edu/viewdoc/download;jsessionid=0513EED48102FDAD1BD940260EF12B11?
doi=10.1.1.548.7490&amp;rep=rep1&amp;type=pdf
https://scialert.net/fulltext/?doi=tasr.2014.396.405
https://www.researchgate.net/publication/333671061_Attacking_IEC-60870-5-104_SCADA_Systems
http://www.salvage-project.com/uploads/4/9/5/5/49558369/art3_-_salvage_2015_-
_cyber_security_in_communication_of_scada_systems_using_iec_61850.pdf
https://eg.uc.pt/bitstream/10316/35720/1/Security%20Probes%20for%20Industrial%20Control%20
Networks.pdf
https://owlcyberdefense.com/products/data-diode-products/software-modules/iec-104/
https://ris.utwente.nl/ws/files/6028066/3-s2_0-B9780128015957000227.pdf
https://waterfall-security.com/static/Waterfall-for-IEC-60870-5-104_FINAL.pdf
http://www.scada.sl/2013/11/last-week-four-guys-of-scada.html
https://repositorio-aberto.up.pt/bitstream/10216/119066/2/315683.pdf
https://www.diva-portal.org/smash/get/diva2:1046339/FULLTEXT01.pdf
https://www.semanticscholar.org/paper/Cybersecurity-analysis-of-a-SCADA-system-under-and-
Rocha/dfa7c12551ebe7b24da8d806e87e946051a57cb9
https://control.com/forums/threads/comparison-between-iec60870-5-103-and-modbus-rtu.20317/
https://www.blackhat.com/docs/us-17/wednesday/us-17-Staggs-Adventures-In-Attacking-Wind-
Farm-Control-Networks.pdf
https://www.securonix.com/web/wp-
content/uploads/2019/08/RSAC_2019_Scada_Attack_Detection_101.pdf
https://dreamlab.net/en/blog/post/fuzzing-ics-protocols/
https://library.e.abb.com/public/f74f9c8be95f4cd09b0b1fbbde699108/4CAE000416_RTU_Secure%2
0communications_web.pdf
https://www.slideshare.net/qqlan/scada-zn
https://virsec.com/virsec-hack-analysis-deep-dive-into-industroyer-aka-crash-override/
https://tutcris.tut.fi/portal/files/16294332/jafary_1534.pdf
http://www.connectivity4ir.co.uk/article/175490/IEC-62351--Secure-communication-in-the-energy-
industry.aspx
https://www.win.tue.nl/~setalle/2017_fauri_encryption.pdf
https://assets.barracuda.com/assets/docs/dms/Barracuda_CloudGen_Firewall_SB_Security_for_ICS_
and_OT.pdf
https://www.checkpoint.com/products/industrial-control-systems-appliances/
https://www.dragos.com/wp-content/uploads/CRASHOVERRIDE.pdf
https://dl.acm.org/doi/fullHtml/10.1145/3381038
https://arxiv.org/pdf/2001.02925.pdf
http://blog.nettedautomation.com/2017/
CRASHOVERRIDE

https://www.dragos.com/wp-content/uploads/CrashOverride-01.pdf
https://www.welivesecurity.com/wp-content/uploads/2017/06/Win32_Industroyer.pdf
https://ec.europa.eu/energy/sites/ener/files/evaluation_of_risks_of_cyber-
incidents_and_on_costs_of_preventing_cyber-incidents_in_the_energy_sector.pdf
https://www.cybersecurityintelligence.com/blog/attack-on-ukraines-power-grid-targeted-
transmission-stations-4530.html
https://www.recordedfuture.com/crashoverride-malware-overview/
https://www.us-cert.gov/ncas/alerts/TA17-163A
https://www.darkreading.com/threat-intelligence/first-malware-designed-solely-for-electric-grids-
caused-2016-ukraine-outage/d/d-id/1329114
https://arstechnica.com/information-technology/2017/06/crash-override-malware-may-sabotage-
electric-grids-but-its-no-stuxnet/
https://www.accenture.com/_acnmedia/pdf-69/accenture-managing-malware-crash-override.pdf
https://www.nixu.com/fi/node/53
https://www.vice.com/en_us/article/zmeyg8/ukraine-power-grid-malware-crashoverride-
industroyer
https://ics.sans.org/media/E-ISAC_SANS_Ukraine_DUC_6.pdf
http://blog.wallix.com/ics-security-russian-hacking
http://web.mit.edu/smadnick/www/wp/2016-22.pdf
https://www.boozallen.com/content/dam/boozallen/documents/2016/09/ukraine-report-when-the-
lights-went-out.pdf
https://www.reuters.com/article/us-ukraine-cybersecurity-sandworm-idUSKBN0UM00N20160108
https://www.nerc.com/pa/CI/ESISAC/Documents/E-ISAC_SANS_Ukraine_DUC_18Mar2016.pdf
https://jsis.washington.edu/news/cyberattack-critical-infrastructure-russia-ukrainian-power-grid-
attacks/
https://www.ifri.org/sites/default/files/atoms/files/desarnaud_cyber_attacks_energy_infrastructure
s_2017_2.pdf
https://blog.trendmicro.com/trendlabs-security-intelligence/sandworm-to-blacken-the-scada-
connection/
https://ec.europa.eu/energy/sites/ener/files/evaluation_of_risks_of_cyber-
incidents_and_on_costs_of_preventing_cyber-incidents_in_the_energy_sector.pdf
https://www.wired.com/2016/03/inside-cunning-unprecedented-hack-ukraines-power-grid/
https://ics.sans.org/media/E-ISAC_SANS_Ukraine_DUC_5.pdf

https://ics.sans.org/media/E-ISAC_SANS_Ukraine_DUC_5.pdf
https://digitalsupport.ge.com/servlet/fileField?retURL=%2Fapex%2FKnowledgeDetail%3Fid%3DkA21
A000000HShPSAW%26lang%3Den_US%26Type%3DArticle__kav&amp;entityId=ka21A000000HccQQAS&amp;fi
eld=File_1__Body__s
https://www.dragos.com/wp-content/uploads/CrashOverride-01.pdf

https://www.dragos.com/resource/crashoverride-analyzing-the-malware-that-attacks-power-grids/
https://www.recordedfuture.com/crashoverride-malware-overview/
https://www.us-cert.gov/ncas/alerts/TA17-163A
https://en.wikipedia.org/wiki/Industroyer
https://en.wikipedia.org/wiki/Crash_Override_Network
https://www.virusbulletin.com/virusbulletin/2019/03/vb2018-paper-anatomy-attack-detecting-and-
defeating-crashoverride/
https://www.nixu.com/blog/crashoverride-threat-electricity-networks
https://www.youtube.com/watch?v=TH17hSH1PGQ
https://www.cyber.nj.gov/threat-center/threat-profiles/ics-malware-variants/crashoverride
https://thecyberwire.com/stories/f0c289436fff6d87ec93227b77dd3d88/crashoverride-its-
aftermath-and-its-implications
https://www.webopedia.com/TERM/C/crashoverride-industroyer-malware.html
https://collaborate.mitre.org/attackics/index.php/Software/S0001
https://www.accenture.com/us-en/blogs/blogs-crashoverride-malware-alert
https://www.incibe-cert.es/en/blog/crashoverride-malware-ics-back-again
https://rhebo.com/en/service/glossar/industroyer-25114/
https://www.csoonline.com/article/3200828/crash-override-malware-that-took-down-a-power-grid-
may-have-been-a-test-run.html
https://www.darkreading.com/threat-intelligence/first-malware-designed-solely-for-electric-grids-
caused-2016-ukraine-outage/d/d-id/1329114
https://www.belden.com/blog/industrial-security/crashoverride-first-malware-platform-designed-
to-take-down-electric-grids
https://www.cyberbit.com/blog/ot-security/industroyer-crashoverride-ot-malware/

https://arstechnica.com/information-technology/2017/06/crash-override-malware-may-sabotage-
electric-grids-but-its-no-stuxnet/
https://blog.claroty.com/crashoverride-a.ka.industroyer-detection-and-alerting-in-claroty-platform
https://www.tenable.com/in-the-news/crashoverride-ics-attack-targets-vulnerable-electrical-grid
https://www.powermag.com/why-crashoverride-is-a-red-flag-for-u-s-power-companies/
https://searchsecurity.techtarget.com/news/450420683/CrashOverride-ICS-attack-targets-
vulnerable-electrical-grid
https://www.fda.gov/media/123073/download
https://cyberx-labs.com/glossary/industroyer-crashoverride-crash-override/
https://tehtris.com/en/egambit-endpoint-security-versus-crashoverride/

https://blog.paloaltonetworks.com/2017/06/crashoverrideindustroyer-protections-palo-alto-
networks-customers/
https://www.waterisac.org/portal/new-vulnerability-discovery-reportedly-abuses-same-protocol-
used-industroyercrashoverride
https://www.msspalert.com/cybersecurity-breaches-and-attacks/u-s-dept-of-homeland-securitys-
crashoverride-malware-warning-to-utilities/
https://iiot-world.com/ics-security/cybersecurity/five-cybersecurity-experts-about-crashoverride-
malware-main-dangers-and-lessons-for-iiot/
https://www.forescout.com/company/blog/crashoverride-protect-your-ics-network-against-the-
newest-malware/
https://search.abb.com/library/Download.aspx?DocumentID=9AKK107045A1003&amp;LanguageCode=en
&amp;DocumentPartId=&amp;Action=Launch
https://blog.checkpoint.com/2017/06/21/crashoverride/
https://www.blackhat.com/us-17/briefings/schedule/#industroyercrashoverride-zero-things-cool-
about-a-threat-group-targeting-the-power-grid-6159
https://www.sans.org/webcasts/notpetya-dragonfly-20-crashoverride-time-active-cyber-defense-ics-
scada-networks-105955
https://humanit.asia/ta17-163a/
https://www.vice.com/en_us/article/zmeyg8/ukraine-power-grid-malware-crashoverride-
industroyer
https://keybase.io/crashoverride
https://medium.com/@sroberts/the-crash-override-chronicles-overall-8389ef178fdf
https://www.oilandgaseng.com/articles/the-most-infamous-cyber-attacks-on-industrial-systems/
https://www.bankinfosecurity.com/power-grid-malware-platform-threatens-industrial-controls-a-
9987
https://www.microsoft.com/en-us/wdsi/threats/malware-encyclopedia-
description?Name=Trojan:Win32/CrashOverride.C!dha&amp;ThreatID=2147726161
https://www.cybersecurityintelligence.com/blog/attack-on-ukraines-power-grid-targeted-
transmission-stations-4530.html
https://thehill.com/policy/cybersecurity/337877-crash-override-malware-heightens-fears-for-us-
electric-grid
https://choice.npr.org/index.html?origin=https://www.npr.org/2017/09/08/548661962/in-crash-
override-zoe-quinn-shares-her-boss-battle-against-online-harassment
https://pylos.co/2018/11/03/crashoverride-when-advanced-actors-look-like-amateurs/
https://www.inguardians.com/dhs-fbi-warn-of-attacks-against-us-energy-manufacturing-companies-
and-employees/
https://blog.emergeits.com/alert-homeland-security-finds-u-s-power-grid-vulnerable-to-
crashoverride-malware-3
https://www.cyberthreatalliance.org/member-share/stuxnet-to-crashoverride-to-trisis-evaluating-
the-history-and-future-of-integrity-based-attacks-on-industrial-environments/
https://rethinkresearch.biz/articles/industroyer-crashoverride-malware-behind-ukraine-utility-
attack/
https://electricenergyonline.com/energy/magazine/1104/article/Security-Sessions-Combating-ICS-
Threats.htm
https://www.smart-energy.com/regional-news/europe-uk/encs-crash-override-virus/
https://securityaffairs.co/wordpress/59989/malware/industroyer-malware.html
https://isssource.com/crashoverride-designed-for-grid-takedown/
https://gigazine.net/gsc_news/en/20170613-crashoverride/
https://www.ethicalhacker.net/members/crashoverride/

https://www.theglobeandmail.com/arts/books-and-media/book-reviews/review-zoe-quinns-crash-
override-and-ellen-paos-reset-explore-the-sexism-of-tech-culture/article36431282/
https://www.forbes.com/sites/kalevleetaru/2017/06/24/crash-override-and-how-cyberwarfare-is-
bringing-conflict-to-the-homefront/#42eb8984277c
https://www.computerworld.com/article/2883732/crash-override-network-combats-online-
harassment.html
https://fticybersecurity.com/2017-11/crashoverride-red-flag-u-s-power-companies/
https://www.publicpower.org/periodical/article/e-isac-offers-special-webinar-june-16-
crashoverride-malware
https://blogs.dnvgl.com/energy/industroyer-and-industrial-control-systems
https://www.theverge.com/2015/1/17/7628567/crash-override
http://digitaleragroup.com/blog/industry-reactions-to-crashoverride-malware-feedback-
friday/?style=turquoise
https://www.us-cert.gov/ics/Recommended-Practices
iso standaarden
https://www.informatiebeveiligingsdienst.nl/wp-content/uploads/2019/04/201902-Handreiking-
Information-Security-Management-System-ISMS-v2.0.pdf
https://info.advisera.com/hubfs/27001Academy/27001Academy_FreeDownloads/NL/Checklist_of_I
SO_27001_Mandatory_Documentation_NL.pdf
https://info.advisera.com/hubfs/27001Academy/27001Academy_FreeDownloads/NL/Checklist_of_I
SO_27001_Mandatory_Documentation_NL.pdf
https://www.iso.org/isoiec-27001-information-security.html
https://www.iso27001security.com/html/faq.html
https://www.iso27001security.com/ISO27k_FAQ.pdf
https://www.iso27001security.com/ISO27k_GDPR_mapping_release_1.pdf

https://webstore.iec.ch/preview/info_isoiec27019%7Bed1.0%7Den.pdf
https://web.archive.org/web/20150924001524/http://www.epri.com/abstracts/Pages/ProductAbstr
act.aspx?ProductId=000000003002003738
https://webstore.iec.ch/publication/25948
https://webstore.iec.ch/publication/6912
https://webstore.iec.ch/publication/6911
https://www.ipcomm.de/protocol/IEC62351/en/sheet.html

UK NCSC Guidelines
BDEW and Oesterreichs Energie
NERC CIP
https://www.energy.gov/ceser/activities/cybersecurity-critical-energy-infrastructure/energy-sector-
cybersecurity-0
https://www.energy.gov/ceser/activities/cybersecurity-critical-energy-infrastructure/energy-sector-
cybersecurity-0-1
https://nvlpubs.nist.gov/nistpubs/ir/2010/NIST.IR.7628.pdf
https://nvlpubs.nist.gov/nistpubs/CSWP/NIST.CSWP.04162018.pdf
https://nvlpubs.nist.gov/nistpubs/SpecialPublications/NIST.SP.800-53r4.pdf
https://nvlpubs.nist.gov/nistpubs/ir/2014/NIST.IR.7628r1.pdf

Bibliografie
(n.d.). Retrieved from rand.org: https://www.rand.org/pubs/research_reports/RR2081.html
(n.d.). Retrieved from dpstele.com: https://dpstele.com/blog%20major%20scada%20hacks
/checklist/network-security-audit-checklist/#related-checklists. (n.d.). Retrieved from process.st:
https://www.process.st/checklist/network-security-audit-checklist/#related-checklists
(2016 ). Retrieved from https://www.fireeye.com/blog/threat-research/2016/01/ukraine-and-
sandworm-team.html
cybersecurity-audit-checklist. (n.d.). Retrieved from reciprocitylabs.com:
https://reciprocitylabs.com/cybersecurity-audit-checklist/
ukraine-power-grid-attack-russia-us. (2016, 2 11). Retrieved from edition.cnn.com:
https://edition.cnn.com/2016/02/11/politics/ukraine-power-grid-attack-russia-us/index.html

ukraine-sees-russian-hand-in-cyber-attacks-on-power-grid-idUSKCN0VL18E. (n.d.). Retrieved from
reuters.com: https://www.reuters.com/article/us-ukraine-cybersecurity/ukraine-sees-
russian-hand-in-cyber-attacks-on-power-grid-idUSKCN0VL18E
https://na.eventscloud.com/file_uploads/aed4bc20e84d2839b83c18bcba7e2876_Owens1.pdf
https://www.wired.com/2016/03/inside-cunning-unprecedented-hack-ukraines-power-grid/

http://web.mit.edu/smadnick/www/wp/2016-22.pdf

https://en.wikipedia.org/wiki/December_2015_Ukraine_power_grid_cyberattack

https://www.wired.com/story/russian-hackers-attack-ukraine/

https://www.linkedin.com/notifications/

https://www.boozallen.com/content/dam/boozallen/documents/2016/09/ukraine-report-when-the-
lights-went-out.pdf

https://www.reuters.com/article/us-ukraine-cybersecurity-sandworm-idUSKBN0UM00N20160108

https://www.wired.com/2016/01/everything-we-know-about-ukraines-power-plant-hack/

https://www.fireeye.com/blog/threat-research/2016/01/ukraine-and-sandworm-team.html

https://www.sans.org/blog/confirmation-of-a-coordinated-attack-on-the-ukrainian-power-grid/
https://www.reuters.com/article/us-ukraine-cybersecurity-sandworm/u-s-firm-blames-russian-
sandworm-hackers-for-ukraine-outage-idUSKBN0UM00N20160108
https://www.reuters.com/article/us-ukraine-crisis-cyber-idUSKBN15U2CN
https://www.wired.com/2014/10/russian-sandworm-hack-isight/
https://blog.trendmicro.com/trendlabs-security-intelligence/sandworm-to-blacken-the-scada-
connection/

https://blog.trendmicro.com/trendlabs-security-intelligence/killdisk-and-blackenergy-are-not-just-
energy-sector-threats/
https://www.nerc.com/pa/CI/ESISAC/Documents/E-ISAC_SANS_Ukraine_DUC_18Mar2016.pdf
https://www.politico.eu/article/ukraine-cyber-war-frontline-russia-malware-attacks/
https://theconversation.com/cyberattack-on-ukraine-grid-heres-how-it-worked-and-perhaps-why-it-
was-done-52802
https://www.ifri.org/sites/default/files/atoms/files/desarnaud_cyber_attacks_energy_infrastructure
s_2017_2.pdf
https://ec.europa.eu/energy/sites/ener/files/evaluation_of_risks_of_cyber-
incidents_and_on_costs_of_preventing_cyber-incidents_in_the_energy_sector.pdf
https://ec.europa.eu/energy/sites/ener/files/evaluation_of_risks_of_cyber-
incidents_and_on_costs_of_preventing_cyber-incidents_in_the_energy_sector.pdf
https://www.wired.com/2016/03/inside-cunning-unprecedented-hack-ukraines-power-grid/
https://www.osti.gov/servlets/purl/1505628
https://jsis.washington.edu/news/cyberattack-critical-infrastructure-russia-ukrainian-power-grid-
attacks/

IEC_60870-5
http://citeseerx.ist.psu.edu/viewdoc/download;jsessionid=0513EED48102FDAD1BD940260EF12B11?
doi=10.1.1.548.7490&amp;rep=rep1&amp;type=pdf
https://scialert.net/fulltext/?doi=tasr.2014.396.405
https://www.researchgate.net/publication/333671061_Attacking_IEC-60870-5-104_SCADA_Systems
http://www.salvage-project.com/uploads/4/9/5/5/49558369/art3_-_salvage_2015_-
_cyber_security_in_communication_of_scada_systems_using_iec_61850.pdf
https://eg.uc.pt/bitstream/10316/35720/1/Security%20Probes%20for%20Industrial%20Control%20
Networks.pdf
https://owlcyberdefense.com/products/data-diode-products/software-modules/iec-104/
https://ris.utwente.nl/ws/files/6028066/3-s2_0-B9780128015957000227.pdf
https://waterfall-security.com/static/Waterfall-for-IEC-60870-5-104_FINAL.pdf
http://www.scada.sl/2013/11/last-week-four-guys-of-scada.html
https://repositorio-aberto.up.pt/bitstream/10216/119066/2/315683.pdf
https://www.diva-portal.org/smash/get/diva2:1046339/FULLTEXT01.pdf
https://www.semanticscholar.org/paper/Cybersecurity-analysis-of-a-SCADA-system-under-and-
Rocha/dfa7c12551ebe7b24da8d806e87e946051a57cb9
https://control.com/forums/threads/comparison-between-iec60870-5-103-and-modbus-rtu.20317/
https://www.blackhat.com/docs/us-17/wednesday/us-17-Staggs-Adventures-In-Attacking-Wind-
Farm-Control-Networks.pdf
https://www.securonix.com/web/wp-
content/uploads/2019/08/RSAC_2019_Scada_Attack_Detection_101.pdf
https://dreamlab.net/en/blog/post/fuzzing-ics-protocols/
https://library.e.abb.com/public/f74f9c8be95f4cd09b0b1fbbde699108/4CAE000416_RTU_Secure%2
0communications_web.pdf
https://www.slideshare.net/qqlan/scada-zn
https://virsec.com/virsec-hack-analysis-deep-dive-into-industroyer-aka-crash-override/
https://tutcris.tut.fi/portal/files/16294332/jafary_1534.pdf
http://www.connectivity4ir.co.uk/article/175490/IEC-62351--Secure-communication-in-the-energy-
industry.aspx
https://www.win.tue.nl/~setalle/2017_fauri_encryption.pdf

https://assets.barracuda.com/assets/docs/dms/Barracuda_CloudGen_Firewall_SB_Security_for_ICS_
and_OT.pdf
https://www.checkpoint.com/products/industrial-control-systems-appliances/
https://www.dragos.com/wp-content/uploads/CRASHOVERRIDE.pdf
https://dl.acm.org/doi/fullHtml/10.1145/3381038
https://arxiv.org/pdf/2001.02925.pdf
http://blog.nettedautomation.com/2017/

https://www.dragos.com/wp-content/uploads/CrashOverride-01.pdf
https://www.welivesecurity.com/wp-content/uploads/2017/06/Win32_Industroyer.pdf
https://ec.europa.eu/energy/sites/ener/files/evaluation_of_risks_of_cyber-
incidents_and_on_costs_of_preventing_cyber-incidents_in_the_energy_sector.pdf
https://www.cybersecurityintelligence.com/blog/attack-on-ukraines-power-grid-targeted-
transmission-stations-4530.html
https://www.recordedfuture.com/crashoverride-malware-overview/
https://www.us-cert.gov/ncas/alerts/TA17-163A
https://www.darkreading.com/threat-intelligence/first-malware-designed-solely-for-electric-grids-
caused-2016-ukraine-outage/d/d-id/1329114
https://arstechnica.com/information-technology/2017/06/crash-override-malware-may-sabotage-
electric-grids-but-its-no-stuxnet/
https://www.accenture.com/_acnmedia/pdf-69/accenture-managing-malware-crash-override.pdf
https://www.nixu.com/fi/node/53
https://www.vice.com/en_us/article/zmeyg8/ukraine-power-grid-malware-crashoverride-
industroyer
https://ics.sans.org/media/E-ISAC_SANS_Ukraine_DUC_6.pdf
http://blog.wallix.com/ics-security-russian-hacking
http://web.mit.edu/smadnick/www/wp/2016-22.pdf
https://www.boozallen.com/content/dam/boozallen/documents/2016/09/ukraine-report-when-the-
lights-went-out.pdf
https://www.reuters.com/article/us-ukraine-cybersecurity-sandworm-idUSKBN0UM00N20160108
https://www.nerc.com/pa/CI/ESISAC/Documents/E-ISAC_SANS_Ukraine_DUC_18Mar2016.pdf
https://jsis.washington.edu/news/cyberattack-critical-infrastructure-russia-ukrainian-power-grid-
attacks/
https://www.ifri.org/sites/default/files/atoms/files/desarnaud_cyber_attacks_energy_infrastructure
s_2017_2.pdf
https://blog.trendmicro.com/trendlabs-security-intelligence/sandworm-to-blacken-the-scada-
connection/
https://ec.europa.eu/energy/sites/ener/files/evaluation_of_risks_of_cyber-
incidents_and_on_costs_of_preventing_cyber-incidents_in_the_energy_sector.pdf
https://www.wired.com/2016/03/inside-cunning-unprecedented-hack-ukraines-power-grid/
https://ics.sans.org/media/E-ISAC_SANS_Ukraine_DUC_5.pdf

https://ics.sans.org/media/E-ISAC_SANS_Ukraine_DUC_5.pdf
https://digitalsupport.ge.com/servlet/fileField?retURL=%2Fapex%2FKnowledgeDetail%3Fid%3DkA21
A000000HShPSAW%26lang%3Den_US%26Type%3DArticle__kav&amp;entityId=ka21A000000HccQQAS&amp;fi
eld=File_1__Body__s
https://www.us-cert.gov/ics/Recommended-Practices

iso standaarden
https://www.informatiebeveiligingsdienst.nl/wp-content/uploads/2019/04/201902-Handreiking-
Information-Security-Management-System-ISMS-v2.0.pdf
https://info.advisera.com/hubfs/27001Academy/27001Academy_FreeDownloads/NL/Checklist_of_I
SO_27001_Mandatory_Documentation_NL.pdf
https://info.advisera.com/hubfs/27001Academy/27001Academy_FreeDownloads/NL/Checklist_of_I
SO_27001_Mandatory_Documentation_NL.pdf
https://www.iso.org/isoiec-27001-information-security.html
https://www.iso27001security.com/html/faq.html
https://www.iso27001security.com/ISO27k_FAQ.pdf
https://www.iso27001security.com/ISO27k_GDPR_mapping_release_1.pdf

https://webstore.iec.ch/preview/info_isoiec27019%7Bed1.0%7Den.pdf
https://web.archive.org/web/20150924001524/http://www.epri.com/abstracts/Pages/ProductAbstr
act.aspx?ProductId=000000003002003738
https://webstore.iec.ch/publication/25948
https://webstore.iec.ch/publication/6912
https://webstore.iec.ch/publication/6911
https://www.ipcomm.de/protocol/IEC62351/en/sheet.html

UK NCSC Guidelines
BDEW and Oesterreichs Energie
NERC CIP
https://www.energy.gov/ceser/activities/cybersecurity-critical-energy-infrastructure/energy-sector-
cybersecurity-0
https://www.energy.gov/ceser/activities/cybersecurity-critical-energy-infrastructure/energy-sector-
cybersecurity-0-1
https://nvlpubs.nist.gov/nistpubs/ir/2010/NIST.IR.7628.pdf
https://nvlpubs.nist.gov/nistpubs/CSWP/NIST.CSWP.04162018.pdf
https://nvlpubs.nist.gov/nistpubs/SpecialPublications/NIST.SP.800-53r4.pdf
https://nvlpubs.nist.gov/nistpubs/ir/2014/NIST.IR.7628r1.pdf

Bibliografie
(n.d.). Retrieved from rand.org: https://www.rand.org/pubs/research_reports/RR2081.html
(n.d.). Retrieved from dpstele.com: https://dpstele.com/blog%20major%20scada%20hacks
/checklist/network-security-audit-checklist/#related-checklists. (n.d.). Retrieved from process.st:
https://www.process.st/checklist/network-security-audit-checklist/#related-checklists
(2016 ). Retrieved from https://www.fireeye.com/blog/threat-research/2016/01/ukraine-and-
sandworm-team.html
cybersecurity-audit-checklist. (n.d.). Retrieved from reciprocitylabs.com:
https://reciprocitylabs.com/cybersecurity-audit-checklist/
ukraine-power-grid-attack-russia-us. (2016, 2 11). Retrieved from edition.cnn.com:
https://edition.cnn.com/2016/02/11/politics/ukraine-power-grid-attack-russia-us/index.html
ukraine-sees-russian-hand-in-cyber-attacks-on-power-grid-idUSKCN0VL18E. (n.d.). Retrieved from
reuters.com: https://www.reuters.com/article/us-ukraine-cybersecurity/ukraine-sees-
russian-hand-in-cyber-attacks-on-power-grid-idUSKCN0VL18E


\section{Conclusie}



\subsubsection{Requirement elicitation technieken}

Requirements elicitation technieken zijn methoden die een onderzoekeer kan gebruiken om de behoeften van de stakeholders in kaart te brengen. De stakeholders  vormen de belangrijkste groep die de doelstelling van een project vastlegd.

Enkele voorbeelden van requirement elcitation technieken zijn:


\begin{enumerate}
	\item  Intervieews
	\item  Brainstorming sessions 
	\item  Use case approach 
	\item  Document analysis 
	\item  Observation
	\item  Prototyping
	\item  Joint applicationdevelopment
	\item  Reverse engineering 
	\item  Survey/ Questionairre 
	\item  Focus groups 
	\item  Interface analysis
	\item  Stakeholder analysis 
	\item  Card sorting laddering 
	\item  Open ended-interview
\end{enumerate}






\usepackage{pgf}
%\newcommand\setform{\pgfqkeys{/form }}
%\setform{field1/.store in=\fieldi,
%	field2/.store in=\fieldii,
%}


%\addtolength{\oddsidemargin}{-.875in}
%\addtolength{\evensidemargin}{-.875in}
%\addtolength{\textwidth}{1.75in}

%\addtolength{\topmargin}{-.875in}
%\addtolength{\textheight}{1.75in}


\newcommand\myform{%
	\fboxrule=0.4pt
	
	
	\fbox{\begin{minipage}{\textwidth}
			\fbox{\begin{minipage}[t][3cm][t]{0.25\textwidth}
					Betrokken partij
			\end{minipage}}%
			\fbox{\begin{minipage}[t][3cm][t]{0.25\textwidth}
					Verantwoordelijk
			\end{minipage}}%
			\fbox{\begin{minipage}[t][3cm][t]{0.44\textwidth}
					datetimestamp here
			\end{minipage}}
			\fbox{\begin{minipage}[t][1cm][t]{0.98\textwidth}
					Korte notitie 
			\end{minipage}}
			
	\end{minipage}}
	
	\fbox{\begin{minipage}{\textwidth}
			
			\fbox{\begin{minipage}[t][3cm][t]{0.98\textwidth}
					test
			\end{minipage}}
	\end{minipage}}
	
	\fbox{\begin{minipage}{\textwidth}
			\fbox{\begin{minipage}[t][10cm][t]{0.98\textwidth}
					Foto incident
			\end{minipage}}%
			
			
	\end{minipage}}
}



\setform{field1 = G. Wales,
	field2 = Mathematics}
\myform



