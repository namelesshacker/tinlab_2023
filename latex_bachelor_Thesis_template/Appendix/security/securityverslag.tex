%\documentclass[twocolumn]{el-author}
\documentclass[11pt]{report}
%\usepackage[...]{...}      This has been commented out as we are not using any additional packages here.  On the whole, they should be unnecessary.
\newcommand{\hH}{\hat{H}}
\newcommand{\D}{^\dagger}
\newcommand{\ua}{\uparrow}
\newcommand{\nc}{\newcommand}
\nc{\da}{\downarrow} \nc{\hc}{\hat{c}} \nc{\hS}{\hat{S}}
\nc{\bra}{\langle} \nc{\ket}{\rangle} \nc{\eq}{equation (\ref}
\nc{\h}{\hat} \nc{\hT}{\h{T}}\nc{\be}{\begin{eqnarray}}
\nc{\ee}{\end{eqnarray}}\nc{\rd}{\textrm{d}}\nc{\e}{eqnarray}\nc{\hR}{\hat{R}}\nc{\Tr}{\mathrm{Tr}}
\nc{\tS}{\tilde{S}}\nc{\tr}{\mathrm{tr}}\nc{\8}{\infty}\nc{\lgs}{\bra\ua,\phi|}\nc{\rgs}{|\ua,\phi\ket}
\nc{\hU}{\hat{U}}\nc{\lfs}{\bra\phi|}\nc{\rfs}{|\phi\ket}\nc{\hZ}{\hat{Z}}\nc{\hd}{\hat{d}}\nc{\mD}{\mathcal{D}}
\nc{\bd}{\bar{d}}\nc{\bc}{\bar{c}}\nc{\mc}{\mathcal}\nc{\ea}{eqnarray}\nc{\mG}{\mathcal{G}}\nc{\bce}{\begin{center}}
\nc{\ece}{\end{center}}
\date{12th April 2023}

\begin{document}

\title{Onderzoek naar de cyberaanval  op \emph{Oekraine} en de impact voor security van SCADA systemen}

\author{G. Bartes}

\abstract{This document describes how to use the el-author.cls file and how to format your \LaTeX submissions
correctly for \emph{Electronics Letters}. It also serves as a template, so that you can simply
copy the text from this example .tex file and replace it with your own.  We have tried to cover the basic
tools and commands you might need, but there may be some more unusual fields, etc, not described. Do not hesitate to contact
us if you encounter any problems.  The structure is as follows: we introduce the basic notations and preamble, and then provide some example text, followed by the references.  For simplicity we have left the source code out of this document and refer the reader to the sample.tex file itself, from which to copy and paste.}

\maketitle

\section{Introduction}


this model, depicted in Fig. 1 and described in Section 2, has been used successfully in the development of safety-critical systems in industry and helps to clarify the behaviors of, and the boundaries between, the environment, sensors, actuators, and software. To be implementable, the system requirements must be feasible with respect to the environment (i.e., should specify only behaviors that obey the environmental constraints) and acceptable software behaviors must be possible given the chosen input/output devices. 
sciencedirect.com/science/article/pii/S0167642315001033#fg0010
https://shemesh.larc.nasa.gov/fm/papers/ExtendingTheFourVariableModel.pdf




\verb"el-author.cls" is used in a similar fashion to the standard \verb"article.cls" file. However, the \verb"el-author.cls" file must be copied into the same directory as the .tex file you wish to compile for submission. Most of the preamble needed for including packages for mathematics or for displaying images is included within the \verb".cls" file itself, whereas more exotic packages will have to be included manually.

If you prefer to review your document in single column format or double spaced you can include this in the options of the document class with the command - inside the square brackets - \verb"[doublespace, onecolumn]".

Tables are straightforward to include (check the \verb".tex" file for details), and will format automatically:
\begin{table}[h]
\processtable{Coefficients and remainders for distribution KK ($k = 0.05$,
$v = 3$, $c_{1} = 1.5$, $c_{2} = 4.5$)}
{\begin{tabular}{|l|l|l|}\hline
$n$ & $a_{n}^{2}$ & $r_{k}(1)$\\\hline
0 & 3.602576748428 & 1.493719547999\\\hline
1 & 1.384791111989 & 0.108928436101\\\hline
2 & 0.108600438794 & 0.000327997399\\\hline
3 & 0.000275794597 & 0.000052202814\\\hline
4 & 0.000027616892 & 0.000024585922\\\hline
5 & 0.000018178621 & 0.000006407300\\\hline
\end{tabular}}{}
\end{table}

Note that we used \verb"[h]" after the \verb"\begin{table}" command to force the table to be included exactly at that location.  The same can be done for
all tables and figures:

\begin{figure}[h]
\centering{\includegraphics[width=60mm]{imagefile1a}}
\caption{The Keldysh contour before extension of the real axis to infinity
\source{}}
\end{figure}

In the next section we provide a short example manuscript, which includes images and their captions.  In \verb"sample.tex" we have added some comments explaining how to use \verb"\source{...}" to include subcaptions, and how to format equations over more than one line.  For more information on submitting and \emph{Electronics Letters} house style, see the author guide at http://www.theiet.org/resources/journals/eletters/authors.cfm.

\section{Situatie}
Starting at 3:30 p.m. on December 23, 2015, the Kyiv,
Prykarpattia, and Chernivtsi electric control center HMIs
began opening and closing circuit breakers without input from
operators. The resulting unauthorized operations resulted in
the loss of power to approximately 225,000 customers across
Ukraine [1] [2]. Operators at the three operations centers were
unable to regain remote control of more than 50 substations
affected by the incident. After six hours and the loss of over
130 MW of load, operators restored power by sending
technicians to the substations and manually controlling the
power system [3] [4] [5].
Complicating the situation and reducing operator
communications, the malicious actors also launched a
telephony-based denial of service attack, using automated
systems to overload the phone systems of the utilities.
Post-power outage analysis found that firmware was
corrupted on serial-to-Ethernet converters at substations,
uninterruptible power supplies (UPS) for both the server room
and the telephony system were remotely turned off, and the
hard drives of numerous computers were corrupted.
This event was the first successful cyber-induced power
outage that disrupted an electric power grid. To mitigate future
attempts at disruption of electrical power by cyber means, it is critical that other electric power organizations learn from the
Ukraine incident.
source: https://na.eventscloud.com/file_uploads/aed4bc20e84d2839b83c18bcba7e2876_Owens1.pdf
https://www.boozallen.com/content/dam/boozallen/documents/2016/09/ukraine-report-when-the-lights-went-out.pdf
https://theconversation.com/cyberattack-on-ukraine-grid-heres-how-it-worked-and-perhaps-why-it-was-done-52802
De aanval is slim opgezet
https://www.sans.org/blog/confirmation-of-a-coordinated-attack-on-the-ukrainian-power-grid/
Cyber aanvallen in het algemeen
https://www.ifri.org/sites/default/files/atoms/files/desarnaud_cyber_attacks_energy_infrastructures_2017_2.pdf
Nearly a year later, on December 17, 2016, a single transmission substation in northern Kiev lost power.
https://jsis.washington.edu/news/cyberattack-critical-infrastructure-russia-ukrainian-power-grid-attacks/
Attacking IEC-60870-5-104 SCADA Systems
https://www.researchgate.net/publication/333671061_Attacking_IEC-60870-5-104_SCADA_Systems

In the typical quantum dot set-up, as described in the introduction, the dot weakly connects together two electron seas, the leads.  It is understood that the phenomenon of the Coulomb blockade limits conductance through the dot unless the charge induced on the dot by the gate is
\begin{align}
  Q=\left(N+\frac{1}{2}\right)e
\end{align}
Consequently, we find sharp peaks in the conductance of the dot at these degeneracy points.  However, for $T<T_{K}$ new behaviour is observed, as in Fig. \ref{kondodotresistance}.  The original conduction peaks of figure of the classic Coulomb blockade exist when the occupancy is effectively half integer.  Hence, we expect at integer occupancy \emph{suppression} of the conductance.  This is indeed observed if $N$ is even.  However, for $T<T_{K}$ and $N$ odd, we see that the conductance is not \emph{fully} suppressed.  The difference is clear: for even occupancy, the spin of the dot will be zero, as there will be as many up as down electrons.  However, for odd filling, the $N+1$th electron will contribute a spin-half, causing the dot to behave as a Kondo-like impurity.  We will discuss what consequences the Kondo-nature of the dot has, but first we will explain exactly \emph{how} it acquires this nature.

\section{Oorzaken}

1. An initial email spear phishing attack lures recipients
into opening an attached Microsoft® document with a
macro that installs Black Energy 3 (BE3) onto
corporate workstations.
2. BE3 and other tools perform reconnaissance and
enumeration of the network and provide an initial
backdoor for the hackers into the corporate network.
3. As a result of network reconnaissance, the malicious
actors discover and access the oblenergos’ Microsoft
Active Directory® servers that contain corporate user
accounts and credentials.
4. With the harvested credentials, the malicious actors use
an encrypted tunnel from an external network to get
inside the oblenergo network, establishing a presence
on the oblenergo control system networks.
5. Malicious actors discover and access the control center
supervisory control and data acquisition (SCADA)
human-machine interface (HMI) servers and
substations. While a router separates corporate and
SCADA networks, the firewall rules are improperly
configured.
6. On December 23, 2015, at 3:30 p.m., the malicious
actors begin their power outage attacks by entering
operations and SCADA networks through backdoors on
the compromised SCADA workstations. The malicious
actors take control away from HMI operators and then
open breakers.
7. The malicious actors perform several other actions with
the intent of complicating the responses of control
operators and increasing the effort required to return the
system to normal operating conditions. These actions
include:
a. Launching a coordinated Telephony Denial of
Service (TDoS) attack that floods call centers to
prevent legitimate calls from getting through.
b. Disabling the UPSs for the control centers.
c. Corrupting the firmware on a remote terminal unit
(RTU) HMI module and serial-to-Ethernet port
servers.
8. Malicious actors execute KillDisk malware in an
attempt to wipe out the control center HMIs and pivotpoint workstations.
http://web.mit.edu/smadnick/www/wp/2016-22.pdf 


\section{4 variabelen model}

\subsection{Monitored ie. sensors}


\subsection{Controlled ie. actuators}


\subsection{Input}

intrusion usingb phishing mail blz  4
install malware on the system blz 5
create a communication channel to the adversary’s command and control.
reconnaissance by collecting credentials and moving through the network to eventually pivot into the network segments where Supervisory Control and Data Acquisition (SCADA) dispatch workstations and servers existed.
reconfiguredthe uninterruptible power supply (UPS)
coded malicious firmware to replace the legitimate firmware on serial-to-Ethernet converters at more than a dozen substations to prevent rebooting to support recovery blz 9
take out the converters so that it would prevent operators from sending remote commands to re-close breakers once the shutdown happened
tap into the SCADA networks through the hijacked VPNs and send commands to deactivate the UPS system they had already reconfigured. I
an internal telephone communications server was targeted effectively cutting off all internal communications with regional offices and distribution substations
actor used the local UPS to schedule a power shutdown of the main datacenter to occur a few hours later. In addition to standard consequences of power loss, a reboot caused the full impact of the KillDisk efforts to take effect.
The hackers had flooded customer call centers with thousands of fake calls, to prevent legitimate callers from being able to report the outage.
Once they neared the end of their assault, the hackers used KillDisk to wipe the files off the  operator station’s causing them to be inoperable. KillDisk is a piece of malware that wipes or  overwrites data in essential system files that causes a computer to crash. The infected computers are unable to reboot because KillDisk also overwrites the master boot record blz 9
source http://web.mit.edu/smadnick/www/wp/2016-22.pdf 


\subsection{Output}



To set up the non-equilibrium Kondo problem - formally - we introduce the two-channel Anderson Hamiltonian
\begin{align}
  H_{2C}=&\sum_{\alpha k\sigma}\epsilon_{\alpha k}\hat{c}^{\dagger}_{\alpha k\sigma}\hat{c}_{\alpha
  k\sigma}+U\hat{d}^{\dagger}_{\uparrow}\hat{d}_{\uparrow}\hat{d}^{\dagger}_{\downarrow}\hat{d}_{\downarrow}\nonumber\\
  &+\sum_{\sigma}\epsilon_{d}\hat{d}^{\dagger}_{\sigma}\hat{d}_{\sigma}+\sum_{\alpha
  k\sigma}[t_{\alpha}\hat{c}^{\dagger}_{\alpha k\sigma}\hat{d}_{\sigma}+h.c.]
\end{align}
The subscript $\alpha$ is the channel label, for the dot case left and right.  The physical idea is that the dot is already at half-integer occupancy.  The Hubbard $U$ is recognized as the charging energy (the energy required to add another electron) which we assume to be much larger than the mean level spacing in the dot, so that we may consider only one level, $\epsilon_{d}$.  The hybridization, $t_{\alpha}$ is the tunneling energy through the potential barriers connecting the dot to the leads, and is assumed to be point like.  It is clear that the dot behaves exactly as the original Anderson impurity model, with the addition of lead indices, and this Hamiltonian has been studied \emph{perturbatively}.  However, the Schrieffer-Wolff transformation can be performed exactly as before:
\begin{align}
  H_{2K}=\sum_{\alpha k\sigma}\epsilon_{\alpha k}\hat{c}^{\dagger}_{\alpha k\sigma}\hat{c}_{\alpha
  k\sigma}+\sum_{\alpha\beta\sigma\tau}\underbrace{\frac{t^{*}_{\alpha}t_{\beta}}{U}}_{J_{\alpha\beta}}\hat{c}^{\dagger}_{\alpha\sigma}(r=0)\sigma^{a}_{\sigma\tau}\hat{c}_{\beta\tau}(r=0)S^{a}
\end{align}
In the following we will assume that the coupling to the left and right leads is identical, $J_{\alpha\beta}$,  we may perform the sum over leads, giving
\begin{align}\label{2lead}
  H_{2K}=\sum_{\alpha k\sigma}\epsilon_{\alpha k}\hat{c}^{\dagger}_{\alpha k\sigma}\hat{c}_{\alpha
  k\sigma}+J\{[\hat{c}_{L\sigma}^{\dagger}(r=0)+\hat{c}_{R\sigma}^{\dagger}(r=0)]\nonumber\\
  \times\sigma^{a}_{\sigma\tau}[\hat{c}_{L\tau}(r=0)+\hat{c}_{R\tau}(r=0)]\}S^{a}
\end{align}

\begin{figure}
\centering{\includegraphics[width=60mm]{imagefile2a}}
\caption{Quantum dot resistance for $T\ll T_{K}$ and $T\gg T_{K}$
\source{For  high temperatures (dashed line) the Coulomb blockade remains}
\source{For lower temperatures (solid line) the Coulomb blockade is overcome}}\label{kondodotresistance}
\end{figure}


\section{Veiligheidsissues}

\section{Maatregelen}


III. CREATING A ROBUST CONTROL SYSTEM ARCHITECTURE
To create a robust control system architecture with a solid
defense, an organization should consider three concepts.
• Identify risk and develop a plan for managing that
risk.
• Implement effective controls to manage the risk.
• Create a defense-in-depth model that allows effective
and efficient security controls. 

Risk assessment has the
following objectives:
• Identify assets and their value
• Identify vulnerabilities and threats
• Calculate threat probability and business impact
• Balance threat impact with security control cost

ls. Asset value includes:
• Value of replacement
• Cost to maintain
• Damage in cost if lost
• Penalties or fines if lost

Oplossingen
1) Isolate Control Systems
2) Baseline, Log, and Continuously Monitor Control Systems
3) Patch, Update, and Maintain
4) Have Contingency Plans
5) After Action Reports and Lessons Learned
6) Ensure Physical Security
7) Ensure Complex Passwords
source: https://na.eventscloud.com/file_uploads/aed4bc20e84d2839b83c18bcba7e2876_Owens1.pdf

Detection of the BlackEnergy malware should be conducted using the latest published YARA signature
https://www.cisa.gov/news-events/ics-alerts/ir-alert-h-16-056-01

requiremnts en de rol van de politiek/statelijke actoren
https://www.reuters.com/article/us-ukraine-cybersecurity-sandworm/u-s-firm-blames-russian-sandworm-hackers-for-ukraine-outage-idUSKBN0UM00N20160108
https://www.reuters.com/article/us-ukraine-crisis-cyber-idUSKBN15U2CN

There may be no better place to witness cyber conflict in action than Ukraine today. Open warfare with Russia, a highly skilled, computer-literate pool of talent and a uniquely vulnerable political, economic and IT environment have made the country the perfect sandbox for those looking to test new cyberweapons, tactics and tools.

Even as Russian tanks crossed the physical border into eastern Ukraine in the spring of 2014, Russian-affiliated hackers were sending malicious code onto Ukraine’s IT systems, providing political chaos as a smokescreen.


The goal, say experts, is to test the West’s defenses. The U.S. and other intelligence agencies have responded by moving into the Ukrainian networks to pick up the signals.


https://www.politico.eu/article/ukraine-cyber-war-frontline-russia-malware-attacks/
First Malware Designed Solely for Electric Grids Caused 2016 Ukraine Outage

https://www.darkreading.com/threat-intelligence/first-malware-designed-solely-for-electric-grids-caused-2016-ukraine-outage






Systeem
https://citeseerx.ist.psu.edu/viewdoc/download;jsessionid=0513EED48102FDAD1BD940260EF12B11?doi=10.1.1.548.7490&amp;rep=rep1&amp;type=pdf





If we assume that the kinetic term takes the same form for the left lead as the right lead (in equilibrium), a simple (Bogoliubov) rotation of basis will transform $H_{2K}$ into the standard one channel Kondo model:
\begin{align}\label{kondolies}
  \hH_{2K}^{1}=\sum_{k}\varepsilon_{k}\hat{c}_{\alpha k\sigma}\D\hat{c}_{\alpha k\sigma}+
  J\sum_{\alpha}\hat{s}(0)\cdot\hat{S}
\end{align}
such that the new Hamiltonian is diagonal in the lead index, and so $\alpha$ behaves as an additional degeneracy.  This procedure is justified if we wish to \emph{perturbatively} analyze the conductance of the dot.  For bias voltages, $V$ much lower than the Kondo temperature, this seems reasonable, as the only true energy scale for the Kondo model is $T_{K}$.

Given that $\hH_{2K}^{1}$ is diagonal in lead index, the same techniques as are used for the equilibrium case apply.  Indeed, performing the poor man's scaling procedure and using Fermi's golden rule, it is straightforward to recover the result, for $T\gg T_{K}$:
\begin{align}
  &G_{1}\sim G_{0}\nu_{0}J\nonumber\\
  &G_{0}\sim{\ln^{2}(T/T_{k})}
\end{align}
A full and more careful treatment, recovers the numerical factors:
\begin{align}
  G_{1}&=\frac{2e^{2}}{h}\frac{4\Gamma_{L}\Gamma_{R}}{(\Gamma_{L}+\Gamma_{R})^{2}}\frac{3\pi^{2}/16}{\ln^{2}(T/T_{k})}\nonumber\\
  &\equiv G_{0}\frac{3\pi^{2}/16}{\ln^{2}(T/T_{k})}
\end{align}
We emphasize that this is valid \emph{only} for $T\gg T_{K}\gg V$.

At temperatures below $T_{K}$, the coupling diverges, so that the dominant term in equation (\ref{kondolies}) is
\begin{align}
  \hH_{coup}=J\sum_{\alpha}\hat{s}(0)\cdot\hat{S}
\end{align}
As was discussed for the one channel problem, the ground state is a singlet, with zero spin, and we expect the scattering in the dot to be suppressed, and so to leading order, the the conductance reduces to $G_{2}=G_{0}$.  Perturbative corrections have been found, \cite{3}, which yield
\begin{align}
  G_{2}=G_{0}\left[1-\left(\frac{\pi T}{T_{K}}\right)^{2}\right]
\end{align}
Thus, we can define two regions for the conductance, both for $V\ll T_{K}$:
\begin{align}
  \begin{array}{cc}
    G_{1}=G_{0}\frac{3\pi^{2}/16}{\ln^{2}(T/T_{k})}, & \;\;\;\;T\gg T_{K} \\
    G_{2}=G_{0}, & \;\;\;\;T\ll T_{K}
  \end{array}
\end{align}
So, we see that as we lower the temperature below $T_{K}$, for an odd-integer Coulomb blockade valley, the conductance is no longer exponentially suppressed.

As we have stressed, these results are valid \emph{only} for $V\ll T_{K}$.  The next step is to introduce an arbitrary voltage via the kinetic term in equation (\ref{kondolies}):
\begin{align}
  \hH_{2K}^{1}=\sum_{k\sigma}(\varepsilon_{k}-eV)\hat{c}_{L k\sigma}\D\hat{c}_{L k\sigma}+
  \sum_{k\sigma}\varepsilon_{k}\hat{c}_{R k\sigma}\D\hat{c}_{R k\sigma}+
  J\sum_{\alpha}\hat{s}(0)\cdot\hat{S}
\end{align}
If we assume that $V\gg T$, we can again divide into two regions:
\begin{align}
  \tilde{G}=\left\{\begin{array}{cc}
     \tilde{G}_{1}, & \;\;\;\;\;\;\;V\gg T_{K} \\
     \tilde{G}_{2}, & \;\;\;\;\;\;\;V\ll T_{K}
   \end{array}\right.
\end{align}
Here, the previous work is based on the idea that $eV$ now plays the same role as temperature.  That is, in the R.G. flow, we cut at $eV$, and the perturbation analysis of \cite{3} is now for low voltage, so we find
\begin{align}
     &\tilde{G}_{1}=G_{0}\frac{3\pi^{2}/16}{\ln^{2}(eV/T_{k})}\nonumber\\
     &\tilde{G}_{2}=G_{0}\left[1-\left(\frac{\pi eV}{T_{K}}\right)^{2}\right]
\end{align}
However, we argue that the approximations used are not entirely reasonable.  From the work of N. d'Ambrumenil and B. Muzykantskii, on the non-equilibrium x-ray problem (to which the Kondo problem can be related), it is \emph{not} sufficient to decouple the leads, rotate basis and then simply reintroduce the voltage.  It is clear (in the very least as a precaution), that a full treatment of the \emph{true} two lead Kondo Hamiltonian of equation (\ref{2lead}) is required.  With this in mind, in the next chapter, we will be following the calculation of Anderson, Yuval and Hamann, in which they map the Kondo Hamiltonian onto a two dimensional Coulomb gas.

%\vfill\pagebreak

\section{Conclusion}
We have derived some results for the two lead Kondo problem in various limits.  We have shown the suppression of the Coulomb blockade, and observed that this suppression can be viewed as a kind of delocalisation caused by the Kondo singlet across the dot.  The above treatment required us to neglect the bias potential and then to rotate our two lead problem to a diagonal basis.  However, it is not clear that this is a controlled or reasonable approach.  In fact, the presence of voltage in the non-diagonal Green's function of related x-ray problems implies that the voltage cannot be treated perturbatively, and that a generalisation of non-eequilibrium Riemann-Hilbert techniques may be necessary.
\section{Verder lezen}

https://ris.utwente.nl/ws/files/6028066/3-s2_0-B9780128015957000227.pdf
https://repositorio-aberto.up.pt/bitstream/10216/119066/2/315683.pdf
https://www.diva-portal.org/smash/get/diva2:1046339/FULLTEXT01.pdf
https://www.semanticscholar.org/paper/Cybersecurity-analysis-of-a-SCADA-system-under-and-Rocha/dfa7c12551ebe7b24da8d806e87e946051a57cb9
https://dreamlab.net/en/blog/post/fuzzing-ics-protocols/
https://www.connectivity4ir.co.uk/article/175490/IEC-62351--Secure-communication-in-the-energy-industry.aspx
https://www.win.tue.nl/~setalle/2017_fauri_encryption.pdf
https://blog.nettedautomation.com/2017/
taxonomie van scada systemen en hun aanval
https://arxiv.org/pdf/2001.02925.pdf
https://www.dragos.com/wp-content/uploads/CrashOverride-01.pdf
https://www.welivesecurity.com/wp-content/uploads/2017/06/Win32_Industroyer.pdf
https://www.cybersecurityintelligence.com/blog/attack-on-ukraines-power-grid-targeted-transmission-stations-4530.html
https://arstechnica.com/information-technology/2017/06/crash-override-malware-may-sabotage-electric-grids-but-its-no-stuxnet/
https://www.vice.com/en/article/zmeyg8/ukraine-power-grid-malware-crashoverride-industroyer
https://www.dragos.com/resource/crashoverride-analyzing-the-malware-that-attacks-power-grids/
https://en.wikipedia.org/wiki/Industroyer
https://en.wikipedia.org/wiki/Crash_Override_Network
https://www.nixu.com/blog/crashoverride-threat-electricity-networks
wat kan een backdoor allemaal
https://www.cyber.nj.gov/threat-center/threat-profiles/ics-malware-variants/crashoverride
https://www.incibe-cert.es/en/blog/crashoverride-malware-ics-back-again
industroyer
https://rhebo.com/en/service/glossar/industroyer-25114/
https://www.paloaltonetworks.com/blog/2017/06/crashoverrideindustroyer-protections-palo-alto-networks-customers/
https://search.abb.com/library/Download.aspx?DocumentID=9AKK107045A1003&amp;LanguageCode=en



crashoverride in december 2016
https://www.dragos.com/wp-content/uploads/CRASHOVERRIDE.pdf
https://www.csoonline.com/article/3200828/crash-override-malware-that-took-down-a-power-grid-may-have-been-a-test-run.html
https://www.powermag.com/why-crashoverride-is-a-red-flag-for-u-s-power-companies/
https://www.webopedia.com/definitions/crashoverride-industroyer-malware/

https://www.msspalert.com/cybersecurity-breaches-and-attacks/u-s-dept-of-homeland-securitys-crashoverride-malware-warning-to-utilities/
https://blog.checkpoint.com/research/crashoverride/
https://thehill.com/policy/cybersecurity/337877-crash-override-malware-heightens-fears-for-us-electric-grid/
https://pylos.co/2018/11/03/crashoverride-when-advanced-actors-look-like-amateurs/
https://www.inguardians.com/dhs-fbi-warn-of-attacks-against-us-energy-manufacturing-companies-and-employees/
https://www.cyberthreatalliance.org/stuxnet-to-crashoverride-to-trisis-evaluating-the-history-and-future-of-integrity-based-attacks-on-industrial-environments/
https://rethinkresearch.biz/articles/industroyer-crashoverride-malware-behind-ukraine-utility-attack/






\vskip3pt
\ack{This work has been supported by The IET}

\vskip5pt

\noindent J. Smith and A. N. Other (\textit{The IET, Stevenage, UK})
\vskip3pt

\noindent E-mail: jsmith@theiet.org

\begin{thebibliography}{}

\bibitem{1}
Anderson, P.: `A poor man's derivation of scaling laws for the Kondo problem', \textit{J. Phys. C.}, 1960, \textbf{3}, p. 2436

\bibitem{2}
Coleman, P.: `1/N expansion for the Kondo lattice', \textit{Phys. Rev. B}, 1983, \textbf{28}, pp. 5255-5262

\bibitem{3}
Ludwig, I. and Ludwig A. W. W.: `Kondo effect induced by a magnetic field', \textit{Phys. Rev. B}, 2001, \textbf{64}, p. 045328

\end{thebibliography}

\end{document}

%\begin{table}[b]
%\processtable{Coefficients and remainders for distribution KK ($k = 0.05$,
%$v = 3$, $c_{1} = 1.5$, $c_{2} = 4.5$)}
%{\begin{tabular}{|l|l|l|}\hline
%$n$ & $a_{n}^{2}$ & $r_{k}(1)$\\\hline
%0 & 3.602576748428 & 1.493719547999\\\hline
%1 & 1.384791111989 & 0.108928436101\\\hline
%2 & 0.108600438794 & 0.000327997399\\\hline
%3 & 0.000275794597 & 0.000052202814\\\hline
%4 & 0.000027616892 & 0.000024585922\\\hline
%5 & 0.000018178621 & 0.000006407300\\\hline
%\end{tabular}}{}
%\end{table}
%
%So, the basic preamble and main body will be:
%\verb"\documentclass[twocolumn]{el-author}"\\
%\verb"\usepackage[...]{packages}"\\
%\verb"\date{12 December 2012}"\\
%\verb"\title{...}"\\
%\verb"\author{...}"\\
%\verb"\abstract{...}"\\
%\verb"\maketitle{...}"\\
%\verb"\begin{document}"\\
%\verb"..."\\
%\verb"\section{...}"\\
%\verb"..."\\
%\verb"\section{..}"\\
%\verb"..."\\
%\verb"\end{document}"
About
About us
Our values
Careers
Press & awards
Blog
Learn