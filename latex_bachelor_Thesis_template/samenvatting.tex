\samenvatting


Introduction: Whilst every study published in ascientific journal contains an abstract, little researchhas been done on the exact format, content andstyle with which an abstract should be written. Thismakes it difficult for authors to adequately sum-marise their work in an abstract.
Methods: In this study, the authors recruited acohort of medical students who had written at leastone scientific paper. Students were anonymouslysurveyed, on their confidence writing abstracts us-ing an online survey, maintaining confidentiality.However, this method may have been subjected toselection bias, where those who have completed ab-stracts but not written a full scientific paper may beexcluded. Use of online surveys may also contributeto selection bias, based on the fact that subject par-ticipation is voluntary and particular characteristicse.g. access to internet, whether the students viewthe site/email providing access to the questionnaire,time available for completion, etc., may differ perindividual and hence reduce the representativenessof the sample regarding the medical student popu-lation (The Writing Centre & University of NorthCarolina at Chapel Hill, n.d.).
Results:  73 students responded and the studyshowed that 37 % of students surveyed rated theirconfidence writing abstracts as ‘very poor,’ with afurther 42 % rating their confidence as ‘poor.’
Discussion: Based on the author’s results, it isclear that students need more guidance on how towrite abstracts. The authors recommend that allstudents wishing to learn how to write an abstractread the National Student Association for MedicalResearch ‘Anatomy of an Abstract’ article. How-ever, further controlled studies should be done toeliminate biases attributed to methodology in thiscohort study to truly determine whether medicalstudents lack confidence in writing abstracts.References:1. Nulty, D. D. (2018) The adequacyof response rates to online and paper surveys: whatcan be done?Assess Eval High Educ, 33(3), 301-14.doi: 10.1080/02602930701293231

Background: The writing and publication of re-search material by medical students is an area thatoccupies the time and efforts of the students them-selves, but does not yet have a large evidence base.Purpose: Consequently, it is important to under-take research that expands this body of knowledge.
Focus: This review aims to assess the confidenceof medical students in writing up abstracts for theirresearch, to gain a better overall picture of medicalstudents’ feelings about undertaking and writing upresearch.Word count: 81

Informative Abstract
Structured abstract includes the following heads: 
• Objectives: Illustrate the background and purpose of the review in one or two sentences in present tense.
• Material and Methods: Write a few lines to present a general picture of the research methodology of article in past tense.
• Result: Describe outcomes in few sentences. 

Abstract
There are two types of abstracts: one is informative abstract which describes the planned end product and result of the review manuscript or specifies the text structure. Second is descriptive abstract which describes the covered subject without specific details. Present tense will be used in the writing. Usually the length of abstract is 200 to 250 words.


Critical abstract
A critical abstract is generally written about a different au-thor’s work and contains all of the information mentionedabove, but also an element of evaluation or critical appraisalof the study, which may include discussion of the reliabilityand validity of the results (Labaree, 2018). For this purpose,references can be included to provide supporting evidence foryour arguments from relevant literature.The critical abstract includes information regarding thearticle e.g. author, title etc. and then briefly provides theirkey findings/conclusion. The main content of the abstractthen highlights the positives and negatives of the article.Examples of things to consider here could include:
•How relevant is this research question?
•Is the hypothesis clearly stated?
•Type of study/trial/research?
•What is the sample size? Is it large enough to providestatistically significant findings?
•Were the methods used appropriate and justified? Couldthey be improved?
•Is the conclusion valid based on the evidence?
•Are there any conflicts of interests?

Keywords

