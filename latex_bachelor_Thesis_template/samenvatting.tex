\samenvatting

Het doel van een samenvatting is om potentiële lezers zo snel mogelijk
te overtuigen van de relevantie van het verslag. Als afstudeerverslagen
gepubliceerd worden, is digitaal zoeken noodzakelijk. Daarom worden in
de samenvatting (\emph{abstract}) vaak kenmerkende woorden en
uitspraken opgenomen. Een samenvatting voor een afstudeerverslag mag
niet meer dan een paar honderd woorden bevatten.


A well-written discussion section includes a statement of important results, reference to previously published relevant literature, comparison of study results with previously reported findings, explanation of results, elucidations of strengths and weaknesses of the study, interpretation of the whole evidence, description of impact of the study and recommendations for the future course of action.


https://www.researchgate.net/publication/284896658_Writing_the_Discussion_Section_Describing_the_Significance_of_the_Study_Findings
https://www.researchgate.net/publication/324573047_Writing_a_discussion_section_How_to_integrate_substantive_and_statistical_expertise
https://www.ncbi.nlm.nih.gov/pmc/articles/PMC3474301/
https://www.sciencedirect.com/science/article/abs/pii/S0889490697853884


Prologue‘A well begun is half done’ Author must thinkbefore hand, about “How to write?” “What to write?” and“Where to submit?”. Having affirmed all of the above,with the data of a well conducted and concluded researchproject in hand, author must think of a “clear message”intended to be given through his write up. A good measureof success is the conclusions drawn from the study, if canbe written in one meaningful sentence.The others considerations to be decided priorly are
i) What is the best format of presentation of the researchdone? eg: as  original article, review, case report, orcorrespondence,? because format is different fordifferent type of articles.
ii) Target audience for the publication and whichjournal?: Aspiring authors will improve their chanceof acceptance if they choose an appropriate journalfor their topic and adhere to conventional rules. Thereason why this decision must be taken in the earlyphases is that from the first draft, the paper must bewritten in the style and format of the specific, journaltargeting particular group of audience.
iii) A thorough literature search is quite essential : 
a) toidentify the knowledge gaps in the existing informationand the proposed paper may be aimed to fill themup. 
b) to avoid duplication if the same message orproject has been published already.  Most journalsdo not wish to consider for publication a paper orwork that has already been reported in a publishedpaper.
iv) Other matters related to authorship, ethical,  andstatistical clearance may be obtained well in advance.


I)  Title1) Title should correctly represent the content andbreadth of the study reported and should  not bemisleading.For example “comparative evaluation of  Propofol– Ketamine and Propofol Fentanyl  in minorSurgery”. On reading the title, we can not know thecontent and breadth of the study; whether dosage,duration, efficiency, and sequalae, of two group arestudied or not whether they are studied as onlyinduction agents or as sole anaesthetic agents; whatgroup of patients? None of the information can behad from this title.

2) It should be clear, concise, and informative. Itshould contain keywords, that capture attention ofthe reader. No abbreviations are used in the title.The decision to read an article often  rests on theappeal of its title.      A More appropriate title could be –“Comparativeevaluation of efficiency of Propofol – Ketamineand Propofol – Fentanyl combination as soleanaesthetic agents in patients undergoing minorambulatory  gynecological operations”.
II) Author
3) Designation,  degree, affiliation and address ofauthors are to be clearly indicated, with additionaldetails like telephone   number, email  address ofthe corresponding author.
III) Abstract  & Keywords
4) Abstract should cover each and every componentof, the study in 150 words for ‘unstructured’ abstractsand 250  words for ‘structured’  abstracts. It shouldstate the purpose of the study or investigation,basic procedures, (selection of study subjects,methodology, main findings, statistical significance),the principal conclusion and implications.
5) The abstract should contain precise information andshould not contain abbreviations.
6) The implications and benefits should commensuratewith the results obtained, and are to be highlighted.
7) Key words (or short phrases) 3 to 10, should belisted covering all the aspects of the study. Usepreferably the terms listed as Medical subjectheadings (MESH) in Index Medicus (Medline)
IV) Introduction and Review of Literature
8) The goal or purpose of the study is clearly stated.The introduction should contain detailed informationabout the problem being studied,   and about thespecific research question/hypothesis.
9) Four or five pertinent publications related to theproblem should be presented and critiqued.  Nodata or conclusions are to be reported.
10) Do not review the literature  extensively.
11) The pertinence of the study is  presented,  inrelation to the current theories and methodsassociated with the problem. The existing gaps  inthe knowledge or conflicting data is to behighlighted.
12) A general overview of the study is presented. Overview serves as   organiser for the sections to followto the reader.


V) Material and Method
13) The selection of the subjects for the study has to bedescribed clearly. Inclusion and exclusion criteriaare to be mentioned with method of allocation togroups.
14) The research design is to be described in detail.Research design is the plan that is chosen to answerthe research question.  The methods, apparatus  andprocedures are to be identified  in sufficient detail toallow other workers to reproduce the results, ifnecessary.
15) Give references of  all the methods used in the studyincluding statistical methods.
16) Identify precisely all drugs and chemicals used,including generic names, doses and routes ofadministration.
17) Methods of elimination of errors viz blinding,introduction of control group and placebo,randomization etc are to be mentioned distinctly.
18) The measurement instrument including itspsychometric qualities is described clearly.  Thepsychometric qualities include validity, reliability,objectivity  and precision.  An example of theinstrument should be gives in the text or in anappendix.For example in the above mentioned study, if ‘homereadiness’ is intended to be studied, the ‘PostAnaesthetic Discharge scoring system’ (PADS)utilised in the study has to be a reliable, and anaccepted one for its objectivity and precision.
19) The data collection procedure is to be clearlydescribed.
20) The setting in which the study took place is described.This information is useful to the reader in decidingwhether results can be applied to his/her setting.
21) The data analysis procedures are stated in preciseterms.

VI) Results
22) Present your results in logical sequence  in the text,tables and illustrations.  Do not repeat in the text allthe data, in the tables or illustrations.
23) Emphasize or summarise important observations.Results  section should contain only  actuals, and noopinions.
24) All the patients included in the study should beaccounted for. There should not be any hesitation inreporting any negative or unexpected result

VII) Discussion & Conclusion
25) The discussion should cover all the debatable  aspectsof the study. The discussion  can go beyond theresults obtained and can cover methodological andthe critical issues. The discussion should not bemisused as a platform to state opinions. Readersshould not be side tracked in to another topic.
26) Relate the observations to the other relevant studies.Bring out similarities and conflicts.
27) The new and important aspects of the study and theconclusions drawn are to be emphasized.  Theimplications  of the findings and their limitations areto be discussed.For example if you find that  Propofol – kelaminecombination fared well except that there was‘excitatory phenomenon’ of Ketamine observed inthese group of patients, this limitation has to bementioned without fail.
28) Scope  and need for future additional research is tobe  discussed.
29) Link conclusions with goals of the study but avoidunqualified statements and conclusions not supportedby your data.
30) State new hypothesis when warranted .Recommendations when appropriate may be included.For eg Propofol does not have any effect on excitatoryphenomenon associated with Ketamine.
31) The conclusions and practical outcomes of the studyshould commensurate with the design used and resultsobtained.  The conclusions and recommendationsmade should not go beyond the limits of the studyconducted i.e. should not over generalize the designand sample used.Suppose the haemodynamics were stable in Ketamine-Propofol group as compared to Propofol – Fentanylgroup, one should not generative that the combinationis recommended for patients with cardiovasculardiseases



Viii) References
32) This is the most disturbing aspect amongst the Indianpublications. It is a wrong notion amongst Indianauthors that providing a long list of referencesincreases the validity (of their article) which is wrong.References are to be written correctly with due care.Correct abbreviated, accepted names, of the journalsto be mentioned. The number of references should
be reasonable (neither too many nor too few); Somejournals specify the number of references to beincluded in a particular type of study.
33) Avoid using ‘abstracts’ as references.  The referencesmust be verified by the author against the originaldocuments.
34) The references are presented according to standardrules of publication as specified by a particularjournal. for eg, whether Vancour style or Harwardstyle is followed.General  Considerations
35) The various sections of the paper are clearlyidentifiable and appropriate. The content of eachsection should correspond to the subtitle used, forinstance, there is no ‘Discussion’ in the ‘Results’section. The transition from one section to nextshould be easy to follow.
36) The terminology has to be uniform through out thepaper. For eg.  abbreviations  should be consistentand units of measurements should be the same inthe text as in tables.
37) The writing style has to be clear and pleasant.  Thereshould not be spelling mistakes. Special care isneeded in following British Vs American spellings.Text is, generally written is passive voice.  Uniform‘tense’ has to be used.
38) Follow the instructions of the journal, you arewriting regarding tables, graphs illustrations, thetext matter, type of  manuscripts etc. to be used inthe article.
39) Follow the ethical guidelines strictly as specifiedby ICMJE. If there is confusion as what is ethicaland what is non ethical and there is no ethicalcommittee to guide, ‘a self test’ may be employed.Ask yourself whether you will be conducting thesimilar study  on your kith and kin. If yes, goahead with your study.

40)  All the direct and  indirect help in the study hasto be acknowledged, without fail.Editors and referees ……………………. but are busypeople whose humanitarian instincts should not be abused;and it is better for all concerned if authors try to submitpapers that are in good working order5 
https://www.researchgate.net/publication/265059173_How_To_Write_A_Scientific_Article_For_A_Medical_Journal

intriductie
Waarom chrijf je dit artikel en waarom nu?
Voor wie is dit artikel bedoeld?
Welk probleem probeer je naar voren te brengen en wat is de achtergrond, wat is de eerste hypothese?
Methods
Hoe is de studie uitgevoerd
Welke materialen zijn gebruikt of wat voor type product is onderzocht
Results
Wat is er gevonden
Hoeveel kan je samenvoegen
Wat kan je laten zien in een tabel en wa in tekst
Discussion
Wat zijn de sterke punten en zwakke punten van de studie
Hoe passen de bevindingen in andere geplubiceerd materieaal
Wat nu, wat is de volgende stap in je onderzoek en kon je met de hypothese de test doorstoom of moest je deze aanpassen

https://www.semanticscholar.org/paper/Kipling%27s-guide-to-writing-a-scientific-paper.-Sharp/d992698098f36dbf33c8ba0c56b6ee92e3219f1b/figure/0
https://www.semanticscholar.org/paper/Kipling's-guide-to-writing-a-scientific-paper.-Sharp/d992698098f36dbf33c8ba0c56b6ee92e3219f1b
https://www.researchgate.net/publication/310912880_How_to_Write_a_Scientific_Research_Paper_International_Journal_of_Research_IJR_e-ISSN_2348-6848_p-_ISSN_2348-795X_Volume_2_Issue_05_May_2015
https://www.researchgate.net/publication/269416739_Publish_or_perish_The_art_of_scientific_writing


ResultsThe results section consists of the organised presentation of the collected data. All measurements that the authors described in the materials and methods section must be reported in the results section and be presented in the same order as they were in that section35. The past tense should be used as results were obtained in the past. Author(s) must ensure that they use proper words when describing the relationship between data or variables. These "data relation words" should be turned into "cause/effect logic and mechanistic words" in the discussion section. A clear example of the use of this appropriate language can be found in the article by O'Connor35.This section should include only data, including negative findings, and not background or methods or results of measurements that were not described in the methods section2. The interpretation of presented data must not be included in this section.Results for primary and secondary outcomes can be reported using tables and figures for additional clarity. The rationale for end-point selection and the reason for the non-collection of information on important non-measured variables must be explained35.Figures and tables should be simple, expand text information rather than repeat it, be consistent with reported data and summarise them23. In addition, they should be comprehensible on their own, that is, with only title, footnotes, abbreviations and comments.

References in this section should be limited to methods developed in the manuscript or to similar methods reported in the literature.Patients' anonymity is essential unless consent for publication is obtained.
https://www.researchgate.net/publication/318761484_Components_and_Structure_of_a_Manuscript

Results SectionThe Results section is the meat of a paper, the most important partof a study. All other sections serve subordinate roles, either preparing thereader for the Results, or providing supplemental information to augmentthe findings (Yang, 1999, p. 63). Sometimes the Results and Discussionare combined into one section. This is particularly useful when preliminarydata must be discussed to show why subsequent data were taken. In thefollowing discussion, Results and Discussions are treated separately.


Results are general statements that present the key results (data)of the research without interpreting their meaning. The author should notinclude the raw data, but should present them as text, illustrations, andtables. All these three forms may be used, but the same data should notbe repeated in more than one form. The results of statistical analysesshould also be stated in this section, but every detail of the analysis shouldbe excluded for the readers are assumed to have known what a nullhypothesis is, a rejection rule, t-test, chi-square test, etc.The text describing data may be any length. However, a briefstatement such as, “The distribution of the respondents’ interest in shortstories are shown in Table 1,” is sufficient. For clarity, long passages oftext are often organized by topic into subsections, with a subheading foreach topic. The subheadings assist the reader to trace paragraphsinteresting to them.The followings are important guidelines to consider in writing theResults section:1.  It is not necessary to include all the collected data during the research.This isn't a diary. Select and emphasize only important and relevantdata that will answer the question or solve the problem raised in theIntroduction section.2.  Do not include information properly belonging to other sections of thepaper such as Materials and Methods, or Discussions (if Results andDiscussions are separated).3.  Prevent repeating the legends for figures or the titles of tables in thetext.4. Explain in the text only those illustrations and tables whosesignificance is not obvious to the reader. Important features that arereadily apparent from the illustrations and tables should be pointed outin the text. Therefore, do not repeat the data presented in theillustrations and tables.


5.  Be sure that the text, illustrations, and tables are consistent with oneanother. Make sure that all numerical values in all every table agreewith the figures or data presented in it.6.  Analyze your data by statistical methods, if appropriate.7.  Be honest. Do not omit data that do not support your hypothesis andconclusion or do not answer the research question.8.  A sentence should not begin with a numeral or symbol. A numeral orsymbol beginning a sentence should be spelled out, or the sentencerewritten.9.  Use the past tense of verbs in the Results section, but use the presenttense when referring to figures and tables. For instances,Seventy per cent respondents said that they got appropriateopportunity for improving speaking in speaking classes, whereas theother 30% denied in this connection.Table 2 reveals the students’ perception collected during the secondcycle of the action research.But,The data in Table 2 were collected during the second cycle of theaction research.10. Do not begin numbers in a sentence with a decimal point. Decimalfractions less than 1 should be written with the numeral 0 before thedecimal point. For instance, instead of writing, “The result of statisticalanalysis was (r) = .619,” you should write, “The result of statisticalanalysis was (r) = 0.619.”


https://www.researchgate.net/publication/260453687_SCIENTIFIC_ARTICLES_STRUCTURE
https://www.researchgate.net/publication/235378471_How_to_write_a_scientific_manuscript_for_publication

The title page
The authors
Institutions
Running title, keywords, word count and correspondence
address
Second title page
Abstract
Introduction
Definition
Fundamentals
The short review
Shortcomings of the existing studies
The aim of the study
Scope of the study
Evaluating the introduction
Patients (or materials) and methods
Fundamentals
Patients or materials
Methods
Design of the study
Statistical methods
Ethical considerations
Evaluation of the methods section
Results
Fundamentals
Analysis of the results
Evaluation of the results
The discussion
Fundamentals
Introduction
Discussion of the results
Advantages and disadvantages of the study
Recommendations by the authors
Evaluation of the discussion
The summary
The references


https://www.researchgate.net/publication/259525553_How_to_write_a_medical_original_article_Advice_from_an_Editor

https://www.researchgate.net/publication/264233797_Writing_and_Publishing_a_Scientific_Paper

https://www.researchgate.net/publication/15449826_Analysis_of_the_structure_of_original_research_papers_An_aid_to_writing_original_papers_for_publication
https://www.researchgate.net/publication/13067810_The_case_for_structuring_discussion_of_scientific_papers
https://www.researchgate.net/publication/23164455_Basic_structure_and_types_of_scientific_papers
https://www.researchgate.net/publication/332396285_How_to_write_the_discussion_section_of_a_scientific_article








https://www.cas.mcmaster.ca/~lawford/papers/ImplementabilityOf4VarSCP2015.pdf

https://lib.dr.iastate.edu/cgi/viewcontent.cgi?referer=https://www.google.com/&httpsredir=1&article=1069&context=cs_techreports

https://citeseerx.ist.psu.edu/viewdoc/download?doi=10.1.1.386.855&rep=rep1&type=pdf

https://www.uni-due.de/imperia/md/content/swe/papers/icsoft16a.pdf

https://www.researchgate.net/publication/220131706_Functional_Documents_for_Computer_Systems


https://personal.utdallas.edu/~chung/SYSM6309/RequirementsModels.pdf



1	BIJLAGE E


https://www.scribbr.nl/scriptie-structuur/theoretisch-kader-van-scriptie/
https://www.scribbr.nl/scriptie-structuur/voorbeeld-theoretisch-kader/
https://docplayer.nl/7419035-Hoofdstuk-2-theoretisch-kader-15.html
https://pure.uva.nl/ws/files/3705146/50412_A01_211_007.pdf
https://pure.uva.nl/ws/files/1140621/108913_UBA003000208_006.pdf
https://plos.org/resource/how-to-write-conclusions/













scriptie conceptueel model
https://www.scribbr.nl/scriptie-structuur/theoretisch-kader-van-scriptie/
https://www.scribbr.nl/scriptie-structuur/theoretisch-kader-van-scriptie/
https://www.scribbr.nl/scriptie-structuur/theoretisch-kader-van-scriptie/
https://www.scribbr.nl/scriptie-structuur/voorbeeld-theoretisch-kader/
https://www.scribbr.nl/scriptie-structuur/hoe-doe-je-literatuuronderzoek/
https://www.scriptiehulpverlening.nl/tips-en-links/meest-gestelde-scriptievragen/hoe-maak-ik-een-conceptueel-model/
https://afstudeerbegeleider.nl/scriptie/theoretisch-kader/
https://www.topscriptie.nl/theoretisch-kader-schrijven/
https://www.scriptium.nl/theoretisch-kader-scriptie/
https://24editor.com/scriptie-theoretisch-kader/
http://www.afstudeersucces.nl/index.php/theoretisch-kader/
https://www.studiemeesters.nl/scriptie/hoe-schrijf-je-een-theoretisch-kader-trechtervorm/
https://www.afstudeergoeroes.nl/onderzoek-en-afstuderen/plan-van-aanpak/onderdelen-van-plan-van-aanpak/9-het-theoretisch-kader/
https://focusopafstuderen.nl/scriptietips/opbouw-van-een-theoretisch-kader/
https://handboeksgpl.sites.uu.nl/verslaglegging/theoretisch-kader/
https://www.studeersnel.nl/nl/document/ncoi-opleidingen/literatuuronderzoek-theoretisch-kader/eindopdracht-literatuuronderzoek-theoretisch-kader/8468334
https://www.scribbr.nl/scriptie-structuur/aanbevelingen-in-je-scriptie/


https://hbo-kennisbank.nl/searchresult?t-0-k=hbo%3Aproduct&has-link=yes&c=2&q=software+development&p=5&t-0-v=info%3Aeu-repo%2Fsemantics%2FbachelorThesis

https://hbo-kennisbank.nl/details/sharekit_hz:oai:surfsharekit.nl:9630d2df-e5c5-4046-9cf0-278835f084b6?q=software&has-link=yes&t-0-k=hbo%3Aproduct&c=2&t-0-v=info%3Aeu-repo%2Fsemantics%2FbachelorThesis
file:///C:/Users/gally/Downloads/file_2e78434f-612f-480c-8954-0fc92f564063_30.05.2013_Final_report_software_for_the_automotive_video_generating_hardware_platform.pdf
https://hbo-kennisbank.nl/details/sharekit_hz:oai:surfsharekit.nl:9ca7c253-e69a-47ba-9bc0-c4bdc751eb0a?q=software&has-link=yes&t-0-k=hbo%3Aproduct&c=2&t-0-v=info%3Aeu-repo%2Fsemantics%2FbachelorThesis
file:///C:/Users/gally/Downloads/file_06517dbf-7d6b-4c15-bd24-3e3351258eec_V1.0_Final_report.pdf
https://hbo-kennisbank.nl/details/sharekit_fontys:oai:surfsharekit.nl:fa1cdfcb-5ebd-49c5-8ad8-8653fb539ef3?q=software+development&has-link=yes&t-0-k=hbo%3Aproduct&c=2&t-0-v=info%3Aeu-repo%2Fsemantics%2FbachelorThesis
file:///C:/Users/gally/Downloads/file_6a0db9b2-c93d-4d27-9edf-870e3def14cb_SLD-2014-340744-Afstuderen-Scriptie-DuivisColin-C.-2192461.pdf
https://hbo-kennisbank.nl/details/sharekit_hh:oai:surfsharekit.nl:aaab2a9e-2c9e-4458-8225-c75dc47077da?q=software+development&has-link=yes&t-0-k=hbo%3Aproduct&c=2&t-0-v=info%3Aeu-repo%2Fsemantics%2FbachelorThesis
file:///C:/Users/gally/Downloads/file_e58b629b-6f7f-4b56-ae45-b350dab1c5eb_afstudeerverslag_dennisvangilst.pdf
https://hbo-kennisbank.nl/details/sharekit_hu:oai:surfsharekit.nl:0c2dc5de-76bf-43bb-89ef-0ab6fccfbbb9?t-0-k=hbo%3Aproduct&has-link=yes&c=2&q=software+development&p=3&t-0-v=info%3Aeu-repo%2Fsemantics%2FbachelorThesis
https://hbo-kennisbank.nl/details/sharekit_hz:oai:surfsharekit.nl:e0e2ad78-d2a8-4575-8282-3930f49e4f43?t-0-k=hbo%3Aproduct&has-link=yes&c=2&q=software+development&p=3&t-0-v=info%3Aeu-repo%2Fsemantics%2FbachelorThesis
file:///C:/Users/gally/Downloads/file_9164d515-255c-43b5-9d19-85945c76dafb_Final-Report-Yongmin-Qiu.pdf











IntroductionWhich is the main theme of the study?What is already known about the theme?What is not yet known about the theme?What are the objectives of the research?Are the objectives clear and well defined?Organize  Introduction  in  a  way  that  the  sequence  of  ideas  is  evident. The text should be informative, concise, and encourage the continuity of reading. 

MethodsWhat is the design of the study? Which is the population of the study (including studied groups and socio-demographic characterization)?Which were the inclusion and exclusion criteria considered?Which were the materials and procedures used?How was the data analysis conducted (including studied variables and statistical tests used to answer each objective, level of significance adopted, and possible transformations applied to the data)?Which were ethical procedures conducted?Write the Methods section in a way that allows its reproduction by other researchers. 

ResultsWhich  results  should  be  presented  to  answer  each  objective  of  the study?What  is  the  most  appropriate  way  to  summarize  each  result,  emphasizing the main findings (text, tables and/or figures)?Which statistical results should be presented to provide credibility to the findings?Besides  numerical  data,  present  a  brief  conclusion  about  the  results, in order to summarize the main findings. Data should not be discussed in this section.

DiscussionWhich are the main answers to the objectives of the study?How are the findings related to those of previous studies found in literature? How do they answer the gap in knowledge evidenced in the Introduction?What are the clinical and scientific implications of the study?What are the limitations of the study?What are the perspectives of future studies on the theme, based on the results and limitations of the present study?The  authors  should  try  to  position  themselves  in  relation  to  the  findings discussed, for this is what determines the contribution of the study to Science.

ConclusionWhat specific results answer to the objectives of the study?What is the novelty found in the results?Write  the  Conclusion  in  one  concise  and  accurate  paragraph,  sticking to the answer. 

AbstractIn a clear and concise manner, what is the objective of the study?What are the essential methodological information that support the results and the conclusion?Which results answer the objective presented?What is the conclusion that answers the objective presented? The abstract is the advertisement of your study. Write it in a clear, reliable, and attractive manner.TitleWhich are the relevant items to attract attention from the intended public?How do the relevant items should be put in order to, in a brief and informative manner, attract attention from readers?The title is the manner by which possible readers will seek to learn about your study. Carefully choose the words and the message you intend to transmit. 

1. Objective:  the  exact  question(s)  addressed  by  thearticle.
2. Design: the basic design of the study.3. Setting: the location and level of clinical care.
4. Patients or Participants: the manner of selection andnumbers of patients or participants who entered andcompleted the study.
5. Interventions: the exact treatment or interventions,if any.
6. Measurements and Results: the methods of assessingpatients and key results.
7. Conclusions: key conclusions including direct clinicalapplications.

1. Purpose:  the primary objective of the review.
2. Data identication: a succinct summary of data sources.
3. Study selection: the number of studies selected forreview and how they were selected.
4. Data  extraction:  the  type  of  guidelines  usedabstracting data and how they were applied.
5. Results  of  data  synthesis:  the  methods  of  datasynthesis  and  key  results.
6. Conclusions:  key  conclusions,  including  potentialapplications and research needs.


Purpose: To ascertain the clinical benefits of digitalistreatment  in  patients  with  chronic  congestive  heartfailure and sinus rhythm.
Data  identification:  An  English-language  literaturesearch using MEDLINE (1966-82), Index Medicus (1960-65), and bibliographic review of textbooks and reviewarticles.
Study  selection:  After  independent  review  by  threeobservers, 16 of 736 originaly identified articles wereselected tha specifically addressed the stated purpose.Data  extraction:  Three  observers  independentlyassessed studies using explict methodologic criteria for evaluating the quality of clinical trials.
Results of data synthesis: Because of deficient selectioncriteria  and  study  methods  in  14  studies,  therapeuticefficacy  could  not  be  adequately  assessed.  Tworandomized, double-blind, placebo-controlled studiessuggested  that  digitalis  could  be  successfullywithdrawn from elderly patients with stable hear failure,whereas patients with a S3 gallop might benefit fromdigitalis.
Conclusions:  The  benefits  of  digitalis  treatment  forpatients with congestive heart failure and synus rhythimare  not  well  established.  To  better  delineate  thetherapeutic  benefits  of  digitalis,  investigators  mustconduct  more  rigorously  designed  trials  involvingpatients  with  newly  diagnosed  failure  and  varyingdegrees of failure.


Original articles1. Objective: the exact question (s) adressed by thearticle.2. Design:  the basic design of the study.
3. Setting: the location and level of clinical care.4. Patients  or  participants:  the  manner  of  selectionand  the  number  of  patients  or  participants  whoentered and completed the study.5. Interventions: the exact treatment or intervention,if any.
6. Main  outcome  measures:  the  primary  studyoutcome  measured  as  planned  before  datacollection began.
7. Results: the key findings.
8. Conclusions:  key  conclusions  including  directclinical applications

Review articles
1. Purpose: the primary objective of the review.
2. Data sources: a succint summary of data sources. 
3. Study selection: the number of studies selected forreview and how they are selected.
4. Data extraction: rules for abstracting data and howthey were applied.
5. Results  of  data  synthesis:  the  methods  of  datasynthesis  and  key  results.Conclusions:  key  conclusions,  including  potentialapplications and research needs.

Objective: To evaluate the safety and immunogenicityin  adults  of  several  different  concentrations  of  anacellular pertussis vaccine.
Design: Double-blind, randomized, placebo-controlled trial.Setting: Medical center immunization clinic.Participants:  One  hundred  eighteen  healthy  adultvolunteers.
Interventions:  Participants  received  standard  adulttetanus-diphtheria vaccine alone or combined with full-strength,    half-strength,    or    quarter-strengthconcentrations of a currently licensed acellular pertussisvaccine used for booster doses in young children. Full-strength vaccine contained 40 micrograms of pertussisproteins, consisting of 86% filamentous hemagglutinin,8% pertussis toxin, 4% 69-kd outer-membrane protein,and 2% agglutinogens.Main outcome measures: Local and systemic reactionswere  assessed  for  14  days  after  vaccination.  Serumsamples  for  antibody  assay  were  obtained  before,  1month after, and 1 year after immunization.
Results: Adverse reactions were few and minor and didnot differ in frequency or severity among the four studygroups.  The  groups  receiving  acellular  pertussisvaccine showed strong antibody responses to pertussisantigens,  which  did  not  significantly  differ  byconcentration  of  vaccine.  After  1  year,  levels  ofantibody  to  pertussis  had  declined  by  approximately50%  but  remained  substantially  higher  thanpreimmunization levels. The four groups did not differin antibody responses to tetanus or diphtheria toxoids.
Conclusions: Routine reimmunization of adults with avaccine  containing  acellular  pertussis  antigens  inaddition  to  diphtheria  and  tetanus  toxoids  cansubstantially enhance pertussis antibody levels withoutan  increase  in  adverse  reactions  or  diminution  inresponse  to  the  diphtheria  and  tetanus  components.Such  a  program  might  materially  reduce  respiratoryillness among both adults and children.

Objective:    To  determine  the  relative  exposure  toenvironmental  tobaccosmoke  for  bar  and  restaurantemployees compared with office employees andwithnonsmokers  exposed  in  the  home  (part  1)  and  to determine whether thisexposure is contributing to anelevated lung cancer risk in these employees(part 2)Data  sources:  MEDLINE  and  bibliographies  fromidentifiedpublications.
Study selection: 
In part 1, published studies of indoorairquality  were  included  if  they  reported  a  meanconcentration  of  carbonmonoxide,  nicotine,  orparticulate matter from measurements taken in one ormore  bars,  restaurants,  offices,  or  residences  with  atleast  one  smoker.  
Inpart  2,  published  epidemiologicstudies  that  reported  a  risk  estimate  forlung  cancerincidence  or  mortality  in  food-service  workers  wereincluded  ifthey  controlled,  directly  or  indirectly,  foractive smoking.
Data extraction: In part 1, a weighted average of themean concentration ofcarbon monoxide, nicotine, andrespirable  suspended  particulates  reportedin  studieswas  calculated  for  bars,  restaurants,  offices,  andresidences.In part 2, the relative lung cancer risk forfood-service workers comparedwith that for the generalpopulation was examined in the six identifiedstudies.
Data synthesis: Levels of environmental tobacco smokeinrestaurants were approximately 1.6 to 2.0 times higherthan in officeworkplaces of other businesses and 1.5times higher than in residences withat least one smoker.Levels in bars were 3.9 to 6.1 times higher than inofficesand  4.4  to  4.5  times  higher  than  in  residences.  Theepidemiologicevidence suggested that there may be a50% increase in lung cancer riskamong food-serviceworkers  that  is  in  part  attributable  to  tobacco  smokeexposure in the workplace.Conclusions:  Environmental  tobacco  smoke  is  asignificant occupational health hazard for food-serviceworkers. To protectthese workers, smoking in bars andrestaurants should be prohibited.

venous  thromboembolism  and  the  incidence  andseverity  of  post-thrombotic  sequelae  have  not  beenwell documented.
Objective: To determine the clinical course of patientsduring the 8 years after their first episode of symptomaticdeep  venous  thrombosis.
Design: Prospective cohort study.Setting: University outpatient thrombosis clinic.Patients: 355 consecutive patients with a first episodeof symptomatic deep venous thrombosis.
Measurements:  Recurrent  venous  thromboembolism,the post-thrombotic syndrome, and death. Potential riskfactors for these outcomes were also evaluated.
Results: The cumulative incidence of recurrent venousthromboembolism was 17.5% after 2 years of follow-up(95% CI, 13.6% to 22.2%), 24.6% after 5 years (CI, 19.6%to 29.7%), and 30.3% after 8 years (CI, 23.6% to 37.0%).The  presence  of  cancer  and  of  impaired  coagulationinhibition  increased  the  risk  for  recurrent  venousthromboembolism (hazard ratios, 1.72 [CI, 1.31 to 2.25]and 1.44 [CI, 1.02 to 2.01], respectively). In contrast,surgery and recent trauma or fracture were associatedwith  a  decreased  risk  for  recurrent  venousthromboembolism (hazard ratios, 0.36 [CI, 0.21 to 0.62]and 0.51 [CI, 0.32 to 0.87], respectively). The cumulativeincidence of the post-thrombotic syndrome was 22.8%after 2 years (CI, 18.0% to 27.5%), 28.0% after 5 years(CI, 22.7% to 33.3%), and 29.1% after 8 years (CI, 23.4%to 34.7%). The development of ipsilateral recurrent deepvenous  thrombosis  was  strongly  associated  with  therisk for the post-thrombotic syndrome (hazard ratio, 6.4;CI, 3.1 to 13.3). Survival after 8 years was 70.2% (CI,64.7% to 75.6%). The presence of cancer increased therisk for death (hazard ratio, 8.1; CI, 3.6 to 18.1).
Conclusions: Patients with symptomatic deep venousthrombosis,  especially  those  without  transient  riskfactors for deep venous thrombosis, have a high risk for recurrent venous thromboembolism that persists formany years. The post-thrombotic syndrome occurs inalmost one third of these patients and is strongly relatedto ipsilateral recurrent deep venous thrombosis. Thesefindings  challenge  the  widely  adopted  use  of  short-course  anticoagulation  therapy  in  patients  withsymptomatic deep venous thrombosis.

Background: Obesity is a major, growing health problem.Observational studies suggest that bariatric surgery ismore  effective  than  nonsurgical  therapy,  but  norandomized, controlled trials have confirmed this.
Objective:  To  ascertain  whether  surgical  therapy  forobesity achieves better weight loss, health, and qualityof life than nonsurgical therapyDesign: Randomized, controlled trial.Setting:  University  departments  of  medicine  andsurgery and an affiliated private hospital.Patients: 80 adults with mild to moderate obesity (bodymass index, 30 kg/m2 to 35 kg/m2) from the generalcommunity.Interventions: Patients were assigned to a program ofvery-low-calorie diets, pharmacotherapy, and lifestylechange  for  24  months  (nonsurgical  group)  or  toplacement  of  a  laparoscopic  adjustable  gastric  band(LAP-BAND System, INAMED Health, Santa Barbara,California) (surgical group).
Measurements: Outcome  measures  were  weightchange,  presence  of  the  metabolic  syndrome,  andchange in quality of life at 2 years.
Results:  At  2  years,  the  surgical  group  had  greaterweight loss, with a mean of 21.6% (95% CI, 19.3% to23.9%) of initial weight lost and 87.2% (CI, 77.7% to96.6%)  of  excess  weight  lost,  while  the  nonsurgicalgroup had a loss of 5.5% (CI, 3.2% to 7.9%) of initialweight and 21.8% (CI, 11.9% to 31.6%) of excess weight(P < 0.001). The metabolic syndrome was initially presentin 15 (38%) patients in each group and was present in 8(24%) nonsurgical patients and 1 (3%) surgical patientat the completion of the study (P < 0.002). Quality of lifeimproved statistically significantly more in the surgicalgroup (8 of 8 subscores of Short Form-36) than in thenonsurgical group (3 of 8 subscores).
Limitations: The study included mildly and moderately obese participants, was not powered for comparison ofadverse  events,  and  examined  outcomes  only  for  24months.
Conclusions:  Surgical  treatment  using  laparoscopicadjustable gastric banding was statistically significantlymore  effective  than  nonsurgical  therapy  in  reducingweight,  resolving  the  metabolic  syndrome,  andimproving quality of life during a 24-month treatmentprogram.


Introduction: Whilst many medical students be-come involved in some form of research during theirmedical school careers, there is often little formalguidance on how to write this research up into apaper that is suitable for publication.
Methods: In this study, we recruited a cohort ofmedical students who had written at least one sci-entific paper. Students were anonymously surveyedon their confidence writing abstracts using an onlinesurvey.
Results:  73 students responded and the studyshowed that 37 % of students surveyed rated theirconfidence writing abstracts as ‘very poor’, with afurther 42 % rating their confidence as ‘poor.’
Discussion: Based on these results, it is clear thatstudents need more guidance on how to write ab-stracts. 
The authors recommend that all studentswishing to learn how to write an abstract read theNational Student Association for Medical Research‘Anatomy of an Abstract’ article.

Critical abstract
A critical abstract is generally written about a different au-thor’s work and contains all of the information mentionedabove, but also an element of evaluation or critical appraisalof the study, which may include discussion of the reliabilityand validity of the results (Labaree, 2018). For this purpose,references can be included to provide supporting evidence foryour arguments from relevant literature.The critical abstract includes information regarding thearticle e.g. author, title etc. and then briefly provides theirkey findings/conclusion. The main content of the abstractthen highlights the positives and negatives of the article.Examples of things to consider here could include:
•How relevant is this research question?
•Is the hypothesis clearly stated?
•Type of study/trial/research?
•What is the sample size? Is it large enough to providestatistically significant findings?
•Were the methods used appropriate and justified? Couldthey be improved?
•Is the conclusion valid based on the evidence?
•Are there any conflicts of interests?


Introduction: Whilst every study published in ascientific journal contains an abstract, little researchhas been done on the exact format, content andstyle with which an abstract should be written. Thismakes it difficult for authors to adequately sum-marise their work in an abstract.
Methods: In this study, the authors recruited acohort of medical students who had written at leastone scientific paper. Students were anonymouslysurveyed, on their confidence writing abstracts us-ing an online survey, maintaining confidentiality.However, this method may have been subjected toselection bias, where those who have completed ab-stracts but not written a full scientific paper may beexcluded. Use of online surveys may also contributeto selection bias, based on the fact that subject par-ticipation is voluntary and particular characteristicse.g. access to internet, whether the students viewthe site/email providing access to the questionnaire,time available for completion, etc., may differ perindividual and hence reduce the representativenessof the sample regarding the medical student popu-lation (The Writing Centre & University of NorthCarolina at Chapel Hill, n.d.).
Results:  73 students responded and the studyshowed that 37 % of students surveyed rated theirconfidence writing abstracts as ‘very poor,’ with afurther 42 % rating their confidence as ‘poor.’
Discussion: Based on the author’s results, it isclear that students need more guidance on how towrite abstracts. The authors recommend that allstudents wishing to learn how to write an abstractread the National Student Association for MedicalResearch ‘Anatomy of an Abstract’ article. How-ever, further controlled studies should be done toeliminate biases attributed to methodology in thiscohort study to truly determine whether medicalstudents lack confidence in writing abstracts.References:1. Nulty, D. D. (2018) The adequacyof response rates to online and paper surveys: whatcan be done?Assess Eval High Educ, 33(3), 301-14.doi: 10.1080/02602930701293231

Background: The writing and publication of re-search material by medical students is an area thatoccupies the time and efforts of the students them-selves, but does not yet have a large evidence base.Purpose: Consequently, it is important to under-take research that expands this body of knowledge.
Focus: This review aims to assess the confidenceof medical students in writing up abstracts for theirresearch, to gain a better overall picture of medicalstudents’ feelings about undertaking and writing upresearch.Word count: 81

Informative Abstract
Structured abstract includes the following heads: 
• Objectives: Illustrate the background and purpose of the review in one or two sentences in present tense.
• Material and Methods: Write a few lines to present a general picture of the research methodology of article in past tense.
• Result: Describe outcomes in few sentences. 

Abstract
There are two types of abstracts: one is informative abstract which describes the planned end product and result of the review manuscript or specifies the text structure. Second is descriptive abstract which describes the covered subject without specific details. Present tense will be used in the writing. Usually the length of abstract is 200 to 250 words.


Keywords

