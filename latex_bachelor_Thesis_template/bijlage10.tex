\hoofdstuk{Verificaie extra}

    De safety en reachability requirements die formeel zijn gespecificeerd worden in Uppaal geverifieerd met de A en E state formule. Andrerere opreratoren zijn

\paragraph{inleiding}
Vanuit deze requiremenst kunnen verdere specificaties opgesteld worden.

Even ter duidelijkheid: een requirement beschrijft wat een programma moet doen, en een specificatie beschrijft hoe men van plan is om deze requirements te realiseren.//
Voorbeeld:// Requirement is dat de sluis meerdere boten moet kunnen verwerken; de specificatie zou hier zijn fdat de sluis minstens twee keer zo groot moet zijn dan de grootste boot die door de sluis kan.



\paragraph{ctl}

CTL formulas are based on the following operators:
A ($\on$ every path")
E ($\there$ exists a path")
X ($\next$ time")
G ($\globally$ or $\always$)
F ($\eventually$ or $\nally)
U ($\until$)
R ($\release$)



Deze zijn als volgt:
\newline
\newline
A[] not maincontroller.rd1 imply
\newline
\newline
A[] maincontroller.rd1 imply
\newline
\newline
A[] not deadlock imply
\newline
\newline
E<> maincontroller.rd1 imply
\newline
\newline
E<> maincontroller.s7
\newline
\newline
E<> maincontroller.s7d
\newline
\newline

\paragraph{Formele specificaties}


\paragraph{Timed automata}


Before we consider a reachability problem, we show how real-time systems can be modoeled as parralel compositions of timed automata [3,5]. We assume an interleavingor asynchroneous semantics for this operation. Let A1 = ($\sum$, S1, $\S^{1}_0$, $X_1$, $I_1$, $T_1$) and $A_2$ = ($\sum_2$, $S1_2$, $\S^{1}_0$, $X_2$, $I_2$, $T_2$) be two timed automata. Assume that the two automata have disjoint sets of clocks, that is $X_1$ $\cap$ $X_2$ = $\emptyset$. Then, the parralel composition of $A_1$, and $A_2$ is the timed automation:

$A_1$ || A2 = ($\Sigma$ $\cup$ $\Sigma_2$, $S_1$ x S2, $\S^{1}_0$ x  $\S^{2}_0$ , $X_1$ $\cup$ $X_2$, I, T),
where I($s_1$,$s_2$)=$I_1$($s_1$) $\wedge$ $I_2$($s_2$) and the edge relation T is given by the following rules:\\ \newline

1 For a $\in$ $\Sigma_1$ $\cap$ $\Sigma_2$, if $\langle$ s1,a, $\varphi$, $\lambda_1$, $s_1$' $\rangle$ $\in$ $T_1$ and $\langle$ s2,a, $\varphi$, $\lambda_2$, $s_2$' $\rangle$ $\in$ $T_2$ \\ then T will contain the transition $\langle$ (s1,s2), a $\varphi$ , $\lambda_1$ $\cup$ $\lambda_2$, ($s_1$',$s_2$') $\rangle$ \\ 
2. For a $\in$ $\Sigma_1$ - $\Sigma_2$, if $\langle$ s, a, $\varphi$, $\lambda$, s' $\in T_1$ and t $\in$ $S_2$ then T will contain the transition $\langle$ (s,t),a, $\varphi$, $\lambda$, (s', t) $\rangle$ \\ \newline
3. For a $\in$ $\Sigma_2$ - $\Sigma_1$, if $\langle$ s, a, $\varphi$, $\lambda$, s' $\in$ $T_2$ and t $\in$ $S_1$ then T will contain the transition $\langle$ (t,s),a, $\varphi$, $\lambda$, (t,s') $\rangle$

Thus the locations of the parralel composition are pairs of locations from the component automata, and the invariant of such a location is the conjunction of the invariants of the component locations. There will be a transition in the parralel compoition for ach pair of transitions from the individual timed automata with the same action The source location of the transition will be the composite location obtained from the source locations of the individual transitions. Te target location will be the compositelocation obtained from the target locations of the individual transitions. The guard will be the conjunction of the guards for the individual transitions, and the set of clocks that are reset will be the union of sets that are reset by the individual transitions. If the action of  a transition is only an action of one of the two processes, then there will be a transition in the parralel composition for each location of the othertimed automation. The source and target locations of the original transition and the location fromthe other automation. All of the other components of the transition will remain the same.



Timed automata
A timed automation[8,99] is a finite augmented with a finite set of  real-valued clocks. We assume that transitions are instantaneous. However, time can elapse when the automation is in a state or location. When a transition occurs, some of the clocks ma be reset to zero. At any instant, the reading clock is equal to the time that has elapsed since the lat time the clock was reset. We assume that time passes at the same rate for all clocks. In order to prevent pathological behaviours, we only consider automata that are non-zeno, that is, only a finite number of transitions can happen within a finite amout of time.

A clock constraint, called a guard, is associated with each transition. The transition can be taken only if the current values of the clocks satisfy the clock constraint. A clock cnstraint is also associated with each location of the automation. This constraint i called the invariant of the location. Time can elapse in the location only as long as the invariant of the location is true. An example of a timed automation is shown in Figure 17.1 The automation consists of two locations s0 and s1, two clocks x and y, and "a" transition from s0 to s1, and a "b" transition from s1 to s0. The automation starts in location s0. It can remain in that location as long as the clock y is less than or equal to 5. As soon as the value of y is greater than or equal to 3, the  automation can make an "a" transition to location s1 and reset the clock y to 0. the automation can remain in location s1 as long as y is less than or equal to 10 and x is less than or equal to 8. When y is at least 4 and x is at least 6, it can make a "b" transition back to location s0 and reset x.

The remainder of this section contains a formal semantics for timed automata in terms of infinite state transition graphs[3,8]. We begin with a precise definition of clock constraints. Let X be a set of clock variables, ranging over the nonneative real numbers $\Re^{+}$. Define the set of clock constraints C(X) as follows:
All inequalities of the form x $\prec$ c or c $\prec$ x are in C(X) where $\prec$ is either < or  $\lveq$ $\sm$ and c is a nonnegative rational number.
If $\varphi_1$ ad $\varphi_{1}$ are in C(X), then $\varphi_1$ $\wedge$ $\varphi$ is in C(X).

Note that if X contains k clocks; then each clock constraints is a convex subset of k-dimensional Eucledian space.Thus, if two points satisfy a clock constraint, then all of the points	on the line sement connecting these points satisfy the clock constraint.
A timed automation is a 6-tuple A = ($\Sigma$, S, $S_0$, X, I, T) such that: \\
$\Sigma$ is a finite alphabet  \\
S is a finite set of locations \\
S0 $\subseteq$ S is a set of starting locations \\
X is a set of clocks \\
I : S $\rightarrow$ C(X) is a mapping from locations to clock constraints called the location invariant. \\
T $\subseteq$ S x $\Sigma$ x C(X) x $2^{x}$ x S is a set of transitions. The 5-tuple $\langle$ s,a,$\varphi$, $\lambda$, s' $\rangle$ corresponds to a transition from location s to location s' labeled with a, a constraint $\varphi$ that specifies when the transition is enabled, and a set of clocks $\lambda$ $\subseteq$ X that  are reset when the transition is executed. \\


We will require that time be allowed to progress to infinity, that is, at each location the upper bound imposed on the clocks be either infinity, or smaller than the maximum bound imposed by the invariant and by the transitions outgoing from the location. In other words, it is possible either to stay at a location forever, or the invariant will force the automation to leave the location, and at that point at least one transition will be enabled. For timed automata, these constraints can be imposed syntactically.

A model for a timed automation A is an infinite state transition graph $\tau$(A) = ($\Sigma$, Q, $Q^{0}$, R). Each state in Q is a pair (s, v) where s $\in$ S is a location and v : X $\rightarrow$  $R^{+}$ is a clock assignement, mapping each clock to a nonnegative real value. The set of initial states $Q_0$ is given by {(s,v)| s $\in$ $S_0$ $\wedge$ $\forall$ x  $\in$  X[v(x) =0]}. \\
\newline
In order to define the state transtion relation for $\tau$(A), we must first introduce some notation. For $\lambda$ $\subseteq$ X, define v[$\lambda$ := 0] to be the clock assignment that is the same as v for clocks in X - $\lambda$ and maps the clocks in $\lambda$ to 0. For d $\in$ $\Re$, define v +d as the clock assignment that maps each clock x $\in$ X to v(x) +d. The clock assignment v -d is defined in the same  manner.
From the brief discussion in the introduction, we know that a timed automation has two basic types of transitions: \\
\newline
Delay transitions correspond to the elapsing of time while staying at some location. \\ \newline 
We write (s, v) $\xrightarrow[]{d}$ (s, v+d), where d $\in$  $R^{+}$, provided that for every 0 $\leq$ e $\leq$ d, the invariant	l(s) holds for v +e. \\
\newline
Action transitions correspond to the execution of a transition	 from T. We write (s,v) $\xrightarrow[]{a}$ (s', v'), where a $\in$ $\Sigma$, provided that there is a transition $\langle$ s,a, $\varphi$, $\lambda$, s' $\rangle$ such that v satisfies $\varphi$ and v=[$\lambda$:=0]. \\ 
\newline

The transition relation R of $\tau$(A) is obtained by combining the delay and action transitions. We will write (s,v) R(s', v') or (s, v) $\xRightarrow[]{f(x)}$   (s', v') if there exists s" and v" such that (s,v) $\xrightarrow[]{d}$ (s", v")$\xrightarrow[]{a}$ (s', v') for some d $\in$ $\Re$.
In this chapter we will describe an algorithm for solving the reachability problem for $\tau$(A): Given a set of initial states $Q_n$, we show how to compute the set of all states q $\in$ Q that are reachable from $Q_0$ by transitions in R. This problem is nontrivial because $\tau$ (A) has an infinite number of states. In order to accomplish this goal, it is necessary to use a finite representation for the infinite state space of $\tau$(A). Developing such representations is the main topic of te following sections.

\paragraph{clock regions}
In the definition	 of timed automata, we allowed the clock constraints that serve as the invariants of locations and the guards of transitions to contain arbitrary rational constants.
We can multiply the constants in each clock constraint by the least common multiple m of the denominators of all the constants to integers. The value of a clock can still be an arbitrary nonnegative real number. Note that applying this transformation can change the clock assignments in the set of reachable states of T(A). Fortunately, this does not cause a mjor problem. Ther reachale states of the original auomation can be obtained from the locations of te transformed automation by applying the inverse transformation, that is, dividing each clock value by m.


Th largest constant in the tranformed in the transformed automation is the product of m and the largest constant in the original automation. Thus, the transformation at worst results in quadratic blowup in the length of the encodings of th lock constraints[3]. This increase in complexity is acceptable, since the transformation simplifies certain operations on clock constraints that will be needed later in the chapter. We will apply this tranformation uniformly to all of th clock constraints that appear in the timed automata the we study. Consequently, in the future we can assume without loss of generality that all constants in clock constraints that we encounter are integers.

In order to obtain a finite representation for the infinite state space of a timed automation, we define clock regions[7,8], which represents sets of clock assignments. If two states, which correspond to the  same location of the timed automation A, agree on the integral parts of all clock constraint in the invariant of a location or in the guard of a transition is satisfied or not. The ordering of the fractional parts of the clock values determines which clock will change its integral part first. This is because clock constraints cn involve only integers, and all clocks increase at the same rate.

For example, let A be a timed automation with two clocks x1 and x2. Let s be a location in A with an outgoing transition e to some other location. Consider two states (s,v) and (s,v') in T(A) that correspond to location s. Suppose that v(x1) = 5.3, v(x2)=7.5, v'(x1)=5.5 and v'(x2) = 7.9. Assume that the guard $\varphi$ associated with e is x1 $\geq$ 8 $\wedge$ x2 $\geq$ 10. It is easy to see that if (s,v) eventually satisfies the guard, then so will (s, v'). \\
\newline

The value of a clock cn get arbitrarily large; however, if the clock is never compared to a constant greater than c, then the value of the clock will have no effect on the computation of A once it exceeds c. Suppose, for instance, that the block x is never compared to a constant greater than 100 in the invariant associated wit a location or in the guard of a transition.

Then, based on the behaviour of A, it is impossible to distinguish between x having the value 101 and x having the value 1001.
Alur,Courcoubetis, and Dill[7,8] show how to formalize this reasoning. For each clock x $\in$ X, let cx, be the largest constant that x is compared with in the invariant of any location or in the guard of any transition. For t $\in$  $\Re^{+}$, let ft(t) be the fractional part of t, and let [t] be the integral part of t. Thus, t = [t] + fr(t). We define an equivalence relation $\cong$ on the set of possible clock assignments as follows: Let v and v' be two clock assignments.
Then v  $\cong$ v' if and only if three conditions are  satisfied: \\
\newline

For all x $\in$ X either v(x) $\geq$ cx, and v'(x) $\geq$ gx or [v(x)] =[v'(x)].
For all x,y $\in$ X such that v(x) $\leq$ cx and v(y) $\leq$ cy, fr(v(x)) $\leq$ fr(v(y)) if and only if fr(v'(x)) 
$\leq$ fr(v'(y))
For all x $\in$ X either v(x) $\leq$ cx,
fr(v(x)) =0 if and only if fr(v'(x)) =0.
it is easy to see that $\cong$ does indeed define anequivalence relation. The equivalence classes of $\cong$ are called regions[7,8].  We will write [v] to denote the region which containsthe clock assignment v.Each region can be represented by specifying \\
\newline

1. for every clock x $\in$ X, once clock constraint from the set {x=c | c=0...., cx} $\cup${c -1 < x < c | c=1,.....cx} $\cup$ {x > cx}
2. for every pairof clocks x, y $\in$ X such that c-1 < x< cand d-1<y< d are clock constraints in the first condition, whether fr(x) is less than, equal to, or greater than fr(y).

Figure 17.7 which is taken from [8], shows the clockregions for a timed automationwith two clocks x and y where cx = 2 and cy =1. In this example, there are a total of 28 regions: 6 corner points, 14 open line segments and 8 open regions.

We will use this observation to show that $\cong$ has finite index and, consequently, that the  number of regions is finite. Our proofof this fact is based on the proof given in [8].





Lemma 43
The number of equivalence classes that $\cong$ induces on C(X) is bounded by
|X|! $\cdot$ $2^{|X|}$ $\cdot$ $\prod$ (2xc +2)
proof
An equivalence class [v] of $\cong$ can be described by a tripple	 of arrays i the following manner. For each block x $\in$ X, the array $\alpha$ tells which of the intervals {[],[]}
contains the value v(x). Thus, the array $\alpha$ represents the cloc assignment v if and only if for each clock x $\in$ X, v(x)$\in$ $\alpha$(x). The number of ways to choose $\alpha$ is $\prod$. \\ 
\newline


Let $X_a$ be th set of clocks with nonzero fractional part. The array $\beta$: $x_a$ $\rightarrow$ {1,....|$A_a$|} is a permutation of $X_a$, which gives the ordering of the fractional parts of the clocks in Xa with respect to $\leq$. Thusm the array $\bta$ represents a clock assignment c if and if for each pair x,y $\in$ $X_a$, if $\beta$(x) < $\beta$(y) then fr(v(x)) $\leq$ fr(v(y)). For a given $\alpha$, the number of ways to choose $\beta$ is bounded by |$X_\alpha$|!  which is bounded by |X|!. \\
\newline

The third component  $\gamma$ is a boolean array indeed by $X_a$ that is used to specify which clocks in $X_a$ have the same fractional part. For each clock c, $\gamma$(x) tells whether or not the fractional part of v(x) equals  the fractional part of its predecessor in the array $\beta$. Thus the array $\gamma$ represents a clock assignment v if and only if for each x \n X, $\gamma$(x) equals 0 exactly when there is a clock $\gamma$ $\in$ $X_alpha$ such that $\beta$(y) = $\beta$(x) +1 and fr(v(x)) equals fr(v(y)). The number of ways of choosing $\gamma$ is bounded by the number of boolean arrays over $X_\alpha$, which is bounded by $2^{|X|}$.
Hence, $\alpha$ encoded the integral parts of he clock assignments, and $\beta$ with $\gamma$ encodes the ordering of their fractioal aprts. It is easy to see that the sets represented bytriples are equivalence of $\cong$ and that every equivalence class is represented by some triple. The bound given in the statement of the lemma is the product of the bounds associated with $\alpha$, $\beta$, and $\gamma$. This completes the proof of the lemma. \\
\newline


The following properties of the equivalence relation $\cong$ are used in later  in this chapter.
Lemma 44
Let v1 and v2 be twoclock assignments1, let $\varphi$ be a clock constraint, and let $\lambda$ $\subseteq$ X be a set of clocks.
1. if v1 $\cong$ v2 and t is a nonnegative integer, then v1 + t $\cong$ v2 +t.
2. if v1 $\cong$ v2, then $\forall$ t1 $\in$ $R^{|+|}$ $\existst_2$ $\in$ $R^{|+|}$[v1 +t1 $\cong$ v2 + t2]
3. if v1 $\cong$ v2, then v1 satisfies $\varphi$ if and only if v2 satisfies $\varphi$
4. If v1 $\cong$ v2, then v1[$\lambda$:=0] $\cong$ v2 [$\lambda$:=0]
\\
\newline

Note that the first property may not hold if t is not an integer. For example, (2.8) $\cong$ (.1, .2),
but (.2, .8) +.3 is not equivalent to (.1,.2) + .3. All of the properties except the second are straightforward to prove and will be left to the reader. A proof if the scond property is sketched below. The proof is not diificultm but it is somewhat tedious. It can be safely skipped when this chapter is read for the first time.

\\
\newline

Proof
Assume that v1 $\cong$ v2. We can assume that t1 > 0 because, otherwise, we can simply choose t2=0. Let X{x1,x2,.....,xn}. We can threat v1 as a vector v1 = $\langle$ a1, ......, an $\rangle$, where $a_i$ is the alue of clock $x_i$ in $v_1$. Similarly, we let v2 = $\langle$ b1, ......, $b_n$ $\rangle$. Since corresponding clocks have the same integer part, we can assume without loss of generality that 0 $\leq$ $a_i$ < 1 and 0 $\leq$ $b_i$ < 1. Also , assume that the clock values are sorrted into increasing order so that $a_1$ $\leq$ $a_2$ $\leq$ ... $\leq$ $a_n$ and b1 $\leq$ $b_2$ $\leq$ .... $\leq$ $b_n$.


case 1
Assume that the largest element in v1 + t1 is less than or equal to 1. This case is trivial. We can easilty choose t2 so that v + t1 $\cong$ v2 + t2

case 2
Assume that  0 $\leq$ t1 < 1. Let the first element of v1 +t1. That is greater than or equal to 1 be $a_k$+t1. Chhoose $\in$ so that $\in$ =0 if $a_k+t_1$ =1 and so that 0< \ni < $b_k$-$b_k-1$ if $a_k$+$t_1$ > 1. Note that $b_k_$-1 < $b_k$ = $b_k$-1, then $a_k$=$a_k$-1 and $a_$+$t_1$ is not the first elment of v1+t1 that is greater than o equal to 1. We will show that v1+t1 $\cong$ v2+(1+ $\in$ - $b_k$). In order to show this we will split the vectors into two parts. Let

L1= $\langle$ $a_1$ + t1, ...., $a_k$-1 + t1    $\rangle$, and
L2= $\langle$  $b_1$ + (1 + $\in$ - $b_k$), ..., $b_k$-1 + (1+ $\in$ - $b_k$)  $\rangle$
In each case it is straightforward to show that

1. all of the elements are positive
2. the elements are sorted in increaing order, and
3. all of the elements are less than 1
Because of these conditios it is easy to see that $L_1$ $\cong$ $L_2$. Similarly, let 

$R_1$= $\langle$ $a_k$ + t1, ...., $a_n$ + t1    $\rangle$, and
$R_2$= $\langle$  $b_k$ + (1 + $\in$ - $b_k$), ..., $b_k$-1 + (1+ $\in$ - $b_k$)  $\rangle$

All of the elements in $R_1$ and $R_2$ are greater than or equal to 1. The fractional parts are given by $R_1$ - 1 and $R_2$ -1, respectively. For these vectors it is straightforward to show that

1. all of the elements are nonnegative
2. the elements are sorted in increasing order, and
3. all of the elements are less than 1

Moreover, an element in one vector is 0 if and only if the corresponding element in the order vector is 0. Thus $R_1$ -1 $\cong$ $R_2$ -1. It follows immediately that $R_1$ $\cong$ $R_2$.
It is not difficult to see that the fractioal parts of $R_2$ precede the fractional parts of $L_2$.
Let i  $\geq$ k and j < k. Then
$b_i$ +  (1 + $\in$ - $b_k$) -1 $\leq$ $b_j$ + (1+ $\in$ - $b_k$).
is equivlent to $b_i$ - $b_j$ $\leq$ 1, which is obviously true. The same relationship holds for the fractioal parts of $R_1$ and $L_1$, that is.
$a_i$ + $t_1$ -1 $\leq$ $a_j$ + $t_1$.

hence , we obtain $R_1$ $\cdot$ $L_1$ $\cong$ $R_2$ $\cdot$ $L_2$, where "$\cdot$" is concatenation of vectors. This shows that for all $t_1$ with 0 $\leq$ t1 < 1, there exits a t2 such that $v_1$ + $t_1$ $\cong$ v2 + $t_2$ and completes the proof of 



case 3
Finally, supppose that $t_1$ $\geq$ 1. Let t1'= $t_1$ - [$t_1$], so that 0 $\leq$ $t_1$ < 1. Find $t_2$ such that $v_1$ + $t_1$ $\cong$ v2 + t2. Then:
v1 + t1 + [t1]  $\cong$ v2 + t2 + [t1].

If we choose t2 = [t1], then we have v1 + t1 $\cong$ v2 + t2 as required. This completes the proof of the second property.

The equivalence relation $\cong$ over clock assignments an be extended to an equvalence relation over the state space of T(A) by requiring that equivalent states have identical locations and equivalent clock assignments: (s,v) $\cong$ (s', v') if and only if s = s' and v $\cong$ v'. The key property of he equivalence reltion $\cong$ is given by the following lemma [5]:


Lemma 45
If v1 $\cong$ v2 and (s, $v_1$)  $\xrightarrow[]{a}$ (s', v'). The transition $\langle$ s, a, $\varphi$, $\lambda$, s' $\rangle$  that takes state (s, v1) to state (s', v1') corresponds to two transiions of the timed automation.

Proof
Assume that v1 $\cong$ v2 and (s, v1)  $\xrightarrow[]{a}$ (s', v'1). The transition $\langle$ s, a, $\varphi$, $\lambda$, s' $\rangle$ that takes state (s, v1) to state (s', v'1) corresponds to two transitions of the timed automation:

a delay transition (s, v1)  $\xrightarrow[]{d1}$ (s, $v_1$ + $d_1$) for some $d_1$ $\geq$ 0, and
an action transition   (s, v1 +d1)  $\xrightarrow[]{a}$ (s', $v_1$') such that $v_1$ + $d_1$ satisfies $\varphi$ and $v'_1$ = ($v_1$ + $d_1$)[$\lambda$ :=0].



Since v1 $\cong$ v2 and v1 satisfies I(s), v2 also satisfies I(s). Furthermore, there exists d2 $\geq$ 0 such that v1 + d1 $\cong$ v2 + d2. Since v1 + d1 satisfies I(s), v2 +d2 also satisfies I(s). Because the clock constraint I(s) is convex and is satisfied by both v2 and v2 + d2, I(s) must be satisfied by v2 + e for all e such that 0 $\leq$ e $\leq$ d2. Consequently, the delay transition (s, v2)  $\xrightarrow[]{d2}$ (s, v2 +d2) is legal.

Since v1 + d1 $\cong$ v2 +d2, both v1+ d1 and v2+ d2 must satisfy the clock constraint for the guard $\varphi$. Thus, the transition $\langle$ s, a, $\varphi$, $\lambda$ , s' $\rangle$ myst also be enabled in the state  *s, $v_2$ + $d_2$
. Let $v'_2$ = ($v_2$ +$d_2$)[$\lambda$ :=0]. Then v'$_2$ is equivalent to $v'_1$. Hence, there is an action transition (s, $v_2$ + $d_2$)  $\xrightarrow[]{a}$  (s', $v'_2$). Combining the delay transition with the action transition, we get (s, $v_2$)  $\xrightarrow[]{a}$ (s', $v'_2$) as required.

As a result of the lemme, we can conostruct a finite state transition raph that is  bisimilaion equivalnt to the infinite state transition graph T(A). The finite state transition graph is called the region graph of A[7,8] and is denoted by R(A). A region is a pair (s, [v]). Since $\cong$  has a finite index, there are only a finite nuber of regions. The states of the region graph are  the regions of A. The construction of R(A) will have the property that whenever (s,v) is a state of T(A), the region (s, [v]) where $s_0$ is an initial state of A and $v_0$ is a clock assignment that assigns 0 to every clock. The transition relation of R(A) is defined so that bisimulation equivalence is guaranteed. There will be a transition labeled with a from the region (s,[v]) to the region (s', [v']) if and only there are assignments $\omega$ $\in$ [v] and $\omega$' $\in$ [v'] such that (s, $\omega$) can make a transition to (s', $\omega$')

We summarize the construction of the region graph R(A) below. Let A = ($\sigma$, S, $S_0$, X, I, T) be a timed automation. Then,
The states of R(A) have the form (s, [v]) where s $\in$ S and [v] i a clock region
The initial states have the form ($s_0$, [v]) where $s_0$ $\in$ $s_0$ and v(x)=0 for all x $\in$ X.
R(A) has a transition ((s,[v]),a, (s',[v'])) if and only if (s, $\omega$)  $\xrightarrow[]{a}$  (s', $\omega$') for some $\omega$ $\in$ [v] and some $\omega$' $\in$ [v'].
We can use Lemma 45 to prove bisimulation equivalene.

Theorem 31
We will show that T(A) and R(A) are bisimilar. Define the bisimulation relation B by (s,v)B(s,[v]). It is easy to see that the initial state ($s_0$, $v_0$) corresponds o the  state ($s_0$, [$v_0$]). Next, we show that for each transtition of T(A), there is a corrresponding transition  of R(A), and vice versa. Suppose first that (s,v)B(s,[v]). Suppose on the other hand that (s,v)B(s,[v]) and that  (s,v)$\xrightarrow[]{a}$(s',[v']). Then there exit $\omega$ $\cong$ v and $\omega$' $\cong$ v' such that (s', v") and (s,v) $\xrightarrow[]{a}$(s', v"). Hence v" $\cong$ $\omega$ $\cong$ v', so [v"] = [v']. By the definition of B, (s', v")
B (s', [v"]), it follows that (s', v")B(s', [v']).


\paragraph{Safety}
Safety Properties are used to verify that something
bad will never happen. Dit kan worden gespecificeerd met de volgende vergelijking

\aqcap\\

\square ( a_0 \implies (( \lnot a_2 \wedge \lnot a_3 ) \mathcal{U} a_1 ) \vee ( \lnot a_2 \wedge \lnot a_3 )) \\

AG(p)
M, s \models AG(p) $\Leftrightarrow$     \forall \pi \in  \sqcap (M,s) \cdot \forall i \cdot M,\pi[i] \models p\\

EG(p)
M, s \models EG(p) $\Leftrightarrow$     \exists \pi \in  \sqcap (M,s) \cdot \forall i \cdot M,\pi[i] \models p\\

AF(p)

EF(p)

AX(p)

EX(p)

A(p \cup q)
M, s \models  A(p \cup q)   $\Leftrightarrow$     \forall \pi \in  \sqcap (M,s) \cdot \exists k \cdot M,\pi[k] \models q \wedge ( \forall i \leq k \cdot M,\pi [i] \models p)\\
E(p \cup q)

A(p \Re q)

E(p \Re q)

\forall x \, (P(x) \to Q(x)) & premise \\
\forall x \, P(x) & premise \\\hspace*{-30pt} \\


P(x_0) & $\forall x \, \mathrm{e}$ 2 \\
Q(x_0) & $\to \mathrm{e}$ 3, 4 \\

\forall x \, Q(x) & $\forall x \, \mathrm{i}$ 3--5 \\







\{a,b\} or \set†{a,b} \\
\langle a,b \rangle or \gens†{a,b} \\


f \colon A \to B \\

f \circ g \\
x \mapsto f(x) \\

\begin{align*}
	f \colon \mathbb{R} &\to \mathbb{R} \\
	x &\mapsto x^2
\end{align*}


\newline \\
M, s $\models$ p $\Leftrightarrow$ p $\in$ L(s) \\
M, s $\models$ $\not$ f1 $\Leftrightarrow$ M, s $\nvdash$ f1 \\
M, s $\models$ f1 $\vee$ f2 $\Leftrightarrow$ M,s $\models$ f1 or M,s $\nvdash$ f2 \\
M, s $\models$ f1 $\wedge$ f2 $\Leftrightarrow$  M,s $\models$ f1 and M,s $\nvdash$ f2 \\
M, s $\models$ $\mathrm{E}$ $g_{1}$ $\Leftrightarrow$ there is a path $\pi$  from ~  s ~   such ~  that  ~ M, $\pi$ $\models$ g1 \\
M, s $\models$ p $\Leftrightarrow$ for every path $\pi$  ~ starting from  ~  s, M, $\pi$ $\models$ g1 \\
M, s $\models$ p $\Leftrightarrow$ s is the first state of $\piand$ M, s $\models$ f1 \\
M, s $\models$ $\not$ $g_{1}$ $\Leftrightarrow$ M, $\pi$  $\nvdash$ g1\\
M, s $\models$ p $\Leftrightarrow$  M, $\pi$  $\models$ g1  or  M, $\pi$  M, $\pi$  $\models$ g2\\
M, s $\models$ p $\Leftrightarrow$ M, $\pi$  $\models$ g1  and  M, $\pi$  M, $\pi$  $\models$ g2 \\
M, s $\models$ p $\Leftrightarrow$ M, $\pi^{1}$ $\models$ g1 \\
M, s $\models$ p $\Leftrightarrow$ there exists a k $\ge$ 0, such that  ~ M, $\pi^{k}$  $\models$ g1\\
M, s $\models$ p $\Leftrightarrow$ for all i $\ge$ 0,M,$\pi^{i}$ $\models$ g1 \\
M, s $\models$ g1 $\bugcup$ g2 $\Leftrightarrow$ ~  there  ~ exists  ~ ak  ~ $\ge$  ~ 0 ~  such ~  that  ~ M,  ~ $\pi^{k}$ $\models$ g2\\
and  ~ for  ~ all ~  0  ~ $\le$ j < k, M,$\pi^{j}$ $\models$ g1
M, s $\models$ p $\Leftrightarrow$ for all j $\ge$ 0, if for ~  every  ~ i < j,M,$\pi^{i}$ $\nvdash$ g1 then M,$\pi^{j}$ $\models$ g2\\


\paragraph{Reachability}
Reachability properties are used to check whether
a given state formula can be satisfied by some
reachable state.

\paragraph{Liveliness}
Liveness properties are used to verify that
something eventually will hold
\paragraph{Security}

\paragraph{Performance}

%%%%%%%%%%%%%%%%%%%%%%%%%%%%%%%%%%%%%%%%%%%%%%%%%%%%%%%%%%%%%%%%%
We think of thevariables innV as the present sate variables and the variables in V'as next state variables. Each variable v i V has a corresponding next state variable in V', which we denote y v'. A valuation for the variables in V and V' can be vieuwed as designating an ordered pair of states or a transition, and we can represent setsof these valuations using formulas as above. We refer to a set of pairs of states as a transition relation. If R is a transition relation, then we write R(V,V') to denote a formula that represents it.
In order to write specifications that describe properties of concurrent systems we need to define a set of atomic propositions AP. Atomic propositionswill typically have the form v=d where v $\in$ V and d $\in$ D. A proposition v =d will be truein a state s if s(v)=d. Whenv is a variable over the  boolean domain{True,False}, it is not necessarly to include both v = True and  v = False in AP.We will write v to indicate that s(v)=True and $\neq$ v to indicate that s(v)=False.
We now show how to derive

blz 16

We now show how to derive Kripke M=(S,$S_0$,R,L) from the first order formulas $S_0$ and R that represent the concurrent system.
The set of states is hthe set of all variations	for V
the set of initial states $S_0$ is the set of all valuations $s_0$ for V that satisfy the formula $S_0$
let s and s' be the two states, then R(s,s') holds if R evaluates to True when each v $\in$ V is assigned the value s(v) and each v' $\in$  V' is assigned the value s'(v).
The labeling function L: S $\to$  $\2^{AP}$ is defined so that L(s) is the subsetof all atomic propositions true in s. If v is a variable over the boolean domain, then v $\in$ L(s) indicates that s(v)=True, and v $\notin$  L(s) indicates that s(v)=False.
L: S $\t$o  $\2^{AP}$  is a function that labels each state with the set of atomic propositions true in that state\\

Because we require that the transition relation of a kripke structuer us always total, we must extend the relation R if some state s has no successor. In this case, we modify R so that R(s,s) holds.
To illustrate the notions defined in this section we consider a simple system with variables x and y that range over D={0,1}. Thus, a valuation for the variables x and y is justa pair ($d_1$, $d_2$) $\in$ D x Dwhre $d_1$ is the value for x and $d_2$ is the value for y.

blz 33
Fairness
A fairness constraint an be an arbitraty set of states, usually described by the formula of the logic. if fairness constraints are interpreted as sets of states, then a fair path must contain an element of each fairness constraint infinetely often. If fairness constrants are interpreted	 as CTL formula, then a path is fair if each constraint is true infnetely often along the path. The path quantifiers in the logic are then restricted fair paths.
Formally, a fairkripke structure is a 4-tuple M = (S,R,L,F), where S, L and R are  defined as before and F $\subseteq$  $\2^{S}$  is a set of fairness constraints ( often called Buchi acceptance conditions) Let $\pi$ = $s_0,s_1$ be a path in M. Define 
inf($\pi$) = {s| s=$s_i$ for infinitely many i}.

We say that $\pi$ is fair if and only if for every P $\in$ F, inf(\p) $\cap$ P $\neq$ $\emptyset$. The semantics of CTL* wth respect to a fair kripke structure is very similar to the semantics of CTL* with respect to ordinary kripke structure. We will write M,s $\models_F$ f to indicate that the state formula f is true in state s of the fair Kripke structure M. Similarly, we write M, $\pi$ $\models$ $_f$ g to indicate that the path formula g is true along path $\pi$  in M. Only clauses 1, 5 and 6 in the origial semanticss change.
1. M, s $\models$  $_f$ p  $\Leftrightarrow$ there exists a fair path from s and p $\in$ L(s)
5. M, s $\models$  $_f$ p  $\Leftrightarrow$ there exists a fair path $\pi$ starting from s such that $\pi$ $\models$ $_f$ g1
6. M, s $\models$  $_f$ p  $\Leftrightarrow$ for all fair paths $\pi$ starting from s, $\pi$ $\models$ $_f$ g1

To illustrate the use of fairness, conider again the communication protocol for reliable channels. There is one fairness constraint for each channelthat expresses the reliability of that channel. A possible choice for the fairness constraint associated with channel i is the set of states that satisfy the formula $\neq$ send $\vee$  $receive_i$. Thus, a computation path is fair if and only if for every channel, infinitly often either a message is received. Other notions of fairness are dealt with in[116].
blz 36
ctl model checking

The model checkingproblem is easy to describe. given a kripke structure M =(S,R,L) that represents a finite-state concurrent system and a termporal logic formula f expressing some desired specification, find the set of all states n S that satisfy f: {s $\in$ S | M, s $\models$ f} \\
Let M = (S,R, L) be a kripke structur. Assume that we want to determine which states in S satisfy the CTL formula f. The algorithm will operate by labeling each state s with the set label(s) of subformulas of f which are true in s. Initially, label(s) is just L(s). The algorithm then goes through a series of stages.During the ith stage, subformulas with i-1 nested CTL operators are processed. When a subformula is processed, it is added to the labelig of each state in which it is true. Onze the algorithm terminates, we will have that M, s $models$ f iff f $\in$ label(s)
blz 40
Fairness constraints
In this subsection we show how to extendthe CTL model checking algorithmto handle fairness constraints. Let M = (S,R,L,F) be a fair kripke structure. Let F = {P1, ..., $P_k$} be the setof fairness constraints. We will say that a strongly connnected component C of the graph of M is fair wth respect to F if and only if for each $P_i$ $\in$ F, there is a state $t_i$ $\in$ (C $\cap$  $P_i$). We first give an algorithmfor checking EG $f_1$ with respect to a fair structure. In order to establish the correctness of this algorith, we need a lemma that is analogous to Lemma 1.As before, let M' be obtained from M by deleting from S all of those states at which $f_1$ does not fairly hold. Thus, M'=(S',R',L',F') where $\S^{'}$ = {s $\in$ S | M,s $\models$ F f1}, R' = R|$_S$'xS', L' = L|$_s$;, and F' ={ $P_i$ $\cap$ S' | $P_i$ $\in$ F}.

Lemma 2 M,s $\models$ F EG f1 iff the followingtow conditions are satisfied:
1. s $\in$ S'
2. There exists a path S' that leads from s to somenode t in a nontrivial fair strongly connected component of thr graph (S',R')

In order to determine if M, s $\models$ f p for some p $\in$ AP, we check M,s $\models$ p $\wedge$ fair using the ordinary model-checking procedure.
blz 68
Fairness in model checking with fixpoint

blz 69

blz 70

blz 71
Counterexamples and whitnesses

blz 72


blz 73


blz 74



blz 121
automata theory
blz 141



blz 171
Equivalence and preorders between systems
blz 172

blz 173


blz 174


blz 175


blz 176


blz 177
simulation relations
blz 178

blz 179

blz 180


blz 232
INvariants
blz 233


blz 234


blz 265

blz 266

blz 267


blz 268 parralel compositioon
Before we consider a reachability problem, we show how real-time systems can be modoeled as parralel compositions of timed automata [3,5]. We assume an interleavingor asynchroneous semantics for this operation. Let A1 = ($\sum$, S1, $\S^{1}_0$, $X_1$, $I_1$, $T_1$) and $A_2$ = ($\sum_2$, $S1_2$, $\S^{1}_0$, $X_2$, $I_2$, $T_2$) be two timed automata. Assume that the two automata have disjoint sets of clocks, that is $X_1$ $\cap$ $X_2$ = $\emptyset$. Then, the parralel composition of $A_1$, and $A_2$ is the timed automation:

$A_1$ || A2 = ($\Sigma$ $\cup$ $\Sigma_2$, $S_1$ x S2, $\S^{1}_0$ x  $\S^{2}_0$ , $X_1$ $\cup$ $X_2$, I, T),
where I($s_1$,$s_2$)=$I_1$($s_1$) $\wedge$ $I_2$($s_2$) and the edge relation T is given by the following rules:

1 For a $\in$ $\Sigma_1$ $\cap$ $\Sigma_2$, if $\langle$ s1,a, $\varphi$, $\lambda_1$, $s_1$' $\rangle$ $\in$ $T_1$ and $\langle$ s2,a, $\varphi$, $\lambda_2$, $s_2$' $\rangle$  $T_2$ then T will contain the transition $\langle$ (s1,s2), a $\varphi$ , $\lambda_1$ $\cup$ $\lambda_2$, ($s_1$',$s_2$') $\rangle$
2. For a $\in$ $\Sigma_1$ - $\Sigma_2$, if $\langle$ s, a, $\varphi$, $\lambda$, s' $\in$ $T_1$ and t $\in$ $S_2$ then T will contain the transition $\langle$ (s,t),a, $\varphi$, $\lambda$, (s', t) $\rangle$
3. For a $\in$ $\Sigma_2$ - $\Sigma_1$, if $\langle$ s, a, $\varphi$, $\lambda$, s' $\in$ $T_2$ and t $\in$ $S_1$ then T will contain the transition $\langle$ (t,s),a, $\varphi$, $\lambda$, (t,s') $\rangle$

Thus the locations of the parralel composition are pairs of locations from the component automata, and the invariant of such a location is the conjunction of the invariants of the component locations. There will be a transition in the parralel compoition for ach pair of transitions from the individual timed automata with the same action The source location of the transition will be the composite location obtained from the source locations of the individual transitions. Te target location will be the compositelocation obtained from the target locations of the individual transitions. The guard will be the conjunction of the guards for the individual transitions, and the set of clocks that are reset will be the union of sets that are reset by the individual transitions. If the action of  a transition is only an action of one of the two processes, then there will be a transition in the parralel composition for each location of the othertimed automation. The source and target locations of the original transition and the location fromthe other automation. All of the other components of the transition will remain the same.


blz 269 modelling with timed automata

blz 274 clock regions

blz 280 clock zones

blz 281
\paragraph{Timed automata}
Timed automata
A timed automation[8,99] is a finite augmented with a finite set of  real-valued clocks. We assume that transitions are instantaneous. However, time can elapse when the automation is in a state or location. When a transition occurs, some of the clocks ma be reset to zero. At any instant, the reading clock is equal to the time that has elapsed since the lat time the clock was reset. We assume that time passes at the same rate for all clocks. In order to prevent pathological behaviours, we only consider automata that are non-zeno, that is, only a finite number of transitions can happen within a finite amout of time.

A clock constraint, called a guard, is associated with each transition. The transition can be taken only if the current values of the clocks satisfy the clock constraint. A clock cnstraint is also associated with each location of the automation. This constraint i called the invariant of the location. Time can elapse in the location only as long as the invariant of the location is true. An example of a timed automation is shown in Figure 17.1 The automation consists of two locations s0 and s1, two clocks x and y, and "a" transition from s0 to s1, and a "b" transition from s1 to s0. The automation starts in location s0. It can remain in that location as long as the clock y is less than or equal to 5. As soon as the value of y is greater than or equal to 3, the  automation can make an "a" transition to location s1 and reset the clock y to 0. the automation can remain in location s1 as long as y is less than or equal to 10 and x is less than or equal to 8. When y is at least 4 and x is at least 6, it can make a "b" transition back to location s0 and reset x.

The remainder of this section contains a formal semantics for timed automata in terms of infinite state transition graphs[3,8]. We begin with a precise definition of clock constraints. Let X be a set of clock variables, ranging over the nonneative real numbers $\Re^{+}$. Define the set of clock constraints C(X) as follows:
All inequalities of the form x $\prec$ c or c $\prec$ x are in C(X) where $\prec$ is either < or  $\leq$ $\sm$ and c is a nonnegative rational number.
If $\varphi_1$ ad $\varphi_{1}$ are in C(X), then $\varphi_1$ $\wedge$ $\varphi$ is in C(X).

Note that if X contains k clocks; then each clock constraints is a convex subset of k-dimensional Eucledian space.Thus, if two points satisfy a clock constraint, then all of the points	on the line sement connecting these points satisfy the clock constraint.
A timed automation is a 6-tuple A = ($\Sigma$, S, $S_0$, X, I, T) such that
$\Sigma$ is a finite alphabet
S is a finite set of locations
S0 $\subseteq$ S is a set of starting locations
X is a set of clocks
I : S $\rightarrow$ C(X) is a mapping from locations to clock constraints called the location invariant.
T $\subseteq$ S x $\Sigma$ x C(X) x $2^{x}$ x S is a set of transitions. The 5-tuple $\langle$ s,a,$\varphi$, $\lambda$, s' $\rangle$ corresponds to a transition from location s to location s' labeled with a, a constraint $\varphi$ that specifies when the transition is enabled, and a set of clocks $\lambda$ $\subseteq$ X that  are reset when the transition is executed.


We will require that time be allowed to progress to infinity, that is, at each location the upper bound imposed on the clocks be either infinity, or smaller than the maximum bound imposed by the invariant and by the transitions outgoing from the location. In other words, it is possible either to stay at a location forever, or the invariant will force the automation to leave the location, and at that point at least one transition will be enabled. For timed automata, these constraints can be imposed syntactically.

A model for a timed automation A is an infinite state transition graph $\tau$(A) = ($\Sigma$, Q, $Q^{0}$, R). Each state in Q is a pair (s, v) where s $\in$ S is a location and v : X $\rightarrow$  $R^{+}$ is a clock assignement, mapping each clock to a nonnegative real value. The set of initial states $Q_0$ is given by {(s,v)| s $\in$ $S_0$ $\wedge$ $\forall$ x  $\in$  X[v(x) =0]}.
In order to define the state transtion relation for $\tau$(A), we musr first introduce some notation. For $\lambda$ $\subseteq$ X, define v[$\lambda$ := 0] to be the clock assignment that is the same as v for clocks in X - $\lambda$ and maps the clocks in $\lambda$ to 0. For d $\in$ $\Re$, define v +d as the clock assignment that maps each clock x $\in$ X to v(x) +d. The clock assignment v -> d is defined in the same  manner. \\
From the brief discussion in the introduction, we know that a timed automation has two basic types of transitions:
Delay transitions correspond to the elapsing of time while staying at some location. We write (s, v) $\xrightarrow[]{d}$ (s, v+d), where d $\in$  $R^{+}$, provided that for every 0 $\leq$ e $\leq$ d, the invariant	l(s) holds for v +e. \\
Action transitions correspond to the execution of a transition	 from T. We write (s,v) $\xrightarrow[]{a}$ (s', v'), where a $\in$ $\Sigma$, provided that there is a transition $\langle$ s,a, $\varphi$, $\lambda$, s' $\rangle$ such that v satisfies $\varphi$ and v=[$\lambda$:=0].

The transition relation R of $\tau$(A) is obtained by combining the delay and action transitions. We will write (s,v) R(s', v') or (s, v) $\xRightarrow[]{f(x)}$   (s', v') if there exists s" and v" such that (s,v) $\xrightarrow[]{d}$ (s", v")$\xrightarrow[]{a}$ (s', v') for some d $\in$ $\Re$.
In this chapter we will describe an algorithm for solving the reachability problem for $\tau$(A): Given a set of initial states $Q_n$, we show how to copute the set of all states q $\in$ Q that are reachable from $Q_0$ by transitions in R. This problem is nontrivial because $\tau$ (A) has an infinite number of states. In order to accomplish this goal, it is necessary to use a finite representation for the infinite state space of $\tau$(A). Developing such representations is the main topic of te following sections.

blz 268 parralel composition

blz 274 clock regions
\paragraph{clock regions}
In the definition	 of timed automata, we allowed the clock constraints that serve as the invariants of locations and the guards of transitions to contain arbitrary rational constants.
We can multiply the constants in each clock constraint by the least common multiple m of the denominators of all the constants to integers. The value of a clock can still be an arbitrary nonnegative real number. Note that applying this transformation can change the clock assignments in the set of reachable states of T(A). Fortunately, this does not cause a mjor problem. Ther reachale states of the original auomation can be obtained from the locations of te transformed automation by applying the inverse transformation, that is, dividing each clock value by m.


Th largest constant in the tranformed in the transformed automation is the product of m and the largest constant in the original automation. Thus, the transformation at worst results in quadratic blowup in the length of the encodings of th lock constraints[3]. This increase in complexity is acceptable, since the transformation simplifies certain operations on clock constraints that will be needed later in the chapter. We will apply this tranformation uniformly to all of th clock constraints that appear in the timed automata the we study. Consequently, in the future we can assume without loss of generality that all constants in clock constraints that we encounter are integers.

In order to obtain a finite representation for the infinite state space of a timed automation, we define clock regions[7,8], which represents sets of clock assignments. If two states, which correspond to the  same location of the timed automation A, agree on the integral parts of all clock constraint in the invariant of a location or in the guard of a transition is satisfied or not. The ordering of the fractional parts of the clock values determines which clock will change its integral part first. This is because clock constraints cn involve only integers, and all clocks increase at the same rate.

For example, let A be a timed automation with two clocks x1 and x2. Let s be a location in A with an outgoing transition e to some other location. Consider two states (s,v) and (s,v') in T(A) that correspond to location s. Suppose that v(x1) = 5.3, v(x2)=7.5, v'(x1)=5.5 and v'(x2) = 7.9. Assume that the guard $\varphi$ associated with e is x1 $\geq$ 8 $\wedge$ x2 $\geq$ 10. It is easy to see that if (s,v) eventually satisfies the guard, then so will (s, v').

The value of a clock cn get arbitrarily large; however, if the clock is never compared to a constant greater than c, then the value of the clock will have no effect on the computation of A once it exceeds c. Suppose, for instance, that the block x is never compared to a constant greater than 100 in the invariant associated wit a location or in the guard of a transition.

Then, based on the behaviour of A, it is impossible to distinguish between x having the value 101 and x having the value 1001.
Alur,Courcoubetis, and Dill[7,8] show how to formalize this reasoning. For each clock x $\in$ X, let cx, be the largest constant that x is compared with in the invariant of any location or in the guard of any transition. For t $\in$  $\Re^{+}$, let ft(t) be the fractional part of t, and let [t] be the integral part of t. Thus, t = [t] + fr(t). We define an equivalence relation $\cong$ on the set of possible clock assignments as follows: Let v and v' be two clock assignments.
Then v  $\cong$ v' if and only if three conditions are  satisfied:

For all x $\in$ X either v(x) $\geq$ cx, and v'(x) $\geq$ gx or [v(x)] =[v'(x)].
For all x,y $\in$ X such that v(x) $\leq$ cx and v(y) $\leq$ cy, fr(v(x)) $\leq$ fr(v(y)) if and only if fr(v'(x)) 
$\leq$ fr(v'(y))
For all x $\in$ X either v(x) $\leq$ cx,
fr(v(x)) =0 if and only if fr(v'(x)) =0.
it is easy to see that $\cong$ does indeed define anequivalence relation. The equivalence classes of $\cong$ are called regions[7,8].  We will write [v] to denote the region which containsthe clock assignment v.Each region can be represented by specifying

1. for every clock x $\in$ X, once clock constraint from the set {x=c | c=0...., cx} $\cup${c -1 < x < c | c=1,.....cx} $\cup$ {x > cx}
2. for every pairof clocks x, y $\in$ X such that c-1 < x< cand d-1<y< d are clock constraints in the first condition, whether fr(x) is less than, equal to, or greater than fr(y).

Figure 17.7 which is taken from [8], shows the clockregions for a timed automationwith two clocks x and y where cx = 2 and cy =1. In this example, there are a total of 28 regions: 6 corner points, 14 open line segments and 8 open regions.

We will use this observation to show that $\cong$ has finite index and, consequently, that the  number of regions is finite. Our proofof this fact is based on the proof given in [8].





Lemma 43
The number of equivalence classes that $\cong$ induces on C(X) is bounded by
|X|! $\cdot$ $2^{|X|}$ $\cdot$ $\prod$ (2xc +2)
proof
An equivalence class [v] of $\cong$ can be described by a tripple	 of arrays i the following manner. For each block x $\in$ X, the array $\alpha$ tells which of the intervals {[],[]}
contains the value v(x). Thus, the array $\alpha$ represents the cloc assignment v if and only if for each clock x $\in$ X, v(x)$\in$ $\alpha$(x). The number of ways to choose $\alpha$ is $\prod$.


Let $X_a$ be th set of clocks with nonzero fractional part. The array $\beta$: $x_a$ $\rightarrow$ {1,....|$A_a$|} is a permutation of $X_a$, which gives the ordering of the fractional parts of the clocks in Xa with respect to $\leq$. Thusm the array $\beta$ represents a clock assignment c if and if for each pair x,y $\in$ $X_a$, if $\beta$(x) < $\beta$(y) then fr(v(x)) $\leq$ fr(v(y)). For a given $\alpha$, the number of ways to choose $\beta$ is bounded by |$X_\alpha$|!  which is bounded by |X|!. \\ \newline

The third component  $\gamma$ is a boolean array indeed by $X_a$ that is used to specify which clocks in $X_a$ have the same fractional part. For each clock c, $\gamma$(x) tells whether or not the fractional part of v(x) equals  the fractional part of its predecessor in the array $\beta$. Thus the array $\gamma$ represents a clock assignment v if and only if for each x \n X, $\gamma$(x) equals 0 exactly when there is a clock $\gamma$ $\in$ $X_alpha$ such that $\beta$(y) = $\beta$(x) +1 and fr(v(x)) equals fr(v(y)). The number of ways of choosing $\gamma$ is bounded by the number of boolean arrays over $X_\alpha$, which is bounded by $2^{|X|}$.
Hence, $\alpha$ encoded the integral parts of he clock assignments, and $\beta$ with $\gamma$ encodes the ordering of their fractioal aprts. It is easy to see that the sets represented bytriples are equivalence of $\cong$ and that every equivalence class is represented by some triple. The bound given in the statement of the lemma is the product of the bounds associated with $\alpha$, $\beta$, and $\gamma$. This completes the proof of the lemma. \\ \newline


The following properties of the equivalence relation $\cong$ are used in later  in this chapter.
Lemma 44
Let v1 and v2 be twoclock assignments1, let $\varphi$ be a clock constraint, and let $\lambda$ $\subseteq$ X be a set of clocks.
1. if v1 $\cong$ v2 and t is a nonnegative integer, then v1 + t $\cong$ v2 +t. \\ \newline
2. if v1 $\cong$ v2, then $\forall$ t1 $\in$ $R^{|+|}$ $\existst_2$ $\in$ $R^{|+|}$[v1 +t1 $\cong$ v2 + t2] \\ \newline
3. if v1 $\cong$ v2, then v1 satisfies $\varphi$ if and only if v2 satisfies $\varphi$ \\ \newline
4. If v1 $\cong$ v2, then v1[$\lambda$:=0] $\cong$ v2 [$\lambda$:=0] \\ \newline

Note that the first property may not hold if t is not an integer. For example, (2.8) $\cong$ (.1, .2),
but (.2, .8) +.3 is not equivalent to (.1,.2) + .3. All of the properties except the second are straightforward to prove and will be left to the reader. A proof if the scond property is sketched below. The proof is not diificultm but it is somewhat tedious. It can be safely skipped when this chapter is read for the first time.



Proof
Assume that v1 $\cong$ v2. We can assume that t1 > 0 because, otherwise, we can simply choose t2=0. Let X{x1,x2,.....,xn}. We can threat v1 as a vector v1 = $\langle$ a1, ......, an $\rangle$, where $a_i$ is the alue of clock $x_i$ in $v_1$. Similarly, we let v2 = $\langle$ b1, ......, $b_n$ $\rangle$. Since corresponding clocks have the same integer part, we can assume without loss of generality that 0 $\leq$ $a_i$ < 1 and 0 $\leq$ $b_i$ < 1. Also , assume that the clock values are sorrted into increasing order so that $a_1$ $\leq$ $a_2$ $\leq$ ... $\leq$ $a_n$ and b1 $\leq$ $b_2$ $\leq$ .... $\leq$ $b_n$.


case 1
Assume that the largest element in v1 + t1 is less than or equal to 1. This case is trivial. We can easilty choose t2 so that v + t1 $\cong$ v2 + t2

case 2
Assume that  0 $\leq$ t1 < 1. Let the first element of v1 +t1. That is greater than or equal to 1 be $a_k$+t1. Chhoose $\in$ so that $\in$ =0 if $a_k$+$t_1$ =1 and so that 0< \ni < $b_k$-$b_k$-1 if $a_k$+$t_1$ > 1. Note that $b_k_$-1 < $b_k$ = $b_k$-1, then $a_k$=$a_k$-1 and $a_$+$t_1$ is not the first elment of v1+t1 that is greater than o equal to 1. We will show that v1+t1 $\cong$ v2+(1+ $\in$ - $b_k$). In order to show this we will split the vectors into two parts. Let

L1= $\langle$ $a_1$ + t1, ...., $a_k$-1 + t1    $\rangle$, and
L2= $\langle$  $b_1$ + (1 + $\in$ - $b_k$), ..., $b_k$-1 + (1+ $\in$ - $b_k$)  $\rangle$
In each case it is straightforward to show that

1. all of the elements are positive
2. the elements are sorted in increaing order, and
3. all of the elements are less than 1
Because of these conditios it is easy to see that $L_1$ $\cong$ $L_2$. Similarly, let 

$R_1$= $\langle$ $a_k$ + t1, ...., $a_n$ + t1    $\rangle$, and
$R_2$= $\langle$  $b_k$ + (1 + $\in$ - $b_k$), ..., $b_k$-1 + (1+ $\in$ - $b_k$)  $\rangle$

All of the elements in $R_1$ and $R_2$ are greater than or equal to 1. The fractional parts are given by $R_1$ - 1 and $R_2$ -1, respectively. For these vectors it is straightforward to show that

1. all of the elements are nonnegative
2. the elements are sorted in increasing order, and
3. all of the elements are less than 1

Moreover, an element in one vector is 0 if and only if the corresponding element in the order vector is 0. Thus $R_1$ -1 $\cong$ $R_2$ -1. It follows immediately that $R_1$ $\cong$ $R_2$.
It is not difficult to see that the fractioal parts of $R_2$ precede the fractional parts of $L_2$.
Let i  $\geq$ k and j < k. Then
$b_i$ +  (1 + $\in$ - $b_k$) -1 $\leq$ $b_j$ + (1+ $\in$ - $b_k$).
is equivlent to $b_i$ - $b_j$ $\leq$ 1, which is obviously true. The same relationship holds for the fractioal parts of $R_1$ and $L_1$, that is.
$a_i$ + $t_1$ -1 $\leq$ $a_j$ + $t_1$.

hence , we obtain $R_1$ $\cdot$ $L_1$ $\cong$ $R_2$ $\cdot$ $L_2$, where "$\cdot$" is concatenation of vectors. This shows that for all $t_1$ with 0 $\leq$ t1 < 1, there exits a t2 such that $v_1$ + $vt_1$ $\cong$ v2 + $t_2$ and completes the proof of 



case 3
Finally, supppose that $t_1$ $\geq$ 1. Let t1'= $t_1$ - [$t_1$], so that 0 $\leq$ $t_1$ < 1. Find $t_2$ such that $v_1$ + $t_1$ $\cong$ v2 + t2. Then:
v1 + t1 + [t1]  $\cong$ v2 + t2 + [t1].

If we choose t2 = [t1], then we have v1 + t1 $\cong$ v2 + t2 as required. This completes the proof of the second property.

The equivalence relation $\cong$ over clock assignments an be extended to an equvalence relation over the state space of T(A) by requiring that equivalent states have identical locations and equivalent clock assignments: (s,v) $\cong$ (s', v') if and only if s = s' and v $\cong$ v'. The key property of he equivalence reltion $\cong$ is given by the following lemma [5]:


Lemma 45
If v1 $\cong$ v2 and (s, $v_1$)  $\xrightarrow[]{a}$ (s', v'). The transition $\langle$ s, a, $\varphi$, $\lambda$, s' $\rangle$  that takes state (s, v1) to state (s', v1') corresponds to two transiions of the timed automation.

Proof
Assume that v1 $\cong$ v2 and (s, v1)  $\xrightarrow[]{a}$ (s', v'1). The transition $\langle$ s, a, $\varphi$, $\lambda$, s' $\rangle$ that takes state (s, v1) to state (s', v'1) corresponds to two transitions of the timed automation:

a delay transition (s, v1)  $\xrightarrow[]{d1}$ (s, $v_1$ + $d_1$) for some $d_1$ $\geq$ 0, and
an action transition   (s, v1 +d1)  $\xrightarrow[]{a}$ (s', $v_1$') such that $v_1$ + $d_1$ satisfies $\varphi$ and $v'_1$ = ($v_1$v + $d_1$)[$\lambda$ :=0].



Since v1 $\cong$ v2 and v1 satisfies I(s), v2 also satisfies I(s). Furthermore, there exists d2 $\geq$ 0 such that v1 + d1 $\cong$ v2 + d2. Since v1 + d1 satisfies I(s), v2 +d2 also satisfies I(s). Because the clock constraint I(s) is convex and is satisfied by both v2 and v2 + d2, I(s) must be satisfied by v2 + e for all e such that 0 $\leq$ e $\leq$ d2. Consequently, the delay transition (s, v2)  $\xrightarrow[]{d2}$ (s, v2 +d2) is legal.

Since v1 + d1 $\cong$ v2 +d2, both v1+ d1 and v2+ d2 must satisfy the clock constraint for the guard $\varphi$. Thus, the transition $\langle$ s, a, $\varphi$, $\lambda$ , s' $\rangle$ myst also be enabled in the state  *s, $v_2$ + $d_2$
. Let $v'_2$ = ($v_2$ +$d_2$)[$\lambda$ :=0]. Then $v'_2$ is equivalent to $v'_1$. Hence, there is an action transition (s, $v_2$ + $d_2$)  $\xrightarrow[]{a}$  (s', $v'_2$). Combining the delay transition with the action transition, we get (s, $v_2$)  $\xrightarrow[]{a}$ (s', $v'_2$) as required.

As a result of the lemme, we can conostruct a finite state transition raph that is  bisimilaion equivalnt to the infinite state transition graph T(A). The finite state transition graph is called the region graph of A[7,8] and is denoted by R(A). A region is a pair (s, [v]). Since $\cong$  has a finite index, there are only a finite nuber of regions. The states of the region graph are  the regions of A. The construction of R(A) will have the property that whenever (s,v) is a state of T(A), the region (s, [v]) where $s_0$ is an initial state of A and $v_0$ is a clock assignment that assigns 0 to every clock. The transition relation of R(A) is defined so that bisimulation equivalence is guaranteed. There will be a transition labeled with a from the region (s,[v]) to the region (s', [v']) if and only there are assignments $\omega$ $\in$ [v] and $\omega$' $\in$ [v'] such that (s, $\omega$) can make a transition to (s', $\omega$')

We summarize the construction of the region graph R(A) below. Let A = ($\sigma$, S, $S_0$, X, I, T) be a timed automation. Then,
The states of R(A) have the form (s, [v]) where s $\in$ S and [v] i a clock region
The initial states have the form ($s_0$, [v]) where $s_0$ $\in$ $s_0$ and v(x)=0 for all x $\in$ X.
R(A) has a transition ((s,[v]),a, (s',[v'])) if and only if (s, $\omega$)  $\xrightarrow[]{a}$  (s', $\omega$') for some $\omega$ $\in$ [v] and some $\omega$' $\in$ [v'].
We can use Lemma 45 to prove bisimulation equivalene.

Theorem 31
We will show that T(A) and R(A) are bisimilar. Define the bisimulation relation B by (s,v)B(s,[v]). It is easy to see that the initial state ($s_0$, $v_0$) corresponds o the  state ($s_0$, [$v_0$]). Next, we show that for each transtition of T(A), there is a corrresponding transition  of R(A), and vice versa. Suppose first that (s,v)B(s,[v]). Suppose on the other hand that (s,v)B(s,[v]) and that  (s,v)$\xrightarrow[]{a}$(s',[v']). Then there exit $\omega$ $\cong$ v and $\omega$' $\cong$ v' such that (s', v") and (s,v) $\xrightarrow[]{a}$(s', v"). Hence v" $\cong$ $\omega$ $\cong$ v', so [v"] = [v']. By the definition of B, (s', v")
B (s', [v"]), it follows that (s', v")B(s', [v']).




blz 280 clock zones
blz 281 Intersection

blz 281 Clock reset


blz 281 elapsing of time
In principle, the three oeraions on clock zones described above can be used to construct a finite representationof the transition graph T(A) corresponding to a timed automation.


Real-time System = Discrete System + Clock Variables by Rajeev Alur

blz 2 actions
The state of a system changes over time. We refer to the state changes of a
system as actions. An action is a pair ($\sigma$,$\sigma$ ') of states that consists of a source
state $\sigma$ and a target state $\sigma$ '. Intuitively, if a system is in the source state $\sigma$,
then the action ($\sigma$,$\sigma$ ') takes the system into the target state $\sigma$'. We say that
an action is enabled in its source state and disabled in all other states. Two
actions ($\sigma$,$\sigma$ '1) and ($\sigma$,$\sigma$ '2) are consecutive if the second action is enabled in
the target state of the rst action|i.e., if ($\sigma$ '1=$\sigma$ '2). The action ($\sigma$,$\sigma$ ') is a nul l
action if ($\sigma$=$\sigma$ ')
.

blz 6 clocks and delays

Formally, the action ($\sigma$,$\sigma$ ') is a system action if for all clock variables x, either
$\sigma$ '(x) = $\sigma$(x) or $\sigma$ '(x) = 0; the action ($\sigma$,$\sigma$ ') is a time action - or delay -if there
is a nonnegative real $\delta$ the duration of the delay|such that $\sigma$ ' = ($\sigma$,$\sigma$ '). System
actions have duration 0. Every null action is, by denition, both a system action and a delay of duration 0.



blz 7 Clock constraints
Let ($\sigma$, $\delta$) be a delay, let $\phi$ be a state predicate, and let $\psi$  be an action
predicate. The characteristic function of $\phi$ maps each nonnegative real e < $\delta$ to
1 if $\phi$ is true for $\sigma$ + e, and otherwise to 0; the characteristic function of   maps
e to 1 iff $\psi$   is enabled in $\sigma$ + e. A state or action predicate varies finitely over the
delay ($\sigma$, $\delta$) if its characteristic function has nitely many discontinuities in the
interval (0,$\delta$). Abstractly, we restrict ourselves to state predicates and action
predicates that vary nitely over all delays.


blz 8 Clock-constrained systems
A clock-constrained system S = ($\phi$, $\psi$ ) is a pair that consists of a timed state
predicate 0|the initial condition of S|and a timed action predicate $\psi$ |the
transition condition of S. The timed behavior $\sigma$ is a behavior of the clock-
constrained system S if (1) the initial condition of S is initially true for $\sigma$
and (2) the transition condition of S is invariantly true for $\sigma$. Every clock-
constrained system S denes, then, the set of its divergent behaviors, which is
denoted by [[S]].



blz 9 Clock-constrained programs


blz 10 Delay predicates


blz 11 Real-time systems
A real-time system S = ($\phi$, $\psi$, $\chi$) is a triple that consists of a clock-constrained system ($\phi$, $\psi$) and a delay predicate $\chi$ the environment condition
of S. The timed behavior $\sigma$ is a behavior of the real-time system S if (1) $\sigma$ is a
behavior of the clock-constrained system that underlies S and (2) the environ-
ment condition of S is invariantly true for $\sigma$. Every real-time system S defines,
then, the set of its divergent behaviors, which is denoted by [[S]].

For example, the following real-time system S2 = ($\phi$, $\psi$, $\chi$)
changes the value of m from 0 to 1 at time 3 at the earliest and at time 5 at the
latest:
$\phi$ = (m =0 $\wedge$ x =0)
$\psi$ = (m $\geq$ 3 $\wedge$ m1' =1)
$\chi$ = (m =0 $\wedge$ x < 5 ) $\vee$ (m=1)

blz 12 Real-time executability

blz 13 Real-time programs

blz 15 Sequential real-time processes


blz 17 Concurrent real-time processes


blz 19 Embedded real-time processes

blz 30 Verification of Safety Properties

A safety property is simply a closed set of behaviors.
\begin{equation*}
	\begin{split}
		x &= 1 \\
		y &= 2 \\
		\hline
		x + y &= 3 
	\end{split}
\end{equation*}




