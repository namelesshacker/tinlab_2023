\documentclass{article}
\usepackage{graphicx} 
\usepackage[dutch]{babel}
\begin{document}
	\sffamily
	\begin{titlepage}
		\centering
		\vfill
		{\bfseries\Huge
			Verslag Tinlab Advanced Algorithms \\
			\vskip2cm
		}
		{\bfseries\Large
			T. Ravensbergen\\ 
			G. Bartes\\
			K. G. Razmjou\\
		}
		{
			\bfseries\normalsize
			69\\
			\vskip1cm
			\today\\
		}    
		\vfill
		\includegraphics[width=4cm]{logohr.png} % also works with logo.pdf
		\vfill
		\vfill
	\end{titlepage}
	\newpage
	\tableofcontents
	
	\newpage
	\section{Inleiding}
	In deze case study wordt %insert inleiding hier
	
	\section{Requirements}
	
	\subsection{Requirements}
	Directe requirements van opdrachtgever:\\
	Na grondige analyse van het Nederlandse sluizenpark is gebleken dat renova-tie van een groot aantal sluizen noodzakelijk is.  Een eerste verkenning heeft onsgeleerd dat het gecombineerd renoveren en automatiseren van het Nederlandsesluizenpark een aanzienlijke verbetering kan opleveren t.a.v.:\\
	- veiligheid\\
	- efficientie\\
	- capaciteit\\
	- onderhoudskosten\\
	- duurzaamheid\\
	In het kader van het onlangs afgesloten klimaatakkoord heeft de Nederlandseoverheid  daarom  besloten  over te gaan tot een ingrijpende renovatie van dediverse sluizen die ons land rijk is. Op het ministerie van infrastructuur en waterstaat is helaas onvoldoende kennis van ict en systemen aanwezig om eenen ander uit te voeren. Wij vragen u een model (of een onderling samenhangend aantal modellen)aan  te leveren, opdat ontwerpen van verschillende, volledig geautomatiseerde sluizen in de toekomst gerealiseerd kunnen worden.\\\\
	Eigen inbreng van deze requirements:\\
	Wij gaan er van uit dat het volgende van ons verwacht wordt:\\
	Maak een model dat als template dient gebruikt te worden voor het automatiseren van verschillende soorten sluizen. Verder moeten overwegingen gemaakt worden die goed onderbouwd zijn.\\\\ Aangezien er van ons alleen een model verwacht wordt, zullen wij ons geheel focussen op de fundamentele werking van de sluis en hierbij zullen wij ons dus niet bezig  houden met fysieke eisen zoals veiligheidshekjes en borden. Onze focus ligt geheel op de werking van de sluis; elke state waar de sluis zich in mag bevinden en welke beslissingen de sluis moet maken op basis van bestaande protocols en benoemde eisen. \\\\
	Deze requirements zullen hieronder uitgewerkt worden, per sluisonderdeel, deze bestaande uit de sluisdeuren, de sloplichten, de waterpomp en de boten.\\
	
	\subsection{Sluisdeuren}
	De sluisdeuren.
	
	\subsection{Stoplichten}
	De stoplichten
	
	\subsection{Waterpomp}
	De waterpomp
	
	\subsection{Boten}
	De meeste sluizen die zich in Nederland bevinden zijn schutsluizen; deze sluizen zijn bedoeld om boten, zowel vrachtschepen als pleziervaart afhangend van de locatie van de sluis, te verwerken. Om deze reden gaan wij deze dus ook verwerken in ons model. Mocht een sluis niet bedoeld zijn om boten te verwerken, dan zou dit model alsnog toegepast kunnen worden opp desbetreffende sluis.
	Boten worden toegevoed aan de queue. Hoe dit gebeurt, dat ligt aan de specifieke sluis.  Sinds wij een template maken, hoeven wij geen rekening te hounden met hoe de schepen in de queue komen. Het enige wat wij hoeven te doen, is de data verwerken.
	
	Overige einsen op basis van eigen inbreng:\\
	
	
	
	\subsection{Specificaties}
	Vanuit deze requiremenst kunnen verdere specificaties opgesteld worden.
	
	Even ter duidelijkheid: een requirement beschrijft wat een programma moet doen, en een specificatie beschrijft hoe men van plan is om deze requirements te realiseren.//
	Voorbeeld:// Requirement is dat de sluis meerdere boten moet kunnen verwerken; de specificatie zou hier zijn fdat de sluis minstens twee keer zo groot moet zijn dan de grootste boot die door de sluis kan.
	
	\subsection{Notities die verwerkt moeten worden}
	
	moet de intitial state altijd in een loop zitten in uppaal?
	wat zijn urgent channels?
	rampen? er staat wel iets in de planning maar kan geen lessen of verdere documentatie of requirements terug vinden?	
	
	
	gesprek wessel:
	main controller slim dat direction een bool is. 
	pomp is te slim, zoiu alleen maar aan of uit moeten gaan, of nog weg en in pompen maar meer niet. niets met waterlevel en aantal schepen.
	schip: niet doen. als een schip zich aanmeld, dan gebeuren er dingen, maar gaat hij naar binnen? je weet niet wat dat schip gaat doen want menselijk gedrag. beter niet het schip uitgebreid maken, maar eerder de sluis. te veel aannames.
	
	wessel model: alleen als wachtrij vol zit, doet de sluis iets.
	deur heeft een parameter zodat er meerdere deuren in de simulator neergezet kunnnen worden. ook bij wachtrij.
	
	stoplichen kunnen er wel in maar als je simpeler wilt, gaan die als eerste weg.
	zes variabelen model is voorgesteld maar niet goed op gereageerd. alleen er van af weten is genoeg.
	rampen alleen voor persoonlijk verslag
	
	
	\subsection{Het vier variabelen model}
	\subsubsection{Monitored variabelen}
	\subsubsection{Controlled variabelen}
	\subsubsection{Input variabelen}
	\subsubsection{Output variabelen}
	
	\subsection{Rampen}
	
	\subsubsection{Ramp 1}
	\begin{description}
		\item[Beschrijving]
		\item[Datum en plaats] 
		\item[Oorzaak]
		%Beschrijf wat er mis ging in termen van het vier variabelen model/requirements/specificaties
	\end{description}
	
	\subsubsection{Ramp 2}
	\begin{description}
		\item[Beschrijving]
		\item[Datum en plaats] 
		\item[Oorzaak]
		%Beschrijf wat er mis ging in termen van het vier variabelen model/requirements/specificaties
	\end{description}
	
	\subsubsection{Ramp 3}
	\begin{description}
		\item[Beschrijving]
		\item[Datum en plaats] 
		\item[Oorzaak]
		%Beschrijf wat er mis ging in termen van het vier variabelen model/requirements/specificaties
	\end{description}
	
	\subsubsection{Ramp 4}
	\subsubsection{Ramp 5}
	\subsubsection{Ramp 6}
	
	
	\section{Research case: De digitale aanval op de Oekrainese krachtcentrale}
	 Dit verslag geeft inzage in een analyse van de Ukraine cyber aanval,
	inclusief hoe de actoren zich zelf toegang gavan tot het controle systeem, welke methoden de acoren hebben gebruikt voor reconnaissance en vastleggen van het systeem, een gedetailleerde omshrijving van de aanval op 15 December 2015, en de methoden die gebruikt zijn door de aanvallers om hun sporen uit te wissen en daarmee het het stoppen van schade toebrengen  nog moeilker maken. Daarnaast wordter  een gedetailleerde omschrijving gevevenv an de beveiliging van de SCADA ccontrol systemen gebaeerd op bst practices, inclusief het control network ontwerp, technieken voor whtelisting, monitoring en loggen, en  opleiding van personeel.
	
	https://na.eventscloud.com/file_uploads/aed4bc20e84d2839b83c18bcba7e2876_Owens1.pdf
	https://www.wired.com/2016/03/inside-cunning-unprecedented-hack-ukraines-power-grid/
	https://www.boozallen.com/content/dam/boozallen/documents/2016/09/ukraine-report-when-the-lights-went-out.pdf
	https://www.reuters.com/article/us-ukraine-cybersecurity-sandworm-idUSKBN0UM00N20160108
	https://www.mandiant.com/resources/blog/ukraine-and-sandworm-team
	https://www.ifri.org/sites/default/files/atoms/files/desarnaud_cyber_attacks_energy_infrastructures_2017_2.pdf
	https://ris.utwente.nl/ws/files/6028066/3-s2_0-B9780128015957000227.pdf
	https://repositorio-aberto.up.pt/bitstream/10216/119066/2/315683.pdf
	https://www.diva-portal.org/smash/get/diva2:1046339/FULLTEXT01.pdf
	https://www.vice.com/en/article/zmeyg8/ukraine-power-grid-malware-crashoverride-industroyer
	
	
	
	Oop 23,december 2015  vind er een cyber aanval plaats op het elektriciteitsnet van de Oekraine. Dit was de eerste bekende aanval op een elektrisch controle  system met corrupte firmware. Daarnaas wordt er een telecom-based denial of service attack met  geautomatieerde systemen om het telefoonverkeer uit te schakelen.
	\cite{Whitehead2017ukrainepoweroutage}
	
	Uit onderzoek\cite{zetter2016GridHack} naar de aanval,  uitgevoerd door Oekraiene sen Amerikaanse militairenblijkt  bleek onder meer dat de power grids in sommige gevallen beter waren beveiligd dan de Amerikaanse. Desondanks was de viligheid niet optimaal door onder andere de  hetgegeven dat werknemers op afstand konden inloggen en geen gebruik van 2-stapsverificatie.
	
	
	\subsection{Literaire analyse}
	
	\subsubsection{Motief}
	Oekraine wijst naar de russen \cite{zetter2016GridHack}
	https://www.wired.com/story/russian-hackers-attack-ukraine/
	https://www.boozallen.com/content/dam/boozallen/documents/2016/09/ukraine-report-when-the-lights-went-out.pdf
	https://www.reuters.com/article/us-ukraine-cybersecurity-sandworm/u-s-firm-blames-russian-sandworm-hackers-for-ukraine-outage-idUSKBN0UM00N20160108
	https://www.reuters.com/article/us-ukraine-crisis-cyber-idUSKBN15U2CN
	https://theconversation.com/cyberattack-on-ukraine-grid-heres-how-it-worked-and-perhaps-why-it-was-done-52802
	https://jsis.washington.edu/news/cyberattack-critical-infrastructure-russia-ukrainian-power-grid-attacks/
	\subsubsection{Situatie Oekraiene}
	https://www.dragos.com/wp-content/uploads/CrashOverride-01.pdf
	https://www.dragos.com/wp-content/uploads/CRASHOVERRIDE.pdf
	\subsubsection{Situatie algemeen}
	https://www.politico.eu/article/ukraine-cyber-war-frontline-russia-malware-attacks/
	https://www.ifri.org/sites/default/files/atoms/files/desarnaud_cyber_attacks_energy_infrastructures_2017_2.pdf
	https://www.cybersecurityintelligence.com/blog/attack-on-ukraines-power-grid-targeted-transmission-stations-4530.html
	
	\subsubsection{Factoren}
	http://web.mit.edu/smadnick/www/wp/2016-22.pdf
	\subsubsection{Oorzaak}
	https://www.sans.org/blog/confirmation-of-a-coordinated-attack-on-the-ukrainian-power-grid/
	https://arstechnica.com/information-technology/2017/06/crash-override-malware-may-sabotage-electric-grids-but-its-no-stuxnet/
	https://www.darkreading.com/threat-intelligence/first-malware-designed-solely-for-electric-grids-caused-2016-ukraine-outage
	https://www.dragos.com/wp-content/uploads/CRASHOVERRIDE.pdf
	\subsubsection{Gebruikte materialen}
	https://en.wikipedia.org/wiki/2015_Ukraine_power_grid_hack
	https://www.cisa.gov/news-events/alerts/2017/06/12/crashoverride-malware
	https://rhebo.com/en/service/glossar/industroyer-25114/
	
	
	\subsubsection{Uitvoering van de aanval}
	https://na.eventscloud.com/file_uploads/aed4bc20e84d2839b83c18bcba7e2876_Owens1.pdf
	https://www.boozallen.com/content/dam/boozallen/documents/2016/09/ukraine-report-when-the-lights-went-out.pdf
	\subsubsection{Oplossingen}
	https://na.eventscloud.com/file_uploads/aed4bc20e84d2839b83c18bcba7e2876_Owens1.pdf
	https://www.cisa.gov/news-events/ics-alerts/ir-alert-h-16-056-01
	\subsubsection{Aanbevelingen}
	
	\subsection{Resultaten}
	\subsubsection{De aanval}
	1. An initial email spear phishing attack lures recipients
	into opening an attached Microsoft® document with a
	macro that installs Black Energy 3 (BE3) onto
	corporate workstations.
	2. BE3 and other tools perform reconnaissance and
	enumeration of the network and provide an initial
	backdoor for the hackers into the corporate network.
	3. As a result of network reconnaissance, the malicious
	actors discover and access the oblenergos’ Microsoft
	Active Directory® servers that contain corporate user
	accounts and credentials.
	4. With the harvested credentials, the malicious actors use
	an encrypted tunnel from an external network to get
	inside the oblenergo network, establishing a presence
	on the oblenergo control system networks.
	5. Malicious actors discover and access the control center
	supervisory control and data acquisition (SCADA)
	human-machine interface (HMI) servers and
	substations. While a router separates corporate and
	SCADA networks, the firewall rules are improperly
	configured.
	6. On December 23, 2015, at 3:30 p.m., the malicious
	actors begin their power outage attacks by entering
	operations and SCADA networks through backdoors on
	the compromised SCADA workstations. The malicious
	actors take control away from HMI operators and then
	open breakers.
	7. The malicious actors perform several other actions with
	the intent of complicating the responses of control
	operators and increasing the effort required to return the
	system to normal operating conditions. These actions
	include:
	a. Launching a coordinated Telephony Denial of
	Service (TDoS) attack that floods call centers to
	prevent legitimate calls from getting through.
	b. Disabling the UPSs for the control centers.
	c. Corrupting the firmware on a remote terminal unit
	(RTU) HMI module and serial-to-Ethernet port
	servers.
	8. Malicious actors execute KillDisk malware in an
	attempt to wipe out the control center HMIs and pivotpoint workstations.
	https://na.eventscloud.com/file_uploads/aed4bc20e84d2839b83c18bcba7e2876_Owens1.pdf
	https://www.boozallen.com/content/dam/boozallen/documents/2016/09/ukraine-report-when-the-lights-went-out.pdf
	\subsubsection{spearfishing}
	\subsubsection{blackenergy}
	\subsubsection{remote access capabilities}
	\subsubsection{serial-to-ethernet communication devices}
	\subsubsection{telephony denial of service attacks}
	
	\subsection{oplossingen}
	Identificeer alle risicos en schrijf een plan foor het managen van de risico's.
	Implementeer  effecteve controle  om het riico te managen.
	Creeer een diepgaand model dat ervoor zor dat er efectieve en efficiente security controls worden uitgevoerd.
	Aangaande de gebeurtenissen in de oekraiene kunnen de volgende security controls worden opgenomen in het securitymodel: Initial access to enterprise network, pivot in interprise network, elevate priviliges, maintainance access, gain access to control system, attack, attack complication, destroy hard drives.
	\cite{Whitehead2017ukrainepoweroutage}
	
	\subsection{Discussie}
	
	\subsection{Verder lezen}
	https://citeseerx.ist.psu.edu/viewdoc/download;jsessionid=0513EED48102FDAD1BD940260EF12B11?doi=10.1.1.548.7490&amp;rep=rep1&amp;type=pdf
	https://scialert.net/fulltext/?doi=tasr.2014.396.405
	https://www.researchgate.net/publication/333671061_Attacking_IEC-60870-5-104_SCADA_Systems
	https://www.welivesecurity.com/wp-content/uploads/2017/06/Win32_Industroyer.pdf
	https://blog.nettedautomation.com/2017/
	https://arxiv.org/pdf/2001.02925.pdf
	https://dl.acm.org/doi/fullHtml/10.1145/3381038
	https://www.win.tue.nl/~setalle/2017_fauri_encryption.pdf
	http://www.connectivity4ir.co.uk/article/175490/IEC-62351--Secure-communication-in-the-energy-industry.aspx
	https://www.virsec.com/resources/blog/virsec-hack-analysis-deep-dive-into-industroyer-aka-crash-override
	https://dreamlab.net/en/blog/post/fuzzing-ics-protocols/
	https://www.blackhat.com/docs/us-17/wednesday/us-17-Staggs-Adventures-In-Attacking-Wind-Farm-Control-Networks.pdf
	https://blog.checkpoint.com/research/crashoverride/
	https://www.blackhat.com/us-17/briefings/schedule/#industroyercrashoverride-zero-things-cool-about-a-threat-group-targeting-the-power-grid-6159
	https://search.abb.com/library/Download.aspx?DocumentID=9AKK107045A1003&amp;LanguageCode=en&amp;DocumentPartId=&amp;Action=Launch
	https://iiot-world.com/ics-security/cybersecurity/five-cybersecurity-experts-about-crashoverride-malware-main-dangers-and-lessons-for-iiot/
	https://www.csoonline.com/article/3200828/crash-override-malware-that-took-down-a-power-grid-may-have-been-a-test-run.html
	https://www.paloaltonetworks.com/blog/2017/06/crashoverrideindustroyer-protections-palo-alto-networks-customers/
	https://www.webopedia.com/definitions/crashoverride-industroyer-malware/
	https://www.cyber.nj.gov/threat-center/threat-profiles/ics-malware-variants/crashoverride
	https://www.nixu.com/blog/crashoverride-threat-electricity-networks
	https://www.virusbulletin.com/virusbulletin/2019/03/vb2018-paper-anatomy-attack-detecting-and-defeating-crashoverride/
	https://en.wikipedia.org/wiki/Crash_Override_Network
	https://en.wikipedia.org/wiki/Industroyer
	https://www.dragos.com/resource/crashoverride-analyzing-the-malware-that-attacks-power-grids/
	https://www.wallix.com/blog/ics-security-russian-hacking
	https://www.nixu.com/fi/node/53
	https://control.com/forums/threads/comparison-between-iec60870-5-103-and-modbus-rtu.20317/
	\section{Modellen}
	
	\subsection{De Kripke structuur}
	
	\subsection{Soorten modellen}
	
	\subsection{Tijd}
	
	\subsection{Guards en invarianten}
	
	\subsection{Deadlock}
	
	\subsection{Zeno gedrag}
	
	\section{Logica}
	
	\subsection{Propositielogica}
	
	\subsection{Predicatenlogica}
	
	\subsection{Kwantoren}
	
	\subsection{Dualiteiten}
	
	\section{Computation tree logic}
	
	\subsection{De computation tree}
	
	\subsection{Operator: AG}

	\subsection{Operator: EG}
	

	Voor alle paden geldt dat waterlevel lager is dan niveau van de kant.
	Voor alle paden geldt dat een omp werkzaam is als alle sluisdeuren dicht zijn.
	Vpoor alle paden geldt dat het aantal schepen in de sluis maximaal 2 is.
	Voor alle padedn  geldt dat een schip nooit langer dan 30 seconden in een sluiskolk zit zonder dat het waterpeil is aangepast.
	\subsection{Operator: EG}
	Er bestaat op elk pad een 

	\subsection{Operator: AF}
	
	\subsection{Operator: EF}
	r is soms een mogelijkheid dat twee schepen in de sluis een verschillende uitvaarrichting hebben.
	\subsection{Operator: AX}

	
	\subsection{Operator: EX}
	
	\subsection{Operator: p U q}
	
	\subsection{Operator: p R q}
	

	Voor alle paden geldt dat een schip alleen kan invaren als de sluisdeur aan de andere zijde is gesloten.
	\subsection{Operator: EX}
	Er bestaat geen situatie dat een pomp actief is terwijl er een sluisdeur open staat
	\subsection{Operator: p U q}
	Vanaf aankomst tot uitvaren is de clocktijd lager dan 30 tijdseenheden 
	\subsection{Operator: p R q}
	Vanaf invaren tot en met uitvarenvan een schip en geldig is x lager dan 15 tijdseenheden
	vanaf aanvaren staat een schip maximaal 40 tijdseenheden in de wahtrij,.

	\subsection{Operator: AF}
	Er is altijd meerdere
	\subsection{Operator: EF}
	r is soms een mogelijkheid dat twee schepen in de sluis een verschillende uitvaarrichting hebben.
	\subsection{Operator: AX}
	Voor alle paden geldt dat een schip alleen kan invaren als de sluisdeur aan de andere zijde is gesloten.
	\subsection{Operator: EX}
	Er bestaat geen situatie dat een pomp actief is terwijl er een sluisdeur open staat
	\subsection{Operator: p U q}
	Vanaf aankomst tot uitvaren is de clocktijd lager dan 30 tijdseenheden 
	\subsection{Operator: p R q}
	Vanaf invaren tot en met uitvarenvan een schip en geldig is x lager dan 15 tijdseenheden
	vanaf aanvaren staat een schip maximaal 40 tijdseenheden in de wahtrij,.

	\subsection{Fairness}
	
	\subsection{Liveness}
	
 
	
	\newpage
	\bibliography{references}
	\bibliographystyle{plain}
\end{document}


